\documentclass[12pt, a4paper]{article}
\usepackage[utf8]{inputenc}
\usepackage{polski}

\usepackage{amsthm}  %pakiet do tworzenia twierdzeń itp.
\usepackage{amsmath} %pakiet do niektórych symboli matematycznych
\usepackage{amssymb} %pakiet do symboli mat., np. \nsubseteq
\usepackage{amsfonts}
\usepackage{graphicx} %obsługa plików graficznych z rozszerzeniem png, jpg
\theoremstyle{definition} %styl dla definicji
\newtheorem{zad}{} 
\title{Multizestaw zadań}
\author{Robert Fidytek}
%\date{\today}
\date{}
\newcounter{liczniksekcji}
\newcommand{\kategoria}[1]{\section{#1}} %olreślamy nazwę kateforii zadań
\newcommand{\zadStart}[1]{\begin{zad}#1\newline} %oznaczenie początku zadania
\newcommand{\zadStop}{\end{zad}}   %oznaczenie końca zadania
%Makra opcjonarne (nie muszą występować):
\newcommand{\rozwStart}[2]{\noindent \textbf{Rozwiązanie (autor #1 , recenzent #2): }\newline} %oznaczenie początku rozwiązania, opcjonarnie można wprowadzić informację o autorze rozwiązania zadania i recenzencie poprawności wykonania rozwiązania zadania
\newcommand{\rozwStop}{\newline}                                            %oznaczenie końca rozwiązania
\newcommand{\odpStart}{\noindent \textbf{Odpowiedź:}\newline}    %oznaczenie początku odpowiedzi końcowej (wypisanie wyniku)
\newcommand{\odpStop}{\newline}                                             %oznaczenie końca odpowiedzi końcowej (wypisanie wyniku)
\newcommand{\testStart}{\noindent \textbf{Test:}\newline} %ewentualne możliwe opcje odpowiedzi testowej: A. ? B. ? C. ? D. ? itd.
\newcommand{\testStop}{\newline} %koniec wprowadzania odpowiedzi testowych
\newcommand{\kluczStart}{\noindent \textbf{Test poprawna odpowiedź:}\newline} %klucz, poprawna odpowiedź pytania testowego (jedna literka): A lub B lub C lub D itd.
\newcommand{\kluczStop}{\newline} %koniec poprawnej odpowiedzi pytania testowego 
\newcommand{\wstawGrafike}[2]{\begin{figure}[h] \includegraphics[scale=#2] {#1} \end{figure}} %gdyby była potrzeba wstawienia obrazka, parametry: nazwa pliku, skala (jak nie wiesz co wpisać, to wpisz 1)

\begin{document}
\maketitle


\kategoria{Dymkowska,Beger/C1.3i}
\zadStart{Zadanie z Dymkowska,Beger C 1.3 i) moja wersja nr [nrWersji]}
%[a]:[2, 3, 4, 5, 6, 7, 8, 9, 10, 11, 12, 13, 14, 15, 16, 17, 18, 19, 20, 21, 22, 23, 24, 25, 26, 27, 28, 29, 30, 31, 32, 33, 34, 35, 36, 37, 38, 39, 40, 41, 42, 43, 44, 45, 46, 47, 48, 49, 50]
%[b]:[2, 3, 4, 5, 6, 7, 8, 9, 10, 11, 12, 13, 14, 15, 16, 17, 18, 19, 20, 21, 22, 23, 24, 25, 26, 27, 28, 29, 30, 31, 32, 33, 34, 35, 36, 37, 38, 39, 40, 41, 42, 43, 44, 45, 46, 47, 48, 49, 50]
%[a2]=[a]*[a]
%[2a]=2*[a]
%([a]*[a]==[b])
Obliczyć całke.$$\int\frac{(x+[a])^{2}}{[b]+x^{2}}dx$$
\zadStop
\rozwStart{Jakub Ulrych}{Pascal Nawrocki}
$$\int\frac{(x+[a])^{2}}{[b]+x^{2}}dx$$
$$\int\frac{x^{2}+[2a]x+[a2]}{[b]+x^{2}}dx$$
$$\int\bigg(\frac{x^{2}+[a2]}{[b]+x^{2}}+\frac{[2a]x}{[b]+x^{2}}\bigg)dx$$
$$\int1dx+\int\frac{[2a]x}{[b]+x^{2}}dx$$
Pierwsza całka jest trywialnie prosta do obliczenia, a drugą policzymy teraz oddzielnie.
$$\int\frac{[2a]x}{[b]+x^{2}}dx$$
Podstawiamy $t=[b]+x^{2}\Rightarrow dt=2xdx \Rightarrow \frac{dt}{2}=xdx$
$$[2a]\int\frac{1}{2t}dt=[a]\int\frac{1}{t}dt$$
$$[a]ln|t|+C=[a]ln|[b]+x^{2}|+C$$
Teraz możemy dokończyć liczenie naszej całki z zadania.
$$\int1dx+\int\frac{[2a]x}{[b]+x^{2}}dx=x+[a]ln|[b]+x^{2}|+C$$
\rozwStop
\odpStart
$$x+[a]ln|[b]+x^{2}|+C$$
\odpStop
\testStart
A.$x+[a]ln|[b]+x^{2}|+C$
B.$[a]ln|[b]+x^{2}|+C$
C.$x+[a]ln|x^{2}|+C$
D.$x+[b]ln|[a]+x|+C$
\testStop
\kluczStart
A
\kluczStop



\end{document}