\documentclass[12pt, a4paper]{article}
\usepackage[utf8]{inputenc}
\usepackage{polski}

\usepackage{amsthm}  %pakiet do tworzenia twierdzeń itp.
\usepackage{amsmath} %pakiet do niektórych symboli matematycznych
\usepackage{amssymb} %pakiet do symboli mat., np. \nsubseteq
\usepackage{amsfonts}
\usepackage{graphicx} %obsługa plików graficznych z rozszerzeniem png, jpg
\theoremstyle{definition} %styl dla definicji
\newtheorem{zad}{} 
\title{Multizestaw zadań}
\author{Robert Fidytek}
%\date{\today}
\date{}
\newcounter{liczniksekcji}
\newcommand{\kategoria}[1]{\section{#1}} %olreślamy nazwę kateforii zadań
\newcommand{\zadStart}[1]{\begin{zad}#1\newline} %oznaczenie początku zadania
\newcommand{\zadStop}{\end{zad}}   %oznaczenie końca zadania
%Makra opcjonarne (nie muszą występować):
\newcommand{\rozwStart}[2]{\noindent \textbf{Rozwiązanie (autor #1 , recenzent #2): }\newline} %oznaczenie początku rozwiązania, opcjonarnie można wprowadzić informację o autorze rozwiązania zadania i recenzencie poprawności wykonania rozwiązania zadania
\newcommand{\rozwStop}{\newline}                                            %oznaczenie końca rozwiązania
\newcommand{\odpStart}{\noindent \textbf{Odpowiedź:}\newline}    %oznaczenie początku odpowiedzi końcowej (wypisanie wyniku)
\newcommand{\odpStop}{\newline}                                             %oznaczenie końca odpowiedzi końcowej (wypisanie wyniku)
\newcommand{\testStart}{\noindent \textbf{Test:}\newline} %ewentualne możliwe opcje odpowiedzi testowej: A. ? B. ? C. ? D. ? itd.
\newcommand{\testStop}{\newline} %koniec wprowadzania odpowiedzi testowych
\newcommand{\kluczStart}{\noindent \textbf{Test poprawna odpowiedź:}\newline} %klucz, poprawna odpowiedź pytania testowego (jedna literka): A lub B lub C lub D itd.
\newcommand{\kluczStop}{\newline} %koniec poprawnej odpowiedzi pytania testowego 
\newcommand{\wstawGrafike}[2]{\begin{figure}[h] \includegraphics[scale=#2] {#1} \end{figure}} %gdyby była potrzeba wstawienia obrazka, parametry: nazwa pliku, skala (jak nie wiesz co wpisać, to wpisz 1)

\begin{document}
\maketitle


\kategoria{Wikieł/Z1.15p}
\zadStart{Zadanie z Wikieł Z 1.15 p) moja wersja nr [nrWersji]}
%[a]:[2,3,4,5,6]
%[b]:[2,3,4,5,6]
%[ab]=[a]*[b]
%[2a]=2*[a]
%[2b]=2*[b]
%[ba]=[b]/[a]
%[ba1]=round([ba],2)
%[bmab]=[b]-[ab]
%[abmb]=[ab]-[b]
%[mbmab]=-[bmab]
%[x1d]=[mbmab]/[2a]
%[x1]=round([x1d],2)
%[a]!=[b] and [b]<[ab] and [x1]>[ba1] and [x1]<[b]
Rozwiązać nierówność: $\big|\frac{x-[b]}{x+[b]}\big|\leq1$
\zadStop
\rozwStart{Pascal Nawrocki}{Jakub Ulrych}
Dziedzina: $x\in\mathbb{R}\symbol{92}\{-[b]\}$.
Korzystamy z własności wartości bezwzględnej:
$$\bigg|\frac{x-[b]}{x+[b]}\bigg|=\frac{|x-[b]|}{|x+[b]|}$$
Rozpatrujemy teraz 3 przypadki, a na końcu weźmiemy sumę ich rozwiązań i wyrzucimy to czego dziedzina nie zawiera.
\begin{enumerate}
\item $x\in(-\infty,-[b])\Rightarrow\frac{-x+[b]}{-x-[b]}\leq1$
$$\frac{-x+[b]}{-x-[b]}\leq1$$
$$\frac{-x+[b]}{-x-[b]}-1\leq0$$
$$\frac{-x+[b]+x+[b]}{-x-[b]}\leq0$$
$$\frac{[2b]}{-x-[b]}\leq0$$
$$x\in(-[b],\infty)$$
Biorąc pod uwagę przedział dostajemy przedział pusty: $x\in\emptyset$.
\item $ [-[b],[b])\Rightarrow\frac{-x+[b]}{x+[b]}\leq1$
$$\frac{-x+[b]}{x+[b]}\leq1$$
$$\frac{-x+[b]}{x+[b]}-1\leq0$$
$$\frac{-x+[b]-x-[b]}{x+[b]}\leq0$$
$$\frac{-2x}{x+[b]}\leq0$$
$$\frac{x}{x+[b]}\geq0$$
Musimy rozpatrzeć dwa przypadki:
\begin {enumerate}
\item
Licznik nieujemny i mianownik dodatni:
$$x\geq0 \wedge x+[b]>0\Rightarrow x\in[0,\infty)$$
\item
Licznik niedodatni i mianownik ujemny
$$x\leq0 \wedge x+[b]<0\Rightarrow x\in(-\infty,-2)$$
Po rozpatrzeniu przypadków i przedziału otrzymamy:
$$x\in[0,[b])$$
\end{enumerate}
\item $x\in[[b],\infty)\Rightarrow\frac{x-[b]}{x+[b]}\leq1$
$$\frac{x-[b]}{x+[b]}-1\leq0$$
$$\frac{x-[b]-x-[b]}{x+[b]}\leq0$$
$$\frac{0}{x-[b]}\geq0$$
Jako, że warunek ten jest spełniony dla każdego x z zadanego przedziału, to otrzymujemy ten przedział:
$$x\in[[b],\infty)$$
\end{enumerate}
Podsumowując i włączając wyznaczoną wcześniej przez nas dziedzinę otrzymujemy: $1\cup2\cup3\Rightarrow x\in[0,\infty)$.
\rozwStop
\odpStart
$x\in[0,\infty)$
\odpStop
\testStart
A.$x\in[0,\infty)$
\\
B.$x\in[[b],\infty)$
\\
C.$x\in\emptyset$
\\
D.$x\in(-\infty,-[a])$
\testStop
\kluczStart
A
\kluczStop
\end{document}