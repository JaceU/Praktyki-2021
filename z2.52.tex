\documentclass[12pt, a4paper]{article}
\usepackage[utf8]{inputenc}
\usepackage{polski}

\usepackage{amsthm}  %pakiet do tworzenia twierdzeń itp.
\usepackage{amsmath} %pakiet do niektórych symboli matematycznych
\usepackage{amssymb} %pakiet do symboli mat., np. \nsubseteq
\usepackage{amsfonts}
\usepackage{graphicx} %obsługa plików graficznych z rozszerzeniem png, jpg
\theoremstyle{definition} %styl dla definicji
\newtheorem{zad}{} 
\title{Multizestaw zadań}
\author{Robert Fidytek}
%\date{\today}
\date{}
\newcounter{liczniksekcji}
\newcommand{\kategoria}[1]{\section{#1}} %olreślamy nazwę kateforii zadań
\newcommand{\zadStart}[1]{\begin{zad}#1\newline} %oznaczenie początku zadania
\newcommand{\zadStop}{\end{zad}}   %oznaczenie końca zadania
%Makra opcjonarne (nie muszą występować):
\newcommand{\rozwStart}[2]{\noindent \textbf{Rozwiązanie (autor #1 , recenzent #2): }\newline} %oznaczenie początku rozwiązania, opcjonarnie można wprowadzić informację o autorze rozwiązania zadania i recenzencie poprawności wykonania rozwiązania zadania
\newcommand{\rozwStop}{\newline}                                            %oznaczenie końca rozwiązania
\newcommand{\odpStart}{\noindent \textbf{Odpowiedź:}\newline}    %oznaczenie początku odpowiedzi końcowej (wypisanie wyniku)
\newcommand{\odpStop}{\newline}                                             %oznaczenie końca odpowiedzi końcowej (wypisanie wyniku)
\newcommand{\testStart}{\noindent \textbf{Test:}\newline} %ewentualne możliwe opcje odpowiedzi testowej: A. ? B. ? C. ? D. ? itd.
\newcommand{\testStop}{\newline} %koniec wprowadzania odpowiedzi testowych
\newcommand{\kluczStart}{\noindent \textbf{Test poprawna odpowiedź:}\newline} %klucz, poprawna odpowiedź pytania testowego (jedna literka): A lub B lub C lub D itd.
\newcommand{\kluczStop}{\newline} %koniec poprawnej odpowiedzi pytania testowego 
\newcommand{\wstawGrafike}[2]{\begin{figure}[h] \includegraphics[scale=#2] {#1} \end{figure}} %gdyby była potrzeba wstawienia obrazka, parametry: nazwa pliku, skala (jak nie wiesz co wpisać, to wpisz 1)

\begin{document}
\maketitle


\kategoria{Wikieł/Z2.52}
\zadStart{Zadanie z Wikieł Z 2.52 moja wersja nr [nrWersji]}
%[a1]:[2,4,6,8,10,12,14,16,18,20]
%[a2]:[1,2,3,4,5,6,7,8,9,10,11,12,13,14,15,16,17,18,19,20]
%[aa2]=[a2]*[a2]
%[x]=[a1]/2
%[cx]=int([x])
%[xx1]=[cx]*[cx]
%[x1]=[cx]
%[r]=[a2]-[x1]
%[rr]=[r]*[r]
%[m]=[rr]-[xx1]
%[rr]>0 and [m]>0 and math.gcd([m],4)==1 and math.gcd([m],9)==1
Okręgi $x^2+y^2-[a1]x=0$ oraz $x^2+(y-m)^2=[aa2]$ są styczne wewnętrznie. Obliczyć m.
\zadStop
\rozwStart{Aleksandra Pasińska}{}
$$(x-[cx])^2+y^2=[xx1]$$
$$S_1=([cx],0), S_2=(0,m)$$
$$r_1=[x1], r_2=[a2]$$
$$S_1S_2^2=(r_2-r_1)^2$$
$$(-[cx])^2+m^2=[rr]$$
$$m^2=[m]$$
$$m=\pm \sqrt{[m]}$$
\rozwStop
\odpStart
$m=\pm \sqrt{[m]}$\\
\odpStop
\testStart
A.$m=\pm \sqrt{[m]}$
B.$m=\sqrt{[cx]}$
C.$m=6$
D.$m=\pm 8$
E.$m=\pm \sqrt{999}$
F.$m= 2$
G.$m=0$
H.$m=\sqrt{[m]}$
I.$m=- \sqrt{[m]}$
\testStop
\kluczStart
A
\kluczStop



\end{document}