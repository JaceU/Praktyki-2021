\documentclass[12pt, a4paper]{article}
\usepackage[utf8]{inputenc}
\usepackage{polski}

\usepackage{amsthm}  %pakiet do tworzenia twierdzeń itp.
\usepackage{amsmath} %pakiet do niektórych symboli matematycznych
\usepackage{amssymb} %pakiet do symboli mat., np. \nsubseteq
\usepackage{amsfonts}
\usepackage{graphicx} %obsługa plików graficznych z rozszerzeniem png, jpg
\theoremstyle{definition} %styl dla definicji
\newtheorem{zad}{} 
\title{Multizestaw zadań}
\author{Robert Fidytek}
%\date{\today}
\date{}
\newcounter{liczniksekcji}
\newcommand{\kategoria}[1]{\section{#1}} %olreślamy nazwę kateforii zadań
\newcommand{\zadStart}[1]{\begin{zad}#1\newline} %oznaczenie początku zadania
\newcommand{\zadStop}{\end{zad}}   %oznaczenie końca zadania
%Makra opcjonarne (nie muszą występować):
\newcommand{\rozwStart}[2]{\noindent \textbf{Rozwiązanie (autor #1 , recenzent #2): }\newline} %oznaczenie początku rozwiązania, opcjonarnie można wprowadzić informację o autorze rozwiązania zadania i recenzencie poprawności wykonania rozwiązania zadania
\newcommand{\rozwStop}{\newline}                                            %oznaczenie końca rozwiązania
\newcommand{\odpStart}{\noindent \textbf{Odpowiedź:}\newline}    %oznaczenie początku odpowiedzi końcowej (wypisanie wyniku)
\newcommand{\odpStop}{\newline}                                             %oznaczenie końca odpowiedzi końcowej (wypisanie wyniku)
\newcommand{\testStart}{\noindent \textbf{Test:}\newline} %ewentualne możliwe opcje odpowiedzi testowej: A. ? B. ? C. ? D. ? itd.
\newcommand{\testStop}{\newline} %koniec wprowadzania odpowiedzi testowych
\newcommand{\kluczStart}{\noindent \textbf{Test poprawna odpowiedź:}\newline} %klucz, poprawna odpowiedź pytania testowego (jedna literka): A lub B lub C lub D itd.
\newcommand{\kluczStop}{\newline} %koniec poprawnej odpowiedzi pytania testowego 
\newcommand{\wstawGrafike}[2]{\begin{figure}[h] \includegraphics[scale=#2] {#1} \end{figure}} %gdyby była potrzeba wstawienia obrazka, parametry: nazwa pliku, skala (jak nie wiesz co wpisać, to wpisz 1)

\begin{document}
\maketitle


\kategoria{Wikieł/Z2.10}
\kategoria{Matematyka z Wikieł Z 2.10)}
\zadStart{Zadanie z Wikieł Z 2.10 moja wersja nr [nrWersji]}
%[A]:[1,2,3,4,5,6,7,8,9,10,11,12,13,14,15,16,17,18,19,20,21,22,23,24,25,26,27,28,29,30,31,32,33,34,35,36,37,38,39,40,41,42,43,44,45,46,47,48,49,50]
%[v1]:[-10,-9,-8,-7,-6,-5,-4,-3,-2,2,3,4,5,6,7,8,9,10,11,12,13,14,15,16,17,18,19,20]
%[v2]:[-10,-9,-8,-7,-6,-5,-4,-3,-2,2,3,4,5,6,7,8,9,10,11,12,13,14,15,16,17,18,19,20]
%[v3]:[-10,-9,-8,-7,-6,-5,-4,-3,-2,2,3,4,5,6,7,8,9,10,11,12,13,14,15,16,17,18,19,20]
%[kv1]=[v1]*[v1]
%[kv2]=[v2]*[v2]
%[kv3]=[v3]*[v3]
%[s]=[kv1]+[kv2]+[kv3]
%[k]=[A]/[s]
%[a1]=[v1]*[k]
%[a2]=[v2]*[k]
%[a3]=[v3]*[k]
%[ka1]=[a1]*[a1]
%[ka2]=[a2]*[a2]
%[ka3]=[a3]*[a3]
%[a]=[ka1]+[ka2]+[ka3]
%[ck]=int([k])
%[ca1]=int([a1])
%[ca2]=int([a2])
%[ca3]=int([a3])
%[ca]=int([a])
%[k].is_integer()==True and math.gcd([ca],4)==1 and math.gcd([ca],9)==1 and math.gcd([ca],16)==1 and math.gcd([ca],25)==1 and math.gcd([ca],49)==1 
Obliczyć długość wektora $\vec a$, jeżeli $\vec a \circ \vec b=[A],$  $\vec a || \vec  b,$  $\vec b=[[v1],[v2],[v3]]$.
\zadStop
\rozwStart{Aleksandra Pasińska}{}
$$\vec a=[[v1]k,[v2]k,[v3]k]$$
$$[A]=\vec a \circ \vec b=a_1b_1+a_2b_2+a_3b_3=([v1])\cdot ([v1])k+([v2])\cdot ([v2])k+([v3])\cdot ([v3])k=$$ 
$$=[kv1]k+[kv2]k+[kv3]k=[s]k$$ 
$$k=\frac{[A]}{[s]}=[ck]$$
$$\vec a=[[ca1],[ca2],[ca3]]$$
$$|\vec a|=\sqrt{([ca1])^2+([ca2])^2+([ca3])^2}=\sqrt{[ca]}$$
\rozwStop
\odpStart
$|\vec a|=\sqrt{[ca]}$
\odpStop
\testStart
A.$|\vec a|=\sqrt{[ca]}$
B.$|\vec a|=9$
C.$|\vec a|=8$
D.$|\vec a|=e^{5}$
E.$|\vec a|=e^{-2}$
F.$|\vec a|=0$
G.$|\vec a|=1$
\testStop
\kluczStart
A
\kluczStop

\end{document}