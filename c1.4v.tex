\documentclass[12pt, a4paper]{article}
\usepackage[utf8]{inputenc}
\usepackage{polski}

\usepackage{amsthm}  %pakiet do tworzenia twierdzeń itp.
\usepackage{amsmath} %pakiet do niektórych symboli matematycznych
\usepackage{amssymb} %pakiet do symboli mat., np. \nsubseteq
\usepackage{amsfonts}
\usepackage{graphicx} %obsługa plików graficznych z rozszerzeniem png, jpg
\theoremstyle{definition} %styl dla definicji
\newtheorem{zad}{} 
\title{Multizestaw zadań}
\author{Robert Fidytek}
%\date{\today}
\date{}
\newcounter{liczniksekcji}
\newcommand{\kategoria}[1]{\section{#1}} %olreślamy nazwę kateforii zadań
\newcommand{\zadStart}[1]{\begin{zad}#1\newline} %oznaczenie początku zadania
\newcommand{\zadStop}{\end{zad}}   %oznaczenie końca zadania
%Makra opcjonarne (nie muszą występować):
\newcommand{\rozwStart}[2]{\noindent \textbf{Rozwiązanie (autor #1 , recenzent #2): }\newline} %oznaczenie początku rozwiązania, opcjonarnie można wprowadzić informację o autorze rozwiązania zadania i recenzencie poprawności wykonania rozwiązania zadania
\newcommand{\rozwStop}{\newline}                                            %oznaczenie końca rozwiązania
\newcommand{\odpStart}{\noindent \textbf{Odpowiedź:}\newline}    %oznaczenie początku odpowiedzi końcowej (wypisanie wyniku)
\newcommand{\odpStop}{\newline}                                             %oznaczenie końca odpowiedzi końcowej (wypisanie wyniku)
\newcommand{\testStart}{\noindent \textbf{Test:}\newline} %ewentualne możliwe opcje odpowiedzi testowej: A. ? B. ? C. ? D. ? itd.
\newcommand{\testStop}{\newline} %koniec wprowadzania odpowiedzi testowych
\newcommand{\kluczStart}{\noindent \textbf{Test poprawna odpowiedź:}\newline} %klucz, poprawna odpowiedź pytania testowego (jedna literka): A lub B lub C lub D itd.
\newcommand{\kluczStop}{\newline} %koniec poprawnej odpowiedzi pytania testowego 
\newcommand{\wstawGrafike}[2]{\begin{figure}[h] \includegraphics[scale=#2] {#1} \end{figure}} %gdyby była potrzeba wstawienia obrazka, parametry: nazwa pliku, skala (jak nie wiesz co wpisać, to wpisz 1)

\begin{document}
\maketitle


\kategoria{Dymkowska,Beger/C1.4v}
\zadStart{Zadanie z Dymkowska,Beger C 1.4 v) moja wersja nr [nrWersji]}
%[a]:[2,3,4,5,6]
%[b]:[2,3,4,5,6]
%[a]=random.randint(2,12)
%[b]=random.randint(2,12) 
Obliczyć, całkując przez podstawienie.$$\int [a]sin([b]^{x})\cdot[b]^{x}dx$$
\zadStop
\rozwStart{Jakub Ulrych}{Pascal Nawrocki}
$$\int [a]sin([b]^{x})\cdot[b]^{x}dx$$
Podstawiamy $t=[b]^{x}\Rightarrow dt=[b]^{x}ln([b])dx \Rightarrow\frac{dt}{ln([b])}=[b]^{x}dx$
$$\int\frac{1}{ln([b])}\cdot[a]sin(t)dt$$
$$\frac{[a]}{ln([b])}\int sin(t)dt$$
$$-\frac{[a]cos(t)}{ln([b])}+C=-\frac{[a]cos([b]^{x})}{ln([b])}+C$$
\rozwStop
\odpStart
$$-\frac{[a]cos([b]^{x})}{ln([b])}+C$$
\odpStop
\testStart
A.$-\frac{[a]cos([b]^{x})}{ln([b])}+C$
B.$\frac{cos([b]^{x})}{ln([b])}+C$
C.$-\frac{[a]cos([b]^{x})}{[a]}+C$
D.$\frac{[a]cos([b]^{x})}{ln([b])}+C$
\testStop
\kluczStart
A
\kluczStop



\end{document}