\documentclass[12pt, a4paper]{article}
\usepackage[utf8]{inputenc}
\usepackage{polski}

\usepackage{amsthm}  %pakiet do tworzenia twierdzeń itp.
\usepackage{amsmath} %pakiet do niektórych symboli matematycznych
\usepackage{amssymb} %pakiet do symboli mat., np. \nsubseteq
\usepackage{amsfonts}
\usepackage{graphicx} %obsługa plików graficznych z rozszerzeniem png, jpg
\theoremstyle{definition} %styl dla definicji
\newtheorem{zad}{} 
\title{Multizestaw zadań}
\author{Robert Fidytek}
%\date{\today}
\date{}
\newcounter{liczniksekcji}
\newcommand{\kategoria}[1]{\section{#1}} %olreślamy nazwę kateforii zadań
\newcommand{\zadStart}[1]{\begin{zad}#1\newline} %oznaczenie początku zadania
\newcommand{\zadStop}{\end{zad}}   %oznaczenie końca zadania
%Makra opcjonarne (nie muszą występować):
\newcommand{\rozwStart}[2]{\noindent \textbf{Rozwiązanie (autor #1 , recenzent #2): }\newline} %oznaczenie początku rozwiązania, opcjonarnie można wprowadzić informację o autorze rozwiązania zadania i recenzencie poprawności wykonania rozwiązania zadania
\newcommand{\rozwStop}{\newline}                                            %oznaczenie końca rozwiązania
\newcommand{\odpStart}{\noindent \textbf{Odpowiedź:}\newline}    %oznaczenie początku odpowiedzi końcowej (wypisanie wyniku)
\newcommand{\odpStop}{\newline}                                             %oznaczenie końca odpowiedzi końcowej (wypisanie wyniku)
\newcommand{\testStart}{\noindent \textbf{Test:}\newline} %ewentualne możliwe opcje odpowiedzi testowej: A. ? B. ? C. ? D. ? itd.
\newcommand{\testStop}{\newline} %koniec wprowadzania odpowiedzi testowych
\newcommand{\kluczStart}{\noindent \textbf{Test poprawna odpowiedź:}\newline} %klucz, poprawna odpowiedź pytania testowego (jedna literka): A lub B lub C lub D itd.
\newcommand{\kluczStop}{\newline} %koniec poprawnej odpowiedzi pytania testowego 
\newcommand{\wstawGrafike}[2]{\begin{figure}[h] \includegraphics[scale=#2] {#1} \end{figure}} %gdyby była potrzeba wstawienia obrazka, parametry: nazwa pliku, skala (jak nie wiesz co wpisać, to wpisz 1)

\begin{document}
\maketitle
\kategoria{Wikieł/C1.1b}
\zadStart{Zadanie z Wikieł C 1.1b moja wersja nr [nrWersji]}
%[a]:[2,3,4,5,6,7,8,9,10]
%[b]:[2,3,4,5,6,7]
%[c]:[2,3,4,5,6]
%[d]:[2,3,4,5]
%[ad]=[a]*[d]
%[bd]=[b]*[d]
%[cd]=[c]*[d]
%[e]=[d]-1
%[f]=[cd]-[b]
%[g]=[e]+[d]
%[h]=[f]+[bd]
%math.gcd([b],[c])==1 and math.gcd([f],[bd])==1 and math.gcd([d],[g])==1 and math.gcd([bd],[h])==1 and math.gcd([ad],[g])==1 and [b]>[c] and [b]>[d]
Oblicz całkę $$\int \frac{[a]x-\sqrt[[b]]{x^{[c]}}}{\sqrt[[d]]{x}} dx.$$
\zadStop
\rozwStart{Justyna Chojecka}{}
$$\int \frac{[a]x-\sqrt[[b]]{x^{[c]}}}{\sqrt[[d]]{x}} dx=\int \frac{[a]x-x^{\frac{[c]}{[b]}}}{x^{\frac{1}{[d]}}}dx=\int[a]x^{\frac{[e]}{[d]}}-x^{\frac{[f]}{[bd]}}dx$$$$=[a]\int x^{\frac{[e]}{[d]}}dx -\int x^{\frac{[f]}{[bd]}}dx=[a]\cdot \frac{[d]}{[g]}x^{\frac{[g]}{[d]}}-\frac{[bd]}{[h]}x^{\frac{[h]}{[bd]}}+C$$$$=\frac{[ad]}{[g]}x^{\frac{[g]}{[d]}}-\frac{[bd]}{[h]}x^{\frac{[h]}{[bd]}}+C$$
\rozwStop
\odpStart
$\frac{[ad]}{[g]}x^{\frac{[g]}{[d]}}-\frac{[bd]}{[h]}x^{\frac{[h]}{[bd]}}+C$
\odpStop
\testStart
A.$\frac{[ad]}{[g]}x^{\frac{[g]}{[d]}}-\frac{[bd]}{[h]}x^{\frac{[h]}{[bd]}}+C$\\
B.$-\frac{[ad]}{[g]}x^{\frac{[g]}{[d]}}+\frac{[bd]}{[h]}x^{\frac{[h]}{[bd]}}+C$\\
C.$x^{\frac{[g]}{[d]}}+\frac{[bd]}{[h]}x^{\frac{[h]}{[bd]}}+C$\\
D.$\frac{[ad]}{[g]}x^{\frac{[g]}{[d]}}-x^{\frac{[h]}{[bd]}}+C$\\
E.$\frac{[g]}{[d]}x^{\frac{[g]}{[d]}}-\frac{[bd]}{[h]}x^{\frac{[h]}{[bd]}}+C$\\
F.$\frac{[ad]}{[g]}x^{\frac{[g]}{[d]}}+\frac{[bd]}{[h]}x^{\frac{[h]}{[bd]}}+C$\\
G.$\frac{[ad]}{[g]}x^{\frac{[g]}{[d]}}+x^{\frac{[h]}{[bd]}}+C$\\
H.$-\frac{[ad]}{[g]}x^{\frac{[g]}{[d]}}-\frac{[bd]}{[h]}x^{\frac{[h]}{[bd]}}+C$\\
I.$-x^{\frac{[g]}{[d]}}+\frac{[bd]}{[h]}x^{\frac{[h]}{[bd]}}+C$\\
\testStop
\kluczStart
A
\kluczStop



\end{document}