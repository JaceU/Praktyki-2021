\documentclass[12pt, a4paper]{article}
\usepackage[utf8]{inputenc}
\usepackage{polski}

\usepackage{amsthm}  %pakiet do tworzenia twierdzeń itp.
\usepackage{amsmath} %pakiet do niektórych symboli matematycznych
\usepackage{amssymb} %pakiet do symboli mat., np. \nsubseteq
\usepackage{amsfonts}
\usepackage{graphicx} %obsługa plików graficznych z rozszerzeniem png, jpg
\theoremstyle{definition} %styl dla definicji
\newtheorem{zad}{} 
\title{Multizestaw zadań}
\author{Robert Fidytek}
%\date{\today}
\date{}
\newcounter{liczniksekcji}
\newcommand{\kategoria}[1]{\section{#1}} %olreślamy nazwę kateforii zadań
\newcommand{\zadStart}[1]{\begin{zad}#1\newline} %oznaczenie początku zadania
\newcommand{\zadStop}{\end{zad}}   %oznaczenie końca zadania
%Makra opcjonarne (nie muszą występować):
\newcommand{\rozwStart}[2]{\noindent \textbf{Rozwiązanie (autor #1 , recenzent #2): }\newline} %oznaczenie początku rozwiązania, opcjonarnie można wprowadzić informację o autorze rozwiązania zadania i recenzencie poprawności wykonania rozwiązania zadania
\newcommand{\rozwStop}{\newline}                                            %oznaczenie końca rozwiązania
\newcommand{\odpStart}{\noindent \textbf{Odpowiedź:}\newline}    %oznaczenie początku odpowiedzi końcowej (wypisanie wyniku)
\newcommand{\odpStop}{\newline}                                             %oznaczenie końca odpowiedzi końcowej (wypisanie wyniku)
\newcommand{\testStart}{\noindent \textbf{Test:}\newline} %ewentualne możliwe opcje odpowiedzi testowej: A. ? B. ? C. ? D. ? itd.
\newcommand{\testStop}{\newline} %koniec wprowadzania odpowiedzi testowych
\newcommand{\kluczStart}{\noindent \textbf{Test poprawna odpowiedź:}\newline} %klucz, poprawna odpowiedź pytania testowego (jedna literka): A lub B lub C lub D itd.
\newcommand{\kluczStop}{\newline} %koniec poprawnej odpowiedzi pytania testowego 
\newcommand{\wstawGrafike}[2]{\begin{figure}[h] \includegraphics[scale=#2] {#1} \end{figure}} %gdyby była potrzeba wstawienia obrazka, parametry: nazwa pliku, skala (jak nie wiesz co wpisać, to wpisz 1)

\begin{document}
\maketitle


\kategoria{Wikieł/Z1.68c}
\zadStart{Zadanie z Wikieł Z 1.68 c) moja wersja nr [nrWersji]}
%[a]:[2,3,4,5,6,7,8]
%[b]:[2,3,4,5,6,7,8]
%[c]:[2,3,4,5,6,7,8]
%[a]=random.randint(2,11)
%[b]=random.randint(2,11)
%[c]=random.randint(2,11)
%[e]=random.randint(2,11)
%[mbmc]=((-1)*[b]-[c])
%[ap2b]=([a]+2*[b])
%[reszta]=(-2*[a]-4*[b]+4*[c])
%[delta]=(pow([ap2b],2)-4*[mbmc]*[reszta])
%[pierwiastek]=pow([delta],1/2)
%[pint]=int([pierwiastek].real)
%[x1]=((-1)*[ap2b]-[pierwiastek])/(2*[mbmc])
%[x2]=((-1)*[ap2b]+[pierwiastek])/(2*[mbmc])
%[x11]=round([x1].real,2)
%[x22]=round([x2].real,2)
%[delta]>0 and [pierwiastek].is_integer()==True and [x1]!=-2 and [x1]!=2 and [x2]!=-2 and [x2]!=-2 and [x2]!=0 and [x22]!=[c] and [x22]!=[b] and [x22]!=[e] and [x11]!=1 and [x22]!=1 and [x11]!=-1 and [x22]!=-1 and [b]!=[a] and [b]!=[e]
Rozwiązać równania $\frac{[a]}{x^{3}+8}-\frac{[b]}{x^{2}-4}=\frac{[c]}{x^{2}-2x+4}$
\zadStop
\rozwStart{Jakub Ulrych}{Pascal Nawrocki}
$$\frac{[a]}{x^{3}+8}-\frac{[b]}{x^{2}-4}=\frac{[c]}{x^{2}-2x+4}$$
założenie: $$x^{3}+8\neq0 \land x^{2}-4\neq0 \land x^{2}-2x+4\neq0$$
$$x\neq-2 \land x\neq2$$
dziedzina:$$x\in \mathbb{R}-\{-2,2\}$$
rozwiązanie:$$\frac{[a]}{x^{3}+8}-\frac{[b]}{x^{2}-4}=\frac{[c]}{x^{2}-2x+4}$$
$$\frac{[a](x-2)-[b](x^{2}-2x+4)-[c](x+2)(x-2)}{(x+2)(x-2)(x^{2}-2x+4)}$$
$$[a](x-2)-[b](x^{2}-2x+4)-[c](x+2)(x-2)=0$$
$$[mbmc]x^{2}+[ap2b]x+([reszta])=0$$
$$\Delta=[ap2b]^{2}-4\cdot([mbmc])\cdot([reszta])=[delta]\Rightarrow\sqrt{\Delta}=[pint]$$
$$x_{1}=\frac{-[ap2b]-[pint]}{2\cdot([mbmc])}=[x11]$$
$$x_{2}=\frac{-[ap2b]+[pint]}{2\cdot([mbmc])}=[x22]$$
\rozwStop
\odpStart
$$x\in\{[x11],[x22]\}$$
\odpStop
\testStart
A.$x\in\{[x11],[x22]\}$
B.$x\in\{[x11],[c]\}$
C.$x\in\{[a],[b]\}$
D.$x\in\{[e],[b]]\}$
\testStop
\kluczStart
A
\kluczStop
\end{document}