\documentclass[12pt, a4paper]{article}
\usepackage[utf8]{inputenc}
\usepackage{polski}

\usepackage{amsthm}  %pakiet do tworzenia twierdzeń itp.
\usepackage{amsmath} %pakiet do niektórych symboli matematycznych
\usepackage{amssymb} %pakiet do symboli mat., np. \nsubseteq
\usepackage{amsfonts}
\usepackage{graphicx} %obsługa plików graficznych z rozszerzeniem png, jpg
\theoremstyle{definition} %styl dla definicji
\newtheorem{zad}{} 
\title{Multizestaw zadań}
\author{Jacek Jabłoński}
%\date{\today}
\date{}
\newcounter{liczniksekcji}
\newcommand{\kategoria}[1]{\section{#1}} %olreślamy nazwę kateforii zadań
\newcommand{\zadStart}[1]{\begin{zad}#1\newline} %oznaczenie początku zadania
\newcommand{\zadStop}{\end{zad}}   %oznaczenie końca zadania
%Makra opcjonarne (nie muszą występować):
\newcommand{\rozwStart}[2]{\noindent \textbf{Rozwiązanie (autor #1 , recenzent #2): }\newline} %oznaczenie początku rozwiązania, opcjonarnie można wprowadzić informację o autorze rozwiązania zadania i recenzencie poprawności wykonania rozwiązania zadania
\newcommand{\rozwStop}{\newline}                                            %oznaczenie końca rozwiązania
\newcommand{\odpStart}{\noindent \textbf{Odpowiedź:}\newline}    %oznaczenie początku odpowiedzi końcowej (wypisanie wyniku)
\newcommand{\odpStop}{\newline}                                             %oznaczenie końca odpowiedzi końcowej (wypisanie wyniku)
\newcommand{\testStart}{\noindent \textbf{Test:}\newline} %ewentualne możliwe opcje odpowiedzi testowej: A. ? B. ? C. ? D. ? itd.
\newcommand{\testStop}{\newline} %koniec wprowadzania odpowiedzi testowych
\newcommand{\kluczStart}{\noindent \textbf{Test poprawna odpowiedź:}\newline} %klucz, poprawna odpowiedź pytania testowego (jedna literka): A lub B lub C lub D itd.
\newcommand{\kluczStop}{\newline} %koniec poprawnej odpowiedzi pytania testowego 
\newcommand{\wstawGrafike}[2]{\begin{figure}[h] \includegraphics[scale=#2] {#1} \end{figure}} %gdyby była potrzeba wstawienia obrazka, parametry: nazwa pliku, skala (jak nie wiesz co wpisać, to wpisz 1)

\begin{document}
\maketitle


\kategoria{Wikieł/z1.82b}
\zadStart{Zadanie z Wikieł z1.82b) moja wersja nr [nrWersji]}
%[p1]:[2,3,4,5,6,7,8]
%[p2]:[1,3,5,7]
%[p3]:[2,3]
%[p4]:[1,3,5]
%[p5]:[1,3,5]
%[r1]=int(math.pow(2,[p3]))
%[r2]=[p3]*[p4]
%[r3]=[p3]*[p5]
%[r4]=[p1]*2
%[r5]=[p3]*2
%[r6]=2-[r5]
%[r7]=[r4]-[r2]
%[r8]=[p2]-[r3]
%[delta]=abs([r7]*[r7] - 4*[r6]*[r8])
%[pdelta]=int(math.pow([delta],(1/2)))
%[x1]=int((-[r7]-[pdelta])/(2*[r6]))
%[x2]=int((-[r7]+[pdelta])/(2*[r6]))
%[c1]=math.sqrt([delta])
%[c2]=math.isqrt([delta])
%[c3]=[r7]*[r7] - 4*[r6]*[r8]
%[y1]=([x1]*[x1]+[p1]*[x1])*2+[p2]
%[y2]=([x2]*[x2]+[p1]*[x2])*2+[p2]
%[f1]=[x1]-1
%[f2]=[x1]-2
%[f3]=[x1]-3
%[f4]=[x2]+1
%[f5]=[x2]+2
%[f6]=[x2]+3
%[f11]=[y1]+2
%[f22]=[y1]+4
%[f33]=[y1]+6
%[f44]=[y2]+2
%[f55]=[y2]+4
%[f66]=[y2]+6
%not([c1]!=[c2]) and [delta]>0 and [y1]<250 and [y2]<250 and [x1]!=[x2] and [c3]>0
Znaleźć współrzędne punktów, w których przecinają się wykresy funkcji.
b) $y=2^{x^2+[p1]x+\frac{[p2]}{2}} \ \ \ \ y=(\frac{1}{[r1]})^{-x^2 - \frac{[p4]}{2}x - \frac{[p5]}{2}}$
\zadStop
\rozwStart{Jacek Jabłoński}{}
$$2^{x^2+[p1]x+\frac{[p2]}{2}} = (\frac{1}{[r1]})^{-x^2 - \frac{[p4]}{2}x - \frac{[p5]}{2}}$$
$$2^{x^2+[p1]x+\frac{[p2]}{2}} =2^{-[p3](-x^2 - \frac{[p4]}{2}x - \frac{[p5]}{2})}$$
$$x^2+[p1]x+\frac{[p2]}{2} = -[p3](-x^2 - \frac{[p4]}{2}x - \frac{[p5]}{2})$$
$$x^2+[p1]x+\frac{[p2]}{2} = [p3]x^2 + \frac{[r2]}{2}x + \frac{[r3]}{2} $$
$$2x^2 + [r4]x + [p2] = [r5]x^2 + [r2]x + [r3] $$
$$[r6]x^2 + ([r7]x) + [r8] = 0 $$
$$\Delta = [delta]$$
$$\sqrt{\Delta} = [pdelta]$$
$$x_1 = \frac{-([r7]) - [pdelta]}{2 \cdot [r6]} = [x1]$$
$$x_2 = \frac{-([r7]) + [pdelta]}{2 \cdot [r6]} = [x2]$$
$$\textrm{Obliczam wartość dla } x = [x1]$$
$$y_1=2^{([x1])^2 +[p1] \cdot [x1] + \frac{[p2]}{2}} = 2^{\frac{[y1]}{2}} $$
$$\textrm{Obliczam wartość dla } x = [x2]$$
$$y_2=2^{([x2])^2 +[p1] \cdot [x2] + \frac{[p2]}{2}} = 2^{\frac{[y2]}{2}} $$
\rozwStop
\odpStart
$$ x=[x1] \ \ \ y=2^{\frac{[y1]}{2}} \ \ \ lub \ \ \ x=[x2] \ \ \ y=2^{\frac{[y2]}{2}} $$
\odpStop
\testStart
A. $$ x=[x1] \ \ \ y=2^{\frac{[y1]}{2}} \ \ \ lub \ \ \ x=[x2] \ \ \ y=2^{\frac{[y2]}{2}} $$
B. $$ x=[f1] \ \ \ y=2^{\frac{[y1]}{2}} \ \ \ lub \ \ \ x=[f1] \ \ \ y=2^{\frac{[y2]}{2}} $$
C. $$ x=[f2] \ \ \ y=2^{\frac{[y1]}{2}} \ \ \ lub \ \ \ x=[f2] \ \ \ y=2^{\frac{[y2]}{2}} $$
D. $$ x=[f3] \ \ \ y=2^{\frac{[y1]}{2}} \ \ \ lub \ \ \ x=[f3] \ \ \ y=2^{\frac{[y2]}{2}} $$
E. $$ x=[x1] \ \ \ y=2^{\frac{[f11]}{2}} \ \ \ lub \ \ \ x=[x2] \ \ \ y=2^{\frac{[f44]}{2}} $$
F. $$ x=[x1] \ \ \ y=2^{\frac{[f22]}{2}} \ \ \ lub \ \ \ x=[x2] \ \ \ y=2^{\frac{[f55]}{2}} $$
G. $$ x=[x1] \ \ \ y=2^{\frac{[f33]}{2}} \ \ \ lub \ \ \ x=[x2] \ \ \ y=2^{\frac{[f66]}{2}} $$
H. $$ x=[f1] \ \ \ y=2^{\frac{[f11]}{2}} \ \ \ lub \ \ \ x=[f1] \ \ \ y=2^{\frac{[f44]}{2}} $$
I. $$ x=[f2] \ \ \ y=2^{\frac{[f22]}{2}} \ \ \ lub \ \ \ x=[f2] \ \ \ y=2^{\frac{[f55]}{2}} $$
\testStop
\kluczStart
A
\kluczStop



\end{document}
