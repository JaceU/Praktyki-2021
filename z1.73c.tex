\documentclass[12pt, a4paper]{article}
\usepackage[utf8]{inputenc}
\usepackage{polski}

\usepackage{amsthm}  %pakiet do tworzenia twierdzeń itp.
\usepackage{amsmath} %pakiet do niektórych symboli matematycznych
\usepackage{amssymb} %pakiet do symboli mat., np. \nsubseteq
\usepackage{amsfonts}
\usepackage{graphicx} %obsługa plików graficznych z rozszerzeniem png, jpg
\theoremstyle{definition} %styl dla definicji
\newtheorem{zad}{} 
\title{Multizestaw zadań}
\author{Robert Fidytek}
%\date{\today}
\date{}
\newcounter{liczniksekcji}
\newcommand{\kategoria}[1]{\section{#1}} %olreślamy nazwę kateforii zadań
\newcommand{\zadStart}[1]{\begin{zad}#1\newline} %oznaczenie początku zadania
\newcommand{\zadStop}{\end{zad}}   %oznaczenie końca zadania
%Makra opcjonarne (nie muszą występować):
\newcommand{\rozwStart}[2]{\noindent \textbf{Rozwiązanie (autor #1 , recenzent #2): }\newline} %oznaczenie początku rozwiązania, opcjonarnie można wprowadzić informację o autorze rozwiązania zadania i recenzencie poprawności wykonania rozwiązania zadania
\newcommand{\rozwStop}{\newline}                                            %oznaczenie końca rozwiązania
\newcommand{\odpStart}{\noindent \textbf{Odpowiedź:}\newline}    %oznaczenie początku odpowiedzi końcowej (wypisanie wyniku)
\newcommand{\odpStop}{\newline}                                             %oznaczenie końca odpowiedzi końcowej (wypisanie wyniku)
\newcommand{\testStart}{\noindent \textbf{Test:}\newline} %ewentualne możliwe opcje odpowiedzi testowej: A. ? B. ? C. ? D. ? itd.
\newcommand{\testStop}{\newline} %koniec wprowadzania odpowiedzi testowych
\newcommand{\kluczStart}{\noindent \textbf{Test poprawna odpowiedź:}\newline} %klucz, poprawna odpowiedź pytania testowego (jedna literka): A lub B lub C lub D itd.
\newcommand{\kluczStop}{\newline} %koniec poprawnej odpowiedzi pytania testowego 
\newcommand{\wstawGrafike}[2]{\begin{figure}[h] \includegraphics[scale=#2] {#1} \end{figure}} %gdyby była potrzeba wstawienia obrazka, parametry: nazwa pliku, skala (jak nie wiesz co wpisać, to wpisz 1)

\begin{document}
\maketitle


\kategoria{Wikieł/Z1.73c}
\zadStart{Zadanie z Wikieł Z 1.73 c) moja wersja nr [nrWersji]}
%[a]:[2,3,4,5,6,7,8,9]
%[b]:[2,3,4,5,6,7,8,9]
%[a]=random.randint(2,17)
%[b]=random.randint(2,17)
%[-a]=(-1)*[a]
%[-ap1]=[-a]+1
%[ap1]=[a]+1
%[2b]=2*[b]
%[2w]=[2b]/[ap1]
%math.gcd([b],[a])==1 and [2w]<[b] and math.gcd([2b],[ap1])==1
Rozwiązać nierówność.$\frac{|[a]x-[b]|}{|x-[b]|}\leq1$
\zadStop
\rozwStart{Jakub Ulrych}{Pascal Nawrocki}
$$\frac{|[a]x-[b]|}{|x-[b]|}\leq1$$
założenie: $$x-[b]\neq0$$
$$x\neq[b]$$
dziedzina:$$x\in\mathbb{R}-\{[b]\}$$
rozwiązanie:$$\frac{|[a]x-[b]|}{|x-[b]|}\leq1$$
$$|[a]x-[b]|\leq|x-[b]|$$
Wyliczamy miejsca zerowe obu wartości bezwzględnych
$$[a]x-[b]=0\Rightarrow x=\frac{[b]}{[a]}$$
$$x-[b]\neq0\Rightarrow x=[b]$$
Tworzymy przedziały
$$\textbf{1)}x\in(-\infty,\frac{[b]}{[a]})\textbf{  2)}x\in[\frac{[b]}{[a]},[b])\textbf{  3)}x\in[[b],\infty)$$
$$\textbf{1)}x\in(-\infty,\frac{[b]}{[a]})$$
$$-[a]x+[b]\leq-x+[b]$$
$$[-ap1]x\leq0$$
$$x\geq0\land x\in(-\infty,\frac{[b]}{[a]})\Rightarrow x\in[0,\frac{[b]}{[a]})$$
$$\textbf{2)}x\in[\frac{[b]}{[a]},[b])$$
$$[a]x-[b]\leq-x+[b]$$
$$x\leq\frac{[2b]}{[ap1]}\land x\in[\frac{[b]}{[a]},[b])\Rightarrow x\in[\frac{[b]}{[a]},\frac{[2b]}{[ap1]}]$$
$$\textbf{  3)}x\in[[b],\infty)$$
$$[a]x-[b]\leq x-[b]$$
$$x\leq0\notin[[b],\infty)\Rightarrow x\in\emptyset$$
$$\textbf{1)} \vee \textbf{2)} \vee \textbf{3)}\Rightarrow x\in[0,\frac{[2b]}{[ap1]}]$$
\rozwStop
\odpStart
$$x\in[0,\frac{[2b]}{[ap1]}]$$
\odpStop
\testStart
A.$$x\in[0,\frac{[2b]}{[ap1]}]$$
B.$$x\in[\frac{[b]}{[a]},\frac{[2b]}{[ap1]}]$$
C.$$x\in[0,\infty]$$
D.$$x\in[0,\frac{[b]}{[a]}]$$
\testStop
\kluczStart
A
\kluczStop
\end{document}