\documentclass[12pt, a4paper]{article}
\usepackage[utf8]{inputenc}
\usepackage{polski}

\usepackage{amsthm}  %pakiet do tworzenia twierdzeń itp.
\usepackage{amsmath} %pakiet do niektórych symboli matematycznych
\usepackage{amssymb} %pakiet do symboli mat., np. \nsubseteq
\usepackage{amsfonts}
\usepackage{graphicx} %obsługa plików graficznych z rozszerzeniem png, jpg
\theoremstyle{definition} %styl dla definicji
\newtheorem{zad}{} 
\title{Multizestaw zadań}
\author{Robert Fidytek}
%\date{\today}
\date{}
\newcounter{liczniksekcji}
\newcommand{\kategoria}[1]{\section{#1}} %olreślamy nazwę kateforii zadań
\newcommand{\zadStart}[1]{\begin{zad}#1\newline} %oznaczenie początku zadania
\newcommand{\zadStop}{\end{zad}}   %oznaczenie końca zadania
%Makra opcjonarne (nie muszą występować):
\newcommand{\rozwStart}[2]{\noindent \textbf{Rozwiązanie (autor #1 , recenzent #2): }\newline} %oznaczenie początku rozwiązania, opcjonarnie można wprowadzić informację o autorze rozwiązania zadania i recenzencie poprawności wykonania rozwiązania zadania
\newcommand{\rozwStop}{\newline}                                            %oznaczenie końca rozwiązania
\newcommand{\odpStart}{\noindent \textbf{Odpowiedź:}\newline}    %oznaczenie początku odpowiedzi końcowej (wypisanie wyniku)
\newcommand{\odpStop}{\newline}                                             %oznaczenie końca odpowiedzi końcowej (wypisanie wyniku)
\newcommand{\testStart}{\noindent \textbf{Test:}\newline} %ewentualne możliwe opcje odpowiedzi testowej: A. ? B. ? C. ? D. ? itd.
\newcommand{\testStop}{\newline} %koniec wprowadzania odpowiedzi testowych
\newcommand{\kluczStart}{\noindent \textbf{Test poprawna odpowiedź:}\newline} %klucz, poprawna odpowiedź pytania testowego (jedna literka): A lub B lub C lub D itd.
\newcommand{\kluczStop}{\newline} %koniec poprawnej odpowiedzi pytania testowego 
\newcommand{\wstawGrafike}[2]{\begin{figure}[h] \includegraphics[scale=#2] {#1} \end{figure}} %gdyby była potrzeba wstawienia obrazka, parametry: nazwa pliku, skala (jak nie wiesz co wpisać, to wpisz 1)

\begin{document}
\maketitle


\kategoria{Wikieł/C1.7b}
\zadStart{Zadanie z Wikieł C 1.7 b) moja wersja nr [nrWersji]}
%[b]:[2,3,4,5,6,7,8,9]
%[a]:[3,4,5,6,7,8,9,10]
%[c]=math.gcd([a],[b])
%[a1]=int([a]/[c])
%[b1]=int([b]/[c])
%[b]<[a]
Obliczyć całkę
$$\int\frac{dx}{[a]x+[b]}.$$
\zadStop
\rozwStart{Adrianna Stobiecka}{}
$$\int\frac{dx}{[a]x+[b]}=\frac{1}{[a]}\int\frac{[a]}{[a]x+[b]}dx=(*)$$
Skorzystamy ze wzoru $$\frac{f'(x)}{f(x)}dx=\ln{|f(x)|}+C, \qquad f(x)\ne0.$$
\\W naszym przypadku $f(x)=[a]x+[b]$ oraz $f'(x)=[a]$. Zakładamy zatem, że $[a]x+[b]\ne0$, czyli $x\ne-\frac{[b1]}{[a1]}$.
\\Ostatecznie otrzymujemy, że 
$$(*)=\frac{1}{[a]}\ln{|[a]x+[b]|}+C.$$
\rozwStop
\odpStart
$\frac{1}{[a]}\ln{|[a]x+[b]|}+C$
\odpStop
\testStart
A.$[a]\ln{|[a]x+[b]|}+C$\\
B.$-[a]\ln{|[a]x+[b]|}+C$\\
C.$-\frac{1}{[a]}\ln{|[a]x+[b]|}+C$\\
D.$\ln{|[a]x+[b]|}+C$\\
E.$\frac{1}{[a]}\ln{|[a]x+[b]|}+C$\\
F.$-\ln{|[a]x+[b]|}+C$\\
G.$\frac{1}{[a]}\ln{|[a]x|}+C$\\
H.$-\frac{1}{[a]}\ln{|[a]x|}+C$\\
I.$-[a]\ln{|[a]x|}+C$
\testStop
\kluczStart
E
\kluczStop



\end{document}