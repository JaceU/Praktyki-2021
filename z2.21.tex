\documentclass[12pt, a4paper]{article}
\usepackage[utf8]{inputenc}
\usepackage{polski}

\usepackage{amsthm}  %pakiet do tworzenia twierdzeń itp.
\usepackage{amsmath} %pakiet do niektórych symboli matematycznych
\usepackage{amssymb} %pakiet do symboli mat., np. \nsubseteq
\usepackage{amsfonts}
\usepackage{graphicx} %obsługa plików graficznych z rozszerzeniem png, jpg
\theoremstyle{definition} %styl dla definicji
\newtheorem{zad}{} 
\title{Multizestaw zadań}
\author{Robert Fidytek}
%\date{\today}
\date{}
\newcounter{liczniksekcji}
\newcommand{\kategoria}[1]{\section{#1}} %olreślamy nazwę kateforii zadań
\newcommand{\zadStart}[1]{\begin{zad}#1\newline} %oznaczenie początku zadania
\newcommand{\zadStop}{\end{zad}}   %oznaczenie końca zadania
%Makra opcjonarne (nie muszą występować):
\newcommand{\rozwStart}[2]{\noindent \textbf{Rozwiązanie (autor #1 , recenzent #2): }\newline} %oznaczenie początku rozwiązania, opcjonarnie można wprowadzić informację o autorze rozwiązania zadania i recenzencie poprawności wykonania rozwiązania zadania
\newcommand{\rozwStop}{\newline}                                            %oznaczenie końca rozwiązania
\newcommand{\odpStart}{\noindent \textbf{Odpowiedź:}\newline}    %oznaczenie początku odpowiedzi końcowej (wypisanie wyniku)
\newcommand{\odpStop}{\newline}                                             %oznaczenie końca odpowiedzi końcowej (wypisanie wyniku)
\newcommand{\testStart}{\noindent \textbf{Test:}\newline} %ewentualne możliwe opcje odpowiedzi testowej: A. ? B. ? C. ? D. ? itd.
\newcommand{\testStop}{\newline} %koniec wprowadzania odpowiedzi testowych
\newcommand{\kluczStart}{\noindent \textbf{Test poprawna odpowiedź:}\newline} %klucz, poprawna odpowiedź pytania testowego (jedna literka): A lub B lub C lub D itd.
\newcommand{\kluczStop}{\newline} %koniec poprawnej odpowiedzi pytania testowego 
\newcommand{\wstawGrafike}[2]{\begin{figure}[h] \includegraphics[scale=#2] {#1} \end{figure}} %gdyby była potrzeba wstawienia obrazka, parametry: nazwa pliku, skala (jak nie wiesz co wpisać, to wpisz 1)

\begin{document}
\maketitle


\kategoria{Wikieł/Z2.21}
\zadStart{Zadanie z Wikieł Z 2.21  moja wersja nr [nrWersji]}
%[p1]=random.randint(2,10)
%[p2]:[2,3,4,5,6,7,8,9,10]
%[p3]:[2,3,4,5,6,7,8,9,10]
%[p4]:[2,3,4,5,6,7,8,9,10]
%[p5]=random.randint(2,10)
%[p4p1]=[p4]*[p1]
%[p2p3]=[p2]*[p3]
%[b]=[p2p3]-[p4p1]
%[p4k]=[p4]*[p4]
%[p4b]=-[p4]*[b]
%[x]=[p3]*[p3]+[p4k]
%[xw]=-([p4b]-[p5]*[p3])
%[xp]=round([xw]/([x]+0.0000001),2)
%[y]=round((-[p4]*[xw])/([p3]*[x]+0.0000001)+[b]/[p3],2)
%[2xp]=2*[xp]
%[2y]=2*[y]
%[xs]=round([2xp]+[p1],2)
%[ys]=round([2y]-[p2],2)
%math.gcd([p4],[p3])==1 and math.gcd([p4p1],[p3])==1 and math.gcd([b],[p3])==1 and math.gcd([xw],[x])==1  and [b]<0

Znaleźć punkt symetryczny do punktu $B(-[p1],[p2])$ względem prostej $[p3]x-[p4]y+[p5]=0.$
\zadStop

\rozwStart{Maja Szabłowska}{}
$$[p3]x-[p4]y+[p5]=0 \Rightarrow [p4]y=[p3]x+[p5] \Rightarrow y=\frac{[p3]}{[p4]}x+\frac{[p5]}{[p4]}$$
Prosta prostopadła do podanej:
$$\frac{[p3]}{[p4]}\cdot a_{2}=-1 \Rightarrow a_{2}=-\frac{[p4]}{[p3]}$$
 $$y=-\frac{[p4]}{[p3]}x+b$$
 Prosta powinna przechodzić przez punkt $B(-[p1],[p2])$, zatem:
 $$[p2]=\frac{[p4]}{[p3]}\cdot [p1] +b$$
 $$\frac{[p2p3]}{[p3]}-\frac{[p4p1]}{[p3]}=b$$
 $$b=\frac{[b]}{[p3]}$$
 Zatem ostatecznie prosta przyjmuje postać $y=-\frac{[p4]}{[p3]}x+\frac{[b]}{[p3]}.$
 
Kolejno znajdujemy punkt przecięcia się tych dwóch prostych, czyli podstawiamy za $y$ wyliczoną prostą.

$$[p3]x-[p4]\left(-\frac{[p4]}{[p3]}x+\frac{[b]}{[p3]}\right)+[p5]=0$$
$$[p3]x+\frac{[p4k]}{[p3]}x+\frac{[p4b]}{[p3]}+[p5]=0$$
$$[x]x=[xw] \iff x=\frac{[xw]}{[x]}=[xp]$$
$$y=-\frac{[p4]}{[p3]}\cdot\frac{[xw]}{[x]}+\frac{[b]}{[p3]}=[y]$$

Zatem punkt $S([xp],[y])$ jest środkiem odcinka AB, gdzie A jest szukanym punktem.

$$[xp]=\frac{-[p1]+x}{2}, \quad [y]=\frac{[p2]+y}{2}$$
$$[2xp]=-[p1]+x, \quad [2y]=[p2]+y$$
$$x=[xs], \quad y=[ys]$$

\rozwStop


\odpStart
$A([xs],[ys])$
\odpStop
\testStart
A.$A([xs],[ys])$\\
B.$A([x],[xw])$\\
C.$A([y],[p1])$\\
D.$A([p1],0)$\\
E.$A([2xp],[ys])$\\
F.$A([b],[p4])$\\
G.$A([p3],[p2])$\\
\testStop
\kluczStart
A
\kluczStop



\end{document}
