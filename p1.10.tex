\documentclass[12pt, a4paper]{article}
\usepackage[utf8]{inputenc}
\usepackage{polski}

\usepackage{amsthm}  %pakiet do tworzenia twierdzeń itp.
\usepackage{amsmath} %pakiet do niektórych symboli matematycznych
\usepackage{amssymb} %pakiet do symboli mat., np. \nsubseteq
\usepackage{amsfonts}
\usepackage{graphicx} %obsługa plików graficznych z rozszerzeniem png, jpg
\theoremstyle{definition} %styl dla definicji
\newtheorem{zad}{} 
\title{Multizestaw zadań}
\author{Robert Fidytek}
%\date{\today}
\date{}
\newcounter{liczniksekcji}
\newcommand{\kategoria}[1]{\section{#1}} %olreślamy nazwę kateforii zadań
\newcommand{\zadStart}[1]{\begin{zad}#1\newline} %oznaczenie początku zadania
\newcommand{\zadStop}{\end{zad}}   %oznaczenie końca zadania
%Makra opcjonarne (nie muszą występować):
\newcommand{\rozwStart}[2]{\noindent \textbf{Rozwiązanie (autor #1 , recenzent #2): }\newline} %oznaczenie początku rozwiązania, opcjonarnie można wprowadzić informację o autorze rozwiązania zadania i recenzencie poprawności wykonania rozwiązania zadania
\newcommand{\rozwStop}{\newline}                                            %oznaczenie końca rozwiązania
\newcommand{\odpStart}{\noindent \textbf{Odpowiedź:}\newline}    %oznaczenie początku odpowiedzi końcowej (wypisanie wyniku)
\newcommand{\odpStop}{\newline}                                             %oznaczenie końca odpowiedzi końcowej (wypisanie wyniku)
\newcommand{\testStart}{\noindent \textbf{Test:}\newline} %ewentualne możliwe opcje odpowiedzi testowej: A. ? B. ? C. ? D. ? itd.
\newcommand{\testStop}{\newline} %koniec wprowadzania odpowiedzi testowych
\newcommand{\kluczStart}{\noindent \textbf{Test poprawna odpowiedź:}\newline} %klucz, poprawna odpowiedź pytania testowego (jedna literka): A lub B lub C lub D itd.
\newcommand{\kluczStop}{\newline} %koniec poprawnej odpowiedzi pytania testowego 
\newcommand{\wstawGrafike}[2]{\begin{figure}[h] \includegraphics[scale=#2] {#1} \end{figure}} %gdyby była potrzeba wstawienia obrazka, parametry: nazwa pliku, skala (jak nie wiesz co wpisać, to wpisz 1)

\begin{document}
\maketitle

\kategoria{Wikieł/P1.10}

\zadStart{Zadanie z Wikieł P 1.10 moja wersja nr [nrWersji]}
%[a]:[6,7,8,9,10,11,12,13,14,15]
%[b]=2*[a]-1
%[c]=2*[b]
%[d]=[a]**2
%[e]=[c]+2
%[f]=[d]-[b]
%[g]=int([f]**(1/2))
%[h]=[g]*2*[g]
Zbadać, czy A = $ \sqrt{ [a] - \sqrt{ [b] } } \cdot \big ( [a] + \sqrt{ [b] } \big ) \cdot \big ( \sqrt{ [c] } - \sqrt{ 2 } \big )$ jest liczbą niewymierną.
\zadStop

\rozwStart{Natalia Danieluk}{}
Uprościmy zapis liczby A:
$$ \text{A} = \sqrt{ [a] - \sqrt{ [b] } \cdot \big ( [a] + \sqrt{ [b] } \big )^2 \cdot \big ( \sqrt{ [c] } - \sqrt{ 2 } \big )^2 } = $$ 
$$ = \sqrt{ \Big ( [a]^2 - \big (\sqrt{ [b] } \big )^2 \Big ) \cdot \big ( [a] + \sqrt{ [b] } \big ) \cdot \big ( [c]  - 2\sqrt{ [c] }\sqrt{ 2 } + 2 \big ) } = $$ 
$$ = \sqrt{ \big ( [d] - [b] \big ) \cdot \big ( [a] + \sqrt{ [b] } \big ) \cdot \big ( [e]  - 4\sqrt{ [b] } \big ) } = $$ 
$$ = \sqrt{ [f] \cdot \big ( [a] + \sqrt{ [b] } \big ) \cdot 4 \cdot \big ( [a]  - \sqrt{ [b] } \big ) } = $$ 
$$ = \sqrt{ [f] } \cdot \sqrt{ 4 } \cdot \sqrt{ [a]^2 - \big (\sqrt{ [b] } \big )^2 } = 
[g] \cdot 2 \cdot \sqrt{ [d] - [b] } = [g] \cdot 2 \cdot [g] = [h]$$ 
\rozwStop

\odpStart
Liczba A = [h] jest liczbą naturalną, a więc nie jest liczbą niewymierną.
\odpStop

\testStart
A. A jest liczbą niewymierną.
B. A nie jest liczbą niewymierną.
\testStop

\kluczStart
B
\kluczStop

\end{document}