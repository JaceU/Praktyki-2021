\documentclass[12pt, a4paper]{article}
\usepackage[utf8]{inputenc}
\usepackage{polski}

\usepackage{amsthm}  %pakiet do tworzenia twierdzeń itp.
\usepackage{amsmath} %pakiet do niektórych symboli matematycznych
\usepackage{amssymb} %pakiet do symboli mat., np. \nsubseteq
\usepackage{amsfonts}
\usepackage{graphicx} %obsługa plików graficznych z rozszerzeniem png, jpg
\theoremstyle{definition} %styl dla definicji
\newtheorem{zad}{} 
\title{Multizestaw zadań}
\author{Robert Fidytek}
%\date{\today}
\date{}
\newcounter{liczniksekcji}
\newcommand{\kategoria}[1]{\section{#1}} %olreślamy nazwę kateforii zadań
\newcommand{\zadStart}[1]{\begin{zad}#1\newline} %oznaczenie początku zadania
\newcommand{\zadStop}{\end{zad}}   %oznaczenie końca zadania
%Makra opcjonarne (nie muszą występować):
\newcommand{\rozwStart}[2]{\noindent \textbf{Rozwiązanie (autor #1 , recenzent #2): }\newline} %oznaczenie początku rozwiązania, opcjonarnie można wprowadzić informację o autorze rozwiązania zadania i recenzencie poprawności wykonania rozwiązania zadania
\newcommand{\rozwStop}{\newline}                                            %oznaczenie końca rozwiązania
\newcommand{\odpStart}{\noindent \textbf{Odpowiedź:}\newline}    %oznaczenie początku odpowiedzi końcowej (wypisanie wyniku)
\newcommand{\odpStop}{\newline}                                             %oznaczenie końca odpowiedzi końcowej (wypisanie wyniku)
\newcommand{\testStart}{\noindent \textbf{Test:}\newline} %ewentualne możliwe opcje odpowiedzi testowej: A. ? B. ? C. ? D. ? itd.
\newcommand{\testStop}{\newline} %koniec wprowadzania odpowiedzi testowych
\newcommand{\kluczStart}{\noindent \textbf{Test poprawna odpowiedź:}\newline} %klucz, poprawna odpowiedź pytania testowego (jedna literka): A lub B lub C lub D itd.
\newcommand{\kluczStop}{\newline} %koniec poprawnej odpowiedzi pytania testowego 
\newcommand{\wstawGrafike}[2]{\begin{figure}[h] \includegraphics[scale=#2] {#1} \end{figure}} %gdyby była potrzeba wstawienia obrazka, parametry: nazwa pliku, skala (jak nie wiesz co wpisać, to wpisz 1)

\begin{document}
\maketitle


\kategoria{Wikieł/Z1.138j}
\zadStart{Zadanie z Wikieł Z 1.138 j) moja wersja nr [nrWersji]}
%[a]:[2,3,4,5,6,7,8,9,10]
%[b]:[2,3,4,5,6,7,8,9,10]
%[c]:[2,3,4,5,6,7,8,9,10]
%[licznik]=(2+[b]*[a])
%[mianownik]=[a]*2
%[gcd]=math.gcd([licznik],[mianownik])
%[licznik2]=int([licznik]/[gcd])
%[mianownik2]=int([mianownik]/[gcd])
%[licznik3]=abs((2-[b]*[a]))
%[gcd2]=math.gcd([licznik3],[mianownik])
%[licznik4]=int([licznik3]/[gcd2])
%[mianownik3]=int([mianownik]/[gcd2])
%[ab]=[a]*[b]
%[gcd]!=0 and [mianownik2]!=0 and [mianownik2]!=1 and [mianownik3]!=0 and [mianownik3]!=1
Wyznaczyć funckję odwrotną do danej funkcji $f$ określonej na zbiorze $\mathcal{D}_{f}$.\\
j) $f(x)=\frac{\pi}{[a]}-[b]arcsin([c]x)\hspace{5mm}\mathcal{D}_{f}=\left[-\frac{1}{[c]},\frac{1}{[c]}\right]$
\zadStop
\rozwStart{Wojciech Przybylski}{Maja Szabłowska}
$$f(x)=\frac{\pi}{[a]}-[b]arcsin([c]x)\hspace{5mm} \mathcal{D}_{f}=\left[-\frac{1}{[c]},\frac{1}{[c]}\right]$$
$$f(-\frac{1}{[c]})=\frac{\pi}{[a]}+[b]\cdot\frac{\pi}{2}=\frac{[licznik2]\pi}{[mianownik2]}, f(\frac{1}{[c]})=\frac{\pi}{[a]}-[b]\cdot\frac{\pi}{2}=\frac{-[licznik4]\pi}{[mianownik3]}$$
$$f(\mathcal{D}_{f})=\left[-\frac{[licznik4]\pi}{[mianownik3]},\frac{[licznik2]\pi}{[mianownik2]}\right]$$
$$y=\frac{\pi}{[a]}-[b]arcsin([c]x)\Rightarrow \frac{\pi}{[a]}-y=[b]arcsin([c]x) \Rightarrow x=\frac{sin\left(\frac{\frac{\pi}{[a]}-y}{[b]}\right)}{[c]}$$
$$y=f^{-1}(x)=\frac{sin\left(\frac{\pi-[a]x}{[ab]}\right)}{[c]} \mbox{ dla } x\in \left[-\frac{[licznik4]\pi}{[mianownik3]},\frac{[licznik2]\pi}{[mianownik2]}\right]$$
\rozwStop
\odpStart
$f^{-1}(x)=\frac{sin\left(\frac{\pi-[a]x}{[ab]}\right)}{[c]} \mbox{ dla } x\in \left[-\frac{[licznik4]\pi}{[mianownik3]},\frac{[licznik2]\pi}{[mianownik2]}\right]$
\odpStop
\testStart
A. $f^{-1}(x)=\frac{sin\left(\frac{\pi-[a]x}{[ab]}\right)}{[c]} \mbox{ dla } x\in \left[-\frac{[licznik4]\pi}{[mianownik3]},\frac{[licznik2]\pi}{[mianownik2]}\right]$\\
B. $f^{-1}(x)=\frac{arcsin\left(\frac{\pi-[a]x}{[ab]}\right)}{[c]} \mbox{ dla } x\in \left[-\frac{[a]\pi}{[mianownik3]},\frac{[licznik2]\pi}{[mianownik2]}\right]$\\
C. $f^{-1}(x)=\frac{sin\left(\frac{\pi-[a]x}{[b]}\right)}{[c]} \mbox{ dla } x\in \left[-\frac{[b]\pi}{[mianownik3]},\frac{[licznik2]\pi}{[mianownik2]}\right]$\\
D. $f^{-1}(x)=\frac{cos\left(\frac{\pi-[a]x}{[ab]}\right)}{[c]} \mbox{ dla } x\in \left[-\frac{[licznik4]}{[mianownik3]},\frac{[licznik2]\pi}{[mianownik2]}\right]$\\
E. $f^{-1}(x)=\frac{sin\left(\frac{\pi+[a]x}{[ab]}\right)}{[c]} \mbox{ dla } x\in \mathbb{R}$\\
F. $f^{-1}(x)=sin\left(\frac{\pi+[a]x}{[ab]}\right) \mbox{ dla } x\in \left[-\frac{[licznik4]\pi}{[mianownik3]},\frac{[licznik2]\pi}{[mianownik2]}\right]$
\testStop
\kluczStart
A
\kluczStop



\end{document}