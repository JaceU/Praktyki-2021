\documentclass[12pt, a4paper]{article}
\usepackage[utf8]{inputenc}
\usepackage{polski}
\usepackage{amsthm}  %pakiet do tworzenia twierdzeń itp.
\usepackage{amsmath} %pakiet do niektórych symboli matematycznych
\usepackage{amssymb} %pakiet do symboli mat., np. \nsubseteq
\usepackage{amsfonts}
\usepackage{graphicx} %obsługa plików graficznych z rozszerzeniem png, jpg
\theoremstyle{definition} %styl dla definicji
\newtheorem{zad}{} 
\title{Multizestaw zadań}
\author{Patryk Wirkus}
%\date{\today}
\date{}
\newcommand{\kategoria}[1]{\section{#1}}
\newcommand{\zadStart}[1]{\begin{zad}#1\newline}
\newcommand{\zadStop}{\end{zad}}
\newcommand{\rozwStart}[2]{\noindent \textbf{Rozwiązanie (autor #1 , recenzent #2): }\newline}
\newcommand{\rozwStop}{\newline}                                           
\newcommand{\odpStart}{\noindent \textbf{Odpowiedź:}\newline}
\newcommand{\odpStop}{\newline}
\newcommand{\testStart}{\noindent \textbf{Test:}\newline}
\newcommand{\testStop}{\newline}
\newcommand{\kluczStart}{\noindent \textbf{Test poprawna odpowiedź:}\newline}
\newcommand{\kluczStop}{\newline}
\newcommand{\wstawGrafike}[2]{\begin{figure}[h] \includegraphics[scale=#2] {#1} \end{figure}}

\begin{document}
\maketitle

\kategoria{Wikieł/Z4.3a}


\zadStart{Zadanie z Wikieł Z 4.3 a) moja wersja nr 1}
Obliczyć granicę funkcji $f(x)=\frac{\sqrt{x+2}-2}{x}$.
\zadStop
\rozwStart{Patryk Wirkus}{Szymon Tokarski}
$$\frac{\sqrt{x+2}-2}{x}=\frac{(\sqrt{x+2}-2)(\sqrt{x+2}+2)}{x(\sqrt{x+2}+2)}=\frac{1}{\sqrt{x+2}+2}$$
\\
$$\lim\limits_{x\to0}\frac{\sqrt{x+2}-2}{x}=[\frac{0}{0}]=
\lim\limits_{x\to0}\frac{x}{x(\sqrt{x+2}+2)} = \frac{1}{\sqrt{2}+2}$$
\rozwStop
\odpStart
$\frac{1}{\sqrt{2}+2}$
\odpStop
\testStart
A.$\frac{1}{\sqrt{2}+2}$
B.$\frac{2}{\sqrt{2}+2}$
C.$0$
D.$\sqrt{2}+2$
E.$\infty$
F.$-\infty$
G.$\sqrt{2}-2$
H.$-2$
I.$2$
\testStop
\kluczStart
A
\kluczStop



\zadStart{Zadanie z Wikieł Z 4.3 a) moja wersja nr 2}
Obliczyć granicę funkcji $f(x)=\frac{\sqrt{x+3}-3}{x}$.
\zadStop
\rozwStart{Patryk Wirkus}{Szymon Tokarski}
$$\frac{\sqrt{x+3}-3}{x}=\frac{(\sqrt{x+3}-3)(\sqrt{x+3}+3)}{x(\sqrt{x+3}+3)}=\frac{1}{\sqrt{x+3}+3}$$
\\
$$\lim\limits_{x\to0}\frac{\sqrt{x+3}-3}{x}=[\frac{0}{0}]=
\lim\limits_{x\to0}\frac{x}{x(\sqrt{x+3}+3)} = \frac{1}{\sqrt{3}+3}$$
\rozwStop
\odpStart
$\frac{1}{\sqrt{3}+3}$
\odpStop
\testStart
A.$\frac{1}{\sqrt{3}+3}$
B.$\frac{3}{\sqrt{3}+3}$
C.$0$
D.$\sqrt{3}+3$
E.$\infty$
F.$-\infty$
G.$\sqrt{3}-3$
H.$-3$
I.$3$
\testStop
\kluczStart
A
\kluczStop



\zadStart{Zadanie z Wikieł Z 4.3 a) moja wersja nr 3}
Obliczyć granicę funkcji $f(x)=\frac{\sqrt{x+5}-5}{x}$.
\zadStop
\rozwStart{Patryk Wirkus}{Szymon Tokarski}
$$\frac{\sqrt{x+5}-5}{x}=\frac{(\sqrt{x+5}-5)(\sqrt{x+5}+5)}{x(\sqrt{x+5}+5)}=\frac{1}{\sqrt{x+5}+5}$$
\\
$$\lim\limits_{x\to0}\frac{\sqrt{x+5}-5}{x}=[\frac{0}{0}]=
\lim\limits_{x\to0}\frac{x}{x(\sqrt{x+5}+5)} = \frac{1}{\sqrt{5}+5}$$
\rozwStop
\odpStart
$\frac{1}{\sqrt{5}+5}$
\odpStop
\testStart
A.$\frac{1}{\sqrt{5}+5}$
B.$\frac{5}{\sqrt{5}+5}$
C.$0$
D.$\sqrt{5}+5$
E.$\infty$
F.$-\infty$
G.$\sqrt{5}-5$
H.$-5$
I.$5$
\testStop
\kluczStart
A
\kluczStop



\zadStart{Zadanie z Wikieł Z 4.3 a) moja wersja nr 4}
Obliczyć granicę funkcji $f(x)=\frac{\sqrt{x+7}-7}{x}$.
\zadStop
\rozwStart{Patryk Wirkus}{Szymon Tokarski}
$$\frac{\sqrt{x+7}-7}{x}=\frac{(\sqrt{x+7}-7)(\sqrt{x+7}+7)}{x(\sqrt{x+7}+7)}=\frac{1}{\sqrt{x+7}+7}$$
\\
$$\lim\limits_{x\to0}\frac{\sqrt{x+7}-7}{x}=[\frac{0}{0}]=
\lim\limits_{x\to0}\frac{x}{x(\sqrt{x+7}+7)} = \frac{1}{\sqrt{7}+7}$$
\rozwStop
\odpStart
$\frac{1}{\sqrt{7}+7}$
\odpStop
\testStart
A.$\frac{1}{\sqrt{7}+7}$
B.$\frac{7}{\sqrt{7}+7}$
C.$0$
D.$\sqrt{7}+7$
E.$\infty$
F.$-\infty$
G.$\sqrt{7}-7$
H.$-7$
I.$7$
\testStop
\kluczStart
A
\kluczStop



\zadStart{Zadanie z Wikieł Z 4.3 a) moja wersja nr 5}
Obliczyć granicę funkcji $f(x)=\frac{\sqrt{x+11}-11}{x}$.
\zadStop
\rozwStart{Patryk Wirkus}{Szymon Tokarski}
$$\frac{\sqrt{x+11}-11}{x}=\frac{(\sqrt{x+11}-11)(\sqrt{x+11}+11)}{x(\sqrt{x+11}+11)}=\frac{1}{\sqrt{x+11}+11}$$
\\
$$\lim\limits_{x\to0}\frac{\sqrt{x+11}-11}{x}=[\frac{0}{0}]=
\lim\limits_{x\to0}\frac{x}{x(\sqrt{x+11}+11)} = \frac{1}{\sqrt{11}+11}$$
\rozwStop
\odpStart
$\frac{1}{\sqrt{11}+11}$
\odpStop
\testStart
A.$\frac{1}{\sqrt{11}+11}$
B.$\frac{11}{\sqrt{11}+11}$
C.$0$
D.$\sqrt{11}+11$
E.$\infty$
F.$-\infty$
G.$\sqrt{11}-11$
H.$-11$
I.$11$
\testStop
\kluczStart
A
\kluczStop



\zadStart{Zadanie z Wikieł Z 4.3 a) moja wersja nr 6}
Obliczyć granicę funkcji $f(x)=\frac{\sqrt{x+13}-13}{x}$.
\zadStop
\rozwStart{Patryk Wirkus}{Szymon Tokarski}
$$\frac{\sqrt{x+13}-13}{x}=\frac{(\sqrt{x+13}-13)(\sqrt{x+13}+13)}{x(\sqrt{x+13}+13)}=\frac{1}{\sqrt{x+13}+13}$$
\\
$$\lim\limits_{x\to0}\frac{\sqrt{x+13}-13}{x}=[\frac{0}{0}]=
\lim\limits_{x\to0}\frac{x}{x(\sqrt{x+13}+13)} = \frac{1}{\sqrt{13}+13}$$
\rozwStop
\odpStart
$\frac{1}{\sqrt{13}+13}$
\odpStop
\testStart
A.$\frac{1}{\sqrt{13}+13}$
B.$\frac{13}{\sqrt{13}+13}$
C.$0$
D.$\sqrt{13}+13$
E.$\infty$
F.$-\infty$
G.$\sqrt{13}-13$
H.$-13$
I.$13$
\testStop
\kluczStart
A
\kluczStop



\zadStart{Zadanie z Wikieł Z 4.3 a) moja wersja nr 7}
Obliczyć granicę funkcji $f(x)=\frac{\sqrt{x+17}-17}{x}$.
\zadStop
\rozwStart{Patryk Wirkus}{Szymon Tokarski}
$$\frac{\sqrt{x+17}-17}{x}=\frac{(\sqrt{x+17}-17)(\sqrt{x+17}+17)}{x(\sqrt{x+17}+17)}=\frac{1}{\sqrt{x+17}+17}$$
\\
$$\lim\limits_{x\to0}\frac{\sqrt{x+17}-17}{x}=[\frac{0}{0}]=
\lim\limits_{x\to0}\frac{x}{x(\sqrt{x+17}+17)} = \frac{1}{\sqrt{17}+17}$$
\rozwStop
\odpStart
$\frac{1}{\sqrt{17}+17}$
\odpStop
\testStart
A.$\frac{1}{\sqrt{17}+17}$
B.$\frac{17}{\sqrt{17}+17}$
C.$0$
D.$\sqrt{17}+17$
E.$\infty$
F.$-\infty$
G.$\sqrt{17}-17$
H.$-17$
I.$17$
\testStop
\kluczStart
A
\kluczStop



\zadStart{Zadanie z Wikieł Z 4.3 a) moja wersja nr 8}
Obliczyć granicę funkcji $f(x)=\frac{\sqrt{x+19}-19}{x}$.
\zadStop
\rozwStart{Patryk Wirkus}{Szymon Tokarski}
$$\frac{\sqrt{x+19}-19}{x}=\frac{(\sqrt{x+19}-19)(\sqrt{x+19}+19)}{x(\sqrt{x+19}+19)}=\frac{1}{\sqrt{x+19}+19}$$
\\
$$\lim\limits_{x\to0}\frac{\sqrt{x+19}-19}{x}=[\frac{0}{0}]=
\lim\limits_{x\to0}\frac{x}{x(\sqrt{x+19}+19)} = \frac{1}{\sqrt{19}+19}$$
\rozwStop
\odpStart
$\frac{1}{\sqrt{19}+19}$
\odpStop
\testStart
A.$\frac{1}{\sqrt{19}+19}$
B.$\frac{19}{\sqrt{19}+19}$
C.$0$
D.$\sqrt{19}+19$
E.$\infty$
F.$-\infty$
G.$\sqrt{19}-19$
H.$-19$
I.$19$
\testStop
\kluczStart
A
\kluczStop



\zadStart{Zadanie z Wikieł Z 4.3 a) moja wersja nr 9}
Obliczyć granicę funkcji $f(x)=\frac{\sqrt{x+23}-23}{x}$.
\zadStop
\rozwStart{Patryk Wirkus}{Szymon Tokarski}
$$\frac{\sqrt{x+23}-23}{x}=\frac{(\sqrt{x+23}-23)(\sqrt{x+23}+23)}{x(\sqrt{x+23}+23)}=\frac{1}{\sqrt{x+23}+23}$$
\\
$$\lim\limits_{x\to0}\frac{\sqrt{x+23}-23}{x}=[\frac{0}{0}]=
\lim\limits_{x\to0}\frac{x}{x(\sqrt{x+23}+23)} = \frac{1}{\sqrt{23}+23}$$
\rozwStop
\odpStart
$\frac{1}{\sqrt{23}+23}$
\odpStop
\testStart
A.$\frac{1}{\sqrt{23}+23}$
B.$\frac{23}{\sqrt{23}+23}$
C.$0$
D.$\sqrt{23}+23$
E.$\infty$
F.$-\infty$
G.$\sqrt{23}-23$
H.$-23$
I.$23$
\testStop
\kluczStart
A
\kluczStop



\zadStart{Zadanie z Wikieł Z 4.3 a) moja wersja nr 10}
Obliczyć granicę funkcji $f(x)=\frac{\sqrt{x+29}-29}{x}$.
\zadStop
\rozwStart{Patryk Wirkus}{Szymon Tokarski}
$$\frac{\sqrt{x+29}-29}{x}=\frac{(\sqrt{x+29}-29)(\sqrt{x+29}+29)}{x(\sqrt{x+29}+29)}=\frac{1}{\sqrt{x+29}+29}$$
\\
$$\lim\limits_{x\to0}\frac{\sqrt{x+29}-29}{x}=[\frac{0}{0}]=
\lim\limits_{x\to0}\frac{x}{x(\sqrt{x+29}+29)} = \frac{1}{\sqrt{29}+29}$$
\rozwStop
\odpStart
$\frac{1}{\sqrt{29}+29}$
\odpStop
\testStart
A.$\frac{1}{\sqrt{29}+29}$
B.$\frac{29}{\sqrt{29}+29}$
C.$0$
D.$\sqrt{29}+29$
E.$\infty$
F.$-\infty$
G.$\sqrt{29}-29$
H.$-29$
I.$29$
\testStop
\kluczStart
A
\kluczStop



\zadStart{Zadanie z Wikieł Z 4.3 a) moja wersja nr 11}
Obliczyć granicę funkcji $f(x)=\frac{\sqrt{x+31}-31}{x}$.
\zadStop
\rozwStart{Patryk Wirkus}{Szymon Tokarski}
$$\frac{\sqrt{x+31}-31}{x}=\frac{(\sqrt{x+31}-31)(\sqrt{x+31}+31)}{x(\sqrt{x+31}+31)}=\frac{1}{\sqrt{x+31}+31}$$
\\
$$\lim\limits_{x\to0}\frac{\sqrt{x+31}-31}{x}=[\frac{0}{0}]=
\lim\limits_{x\to0}\frac{x}{x(\sqrt{x+31}+31)} = \frac{1}{\sqrt{31}+31}$$
\rozwStop
\odpStart
$\frac{1}{\sqrt{31}+31}$
\odpStop
\testStart
A.$\frac{1}{\sqrt{31}+31}$
B.$\frac{31}{\sqrt{31}+31}$
C.$0$
D.$\sqrt{31}+31$
E.$\infty$
F.$-\infty$
G.$\sqrt{31}-31$
H.$-31$
I.$31$
\testStop
\kluczStart
A
\kluczStop



\zadStart{Zadanie z Wikieł Z 4.3 a) moja wersja nr 12}
Obliczyć granicę funkcji $f(x)=\frac{\sqrt{x+37}-37}{x}$.
\zadStop
\rozwStart{Patryk Wirkus}{Szymon Tokarski}
$$\frac{\sqrt{x+37}-37}{x}=\frac{(\sqrt{x+37}-37)(\sqrt{x+37}+37)}{x(\sqrt{x+37}+37)}=\frac{1}{\sqrt{x+37}+37}$$
\\
$$\lim\limits_{x\to0}\frac{\sqrt{x+37}-37}{x}=[\frac{0}{0}]=
\lim\limits_{x\to0}\frac{x}{x(\sqrt{x+37}+37)} = \frac{1}{\sqrt{37}+37}$$
\rozwStop
\odpStart
$\frac{1}{\sqrt{37}+37}$
\odpStop
\testStart
A.$\frac{1}{\sqrt{37}+37}$
B.$\frac{37}{\sqrt{37}+37}$
C.$0$
D.$\sqrt{37}+37$
E.$\infty$
F.$-\infty$
G.$\sqrt{37}-37$
H.$-37$
I.$37$
\testStop
\kluczStart
A
\kluczStop



\zadStart{Zadanie z Wikieł Z 4.3 a) moja wersja nr 13}
Obliczyć granicę funkcji $f(x)=\frac{\sqrt{x+41}-41}{x}$.
\zadStop
\rozwStart{Patryk Wirkus}{Szymon Tokarski}
$$\frac{\sqrt{x+41}-41}{x}=\frac{(\sqrt{x+41}-41)(\sqrt{x+41}+41)}{x(\sqrt{x+41}+41)}=\frac{1}{\sqrt{x+41}+41}$$
\\
$$\lim\limits_{x\to0}\frac{\sqrt{x+41}-41}{x}=[\frac{0}{0}]=
\lim\limits_{x\to0}\frac{x}{x(\sqrt{x+41}+41)} = \frac{1}{\sqrt{41}+41}$$
\rozwStop
\odpStart
$\frac{1}{\sqrt{41}+41}$
\odpStop
\testStart
A.$\frac{1}{\sqrt{41}+41}$
B.$\frac{41}{\sqrt{41}+41}$
C.$0$
D.$\sqrt{41}+41$
E.$\infty$
F.$-\infty$
G.$\sqrt{41}-41$
H.$-41$
I.$41$
\testStop
\kluczStart
A
\kluczStop



\zadStart{Zadanie z Wikieł Z 4.3 a) moja wersja nr 14}
Obliczyć granicę funkcji $f(x)=\frac{\sqrt{x+43}-43}{x}$.
\zadStop
\rozwStart{Patryk Wirkus}{Szymon Tokarski}
$$\frac{\sqrt{x+43}-43}{x}=\frac{(\sqrt{x+43}-43)(\sqrt{x+43}+43)}{x(\sqrt{x+43}+43)}=\frac{1}{\sqrt{x+43}+43}$$
\\
$$\lim\limits_{x\to0}\frac{\sqrt{x+43}-43}{x}=[\frac{0}{0}]=
\lim\limits_{x\to0}\frac{x}{x(\sqrt{x+43}+43)} = \frac{1}{\sqrt{43}+43}$$
\rozwStop
\odpStart
$\frac{1}{\sqrt{43}+43}$
\odpStop
\testStart
A.$\frac{1}{\sqrt{43}+43}$
B.$\frac{43}{\sqrt{43}+43}$
C.$0$
D.$\sqrt{43}+43$
E.$\infty$
F.$-\infty$
G.$\sqrt{43}-43$
H.$-43$
I.$43$
\testStop
\kluczStart
A
\kluczStop



\zadStart{Zadanie z Wikieł Z 4.3 a) moja wersja nr 15}
Obliczyć granicę funkcji $f(x)=\frac{\sqrt{x+47}-47}{x}$.
\zadStop
\rozwStart{Patryk Wirkus}{Szymon Tokarski}
$$\frac{\sqrt{x+47}-47}{x}=\frac{(\sqrt{x+47}-47)(\sqrt{x+47}+47)}{x(\sqrt{x+47}+47)}=\frac{1}{\sqrt{x+47}+47}$$
\\
$$\lim\limits_{x\to0}\frac{\sqrt{x+47}-47}{x}=[\frac{0}{0}]=
\lim\limits_{x\to0}\frac{x}{x(\sqrt{x+47}+47)} = \frac{1}{\sqrt{47}+47}$$
\rozwStop
\odpStart
$\frac{1}{\sqrt{47}+47}$
\odpStop
\testStart
A.$\frac{1}{\sqrt{47}+47}$
B.$\frac{47}{\sqrt{47}+47}$
C.$0$
D.$\sqrt{47}+47$
E.$\infty$
F.$-\infty$
G.$\sqrt{47}-47$
H.$-47$
I.$47$
\testStop
\kluczStart
A
\kluczStop



\zadStart{Zadanie z Wikieł Z 4.3 a) moja wersja nr 16}
Obliczyć granicę funkcji $f(x)=\frac{\sqrt{x+53}-53}{x}$.
\zadStop
\rozwStart{Patryk Wirkus}{Szymon Tokarski}
$$\frac{\sqrt{x+53}-53}{x}=\frac{(\sqrt{x+53}-53)(\sqrt{x+53}+53)}{x(\sqrt{x+53}+53)}=\frac{1}{\sqrt{x+53}+53}$$
\\
$$\lim\limits_{x\to0}\frac{\sqrt{x+53}-53}{x}=[\frac{0}{0}]=
\lim\limits_{x\to0}\frac{x}{x(\sqrt{x+53}+53)} = \frac{1}{\sqrt{53}+53}$$
\rozwStop
\odpStart
$\frac{1}{\sqrt{53}+53}$
\odpStop
\testStart
A.$\frac{1}{\sqrt{53}+53}$
B.$\frac{53}{\sqrt{53}+53}$
C.$0$
D.$\sqrt{53}+53$
E.$\infty$
F.$-\infty$
G.$\sqrt{53}-53$
H.$-53$
I.$53$
\testStop
\kluczStart
A
\kluczStop



\zadStart{Zadanie z Wikieł Z 4.3 a) moja wersja nr 17}
Obliczyć granicę funkcji $f(x)=\frac{\sqrt{x+59}-59}{x}$.
\zadStop
\rozwStart{Patryk Wirkus}{Szymon Tokarski}
$$\frac{\sqrt{x+59}-59}{x}=\frac{(\sqrt{x+59}-59)(\sqrt{x+59}+59)}{x(\sqrt{x+59}+59)}=\frac{1}{\sqrt{x+59}+59}$$
\\
$$\lim\limits_{x\to0}\frac{\sqrt{x+59}-59}{x}=[\frac{0}{0}]=
\lim\limits_{x\to0}\frac{x}{x(\sqrt{x+59}+59)} = \frac{1}{\sqrt{59}+59}$$
\rozwStop
\odpStart
$\frac{1}{\sqrt{59}+59}$
\odpStop
\testStart
A.$\frac{1}{\sqrt{59}+59}$
B.$\frac{59}{\sqrt{59}+59}$
C.$0$
D.$\sqrt{59}+59$
E.$\infty$
F.$-\infty$
G.$\sqrt{59}-59$
H.$-59$
I.$59$
\testStop
\kluczStart
A
\kluczStop



\zadStart{Zadanie z Wikieł Z 4.3 a) moja wersja nr 18}
Obliczyć granicę funkcji $f(x)=\frac{\sqrt{x+61}-61}{x}$.
\zadStop
\rozwStart{Patryk Wirkus}{Szymon Tokarski}
$$\frac{\sqrt{x+61}-61}{x}=\frac{(\sqrt{x+61}-61)(\sqrt{x+61}+61)}{x(\sqrt{x+61}+61)}=\frac{1}{\sqrt{x+61}+61}$$
\\
$$\lim\limits_{x\to0}\frac{\sqrt{x+61}-61}{x}=[\frac{0}{0}]=
\lim\limits_{x\to0}\frac{x}{x(\sqrt{x+61}+61)} = \frac{1}{\sqrt{61}+61}$$
\rozwStop
\odpStart
$\frac{1}{\sqrt{61}+61}$
\odpStop
\testStart
A.$\frac{1}{\sqrt{61}+61}$
B.$\frac{61}{\sqrt{61}+61}$
C.$0$
D.$\sqrt{61}+61$
E.$\infty$
F.$-\infty$
G.$\sqrt{61}-61$
H.$-61$
I.$61$
\testStop
\kluczStart
A
\kluczStop



\zadStart{Zadanie z Wikieł Z 4.3 a) moja wersja nr 19}
Obliczyć granicę funkcji $f(x)=\frac{\sqrt{x+67}-67}{x}$.
\zadStop
\rozwStart{Patryk Wirkus}{Szymon Tokarski}
$$\frac{\sqrt{x+67}-67}{x}=\frac{(\sqrt{x+67}-67)(\sqrt{x+67}+67)}{x(\sqrt{x+67}+67)}=\frac{1}{\sqrt{x+67}+67}$$
\\
$$\lim\limits_{x\to0}\frac{\sqrt{x+67}-67}{x}=[\frac{0}{0}]=
\lim\limits_{x\to0}\frac{x}{x(\sqrt{x+67}+67)} = \frac{1}{\sqrt{67}+67}$$
\rozwStop
\odpStart
$\frac{1}{\sqrt{67}+67}$
\odpStop
\testStart
A.$\frac{1}{\sqrt{67}+67}$
B.$\frac{67}{\sqrt{67}+67}$
C.$0$
D.$\sqrt{67}+67$
E.$\infty$
F.$-\infty$
G.$\sqrt{67}-67$
H.$-67$
I.$67$
\testStop
\kluczStart
A
\kluczStop



\zadStart{Zadanie z Wikieł Z 4.3 a) moja wersja nr 20}
Obliczyć granicę funkcji $f(x)=\frac{\sqrt{x+71}-71}{x}$.
\zadStop
\rozwStart{Patryk Wirkus}{Szymon Tokarski}
$$\frac{\sqrt{x+71}-71}{x}=\frac{(\sqrt{x+71}-71)(\sqrt{x+71}+71)}{x(\sqrt{x+71}+71)}=\frac{1}{\sqrt{x+71}+71}$$
\\
$$\lim\limits_{x\to0}\frac{\sqrt{x+71}-71}{x}=[\frac{0}{0}]=
\lim\limits_{x\to0}\frac{x}{x(\sqrt{x+71}+71)} = \frac{1}{\sqrt{71}+71}$$
\rozwStop
\odpStart
$\frac{1}{\sqrt{71}+71}$
\odpStop
\testStart
A.$\frac{1}{\sqrt{71}+71}$
B.$\frac{71}{\sqrt{71}+71}$
C.$0$
D.$\sqrt{71}+71$
E.$\infty$
F.$-\infty$
G.$\sqrt{71}-71$
H.$-71$
I.$71$
\testStop
\kluczStart
A
\kluczStop



\zadStart{Zadanie z Wikieł Z 4.3 a) moja wersja nr 21}
Obliczyć granicę funkcji $f(x)=\frac{\sqrt{x+73}-73}{x}$.
\zadStop
\rozwStart{Patryk Wirkus}{Szymon Tokarski}
$$\frac{\sqrt{x+73}-73}{x}=\frac{(\sqrt{x+73}-73)(\sqrt{x+73}+73)}{x(\sqrt{x+73}+73)}=\frac{1}{\sqrt{x+73}+73}$$
\\
$$\lim\limits_{x\to0}\frac{\sqrt{x+73}-73}{x}=[\frac{0}{0}]=
\lim\limits_{x\to0}\frac{x}{x(\sqrt{x+73}+73)} = \frac{1}{\sqrt{73}+73}$$
\rozwStop
\odpStart
$\frac{1}{\sqrt{73}+73}$
\odpStop
\testStart
A.$\frac{1}{\sqrt{73}+73}$
B.$\frac{73}{\sqrt{73}+73}$
C.$0$
D.$\sqrt{73}+73$
E.$\infty$
F.$-\infty$
G.$\sqrt{73}-73$
H.$-73$
I.$73$
\testStop
\kluczStart
A
\kluczStop



\zadStart{Zadanie z Wikieł Z 4.3 a) moja wersja nr 22}
Obliczyć granicę funkcji $f(x)=\frac{\sqrt{x+79}-79}{x}$.
\zadStop
\rozwStart{Patryk Wirkus}{Szymon Tokarski}
$$\frac{\sqrt{x+79}-79}{x}=\frac{(\sqrt{x+79}-79)(\sqrt{x+79}+79)}{x(\sqrt{x+79}+79)}=\frac{1}{\sqrt{x+79}+79}$$
\\
$$\lim\limits_{x\to0}\frac{\sqrt{x+79}-79}{x}=[\frac{0}{0}]=
\lim\limits_{x\to0}\frac{x}{x(\sqrt{x+79}+79)} = \frac{1}{\sqrt{79}+79}$$
\rozwStop
\odpStart
$\frac{1}{\sqrt{79}+79}$
\odpStop
\testStart
A.$\frac{1}{\sqrt{79}+79}$
B.$\frac{79}{\sqrt{79}+79}$
C.$0$
D.$\sqrt{79}+79$
E.$\infty$
F.$-\infty$
G.$\sqrt{79}-79$
H.$-79$
I.$79$
\testStop
\kluczStart
A
\kluczStop



\zadStart{Zadanie z Wikieł Z 4.3 a) moja wersja nr 23}
Obliczyć granicę funkcji $f(x)=\frac{\sqrt{x+83}-83}{x}$.
\zadStop
\rozwStart{Patryk Wirkus}{Szymon Tokarski}
$$\frac{\sqrt{x+83}-83}{x}=\frac{(\sqrt{x+83}-83)(\sqrt{x+83}+83)}{x(\sqrt{x+83}+83)}=\frac{1}{\sqrt{x+83}+83}$$
\\
$$\lim\limits_{x\to0}\frac{\sqrt{x+83}-83}{x}=[\frac{0}{0}]=
\lim\limits_{x\to0}\frac{x}{x(\sqrt{x+83}+83)} = \frac{1}{\sqrt{83}+83}$$
\rozwStop
\odpStart
$\frac{1}{\sqrt{83}+83}$
\odpStop
\testStart
A.$\frac{1}{\sqrt{83}+83}$
B.$\frac{83}{\sqrt{83}+83}$
C.$0$
D.$\sqrt{83}+83$
E.$\infty$
F.$-\infty$
G.$\sqrt{83}-83$
H.$-83$
I.$83$
\testStop
\kluczStart
A
\kluczStop



\zadStart{Zadanie z Wikieł Z 4.3 a) moja wersja nr 24}
Obliczyć granicę funkcji $f(x)=\frac{\sqrt{x+89}-89}{x}$.
\zadStop
\rozwStart{Patryk Wirkus}{Szymon Tokarski}
$$\frac{\sqrt{x+89}-89}{x}=\frac{(\sqrt{x+89}-89)(\sqrt{x+89}+89)}{x(\sqrt{x+89}+89)}=\frac{1}{\sqrt{x+89}+89}$$
\\
$$\lim\limits_{x\to0}\frac{\sqrt{x+89}-89}{x}=[\frac{0}{0}]=
\lim\limits_{x\to0}\frac{x}{x(\sqrt{x+89}+89)} = \frac{1}{\sqrt{89}+89}$$
\rozwStop
\odpStart
$\frac{1}{\sqrt{89}+89}$
\odpStop
\testStart
A.$\frac{1}{\sqrt{89}+89}$
B.$\frac{89}{\sqrt{89}+89}$
C.$0$
D.$\sqrt{89}+89$
E.$\infty$
F.$-\infty$
G.$\sqrt{89}-89$
H.$-89$
I.$89$
\testStop
\kluczStart
A
\kluczStop



\zadStart{Zadanie z Wikieł Z 4.3 a) moja wersja nr 25}
Obliczyć granicę funkcji $f(x)=\frac{\sqrt{x+97}-97}{x}$.
\zadStop
\rozwStart{Patryk Wirkus}{Szymon Tokarski}
$$\frac{\sqrt{x+97}-97}{x}=\frac{(\sqrt{x+97}-97)(\sqrt{x+97}+97)}{x(\sqrt{x+97}+97)}=\frac{1}{\sqrt{x+97}+97}$$
\\
$$\lim\limits_{x\to0}\frac{\sqrt{x+97}-97}{x}=[\frac{0}{0}]=
\lim\limits_{x\to0}\frac{x}{x(\sqrt{x+97}+97)} = \frac{1}{\sqrt{97}+97}$$
\rozwStop
\odpStart
$\frac{1}{\sqrt{97}+97}$
\odpStop
\testStart
A.$\frac{1}{\sqrt{97}+97}$
B.$\frac{97}{\sqrt{97}+97}$
C.$0$
D.$\sqrt{97}+97$
E.$\infty$
F.$-\infty$
G.$\sqrt{97}-97$
H.$-97$
I.$97$
\testStop
\kluczStart
A
\kluczStop



\zadStart{Zadanie z Wikieł Z 4.3 a) moja wersja nr 26}
Obliczyć granicę funkcji $f(x)=\frac{\sqrt{x+101}-101}{x}$.
\zadStop
\rozwStart{Patryk Wirkus}{Szymon Tokarski}
$$\frac{\sqrt{x+101}-101}{x}=\frac{(\sqrt{x+101}-101)(\sqrt{x+101}+101)}{x(\sqrt{x+101}+101)}=\frac{1}{\sqrt{x+101}+101}$$
\\
$$\lim\limits_{x\to0}\frac{\sqrt{x+101}-101}{x}=[\frac{0}{0}]=
\lim\limits_{x\to0}\frac{x}{x(\sqrt{x+101}+101)} = \frac{1}{\sqrt{101}+101}$$
\rozwStop
\odpStart
$\frac{1}{\sqrt{101}+101}$
\odpStop
\testStart
A.$\frac{1}{\sqrt{101}+101}$
B.$\frac{101}{\sqrt{101}+101}$
C.$0$
D.$\sqrt{101}+101$
E.$\infty$
F.$-\infty$
G.$\sqrt{101}-101$
H.$-101$
I.$101$
\testStop
\kluczStart
A
\kluczStop



\zadStart{Zadanie z Wikieł Z 4.3 a) moja wersja nr 27}
Obliczyć granicę funkcji $f(x)=\frac{\sqrt{x+103}-103}{x}$.
\zadStop
\rozwStart{Patryk Wirkus}{Szymon Tokarski}
$$\frac{\sqrt{x+103}-103}{x}=\frac{(\sqrt{x+103}-103)(\sqrt{x+103}+103)}{x(\sqrt{x+103}+103)}=\frac{1}{\sqrt{x+103}+103}$$
\\
$$\lim\limits_{x\to0}\frac{\sqrt{x+103}-103}{x}=[\frac{0}{0}]=
\lim\limits_{x\to0}\frac{x}{x(\sqrt{x+103}+103)} = \frac{1}{\sqrt{103}+103}$$
\rozwStop
\odpStart
$\frac{1}{\sqrt{103}+103}$
\odpStop
\testStart
A.$\frac{1}{\sqrt{103}+103}$
B.$\frac{103}{\sqrt{103}+103}$
C.$0$
D.$\sqrt{103}+103$
E.$\infty$
F.$-\infty$
G.$\sqrt{103}-103$
H.$-103$
I.$103$
\testStop
\kluczStart
A
\kluczStop



\zadStart{Zadanie z Wikieł Z 4.3 a) moja wersja nr 28}
Obliczyć granicę funkcji $f(x)=\frac{\sqrt{x+107}-107}{x}$.
\zadStop
\rozwStart{Patryk Wirkus}{Szymon Tokarski}
$$\frac{\sqrt{x+107}-107}{x}=\frac{(\sqrt{x+107}-107)(\sqrt{x+107}+107)}{x(\sqrt{x+107}+107)}=\frac{1}{\sqrt{x+107}+107}$$
\\
$$\lim\limits_{x\to0}\frac{\sqrt{x+107}-107}{x}=[\frac{0}{0}]=
\lim\limits_{x\to0}\frac{x}{x(\sqrt{x+107}+107)} = \frac{1}{\sqrt{107}+107}$$
\rozwStop
\odpStart
$\frac{1}{\sqrt{107}+107}$
\odpStop
\testStart
A.$\frac{1}{\sqrt{107}+107}$
B.$\frac{107}{\sqrt{107}+107}$
C.$0$
D.$\sqrt{107}+107$
E.$\infty$
F.$-\infty$
G.$\sqrt{107}-107$
H.$-107$
I.$107$
\testStop
\kluczStart
A
\kluczStop



\zadStart{Zadanie z Wikieł Z 4.3 a) moja wersja nr 29}
Obliczyć granicę funkcji $f(x)=\frac{\sqrt{x+109}-109}{x}$.
\zadStop
\rozwStart{Patryk Wirkus}{Szymon Tokarski}
$$\frac{\sqrt{x+109}-109}{x}=\frac{(\sqrt{x+109}-109)(\sqrt{x+109}+109)}{x(\sqrt{x+109}+109)}=\frac{1}{\sqrt{x+109}+109}$$
\\
$$\lim\limits_{x\to0}\frac{\sqrt{x+109}-109}{x}=[\frac{0}{0}]=
\lim\limits_{x\to0}\frac{x}{x(\sqrt{x+109}+109)} = \frac{1}{\sqrt{109}+109}$$
\rozwStop
\odpStart
$\frac{1}{\sqrt{109}+109}$
\odpStop
\testStart
A.$\frac{1}{\sqrt{109}+109}$
B.$\frac{109}{\sqrt{109}+109}$
C.$0$
D.$\sqrt{109}+109$
E.$\infty$
F.$-\infty$
G.$\sqrt{109}-109$
H.$-109$
I.$109$
\testStop
\kluczStart
A
\kluczStop



\zadStart{Zadanie z Wikieł Z 4.3 a) moja wersja nr 30}
Obliczyć granicę funkcji $f(x)=\frac{\sqrt{x+113}-113}{x}$.
\zadStop
\rozwStart{Patryk Wirkus}{Szymon Tokarski}
$$\frac{\sqrt{x+113}-113}{x}=\frac{(\sqrt{x+113}-113)(\sqrt{x+113}+113)}{x(\sqrt{x+113}+113)}=\frac{1}{\sqrt{x+113}+113}$$
\\
$$\lim\limits_{x\to0}\frac{\sqrt{x+113}-113}{x}=[\frac{0}{0}]=
\lim\limits_{x\to0}\frac{x}{x(\sqrt{x+113}+113)} = \frac{1}{\sqrt{113}+113}$$
\rozwStop
\odpStart
$\frac{1}{\sqrt{113}+113}$
\odpStop
\testStart
A.$\frac{1}{\sqrt{113}+113}$
B.$\frac{113}{\sqrt{113}+113}$
C.$0$
D.$\sqrt{113}+113$
E.$\infty$
F.$-\infty$
G.$\sqrt{113}-113$
H.$-113$
I.$113$
\testStop
\kluczStart
A
\kluczStop



\zadStart{Zadanie z Wikieł Z 4.3 a) moja wersja nr 31}
Obliczyć granicę funkcji $f(x)=\frac{\sqrt{x+127}-127}{x}$.
\zadStop
\rozwStart{Patryk Wirkus}{Szymon Tokarski}
$$\frac{\sqrt{x+127}-127}{x}=\frac{(\sqrt{x+127}-127)(\sqrt{x+127}+127)}{x(\sqrt{x+127}+127)}=\frac{1}{\sqrt{x+127}+127}$$
\\
$$\lim\limits_{x\to0}\frac{\sqrt{x+127}-127}{x}=[\frac{0}{0}]=
\lim\limits_{x\to0}\frac{x}{x(\sqrt{x+127}+127)} = \frac{1}{\sqrt{127}+127}$$
\rozwStop
\odpStart
$\frac{1}{\sqrt{127}+127}$
\odpStop
\testStart
A.$\frac{1}{\sqrt{127}+127}$
B.$\frac{127}{\sqrt{127}+127}$
C.$0$
D.$\sqrt{127}+127$
E.$\infty$
F.$-\infty$
G.$\sqrt{127}-127$
H.$-127$
I.$127$
\testStop
\kluczStart
A
\kluczStop



\zadStart{Zadanie z Wikieł Z 4.3 a) moja wersja nr 32}
Obliczyć granicę funkcji $f(x)=\frac{\sqrt{x+131}-131}{x}$.
\zadStop
\rozwStart{Patryk Wirkus}{Szymon Tokarski}
$$\frac{\sqrt{x+131}-131}{x}=\frac{(\sqrt{x+131}-131)(\sqrt{x+131}+131)}{x(\sqrt{x+131}+131)}=\frac{1}{\sqrt{x+131}+131}$$
\\
$$\lim\limits_{x\to0}\frac{\sqrt{x+131}-131}{x}=[\frac{0}{0}]=
\lim\limits_{x\to0}\frac{x}{x(\sqrt{x+131}+131)} = \frac{1}{\sqrt{131}+131}$$
\rozwStop
\odpStart
$\frac{1}{\sqrt{131}+131}$
\odpStop
\testStart
A.$\frac{1}{\sqrt{131}+131}$
B.$\frac{131}{\sqrt{131}+131}$
C.$0$
D.$\sqrt{131}+131$
E.$\infty$
F.$-\infty$
G.$\sqrt{131}-131$
H.$-131$
I.$131$
\testStop
\kluczStart
A
\kluczStop



\zadStart{Zadanie z Wikieł Z 4.3 a) moja wersja nr 33}
Obliczyć granicę funkcji $f(x)=\frac{\sqrt{x+137}-137}{x}$.
\zadStop
\rozwStart{Patryk Wirkus}{Szymon Tokarski}
$$\frac{\sqrt{x+137}-137}{x}=\frac{(\sqrt{x+137}-137)(\sqrt{x+137}+137)}{x(\sqrt{x+137}+137)}=\frac{1}{\sqrt{x+137}+137}$$
\\
$$\lim\limits_{x\to0}\frac{\sqrt{x+137}-137}{x}=[\frac{0}{0}]=
\lim\limits_{x\to0}\frac{x}{x(\sqrt{x+137}+137)} = \frac{1}{\sqrt{137}+137}$$
\rozwStop
\odpStart
$\frac{1}{\sqrt{137}+137}$
\odpStop
\testStart
A.$\frac{1}{\sqrt{137}+137}$
B.$\frac{137}{\sqrt{137}+137}$
C.$0$
D.$\sqrt{137}+137$
E.$\infty$
F.$-\infty$
G.$\sqrt{137}-137$
H.$-137$
I.$137$
\testStop
\kluczStart
A
\kluczStop



\zadStart{Zadanie z Wikieł Z 4.3 a) moja wersja nr 34}
Obliczyć granicę funkcji $f(x)=\frac{\sqrt{x+139}-139}{x}$.
\zadStop
\rozwStart{Patryk Wirkus}{Szymon Tokarski}
$$\frac{\sqrt{x+139}-139}{x}=\frac{(\sqrt{x+139}-139)(\sqrt{x+139}+139)}{x(\sqrt{x+139}+139)}=\frac{1}{\sqrt{x+139}+139}$$
\\
$$\lim\limits_{x\to0}\frac{\sqrt{x+139}-139}{x}=[\frac{0}{0}]=
\lim\limits_{x\to0}\frac{x}{x(\sqrt{x+139}+139)} = \frac{1}{\sqrt{139}+139}$$
\rozwStop
\odpStart
$\frac{1}{\sqrt{139}+139}$
\odpStop
\testStart
A.$\frac{1}{\sqrt{139}+139}$
B.$\frac{139}{\sqrt{139}+139}$
C.$0$
D.$\sqrt{139}+139$
E.$\infty$
F.$-\infty$
G.$\sqrt{139}-139$
H.$-139$
I.$139$
\testStop
\kluczStart
A
\kluczStop



\zadStart{Zadanie z Wikieł Z 4.3 a) moja wersja nr 35}
Obliczyć granicę funkcji $f(x)=\frac{\sqrt{x+149}-149}{x}$.
\zadStop
\rozwStart{Patryk Wirkus}{Szymon Tokarski}
$$\frac{\sqrt{x+149}-149}{x}=\frac{(\sqrt{x+149}-149)(\sqrt{x+149}+149)}{x(\sqrt{x+149}+149)}=\frac{1}{\sqrt{x+149}+149}$$
\\
$$\lim\limits_{x\to0}\frac{\sqrt{x+149}-149}{x}=[\frac{0}{0}]=
\lim\limits_{x\to0}\frac{x}{x(\sqrt{x+149}+149)} = \frac{1}{\sqrt{149}+149}$$
\rozwStop
\odpStart
$\frac{1}{\sqrt{149}+149}$
\odpStop
\testStart
A.$\frac{1}{\sqrt{149}+149}$
B.$\frac{149}{\sqrt{149}+149}$
C.$0$
D.$\sqrt{149}+149$
E.$\infty$
F.$-\infty$
G.$\sqrt{149}-149$
H.$-149$
I.$149$
\testStop
\kluczStart
A
\kluczStop



\zadStart{Zadanie z Wikieł Z 4.3 a) moja wersja nr 36}
Obliczyć granicę funkcji $f(x)=\frac{\sqrt{x+151}-151}{x}$.
\zadStop
\rozwStart{Patryk Wirkus}{Szymon Tokarski}
$$\frac{\sqrt{x+151}-151}{x}=\frac{(\sqrt{x+151}-151)(\sqrt{x+151}+151)}{x(\sqrt{x+151}+151)}=\frac{1}{\sqrt{x+151}+151}$$
\\
$$\lim\limits_{x\to0}\frac{\sqrt{x+151}-151}{x}=[\frac{0}{0}]=
\lim\limits_{x\to0}\frac{x}{x(\sqrt{x+151}+151)} = \frac{1}{\sqrt{151}+151}$$
\rozwStop
\odpStart
$\frac{1}{\sqrt{151}+151}$
\odpStop
\testStart
A.$\frac{1}{\sqrt{151}+151}$
B.$\frac{151}{\sqrt{151}+151}$
C.$0$
D.$\sqrt{151}+151$
E.$\infty$
F.$-\infty$
G.$\sqrt{151}-151$
H.$-151$
I.$151$
\testStop
\kluczStart
A
\kluczStop



\zadStart{Zadanie z Wikieł Z 4.3 a) moja wersja nr 37}
Obliczyć granicę funkcji $f(x)=\frac{\sqrt{x+157}-157}{x}$.
\zadStop
\rozwStart{Patryk Wirkus}{Szymon Tokarski}
$$\frac{\sqrt{x+157}-157}{x}=\frac{(\sqrt{x+157}-157)(\sqrt{x+157}+157)}{x(\sqrt{x+157}+157)}=\frac{1}{\sqrt{x+157}+157}$$
\\
$$\lim\limits_{x\to0}\frac{\sqrt{x+157}-157}{x}=[\frac{0}{0}]=
\lim\limits_{x\to0}\frac{x}{x(\sqrt{x+157}+157)} = \frac{1}{\sqrt{157}+157}$$
\rozwStop
\odpStart
$\frac{1}{\sqrt{157}+157}$
\odpStop
\testStart
A.$\frac{1}{\sqrt{157}+157}$
B.$\frac{157}{\sqrt{157}+157}$
C.$0$
D.$\sqrt{157}+157$
E.$\infty$
F.$-\infty$
G.$\sqrt{157}-157$
H.$-157$
I.$157$
\testStop
\kluczStart
A
\kluczStop



\zadStart{Zadanie z Wikieł Z 4.3 a) moja wersja nr 38}
Obliczyć granicę funkcji $f(x)=\frac{\sqrt{x+163}-163}{x}$.
\zadStop
\rozwStart{Patryk Wirkus}{Szymon Tokarski}
$$\frac{\sqrt{x+163}-163}{x}=\frac{(\sqrt{x+163}-163)(\sqrt{x+163}+163)}{x(\sqrt{x+163}+163)}=\frac{1}{\sqrt{x+163}+163}$$
\\
$$\lim\limits_{x\to0}\frac{\sqrt{x+163}-163}{x}=[\frac{0}{0}]=
\lim\limits_{x\to0}\frac{x}{x(\sqrt{x+163}+163)} = \frac{1}{\sqrt{163}+163}$$
\rozwStop
\odpStart
$\frac{1}{\sqrt{163}+163}$
\odpStop
\testStart
A.$\frac{1}{\sqrt{163}+163}$
B.$\frac{163}{\sqrt{163}+163}$
C.$0$
D.$\sqrt{163}+163$
E.$\infty$
F.$-\infty$
G.$\sqrt{163}-163$
H.$-163$
I.$163$
\testStop
\kluczStart
A
\kluczStop



\zadStart{Zadanie z Wikieł Z 4.3 a) moja wersja nr 39}
Obliczyć granicę funkcji $f(x)=\frac{\sqrt{x+167}-167}{x}$.
\zadStop
\rozwStart{Patryk Wirkus}{Szymon Tokarski}
$$\frac{\sqrt{x+167}-167}{x}=\frac{(\sqrt{x+167}-167)(\sqrt{x+167}+167)}{x(\sqrt{x+167}+167)}=\frac{1}{\sqrt{x+167}+167}$$
\\
$$\lim\limits_{x\to0}\frac{\sqrt{x+167}-167}{x}=[\frac{0}{0}]=
\lim\limits_{x\to0}\frac{x}{x(\sqrt{x+167}+167)} = \frac{1}{\sqrt{167}+167}$$
\rozwStop
\odpStart
$\frac{1}{\sqrt{167}+167}$
\odpStop
\testStart
A.$\frac{1}{\sqrt{167}+167}$
B.$\frac{167}{\sqrt{167}+167}$
C.$0$
D.$\sqrt{167}+167$
E.$\infty$
F.$-\infty$
G.$\sqrt{167}-167$
H.$-167$
I.$167$
\testStop
\kluczStart
A
\kluczStop



\zadStart{Zadanie z Wikieł Z 4.3 a) moja wersja nr 40}
Obliczyć granicę funkcji $f(x)=\frac{\sqrt{x+173}-173}{x}$.
\zadStop
\rozwStart{Patryk Wirkus}{Szymon Tokarski}
$$\frac{\sqrt{x+173}-173}{x}=\frac{(\sqrt{x+173}-173)(\sqrt{x+173}+173)}{x(\sqrt{x+173}+173)}=\frac{1}{\sqrt{x+173}+173}$$
\\
$$\lim\limits_{x\to0}\frac{\sqrt{x+173}-173}{x}=[\frac{0}{0}]=
\lim\limits_{x\to0}\frac{x}{x(\sqrt{x+173}+173)} = \frac{1}{\sqrt{173}+173}$$
\rozwStop
\odpStart
$\frac{1}{\sqrt{173}+173}$
\odpStop
\testStart
A.$\frac{1}{\sqrt{173}+173}$
B.$\frac{173}{\sqrt{173}+173}$
C.$0$
D.$\sqrt{173}+173$
E.$\infty$
F.$-\infty$
G.$\sqrt{173}-173$
H.$-173$
I.$173$
\testStop
\kluczStart
A
\kluczStop



\zadStart{Zadanie z Wikieł Z 4.3 a) moja wersja nr 41}
Obliczyć granicę funkcji $f(x)=\frac{\sqrt{x+179}-179}{x}$.
\zadStop
\rozwStart{Patryk Wirkus}{Szymon Tokarski}
$$\frac{\sqrt{x+179}-179}{x}=\frac{(\sqrt{x+179}-179)(\sqrt{x+179}+179)}{x(\sqrt{x+179}+179)}=\frac{1}{\sqrt{x+179}+179}$$
\\
$$\lim\limits_{x\to0}\frac{\sqrt{x+179}-179}{x}=[\frac{0}{0}]=
\lim\limits_{x\to0}\frac{x}{x(\sqrt{x+179}+179)} = \frac{1}{\sqrt{179}+179}$$
\rozwStop
\odpStart
$\frac{1}{\sqrt{179}+179}$
\odpStop
\testStart
A.$\frac{1}{\sqrt{179}+179}$
B.$\frac{179}{\sqrt{179}+179}$
C.$0$
D.$\sqrt{179}+179$
E.$\infty$
F.$-\infty$
G.$\sqrt{179}-179$
H.$-179$
I.$179$
\testStop
\kluczStart
A
\kluczStop



\zadStart{Zadanie z Wikieł Z 4.3 a) moja wersja nr 42}
Obliczyć granicę funkcji $f(x)=\frac{\sqrt{x+181}-181}{x}$.
\zadStop
\rozwStart{Patryk Wirkus}{Szymon Tokarski}
$$\frac{\sqrt{x+181}-181}{x}=\frac{(\sqrt{x+181}-181)(\sqrt{x+181}+181)}{x(\sqrt{x+181}+181)}=\frac{1}{\sqrt{x+181}+181}$$
\\
$$\lim\limits_{x\to0}\frac{\sqrt{x+181}-181}{x}=[\frac{0}{0}]=
\lim\limits_{x\to0}\frac{x}{x(\sqrt{x+181}+181)} = \frac{1}{\sqrt{181}+181}$$
\rozwStop
\odpStart
$\frac{1}{\sqrt{181}+181}$
\odpStop
\testStart
A.$\frac{1}{\sqrt{181}+181}$
B.$\frac{181}{\sqrt{181}+181}$
C.$0$
D.$\sqrt{181}+181$
E.$\infty$
F.$-\infty$
G.$\sqrt{181}-181$
H.$-181$
I.$181$
\testStop
\kluczStart
A
\kluczStop



\zadStart{Zadanie z Wikieł Z 4.3 a) moja wersja nr 43}
Obliczyć granicę funkcji $f(x)=\frac{\sqrt{x+191}-191}{x}$.
\zadStop
\rozwStart{Patryk Wirkus}{Szymon Tokarski}
$$\frac{\sqrt{x+191}-191}{x}=\frac{(\sqrt{x+191}-191)(\sqrt{x+191}+191)}{x(\sqrt{x+191}+191)}=\frac{1}{\sqrt{x+191}+191}$$
\\
$$\lim\limits_{x\to0}\frac{\sqrt{x+191}-191}{x}=[\frac{0}{0}]=
\lim\limits_{x\to0}\frac{x}{x(\sqrt{x+191}+191)} = \frac{1}{\sqrt{191}+191}$$
\rozwStop
\odpStart
$\frac{1}{\sqrt{191}+191}$
\odpStop
\testStart
A.$\frac{1}{\sqrt{191}+191}$
B.$\frac{191}{\sqrt{191}+191}$
C.$0$
D.$\sqrt{191}+191$
E.$\infty$
F.$-\infty$
G.$\sqrt{191}-191$
H.$-191$
I.$191$
\testStop
\kluczStart
A
\kluczStop



\zadStart{Zadanie z Wikieł Z 4.3 a) moja wersja nr 44}
Obliczyć granicę funkcji $f(x)=\frac{\sqrt{x+193}-193}{x}$.
\zadStop
\rozwStart{Patryk Wirkus}{Szymon Tokarski}
$$\frac{\sqrt{x+193}-193}{x}=\frac{(\sqrt{x+193}-193)(\sqrt{x+193}+193)}{x(\sqrt{x+193}+193)}=\frac{1}{\sqrt{x+193}+193}$$
\\
$$\lim\limits_{x\to0}\frac{\sqrt{x+193}-193}{x}=[\frac{0}{0}]=
\lim\limits_{x\to0}\frac{x}{x(\sqrt{x+193}+193)} = \frac{1}{\sqrt{193}+193}$$
\rozwStop
\odpStart
$\frac{1}{\sqrt{193}+193}$
\odpStop
\testStart
A.$\frac{1}{\sqrt{193}+193}$
B.$\frac{193}{\sqrt{193}+193}$
C.$0$
D.$\sqrt{193}+193$
E.$\infty$
F.$-\infty$
G.$\sqrt{193}-193$
H.$-193$
I.$193$
\testStop
\kluczStart
A
\kluczStop



\zadStart{Zadanie z Wikieł Z 4.3 a) moja wersja nr 45}
Obliczyć granicę funkcji $f(x)=\frac{\sqrt{x+197}-197}{x}$.
\zadStop
\rozwStart{Patryk Wirkus}{Szymon Tokarski}
$$\frac{\sqrt{x+197}-197}{x}=\frac{(\sqrt{x+197}-197)(\sqrt{x+197}+197)}{x(\sqrt{x+197}+197)}=\frac{1}{\sqrt{x+197}+197}$$
\\
$$\lim\limits_{x\to0}\frac{\sqrt{x+197}-197}{x}=[\frac{0}{0}]=
\lim\limits_{x\to0}\frac{x}{x(\sqrt{x+197}+197)} = \frac{1}{\sqrt{197}+197}$$
\rozwStop
\odpStart
$\frac{1}{\sqrt{197}+197}$
\odpStop
\testStart
A.$\frac{1}{\sqrt{197}+197}$
B.$\frac{197}{\sqrt{197}+197}$
C.$0$
D.$\sqrt{197}+197$
E.$\infty$
F.$-\infty$
G.$\sqrt{197}-197$
H.$-197$
I.$197$
\testStop
\kluczStart
A
\kluczStop



\zadStart{Zadanie z Wikieł Z 4.3 a) moja wersja nr 46}
Obliczyć granicę funkcji $f(x)=\frac{\sqrt{x+199}-199}{x}$.
\zadStop
\rozwStart{Patryk Wirkus}{Szymon Tokarski}
$$\frac{\sqrt{x+199}-199}{x}=\frac{(\sqrt{x+199}-199)(\sqrt{x+199}+199)}{x(\sqrt{x+199}+199)}=\frac{1}{\sqrt{x+199}+199}$$
\\
$$\lim\limits_{x\to0}\frac{\sqrt{x+199}-199}{x}=[\frac{0}{0}]=
\lim\limits_{x\to0}\frac{x}{x(\sqrt{x+199}+199)} = \frac{1}{\sqrt{199}+199}$$
\rozwStop
\odpStart
$\frac{1}{\sqrt{199}+199}$
\odpStop
\testStart
A.$\frac{1}{\sqrt{199}+199}$
B.$\frac{199}{\sqrt{199}+199}$
C.$0$
D.$\sqrt{199}+199$
E.$\infty$
F.$-\infty$
G.$\sqrt{199}-199$
H.$-199$
I.$199$
\testStop
\kluczStart
A
\kluczStop



\zadStart{Zadanie z Wikieł Z 4.3 a) moja wersja nr 47}
Obliczyć granicę funkcji $f(x)=\frac{\sqrt{x+211}-211}{x}$.
\zadStop
\rozwStart{Patryk Wirkus}{Szymon Tokarski}
$$\frac{\sqrt{x+211}-211}{x}=\frac{(\sqrt{x+211}-211)(\sqrt{x+211}+211)}{x(\sqrt{x+211}+211)}=\frac{1}{\sqrt{x+211}+211}$$
\\
$$\lim\limits_{x\to0}\frac{\sqrt{x+211}-211}{x}=[\frac{0}{0}]=
\lim\limits_{x\to0}\frac{x}{x(\sqrt{x+211}+211)} = \frac{1}{\sqrt{211}+211}$$
\rozwStop
\odpStart
$\frac{1}{\sqrt{211}+211}$
\odpStop
\testStart
A.$\frac{1}{\sqrt{211}+211}$
B.$\frac{211}{\sqrt{211}+211}$
C.$0$
D.$\sqrt{211}+211$
E.$\infty$
F.$-\infty$
G.$\sqrt{211}-211$
H.$-211$
I.$211$
\testStop
\kluczStart
A
\kluczStop



\zadStart{Zadanie z Wikieł Z 4.3 a) moja wersja nr 48}
Obliczyć granicę funkcji $f(x)=\frac{\sqrt{x+223}-223}{x}$.
\zadStop
\rozwStart{Patryk Wirkus}{Szymon Tokarski}
$$\frac{\sqrt{x+223}-223}{x}=\frac{(\sqrt{x+223}-223)(\sqrt{x+223}+223)}{x(\sqrt{x+223}+223)}=\frac{1}{\sqrt{x+223}+223}$$
\\
$$\lim\limits_{x\to0}\frac{\sqrt{x+223}-223}{x}=[\frac{0}{0}]=
\lim\limits_{x\to0}\frac{x}{x(\sqrt{x+223}+223)} = \frac{1}{\sqrt{223}+223}$$
\rozwStop
\odpStart
$\frac{1}{\sqrt{223}+223}$
\odpStop
\testStart
A.$\frac{1}{\sqrt{223}+223}$
B.$\frac{223}{\sqrt{223}+223}$
C.$0$
D.$\sqrt{223}+223$
E.$\infty$
F.$-\infty$
G.$\sqrt{223}-223$
H.$-223$
I.$223$
\testStop
\kluczStart
A
\kluczStop



\zadStart{Zadanie z Wikieł Z 4.3 a) moja wersja nr 49}
Obliczyć granicę funkcji $f(x)=\frac{\sqrt{x+227}-227}{x}$.
\zadStop
\rozwStart{Patryk Wirkus}{Szymon Tokarski}
$$\frac{\sqrt{x+227}-227}{x}=\frac{(\sqrt{x+227}-227)(\sqrt{x+227}+227)}{x(\sqrt{x+227}+227)}=\frac{1}{\sqrt{x+227}+227}$$
\\
$$\lim\limits_{x\to0}\frac{\sqrt{x+227}-227}{x}=[\frac{0}{0}]=
\lim\limits_{x\to0}\frac{x}{x(\sqrt{x+227}+227)} = \frac{1}{\sqrt{227}+227}$$
\rozwStop
\odpStart
$\frac{1}{\sqrt{227}+227}$
\odpStop
\testStart
A.$\frac{1}{\sqrt{227}+227}$
B.$\frac{227}{\sqrt{227}+227}$
C.$0$
D.$\sqrt{227}+227$
E.$\infty$
F.$-\infty$
G.$\sqrt{227}-227$
H.$-227$
I.$227$
\testStop
\kluczStart
A
\kluczStop



\zadStart{Zadanie z Wikieł Z 4.3 a) moja wersja nr 50}
Obliczyć granicę funkcji $f(x)=\frac{\sqrt{x+229}-229}{x}$.
\zadStop
\rozwStart{Patryk Wirkus}{Szymon Tokarski}
$$\frac{\sqrt{x+229}-229}{x}=\frac{(\sqrt{x+229}-229)(\sqrt{x+229}+229)}{x(\sqrt{x+229}+229)}=\frac{1}{\sqrt{x+229}+229}$$
\\
$$\lim\limits_{x\to0}\frac{\sqrt{x+229}-229}{x}=[\frac{0}{0}]=
\lim\limits_{x\to0}\frac{x}{x(\sqrt{x+229}+229)} = \frac{1}{\sqrt{229}+229}$$
\rozwStop
\odpStart
$\frac{1}{\sqrt{229}+229}$
\odpStop
\testStart
A.$\frac{1}{\sqrt{229}+229}$
B.$\frac{229}{\sqrt{229}+229}$
C.$0$
D.$\sqrt{229}+229$
E.$\infty$
F.$-\infty$
G.$\sqrt{229}-229$
H.$-229$
I.$229$
\testStop
\kluczStart
A
\kluczStop



\zadStart{Zadanie z Wikieł Z 4.3 a) moja wersja nr 51}
Obliczyć granicę funkcji $f(x)=\frac{\sqrt{x+233}-233}{x}$.
\zadStop
\rozwStart{Patryk Wirkus}{Szymon Tokarski}
$$\frac{\sqrt{x+233}-233}{x}=\frac{(\sqrt{x+233}-233)(\sqrt{x+233}+233)}{x(\sqrt{x+233}+233)}=\frac{1}{\sqrt{x+233}+233}$$
\\
$$\lim\limits_{x\to0}\frac{\sqrt{x+233}-233}{x}=[\frac{0}{0}]=
\lim\limits_{x\to0}\frac{x}{x(\sqrt{x+233}+233)} = \frac{1}{\sqrt{233}+233}$$
\rozwStop
\odpStart
$\frac{1}{\sqrt{233}+233}$
\odpStop
\testStart
A.$\frac{1}{\sqrt{233}+233}$
B.$\frac{233}{\sqrt{233}+233}$
C.$0$
D.$\sqrt{233}+233$
E.$\infty$
F.$-\infty$
G.$\sqrt{233}-233$
H.$-233$
I.$233$
\testStop
\kluczStart
A
\kluczStop



\zadStart{Zadanie z Wikieł Z 4.3 a) moja wersja nr 52}
Obliczyć granicę funkcji $f(x)=\frac{\sqrt{x+239}-239}{x}$.
\zadStop
\rozwStart{Patryk Wirkus}{Szymon Tokarski}
$$\frac{\sqrt{x+239}-239}{x}=\frac{(\sqrt{x+239}-239)(\sqrt{x+239}+239)}{x(\sqrt{x+239}+239)}=\frac{1}{\sqrt{x+239}+239}$$
\\
$$\lim\limits_{x\to0}\frac{\sqrt{x+239}-239}{x}=[\frac{0}{0}]=
\lim\limits_{x\to0}\frac{x}{x(\sqrt{x+239}+239)} = \frac{1}{\sqrt{239}+239}$$
\rozwStop
\odpStart
$\frac{1}{\sqrt{239}+239}$
\odpStop
\testStart
A.$\frac{1}{\sqrt{239}+239}$
B.$\frac{239}{\sqrt{239}+239}$
C.$0$
D.$\sqrt{239}+239$
E.$\infty$
F.$-\infty$
G.$\sqrt{239}-239$
H.$-239$
I.$239$
\testStop
\kluczStart
A
\kluczStop



\zadStart{Zadanie z Wikieł Z 4.3 a) moja wersja nr 53}
Obliczyć granicę funkcji $f(x)=\frac{\sqrt{x+241}-241}{x}$.
\zadStop
\rozwStart{Patryk Wirkus}{Szymon Tokarski}
$$\frac{\sqrt{x+241}-241}{x}=\frac{(\sqrt{x+241}-241)(\sqrt{x+241}+241)}{x(\sqrt{x+241}+241)}=\frac{1}{\sqrt{x+241}+241}$$
\\
$$\lim\limits_{x\to0}\frac{\sqrt{x+241}-241}{x}=[\frac{0}{0}]=
\lim\limits_{x\to0}\frac{x}{x(\sqrt{x+241}+241)} = \frac{1}{\sqrt{241}+241}$$
\rozwStop
\odpStart
$\frac{1}{\sqrt{241}+241}$
\odpStop
\testStart
A.$\frac{1}{\sqrt{241}+241}$
B.$\frac{241}{\sqrt{241}+241}$
C.$0$
D.$\sqrt{241}+241$
E.$\infty$
F.$-\infty$
G.$\sqrt{241}-241$
H.$-241$
I.$241$
\testStop
\kluczStart
A
\kluczStop



\zadStart{Zadanie z Wikieł Z 4.3 a) moja wersja nr 54}
Obliczyć granicę funkcji $f(x)=\frac{\sqrt{x+251}-251}{x}$.
\zadStop
\rozwStart{Patryk Wirkus}{Szymon Tokarski}
$$\frac{\sqrt{x+251}-251}{x}=\frac{(\sqrt{x+251}-251)(\sqrt{x+251}+251)}{x(\sqrt{x+251}+251)}=\frac{1}{\sqrt{x+251}+251}$$
\\
$$\lim\limits_{x\to0}\frac{\sqrt{x+251}-251}{x}=[\frac{0}{0}]=
\lim\limits_{x\to0}\frac{x}{x(\sqrt{x+251}+251)} = \frac{1}{\sqrt{251}+251}$$
\rozwStop
\odpStart
$\frac{1}{\sqrt{251}+251}$
\odpStop
\testStart
A.$\frac{1}{\sqrt{251}+251}$
B.$\frac{251}{\sqrt{251}+251}$
C.$0$
D.$\sqrt{251}+251$
E.$\infty$
F.$-\infty$
G.$\sqrt{251}-251$
H.$-251$
I.$251$
\testStop
\kluczStart
A
\kluczStop



\zadStart{Zadanie z Wikieł Z 4.3 a) moja wersja nr 55}
Obliczyć granicę funkcji $f(x)=\frac{\sqrt{x+257}-257}{x}$.
\zadStop
\rozwStart{Patryk Wirkus}{Szymon Tokarski}
$$\frac{\sqrt{x+257}-257}{x}=\frac{(\sqrt{x+257}-257)(\sqrt{x+257}+257)}{x(\sqrt{x+257}+257)}=\frac{1}{\sqrt{x+257}+257}$$
\\
$$\lim\limits_{x\to0}\frac{\sqrt{x+257}-257}{x}=[\frac{0}{0}]=
\lim\limits_{x\to0}\frac{x}{x(\sqrt{x+257}+257)} = \frac{1}{\sqrt{257}+257}$$
\rozwStop
\odpStart
$\frac{1}{\sqrt{257}+257}$
\odpStop
\testStart
A.$\frac{1}{\sqrt{257}+257}$
B.$\frac{257}{\sqrt{257}+257}$
C.$0$
D.$\sqrt{257}+257$
E.$\infty$
F.$-\infty$
G.$\sqrt{257}-257$
H.$-257$
I.$257$
\testStop
\kluczStart
A
\kluczStop



\zadStart{Zadanie z Wikieł Z 4.3 a) moja wersja nr 56}
Obliczyć granicę funkcji $f(x)=\frac{\sqrt{x+263}-263}{x}$.
\zadStop
\rozwStart{Patryk Wirkus}{Szymon Tokarski}
$$\frac{\sqrt{x+263}-263}{x}=\frac{(\sqrt{x+263}-263)(\sqrt{x+263}+263)}{x(\sqrt{x+263}+263)}=\frac{1}{\sqrt{x+263}+263}$$
\\
$$\lim\limits_{x\to0}\frac{\sqrt{x+263}-263}{x}=[\frac{0}{0}]=
\lim\limits_{x\to0}\frac{x}{x(\sqrt{x+263}+263)} = \frac{1}{\sqrt{263}+263}$$
\rozwStop
\odpStart
$\frac{1}{\sqrt{263}+263}$
\odpStop
\testStart
A.$\frac{1}{\sqrt{263}+263}$
B.$\frac{263}{\sqrt{263}+263}$
C.$0$
D.$\sqrt{263}+263$
E.$\infty$
F.$-\infty$
G.$\sqrt{263}-263$
H.$-263$
I.$263$
\testStop
\kluczStart
A
\kluczStop



\zadStart{Zadanie z Wikieł Z 4.3 a) moja wersja nr 57}
Obliczyć granicę funkcji $f(x)=\frac{\sqrt{x+269}-269}{x}$.
\zadStop
\rozwStart{Patryk Wirkus}{Szymon Tokarski}
$$\frac{\sqrt{x+269}-269}{x}=\frac{(\sqrt{x+269}-269)(\sqrt{x+269}+269)}{x(\sqrt{x+269}+269)}=\frac{1}{\sqrt{x+269}+269}$$
\\
$$\lim\limits_{x\to0}\frac{\sqrt{x+269}-269}{x}=[\frac{0}{0}]=
\lim\limits_{x\to0}\frac{x}{x(\sqrt{x+269}+269)} = \frac{1}{\sqrt{269}+269}$$
\rozwStop
\odpStart
$\frac{1}{\sqrt{269}+269}$
\odpStop
\testStart
A.$\frac{1}{\sqrt{269}+269}$
B.$\frac{269}{\sqrt{269}+269}$
C.$0$
D.$\sqrt{269}+269$
E.$\infty$
F.$-\infty$
G.$\sqrt{269}-269$
H.$-269$
I.$269$
\testStop
\kluczStart
A
\kluczStop



\zadStart{Zadanie z Wikieł Z 4.3 a) moja wersja nr 58}
Obliczyć granicę funkcji $f(x)=\frac{\sqrt{x+271}-271}{x}$.
\zadStop
\rozwStart{Patryk Wirkus}{Szymon Tokarski}
$$\frac{\sqrt{x+271}-271}{x}=\frac{(\sqrt{x+271}-271)(\sqrt{x+271}+271)}{x(\sqrt{x+271}+271)}=\frac{1}{\sqrt{x+271}+271}$$
\\
$$\lim\limits_{x\to0}\frac{\sqrt{x+271}-271}{x}=[\frac{0}{0}]=
\lim\limits_{x\to0}\frac{x}{x(\sqrt{x+271}+271)} = \frac{1}{\sqrt{271}+271}$$
\rozwStop
\odpStart
$\frac{1}{\sqrt{271}+271}$
\odpStop
\testStart
A.$\frac{1}{\sqrt{271}+271}$
B.$\frac{271}{\sqrt{271}+271}$
C.$0$
D.$\sqrt{271}+271$
E.$\infty$
F.$-\infty$
G.$\sqrt{271}-271$
H.$-271$
I.$271$
\testStop
\kluczStart
A
\kluczStop



\zadStart{Zadanie z Wikieł Z 4.3 a) moja wersja nr 59}
Obliczyć granicę funkcji $f(x)=\frac{\sqrt{x+277}-277}{x}$.
\zadStop
\rozwStart{Patryk Wirkus}{Szymon Tokarski}
$$\frac{\sqrt{x+277}-277}{x}=\frac{(\sqrt{x+277}-277)(\sqrt{x+277}+277)}{x(\sqrt{x+277}+277)}=\frac{1}{\sqrt{x+277}+277}$$
\\
$$\lim\limits_{x\to0}\frac{\sqrt{x+277}-277}{x}=[\frac{0}{0}]=
\lim\limits_{x\to0}\frac{x}{x(\sqrt{x+277}+277)} = \frac{1}{\sqrt{277}+277}$$
\rozwStop
\odpStart
$\frac{1}{\sqrt{277}+277}$
\odpStop
\testStart
A.$\frac{1}{\sqrt{277}+277}$
B.$\frac{277}{\sqrt{277}+277}$
C.$0$
D.$\sqrt{277}+277$
E.$\infty$
F.$-\infty$
G.$\sqrt{277}-277$
H.$-277$
I.$277$
\testStop
\kluczStart
A
\kluczStop



\zadStart{Zadanie z Wikieł Z 4.3 a) moja wersja nr 60}
Obliczyć granicę funkcji $f(x)=\frac{\sqrt{x+281}-281}{x}$.
\zadStop
\rozwStart{Patryk Wirkus}{Szymon Tokarski}
$$\frac{\sqrt{x+281}-281}{x}=\frac{(\sqrt{x+281}-281)(\sqrt{x+281}+281)}{x(\sqrt{x+281}+281)}=\frac{1}{\sqrt{x+281}+281}$$
\\
$$\lim\limits_{x\to0}\frac{\sqrt{x+281}-281}{x}=[\frac{0}{0}]=
\lim\limits_{x\to0}\frac{x}{x(\sqrt{x+281}+281)} = \frac{1}{\sqrt{281}+281}$$
\rozwStop
\odpStart
$\frac{1}{\sqrt{281}+281}$
\odpStop
\testStart
A.$\frac{1}{\sqrt{281}+281}$
B.$\frac{281}{\sqrt{281}+281}$
C.$0$
D.$\sqrt{281}+281$
E.$\infty$
F.$-\infty$
G.$\sqrt{281}-281$
H.$-281$
I.$281$
\testStop
\kluczStart
A
\kluczStop



\zadStart{Zadanie z Wikieł Z 4.3 a) moja wersja nr 61}
Obliczyć granicę funkcji $f(x)=\frac{\sqrt{x+283}-283}{x}$.
\zadStop
\rozwStart{Patryk Wirkus}{Szymon Tokarski}
$$\frac{\sqrt{x+283}-283}{x}=\frac{(\sqrt{x+283}-283)(\sqrt{x+283}+283)}{x(\sqrt{x+283}+283)}=\frac{1}{\sqrt{x+283}+283}$$
\\
$$\lim\limits_{x\to0}\frac{\sqrt{x+283}-283}{x}=[\frac{0}{0}]=
\lim\limits_{x\to0}\frac{x}{x(\sqrt{x+283}+283)} = \frac{1}{\sqrt{283}+283}$$
\rozwStop
\odpStart
$\frac{1}{\sqrt{283}+283}$
\odpStop
\testStart
A.$\frac{1}{\sqrt{283}+283}$
B.$\frac{283}{\sqrt{283}+283}$
C.$0$
D.$\sqrt{283}+283$
E.$\infty$
F.$-\infty$
G.$\sqrt{283}-283$
H.$-283$
I.$283$
\testStop
\kluczStart
A
\kluczStop



\zadStart{Zadanie z Wikieł Z 4.3 a) moja wersja nr 62}
Obliczyć granicę funkcji $f(x)=\frac{\sqrt{x+293}-293}{x}$.
\zadStop
\rozwStart{Patryk Wirkus}{Szymon Tokarski}
$$\frac{\sqrt{x+293}-293}{x}=\frac{(\sqrt{x+293}-293)(\sqrt{x+293}+293)}{x(\sqrt{x+293}+293)}=\frac{1}{\sqrt{x+293}+293}$$
\\
$$\lim\limits_{x\to0}\frac{\sqrt{x+293}-293}{x}=[\frac{0}{0}]=
\lim\limits_{x\to0}\frac{x}{x(\sqrt{x+293}+293)} = \frac{1}{\sqrt{293}+293}$$
\rozwStop
\odpStart
$\frac{1}{\sqrt{293}+293}$
\odpStop
\testStart
A.$\frac{1}{\sqrt{293}+293}$
B.$\frac{293}{\sqrt{293}+293}$
C.$0$
D.$\sqrt{293}+293$
E.$\infty$
F.$-\infty$
G.$\sqrt{293}-293$
H.$-293$
I.$293$
\testStop
\kluczStart
A
\kluczStop



\zadStart{Zadanie z Wikieł Z 4.3 a) moja wersja nr 63}
Obliczyć granicę funkcji $f(x)=\frac{\sqrt{x+307}-307}{x}$.
\zadStop
\rozwStart{Patryk Wirkus}{Szymon Tokarski}
$$\frac{\sqrt{x+307}-307}{x}=\frac{(\sqrt{x+307}-307)(\sqrt{x+307}+307)}{x(\sqrt{x+307}+307)}=\frac{1}{\sqrt{x+307}+307}$$
\\
$$\lim\limits_{x\to0}\frac{\sqrt{x+307}-307}{x}=[\frac{0}{0}]=
\lim\limits_{x\to0}\frac{x}{x(\sqrt{x+307}+307)} = \frac{1}{\sqrt{307}+307}$$
\rozwStop
\odpStart
$\frac{1}{\sqrt{307}+307}$
\odpStop
\testStart
A.$\frac{1}{\sqrt{307}+307}$
B.$\frac{307}{\sqrt{307}+307}$
C.$0$
D.$\sqrt{307}+307$
E.$\infty$
F.$-\infty$
G.$\sqrt{307}-307$
H.$-307$
I.$307$
\testStop
\kluczStart
A
\kluczStop



\zadStart{Zadanie z Wikieł Z 4.3 a) moja wersja nr 64}
Obliczyć granicę funkcji $f(x)=\frac{\sqrt{x+311}-311}{x}$.
\zadStop
\rozwStart{Patryk Wirkus}{Szymon Tokarski}
$$\frac{\sqrt{x+311}-311}{x}=\frac{(\sqrt{x+311}-311)(\sqrt{x+311}+311)}{x(\sqrt{x+311}+311)}=\frac{1}{\sqrt{x+311}+311}$$
\\
$$\lim\limits_{x\to0}\frac{\sqrt{x+311}-311}{x}=[\frac{0}{0}]=
\lim\limits_{x\to0}\frac{x}{x(\sqrt{x+311}+311)} = \frac{1}{\sqrt{311}+311}$$
\rozwStop
\odpStart
$\frac{1}{\sqrt{311}+311}$
\odpStop
\testStart
A.$\frac{1}{\sqrt{311}+311}$
B.$\frac{311}{\sqrt{311}+311}$
C.$0$
D.$\sqrt{311}+311$
E.$\infty$
F.$-\infty$
G.$\sqrt{311}-311$
H.$-311$
I.$311$
\testStop
\kluczStart
A
\kluczStop



\zadStart{Zadanie z Wikieł Z 4.3 a) moja wersja nr 65}
Obliczyć granicę funkcji $f(x)=\frac{\sqrt{x+313}-313}{x}$.
\zadStop
\rozwStart{Patryk Wirkus}{Szymon Tokarski}
$$\frac{\sqrt{x+313}-313}{x}=\frac{(\sqrt{x+313}-313)(\sqrt{x+313}+313)}{x(\sqrt{x+313}+313)}=\frac{1}{\sqrt{x+313}+313}$$
\\
$$\lim\limits_{x\to0}\frac{\sqrt{x+313}-313}{x}=[\frac{0}{0}]=
\lim\limits_{x\to0}\frac{x}{x(\sqrt{x+313}+313)} = \frac{1}{\sqrt{313}+313}$$
\rozwStop
\odpStart
$\frac{1}{\sqrt{313}+313}$
\odpStop
\testStart
A.$\frac{1}{\sqrt{313}+313}$
B.$\frac{313}{\sqrt{313}+313}$
C.$0$
D.$\sqrt{313}+313$
E.$\infty$
F.$-\infty$
G.$\sqrt{313}-313$
H.$-313$
I.$313$
\testStop
\kluczStart
A
\kluczStop



\zadStart{Zadanie z Wikieł Z 4.3 a) moja wersja nr 66}
Obliczyć granicę funkcji $f(x)=\frac{\sqrt{x+317}-317}{x}$.
\zadStop
\rozwStart{Patryk Wirkus}{Szymon Tokarski}
$$\frac{\sqrt{x+317}-317}{x}=\frac{(\sqrt{x+317}-317)(\sqrt{x+317}+317)}{x(\sqrt{x+317}+317)}=\frac{1}{\sqrt{x+317}+317}$$
\\
$$\lim\limits_{x\to0}\frac{\sqrt{x+317}-317}{x}=[\frac{0}{0}]=
\lim\limits_{x\to0}\frac{x}{x(\sqrt{x+317}+317)} = \frac{1}{\sqrt{317}+317}$$
\rozwStop
\odpStart
$\frac{1}{\sqrt{317}+317}$
\odpStop
\testStart
A.$\frac{1}{\sqrt{317}+317}$
B.$\frac{317}{\sqrt{317}+317}$
C.$0$
D.$\sqrt{317}+317$
E.$\infty$
F.$-\infty$
G.$\sqrt{317}-317$
H.$-317$
I.$317$
\testStop
\kluczStart
A
\kluczStop



\zadStart{Zadanie z Wikieł Z 4.3 a) moja wersja nr 67}
Obliczyć granicę funkcji $f(x)=\frac{\sqrt{x+331}-331}{x}$.
\zadStop
\rozwStart{Patryk Wirkus}{Szymon Tokarski}
$$\frac{\sqrt{x+331}-331}{x}=\frac{(\sqrt{x+331}-331)(\sqrt{x+331}+331)}{x(\sqrt{x+331}+331)}=\frac{1}{\sqrt{x+331}+331}$$
\\
$$\lim\limits_{x\to0}\frac{\sqrt{x+331}-331}{x}=[\frac{0}{0}]=
\lim\limits_{x\to0}\frac{x}{x(\sqrt{x+331}+331)} = \frac{1}{\sqrt{331}+331}$$
\rozwStop
\odpStart
$\frac{1}{\sqrt{331}+331}$
\odpStop
\testStart
A.$\frac{1}{\sqrt{331}+331}$
B.$\frac{331}{\sqrt{331}+331}$
C.$0$
D.$\sqrt{331}+331$
E.$\infty$
F.$-\infty$
G.$\sqrt{331}-331$
H.$-331$
I.$331$
\testStop
\kluczStart
A
\kluczStop



\zadStart{Zadanie z Wikieł Z 4.3 a) moja wersja nr 68}
Obliczyć granicę funkcji $f(x)=\frac{\sqrt{x+337}-337}{x}$.
\zadStop
\rozwStart{Patryk Wirkus}{Szymon Tokarski}
$$\frac{\sqrt{x+337}-337}{x}=\frac{(\sqrt{x+337}-337)(\sqrt{x+337}+337)}{x(\sqrt{x+337}+337)}=\frac{1}{\sqrt{x+337}+337}$$
\\
$$\lim\limits_{x\to0}\frac{\sqrt{x+337}-337}{x}=[\frac{0}{0}]=
\lim\limits_{x\to0}\frac{x}{x(\sqrt{x+337}+337)} = \frac{1}{\sqrt{337}+337}$$
\rozwStop
\odpStart
$\frac{1}{\sqrt{337}+337}$
\odpStop
\testStart
A.$\frac{1}{\sqrt{337}+337}$
B.$\frac{337}{\sqrt{337}+337}$
C.$0$
D.$\sqrt{337}+337$
E.$\infty$
F.$-\infty$
G.$\sqrt{337}-337$
H.$-337$
I.$337$
\testStop
\kluczStart
A
\kluczStop



\zadStart{Zadanie z Wikieł Z 4.3 a) moja wersja nr 69}
Obliczyć granicę funkcji $f(x)=\frac{\sqrt{x+347}-347}{x}$.
\zadStop
\rozwStart{Patryk Wirkus}{Szymon Tokarski}
$$\frac{\sqrt{x+347}-347}{x}=\frac{(\sqrt{x+347}-347)(\sqrt{x+347}+347)}{x(\sqrt{x+347}+347)}=\frac{1}{\sqrt{x+347}+347}$$
\\
$$\lim\limits_{x\to0}\frac{\sqrt{x+347}-347}{x}=[\frac{0}{0}]=
\lim\limits_{x\to0}\frac{x}{x(\sqrt{x+347}+347)} = \frac{1}{\sqrt{347}+347}$$
\rozwStop
\odpStart
$\frac{1}{\sqrt{347}+347}$
\odpStop
\testStart
A.$\frac{1}{\sqrt{347}+347}$
B.$\frac{347}{\sqrt{347}+347}$
C.$0$
D.$\sqrt{347}+347$
E.$\infty$
F.$-\infty$
G.$\sqrt{347}-347$
H.$-347$
I.$347$
\testStop
\kluczStart
A
\kluczStop



\zadStart{Zadanie z Wikieł Z 4.3 a) moja wersja nr 70}
Obliczyć granicę funkcji $f(x)=\frac{\sqrt{x+349}-349}{x}$.
\zadStop
\rozwStart{Patryk Wirkus}{Szymon Tokarski}
$$\frac{\sqrt{x+349}-349}{x}=\frac{(\sqrt{x+349}-349)(\sqrt{x+349}+349)}{x(\sqrt{x+349}+349)}=\frac{1}{\sqrt{x+349}+349}$$
\\
$$\lim\limits_{x\to0}\frac{\sqrt{x+349}-349}{x}=[\frac{0}{0}]=
\lim\limits_{x\to0}\frac{x}{x(\sqrt{x+349}+349)} = \frac{1}{\sqrt{349}+349}$$
\rozwStop
\odpStart
$\frac{1}{\sqrt{349}+349}$
\odpStop
\testStart
A.$\frac{1}{\sqrt{349}+349}$
B.$\frac{349}{\sqrt{349}+349}$
C.$0$
D.$\sqrt{349}+349$
E.$\infty$
F.$-\infty$
G.$\sqrt{349}-349$
H.$-349$
I.$349$
\testStop
\kluczStart
A
\kluczStop



\zadStart{Zadanie z Wikieł Z 4.3 a) moja wersja nr 71}
Obliczyć granicę funkcji $f(x)=\frac{\sqrt{x+353}-353}{x}$.
\zadStop
\rozwStart{Patryk Wirkus}{Szymon Tokarski}
$$\frac{\sqrt{x+353}-353}{x}=\frac{(\sqrt{x+353}-353)(\sqrt{x+353}+353)}{x(\sqrt{x+353}+353)}=\frac{1}{\sqrt{x+353}+353}$$
\\
$$\lim\limits_{x\to0}\frac{\sqrt{x+353}-353}{x}=[\frac{0}{0}]=
\lim\limits_{x\to0}\frac{x}{x(\sqrt{x+353}+353)} = \frac{1}{\sqrt{353}+353}$$
\rozwStop
\odpStart
$\frac{1}{\sqrt{353}+353}$
\odpStop
\testStart
A.$\frac{1}{\sqrt{353}+353}$
B.$\frac{353}{\sqrt{353}+353}$
C.$0$
D.$\sqrt{353}+353$
E.$\infty$
F.$-\infty$
G.$\sqrt{353}-353$
H.$-353$
I.$353$
\testStop
\kluczStart
A
\kluczStop





\end{document}
