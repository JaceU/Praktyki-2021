\documentclass[12pt, a4paper]{article}
\usepackage[utf8]{inputenc}
\usepackage{polski}

\usepackage{amsthm}  %pakiet do tworzenia twierdzeń itp.
\usepackage{amsmath} %pakiet do niektórych symboli matematycznych
\usepackage{amssymb} %pakiet do symboli mat., np. \nsubseteq
\usepackage{amsfonts}
\usepackage{graphicx} %obsługa plików graficznych z rozszerzeniem png, jpg
\theoremstyle{definition} %styl dla definicji
\newtheorem{zad}{} 
\title{Multizestaw zadań}
\author{Robert Fidytek}
%\date{\today}
\date{}
\newcounter{liczniksekcji}
\newcommand{\kategoria}[1]{\section{#1}} %olreślamy nazwę kateforii zadań
\newcommand{\zadStart}[1]{\begin{zad}#1\newline} %oznaczenie początku zadania
\newcommand{\zadStop}{\end{zad}}   %oznaczenie końca zadania
%Makra opcjonarne (nie muszą występować):
\newcommand{\rozwStart}[2]{\noindent \textbf{Rozwiązanie (autor #1 , recenzent #2): }\newline} %oznaczenie początku rozwiązania, opcjonarnie można wprowadzić informację o autorze rozwiązania zadania i recenzencie poprawności wykonania rozwiązania zadania
\newcommand{\rozwStop}{\newline}                                            %oznaczenie końca rozwiązania
\newcommand{\odpStart}{\noindent \textbf{Odpowiedź:}\newline}    %oznaczenie początku odpowiedzi końcowej (wypisanie wyniku)
\newcommand{\odpStop}{\newline}                                             %oznaczenie końca odpowiedzi końcowej (wypisanie wyniku)
\newcommand{\testStart}{\noindent \textbf{Test:}\newline} %ewentualne możliwe opcje odpowiedzi testowej: A. ? B. ? C. ? D. ? itd.
\newcommand{\testStop}{\newline} %koniec wprowadzania odpowiedzi testowych
\newcommand{\kluczStart}{\noindent \textbf{Test poprawna odpowiedź:}\newline} %klucz, poprawna odpowiedź pytania testowego (jedna literka): A lub B lub C lub D itd.
\newcommand{\kluczStop}{\newline} %koniec poprawnej odpowiedzi pytania testowego 
\newcommand{\wstawGrafike}[2]{\begin{figure}[h] \includegraphics[scale=#2] {#1} \end{figure}} %gdyby była potrzeba wstawienia obrazka, parametry: nazwa pliku, skala (jak nie wiesz co wpisać, to wpisz 1)

\begin{document}
\maketitle


\kategoria{Wikieł/Z1.68e}
\zadStart{Zadanie z Wikieł Z 1.68 e) moja wersja nr [nrWersji]}
%[a]:[2,3,4,5,6]
%[b]:[2,3,4,5,6]
%[a]=random.randint(2,13)
%[b]=random.randint(2,13)
%[ba]=[b]*[a]
%[2b]=2*[b]
%[1ba]=1+[ba]
%[w]=([1ba])/([2b])
%[pomoc]=1/[b]
%[w]<[a] and [w]>[pomoc] and math.gcd([1ba],[2b])==1
Rozwiązać równania $\frac{1}{|[a]-x|}=\frac{[b]}{|[b]x-1|}$
\zadStop
\rozwStart{Jakub Ulrych}{Pascal Nawrocki}
założenie:
$$[a]-x\neq0 \land [b]x-1\neq0$$
$$x\neq[a]\land x\neq\frac{1}{[b]}$$
dziedzina:$$x\in\mathbb{R}-\{[a],\frac{1}{[b]}\}$$
rozwiązanie:
$$\frac{1}{|[a]-x|}=\frac{[b]}{|[b]x-1|}$$
$$[b]|[a]-x|=|[b]x-1|$$
$$\text{Liczymy miejsca zerowe wartości bezwzględnych: }[a]-x=0\vee[b]x-1=0$$
$$x=[a]\vee x=\frac{1}{[b]}$$
$$\textbf{1)}x\in(-\infty,\frac{1}{[b]})$$
$$[b]([a]-x)=-[b]x+1$$
$$[ba]\neq1$$
$$\textbf{2)}x\in[\frac{1}{[b]},[a])$$
$$[b]([a]-x)=[b]x-1$$
$$x=\frac{1+[b]\cdot[a]}{2\cdot[b]}=\frac{1+[ba]}{[2b]}=\frac{[1ba]}{[2b]}\in\text{dziedziny}$$
$$\textbf{3)}x\in[[a],\infty)$$
$$[b](-[a]+x)=[b]x-1$$
$$[ba]\neq1$$
\rozwStop
\odpStart
$$x=\frac{[1ba]}{[2b]}$$
\odpStop
\testStart
A.$x=\frac{[1ba]}{[2b]}$
B.$x=-\frac{[1ba]}{[2b]}$
C.$x\in\{-\infty,[a]\}$
D.$x\in\{[a],\infty\}$
\testStop
\kluczStart
A
\kluczStop
\end{document}