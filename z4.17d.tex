\documentclass[12pt, a4paper]{article}
\usepackage[utf8]{inputenc}
\usepackage{polski}

\usepackage{amsthm}  %pakiet do tworzenia twierdzeń itp.
\usepackage{amsmath} %pakiet do niektórych symboli matematycznych
\usepackage{amssymb} %pakiet do symboli mat., np. \nsubseteq
\usepackage{amsfonts}
\usepackage{graphicx} %obsługa plików graficznych z rozszerzeniem png, jpg
\theoremstyle{definition} %styl dla definicji
\newtheorem{zad}{} 
\title{Multizestaw zadań}
\author{Robert Fidytek}
%\date{\today}
\date{}
\newcounter{liczniksekcji}
\newcommand{\kategoria}[1]{\section{#1}} %olreślamy nazwę kateforii zadań
\newcommand{\zadStart}[1]{\begin{zad}#1\newline} %oznaczenie początku zadania
\newcommand{\zadStop}{\end{zad}}   %oznaczenie końca zadania
%Makra opcjonarne (nie muszą występować):
\newcommand{\rozwStart}[2]{\noindent \textbf{Rozwiązanie (autor #1 , recenzent #2): }\newline} %oznaczenie początku rozwiązania, opcjonarnie można wprowadzić informację o autorze rozwiązania zadania i recenzencie poprawności wykonania rozwiązania zadania
\newcommand{\rozwStop}{\newline}                                            %oznaczenie końca rozwiązania
\newcommand{\odpStart}{\noindent \textbf{Odpowiedź:}\newline}    %oznaczenie początku odpowiedzi końcowej (wypisanie wyniku)
\newcommand{\odpStop}{\newline}                                             %oznaczenie końca odpowiedzi końcowej (wypisanie wyniku)
\newcommand{\testStart}{\noindent \textbf{Test:}\newline} %ewentualne możliwe opcje odpowiedzi testowej: A. ? B. ? C. ? D. ? itd.
\newcommand{\testStop}{\newline} %koniec wprowadzania odpowiedzi testowych
\newcommand{\kluczStart}{\noindent \textbf{Test poprawna odpowiedź:}\newline} %klucz, poprawna odpowiedź pytania testowego (jedna literka): A lub B lub C lub D itd.
\newcommand{\kluczStop}{\newline} %koniec poprawnej odpowiedzi pytania testowego 
\newcommand{\wstawGrafike}[2]{\begin{figure}[h] \includegraphics[scale=#2] {#1} \end{figure}} %gdyby była potrzeba wstawienia obrazka, parametry: nazwa pliku, skala (jak nie wiesz co wpisać, to wpisz 1)

\begin{document}
\maketitle


\kategoria{Wikieł/Z4.17d}
\zadStart{Zadanie z Wikieł Z 4.17 d) moja wersja nr [nrWersji]}
%[a]:[2,3,4,5,6,7,8,9,10,11,12,13,14,15,16,17,18,19,20,21,22,23,24,25,26,27,28,29,30]
%[x]=[a]+1
%[y]=[a]-1
Zbadać ciągłość funkcji $$f(x)=\frac{arcsin(x+[a])}{x^2+[a]x}.$$
\zadStop
\rozwStart{Aleksandra Pasińska}{}
$$D:x^2+[a]x\neq 0$$
$$x(x+[a])\neq0$$
$$x\neq0, x\neq-[a]$$
$$arcsin(x+[a]),x\in\langle-[x],-[y]\rangle$$
$$D:x\in\langle-[x],-[a])\cup(-[a],-[y]\rangle$$
$$f(x)= \left\{ \begin{array}{ll}
\frac{arcsin(x+[a])}{x^2+[a]x} & \textrm{dla $x\in \langle-[x],-[a])$}\\
\frac{arcsin(x+[a])}{x^2+[a]x} & \textrm{dla $x\in (-[a],-[y]\rangle$} 
\end{array} \right.$$
\rozwStop
\odpStart
Funkcja jest ciągła w swojej dziedzinie naturalnej D$:x\in\langle-[x],-[a])\cup(-[a],-[y]\rangle$.\\
\odpStop
\testStart
A.D$:x\in\langle-[x],-[a])\cup(-[a],-[y]\rangle$
B.$L=\infty, P=\infty$
C.$L=-\infty, P=-\infty$
D.$L=\infty, P=-\infty$
E.$L=-1, P=\infty$
F.$L=-\infty, P=0$
G.$L=-\infty, P=1$
H.$L=5, P=\infty$
I.$L=-9, P=\infty$
\testStop
\kluczStart
A
\kluczStop



\end{document}