\documentclass[12pt, a4paper]{article}
\usepackage[utf8]{inputenc}
\usepackage{polski}
\usepackage{amsthm}  %pakiet do tworzenia twierdzeń itp.
\usepackage{amsmath} %pakiet do niektórych symboli matematycznych
\usepackage{amssymb} %pakiet do symboli mat., np. \nsubseteq
\usepackage{amsfonts}
\usepackage{graphicx} %obsługa plików graficznych z rozszerzeniem png, jpg
\theoremstyle{definition} %styl dla definicji
\newtheorem{zad}{} 
\title{Multizestaw zadań}
\author{Radosław Grzyb}
%\date{\today}
\date{}
\newcounter{liczniksekcji}
\newcommand{\kategoria}[1]{\section{#1}} %olreślamy nazwę kateforii zadań
\newcommand{\zadStart}[1]{\begin{zad}#1\newline} %oznaczenie początku zadania
\newcommand{\zadStop}{\end{zad}}   %oznaczenie końca zadania
%Makra opcjonarne (nie muszą występować):
\newcommand{\rozwStart}[2]{\noindent \textbf{Rozwiązanie (autor #1 , recenzent #2): }\newline} %oznaczenie początku rozwiązania, opcjonarnie można wprowadzić informację o autorze rozwiązania zadania i recenzencie poprawności wykonania rozwiązania zadania
\newcommand{\rozwStop}{\newline}                                            %oznaczenie końca rozwiązania
\newcommand{\odpStart}{\noindent \textbf{Odpowiedź:}\newline}    %oznaczenie początku odpowiedzi końcowej (wypisanie wyniku)
\newcommand{\odpStop}{\newline}                                             %oznaczenie końca odpowiedzi końcowej (wypisanie wyniku)
\newcommand{\testStart}{\noindent \textbf{Test:}\newline} %ewentualne możliwe opcje odpowiedzi testowej: A. ? B. ? C. ? D. ? itd.
\newcommand{\testStop}{\newline} %koniec wprowadzania odpowiedzi testowych
\newcommand{\kluczStart}{\noindent \textbf{Test poprawna odpowiedź:}\newline} %klucz, poprawna odpowiedź pytania testowego (jedna literka): A lub B lub C lub D itd.
\newcommand{\kluczStop}{\newline} %koniec poprawnej odpowiedzi pytania testowego 
\newcommand{\wstawGrafike}[2]{\begin{figure}[h] \includegraphics[scale=#2] {#1} \end{figure}} %gdyby była potrzeba wstawienia obrazka, parametry: nazwa pliku, skala (jak nie wiesz co wpisać, to wpisz 1)
\begin{document}
\maketitle
\kategoria{Wikieł/Z1.82a}
\zadStart{Zadanie z Wikieł Z 1.82a moja wersja nr [nrWersji]}
%[p1]:[2,3,4,5,6,7,8]
%[p2]:[2,3,4,5]
%[p3]:[2,4,5,8,10]
%[C]=1/[p3]
%[vycl]=[p1]**2-[p2]
%[w1]=[p3]**(-[p2])
%[w2]=[p3]**[vycl]
%[v222]=1000*[w1]
%[w2]<1000 and ([v222]).is_integer() is True
Znaleźć współrzędne puntków, w których przecinają się wykresy funkcji.
$$y=[p3]^{x^{2}-[p2]}$$ oraz $$y=[C]^{[p1]x+[p2]}$$
\zadStop
\rozwStart{Radosław Grzyb}{}
Oznaczamy:
$$[p3]^{x^{2}-[p2]}=[C]^{[p1]x+[p2]}$$
$$[p3]^{x^{2}-[p2]}=(\frac{1}{[p3]})^{[p1]x+[p2]}$$
$$[p3]^{x^{2}-[p2]}=[p3]^{-[p1]x-[p2]}$$
Logarytmując obie strony równania otrzymujemy:\\
$$x^{2}-[p2]=-[p1]x-[p2]$$
$$x^{2}+[p1]x=0$$
$$x(x+[p1])=0\implies x=0 \lor x=-[p1]$$
Znaleźlismy współrzędne x naszego równania kwadratowego. Wykorzystamy je do znalezienia współrzędnych y i tym samym otrzymamy punkty przecięcia funkcji.\\\\
Dla $x_{1}=0$ mamy:\\\\
$[p3]^{0^{2}-[p2]}=[w1]=y_{1}$\\\\
Zaś dla $x_{2}=-[p1]$:\\\\
$[p3]^{(-[p1])^{2}-[p2]}=[w2]=y_{2}$\\\\
Mając obie współrzędne naszych punktów, możemy je zapisać:\\\\
$$p_{1}=(0,[w1])$$
$$p_{2}=(-[p1],[w2])$$
\rozwStop
\odpStart
$$p_{1}=(0,[w1])$$
$$p_{2}=(-[p1],[w2])$$
\odpStop
\testStart
A.$$(-[p1],[w2])$$
B.$$p_{1}=(1,[w1])$$
$$p_{2}=(-[p1],[w2])$$
C.$$2$$
D.$$(0,[w1])$$
E.$$p_{1}=(0,[w1])$$
$$p_{2}=(-[p1],[w2])$$ 
\testStop
\kluczStart
E
\kluczStop
\end{document}