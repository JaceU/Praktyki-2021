\documentclass[12pt, a4paper]{article}
\usepackage[utf8]{inputenc}
\usepackage{polski}

\usepackage{amsthm}  %pakiet do tworzenia twierdzeń itp.
\usepackage{amsmath} %pakiet do niektórych symboli matematycznych
\usepackage{amssymb} %pakiet do symboli mat., np. \nsubseteq
\usepackage{amsfonts}
\usepackage{graphicx} %obsługa plików graficznych z rozszerzeniem png, jpg
\theoremstyle{definition} %styl dla definicji
\newtheorem{zad}{} 
\title{Multizestaw zadań}
\author{Robert Fidytek}
%\date{\today}
\date{}
\newcounter{liczniksekcji}
\newcommand{\kategoria}[1]{\section{#1}} %olreślamy nazwę kateforii zadań
\newcommand{\zadStart}[1]{\begin{zad}#1\newline} %oznaczenie początku zadania
\newcommand{\zadStop}{\end{zad}}   %oznaczenie końca zadania
%Makra opcjonarne (nie muszą występować):
\newcommand{\rozwStart}[2]{\noindent \textbf{Rozwiązanie (autor #1 , recenzent #2): }\newline} %oznaczenie początku rozwiązania, opcjonarnie można wprowadzić informację o autorze rozwiązania zadania i recenzencie poprawności wykonania rozwiązania zadania
\newcommand{\rozwStop}{\newline}                                            %oznaczenie końca rozwiązania
\newcommand{\odpStart}{\noindent \textbf{Odpowiedź:}\newline}    %oznaczenie początku odpowiedzi końcowej (wypisanie wyniku)
\newcommand{\odpStop}{\newline}                                             %oznaczenie końca odpowiedzi końcowej (wypisanie wyniku)
\newcommand{\testStart}{\noindent \textbf{Test:}\newline} %ewentualne możliwe opcje odpowiedzi testowej: A. ? B. ? C. ? D. ? itd.
\newcommand{\testStop}{\newline} %koniec wprowadzania odpowiedzi testowych
\newcommand{\kluczStart}{\noindent \textbf{Test poprawna odpowiedź:}\newline} %klucz, poprawna odpowiedź pytania testowego (jedna literka): A lub B lub C lub D itd.
\newcommand{\kluczStop}{\newline} %koniec poprawnej odpowiedzi pytania testowego 
\newcommand{\wstawGrafike}[2]{\begin{figure}[h] \includegraphics[scale=#2] {#1} \end{figure}} %gdyby była potrzeba wstawienia obrazka, parametry: nazwa pliku, skala (jak nie wiesz co wpisać, to wpisz 1)

\begin{document}
\maketitle


\kategoria{Wikieł/Z3.3c}
\zadStart{Zadanie z Wikieł Z 3.3 c moja wersja nr [nrWersji]}
%[p1]:[2,3,4,5,6,7,8]
%[p2]:[4,5,6,7,10]
%[p3]:[2,3,4,5,6,7]
%[p4]:[3,6,7,8,9]
%[p5]:[1,2,4,5,10]
%[a]=random.randint(3,20)
%[b]=random.randint(4,25)
%[p11]=[p1]-1
%[p21]=[p2]-1
%[p31]=[p3]-1
%[p41]=[p4]-1
%[p51]=[p5]-1
%[p123]=[p11]+[p21]-[p31]
%[ap]=round([a]/([p123]+0.0001),2)
%[p45]=[p41]+[p51]
%[bp]=round([b]/(2*[ap]+[p45]+0.0001),2)
%[abp]=round([ap]*[bp],2)
%([p1]+[p2]-[p3])!=([p4]+[p5]) and [a]!=[b] and [p1]!=[p2] and [p1]!=[p3] and [p2]!=[p3] and [p123]!=0 and (2*[ap]+[p45])!=0 and [p1]!=[bp]
Znaleźć wyraz pierwszy $a_{1}$ oraz różnicę $r$ ciągu arytmetycznego ($a_{n}$), w którym \\
b) $a_{[p1]}+a_{[p2]}-a_{[p3]}=[a]$ oraz $a_{[p4]}+ a_{[p5]}=[b]$
\zadStop
\rozwStart{Wojciech Przybylski}{}
$$a_{[p1]}+a_{[p2]}-a_{[p3]}=[a]\Rightarrow a_{1}+[p11]r+a_{1}+[p21]r-a_{1}-[p31]r=[a]$$
$$a_{1}+[p123]r=[a] \Rightarrow a_{1}=[ap]r$$
$$a_{[p4]}+ a_{[p5]}=[b] \Rightarrow 2a_{1}+[p45]r=[b] \Rightarrow 2([ap]r)+[p45]r=[b]$$
$$r=[bp] \Rightarrow a_{1}=[ap]\cdot [bp] \Rightarrow a_{1}=[abp] $$
$$
 \left\{ \begin{array}{ll}
r=[bp] & \\
a_{1}=[abp] &
\end{array} \right.
$$
\rozwStop
\odpStart
$r=[bp]$ oraz $a_{1}=[abp]$
\odpStop
\testStart
A. $r=[bp]$ oraz $a_{1}=[abp]$\\
B. Ciąg jest nieskończony\\
C. $r=[bp]$ oraz $a_{1}=0$\\
D. $r=0$ oraz $a_{1}=[bp]$\\
E. $r=[p1]$ oraz $a_{1}=[p1]$\\
F. $r=[p2]$ oraz $a_{1}=[p2]$\\
G. $r=[p3]$ oraz $a_{1}=[p3]$\\
H. $r=[p4]$ oraz $a_{1}=[p5]$\\
I. $r=[p1]$ oraz $a_{1}=[p2]$
\testStop
\kluczStart
A
\kluczStop



\end{document}