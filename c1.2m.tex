\documentclass[12pt, a4paper]{article}
\usepackage[utf8]{inputenc}
\usepackage{polski}

\usepackage{amsthm}  %pakiet do tworzenia twierdzeń itp.
\usepackage{amsmath} %pakiet do niektórych symboli matematycznych
\usepackage{amssymb} %pakiet do symboli mat., np. \nsubseteq
\usepackage{amsfonts}
\usepackage{graphicx} %obsługa plików graficznych z rozszerzeniem png, jpg
\theoremstyle{definition} %styl dla definicji
\newtheorem{zad}{} 
\title{Multizestaw zadań}
\author{Robert Fidytek}
%\date{\today}
\date{}
\newcounter{liczniksekcji}
\newcommand{\kategoria}[1]{\section{#1}} %olreślamy nazwę kateforii zadań
\newcommand{\zadStart}[1]{\begin{zad}#1\newline} %oznaczenie początku zadania
\newcommand{\zadStop}{\end{zad}}   %oznaczenie końca zadania
%Makra opcjonarne (nie muszą występować):
\newcommand{\rozwStart}[2]{\noindent \textbf{Rozwiązanie (autor #1 , recenzent #2): }\newline} %oznaczenie początku rozwiązania, opcjonarnie można wprowadzić informację o autorze rozwiązania zadania i recenzencie poprawności wykonania rozwiązania zadania
\newcommand{\rozwStop}{\newline}                                            %oznaczenie końca rozwiązania
\newcommand{\odpStart}{\noindent \textbf{Odpowiedź:}\newline}    %oznaczenie początku odpowiedzi końcowej (wypisanie wyniku)
\newcommand{\odpStop}{\newline}                                             %oznaczenie końca odpowiedzi końcowej (wypisanie wyniku)
\newcommand{\testStart}{\noindent \textbf{Test:}\newline} %ewentualne możliwe opcje odpowiedzi testowej: A. ? B. ? C. ? D. ? itd.
\newcommand{\testStop}{\newline} %koniec wprowadzania odpowiedzi testowych
\newcommand{\kluczStart}{\noindent \textbf{Test poprawna odpowiedź:}\newline} %klucz, poprawna odpowiedź pytania testowego (jedna literka): A lub B lub C lub D itd.
\newcommand{\kluczStop}{\newline} %koniec poprawnej odpowiedzi pytania testowego 
\newcommand{\wstawGrafike}[2]{\begin{figure}[h] \includegraphics[scale=#2] {#1} \end{figure}} %gdyby była potrzeba wstawienia obrazka, parametry: nazwa pliku, skala (jak nie wiesz co wpisać, to wpisz 1)

\begin{document}
\maketitle
\kategoria{Wikieł/C1.2m}
\zadStart{Zadanie z Wikieł C 1.2m moja wersja nr [nrWersji]}
%[a]:[2,3,4,5,6,7,8,9,10,11,12]
%[b]:[2,3,4,5,6,7,8,9,10,11,12]
%[a]!=[b]
Oblicz całkę $$\int \frac{[a]cos^{3}(x)-[b]}{cos^{2}(x)} dx.$$
\zadStop
\rozwStart{Justyna Chojecka}{}
$$\int \frac{[a]cos^{3}(x)-[b]}{cos^{2}(x)} dx=\int\left(\frac{[a]cos^{3}(x)}{cos^{2}(x)}-\frac{[b]}{cos^{2}(x)}\right)dx$$$$=\int\left([a]cos(x)-\frac{[b]}{cos^{2}(x)}\right)dx=[a]\int cos(x)dx-[b]\int \frac{1}{cos^{2}(x)}dx$$$$=[a]sin(x)-[b]tg(x)+C$$
\rozwStop
\odpStart
$[a]sin(x)-[b]tg(x)+C$
\odpStop
\testStart
A.$[a]sin(x)-[b]tg(x)+C$\\
B.$[a]cos(x)-[b]tg(x)+C$\\
C.$[b]sin(x)-[a]tg(x)+C$\\
D.$-[a]cos(x)-[b]tg(x)+C$\\
E.$[a]sin(x)+[b]tg(x)+C$\\
F.$-[a]sin(x)-[b]tg(x)+C$\\
G.$[b]sin(x)+[a]tg(x)+C$\\
H.$-[b]sin(x)+[a]tg(x)+C$\\
I.$-[a]sin(x)+[b]tg(x)+C$\\
\testStop
\kluczStart
A
\kluczStop



\end{document}