\documentclass[12pt, a4paper]{article}
\usepackage[utf8]{inputenc}
\usepackage{polski}

\usepackage{amsthm}  %pakiet do tworzenia twierdzeń itp.
\usepackage{amsmath} %pakiet do niektórych symboli matematycznych
\usepackage{amssymb} %pakiet do symboli mat., np. \nsubseteq
\usepackage{amsfonts}
\usepackage{graphicx} %obsługa plików graficznych z rozszerzeniem png, jpg
\theoremstyle{definition} %styl dla definicji
\newtheorem{zad}{} 
\title{Multizestaw zadań}
\author{Robert Fidytek}
%\date{\today}
\date{}
\newcounter{liczniksekcji}
\newcommand{\kategoria}[1]{\section{#1}} %olreślamy nazwę kateforii zadań
\newcommand{\zadStart}[1]{\begin{zad}#1\newline} %oznaczenie początku zadania
\newcommand{\zadStop}{\end{zad}}   %oznaczenie końca zadania
%Makra opcjonarne (nie muszą występować):
\newcommand{\rozwStart}[2]{\noindent \textbf{Rozwiązanie (autor #1 , recenzent #2): }\newline} %oznaczenie początku rozwiązania, opcjonarnie można wprowadzić informację o autorze rozwiązania zadania i recenzencie poprawności wykonania rozwiązania zadania
\newcommand{\rozwStop}{\newline}                                            %oznaczenie końca rozwiązania
\newcommand{\odpStart}{\noindent \textbf{Odpowiedź:}\newline}    %oznaczenie początku odpowiedzi końcowej (wypisanie wyniku)
\newcommand{\odpStop}{\newline}                                             %oznaczenie końca odpowiedzi końcowej (wypisanie wyniku)
\newcommand{\testStart}{\noindent \textbf{Test:}\newline} %ewentualne możliwe opcje odpowiedzi testowej: A. ? B. ? C. ? D. ? itd.
\newcommand{\testStop}{\newline} %koniec wprowadzania odpowiedzi testowych
\newcommand{\kluczStart}{\noindent \textbf{Test poprawna odpowiedź:}\newline} %klucz, poprawna odpowiedź pytania testowego (jedna literka): A lub B lub C lub D itd.
\newcommand{\kluczStop}{\newline} %koniec poprawnej odpowiedzi pytania testowego 
\newcommand{\wstawGrafike}[2]{\begin{figure}[h] \centering \includegraphics[scale=#2] {#1} \end{figure}} %gdyby była potrzeba wstawienia obrazka, parametry: nazwa pliku, skala (jak nie wiesz co wpisać, to wpisz 1)

\begin{document}
\maketitle

\kategoria{Wikieł/Z5.43a}

\zadStart{Zadanie z Wikieł Z 5.43 a) moja wersja nr [nrWersji]}
%[a]:[2,3,4,5,6,7,8,9,10,11]
%[b]:[2,3,4,5,6,7,8,9,10,11]
%[c]:[1,2,3,4,5,6,7,8,9,10,11]
%[d]=[b]*2
%[e]=math.gcd([b],[c])
%[f]=int([b]/[e])
%[g]=int([c]/[e])
%[j]=[a]*[b]
%[h]=[a]*[b]+[j]
%[i]=[a]*[c]
%[c]!=[b]
Obliczyć i przedstawić w najprostszej postaci pochodną funkcji f.
$$f(x) = [a]x \sqrt{[b]x^2 - [c]}$$
\zadStop

\rozwStart{Natalia Danieluk}{}
Dziedzina: $\quad \mathcal{D}_f= (-\infty,-\sqrt{\frac{[g]}{[f]}}\rangle\cup\langle\sqrt{\frac{[g]}{[f]}},\infty)$
$$f'(x) = [a]x'\sqrt{[b]x^2 - [c]} + (\sqrt{[b]x^2 - [c]})'[a]x = [a]\sqrt{[b]x^2 - [c]} + \frac{1}{2}\cdot \frac{1}{\sqrt{[b]x^2 - [c]}}\cdot [d]x \cdot [a]x = $$
$$= \frac{[a]([b]x^2 - [c])}{\sqrt{[b]x^2 - [c]}} + \frac{[j]x^2}{\sqrt{[b]x^2 - [c]}} = \frac{[h]x^2 - [i]}{\sqrt{[b]x^2 - [c]}}$$
\rozwStop

\odpStart
$f'(x) = \frac{[h]x^2 - [i]}{\sqrt{[b]x^2 - [c]}}$
\odpStop

\testStart
A. $f'(x) = [a]x \sqrt{[b]x^2 - [c]}$
B. $f'(x) = \frac{[i]x^2 - [h]}{\sqrt{[b]x^2 - [c]}}$
C. $f'(x) = \frac{[h]x^2 + [i]}{\sqrt{[b]x^2 - [c]}}$
D. $f'(x) = \frac{[h]x^2 - [i]}{\sqrt{[b]x^2 - [c]}}$
\testStop

\kluczStart
D
\kluczStop

\end{document}