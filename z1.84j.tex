\documentclass[12pt, a4paper]{article}
\usepackage[utf8]{inputenc}
\usepackage{polski}
\usepackage{amsthm}  %pakiet do tworzenia twierdzeń itp.
\usepackage{amsmath} %pakiet do niektórych symboli matematycznych
\usepackage{amssymb} %pakiet do symboli mat., np. \nsubseteq
\usepackage{amsfonts}
\usepackage{graphicx} %obsługa plików graficznych z rozszerzeniem png, jpg
\theoremstyle{definition} %styl dla definicji
\newtheorem{zad}{} 
\title{Multizestaw zadań}
\author{Radosław Grzyb}
%\date{\today}
\date{}
\newcounter{liczniksekcji}
\newcommand{\kategoria}[1]{\section{#1}} %olreślamy nazwę kateforii zadań
\newcommand{\zadStart}[1]{\begin{zad}#1\newline} %oznaczenie początku zadania
\newcommand{\zadStop}{\end{zad}}   %oznaczenie końca zadania
%Makra opcjonarne (nie muszą występować):
\newcommand{\rozwStart}[2]{\noindent \textbf{Rozwiązanie (autor #1 , recenzent #2): }\newline} %oznaczenie początku rozwiązania, opcjonarnie można wprowadzić informację o autorze rozwiązania zadania i recenzencie poprawności wykonania rozwiązania zadania
\newcommand{\rozwStop}{\newline}                                            %oznaczenie końca rozwiązania
\newcommand{\odpStart}{\noindent \textbf{Odpowiedź:}\newline}    %oznaczenie początku odpowiedzi końcowej (wypisanie wyniku)
\newcommand{\odpStop}{\newline}                                             %oznaczenie końca odpowiedzi końcowej (wypisanie wyniku)
\newcommand{\testStart}{\noindent \textbf{Test:}\newline} %ewentualne możliwe opcje odpowiedzi testowej: A. ? B. ? C. ? D. ? itd.
\newcommand{\testStop}{\newline} %koniec wprowadzania odpowiedzi testowych
\newcommand{\kluczStart}{\noindent \textbf{Test poprawna odpowiedź:}\newline} %klucz, poprawna odpowiedź pytania testowego (jedna literka): A lub B lub C lub D itd.
\newcommand{\kluczStop}{\newline} %koniec poprawnej odpowiedzi pytania testowego 
\newcommand{\wstawGrafike}[2]{\begin{figure}[h] \includegraphics[scale=#2] {#1} \end{figure}} %gdyby była potrzeba wstawienia obrazka, parametry: nazwa pliku, skala (jak nie wiesz co wpisać, to wpisz 1)
\begin{document}
\maketitle
\kategoria{Wikieł/Z1.84j}
\zadStart{Zadanie z Wikieł Z 1.84j moja wersja nr [nrWersji]}
%[p1]:[2,3,5]
%[p2]:[2,3,4,6,7,8,9,10,11]
%[p3]:[2,3,4,6,7,8,9,10,11]
%[delta]=[p2]**2-4*[p3]*(-7)
%[deltaf]=math.sqrt([delta])
%[deltap]=int([deltaf])
%[c1]=[p1]**4
%[x1]=([p2]-[deltap])/(2*[p3])
%[x2]=([p2]+[deltap])/(2*[p3])
%[x11]=10*[x1]
%[x111]=100*[x1]
%[x22]=10*[x2]
%[x222]=100*[x2]
%[delta]>0 and ([deltaf]).is_integer() is True and (([x11]).is_integer() is True or ([x111]).is_integer() is True) and (([x22]).is_integer() is True or ([x222]).is_integer() is True)
Rozwiązać równanie:
$$[p1]^{[p3]x^{2}-[p2]x-\frac{5}{2}}=[c1]\sqrt{[p1]}$$
\zadStop
\rozwStart{Radosław Grzyb}{}
$$[p1]^{[p3]x^{2}-[p2]x-\frac{5}{2}}=[p1]^{4}\cdot[p1]^{\frac{1}{2}}$$
$$[p1]^{[p3]x^{2}-[p2]x-\frac{5}{2}}=[p1]^{\frac{9}{2}}$$
Logarytmując obie strony równania otrzymujemy:
$$[p3]x^{2}-[p2]x-\frac{5}{2}=\frac{9}{2}$$
$$[p3]x^{2}-[p2]x-7=0$$
Otrzymaliśmy równanie kwadratowe. Czas policzyć deltę i znaleźć miejsca zerowe:
$$\Delta=(-[p2])^{2}-4\cdot[p3]\cdot(-7)=[delta]\implies \sqrt{\Delta}=[deltap]$$\\
$$x_{1}=\frac{[p2]-[deltap]}{2\cdot[p3]}=[x1]$$\\\\
$$x_{2}=\frac{[p2]+[deltap]}{2\cdot[p3]}=[x2]$$
\rozwStop
\odpStart
$[x1] \vee [x2]$
\odpStop
\testStart
A.$0.12 \vee 10$
B.$[x1] \vee [x2]$
C.$-5 \vee 0.94$
D.$11.5 \vee 50.0$
\testStop
\kluczStart
B
\kluczStop
\end{document}
