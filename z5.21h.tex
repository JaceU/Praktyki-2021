\documentclass[12pt, a4paper]{article}
\usepackage[utf8]{inputenc}
\usepackage{polski}

\usepackage{amsthm}  %pakiet do tworzenia twierdzeń itp.
\usepackage{amsmath} %pakiet do niektórych symboli matematycznych
\usepackage{amssymb} %pakiet do symboli mat., np. \nsubseteq
\usepackage{amsfonts}
\usepackage{graphicx} %obsługa plików graficznych z rozszerzeniem png, jpg
\theoremstyle{definition} %styl dla definicji
\newtheorem{zad}{} 
\title{Multizestaw zadań}
\author{Laura Mieczkowska}
%\date{\today}
\date{}
\newcounter{liczniksekcji}
\newcommand{\kategoria}[1]{\section{#1}} %olreślamy nazwę kateforii zadań
\newcommand{\zadStart}[1]{\begin{zad}#1\newline} %oznaczenie początku zadania
\newcommand{\zadStop}{\end{zad}}   %oznaczenie końca zadania
%Makra opcjonarne (nie muszą występować):
\newcommand{\rozwStart}[2]{\noindent \textbf{Rozwiązanie (autor #1 , recenzent #2): }\newline} %oznaczenie początku rozwiązania, opcjonarnie można wprowadzić informację o autorze rozwiązania zadania i recenzencie poprawności wykonania rozwiązania zadania
\newcommand{\rozwStop}{\newline}                                            %oznaczenie końca rozwiązania
\newcommand{\odpStart}{\noindent \textbf{Odpowiedź:}\newline}    %oznaczenie początku odpowiedzi końcowej (wypisanie wyniku)
\newcommand{\odpStop}{\newline}                                             %oznaczenie końca odpowiedzi końcowej (wypisanie wyniku)
\newcommand{\testStart}{\noindent \textbf{Test:}\newline} %ewentualne możliwe opcje odpowiedzi testowej: A. ? B. ? C. ? D. ? itd.
\newcommand{\testStop}{\newline} %koniec wprowadzania odpowiedzi testowych
\newcommand{\kluczStart}{\noindent \textbf{Test poprawna odpowiedź:}\newline} %klucz, poprawna odpowiedź pytania testowego (jedna literka): A lub B lub C lub D itd.
\newcommand{\kluczStop}{\newline} %koniec poprawnej odpowiedzi pytania testowego 
\newcommand{\wstawGrafike}[2]{\begin{figure}[h] \includegraphics[scale=#2] {#1} \end{figure}} %gdyby była potrzeba wstawienia obrazka, parametry: nazwa pliku, skala (jak nie wiesz co wpisać, to wpisz 1)

\begin{document}
\maketitle


\kategoria{Wikieł/Z5.21h}
\zadStart{Zadanie z Wikieł Z 5.21 h) moja wersja nr [nrWersji]}
%[a]:[4,6,8,10,12,14,16,18]
%[b1]=[a]/2
%[b]=int([b1])

Wyznaczyć przedziały monotoniczności funkcji $f(x)=[a]\sqrt{x}lnx$.
\zadStop
\rozwStart{Laura Mieczkowska}{}
$$f(x)=[a]\sqrt{x}lnx$$
Dziedziną funkcji $f$ jest zbiór liczb rzeczywistych dodatnich tj. $\mathnormal{D_f}=(0;\infty).$
\\\\Obliczamy pochodną funkcji
$$f'(x)=[a]\cdot((\sqrt{x})'lnx+\sqrt{x}(lnx)')=[a]\cdot\bigg(\frac{lnx}{2\sqrt{x}}+\frac{1}{\sqrt{x}}\bigg)=
[a]\cdot\bigg(\frac{lnx+2}{2\sqrt{x}}\bigg)=\frac{[b](lnx+2)}{\sqrt{x}}$$
\\
$$f'(x)>0\Leftrightarrow \frac{[b](lnx+2)}{\sqrt{x}}>0 \Leftrightarrow [b](lnx+2)>0 \Leftrightarrow$$
$$\Leftrightarrow (lnx+2)>0 \Leftrightarrow lnx>-2 \Leftrightarrow x>e^{-2}$$
\\Więc funkcja $f(x)=[a]\sqrt{x}lnx$ jest rosnąca na przedziale $(e^{-2};\infty)$ oraz malejąca na przedziale $(0,e^{-2})$.

\odpStart
$\nearrow$ w $(e^{-2};\infty)$,$\searrow$ w $(0,e^{-2})$
\odpStop
\testStart
A. $\nearrow$ w $(e^{-2};\infty)$,$\searrow$ w $(0,e^{-2})$\\
B. $\searrow$ w $(e^{-2};\infty)$,$\nearrow$ w $(0,e^{-2})$ \\
C. $\nearrow$ w $(-\infty;0)$,$\searrow$ w $(0,e^{-2})$ \\
D. $\searrow$ w $(-\infty;0)$,$\nearrow$ w $(0,e^{-2})$ 
\testStop
\kluczStart
A
\kluczStop



\end{document}