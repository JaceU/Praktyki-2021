\documentclass[12pt, a4paper]{article}
\usepackage[utf8]{inputenc}
\usepackage{polski}

\usepackage{amsthm}  %pakiet do tworzenia twierdzeń itp.
\usepackage{amsmath} %pakiet do niektórych symboli matematycznych
\usepackage{amssymb} %pakiet do symboli mat., np. \nsubseteq
\usepackage{amsfonts}
\usepackage{graphicx} %obsługa plików graficznych z rozszerzeniem png, jpg
\theoremstyle{definition} %styl dla definicji
\newtheorem{zad}{} 
\title{Multizestaw zadań}
\author{Robert Fidytek}
%\date{\today}
\date{}
\newcounter{liczniksekcji}
\newcommand{\kategoria}[1]{\section{#1}} %olreślamy nazwę kateforii zadań
\newcommand{\zadStart}[1]{\begin{zad}#1\newline} %oznaczenie początku zadania
\newcommand{\zadStop}{\end{zad}}   %oznaczenie końca zadania
%Makra opcjonarne (nie muszą występować):
\newcommand{\rozwStart}[2]{\noindent \textbf{Rozwiązanie (autor #1 , recenzent #2): }\newline} %oznaczenie początku rozwiązania, opcjonarnie można wprowadzić informację o autorze rozwiązania zadania i recenzencie poprawności wykonania rozwiązania zadania
\newcommand{\rozwStop}{\newline}                                            %oznaczenie końca rozwiązania
\newcommand{\odpStart}{\noindent \textbf{Odpowiedź:}\newline}    %oznaczenie początku odpowiedzi końcowej (wypisanie wyniku)
\newcommand{\odpStop}{\newline}                                             %oznaczenie końca odpowiedzi końcowej (wypisanie wyniku)
\newcommand{\testStart}{\noindent \textbf{Test:}\newline} %ewentualne możliwe opcje odpowiedzi testowej: A. ? B. ? C. ? D. ? itd.
\newcommand{\testStop}{\newline} %koniec wprowadzania odpowiedzi testowych
\newcommand{\kluczStart}{\noindent \textbf{Test poprawna odpowiedź:}\newline} %klucz, poprawna odpowiedź pytania testowego (jedna literka): A lub B lub C lub D itd.
\newcommand{\kluczStop}{\newline} %koniec poprawnej odpowiedzi pytania testowego 
\newcommand{\wstawGrafike}[2]{\begin{figure}[h] \includegraphics[scale=#2] {#1} \end{figure}} %gdyby była potrzeba wstawienia obrazka, parametry: nazwa pliku, skala (jak nie wiesz co wpisać, to wpisz 1)

\begin{document}
\maketitle



\kategoria{Wikieł/Z1.15l}
\zadStart{Zadanie z Wikieł Z 1.15 l) moja wersja nr [nrWersji]}
%[a]:[2,3,4]
%[a6]=[a]*[a]*[a]*[a]*[a]*[a]
%[a5]=[a]*[a]*[a]*[a]*[a]
%[a4]=[a]*[a]*[a]*[a]
%[a3]=[a]*[a]*[a]
%[a2]=[a]*[a]
%[a56]=6*[a5]
%[a415]=15*[a4]
%[a320]=20*[a3]
%[a215]=15*[a2]
%[a16]=[a]*6


Rozwinąć według wzoru Newtona: $(\sqrt{x}+[a]y)^6$
\zadStop
\rozwStart{Pascal Nawrocki}{Jakub Ulrych}
Na początku przypomnijmy wzór Newtona:$$(a+b)^n={n\choose 0}a^{n}b^{0}+{n\choose 1}a^{n-1}b^{1}+{n\choose 2}a^{n-2}b^{2}+\dots+{n\choose n}a^{0}b^{n}=\sum_{k=0}^{n} {n\choose k}a^{n-k}b^{k}$$
Zatem:
$$(\sqrt{x}+[a]y)^6=$$
\begin{equation}
\begin{split}
&={6\choose0}(\sqrt{x})^6([a]y)^0+{6\choose1}(\sqrt{x})^5([a]y)^1+{6\choose2}(\sqrt{x})^4([a]y)^2+{6\choose3}(\sqrt{x})^3([a]y)^3\\&+{6\choose4}(\sqrt{x})^2([a]y)^4+{6\choose5}(\sqrt{x})^1([a]y)^5+{6\choose6}(\sqrt{x})^0([a]y)^6=
\end{split}
\end{equation}
$$=1\cdot x^3\cdot1+6\cdot x^2\sqrt{x}\cdot[a]y+15\cdot x^2\cdot[a2]y^2+20\cdot x\sqrt{x}\cdot[a3]y^3\cdot15\cdot x\cdot[a4]y^4+6\cdot \sqrt{x}\cdot[a5]y^5+1\cdot1\cdot[a6]y^6$$
$$=[a6]y^6+[a56]\sqrt{x}y^5+[a415]xy^4+[a320]x\sqrt{x}y^3+[a215]x^2y^2+[a16]x^2\sqrt{x}y+x^3$$
\odpStop
\testStart
A.$[a6]y^6+[a56]\sqrt{x}y^5+[a415]xy^4+[a320]x\sqrt{x}y^3+[a215]x^2y^2+[a16]x^2\sqrt{x}y+x^3$
\\
B.$[a6]y^6+[a56]\sqrt{x}y^5+[a415]xy^4+[a320]x\sqrt{x}y^3$
\\
C.$[a6]y^6+[a56]\sqrt{x}y^5+[a415]xy^4-[a320]x\sqrt{x}y^3-[a215]x^2y^2+[a16]x^2\sqrt{x}y+x^3$
\\
D.$[a6]y^6+[a320]x\sqrt{x}y^3+[a215]x^2y^2+[a16]x^2\sqrt{x}y+x^3$
\testStop
\kluczStart
A
\kluczStop


\end{document}