\documentclass[12pt, a4paper]{article}
\usepackage[utf8]{inputenc}
\usepackage{polski}

\usepackage{amsthm}  %pakiet do tworzenia twierdzeń itp.
\usepackage{amsmath} %pakiet do niektórych symboli matematycznych
\usepackage{amssymb} %pakiet do symboli mat., np. \nsubseteq
\usepackage{amsfonts}
\usepackage{graphicx} %obsługa plików graficznych z rozszerzeniem png, jpg
\theoremstyle{definition} %styl dla definicji
\newtheorem{zad}{} 
\title{Multizestaw zadań}
\author{Robert Fidytek}
%\date{\today}
\date{}
\newcounter{liczniksekcji}
\newcommand{\kategoria}[1]{\section{#1}} %olreślamy nazwę kateforii zadań
\newcommand{\zadStart}[1]{\begin{zad}#1\newline} %oznaczenie początku zadania
\newcommand{\zadStop}{\end{zad}}   %oznaczenie końca zadania
%Makra opcjonarne (nie muszą występować):
\newcommand{\rozwStart}[2]{\noindent \textbf{Rozwiązanie (autor #1 , recenzent #2): }\newline} %oznaczenie początku rozwiązania, opcjonarnie można wprowadzić informację o autorze rozwiązania zadania i recenzencie poprawności wykonania rozwiązania zadania
\newcommand{\rozwStop}{\newline}                                            %oznaczenie końca rozwiązania
\newcommand{\odpStart}{\noindent \textbf{Odpowiedź:}\newline}    %oznaczenie początku odpowiedzi końcowej (wypisanie wyniku)
\newcommand{\odpStop}{\newline}                                             %oznaczenie końca odpowiedzi końcowej (wypisanie wyniku)
\newcommand{\testStart}{\noindent \textbf{Test:}\newline} %ewentualne możliwe opcje odpowiedzi testowej: A. ? B. ? C. ? D. ? itd.
\newcommand{\testStop}{\newline} %koniec wprowadzania odpowiedzi testowych
\newcommand{\kluczStart}{\noindent \textbf{Test poprawna odpowiedź:}\newline} %klucz, poprawna odpowiedź pytania testowego (jedna literka): A lub B lub C lub D itd.
\newcommand{\kluczStop}{\newline} %koniec poprawnej odpowiedzi pytania testowego 
\newcommand{\wstawGrafike}[2]{\begin{figure}[h] \includegraphics[scale=#2] {#1} \end{figure}} %gdyby była potrzeba wstawienia obrazka, parametry: nazwa pliku, skala (jak nie wiesz co wpisać, to wpisz 1)

\begin{document}
\maketitle


\kategoria{Wikieł/Z1.93i}
\zadStart{Zadanie z Wikieł Z 1.93 i) moja wersja nr [nrWersji]}
%[b]:[1,2,3,4,5,6,7,8,9,10,11,12,13,14,15,16,17,18,19,20,21,22,23,24]
%[a]:[1,2,3,4,5,6,7,8,9,10,11,12,13,14,15,16,17,18,19,20,21,22,23,24]
%[ab]=[a]+[b]
%[c1]=[b]*[a]
%[c]=[c1]+1
%[d]=(([ab]**2)-(4*[c]))
%[d4]=int([d]/4)
%[abb]=[ab]/2
%[ab2]=int([abb])
%[z1]=([ab]-(pow([d],1/2)))/2
%[z2]=([ab]+(pow([d],1/2)))/2
%[d]>0 and [z1]>[b] and [z2]>[b] and [z1]<[a] and [z2]<[a] and math.gcd([d],4)==4 and math.gcd([d],9)<4 and math.gcd([d],16)<5 and math.gcd([d],25)<6 and math.gcd([d],36)<13 and [abb].is_integer()==True
Rozwiązać równanie $\log{([a]-x)} + 2\log{\sqrt{x-[b]}} = 0$
\zadStop
\rozwStart{Małgorzata Ugowska}{}
Szukamy dziedziny:
$$[a]-x>0 \quad \land \quad x-[b] >0 \quad \Longrightarrow \quad D = ([b], [a])$$
Następnie szukamy rozwiązania równania.
$$\log{([a]-x)} + 2\log{\sqrt{x-[b]}} = 0 \quad \Longleftrightarrow \quad \log{([a]-x)} + \log{(x-[b])} = 0 $$
$$ \Longleftrightarrow \quad \log{([a]-x)(x-[b])} = 0 \quad \Longleftrightarrow \quad ([a]-x)(x-[b]) =1 $$
$$ \Longleftrightarrow \quad -x^2 +[ab]x-[c1]=1 \quad \Longleftrightarrow \quad x^2 -[ab]x+[c]=0 $$
$$ \bigtriangleup = [ab]^2-4 \cdot 1 \cdot [c] = [d] \quad  \Longrightarrow \quad \sqrt{\bigtriangleup}=\sqrt{[d]} = \sqrt{4 \cdot [d4]} = 2 \sqrt{[d4]}$$
$$ x_1=\frac{[ab]-2 \sqrt{[d4]}}{2} = [ab2]-\sqrt{[d4]} \in D, \qquad x_2=\frac{[ab]+2 \sqrt{[d4]}}{2} = [ab2]+\sqrt{[d4]} \in D$$
\rozwStop
\odpStart
$x \in \{[ab2]-\sqrt{[d4]}, [ab2]+\sqrt{[d4]}\}$
\odpStop
\testStart
A. $x \in \{[ab2]-\sqrt{[d4]}, [ab2]+\sqrt{[d4]}\}$\\
B. $x \in \{[ab2], [ab]\}$\\
C. $x \in \{0, \frac{1}{2}\}$\\
D. $x \in \{\frac{5}{3}, \frac{-1}{3}\}$\\
E. $x \in \{[ab]-\sqrt{[d4]}, [ab]+\sqrt{[d4]}\}$
\testStop
\kluczStart
A
\kluczStop



\end{document}