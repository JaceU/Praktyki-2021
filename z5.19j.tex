\documentclass[12pt, a4paper]{article}
\usepackage[utf8]{inputenc}
\usepackage{polski}

\usepackage{amsthm}  %pakiet do tworzenia twierdzeń itp.
\usepackage{amsmath} %pakiet do niektórych symboli matematycznych
\usepackage{amssymb} %pakiet do symboli mat., np. \nsubseteq
\usepackage{amsfonts}
\usepackage{graphicx} %obsługa plików graficznych z rozszerzeniem png, jpg
\theoremstyle{definition} %styl dla definicji
\newtheorem{zad}{} 
\title{Multizestaw zadań}
\author{Robert Fidytek}
%\date{\today}
\date{}
\newcounter{liczniksekcji}
\newcommand{\kategoria}[1]{\section{#1}} %olreślamy nazwę kateforii zadań
\newcommand{\zadStart}[1]{\begin{zad}#1\newline} %oznaczenie początku zadania
\newcommand{\zadStop}{\end{zad}}   %oznaczenie końca zadania
%Makra opcjonarne (nie muszą występować):
\newcommand{\rozwStart}[2]{\noindent \textbf{Rozwiązanie (autor #1 , recenzent #2): }\newline} %oznaczenie początku rozwiązania, opcjonarnie można wprowadzić informację o autorze rozwiązania zadania i recenzencie poprawności wykonania rozwiązania zadania
\newcommand{\rozwStop}{\newline}                                            %oznaczenie końca rozwiązania
\newcommand{\odpStart}{\noindent \textbf{Odpowiedź:}\newline}    %oznaczenie początku odpowiedzi końcowej (wypisanie wyniku)
\newcommand{\odpStop}{\newline}                                             %oznaczenie końca odpowiedzi końcowej (wypisanie wyniku)
\newcommand{\testStart}{\noindent \textbf{Test:}\newline} %ewentualne możliwe opcje odpowiedzi testowej: A. ? B. ? C. ? D. ? itd.
\newcommand{\testStop}{\newline} %koniec wprowadzania odpowiedzi testowych
\newcommand{\kluczStart}{\noindent \textbf{Test poprawna odpowiedź:}\newline} %klucz, poprawna odpowiedź pytania testowego (jedna literka): A lub B lub C lub D itd.
\newcommand{\kluczStop}{\newline} %koniec poprawnej odpowiedzi pytania testowego 
\newcommand{\wstawGrafike}[2]{\begin{figure}[h] \includegraphics[scale=#2] {#1} \end{figure}} %gdyby była potrzeba wstawienia obrazka, parametry: nazwa pliku, skala (jak nie wiesz co wpisać, to wpisz 1)

\begin{document}
\maketitle


\kategoria{Wikieł/Z5.19 j}
\zadStart{Zadanie z Wikieł Z 5.19 j) moja wersja nr [nrWersji]}
%[a]:[2,3,4,5,6,7,8,9]
%[b]:[2,3,4,5,6,7,8,9]
%[c]=2*[a]
%[d]=[b]*[c]
%[a]!=0
Oblicz granicę $\lim_{x \rightarrow \frac{1}{2}} [b]\sin([c]x-[a])\tg(\pi x)$.
\zadStop
\rozwStart{Joanna Świerzbin}{}
$$\lim_{x \rightarrow \frac{1}{2}} [b]\sin([c]x-[a])\tg(\pi x)=-[b] \lim_{x \rightarrow \frac{1}{2}} \sin([a]-[c]x)\tg(\pi x)=$$
$$=-[b] \lim_{x \rightarrow \frac{1}{2}} \frac{\sin([a]-[c]x) \sin(\pi x)}{\cos(\pi x)}$$
Otrzymujemy $ \left[ \frac{0}{0} \right] $ więc możemy skorzystać z twierdzenia de l'Hospitala.
$$-[b] \lim_{x \rightarrow \frac{1}{2}} \frac{\sin([a]-[c]x) \sin(\pi x)}{\cos(\pi x)}=$$
$$=-[b] \lim_{x \rightarrow \frac{1}{2}} \frac{\cos([a]-[c]x) (-[c]) \sin(\pi x) -\pi \cos(\pi x) \sin([a]-[c]x)}{-\pi \sin(\pi x)}=$$
$$=-[b] \lim_{x \rightarrow \frac{1}{2}} \frac{[c] \cos([a]-[c]x) \sin(\pi x) }{\pi \sin(\pi x)}-[b] \lim_{x \rightarrow \frac{1}{2}} \frac{\pi \cos(\pi x) \sin([a]-[c]x)}{\pi \sin(\pi x)} = $$
$$=-[b] \lim_{x \rightarrow \frac{1}{2}} \frac{[c] \cos([a]-[c]x)}{\pi}-[b] \lim_{x \rightarrow \frac{1}{2}} \frac{ \cos(\pi x) \sin([a]-[c]x)}{ \sin(\pi x)} = $$
$$=-\frac{[b]\cdot[c]}{\pi} -0 = -\frac{[d]}{\pi}$$
\rozwStop
\odpStart
$\lim_{x \rightarrow \frac{1}{2}} [b]\sin([c]x-[a])\tg(\pi x) =  -\frac{[d]}{\pi} $
\odpStop
\testStart
A. $\lim_{x \rightarrow \frac{1}{2}} [b]\sin([c]x-[a])\tg(\pi x) =  -\frac{[d]}{\pi} $\\
B. $\lim_{x \rightarrow \frac{1}{2}} [b]\sin([c]x-[a])\tg(\pi x) =  -\frac{1}{\pi} $\\
C. $\lim_{x \rightarrow \frac{1}{2}} [b]\sin([c]x-[a])\tg(\pi x) =  \frac{[d]}{\pi} $\\
D. $\lim_{x \rightarrow \frac{1}{2}} [b]\sin([c]x-[a])\tg(\pi x) =  [d] $\\
E. $\lim_{x \rightarrow \frac{1}{2}} [b]\sin([c]x-[a])\tg(\pi x) =  0 $\\
F. $\lim_{x \rightarrow \frac{1}{2}} [b]\sin([c]x-[a])\tg(\pi x) =  1 $
\testStop
\kluczStart
A
\kluczStop



\end{document}