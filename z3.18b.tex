\documentclass[12pt, a4paper]{article}
\usepackage[utf8]{inputenc}
\usepackage{polski}
\usepackage{amsthm}  %pakiet do tworzenia twierdzeń itp.
\usepackage{amsmath} %pakiet do niektórych symboli matematycznych
\usepackage{amssymb} %pakiet do symboli mat., np. \nsubseteq
\usepackage{amsfonts}
\usepackage{graphicx} %obsługa plików graficznych z rozszerzeniem png, jpg
\theoremstyle{definition} %styl dla definicji
\newtheorem{zad}{} 
\title{Multizestaw zadań}
\author{Patryk Wirkus}
%\date{\today}
\date{}
\newcommand{\kategoria}[1]{\section{#1}}
\newcommand{\zadStart}[1]{\begin{zad}#1\newline}
\newcommand{\zadStop}{\end{zad}}
\newcommand{\rozwStart}[2]{\noindent \textbf{Rozwiązanie (autor #1 , recenzent #2): }\newline}
\newcommand{\rozwStop}{\newline}                                           
\newcommand{\odpStart}{\noindent \textbf{Odpowiedź:}\newline}
\newcommand{\odpStop}{\newline}
\newcommand{\testStart}{\noindent \textbf{Test:}\newline}
\newcommand{\testStop}{\newline}
\newcommand{\kluczStart}{\noindent \textbf{Test poprawna odpowiedź:}\newline}
\newcommand{\kluczStop}{\newline}
\newcommand{\wstawGrafike}[2]{\begin{figure}[h] \includegraphics[scale=#2] {#1} \end{figure}}

\begin{document}
\maketitle

\kategoria{Wikieł/Z3.18b}


\zadStart{Zadanie z Wikieł Z 3.18 b) moja wersja nr 1}

Zamień poniższe ułamki dziesiętne okresowe na ułamki zwykłe $2,1(23)$.
\zadStop
\rozwStart{Patryk Wirkus}{Martyna Czarnobaj}
$$2,1(23)=2,1232323=2,1+(0,023+0,00023+...)=2,1+\frac{0,023}{1-0,01}$$
$$=2,1+\frac{23}{990}=\frac{21\cdot99+23}{990}$$
\rozwStop
\odpStart
$\frac{21\cdot99+23}{990}$
\odpStop
\testStart
A.$\frac{21\cdot99+23}{990}$\\ B.$-\frac{21\cdot99+23}{990}$\\ C.$2,1$\\ D.$\frac{23\cdot100}{9900}$
\testStop
\kluczStart
A
\kluczStop



\zadStart{Zadanie z Wikieł Z 3.18 b) moja wersja nr 2}

Zamień poniższe ułamki dziesiętne okresowe na ułamki zwykłe $2,1(24)$.
\zadStop
\rozwStart{Patryk Wirkus}{Martyna Czarnobaj}
$$2,1(24)=2,1242424=2,1+(0,024+0,00024+...)=2,1+\frac{0,024}{1-0,01}$$
$$=2,1+\frac{24}{990}=\frac{21\cdot99+24}{990}$$
\rozwStop
\odpStart
$\frac{21\cdot99+24}{990}$
\odpStop
\testStart
A.$\frac{21\cdot99+24}{990}$\\ B.$-\frac{21\cdot99+24}{990}$\\ C.$2,1$\\ D.$\frac{24\cdot100}{9900}$
\testStop
\kluczStart
A
\kluczStop



\zadStart{Zadanie z Wikieł Z 3.18 b) moja wersja nr 3}

Zamień poniższe ułamki dziesiętne okresowe na ułamki zwykłe $2,1(25)$.
\zadStop
\rozwStart{Patryk Wirkus}{Martyna Czarnobaj}
$$2,1(25)=2,1252525=2,1+(0,025+0,00025+...)=2,1+\frac{0,025}{1-0,01}$$
$$=2,1+\frac{25}{990}=\frac{21\cdot99+25}{990}$$
\rozwStop
\odpStart
$\frac{21\cdot99+25}{990}$
\odpStop
\testStart
A.$\frac{21\cdot99+25}{990}$\\ B.$-\frac{21\cdot99+25}{990}$\\ C.$2,1$\\ D.$\frac{25\cdot100}{9900}$
\testStop
\kluczStart
A
\kluczStop



\zadStart{Zadanie z Wikieł Z 3.18 b) moja wersja nr 4}

Zamień poniższe ułamki dziesiętne okresowe na ułamki zwykłe $2,1(26)$.
\zadStop
\rozwStart{Patryk Wirkus}{Martyna Czarnobaj}
$$2,1(26)=2,1262626=2,1+(0,026+0,00026+...)=2,1+\frac{0,026}{1-0,01}$$
$$=2,1+\frac{26}{990}=\frac{21\cdot99+26}{990}$$
\rozwStop
\odpStart
$\frac{21\cdot99+26}{990}$
\odpStop
\testStart
A.$\frac{21\cdot99+26}{990}$\\ B.$-\frac{21\cdot99+26}{990}$\\ C.$2,1$\\ D.$\frac{26\cdot100}{9900}$
\testStop
\kluczStart
A
\kluczStop



\zadStart{Zadanie z Wikieł Z 3.18 b) moja wersja nr 5}

Zamień poniższe ułamki dziesiętne okresowe na ułamki zwykłe $2,1(27)$.
\zadStop
\rozwStart{Patryk Wirkus}{Martyna Czarnobaj}
$$2,1(27)=2,1272727=2,1+(0,027+0,00027+...)=2,1+\frac{0,027}{1-0,01}$$
$$=2,1+\frac{27}{990}=\frac{21\cdot99+27}{990}$$
\rozwStop
\odpStart
$\frac{21\cdot99+27}{990}$
\odpStop
\testStart
A.$\frac{21\cdot99+27}{990}$\\ B.$-\frac{21\cdot99+27}{990}$\\ C.$2,1$\\ D.$\frac{27\cdot100}{9900}$
\testStop
\kluczStart
A
\kluczStop



\zadStart{Zadanie z Wikieł Z 3.18 b) moja wersja nr 6}

Zamień poniższe ułamki dziesiętne okresowe na ułamki zwykłe $2,1(28)$.
\zadStop
\rozwStart{Patryk Wirkus}{Martyna Czarnobaj}
$$2,1(28)=2,1282828=2,1+(0,028+0,00028+...)=2,1+\frac{0,028}{1-0,01}$$
$$=2,1+\frac{28}{990}=\frac{21\cdot99+28}{990}$$
\rozwStop
\odpStart
$\frac{21\cdot99+28}{990}$
\odpStop
\testStart
A.$\frac{21\cdot99+28}{990}$\\ B.$-\frac{21\cdot99+28}{990}$\\ C.$2,1$\\ D.$\frac{28\cdot100}{9900}$
\testStop
\kluczStart
A
\kluczStop



\zadStart{Zadanie z Wikieł Z 3.18 b) moja wersja nr 7}

Zamień poniższe ułamki dziesiętne okresowe na ułamki zwykłe $2,1(29)$.
\zadStop
\rozwStart{Patryk Wirkus}{Martyna Czarnobaj}
$$2,1(29)=2,1292929=2,1+(0,029+0,00029+...)=2,1+\frac{0,029}{1-0,01}$$
$$=2,1+\frac{29}{990}=\frac{21\cdot99+29}{990}$$
\rozwStop
\odpStart
$\frac{21\cdot99+29}{990}$
\odpStop
\testStart
A.$\frac{21\cdot99+29}{990}$\\ B.$-\frac{21\cdot99+29}{990}$\\ C.$2,1$\\ D.$\frac{29\cdot100}{9900}$
\testStop
\kluczStart
A
\kluczStop



\zadStart{Zadanie z Wikieł Z 3.18 b) moja wersja nr 8}

Zamień poniższe ułamki dziesiętne okresowe na ułamki zwykłe $2,1(34)$.
\zadStop
\rozwStart{Patryk Wirkus}{Martyna Czarnobaj}
$$2,1(34)=2,1343434=2,1+(0,034+0,00034+...)=2,1+\frac{0,034}{1-0,01}$$
$$=2,1+\frac{34}{990}=\frac{21\cdot99+34}{990}$$
\rozwStop
\odpStart
$\frac{21\cdot99+34}{990}$
\odpStop
\testStart
A.$\frac{21\cdot99+34}{990}$\\ B.$-\frac{21\cdot99+34}{990}$\\ C.$2,1$\\ D.$\frac{34\cdot100}{9900}$
\testStop
\kluczStart
A
\kluczStop



\zadStart{Zadanie z Wikieł Z 3.18 b) moja wersja nr 9}

Zamień poniższe ułamki dziesiętne okresowe na ułamki zwykłe $2,1(35)$.
\zadStop
\rozwStart{Patryk Wirkus}{Martyna Czarnobaj}
$$2,1(35)=2,1353535=2,1+(0,035+0,00035+...)=2,1+\frac{0,035}{1-0,01}$$
$$=2,1+\frac{35}{990}=\frac{21\cdot99+35}{990}$$
\rozwStop
\odpStart
$\frac{21\cdot99+35}{990}$
\odpStop
\testStart
A.$\frac{21\cdot99+35}{990}$\\ B.$-\frac{21\cdot99+35}{990}$\\ C.$2,1$\\ D.$\frac{35\cdot100}{9900}$
\testStop
\kluczStart
A
\kluczStop



\zadStart{Zadanie z Wikieł Z 3.18 b) moja wersja nr 10}

Zamień poniższe ułamki dziesiętne okresowe na ułamki zwykłe $2,1(36)$.
\zadStop
\rozwStart{Patryk Wirkus}{Martyna Czarnobaj}
$$2,1(36)=2,1363636=2,1+(0,036+0,00036+...)=2,1+\frac{0,036}{1-0,01}$$
$$=2,1+\frac{36}{990}=\frac{21\cdot99+36}{990}$$
\rozwStop
\odpStart
$\frac{21\cdot99+36}{990}$
\odpStop
\testStart
A.$\frac{21\cdot99+36}{990}$\\ B.$-\frac{21\cdot99+36}{990}$\\ C.$2,1$\\ D.$\frac{36\cdot100}{9900}$
\testStop
\kluczStart
A
\kluczStop



\zadStart{Zadanie z Wikieł Z 3.18 b) moja wersja nr 11}

Zamień poniższe ułamki dziesiętne okresowe na ułamki zwykłe $2,1(37)$.
\zadStop
\rozwStart{Patryk Wirkus}{Martyna Czarnobaj}
$$2,1(37)=2,1373737=2,1+(0,037+0,00037+...)=2,1+\frac{0,037}{1-0,01}$$
$$=2,1+\frac{37}{990}=\frac{21\cdot99+37}{990}$$
\rozwStop
\odpStart
$\frac{21\cdot99+37}{990}$
\odpStop
\testStart
A.$\frac{21\cdot99+37}{990}$\\ B.$-\frac{21\cdot99+37}{990}$\\ C.$2,1$\\ D.$\frac{37\cdot100}{9900}$
\testStop
\kluczStart
A
\kluczStop



\zadStart{Zadanie z Wikieł Z 3.18 b) moja wersja nr 12}

Zamień poniższe ułamki dziesiętne okresowe na ułamki zwykłe $2,1(38)$.
\zadStop
\rozwStart{Patryk Wirkus}{Martyna Czarnobaj}
$$2,1(38)=2,1383838=2,1+(0,038+0,00038+...)=2,1+\frac{0,038}{1-0,01}$$
$$=2,1+\frac{38}{990}=\frac{21\cdot99+38}{990}$$
\rozwStop
\odpStart
$\frac{21\cdot99+38}{990}$
\odpStop
\testStart
A.$\frac{21\cdot99+38}{990}$\\ B.$-\frac{21\cdot99+38}{990}$\\ C.$2,1$\\ D.$\frac{38\cdot100}{9900}$
\testStop
\kluczStart
A
\kluczStop



\zadStart{Zadanie z Wikieł Z 3.18 b) moja wersja nr 13}

Zamień poniższe ułamki dziesiętne okresowe na ułamki zwykłe $2,1(39)$.
\zadStop
\rozwStart{Patryk Wirkus}{Martyna Czarnobaj}
$$2,1(39)=2,1393939=2,1+(0,039+0,00039+...)=2,1+\frac{0,039}{1-0,01}$$
$$=2,1+\frac{39}{990}=\frac{21\cdot99+39}{990}$$
\rozwStop
\odpStart
$\frac{21\cdot99+39}{990}$
\odpStop
\testStart
A.$\frac{21\cdot99+39}{990}$\\ B.$-\frac{21\cdot99+39}{990}$\\ C.$2,1$\\ D.$\frac{39\cdot100}{9900}$
\testStop
\kluczStart
A
\kluczStop



\zadStart{Zadanie z Wikieł Z 3.18 b) moja wersja nr 14}

Zamień poniższe ułamki dziesiętne okresowe na ułamki zwykłe $2,1(45)$.
\zadStop
\rozwStart{Patryk Wirkus}{Martyna Czarnobaj}
$$2,1(45)=2,1454545=2,1+(0,045+0,00045+...)=2,1+\frac{0,045}{1-0,01}$$
$$=2,1+\frac{45}{990}=\frac{21\cdot99+45}{990}$$
\rozwStop
\odpStart
$\frac{21\cdot99+45}{990}$
\odpStop
\testStart
A.$\frac{21\cdot99+45}{990}$\\ B.$-\frac{21\cdot99+45}{990}$\\ C.$2,1$\\ D.$\frac{45\cdot100}{9900}$
\testStop
\kluczStart
A
\kluczStop



\zadStart{Zadanie z Wikieł Z 3.18 b) moja wersja nr 15}

Zamień poniższe ułamki dziesiętne okresowe na ułamki zwykłe $2,1(46)$.
\zadStop
\rozwStart{Patryk Wirkus}{Martyna Czarnobaj}
$$2,1(46)=2,1464646=2,1+(0,046+0,00046+...)=2,1+\frac{0,046}{1-0,01}$$
$$=2,1+\frac{46}{990}=\frac{21\cdot99+46}{990}$$
\rozwStop
\odpStart
$\frac{21\cdot99+46}{990}$
\odpStop
\testStart
A.$\frac{21\cdot99+46}{990}$\\ B.$-\frac{21\cdot99+46}{990}$\\ C.$2,1$\\ D.$\frac{46\cdot100}{9900}$
\testStop
\kluczStart
A
\kluczStop



\zadStart{Zadanie z Wikieł Z 3.18 b) moja wersja nr 16}

Zamień poniższe ułamki dziesiętne okresowe na ułamki zwykłe $2,1(47)$.
\zadStop
\rozwStart{Patryk Wirkus}{Martyna Czarnobaj}
$$2,1(47)=2,1474747=2,1+(0,047+0,00047+...)=2,1+\frac{0,047}{1-0,01}$$
$$=2,1+\frac{47}{990}=\frac{21\cdot99+47}{990}$$
\rozwStop
\odpStart
$\frac{21\cdot99+47}{990}$
\odpStop
\testStart
A.$\frac{21\cdot99+47}{990}$\\ B.$-\frac{21\cdot99+47}{990}$\\ C.$2,1$\\ D.$\frac{47\cdot100}{9900}$
\testStop
\kluczStart
A
\kluczStop



\zadStart{Zadanie z Wikieł Z 3.18 b) moja wersja nr 17}

Zamień poniższe ułamki dziesiętne okresowe na ułamki zwykłe $2,1(48)$.
\zadStop
\rozwStart{Patryk Wirkus}{Martyna Czarnobaj}
$$2,1(48)=2,1484848=2,1+(0,048+0,00048+...)=2,1+\frac{0,048}{1-0,01}$$
$$=2,1+\frac{48}{990}=\frac{21\cdot99+48}{990}$$
\rozwStop
\odpStart
$\frac{21\cdot99+48}{990}$
\odpStop
\testStart
A.$\frac{21\cdot99+48}{990}$\\ B.$-\frac{21\cdot99+48}{990}$\\ C.$2,1$\\ D.$\frac{48\cdot100}{9900}$
\testStop
\kluczStart
A
\kluczStop



\zadStart{Zadanie z Wikieł Z 3.18 b) moja wersja nr 18}

Zamień poniższe ułamki dziesiętne okresowe na ułamki zwykłe $2,1(49)$.
\zadStop
\rozwStart{Patryk Wirkus}{Martyna Czarnobaj}
$$2,1(49)=2,1494949=2,1+(0,049+0,00049+...)=2,1+\frac{0,049}{1-0,01}$$
$$=2,1+\frac{49}{990}=\frac{21\cdot99+49}{990}$$
\rozwStop
\odpStart
$\frac{21\cdot99+49}{990}$
\odpStop
\testStart
A.$\frac{21\cdot99+49}{990}$\\ B.$-\frac{21\cdot99+49}{990}$\\ C.$2,1$\\ D.$\frac{49\cdot100}{9900}$
\testStop
\kluczStart
A
\kluczStop



\zadStart{Zadanie z Wikieł Z 3.18 b) moja wersja nr 19}

Zamień poniższe ułamki dziesiętne okresowe na ułamki zwykłe $2,1(56)$.
\zadStop
\rozwStart{Patryk Wirkus}{Martyna Czarnobaj}
$$2,1(56)=2,1565656=2,1+(0,056+0,00056+...)=2,1+\frac{0,056}{1-0,01}$$
$$=2,1+\frac{56}{990}=\frac{21\cdot99+56}{990}$$
\rozwStop
\odpStart
$\frac{21\cdot99+56}{990}$
\odpStop
\testStart
A.$\frac{21\cdot99+56}{990}$\\ B.$-\frac{21\cdot99+56}{990}$\\ C.$2,1$\\ D.$\frac{56\cdot100}{9900}$
\testStop
\kluczStart
A
\kluczStop



\zadStart{Zadanie z Wikieł Z 3.18 b) moja wersja nr 20}

Zamień poniższe ułamki dziesiętne okresowe na ułamki zwykłe $2,1(57)$.
\zadStop
\rozwStart{Patryk Wirkus}{Martyna Czarnobaj}
$$2,1(57)=2,1575757=2,1+(0,057+0,00057+...)=2,1+\frac{0,057}{1-0,01}$$
$$=2,1+\frac{57}{990}=\frac{21\cdot99+57}{990}$$
\rozwStop
\odpStart
$\frac{21\cdot99+57}{990}$
\odpStop
\testStart
A.$\frac{21\cdot99+57}{990}$\\ B.$-\frac{21\cdot99+57}{990}$\\ C.$2,1$\\ D.$\frac{57\cdot100}{9900}$
\testStop
\kluczStart
A
\kluczStop



\zadStart{Zadanie z Wikieł Z 3.18 b) moja wersja nr 21}

Zamień poniższe ułamki dziesiętne okresowe na ułamki zwykłe $2,1(58)$.
\zadStop
\rozwStart{Patryk Wirkus}{Martyna Czarnobaj}
$$2,1(58)=2,1585858=2,1+(0,058+0,00058+...)=2,1+\frac{0,058}{1-0,01}$$
$$=2,1+\frac{58}{990}=\frac{21\cdot99+58}{990}$$
\rozwStop
\odpStart
$\frac{21\cdot99+58}{990}$
\odpStop
\testStart
A.$\frac{21\cdot99+58}{990}$\\ B.$-\frac{21\cdot99+58}{990}$\\ C.$2,1$\\ D.$\frac{58\cdot100}{9900}$
\testStop
\kluczStart
A
\kluczStop



\zadStart{Zadanie z Wikieł Z 3.18 b) moja wersja nr 22}

Zamień poniższe ułamki dziesiętne okresowe na ułamki zwykłe $2,1(59)$.
\zadStop
\rozwStart{Patryk Wirkus}{Martyna Czarnobaj}
$$2,1(59)=2,1595959=2,1+(0,059+0,00059+...)=2,1+\frac{0,059}{1-0,01}$$
$$=2,1+\frac{59}{990}=\frac{21\cdot99+59}{990}$$
\rozwStop
\odpStart
$\frac{21\cdot99+59}{990}$
\odpStop
\testStart
A.$\frac{21\cdot99+59}{990}$\\ B.$-\frac{21\cdot99+59}{990}$\\ C.$2,1$\\ D.$\frac{59\cdot100}{9900}$
\testStop
\kluczStart
A
\kluczStop



\zadStart{Zadanie z Wikieł Z 3.18 b) moja wersja nr 23}

Zamień poniższe ułamki dziesiętne okresowe na ułamki zwykłe $2,1(67)$.
\zadStop
\rozwStart{Patryk Wirkus}{Martyna Czarnobaj}
$$2,1(67)=2,1676767=2,1+(0,067+0,00067+...)=2,1+\frac{0,067}{1-0,01}$$
$$=2,1+\frac{67}{990}=\frac{21\cdot99+67}{990}$$
\rozwStop
\odpStart
$\frac{21\cdot99+67}{990}$
\odpStop
\testStart
A.$\frac{21\cdot99+67}{990}$\\ B.$-\frac{21\cdot99+67}{990}$\\ C.$2,1$\\ D.$\frac{67\cdot100}{9900}$
\testStop
\kluczStart
A
\kluczStop



\zadStart{Zadanie z Wikieł Z 3.18 b) moja wersja nr 24}

Zamień poniższe ułamki dziesiętne okresowe na ułamki zwykłe $2,1(68)$.
\zadStop
\rozwStart{Patryk Wirkus}{Martyna Czarnobaj}
$$2,1(68)=2,1686868=2,1+(0,068+0,00068+...)=2,1+\frac{0,068}{1-0,01}$$
$$=2,1+\frac{68}{990}=\frac{21\cdot99+68}{990}$$
\rozwStop
\odpStart
$\frac{21\cdot99+68}{990}$
\odpStop
\testStart
A.$\frac{21\cdot99+68}{990}$\\ B.$-\frac{21\cdot99+68}{990}$\\ C.$2,1$\\ D.$\frac{68\cdot100}{9900}$
\testStop
\kluczStart
A
\kluczStop



\zadStart{Zadanie z Wikieł Z 3.18 b) moja wersja nr 25}

Zamień poniższe ułamki dziesiętne okresowe na ułamki zwykłe $2,1(69)$.
\zadStop
\rozwStart{Patryk Wirkus}{Martyna Czarnobaj}
$$2,1(69)=2,1696969=2,1+(0,069+0,00069+...)=2,1+\frac{0,069}{1-0,01}$$
$$=2,1+\frac{69}{990}=\frac{21\cdot99+69}{990}$$
\rozwStop
\odpStart
$\frac{21\cdot99+69}{990}$
\odpStop
\testStart
A.$\frac{21\cdot99+69}{990}$\\ B.$-\frac{21\cdot99+69}{990}$\\ C.$2,1$\\ D.$\frac{69\cdot100}{9900}$
\testStop
\kluczStart
A
\kluczStop



\zadStart{Zadanie z Wikieł Z 3.18 b) moja wersja nr 26}

Zamień poniższe ułamki dziesiętne okresowe na ułamki zwykłe $2,1(78)$.
\zadStop
\rozwStart{Patryk Wirkus}{Martyna Czarnobaj}
$$2,1(78)=2,1787878=2,1+(0,078+0,00078+...)=2,1+\frac{0,078}{1-0,01}$$
$$=2,1+\frac{78}{990}=\frac{21\cdot99+78}{990}$$
\rozwStop
\odpStart
$\frac{21\cdot99+78}{990}$
\odpStop
\testStart
A.$\frac{21\cdot99+78}{990}$\\ B.$-\frac{21\cdot99+78}{990}$\\ C.$2,1$\\ D.$\frac{78\cdot100}{9900}$
\testStop
\kluczStart
A
\kluczStop



\zadStart{Zadanie z Wikieł Z 3.18 b) moja wersja nr 27}

Zamień poniższe ułamki dziesiętne okresowe na ułamki zwykłe $2,1(79)$.
\zadStop
\rozwStart{Patryk Wirkus}{Martyna Czarnobaj}
$$2,1(79)=2,1797979=2,1+(0,079+0,00079+...)=2,1+\frac{0,079}{1-0,01}$$
$$=2,1+\frac{79}{990}=\frac{21\cdot99+79}{990}$$
\rozwStop
\odpStart
$\frac{21\cdot99+79}{990}$
\odpStop
\testStart
A.$\frac{21\cdot99+79}{990}$\\ B.$-\frac{21\cdot99+79}{990}$\\ C.$2,1$\\ D.$\frac{79\cdot100}{9900}$
\testStop
\kluczStart
A
\kluczStop



\zadStart{Zadanie z Wikieł Z 3.18 b) moja wersja nr 28}

Zamień poniższe ułamki dziesiętne okresowe na ułamki zwykłe $2,1(89)$.
\zadStop
\rozwStart{Patryk Wirkus}{Martyna Czarnobaj}
$$2,1(89)=2,1898989=2,1+(0,089+0,00089+...)=2,1+\frac{0,089}{1-0,01}$$
$$=2,1+\frac{89}{990}=\frac{21\cdot99+89}{990}$$
\rozwStop
\odpStart
$\frac{21\cdot99+89}{990}$
\odpStop
\testStart
A.$\frac{21\cdot99+89}{990}$\\ B.$-\frac{21\cdot99+89}{990}$\\ C.$2,1$\\ D.$\frac{89\cdot100}{9900}$
\testStop
\kluczStart
A
\kluczStop



\zadStart{Zadanie z Wikieł Z 3.18 b) moja wersja nr 29}

Zamień poniższe ułamki dziesiętne okresowe na ułamki zwykłe $2,3(45)$.
\zadStop
\rozwStart{Patryk Wirkus}{Martyna Czarnobaj}
$$2,3(45)=2,3454545=2,3+(0,045+0,00045+...)=2,3+\frac{0,045}{1-0,01}$$
$$=2,3+\frac{45}{990}=\frac{23\cdot99+45}{990}$$
\rozwStop
\odpStart
$\frac{23\cdot99+45}{990}$
\odpStop
\testStart
A.$\frac{23\cdot99+45}{990}$\\ B.$-\frac{23\cdot99+45}{990}$\\ C.$2,3$\\ D.$\frac{45\cdot100}{9900}$
\testStop
\kluczStart
A
\kluczStop



\zadStart{Zadanie z Wikieł Z 3.18 b) moja wersja nr 30}

Zamień poniższe ułamki dziesiętne okresowe na ułamki zwykłe $2,3(46)$.
\zadStop
\rozwStart{Patryk Wirkus}{Martyna Czarnobaj}
$$2,3(46)=2,3464646=2,3+(0,046+0,00046+...)=2,3+\frac{0,046}{1-0,01}$$
$$=2,3+\frac{46}{990}=\frac{23\cdot99+46}{990}$$
\rozwStop
\odpStart
$\frac{23\cdot99+46}{990}$
\odpStop
\testStart
A.$\frac{23\cdot99+46}{990}$\\ B.$-\frac{23\cdot99+46}{990}$\\ C.$2,3$\\ D.$\frac{46\cdot100}{9900}$
\testStop
\kluczStart
A
\kluczStop



\zadStart{Zadanie z Wikieł Z 3.18 b) moja wersja nr 31}

Zamień poniższe ułamki dziesiętne okresowe na ułamki zwykłe $2,3(47)$.
\zadStop
\rozwStart{Patryk Wirkus}{Martyna Czarnobaj}
$$2,3(47)=2,3474747=2,3+(0,047+0,00047+...)=2,3+\frac{0,047}{1-0,01}$$
$$=2,3+\frac{47}{990}=\frac{23\cdot99+47}{990}$$
\rozwStop
\odpStart
$\frac{23\cdot99+47}{990}$
\odpStop
\testStart
A.$\frac{23\cdot99+47}{990}$\\ B.$-\frac{23\cdot99+47}{990}$\\ C.$2,3$\\ D.$\frac{47\cdot100}{9900}$
\testStop
\kluczStart
A
\kluczStop



\zadStart{Zadanie z Wikieł Z 3.18 b) moja wersja nr 32}

Zamień poniższe ułamki dziesiętne okresowe na ułamki zwykłe $2,3(48)$.
\zadStop
\rozwStart{Patryk Wirkus}{Martyna Czarnobaj}
$$2,3(48)=2,3484848=2,3+(0,048+0,00048+...)=2,3+\frac{0,048}{1-0,01}$$
$$=2,3+\frac{48}{990}=\frac{23\cdot99+48}{990}$$
\rozwStop
\odpStart
$\frac{23\cdot99+48}{990}$
\odpStop
\testStart
A.$\frac{23\cdot99+48}{990}$\\ B.$-\frac{23\cdot99+48}{990}$\\ C.$2,3$\\ D.$\frac{48\cdot100}{9900}$
\testStop
\kluczStart
A
\kluczStop



\zadStart{Zadanie z Wikieł Z 3.18 b) moja wersja nr 33}

Zamień poniższe ułamki dziesiętne okresowe na ułamki zwykłe $2,3(49)$.
\zadStop
\rozwStart{Patryk Wirkus}{Martyna Czarnobaj}
$$2,3(49)=2,3494949=2,3+(0,049+0,00049+...)=2,3+\frac{0,049}{1-0,01}$$
$$=2,3+\frac{49}{990}=\frac{23\cdot99+49}{990}$$
\rozwStop
\odpStart
$\frac{23\cdot99+49}{990}$
\odpStop
\testStart
A.$\frac{23\cdot99+49}{990}$\\ B.$-\frac{23\cdot99+49}{990}$\\ C.$2,3$\\ D.$\frac{49\cdot100}{9900}$
\testStop
\kluczStart
A
\kluczStop



\zadStart{Zadanie z Wikieł Z 3.18 b) moja wersja nr 34}

Zamień poniższe ułamki dziesiętne okresowe na ułamki zwykłe $2,3(56)$.
\zadStop
\rozwStart{Patryk Wirkus}{Martyna Czarnobaj}
$$2,3(56)=2,3565656=2,3+(0,056+0,00056+...)=2,3+\frac{0,056}{1-0,01}$$
$$=2,3+\frac{56}{990}=\frac{23\cdot99+56}{990}$$
\rozwStop
\odpStart
$\frac{23\cdot99+56}{990}$
\odpStop
\testStart
A.$\frac{23\cdot99+56}{990}$\\ B.$-\frac{23\cdot99+56}{990}$\\ C.$2,3$\\ D.$\frac{56\cdot100}{9900}$
\testStop
\kluczStart
A
\kluczStop



\zadStart{Zadanie z Wikieł Z 3.18 b) moja wersja nr 35}

Zamień poniższe ułamki dziesiętne okresowe na ułamki zwykłe $2,3(57)$.
\zadStop
\rozwStart{Patryk Wirkus}{Martyna Czarnobaj}
$$2,3(57)=2,3575757=2,3+(0,057+0,00057+...)=2,3+\frac{0,057}{1-0,01}$$
$$=2,3+\frac{57}{990}=\frac{23\cdot99+57}{990}$$
\rozwStop
\odpStart
$\frac{23\cdot99+57}{990}$
\odpStop
\testStart
A.$\frac{23\cdot99+57}{990}$\\ B.$-\frac{23\cdot99+57}{990}$\\ C.$2,3$\\ D.$\frac{57\cdot100}{9900}$
\testStop
\kluczStart
A
\kluczStop



\zadStart{Zadanie z Wikieł Z 3.18 b) moja wersja nr 36}

Zamień poniższe ułamki dziesiętne okresowe na ułamki zwykłe $2,3(58)$.
\zadStop
\rozwStart{Patryk Wirkus}{Martyna Czarnobaj}
$$2,3(58)=2,3585858=2,3+(0,058+0,00058+...)=2,3+\frac{0,058}{1-0,01}$$
$$=2,3+\frac{58}{990}=\frac{23\cdot99+58}{990}$$
\rozwStop
\odpStart
$\frac{23\cdot99+58}{990}$
\odpStop
\testStart
A.$\frac{23\cdot99+58}{990}$\\ B.$-\frac{23\cdot99+58}{990}$\\ C.$2,3$\\ D.$\frac{58\cdot100}{9900}$
\testStop
\kluczStart
A
\kluczStop



\zadStart{Zadanie z Wikieł Z 3.18 b) moja wersja nr 37}

Zamień poniższe ułamki dziesiętne okresowe na ułamki zwykłe $2,3(59)$.
\zadStop
\rozwStart{Patryk Wirkus}{Martyna Czarnobaj}
$$2,3(59)=2,3595959=2,3+(0,059+0,00059+...)=2,3+\frac{0,059}{1-0,01}$$
$$=2,3+\frac{59}{990}=\frac{23\cdot99+59}{990}$$
\rozwStop
\odpStart
$\frac{23\cdot99+59}{990}$
\odpStop
\testStart
A.$\frac{23\cdot99+59}{990}$\\ B.$-\frac{23\cdot99+59}{990}$\\ C.$2,3$\\ D.$\frac{59\cdot100}{9900}$
\testStop
\kluczStart
A
\kluczStop



\zadStart{Zadanie z Wikieł Z 3.18 b) moja wersja nr 38}

Zamień poniższe ułamki dziesiętne okresowe na ułamki zwykłe $2,3(67)$.
\zadStop
\rozwStart{Patryk Wirkus}{Martyna Czarnobaj}
$$2,3(67)=2,3676767=2,3+(0,067+0,00067+...)=2,3+\frac{0,067}{1-0,01}$$
$$=2,3+\frac{67}{990}=\frac{23\cdot99+67}{990}$$
\rozwStop
\odpStart
$\frac{23\cdot99+67}{990}$
\odpStop
\testStart
A.$\frac{23\cdot99+67}{990}$\\ B.$-\frac{23\cdot99+67}{990}$\\ C.$2,3$\\ D.$\frac{67\cdot100}{9900}$
\testStop
\kluczStart
A
\kluczStop



\zadStart{Zadanie z Wikieł Z 3.18 b) moja wersja nr 39}

Zamień poniższe ułamki dziesiętne okresowe na ułamki zwykłe $2,3(68)$.
\zadStop
\rozwStart{Patryk Wirkus}{Martyna Czarnobaj}
$$2,3(68)=2,3686868=2,3+(0,068+0,00068+...)=2,3+\frac{0,068}{1-0,01}$$
$$=2,3+\frac{68}{990}=\frac{23\cdot99+68}{990}$$
\rozwStop
\odpStart
$\frac{23\cdot99+68}{990}$
\odpStop
\testStart
A.$\frac{23\cdot99+68}{990}$\\ B.$-\frac{23\cdot99+68}{990}$\\ C.$2,3$\\ D.$\frac{68\cdot100}{9900}$
\testStop
\kluczStart
A
\kluczStop



\zadStart{Zadanie z Wikieł Z 3.18 b) moja wersja nr 40}

Zamień poniższe ułamki dziesiętne okresowe na ułamki zwykłe $2,3(69)$.
\zadStop
\rozwStart{Patryk Wirkus}{Martyna Czarnobaj}
$$2,3(69)=2,3696969=2,3+(0,069+0,00069+...)=2,3+\frac{0,069}{1-0,01}$$
$$=2,3+\frac{69}{990}=\frac{23\cdot99+69}{990}$$
\rozwStop
\odpStart
$\frac{23\cdot99+69}{990}$
\odpStop
\testStart
A.$\frac{23\cdot99+69}{990}$\\ B.$-\frac{23\cdot99+69}{990}$\\ C.$2,3$\\ D.$\frac{69\cdot100}{9900}$
\testStop
\kluczStart
A
\kluczStop



\zadStart{Zadanie z Wikieł Z 3.18 b) moja wersja nr 41}

Zamień poniższe ułamki dziesiętne okresowe na ułamki zwykłe $2,3(78)$.
\zadStop
\rozwStart{Patryk Wirkus}{Martyna Czarnobaj}
$$2,3(78)=2,3787878=2,3+(0,078+0,00078+...)=2,3+\frac{0,078}{1-0,01}$$
$$=2,3+\frac{78}{990}=\frac{23\cdot99+78}{990}$$
\rozwStop
\odpStart
$\frac{23\cdot99+78}{990}$
\odpStop
\testStart
A.$\frac{23\cdot99+78}{990}$\\ B.$-\frac{23\cdot99+78}{990}$\\ C.$2,3$\\ D.$\frac{78\cdot100}{9900}$
\testStop
\kluczStart
A
\kluczStop



\zadStart{Zadanie z Wikieł Z 3.18 b) moja wersja nr 42}

Zamień poniższe ułamki dziesiętne okresowe na ułamki zwykłe $2,3(79)$.
\zadStop
\rozwStart{Patryk Wirkus}{Martyna Czarnobaj}
$$2,3(79)=2,3797979=2,3+(0,079+0,00079+...)=2,3+\frac{0,079}{1-0,01}$$
$$=2,3+\frac{79}{990}=\frac{23\cdot99+79}{990}$$
\rozwStop
\odpStart
$\frac{23\cdot99+79}{990}$
\odpStop
\testStart
A.$\frac{23\cdot99+79}{990}$\\ B.$-\frac{23\cdot99+79}{990}$\\ C.$2,3$\\ D.$\frac{79\cdot100}{9900}$
\testStop
\kluczStart
A
\kluczStop



\zadStart{Zadanie z Wikieł Z 3.18 b) moja wersja nr 43}

Zamień poniższe ułamki dziesiętne okresowe na ułamki zwykłe $2,3(89)$.
\zadStop
\rozwStart{Patryk Wirkus}{Martyna Czarnobaj}
$$2,3(89)=2,3898989=2,3+(0,089+0,00089+...)=2,3+\frac{0,089}{1-0,01}$$
$$=2,3+\frac{89}{990}=\frac{23\cdot99+89}{990}$$
\rozwStop
\odpStart
$\frac{23\cdot99+89}{990}$
\odpStop
\testStart
A.$\frac{23\cdot99+89}{990}$\\ B.$-\frac{23\cdot99+89}{990}$\\ C.$2,3$\\ D.$\frac{89\cdot100}{9900}$
\testStop
\kluczStart
A
\kluczStop



\zadStart{Zadanie z Wikieł Z 3.18 b) moja wersja nr 44}

Zamień poniższe ułamki dziesiętne okresowe na ułamki zwykłe $2,5(67)$.
\zadStop
\rozwStart{Patryk Wirkus}{Martyna Czarnobaj}
$$2,5(67)=2,5676767=2,5+(0,067+0,00067+...)=2,5+\frac{0,067}{1-0,01}$$
$$=2,5+\frac{67}{990}=\frac{25\cdot99+67}{990}$$
\rozwStop
\odpStart
$\frac{25\cdot99+67}{990}$
\odpStop
\testStart
A.$\frac{25\cdot99+67}{990}$\\ B.$-\frac{25\cdot99+67}{990}$\\ C.$2,5$\\ D.$\frac{67\cdot100}{9900}$
\testStop
\kluczStart
A
\kluczStop



\zadStart{Zadanie z Wikieł Z 3.18 b) moja wersja nr 45}

Zamień poniższe ułamki dziesiętne okresowe na ułamki zwykłe $2,5(68)$.
\zadStop
\rozwStart{Patryk Wirkus}{Martyna Czarnobaj}
$$2,5(68)=2,5686868=2,5+(0,068+0,00068+...)=2,5+\frac{0,068}{1-0,01}$$
$$=2,5+\frac{68}{990}=\frac{25\cdot99+68}{990}$$
\rozwStop
\odpStart
$\frac{25\cdot99+68}{990}$
\odpStop
\testStart
A.$\frac{25\cdot99+68}{990}$\\ B.$-\frac{25\cdot99+68}{990}$\\ C.$2,5$\\ D.$\frac{68\cdot100}{9900}$
\testStop
\kluczStart
A
\kluczStop



\zadStart{Zadanie z Wikieł Z 3.18 b) moja wersja nr 46}

Zamień poniższe ułamki dziesiętne okresowe na ułamki zwykłe $2,5(69)$.
\zadStop
\rozwStart{Patryk Wirkus}{Martyna Czarnobaj}
$$2,5(69)=2,5696969=2,5+(0,069+0,00069+...)=2,5+\frac{0,069}{1-0,01}$$
$$=2,5+\frac{69}{990}=\frac{25\cdot99+69}{990}$$
\rozwStop
\odpStart
$\frac{25\cdot99+69}{990}$
\odpStop
\testStart
A.$\frac{25\cdot99+69}{990}$\\ B.$-\frac{25\cdot99+69}{990}$\\ C.$2,5$\\ D.$\frac{69\cdot100}{9900}$
\testStop
\kluczStart
A
\kluczStop



\zadStart{Zadanie z Wikieł Z 3.18 b) moja wersja nr 47}

Zamień poniższe ułamki dziesiętne okresowe na ułamki zwykłe $2,5(78)$.
\zadStop
\rozwStart{Patryk Wirkus}{Martyna Czarnobaj}
$$2,5(78)=2,5787878=2,5+(0,078+0,00078+...)=2,5+\frac{0,078}{1-0,01}$$
$$=2,5+\frac{78}{990}=\frac{25\cdot99+78}{990}$$
\rozwStop
\odpStart
$\frac{25\cdot99+78}{990}$
\odpStop
\testStart
A.$\frac{25\cdot99+78}{990}$\\ B.$-\frac{25\cdot99+78}{990}$\\ C.$2,5$\\ D.$\frac{78\cdot100}{9900}$
\testStop
\kluczStart
A
\kluczStop



\zadStart{Zadanie z Wikieł Z 3.18 b) moja wersja nr 48}

Zamień poniższe ułamki dziesiętne okresowe na ułamki zwykłe $2,5(79)$.
\zadStop
\rozwStart{Patryk Wirkus}{Martyna Czarnobaj}
$$2,5(79)=2,5797979=2,5+(0,079+0,00079+...)=2,5+\frac{0,079}{1-0,01}$$
$$=2,5+\frac{79}{990}=\frac{25\cdot99+79}{990}$$
\rozwStop
\odpStart
$\frac{25\cdot99+79}{990}$
\odpStop
\testStart
A.$\frac{25\cdot99+79}{990}$\\ B.$-\frac{25\cdot99+79}{990}$\\ C.$2,5$\\ D.$\frac{79\cdot100}{9900}$
\testStop
\kluczStart
A
\kluczStop



\zadStart{Zadanie z Wikieł Z 3.18 b) moja wersja nr 49}

Zamień poniższe ułamki dziesiętne okresowe na ułamki zwykłe $2,5(89)$.
\zadStop
\rozwStart{Patryk Wirkus}{Martyna Czarnobaj}
$$2,5(89)=2,5898989=2,5+(0,089+0,00089+...)=2,5+\frac{0,089}{1-0,01}$$
$$=2,5+\frac{89}{990}=\frac{25\cdot99+89}{990}$$
\rozwStop
\odpStart
$\frac{25\cdot99+89}{990}$
\odpStop
\testStart
A.$\frac{25\cdot99+89}{990}$\\ B.$-\frac{25\cdot99+89}{990}$\\ C.$2,5$\\ D.$\frac{89\cdot100}{9900}$
\testStop
\kluczStart
A
\kluczStop



\zadStart{Zadanie z Wikieł Z 3.18 b) moja wersja nr 50}

Zamień poniższe ułamki dziesiętne okresowe na ułamki zwykłe $2,7(89)$.
\zadStop
\rozwStart{Patryk Wirkus}{Martyna Czarnobaj}
$$2,7(89)=2,7898989=2,7+(0,089+0,00089+...)=2,7+\frac{0,089}{1-0,01}$$
$$=2,7+\frac{89}{990}=\frac{27\cdot99+89}{990}$$
\rozwStop
\odpStart
$\frac{27\cdot99+89}{990}$
\odpStop
\testStart
A.$\frac{27\cdot99+89}{990}$\\ B.$-\frac{27\cdot99+89}{990}$\\ C.$2,7$\\ D.$\frac{89\cdot100}{9900}$
\testStop
\kluczStart
A
\kluczStop





\end{document}
