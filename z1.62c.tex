\documentclass[12pt, a4paper]{article}
\usepackage[utf8]{inputenc}
\usepackage{polski}
\usepackage{amsthm}  %pakiet do tworzenia twierdzeń itp.
\usepackage{amsmath} %pakiet do niektórych symboli matematycznych
\usepackage{amssymb} %pakiet do symboli mat., np. \nsubseteq
\usepackage{amsfonts}
\usepackage{graphicx} %obsługa plików graficznych z rozszerzeniem png, jpg
\theoremstyle{definition} %styl dla definicji
\newtheorem{zad}{} 
\title{Multizestaw zadań}
\author{Patryk Wirkus}
%\date{\today}
\date{}
\newcommand{\kategoria}[1]{\section{#1}}
\newcommand{\zadStart}[1]{\begin{zad}#1\newline}
\newcommand{\zadStop}{\end{zad}}
\newcommand{\rozwStart}[2]{\noindent \textbf{Rozwiązanie (autor #1 , recenzent #2): }\newline}
\newcommand{\rozwStop}{\newline}                                           
\newcommand{\odpStart}{\noindent \textbf{Odpowiedź:}\newline}
\newcommand{\odpStop}{\newline}
\newcommand{\testStart}{\noindent \textbf{Test:}\newline}
\newcommand{\testStop}{\newline}
\newcommand{\kluczStart}{\noindent \textbf{Test poprawna odpowiedź:}\newline}
\newcommand{\kluczStop}{\newline}
\newcommand{\wstawGrafike}[2]{\begin{figure}[h] \includegraphics[scale=#2] {#1} \end{figure}}

\begin{document}
\maketitle

\kategoria{Wikieł/1.62c}


\zadStart{Zadanie z Wikieł Z 1.62 c) moja wersja nr 1}

Rozwiązać nierówności $(2-x)(x+1)^{2}(3-x)^{3}\le0$.
\zadStop
\rozwStart{Patryk Wirkus}{Laura Mieczkowska}
Miejsca zerowe naszego wielomianu to: $2, -1, 3$.\\
Wielomian jest stopnia parzystego, ponadto znak współczynnika przy\linebreak najwyższej potędze x jest ujemny.\\ W związku z tym wykres wielomianu zaczyna się od lewej strony powyżej osi OX.\\
Ponadto w punkcie $-1$ wykres odbija się od osi poziomej.\\
A więc $$x \in \{-1\} \cup [2,3].$$
\rozwStop
\odpStart
$x \in \{-1\} \cup [2,3]$
\odpStop
\testStart
A.$x \in \{-1\} \cup [2,3]$\\
B.$x \in \{1\} \cup (2,3)$\\
C.$x \in \{-1\} \cup (2,3]$\\
D.$x \in \{1\} \cup (2,3]$\\
E.$x \in \{-1\} \cup [2,3)$\\
F.$x \in \{1\} \cup [2,3)$\\
G.$x \in \{-1\} \cup (2,3)$\\
H.$x \in \{1\} \cup [2,3]$
\testStop
\kluczStart
A
\kluczStop



\zadStart{Zadanie z Wikieł Z 1.62 c) moja wersja nr 2}

Rozwiązać nierówności $(2-x)(x+1)^{2}(4-x)^{3}\le0$.
\zadStop
\rozwStart{Patryk Wirkus}{Laura Mieczkowska}
Miejsca zerowe naszego wielomianu to: $2, -1, 4$.\\
Wielomian jest stopnia parzystego, ponadto znak współczynnika przy\linebreak najwyższej potędze x jest ujemny.\\ W związku z tym wykres wielomianu zaczyna się od lewej strony powyżej osi OX.\\
Ponadto w punkcie $-1$ wykres odbija się od osi poziomej.\\
A więc $$x \in \{-1\} \cup [2,4].$$
\rozwStop
\odpStart
$x \in \{-1\} \cup [2,4]$
\odpStop
\testStart
A.$x \in \{-1\} \cup [2,4]$\\
B.$x \in \{1\} \cup (2,4)$\\
C.$x \in \{-1\} \cup (2,4]$\\
D.$x \in \{1\} \cup (2,4]$\\
E.$x \in \{-1\} \cup [2,4)$\\
F.$x \in \{1\} \cup [2,4)$\\
G.$x \in \{-1\} \cup (2,4)$\\
H.$x \in \{1\} \cup [2,4]$
\testStop
\kluczStart
A
\kluczStop



\zadStart{Zadanie z Wikieł Z 1.62 c) moja wersja nr 3}

Rozwiązać nierówności $(2-x)(x+1)^{2}(5-x)^{3}\le0$.
\zadStop
\rozwStart{Patryk Wirkus}{Laura Mieczkowska}
Miejsca zerowe naszego wielomianu to: $2, -1, 5$.\\
Wielomian jest stopnia parzystego, ponadto znak współczynnika przy\linebreak najwyższej potędze x jest ujemny.\\ W związku z tym wykres wielomianu zaczyna się od lewej strony powyżej osi OX.\\
Ponadto w punkcie $-1$ wykres odbija się od osi poziomej.\\
A więc $$x \in \{-1\} \cup [2,5].$$
\rozwStop
\odpStart
$x \in \{-1\} \cup [2,5]$
\odpStop
\testStart
A.$x \in \{-1\} \cup [2,5]$\\
B.$x \in \{1\} \cup (2,5)$\\
C.$x \in \{-1\} \cup (2,5]$\\
D.$x \in \{1\} \cup (2,5]$\\
E.$x \in \{-1\} \cup [2,5)$\\
F.$x \in \{1\} \cup [2,5)$\\
G.$x \in \{-1\} \cup (2,5)$\\
H.$x \in \{1\} \cup [2,5]$
\testStop
\kluczStart
A
\kluczStop



\zadStart{Zadanie z Wikieł Z 1.62 c) moja wersja nr 4}

Rozwiązać nierówności $(2-x)(x+1)^{2}(6-x)^{3}\le0$.
\zadStop
\rozwStart{Patryk Wirkus}{Laura Mieczkowska}
Miejsca zerowe naszego wielomianu to: $2, -1, 6$.\\
Wielomian jest stopnia parzystego, ponadto znak współczynnika przy\linebreak najwyższej potędze x jest ujemny.\\ W związku z tym wykres wielomianu zaczyna się od lewej strony powyżej osi OX.\\
Ponadto w punkcie $-1$ wykres odbija się od osi poziomej.\\
A więc $$x \in \{-1\} \cup [2,6].$$
\rozwStop
\odpStart
$x \in \{-1\} \cup [2,6]$
\odpStop
\testStart
A.$x \in \{-1\} \cup [2,6]$\\
B.$x \in \{1\} \cup (2,6)$\\
C.$x \in \{-1\} \cup (2,6]$\\
D.$x \in \{1\} \cup (2,6]$\\
E.$x \in \{-1\} \cup [2,6)$\\
F.$x \in \{1\} \cup [2,6)$\\
G.$x \in \{-1\} \cup (2,6)$\\
H.$x \in \{1\} \cup [2,6]$
\testStop
\kluczStart
A
\kluczStop



\zadStart{Zadanie z Wikieł Z 1.62 c) moja wersja nr 5}

Rozwiązać nierówności $(2-x)(x+1)^{2}(7-x)^{3}\le0$.
\zadStop
\rozwStart{Patryk Wirkus}{Laura Mieczkowska}
Miejsca zerowe naszego wielomianu to: $2, -1, 7$.\\
Wielomian jest stopnia parzystego, ponadto znak współczynnika przy\linebreak najwyższej potędze x jest ujemny.\\ W związku z tym wykres wielomianu zaczyna się od lewej strony powyżej osi OX.\\
Ponadto w punkcie $-1$ wykres odbija się od osi poziomej.\\
A więc $$x \in \{-1\} \cup [2,7].$$
\rozwStop
\odpStart
$x \in \{-1\} \cup [2,7]$
\odpStop
\testStart
A.$x \in \{-1\} \cup [2,7]$\\
B.$x \in \{1\} \cup (2,7)$\\
C.$x \in \{-1\} \cup (2,7]$\\
D.$x \in \{1\} \cup (2,7]$\\
E.$x \in \{-1\} \cup [2,7)$\\
F.$x \in \{1\} \cup [2,7)$\\
G.$x \in \{-1\} \cup (2,7)$\\
H.$x \in \{1\} \cup [2,7]$
\testStop
\kluczStart
A
\kluczStop



\zadStart{Zadanie z Wikieł Z 1.62 c) moja wersja nr 6}

Rozwiązać nierówności $(2-x)(x+1)^{2}(8-x)^{3}\le0$.
\zadStop
\rozwStart{Patryk Wirkus}{Laura Mieczkowska}
Miejsca zerowe naszego wielomianu to: $2, -1, 8$.\\
Wielomian jest stopnia parzystego, ponadto znak współczynnika przy\linebreak najwyższej potędze x jest ujemny.\\ W związku z tym wykres wielomianu zaczyna się od lewej strony powyżej osi OX.\\
Ponadto w punkcie $-1$ wykres odbija się od osi poziomej.\\
A więc $$x \in \{-1\} \cup [2,8].$$
\rozwStop
\odpStart
$x \in \{-1\} \cup [2,8]$
\odpStop
\testStart
A.$x \in \{-1\} \cup [2,8]$\\
B.$x \in \{1\} \cup (2,8)$\\
C.$x \in \{-1\} \cup (2,8]$\\
D.$x \in \{1\} \cup (2,8]$\\
E.$x \in \{-1\} \cup [2,8)$\\
F.$x \in \{1\} \cup [2,8)$\\
G.$x \in \{-1\} \cup (2,8)$\\
H.$x \in \{1\} \cup [2,8]$
\testStop
\kluczStart
A
\kluczStop



\zadStart{Zadanie z Wikieł Z 1.62 c) moja wersja nr 7}

Rozwiązać nierówności $(2-x)(x+1)^{2}(9-x)^{3}\le0$.
\zadStop
\rozwStart{Patryk Wirkus}{Laura Mieczkowska}
Miejsca zerowe naszego wielomianu to: $2, -1, 9$.\\
Wielomian jest stopnia parzystego, ponadto znak współczynnika przy\linebreak najwyższej potędze x jest ujemny.\\ W związku z tym wykres wielomianu zaczyna się od lewej strony powyżej osi OX.\\
Ponadto w punkcie $-1$ wykres odbija się od osi poziomej.\\
A więc $$x \in \{-1\} \cup [2,9].$$
\rozwStop
\odpStart
$x \in \{-1\} \cup [2,9]$
\odpStop
\testStart
A.$x \in \{-1\} \cup [2,9]$\\
B.$x \in \{1\} \cup (2,9)$\\
C.$x \in \{-1\} \cup (2,9]$\\
D.$x \in \{1\} \cup (2,9]$\\
E.$x \in \{-1\} \cup [2,9)$\\
F.$x \in \{1\} \cup [2,9)$\\
G.$x \in \{-1\} \cup (2,9)$\\
H.$x \in \{1\} \cup [2,9]$
\testStop
\kluczStart
A
\kluczStop



\zadStart{Zadanie z Wikieł Z 1.62 c) moja wersja nr 8}

Rozwiązać nierówności $(2-x)(x+1)^{2}(10-x)^{3}\le0$.
\zadStop
\rozwStart{Patryk Wirkus}{Laura Mieczkowska}
Miejsca zerowe naszego wielomianu to: $2, -1, 10$.\\
Wielomian jest stopnia parzystego, ponadto znak współczynnika przy\linebreak najwyższej potędze x jest ujemny.\\ W związku z tym wykres wielomianu zaczyna się od lewej strony powyżej osi OX.\\
Ponadto w punkcie $-1$ wykres odbija się od osi poziomej.\\
A więc $$x \in \{-1\} \cup [2,10].$$
\rozwStop
\odpStart
$x \in \{-1\} \cup [2,10]$
\odpStop
\testStart
A.$x \in \{-1\} \cup [2,10]$\\
B.$x \in \{1\} \cup (2,10)$\\
C.$x \in \{-1\} \cup (2,10]$\\
D.$x \in \{1\} \cup (2,10]$\\
E.$x \in \{-1\} \cup [2,10)$\\
F.$x \in \{1\} \cup [2,10)$\\
G.$x \in \{-1\} \cup (2,10)$\\
H.$x \in \{1\} \cup [2,10]$
\testStop
\kluczStart
A
\kluczStop



\zadStart{Zadanie z Wikieł Z 1.62 c) moja wersja nr 9}

Rozwiązać nierówności $(2-x)(x+1)^{2}(11-x)^{3}\le0$.
\zadStop
\rozwStart{Patryk Wirkus}{Laura Mieczkowska}
Miejsca zerowe naszego wielomianu to: $2, -1, 11$.\\
Wielomian jest stopnia parzystego, ponadto znak współczynnika przy\linebreak najwyższej potędze x jest ujemny.\\ W związku z tym wykres wielomianu zaczyna się od lewej strony powyżej osi OX.\\
Ponadto w punkcie $-1$ wykres odbija się od osi poziomej.\\
A więc $$x \in \{-1\} \cup [2,11].$$
\rozwStop
\odpStart
$x \in \{-1\} \cup [2,11]$
\odpStop
\testStart
A.$x \in \{-1\} \cup [2,11]$\\
B.$x \in \{1\} \cup (2,11)$\\
C.$x \in \{-1\} \cup (2,11]$\\
D.$x \in \{1\} \cup (2,11]$\\
E.$x \in \{-1\} \cup [2,11)$\\
F.$x \in \{1\} \cup [2,11)$\\
G.$x \in \{-1\} \cup (2,11)$\\
H.$x \in \{1\} \cup [2,11]$
\testStop
\kluczStart
A
\kluczStop



\zadStart{Zadanie z Wikieł Z 1.62 c) moja wersja nr 10}

Rozwiązać nierówności $(2-x)(x+1)^{2}(12-x)^{3}\le0$.
\zadStop
\rozwStart{Patryk Wirkus}{Laura Mieczkowska}
Miejsca zerowe naszego wielomianu to: $2, -1, 12$.\\
Wielomian jest stopnia parzystego, ponadto znak współczynnika przy\linebreak najwyższej potędze x jest ujemny.\\ W związku z tym wykres wielomianu zaczyna się od lewej strony powyżej osi OX.\\
Ponadto w punkcie $-1$ wykres odbija się od osi poziomej.\\
A więc $$x \in \{-1\} \cup [2,12].$$
\rozwStop
\odpStart
$x \in \{-1\} \cup [2,12]$
\odpStop
\testStart
A.$x \in \{-1\} \cup [2,12]$\\
B.$x \in \{1\} \cup (2,12)$\\
C.$x \in \{-1\} \cup (2,12]$\\
D.$x \in \{1\} \cup (2,12]$\\
E.$x \in \{-1\} \cup [2,12)$\\
F.$x \in \{1\} \cup [2,12)$\\
G.$x \in \{-1\} \cup (2,12)$\\
H.$x \in \{1\} \cup [2,12]$
\testStop
\kluczStart
A
\kluczStop



\zadStart{Zadanie z Wikieł Z 1.62 c) moja wersja nr 11}

Rozwiązać nierówności $(2-x)(x+1)^{2}(13-x)^{3}\le0$.
\zadStop
\rozwStart{Patryk Wirkus}{Laura Mieczkowska}
Miejsca zerowe naszego wielomianu to: $2, -1, 13$.\\
Wielomian jest stopnia parzystego, ponadto znak współczynnika przy\linebreak najwyższej potędze x jest ujemny.\\ W związku z tym wykres wielomianu zaczyna się od lewej strony powyżej osi OX.\\
Ponadto w punkcie $-1$ wykres odbija się od osi poziomej.\\
A więc $$x \in \{-1\} \cup [2,13].$$
\rozwStop
\odpStart
$x \in \{-1\} \cup [2,13]$
\odpStop
\testStart
A.$x \in \{-1\} \cup [2,13]$\\
B.$x \in \{1\} \cup (2,13)$\\
C.$x \in \{-1\} \cup (2,13]$\\
D.$x \in \{1\} \cup (2,13]$\\
E.$x \in \{-1\} \cup [2,13)$\\
F.$x \in \{1\} \cup [2,13)$\\
G.$x \in \{-1\} \cup (2,13)$\\
H.$x \in \{1\} \cup [2,13]$
\testStop
\kluczStart
A
\kluczStop



\zadStart{Zadanie z Wikieł Z 1.62 c) moja wersja nr 12}

Rozwiązać nierówności $(2-x)(x+1)^{2}(14-x)^{3}\le0$.
\zadStop
\rozwStart{Patryk Wirkus}{Laura Mieczkowska}
Miejsca zerowe naszego wielomianu to: $2, -1, 14$.\\
Wielomian jest stopnia parzystego, ponadto znak współczynnika przy\linebreak najwyższej potędze x jest ujemny.\\ W związku z tym wykres wielomianu zaczyna się od lewej strony powyżej osi OX.\\
Ponadto w punkcie $-1$ wykres odbija się od osi poziomej.\\
A więc $$x \in \{-1\} \cup [2,14].$$
\rozwStop
\odpStart
$x \in \{-1\} \cup [2,14]$
\odpStop
\testStart
A.$x \in \{-1\} \cup [2,14]$\\
B.$x \in \{1\} \cup (2,14)$\\
C.$x \in \{-1\} \cup (2,14]$\\
D.$x \in \{1\} \cup (2,14]$\\
E.$x \in \{-1\} \cup [2,14)$\\
F.$x \in \{1\} \cup [2,14)$\\
G.$x \in \{-1\} \cup (2,14)$\\
H.$x \in \{1\} \cup [2,14]$
\testStop
\kluczStart
A
\kluczStop



\zadStart{Zadanie z Wikieł Z 1.62 c) moja wersja nr 13}

Rozwiązać nierówności $(2-x)(x+1)^{2}(15-x)^{3}\le0$.
\zadStop
\rozwStart{Patryk Wirkus}{Laura Mieczkowska}
Miejsca zerowe naszego wielomianu to: $2, -1, 15$.\\
Wielomian jest stopnia parzystego, ponadto znak współczynnika przy\linebreak najwyższej potędze x jest ujemny.\\ W związku z tym wykres wielomianu zaczyna się od lewej strony powyżej osi OX.\\
Ponadto w punkcie $-1$ wykres odbija się od osi poziomej.\\
A więc $$x \in \{-1\} \cup [2,15].$$
\rozwStop
\odpStart
$x \in \{-1\} \cup [2,15]$
\odpStop
\testStart
A.$x \in \{-1\} \cup [2,15]$\\
B.$x \in \{1\} \cup (2,15)$\\
C.$x \in \{-1\} \cup (2,15]$\\
D.$x \in \{1\} \cup (2,15]$\\
E.$x \in \{-1\} \cup [2,15)$\\
F.$x \in \{1\} \cup [2,15)$\\
G.$x \in \{-1\} \cup (2,15)$\\
H.$x \in \{1\} \cup [2,15]$
\testStop
\kluczStart
A
\kluczStop



\zadStart{Zadanie z Wikieł Z 1.62 c) moja wersja nr 14}

Rozwiązać nierówności $(3-x)(x+1)^{2}(4-x)^{3}\le0$.
\zadStop
\rozwStart{Patryk Wirkus}{Laura Mieczkowska}
Miejsca zerowe naszego wielomianu to: $3, -1, 4$.\\
Wielomian jest stopnia parzystego, ponadto znak współczynnika przy\linebreak najwyższej potędze x jest ujemny.\\ W związku z tym wykres wielomianu zaczyna się od lewej strony powyżej osi OX.\\
Ponadto w punkcie $-1$ wykres odbija się od osi poziomej.\\
A więc $$x \in \{-1\} \cup [3,4].$$
\rozwStop
\odpStart
$x \in \{-1\} \cup [3,4]$
\odpStop
\testStart
A.$x \in \{-1\} \cup [3,4]$\\
B.$x \in \{1\} \cup (3,4)$\\
C.$x \in \{-1\} \cup (3,4]$\\
D.$x \in \{1\} \cup (3,4]$\\
E.$x \in \{-1\} \cup [3,4)$\\
F.$x \in \{1\} \cup [3,4)$\\
G.$x \in \{-1\} \cup (3,4)$\\
H.$x \in \{1\} \cup [3,4]$
\testStop
\kluczStart
A
\kluczStop



\zadStart{Zadanie z Wikieł Z 1.62 c) moja wersja nr 15}

Rozwiązać nierówności $(3-x)(x+1)^{2}(5-x)^{3}\le0$.
\zadStop
\rozwStart{Patryk Wirkus}{Laura Mieczkowska}
Miejsca zerowe naszego wielomianu to: $3, -1, 5$.\\
Wielomian jest stopnia parzystego, ponadto znak współczynnika przy\linebreak najwyższej potędze x jest ujemny.\\ W związku z tym wykres wielomianu zaczyna się od lewej strony powyżej osi OX.\\
Ponadto w punkcie $-1$ wykres odbija się od osi poziomej.\\
A więc $$x \in \{-1\} \cup [3,5].$$
\rozwStop
\odpStart
$x \in \{-1\} \cup [3,5]$
\odpStop
\testStart
A.$x \in \{-1\} \cup [3,5]$\\
B.$x \in \{1\} \cup (3,5)$\\
C.$x \in \{-1\} \cup (3,5]$\\
D.$x \in \{1\} \cup (3,5]$\\
E.$x \in \{-1\} \cup [3,5)$\\
F.$x \in \{1\} \cup [3,5)$\\
G.$x \in \{-1\} \cup (3,5)$\\
H.$x \in \{1\} \cup [3,5]$
\testStop
\kluczStart
A
\kluczStop



\zadStart{Zadanie z Wikieł Z 1.62 c) moja wersja nr 16}

Rozwiązać nierówności $(3-x)(x+1)^{2}(6-x)^{3}\le0$.
\zadStop
\rozwStart{Patryk Wirkus}{Laura Mieczkowska}
Miejsca zerowe naszego wielomianu to: $3, -1, 6$.\\
Wielomian jest stopnia parzystego, ponadto znak współczynnika przy\linebreak najwyższej potędze x jest ujemny.\\ W związku z tym wykres wielomianu zaczyna się od lewej strony powyżej osi OX.\\
Ponadto w punkcie $-1$ wykres odbija się od osi poziomej.\\
A więc $$x \in \{-1\} \cup [3,6].$$
\rozwStop
\odpStart
$x \in \{-1\} \cup [3,6]$
\odpStop
\testStart
A.$x \in \{-1\} \cup [3,6]$\\
B.$x \in \{1\} \cup (3,6)$\\
C.$x \in \{-1\} \cup (3,6]$\\
D.$x \in \{1\} \cup (3,6]$\\
E.$x \in \{-1\} \cup [3,6)$\\
F.$x \in \{1\} \cup [3,6)$\\
G.$x \in \{-1\} \cup (3,6)$\\
H.$x \in \{1\} \cup [3,6]$
\testStop
\kluczStart
A
\kluczStop



\zadStart{Zadanie z Wikieł Z 1.62 c) moja wersja nr 17}

Rozwiązać nierówności $(3-x)(x+1)^{2}(7-x)^{3}\le0$.
\zadStop
\rozwStart{Patryk Wirkus}{Laura Mieczkowska}
Miejsca zerowe naszego wielomianu to: $3, -1, 7$.\\
Wielomian jest stopnia parzystego, ponadto znak współczynnika przy\linebreak najwyższej potędze x jest ujemny.\\ W związku z tym wykres wielomianu zaczyna się od lewej strony powyżej osi OX.\\
Ponadto w punkcie $-1$ wykres odbija się od osi poziomej.\\
A więc $$x \in \{-1\} \cup [3,7].$$
\rozwStop
\odpStart
$x \in \{-1\} \cup [3,7]$
\odpStop
\testStart
A.$x \in \{-1\} \cup [3,7]$\\
B.$x \in \{1\} \cup (3,7)$\\
C.$x \in \{-1\} \cup (3,7]$\\
D.$x \in \{1\} \cup (3,7]$\\
E.$x \in \{-1\} \cup [3,7)$\\
F.$x \in \{1\} \cup [3,7)$\\
G.$x \in \{-1\} \cup (3,7)$\\
H.$x \in \{1\} \cup [3,7]$
\testStop
\kluczStart
A
\kluczStop



\zadStart{Zadanie z Wikieł Z 1.62 c) moja wersja nr 18}

Rozwiązać nierówności $(3-x)(x+1)^{2}(8-x)^{3}\le0$.
\zadStop
\rozwStart{Patryk Wirkus}{Laura Mieczkowska}
Miejsca zerowe naszego wielomianu to: $3, -1, 8$.\\
Wielomian jest stopnia parzystego, ponadto znak współczynnika przy\linebreak najwyższej potędze x jest ujemny.\\ W związku z tym wykres wielomianu zaczyna się od lewej strony powyżej osi OX.\\
Ponadto w punkcie $-1$ wykres odbija się od osi poziomej.\\
A więc $$x \in \{-1\} \cup [3,8].$$
\rozwStop
\odpStart
$x \in \{-1\} \cup [3,8]$
\odpStop
\testStart
A.$x \in \{-1\} \cup [3,8]$\\
B.$x \in \{1\} \cup (3,8)$\\
C.$x \in \{-1\} \cup (3,8]$\\
D.$x \in \{1\} \cup (3,8]$\\
E.$x \in \{-1\} \cup [3,8)$\\
F.$x \in \{1\} \cup [3,8)$\\
G.$x \in \{-1\} \cup (3,8)$\\
H.$x \in \{1\} \cup [3,8]$
\testStop
\kluczStart
A
\kluczStop



\zadStart{Zadanie z Wikieł Z 1.62 c) moja wersja nr 19}

Rozwiązać nierówności $(3-x)(x+1)^{2}(9-x)^{3}\le0$.
\zadStop
\rozwStart{Patryk Wirkus}{Laura Mieczkowska}
Miejsca zerowe naszego wielomianu to: $3, -1, 9$.\\
Wielomian jest stopnia parzystego, ponadto znak współczynnika przy\linebreak najwyższej potędze x jest ujemny.\\ W związku z tym wykres wielomianu zaczyna się od lewej strony powyżej osi OX.\\
Ponadto w punkcie $-1$ wykres odbija się od osi poziomej.\\
A więc $$x \in \{-1\} \cup [3,9].$$
\rozwStop
\odpStart
$x \in \{-1\} \cup [3,9]$
\odpStop
\testStart
A.$x \in \{-1\} \cup [3,9]$\\
B.$x \in \{1\} \cup (3,9)$\\
C.$x \in \{-1\} \cup (3,9]$\\
D.$x \in \{1\} \cup (3,9]$\\
E.$x \in \{-1\} \cup [3,9)$\\
F.$x \in \{1\} \cup [3,9)$\\
G.$x \in \{-1\} \cup (3,9)$\\
H.$x \in \{1\} \cup [3,9]$
\testStop
\kluczStart
A
\kluczStop



\zadStart{Zadanie z Wikieł Z 1.62 c) moja wersja nr 20}

Rozwiązać nierówności $(3-x)(x+1)^{2}(10-x)^{3}\le0$.
\zadStop
\rozwStart{Patryk Wirkus}{Laura Mieczkowska}
Miejsca zerowe naszego wielomianu to: $3, -1, 10$.\\
Wielomian jest stopnia parzystego, ponadto znak współczynnika przy\linebreak najwyższej potędze x jest ujemny.\\ W związku z tym wykres wielomianu zaczyna się od lewej strony powyżej osi OX.\\
Ponadto w punkcie $-1$ wykres odbija się od osi poziomej.\\
A więc $$x \in \{-1\} \cup [3,10].$$
\rozwStop
\odpStart
$x \in \{-1\} \cup [3,10]$
\odpStop
\testStart
A.$x \in \{-1\} \cup [3,10]$\\
B.$x \in \{1\} \cup (3,10)$\\
C.$x \in \{-1\} \cup (3,10]$\\
D.$x \in \{1\} \cup (3,10]$\\
E.$x \in \{-1\} \cup [3,10)$\\
F.$x \in \{1\} \cup [3,10)$\\
G.$x \in \{-1\} \cup (3,10)$\\
H.$x \in \{1\} \cup [3,10]$
\testStop
\kluczStart
A
\kluczStop



\zadStart{Zadanie z Wikieł Z 1.62 c) moja wersja nr 21}

Rozwiązać nierówności $(3-x)(x+1)^{2}(11-x)^{3}\le0$.
\zadStop
\rozwStart{Patryk Wirkus}{Laura Mieczkowska}
Miejsca zerowe naszego wielomianu to: $3, -1, 11$.\\
Wielomian jest stopnia parzystego, ponadto znak współczynnika przy\linebreak najwyższej potędze x jest ujemny.\\ W związku z tym wykres wielomianu zaczyna się od lewej strony powyżej osi OX.\\
Ponadto w punkcie $-1$ wykres odbija się od osi poziomej.\\
A więc $$x \in \{-1\} \cup [3,11].$$
\rozwStop
\odpStart
$x \in \{-1\} \cup [3,11]$
\odpStop
\testStart
A.$x \in \{-1\} \cup [3,11]$\\
B.$x \in \{1\} \cup (3,11)$\\
C.$x \in \{-1\} \cup (3,11]$\\
D.$x \in \{1\} \cup (3,11]$\\
E.$x \in \{-1\} \cup [3,11)$\\
F.$x \in \{1\} \cup [3,11)$\\
G.$x \in \{-1\} \cup (3,11)$\\
H.$x \in \{1\} \cup [3,11]$
\testStop
\kluczStart
A
\kluczStop



\zadStart{Zadanie z Wikieł Z 1.62 c) moja wersja nr 22}

Rozwiązać nierówności $(3-x)(x+1)^{2}(12-x)^{3}\le0$.
\zadStop
\rozwStart{Patryk Wirkus}{Laura Mieczkowska}
Miejsca zerowe naszego wielomianu to: $3, -1, 12$.\\
Wielomian jest stopnia parzystego, ponadto znak współczynnika przy\linebreak najwyższej potędze x jest ujemny.\\ W związku z tym wykres wielomianu zaczyna się od lewej strony powyżej osi OX.\\
Ponadto w punkcie $-1$ wykres odbija się od osi poziomej.\\
A więc $$x \in \{-1\} \cup [3,12].$$
\rozwStop
\odpStart
$x \in \{-1\} \cup [3,12]$
\odpStop
\testStart
A.$x \in \{-1\} \cup [3,12]$\\
B.$x \in \{1\} \cup (3,12)$\\
C.$x \in \{-1\} \cup (3,12]$\\
D.$x \in \{1\} \cup (3,12]$\\
E.$x \in \{-1\} \cup [3,12)$\\
F.$x \in \{1\} \cup [3,12)$\\
G.$x \in \{-1\} \cup (3,12)$\\
H.$x \in \{1\} \cup [3,12]$
\testStop
\kluczStart
A
\kluczStop



\zadStart{Zadanie z Wikieł Z 1.62 c) moja wersja nr 23}

Rozwiązać nierówności $(3-x)(x+1)^{2}(13-x)^{3}\le0$.
\zadStop
\rozwStart{Patryk Wirkus}{Laura Mieczkowska}
Miejsca zerowe naszego wielomianu to: $3, -1, 13$.\\
Wielomian jest stopnia parzystego, ponadto znak współczynnika przy\linebreak najwyższej potędze x jest ujemny.\\ W związku z tym wykres wielomianu zaczyna się od lewej strony powyżej osi OX.\\
Ponadto w punkcie $-1$ wykres odbija się od osi poziomej.\\
A więc $$x \in \{-1\} \cup [3,13].$$
\rozwStop
\odpStart
$x \in \{-1\} \cup [3,13]$
\odpStop
\testStart
A.$x \in \{-1\} \cup [3,13]$\\
B.$x \in \{1\} \cup (3,13)$\\
C.$x \in \{-1\} \cup (3,13]$\\
D.$x \in \{1\} \cup (3,13]$\\
E.$x \in \{-1\} \cup [3,13)$\\
F.$x \in \{1\} \cup [3,13)$\\
G.$x \in \{-1\} \cup (3,13)$\\
H.$x \in \{1\} \cup [3,13]$
\testStop
\kluczStart
A
\kluczStop



\zadStart{Zadanie z Wikieł Z 1.62 c) moja wersja nr 24}

Rozwiązać nierówności $(3-x)(x+1)^{2}(14-x)^{3}\le0$.
\zadStop
\rozwStart{Patryk Wirkus}{Laura Mieczkowska}
Miejsca zerowe naszego wielomianu to: $3, -1, 14$.\\
Wielomian jest stopnia parzystego, ponadto znak współczynnika przy\linebreak najwyższej potędze x jest ujemny.\\ W związku z tym wykres wielomianu zaczyna się od lewej strony powyżej osi OX.\\
Ponadto w punkcie $-1$ wykres odbija się od osi poziomej.\\
A więc $$x \in \{-1\} \cup [3,14].$$
\rozwStop
\odpStart
$x \in \{-1\} \cup [3,14]$
\odpStop
\testStart
A.$x \in \{-1\} \cup [3,14]$\\
B.$x \in \{1\} \cup (3,14)$\\
C.$x \in \{-1\} \cup (3,14]$\\
D.$x \in \{1\} \cup (3,14]$\\
E.$x \in \{-1\} \cup [3,14)$\\
F.$x \in \{1\} \cup [3,14)$\\
G.$x \in \{-1\} \cup (3,14)$\\
H.$x \in \{1\} \cup [3,14]$
\testStop
\kluczStart
A
\kluczStop



\zadStart{Zadanie z Wikieł Z 1.62 c) moja wersja nr 25}

Rozwiązać nierówności $(3-x)(x+1)^{2}(15-x)^{3}\le0$.
\zadStop
\rozwStart{Patryk Wirkus}{Laura Mieczkowska}
Miejsca zerowe naszego wielomianu to: $3, -1, 15$.\\
Wielomian jest stopnia parzystego, ponadto znak współczynnika przy\linebreak najwyższej potędze x jest ujemny.\\ W związku z tym wykres wielomianu zaczyna się od lewej strony powyżej osi OX.\\
Ponadto w punkcie $-1$ wykres odbija się od osi poziomej.\\
A więc $$x \in \{-1\} \cup [3,15].$$
\rozwStop
\odpStart
$x \in \{-1\} \cup [3,15]$
\odpStop
\testStart
A.$x \in \{-1\} \cup [3,15]$\\
B.$x \in \{1\} \cup (3,15)$\\
C.$x \in \{-1\} \cup (3,15]$\\
D.$x \in \{1\} \cup (3,15]$\\
E.$x \in \{-1\} \cup [3,15)$\\
F.$x \in \{1\} \cup [3,15)$\\
G.$x \in \{-1\} \cup (3,15)$\\
H.$x \in \{1\} \cup [3,15]$
\testStop
\kluczStart
A
\kluczStop



\zadStart{Zadanie z Wikieł Z 1.62 c) moja wersja nr 26}

Rozwiązać nierówności $(3-x)(x+2)^{2}(4-x)^{3}\le0$.
\zadStop
\rozwStart{Patryk Wirkus}{Laura Mieczkowska}
Miejsca zerowe naszego wielomianu to: $3, -2, 4$.\\
Wielomian jest stopnia parzystego, ponadto znak współczynnika przy\linebreak najwyższej potędze x jest ujemny.\\ W związku z tym wykres wielomianu zaczyna się od lewej strony powyżej osi OX.\\
Ponadto w punkcie $-2$ wykres odbija się od osi poziomej.\\
A więc $$x \in \{-2\} \cup [3,4].$$
\rozwStop
\odpStart
$x \in \{-2\} \cup [3,4]$
\odpStop
\testStart
A.$x \in \{-2\} \cup [3,4]$\\
B.$x \in \{2\} \cup (3,4)$\\
C.$x \in \{-2\} \cup (3,4]$\\
D.$x \in \{2\} \cup (3,4]$\\
E.$x \in \{-2\} \cup [3,4)$\\
F.$x \in \{2\} \cup [3,4)$\\
G.$x \in \{-2\} \cup (3,4)$\\
H.$x \in \{2\} \cup [3,4]$
\testStop
\kluczStart
A
\kluczStop



\zadStart{Zadanie z Wikieł Z 1.62 c) moja wersja nr 27}

Rozwiązać nierówności $(3-x)(x+2)^{2}(5-x)^{3}\le0$.
\zadStop
\rozwStart{Patryk Wirkus}{Laura Mieczkowska}
Miejsca zerowe naszego wielomianu to: $3, -2, 5$.\\
Wielomian jest stopnia parzystego, ponadto znak współczynnika przy\linebreak najwyższej potędze x jest ujemny.\\ W związku z tym wykres wielomianu zaczyna się od lewej strony powyżej osi OX.\\
Ponadto w punkcie $-2$ wykres odbija się od osi poziomej.\\
A więc $$x \in \{-2\} \cup [3,5].$$
\rozwStop
\odpStart
$x \in \{-2\} \cup [3,5]$
\odpStop
\testStart
A.$x \in \{-2\} \cup [3,5]$\\
B.$x \in \{2\} \cup (3,5)$\\
C.$x \in \{-2\} \cup (3,5]$\\
D.$x \in \{2\} \cup (3,5]$\\
E.$x \in \{-2\} \cup [3,5)$\\
F.$x \in \{2\} \cup [3,5)$\\
G.$x \in \{-2\} \cup (3,5)$\\
H.$x \in \{2\} \cup [3,5]$
\testStop
\kluczStart
A
\kluczStop



\zadStart{Zadanie z Wikieł Z 1.62 c) moja wersja nr 28}

Rozwiązać nierówności $(3-x)(x+2)^{2}(6-x)^{3}\le0$.
\zadStop
\rozwStart{Patryk Wirkus}{Laura Mieczkowska}
Miejsca zerowe naszego wielomianu to: $3, -2, 6$.\\
Wielomian jest stopnia parzystego, ponadto znak współczynnika przy\linebreak najwyższej potędze x jest ujemny.\\ W związku z tym wykres wielomianu zaczyna się od lewej strony powyżej osi OX.\\
Ponadto w punkcie $-2$ wykres odbija się od osi poziomej.\\
A więc $$x \in \{-2\} \cup [3,6].$$
\rozwStop
\odpStart
$x \in \{-2\} \cup [3,6]$
\odpStop
\testStart
A.$x \in \{-2\} \cup [3,6]$\\
B.$x \in \{2\} \cup (3,6)$\\
C.$x \in \{-2\} \cup (3,6]$\\
D.$x \in \{2\} \cup (3,6]$\\
E.$x \in \{-2\} \cup [3,6)$\\
F.$x \in \{2\} \cup [3,6)$\\
G.$x \in \{-2\} \cup (3,6)$\\
H.$x \in \{2\} \cup [3,6]$
\testStop
\kluczStart
A
\kluczStop



\zadStart{Zadanie z Wikieł Z 1.62 c) moja wersja nr 29}

Rozwiązać nierówności $(3-x)(x+2)^{2}(7-x)^{3}\le0$.
\zadStop
\rozwStart{Patryk Wirkus}{Laura Mieczkowska}
Miejsca zerowe naszego wielomianu to: $3, -2, 7$.\\
Wielomian jest stopnia parzystego, ponadto znak współczynnika przy\linebreak najwyższej potędze x jest ujemny.\\ W związku z tym wykres wielomianu zaczyna się od lewej strony powyżej osi OX.\\
Ponadto w punkcie $-2$ wykres odbija się od osi poziomej.\\
A więc $$x \in \{-2\} \cup [3,7].$$
\rozwStop
\odpStart
$x \in \{-2\} \cup [3,7]$
\odpStop
\testStart
A.$x \in \{-2\} \cup [3,7]$\\
B.$x \in \{2\} \cup (3,7)$\\
C.$x \in \{-2\} \cup (3,7]$\\
D.$x \in \{2\} \cup (3,7]$\\
E.$x \in \{-2\} \cup [3,7)$\\
F.$x \in \{2\} \cup [3,7)$\\
G.$x \in \{-2\} \cup (3,7)$\\
H.$x \in \{2\} \cup [3,7]$
\testStop
\kluczStart
A
\kluczStop



\zadStart{Zadanie z Wikieł Z 1.62 c) moja wersja nr 30}

Rozwiązać nierówności $(3-x)(x+2)^{2}(8-x)^{3}\le0$.
\zadStop
\rozwStart{Patryk Wirkus}{Laura Mieczkowska}
Miejsca zerowe naszego wielomianu to: $3, -2, 8$.\\
Wielomian jest stopnia parzystego, ponadto znak współczynnika przy\linebreak najwyższej potędze x jest ujemny.\\ W związku z tym wykres wielomianu zaczyna się od lewej strony powyżej osi OX.\\
Ponadto w punkcie $-2$ wykres odbija się od osi poziomej.\\
A więc $$x \in \{-2\} \cup [3,8].$$
\rozwStop
\odpStart
$x \in \{-2\} \cup [3,8]$
\odpStop
\testStart
A.$x \in \{-2\} \cup [3,8]$\\
B.$x \in \{2\} \cup (3,8)$\\
C.$x \in \{-2\} \cup (3,8]$\\
D.$x \in \{2\} \cup (3,8]$\\
E.$x \in \{-2\} \cup [3,8)$\\
F.$x \in \{2\} \cup [3,8)$\\
G.$x \in \{-2\} \cup (3,8)$\\
H.$x \in \{2\} \cup [3,8]$
\testStop
\kluczStart
A
\kluczStop



\zadStart{Zadanie z Wikieł Z 1.62 c) moja wersja nr 31}

Rozwiązać nierówności $(3-x)(x+2)^{2}(9-x)^{3}\le0$.
\zadStop
\rozwStart{Patryk Wirkus}{Laura Mieczkowska}
Miejsca zerowe naszego wielomianu to: $3, -2, 9$.\\
Wielomian jest stopnia parzystego, ponadto znak współczynnika przy\linebreak najwyższej potędze x jest ujemny.\\ W związku z tym wykres wielomianu zaczyna się od lewej strony powyżej osi OX.\\
Ponadto w punkcie $-2$ wykres odbija się od osi poziomej.\\
A więc $$x \in \{-2\} \cup [3,9].$$
\rozwStop
\odpStart
$x \in \{-2\} \cup [3,9]$
\odpStop
\testStart
A.$x \in \{-2\} \cup [3,9]$\\
B.$x \in \{2\} \cup (3,9)$\\
C.$x \in \{-2\} \cup (3,9]$\\
D.$x \in \{2\} \cup (3,9]$\\
E.$x \in \{-2\} \cup [3,9)$\\
F.$x \in \{2\} \cup [3,9)$\\
G.$x \in \{-2\} \cup (3,9)$\\
H.$x \in \{2\} \cup [3,9]$
\testStop
\kluczStart
A
\kluczStop



\zadStart{Zadanie z Wikieł Z 1.62 c) moja wersja nr 32}

Rozwiązać nierówności $(3-x)(x+2)^{2}(10-x)^{3}\le0$.
\zadStop
\rozwStart{Patryk Wirkus}{Laura Mieczkowska}
Miejsca zerowe naszego wielomianu to: $3, -2, 10$.\\
Wielomian jest stopnia parzystego, ponadto znak współczynnika przy\linebreak najwyższej potędze x jest ujemny.\\ W związku z tym wykres wielomianu zaczyna się od lewej strony powyżej osi OX.\\
Ponadto w punkcie $-2$ wykres odbija się od osi poziomej.\\
A więc $$x \in \{-2\} \cup [3,10].$$
\rozwStop
\odpStart
$x \in \{-2\} \cup [3,10]$
\odpStop
\testStart
A.$x \in \{-2\} \cup [3,10]$\\
B.$x \in \{2\} \cup (3,10)$\\
C.$x \in \{-2\} \cup (3,10]$\\
D.$x \in \{2\} \cup (3,10]$\\
E.$x \in \{-2\} \cup [3,10)$\\
F.$x \in \{2\} \cup [3,10)$\\
G.$x \in \{-2\} \cup (3,10)$\\
H.$x \in \{2\} \cup [3,10]$
\testStop
\kluczStart
A
\kluczStop



\zadStart{Zadanie z Wikieł Z 1.62 c) moja wersja nr 33}

Rozwiązać nierówności $(3-x)(x+2)^{2}(11-x)^{3}\le0$.
\zadStop
\rozwStart{Patryk Wirkus}{Laura Mieczkowska}
Miejsca zerowe naszego wielomianu to: $3, -2, 11$.\\
Wielomian jest stopnia parzystego, ponadto znak współczynnika przy\linebreak najwyższej potędze x jest ujemny.\\ W związku z tym wykres wielomianu zaczyna się od lewej strony powyżej osi OX.\\
Ponadto w punkcie $-2$ wykres odbija się od osi poziomej.\\
A więc $$x \in \{-2\} \cup [3,11].$$
\rozwStop
\odpStart
$x \in \{-2\} \cup [3,11]$
\odpStop
\testStart
A.$x \in \{-2\} \cup [3,11]$\\
B.$x \in \{2\} \cup (3,11)$\\
C.$x \in \{-2\} \cup (3,11]$\\
D.$x \in \{2\} \cup (3,11]$\\
E.$x \in \{-2\} \cup [3,11)$\\
F.$x \in \{2\} \cup [3,11)$\\
G.$x \in \{-2\} \cup (3,11)$\\
H.$x \in \{2\} \cup [3,11]$
\testStop
\kluczStart
A
\kluczStop



\zadStart{Zadanie z Wikieł Z 1.62 c) moja wersja nr 34}

Rozwiązać nierówności $(3-x)(x+2)^{2}(12-x)^{3}\le0$.
\zadStop
\rozwStart{Patryk Wirkus}{Laura Mieczkowska}
Miejsca zerowe naszego wielomianu to: $3, -2, 12$.\\
Wielomian jest stopnia parzystego, ponadto znak współczynnika przy\linebreak najwyższej potędze x jest ujemny.\\ W związku z tym wykres wielomianu zaczyna się od lewej strony powyżej osi OX.\\
Ponadto w punkcie $-2$ wykres odbija się od osi poziomej.\\
A więc $$x \in \{-2\} \cup [3,12].$$
\rozwStop
\odpStart
$x \in \{-2\} \cup [3,12]$
\odpStop
\testStart
A.$x \in \{-2\} \cup [3,12]$\\
B.$x \in \{2\} \cup (3,12)$\\
C.$x \in \{-2\} \cup (3,12]$\\
D.$x \in \{2\} \cup (3,12]$\\
E.$x \in \{-2\} \cup [3,12)$\\
F.$x \in \{2\} \cup [3,12)$\\
G.$x \in \{-2\} \cup (3,12)$\\
H.$x \in \{2\} \cup [3,12]$
\testStop
\kluczStart
A
\kluczStop



\zadStart{Zadanie z Wikieł Z 1.62 c) moja wersja nr 35}

Rozwiązać nierówności $(3-x)(x+2)^{2}(13-x)^{3}\le0$.
\zadStop
\rozwStart{Patryk Wirkus}{Laura Mieczkowska}
Miejsca zerowe naszego wielomianu to: $3, -2, 13$.\\
Wielomian jest stopnia parzystego, ponadto znak współczynnika przy\linebreak najwyższej potędze x jest ujemny.\\ W związku z tym wykres wielomianu zaczyna się od lewej strony powyżej osi OX.\\
Ponadto w punkcie $-2$ wykres odbija się od osi poziomej.\\
A więc $$x \in \{-2\} \cup [3,13].$$
\rozwStop
\odpStart
$x \in \{-2\} \cup [3,13]$
\odpStop
\testStart
A.$x \in \{-2\} \cup [3,13]$\\
B.$x \in \{2\} \cup (3,13)$\\
C.$x \in \{-2\} \cup (3,13]$\\
D.$x \in \{2\} \cup (3,13]$\\
E.$x \in \{-2\} \cup [3,13)$\\
F.$x \in \{2\} \cup [3,13)$\\
G.$x \in \{-2\} \cup (3,13)$\\
H.$x \in \{2\} \cup [3,13]$
\testStop
\kluczStart
A
\kluczStop



\zadStart{Zadanie z Wikieł Z 1.62 c) moja wersja nr 36}

Rozwiązać nierówności $(3-x)(x+2)^{2}(14-x)^{3}\le0$.
\zadStop
\rozwStart{Patryk Wirkus}{Laura Mieczkowska}
Miejsca zerowe naszego wielomianu to: $3, -2, 14$.\\
Wielomian jest stopnia parzystego, ponadto znak współczynnika przy\linebreak najwyższej potędze x jest ujemny.\\ W związku z tym wykres wielomianu zaczyna się od lewej strony powyżej osi OX.\\
Ponadto w punkcie $-2$ wykres odbija się od osi poziomej.\\
A więc $$x \in \{-2\} \cup [3,14].$$
\rozwStop
\odpStart
$x \in \{-2\} \cup [3,14]$
\odpStop
\testStart
A.$x \in \{-2\} \cup [3,14]$\\
B.$x \in \{2\} \cup (3,14)$\\
C.$x \in \{-2\} \cup (3,14]$\\
D.$x \in \{2\} \cup (3,14]$\\
E.$x \in \{-2\} \cup [3,14)$\\
F.$x \in \{2\} \cup [3,14)$\\
G.$x \in \{-2\} \cup (3,14)$\\
H.$x \in \{2\} \cup [3,14]$
\testStop
\kluczStart
A
\kluczStop



\zadStart{Zadanie z Wikieł Z 1.62 c) moja wersja nr 37}

Rozwiązać nierówności $(3-x)(x+2)^{2}(15-x)^{3}\le0$.
\zadStop
\rozwStart{Patryk Wirkus}{Laura Mieczkowska}
Miejsca zerowe naszego wielomianu to: $3, -2, 15$.\\
Wielomian jest stopnia parzystego, ponadto znak współczynnika przy\linebreak najwyższej potędze x jest ujemny.\\ W związku z tym wykres wielomianu zaczyna się od lewej strony powyżej osi OX.\\
Ponadto w punkcie $-2$ wykres odbija się od osi poziomej.\\
A więc $$x \in \{-2\} \cup [3,15].$$
\rozwStop
\odpStart
$x \in \{-2\} \cup [3,15]$
\odpStop
\testStart
A.$x \in \{-2\} \cup [3,15]$\\
B.$x \in \{2\} \cup (3,15)$\\
C.$x \in \{-2\} \cup (3,15]$\\
D.$x \in \{2\} \cup (3,15]$\\
E.$x \in \{-2\} \cup [3,15)$\\
F.$x \in \{2\} \cup [3,15)$\\
G.$x \in \{-2\} \cup (3,15)$\\
H.$x \in \{2\} \cup [3,15]$
\testStop
\kluczStart
A
\kluczStop



\zadStart{Zadanie z Wikieł Z 1.62 c) moja wersja nr 38}

Rozwiązać nierówności $(4-x)(x+1)^{2}(5-x)^{3}\le0$.
\zadStop
\rozwStart{Patryk Wirkus}{Laura Mieczkowska}
Miejsca zerowe naszego wielomianu to: $4, -1, 5$.\\
Wielomian jest stopnia parzystego, ponadto znak współczynnika przy\linebreak najwyższej potędze x jest ujemny.\\ W związku z tym wykres wielomianu zaczyna się od lewej strony powyżej osi OX.\\
Ponadto w punkcie $-1$ wykres odbija się od osi poziomej.\\
A więc $$x \in \{-1\} \cup [4,5].$$
\rozwStop
\odpStart
$x \in \{-1\} \cup [4,5]$
\odpStop
\testStart
A.$x \in \{-1\} \cup [4,5]$\\
B.$x \in \{1\} \cup (4,5)$\\
C.$x \in \{-1\} \cup (4,5]$\\
D.$x \in \{1\} \cup (4,5]$\\
E.$x \in \{-1\} \cup [4,5)$\\
F.$x \in \{1\} \cup [4,5)$\\
G.$x \in \{-1\} \cup (4,5)$\\
H.$x \in \{1\} \cup [4,5]$
\testStop
\kluczStart
A
\kluczStop



\zadStart{Zadanie z Wikieł Z 1.62 c) moja wersja nr 39}

Rozwiązać nierówności $(4-x)(x+1)^{2}(6-x)^{3}\le0$.
\zadStop
\rozwStart{Patryk Wirkus}{Laura Mieczkowska}
Miejsca zerowe naszego wielomianu to: $4, -1, 6$.\\
Wielomian jest stopnia parzystego, ponadto znak współczynnika przy\linebreak najwyższej potędze x jest ujemny.\\ W związku z tym wykres wielomianu zaczyna się od lewej strony powyżej osi OX.\\
Ponadto w punkcie $-1$ wykres odbija się od osi poziomej.\\
A więc $$x \in \{-1\} \cup [4,6].$$
\rozwStop
\odpStart
$x \in \{-1\} \cup [4,6]$
\odpStop
\testStart
A.$x \in \{-1\} \cup [4,6]$\\
B.$x \in \{1\} \cup (4,6)$\\
C.$x \in \{-1\} \cup (4,6]$\\
D.$x \in \{1\} \cup (4,6]$\\
E.$x \in \{-1\} \cup [4,6)$\\
F.$x \in \{1\} \cup [4,6)$\\
G.$x \in \{-1\} \cup (4,6)$\\
H.$x \in \{1\} \cup [4,6]$
\testStop
\kluczStart
A
\kluczStop



\zadStart{Zadanie z Wikieł Z 1.62 c) moja wersja nr 40}

Rozwiązać nierówności $(4-x)(x+1)^{2}(7-x)^{3}\le0$.
\zadStop
\rozwStart{Patryk Wirkus}{Laura Mieczkowska}
Miejsca zerowe naszego wielomianu to: $4, -1, 7$.\\
Wielomian jest stopnia parzystego, ponadto znak współczynnika przy\linebreak najwyższej potędze x jest ujemny.\\ W związku z tym wykres wielomianu zaczyna się od lewej strony powyżej osi OX.\\
Ponadto w punkcie $-1$ wykres odbija się od osi poziomej.\\
A więc $$x \in \{-1\} \cup [4,7].$$
\rozwStop
\odpStart
$x \in \{-1\} \cup [4,7]$
\odpStop
\testStart
A.$x \in \{-1\} \cup [4,7]$\\
B.$x \in \{1\} \cup (4,7)$\\
C.$x \in \{-1\} \cup (4,7]$\\
D.$x \in \{1\} \cup (4,7]$\\
E.$x \in \{-1\} \cup [4,7)$\\
F.$x \in \{1\} \cup [4,7)$\\
G.$x \in \{-1\} \cup (4,7)$\\
H.$x \in \{1\} \cup [4,7]$
\testStop
\kluczStart
A
\kluczStop



\zadStart{Zadanie z Wikieł Z 1.62 c) moja wersja nr 41}

Rozwiązać nierówności $(4-x)(x+1)^{2}(8-x)^{3}\le0$.
\zadStop
\rozwStart{Patryk Wirkus}{Laura Mieczkowska}
Miejsca zerowe naszego wielomianu to: $4, -1, 8$.\\
Wielomian jest stopnia parzystego, ponadto znak współczynnika przy\linebreak najwyższej potędze x jest ujemny.\\ W związku z tym wykres wielomianu zaczyna się od lewej strony powyżej osi OX.\\
Ponadto w punkcie $-1$ wykres odbija się od osi poziomej.\\
A więc $$x \in \{-1\} \cup [4,8].$$
\rozwStop
\odpStart
$x \in \{-1\} \cup [4,8]$
\odpStop
\testStart
A.$x \in \{-1\} \cup [4,8]$\\
B.$x \in \{1\} \cup (4,8)$\\
C.$x \in \{-1\} \cup (4,8]$\\
D.$x \in \{1\} \cup (4,8]$\\
E.$x \in \{-1\} \cup [4,8)$\\
F.$x \in \{1\} \cup [4,8)$\\
G.$x \in \{-1\} \cup (4,8)$\\
H.$x \in \{1\} \cup [4,8]$
\testStop
\kluczStart
A
\kluczStop



\zadStart{Zadanie z Wikieł Z 1.62 c) moja wersja nr 42}

Rozwiązać nierówności $(4-x)(x+1)^{2}(9-x)^{3}\le0$.
\zadStop
\rozwStart{Patryk Wirkus}{Laura Mieczkowska}
Miejsca zerowe naszego wielomianu to: $4, -1, 9$.\\
Wielomian jest stopnia parzystego, ponadto znak współczynnika przy\linebreak najwyższej potędze x jest ujemny.\\ W związku z tym wykres wielomianu zaczyna się od lewej strony powyżej osi OX.\\
Ponadto w punkcie $-1$ wykres odbija się od osi poziomej.\\
A więc $$x \in \{-1\} \cup [4,9].$$
\rozwStop
\odpStart
$x \in \{-1\} \cup [4,9]$
\odpStop
\testStart
A.$x \in \{-1\} \cup [4,9]$\\
B.$x \in \{1\} \cup (4,9)$\\
C.$x \in \{-1\} \cup (4,9]$\\
D.$x \in \{1\} \cup (4,9]$\\
E.$x \in \{-1\} \cup [4,9)$\\
F.$x \in \{1\} \cup [4,9)$\\
G.$x \in \{-1\} \cup (4,9)$\\
H.$x \in \{1\} \cup [4,9]$
\testStop
\kluczStart
A
\kluczStop



\zadStart{Zadanie z Wikieł Z 1.62 c) moja wersja nr 43}

Rozwiązać nierówności $(4-x)(x+1)^{2}(10-x)^{3}\le0$.
\zadStop
\rozwStart{Patryk Wirkus}{Laura Mieczkowska}
Miejsca zerowe naszego wielomianu to: $4, -1, 10$.\\
Wielomian jest stopnia parzystego, ponadto znak współczynnika przy\linebreak najwyższej potędze x jest ujemny.\\ W związku z tym wykres wielomianu zaczyna się od lewej strony powyżej osi OX.\\
Ponadto w punkcie $-1$ wykres odbija się od osi poziomej.\\
A więc $$x \in \{-1\} \cup [4,10].$$
\rozwStop
\odpStart
$x \in \{-1\} \cup [4,10]$
\odpStop
\testStart
A.$x \in \{-1\} \cup [4,10]$\\
B.$x \in \{1\} \cup (4,10)$\\
C.$x \in \{-1\} \cup (4,10]$\\
D.$x \in \{1\} \cup (4,10]$\\
E.$x \in \{-1\} \cup [4,10)$\\
F.$x \in \{1\} \cup [4,10)$\\
G.$x \in \{-1\} \cup (4,10)$\\
H.$x \in \{1\} \cup [4,10]$
\testStop
\kluczStart
A
\kluczStop



\zadStart{Zadanie z Wikieł Z 1.62 c) moja wersja nr 44}

Rozwiązać nierówności $(4-x)(x+1)^{2}(11-x)^{3}\le0$.
\zadStop
\rozwStart{Patryk Wirkus}{Laura Mieczkowska}
Miejsca zerowe naszego wielomianu to: $4, -1, 11$.\\
Wielomian jest stopnia parzystego, ponadto znak współczynnika przy\linebreak najwyższej potędze x jest ujemny.\\ W związku z tym wykres wielomianu zaczyna się od lewej strony powyżej osi OX.\\
Ponadto w punkcie $-1$ wykres odbija się od osi poziomej.\\
A więc $$x \in \{-1\} \cup [4,11].$$
\rozwStop
\odpStart
$x \in \{-1\} \cup [4,11]$
\odpStop
\testStart
A.$x \in \{-1\} \cup [4,11]$\\
B.$x \in \{1\} \cup (4,11)$\\
C.$x \in \{-1\} \cup (4,11]$\\
D.$x \in \{1\} \cup (4,11]$\\
E.$x \in \{-1\} \cup [4,11)$\\
F.$x \in \{1\} \cup [4,11)$\\
G.$x \in \{-1\} \cup (4,11)$\\
H.$x \in \{1\} \cup [4,11]$
\testStop
\kluczStart
A
\kluczStop



\zadStart{Zadanie z Wikieł Z 1.62 c) moja wersja nr 45}

Rozwiązać nierówności $(4-x)(x+1)^{2}(12-x)^{3}\le0$.
\zadStop
\rozwStart{Patryk Wirkus}{Laura Mieczkowska}
Miejsca zerowe naszego wielomianu to: $4, -1, 12$.\\
Wielomian jest stopnia parzystego, ponadto znak współczynnika przy\linebreak najwyższej potędze x jest ujemny.\\ W związku z tym wykres wielomianu zaczyna się od lewej strony powyżej osi OX.\\
Ponadto w punkcie $-1$ wykres odbija się od osi poziomej.\\
A więc $$x \in \{-1\} \cup [4,12].$$
\rozwStop
\odpStart
$x \in \{-1\} \cup [4,12]$
\odpStop
\testStart
A.$x \in \{-1\} \cup [4,12]$\\
B.$x \in \{1\} \cup (4,12)$\\
C.$x \in \{-1\} \cup (4,12]$\\
D.$x \in \{1\} \cup (4,12]$\\
E.$x \in \{-1\} \cup [4,12)$\\
F.$x \in \{1\} \cup [4,12)$\\
G.$x \in \{-1\} \cup (4,12)$\\
H.$x \in \{1\} \cup [4,12]$
\testStop
\kluczStart
A
\kluczStop



\zadStart{Zadanie z Wikieł Z 1.62 c) moja wersja nr 46}

Rozwiązać nierówności $(4-x)(x+1)^{2}(13-x)^{3}\le0$.
\zadStop
\rozwStart{Patryk Wirkus}{Laura Mieczkowska}
Miejsca zerowe naszego wielomianu to: $4, -1, 13$.\\
Wielomian jest stopnia parzystego, ponadto znak współczynnika przy\linebreak najwyższej potędze x jest ujemny.\\ W związku z tym wykres wielomianu zaczyna się od lewej strony powyżej osi OX.\\
Ponadto w punkcie $-1$ wykres odbija się od osi poziomej.\\
A więc $$x \in \{-1\} \cup [4,13].$$
\rozwStop
\odpStart
$x \in \{-1\} \cup [4,13]$
\odpStop
\testStart
A.$x \in \{-1\} \cup [4,13]$\\
B.$x \in \{1\} \cup (4,13)$\\
C.$x \in \{-1\} \cup (4,13]$\\
D.$x \in \{1\} \cup (4,13]$\\
E.$x \in \{-1\} \cup [4,13)$\\
F.$x \in \{1\} \cup [4,13)$\\
G.$x \in \{-1\} \cup (4,13)$\\
H.$x \in \{1\} \cup [4,13]$
\testStop
\kluczStart
A
\kluczStop



\zadStart{Zadanie z Wikieł Z 1.62 c) moja wersja nr 47}

Rozwiązać nierówności $(4-x)(x+1)^{2}(14-x)^{3}\le0$.
\zadStop
\rozwStart{Patryk Wirkus}{Laura Mieczkowska}
Miejsca zerowe naszego wielomianu to: $4, -1, 14$.\\
Wielomian jest stopnia parzystego, ponadto znak współczynnika przy\linebreak najwyższej potędze x jest ujemny.\\ W związku z tym wykres wielomianu zaczyna się od lewej strony powyżej osi OX.\\
Ponadto w punkcie $-1$ wykres odbija się od osi poziomej.\\
A więc $$x \in \{-1\} \cup [4,14].$$
\rozwStop
\odpStart
$x \in \{-1\} \cup [4,14]$
\odpStop
\testStart
A.$x \in \{-1\} \cup [4,14]$\\
B.$x \in \{1\} \cup (4,14)$\\
C.$x \in \{-1\} \cup (4,14]$\\
D.$x \in \{1\} \cup (4,14]$\\
E.$x \in \{-1\} \cup [4,14)$\\
F.$x \in \{1\} \cup [4,14)$\\
G.$x \in \{-1\} \cup (4,14)$\\
H.$x \in \{1\} \cup [4,14]$
\testStop
\kluczStart
A
\kluczStop



\zadStart{Zadanie z Wikieł Z 1.62 c) moja wersja nr 48}

Rozwiązać nierówności $(4-x)(x+1)^{2}(15-x)^{3}\le0$.
\zadStop
\rozwStart{Patryk Wirkus}{Laura Mieczkowska}
Miejsca zerowe naszego wielomianu to: $4, -1, 15$.\\
Wielomian jest stopnia parzystego, ponadto znak współczynnika przy\linebreak najwyższej potędze x jest ujemny.\\ W związku z tym wykres wielomianu zaczyna się od lewej strony powyżej osi OX.\\
Ponadto w punkcie $-1$ wykres odbija się od osi poziomej.\\
A więc $$x \in \{-1\} \cup [4,15].$$
\rozwStop
\odpStart
$x \in \{-1\} \cup [4,15]$
\odpStop
\testStart
A.$x \in \{-1\} \cup [4,15]$\\
B.$x \in \{1\} \cup (4,15)$\\
C.$x \in \{-1\} \cup (4,15]$\\
D.$x \in \{1\} \cup (4,15]$\\
E.$x \in \{-1\} \cup [4,15)$\\
F.$x \in \{1\} \cup [4,15)$\\
G.$x \in \{-1\} \cup (4,15)$\\
H.$x \in \{1\} \cup [4,15]$
\testStop
\kluczStart
A
\kluczStop



\zadStart{Zadanie z Wikieł Z 1.62 c) moja wersja nr 49}

Rozwiązać nierówności $(4-x)(x+2)^{2}(5-x)^{3}\le0$.
\zadStop
\rozwStart{Patryk Wirkus}{Laura Mieczkowska}
Miejsca zerowe naszego wielomianu to: $4, -2, 5$.\\
Wielomian jest stopnia parzystego, ponadto znak współczynnika przy\linebreak najwyższej potędze x jest ujemny.\\ W związku z tym wykres wielomianu zaczyna się od lewej strony powyżej osi OX.\\
Ponadto w punkcie $-2$ wykres odbija się od osi poziomej.\\
A więc $$x \in \{-2\} \cup [4,5].$$
\rozwStop
\odpStart
$x \in \{-2\} \cup [4,5]$
\odpStop
\testStart
A.$x \in \{-2\} \cup [4,5]$\\
B.$x \in \{2\} \cup (4,5)$\\
C.$x \in \{-2\} \cup (4,5]$\\
D.$x \in \{2\} \cup (4,5]$\\
E.$x \in \{-2\} \cup [4,5)$\\
F.$x \in \{2\} \cup [4,5)$\\
G.$x \in \{-2\} \cup (4,5)$\\
H.$x \in \{2\} \cup [4,5]$
\testStop
\kluczStart
A
\kluczStop



\zadStart{Zadanie z Wikieł Z 1.62 c) moja wersja nr 50}

Rozwiązać nierówności $(4-x)(x+2)^{2}(6-x)^{3}\le0$.
\zadStop
\rozwStart{Patryk Wirkus}{Laura Mieczkowska}
Miejsca zerowe naszego wielomianu to: $4, -2, 6$.\\
Wielomian jest stopnia parzystego, ponadto znak współczynnika przy\linebreak najwyższej potędze x jest ujemny.\\ W związku z tym wykres wielomianu zaczyna się od lewej strony powyżej osi OX.\\
Ponadto w punkcie $-2$ wykres odbija się od osi poziomej.\\
A więc $$x \in \{-2\} \cup [4,6].$$
\rozwStop
\odpStart
$x \in \{-2\} \cup [4,6]$
\odpStop
\testStart
A.$x \in \{-2\} \cup [4,6]$\\
B.$x \in \{2\} \cup (4,6)$\\
C.$x \in \{-2\} \cup (4,6]$\\
D.$x \in \{2\} \cup (4,6]$\\
E.$x \in \{-2\} \cup [4,6)$\\
F.$x \in \{2\} \cup [4,6)$\\
G.$x \in \{-2\} \cup (4,6)$\\
H.$x \in \{2\} \cup [4,6]$
\testStop
\kluczStart
A
\kluczStop



\zadStart{Zadanie z Wikieł Z 1.62 c) moja wersja nr 51}

Rozwiązać nierówności $(4-x)(x+2)^{2}(7-x)^{3}\le0$.
\zadStop
\rozwStart{Patryk Wirkus}{Laura Mieczkowska}
Miejsca zerowe naszego wielomianu to: $4, -2, 7$.\\
Wielomian jest stopnia parzystego, ponadto znak współczynnika przy\linebreak najwyższej potędze x jest ujemny.\\ W związku z tym wykres wielomianu zaczyna się od lewej strony powyżej osi OX.\\
Ponadto w punkcie $-2$ wykres odbija się od osi poziomej.\\
A więc $$x \in \{-2\} \cup [4,7].$$
\rozwStop
\odpStart
$x \in \{-2\} \cup [4,7]$
\odpStop
\testStart
A.$x \in \{-2\} \cup [4,7]$\\
B.$x \in \{2\} \cup (4,7)$\\
C.$x \in \{-2\} \cup (4,7]$\\
D.$x \in \{2\} \cup (4,7]$\\
E.$x \in \{-2\} \cup [4,7)$\\
F.$x \in \{2\} \cup [4,7)$\\
G.$x \in \{-2\} \cup (4,7)$\\
H.$x \in \{2\} \cup [4,7]$
\testStop
\kluczStart
A
\kluczStop



\zadStart{Zadanie z Wikieł Z 1.62 c) moja wersja nr 52}

Rozwiązać nierówności $(4-x)(x+2)^{2}(8-x)^{3}\le0$.
\zadStop
\rozwStart{Patryk Wirkus}{Laura Mieczkowska}
Miejsca zerowe naszego wielomianu to: $4, -2, 8$.\\
Wielomian jest stopnia parzystego, ponadto znak współczynnika przy\linebreak najwyższej potędze x jest ujemny.\\ W związku z tym wykres wielomianu zaczyna się od lewej strony powyżej osi OX.\\
Ponadto w punkcie $-2$ wykres odbija się od osi poziomej.\\
A więc $$x \in \{-2\} \cup [4,8].$$
\rozwStop
\odpStart
$x \in \{-2\} \cup [4,8]$
\odpStop
\testStart
A.$x \in \{-2\} \cup [4,8]$\\
B.$x \in \{2\} \cup (4,8)$\\
C.$x \in \{-2\} \cup (4,8]$\\
D.$x \in \{2\} \cup (4,8]$\\
E.$x \in \{-2\} \cup [4,8)$\\
F.$x \in \{2\} \cup [4,8)$\\
G.$x \in \{-2\} \cup (4,8)$\\
H.$x \in \{2\} \cup [4,8]$
\testStop
\kluczStart
A
\kluczStop



\zadStart{Zadanie z Wikieł Z 1.62 c) moja wersja nr 53}

Rozwiązać nierówności $(4-x)(x+2)^{2}(9-x)^{3}\le0$.
\zadStop
\rozwStart{Patryk Wirkus}{Laura Mieczkowska}
Miejsca zerowe naszego wielomianu to: $4, -2, 9$.\\
Wielomian jest stopnia parzystego, ponadto znak współczynnika przy\linebreak najwyższej potędze x jest ujemny.\\ W związku z tym wykres wielomianu zaczyna się od lewej strony powyżej osi OX.\\
Ponadto w punkcie $-2$ wykres odbija się od osi poziomej.\\
A więc $$x \in \{-2\} \cup [4,9].$$
\rozwStop
\odpStart
$x \in \{-2\} \cup [4,9]$
\odpStop
\testStart
A.$x \in \{-2\} \cup [4,9]$\\
B.$x \in \{2\} \cup (4,9)$\\
C.$x \in \{-2\} \cup (4,9]$\\
D.$x \in \{2\} \cup (4,9]$\\
E.$x \in \{-2\} \cup [4,9)$\\
F.$x \in \{2\} \cup [4,9)$\\
G.$x \in \{-2\} \cup (4,9)$\\
H.$x \in \{2\} \cup [4,9]$
\testStop
\kluczStart
A
\kluczStop



\zadStart{Zadanie z Wikieł Z 1.62 c) moja wersja nr 54}

Rozwiązać nierówności $(4-x)(x+2)^{2}(10-x)^{3}\le0$.
\zadStop
\rozwStart{Patryk Wirkus}{Laura Mieczkowska}
Miejsca zerowe naszego wielomianu to: $4, -2, 10$.\\
Wielomian jest stopnia parzystego, ponadto znak współczynnika przy\linebreak najwyższej potędze x jest ujemny.\\ W związku z tym wykres wielomianu zaczyna się od lewej strony powyżej osi OX.\\
Ponadto w punkcie $-2$ wykres odbija się od osi poziomej.\\
A więc $$x \in \{-2\} \cup [4,10].$$
\rozwStop
\odpStart
$x \in \{-2\} \cup [4,10]$
\odpStop
\testStart
A.$x \in \{-2\} \cup [4,10]$\\
B.$x \in \{2\} \cup (4,10)$\\
C.$x \in \{-2\} \cup (4,10]$\\
D.$x \in \{2\} \cup (4,10]$\\
E.$x \in \{-2\} \cup [4,10)$\\
F.$x \in \{2\} \cup [4,10)$\\
G.$x \in \{-2\} \cup (4,10)$\\
H.$x \in \{2\} \cup [4,10]$
\testStop
\kluczStart
A
\kluczStop



\zadStart{Zadanie z Wikieł Z 1.62 c) moja wersja nr 55}

Rozwiązać nierówności $(4-x)(x+2)^{2}(11-x)^{3}\le0$.
\zadStop
\rozwStart{Patryk Wirkus}{Laura Mieczkowska}
Miejsca zerowe naszego wielomianu to: $4, -2, 11$.\\
Wielomian jest stopnia parzystego, ponadto znak współczynnika przy\linebreak najwyższej potędze x jest ujemny.\\ W związku z tym wykres wielomianu zaczyna się od lewej strony powyżej osi OX.\\
Ponadto w punkcie $-2$ wykres odbija się od osi poziomej.\\
A więc $$x \in \{-2\} \cup [4,11].$$
\rozwStop
\odpStart
$x \in \{-2\} \cup [4,11]$
\odpStop
\testStart
A.$x \in \{-2\} \cup [4,11]$\\
B.$x \in \{2\} \cup (4,11)$\\
C.$x \in \{-2\} \cup (4,11]$\\
D.$x \in \{2\} \cup (4,11]$\\
E.$x \in \{-2\} \cup [4,11)$\\
F.$x \in \{2\} \cup [4,11)$\\
G.$x \in \{-2\} \cup (4,11)$\\
H.$x \in \{2\} \cup [4,11]$
\testStop
\kluczStart
A
\kluczStop



\zadStart{Zadanie z Wikieł Z 1.62 c) moja wersja nr 56}

Rozwiązać nierówności $(4-x)(x+2)^{2}(12-x)^{3}\le0$.
\zadStop
\rozwStart{Patryk Wirkus}{Laura Mieczkowska}
Miejsca zerowe naszego wielomianu to: $4, -2, 12$.\\
Wielomian jest stopnia parzystego, ponadto znak współczynnika przy\linebreak najwyższej potędze x jest ujemny.\\ W związku z tym wykres wielomianu zaczyna się od lewej strony powyżej osi OX.\\
Ponadto w punkcie $-2$ wykres odbija się od osi poziomej.\\
A więc $$x \in \{-2\} \cup [4,12].$$
\rozwStop
\odpStart
$x \in \{-2\} \cup [4,12]$
\odpStop
\testStart
A.$x \in \{-2\} \cup [4,12]$\\
B.$x \in \{2\} \cup (4,12)$\\
C.$x \in \{-2\} \cup (4,12]$\\
D.$x \in \{2\} \cup (4,12]$\\
E.$x \in \{-2\} \cup [4,12)$\\
F.$x \in \{2\} \cup [4,12)$\\
G.$x \in \{-2\} \cup (4,12)$\\
H.$x \in \{2\} \cup [4,12]$
\testStop
\kluczStart
A
\kluczStop



\zadStart{Zadanie z Wikieł Z 1.62 c) moja wersja nr 57}

Rozwiązać nierówności $(4-x)(x+2)^{2}(13-x)^{3}\le0$.
\zadStop
\rozwStart{Patryk Wirkus}{Laura Mieczkowska}
Miejsca zerowe naszego wielomianu to: $4, -2, 13$.\\
Wielomian jest stopnia parzystego, ponadto znak współczynnika przy\linebreak najwyższej potędze x jest ujemny.\\ W związku z tym wykres wielomianu zaczyna się od lewej strony powyżej osi OX.\\
Ponadto w punkcie $-2$ wykres odbija się od osi poziomej.\\
A więc $$x \in \{-2\} \cup [4,13].$$
\rozwStop
\odpStart
$x \in \{-2\} \cup [4,13]$
\odpStop
\testStart
A.$x \in \{-2\} \cup [4,13]$\\
B.$x \in \{2\} \cup (4,13)$\\
C.$x \in \{-2\} \cup (4,13]$\\
D.$x \in \{2\} \cup (4,13]$\\
E.$x \in \{-2\} \cup [4,13)$\\
F.$x \in \{2\} \cup [4,13)$\\
G.$x \in \{-2\} \cup (4,13)$\\
H.$x \in \{2\} \cup [4,13]$
\testStop
\kluczStart
A
\kluczStop



\zadStart{Zadanie z Wikieł Z 1.62 c) moja wersja nr 58}

Rozwiązać nierówności $(4-x)(x+2)^{2}(14-x)^{3}\le0$.
\zadStop
\rozwStart{Patryk Wirkus}{Laura Mieczkowska}
Miejsca zerowe naszego wielomianu to: $4, -2, 14$.\\
Wielomian jest stopnia parzystego, ponadto znak współczynnika przy\linebreak najwyższej potędze x jest ujemny.\\ W związku z tym wykres wielomianu zaczyna się od lewej strony powyżej osi OX.\\
Ponadto w punkcie $-2$ wykres odbija się od osi poziomej.\\
A więc $$x \in \{-2\} \cup [4,14].$$
\rozwStop
\odpStart
$x \in \{-2\} \cup [4,14]$
\odpStop
\testStart
A.$x \in \{-2\} \cup [4,14]$\\
B.$x \in \{2\} \cup (4,14)$\\
C.$x \in \{-2\} \cup (4,14]$\\
D.$x \in \{2\} \cup (4,14]$\\
E.$x \in \{-2\} \cup [4,14)$\\
F.$x \in \{2\} \cup [4,14)$\\
G.$x \in \{-2\} \cup (4,14)$\\
H.$x \in \{2\} \cup [4,14]$
\testStop
\kluczStart
A
\kluczStop



\zadStart{Zadanie z Wikieł Z 1.62 c) moja wersja nr 59}

Rozwiązać nierówności $(4-x)(x+2)^{2}(15-x)^{3}\le0$.
\zadStop
\rozwStart{Patryk Wirkus}{Laura Mieczkowska}
Miejsca zerowe naszego wielomianu to: $4, -2, 15$.\\
Wielomian jest stopnia parzystego, ponadto znak współczynnika przy\linebreak najwyższej potędze x jest ujemny.\\ W związku z tym wykres wielomianu zaczyna się od lewej strony powyżej osi OX.\\
Ponadto w punkcie $-2$ wykres odbija się od osi poziomej.\\
A więc $$x \in \{-2\} \cup [4,15].$$
\rozwStop
\odpStart
$x \in \{-2\} \cup [4,15]$
\odpStop
\testStart
A.$x \in \{-2\} \cup [4,15]$\\
B.$x \in \{2\} \cup (4,15)$\\
C.$x \in \{-2\} \cup (4,15]$\\
D.$x \in \{2\} \cup (4,15]$\\
E.$x \in \{-2\} \cup [4,15)$\\
F.$x \in \{2\} \cup [4,15)$\\
G.$x \in \{-2\} \cup (4,15)$\\
H.$x \in \{2\} \cup [4,15]$
\testStop
\kluczStart
A
\kluczStop



\zadStart{Zadanie z Wikieł Z 1.62 c) moja wersja nr 60}

Rozwiązać nierówności $(4-x)(x+3)^{2}(5-x)^{3}\le0$.
\zadStop
\rozwStart{Patryk Wirkus}{Laura Mieczkowska}
Miejsca zerowe naszego wielomianu to: $4, -3, 5$.\\
Wielomian jest stopnia parzystego, ponadto znak współczynnika przy\linebreak najwyższej potędze x jest ujemny.\\ W związku z tym wykres wielomianu zaczyna się od lewej strony powyżej osi OX.\\
Ponadto w punkcie $-3$ wykres odbija się od osi poziomej.\\
A więc $$x \in \{-3\} \cup [4,5].$$
\rozwStop
\odpStart
$x \in \{-3\} \cup [4,5]$
\odpStop
\testStart
A.$x \in \{-3\} \cup [4,5]$\\
B.$x \in \{3\} \cup (4,5)$\\
C.$x \in \{-3\} \cup (4,5]$\\
D.$x \in \{3\} \cup (4,5]$\\
E.$x \in \{-3\} \cup [4,5)$\\
F.$x \in \{3\} \cup [4,5)$\\
G.$x \in \{-3\} \cup (4,5)$\\
H.$x \in \{3\} \cup [4,5]$
\testStop
\kluczStart
A
\kluczStop



\zadStart{Zadanie z Wikieł Z 1.62 c) moja wersja nr 61}

Rozwiązać nierówności $(4-x)(x+3)^{2}(6-x)^{3}\le0$.
\zadStop
\rozwStart{Patryk Wirkus}{Laura Mieczkowska}
Miejsca zerowe naszego wielomianu to: $4, -3, 6$.\\
Wielomian jest stopnia parzystego, ponadto znak współczynnika przy\linebreak najwyższej potędze x jest ujemny.\\ W związku z tym wykres wielomianu zaczyna się od lewej strony powyżej osi OX.\\
Ponadto w punkcie $-3$ wykres odbija się od osi poziomej.\\
A więc $$x \in \{-3\} \cup [4,6].$$
\rozwStop
\odpStart
$x \in \{-3\} \cup [4,6]$
\odpStop
\testStart
A.$x \in \{-3\} \cup [4,6]$\\
B.$x \in \{3\} \cup (4,6)$\\
C.$x \in \{-3\} \cup (4,6]$\\
D.$x \in \{3\} \cup (4,6]$\\
E.$x \in \{-3\} \cup [4,6)$\\
F.$x \in \{3\} \cup [4,6)$\\
G.$x \in \{-3\} \cup (4,6)$\\
H.$x \in \{3\} \cup [4,6]$
\testStop
\kluczStart
A
\kluczStop



\zadStart{Zadanie z Wikieł Z 1.62 c) moja wersja nr 62}

Rozwiązać nierówności $(4-x)(x+3)^{2}(7-x)^{3}\le0$.
\zadStop
\rozwStart{Patryk Wirkus}{Laura Mieczkowska}
Miejsca zerowe naszego wielomianu to: $4, -3, 7$.\\
Wielomian jest stopnia parzystego, ponadto znak współczynnika przy\linebreak najwyższej potędze x jest ujemny.\\ W związku z tym wykres wielomianu zaczyna się od lewej strony powyżej osi OX.\\
Ponadto w punkcie $-3$ wykres odbija się od osi poziomej.\\
A więc $$x \in \{-3\} \cup [4,7].$$
\rozwStop
\odpStart
$x \in \{-3\} \cup [4,7]$
\odpStop
\testStart
A.$x \in \{-3\} \cup [4,7]$\\
B.$x \in \{3\} \cup (4,7)$\\
C.$x \in \{-3\} \cup (4,7]$\\
D.$x \in \{3\} \cup (4,7]$\\
E.$x \in \{-3\} \cup [4,7)$\\
F.$x \in \{3\} \cup [4,7)$\\
G.$x \in \{-3\} \cup (4,7)$\\
H.$x \in \{3\} \cup [4,7]$
\testStop
\kluczStart
A
\kluczStop



\zadStart{Zadanie z Wikieł Z 1.62 c) moja wersja nr 63}

Rozwiązać nierówności $(4-x)(x+3)^{2}(8-x)^{3}\le0$.
\zadStop
\rozwStart{Patryk Wirkus}{Laura Mieczkowska}
Miejsca zerowe naszego wielomianu to: $4, -3, 8$.\\
Wielomian jest stopnia parzystego, ponadto znak współczynnika przy\linebreak najwyższej potędze x jest ujemny.\\ W związku z tym wykres wielomianu zaczyna się od lewej strony powyżej osi OX.\\
Ponadto w punkcie $-3$ wykres odbija się od osi poziomej.\\
A więc $$x \in \{-3\} \cup [4,8].$$
\rozwStop
\odpStart
$x \in \{-3\} \cup [4,8]$
\odpStop
\testStart
A.$x \in \{-3\} \cup [4,8]$\\
B.$x \in \{3\} \cup (4,8)$\\
C.$x \in \{-3\} \cup (4,8]$\\
D.$x \in \{3\} \cup (4,8]$\\
E.$x \in \{-3\} \cup [4,8)$\\
F.$x \in \{3\} \cup [4,8)$\\
G.$x \in \{-3\} \cup (4,8)$\\
H.$x \in \{3\} \cup [4,8]$
\testStop
\kluczStart
A
\kluczStop



\zadStart{Zadanie z Wikieł Z 1.62 c) moja wersja nr 64}

Rozwiązać nierówności $(4-x)(x+3)^{2}(9-x)^{3}\le0$.
\zadStop
\rozwStart{Patryk Wirkus}{Laura Mieczkowska}
Miejsca zerowe naszego wielomianu to: $4, -3, 9$.\\
Wielomian jest stopnia parzystego, ponadto znak współczynnika przy\linebreak najwyższej potędze x jest ujemny.\\ W związku z tym wykres wielomianu zaczyna się od lewej strony powyżej osi OX.\\
Ponadto w punkcie $-3$ wykres odbija się od osi poziomej.\\
A więc $$x \in \{-3\} \cup [4,9].$$
\rozwStop
\odpStart
$x \in \{-3\} \cup [4,9]$
\odpStop
\testStart
A.$x \in \{-3\} \cup [4,9]$\\
B.$x \in \{3\} \cup (4,9)$\\
C.$x \in \{-3\} \cup (4,9]$\\
D.$x \in \{3\} \cup (4,9]$\\
E.$x \in \{-3\} \cup [4,9)$\\
F.$x \in \{3\} \cup [4,9)$\\
G.$x \in \{-3\} \cup (4,9)$\\
H.$x \in \{3\} \cup [4,9]$
\testStop
\kluczStart
A
\kluczStop



\zadStart{Zadanie z Wikieł Z 1.62 c) moja wersja nr 65}

Rozwiązać nierówności $(4-x)(x+3)^{2}(10-x)^{3}\le0$.
\zadStop
\rozwStart{Patryk Wirkus}{Laura Mieczkowska}
Miejsca zerowe naszego wielomianu to: $4, -3, 10$.\\
Wielomian jest stopnia parzystego, ponadto znak współczynnika przy\linebreak najwyższej potędze x jest ujemny.\\ W związku z tym wykres wielomianu zaczyna się od lewej strony powyżej osi OX.\\
Ponadto w punkcie $-3$ wykres odbija się od osi poziomej.\\
A więc $$x \in \{-3\} \cup [4,10].$$
\rozwStop
\odpStart
$x \in \{-3\} \cup [4,10]$
\odpStop
\testStart
A.$x \in \{-3\} \cup [4,10]$\\
B.$x \in \{3\} \cup (4,10)$\\
C.$x \in \{-3\} \cup (4,10]$\\
D.$x \in \{3\} \cup (4,10]$\\
E.$x \in \{-3\} \cup [4,10)$\\
F.$x \in \{3\} \cup [4,10)$\\
G.$x \in \{-3\} \cup (4,10)$\\
H.$x \in \{3\} \cup [4,10]$
\testStop
\kluczStart
A
\kluczStop



\zadStart{Zadanie z Wikieł Z 1.62 c) moja wersja nr 66}

Rozwiązać nierówności $(4-x)(x+3)^{2}(11-x)^{3}\le0$.
\zadStop
\rozwStart{Patryk Wirkus}{Laura Mieczkowska}
Miejsca zerowe naszego wielomianu to: $4, -3, 11$.\\
Wielomian jest stopnia parzystego, ponadto znak współczynnika przy\linebreak najwyższej potędze x jest ujemny.\\ W związku z tym wykres wielomianu zaczyna się od lewej strony powyżej osi OX.\\
Ponadto w punkcie $-3$ wykres odbija się od osi poziomej.\\
A więc $$x \in \{-3\} \cup [4,11].$$
\rozwStop
\odpStart
$x \in \{-3\} \cup [4,11]$
\odpStop
\testStart
A.$x \in \{-3\} \cup [4,11]$\\
B.$x \in \{3\} \cup (4,11)$\\
C.$x \in \{-3\} \cup (4,11]$\\
D.$x \in \{3\} \cup (4,11]$\\
E.$x \in \{-3\} \cup [4,11)$\\
F.$x \in \{3\} \cup [4,11)$\\
G.$x \in \{-3\} \cup (4,11)$\\
H.$x \in \{3\} \cup [4,11]$
\testStop
\kluczStart
A
\kluczStop



\zadStart{Zadanie z Wikieł Z 1.62 c) moja wersja nr 67}

Rozwiązać nierówności $(4-x)(x+3)^{2}(12-x)^{3}\le0$.
\zadStop
\rozwStart{Patryk Wirkus}{Laura Mieczkowska}
Miejsca zerowe naszego wielomianu to: $4, -3, 12$.\\
Wielomian jest stopnia parzystego, ponadto znak współczynnika przy\linebreak najwyższej potędze x jest ujemny.\\ W związku z tym wykres wielomianu zaczyna się od lewej strony powyżej osi OX.\\
Ponadto w punkcie $-3$ wykres odbija się od osi poziomej.\\
A więc $$x \in \{-3\} \cup [4,12].$$
\rozwStop
\odpStart
$x \in \{-3\} \cup [4,12]$
\odpStop
\testStart
A.$x \in \{-3\} \cup [4,12]$\\
B.$x \in \{3\} \cup (4,12)$\\
C.$x \in \{-3\} \cup (4,12]$\\
D.$x \in \{3\} \cup (4,12]$\\
E.$x \in \{-3\} \cup [4,12)$\\
F.$x \in \{3\} \cup [4,12)$\\
G.$x \in \{-3\} \cup (4,12)$\\
H.$x \in \{3\} \cup [4,12]$
\testStop
\kluczStart
A
\kluczStop



\zadStart{Zadanie z Wikieł Z 1.62 c) moja wersja nr 68}

Rozwiązać nierówności $(4-x)(x+3)^{2}(13-x)^{3}\le0$.
\zadStop
\rozwStart{Patryk Wirkus}{Laura Mieczkowska}
Miejsca zerowe naszego wielomianu to: $4, -3, 13$.\\
Wielomian jest stopnia parzystego, ponadto znak współczynnika przy\linebreak najwyższej potędze x jest ujemny.\\ W związku z tym wykres wielomianu zaczyna się od lewej strony powyżej osi OX.\\
Ponadto w punkcie $-3$ wykres odbija się od osi poziomej.\\
A więc $$x \in \{-3\} \cup [4,13].$$
\rozwStop
\odpStart
$x \in \{-3\} \cup [4,13]$
\odpStop
\testStart
A.$x \in \{-3\} \cup [4,13]$\\
B.$x \in \{3\} \cup (4,13)$\\
C.$x \in \{-3\} \cup (4,13]$\\
D.$x \in \{3\} \cup (4,13]$\\
E.$x \in \{-3\} \cup [4,13)$\\
F.$x \in \{3\} \cup [4,13)$\\
G.$x \in \{-3\} \cup (4,13)$\\
H.$x \in \{3\} \cup [4,13]$
\testStop
\kluczStart
A
\kluczStop



\zadStart{Zadanie z Wikieł Z 1.62 c) moja wersja nr 69}

Rozwiązać nierówności $(4-x)(x+3)^{2}(14-x)^{3}\le0$.
\zadStop
\rozwStart{Patryk Wirkus}{Laura Mieczkowska}
Miejsca zerowe naszego wielomianu to: $4, -3, 14$.\\
Wielomian jest stopnia parzystego, ponadto znak współczynnika przy\linebreak najwyższej potędze x jest ujemny.\\ W związku z tym wykres wielomianu zaczyna się od lewej strony powyżej osi OX.\\
Ponadto w punkcie $-3$ wykres odbija się od osi poziomej.\\
A więc $$x \in \{-3\} \cup [4,14].$$
\rozwStop
\odpStart
$x \in \{-3\} \cup [4,14]$
\odpStop
\testStart
A.$x \in \{-3\} \cup [4,14]$\\
B.$x \in \{3\} \cup (4,14)$\\
C.$x \in \{-3\} \cup (4,14]$\\
D.$x \in \{3\} \cup (4,14]$\\
E.$x \in \{-3\} \cup [4,14)$\\
F.$x \in \{3\} \cup [4,14)$\\
G.$x \in \{-3\} \cup (4,14)$\\
H.$x \in \{3\} \cup [4,14]$
\testStop
\kluczStart
A
\kluczStop



\zadStart{Zadanie z Wikieł Z 1.62 c) moja wersja nr 70}

Rozwiązać nierówności $(4-x)(x+3)^{2}(15-x)^{3}\le0$.
\zadStop
\rozwStart{Patryk Wirkus}{Laura Mieczkowska}
Miejsca zerowe naszego wielomianu to: $4, -3, 15$.\\
Wielomian jest stopnia parzystego, ponadto znak współczynnika przy\linebreak najwyższej potędze x jest ujemny.\\ W związku z tym wykres wielomianu zaczyna się od lewej strony powyżej osi OX.\\
Ponadto w punkcie $-3$ wykres odbija się od osi poziomej.\\
A więc $$x \in \{-3\} \cup [4,15].$$
\rozwStop
\odpStart
$x \in \{-3\} \cup [4,15]$
\odpStop
\testStart
A.$x \in \{-3\} \cup [4,15]$\\
B.$x \in \{3\} \cup (4,15)$\\
C.$x \in \{-3\} \cup (4,15]$\\
D.$x \in \{3\} \cup (4,15]$\\
E.$x \in \{-3\} \cup [4,15)$\\
F.$x \in \{3\} \cup [4,15)$\\
G.$x \in \{-3\} \cup (4,15)$\\
H.$x \in \{3\} \cup [4,15]$
\testStop
\kluczStart
A
\kluczStop



\zadStart{Zadanie z Wikieł Z 1.62 c) moja wersja nr 71}

Rozwiązać nierówności $(5-x)(x+1)^{2}(6-x)^{3}\le0$.
\zadStop
\rozwStart{Patryk Wirkus}{Laura Mieczkowska}
Miejsca zerowe naszego wielomianu to: $5, -1, 6$.\\
Wielomian jest stopnia parzystego, ponadto znak współczynnika przy\linebreak najwyższej potędze x jest ujemny.\\ W związku z tym wykres wielomianu zaczyna się od lewej strony powyżej osi OX.\\
Ponadto w punkcie $-1$ wykres odbija się od osi poziomej.\\
A więc $$x \in \{-1\} \cup [5,6].$$
\rozwStop
\odpStart
$x \in \{-1\} \cup [5,6]$
\odpStop
\testStart
A.$x \in \{-1\} \cup [5,6]$\\
B.$x \in \{1\} \cup (5,6)$\\
C.$x \in \{-1\} \cup (5,6]$\\
D.$x \in \{1\} \cup (5,6]$\\
E.$x \in \{-1\} \cup [5,6)$\\
F.$x \in \{1\} \cup [5,6)$\\
G.$x \in \{-1\} \cup (5,6)$\\
H.$x \in \{1\} \cup [5,6]$
\testStop
\kluczStart
A
\kluczStop



\zadStart{Zadanie z Wikieł Z 1.62 c) moja wersja nr 72}

Rozwiązać nierówności $(5-x)(x+1)^{2}(7-x)^{3}\le0$.
\zadStop
\rozwStart{Patryk Wirkus}{Laura Mieczkowska}
Miejsca zerowe naszego wielomianu to: $5, -1, 7$.\\
Wielomian jest stopnia parzystego, ponadto znak współczynnika przy\linebreak najwyższej potędze x jest ujemny.\\ W związku z tym wykres wielomianu zaczyna się od lewej strony powyżej osi OX.\\
Ponadto w punkcie $-1$ wykres odbija się od osi poziomej.\\
A więc $$x \in \{-1\} \cup [5,7].$$
\rozwStop
\odpStart
$x \in \{-1\} \cup [5,7]$
\odpStop
\testStart
A.$x \in \{-1\} \cup [5,7]$\\
B.$x \in \{1\} \cup (5,7)$\\
C.$x \in \{-1\} \cup (5,7]$\\
D.$x \in \{1\} \cup (5,7]$\\
E.$x \in \{-1\} \cup [5,7)$\\
F.$x \in \{1\} \cup [5,7)$\\
G.$x \in \{-1\} \cup (5,7)$\\
H.$x \in \{1\} \cup [5,7]$
\testStop
\kluczStart
A
\kluczStop



\zadStart{Zadanie z Wikieł Z 1.62 c) moja wersja nr 73}

Rozwiązać nierówności $(5-x)(x+1)^{2}(8-x)^{3}\le0$.
\zadStop
\rozwStart{Patryk Wirkus}{Laura Mieczkowska}
Miejsca zerowe naszego wielomianu to: $5, -1, 8$.\\
Wielomian jest stopnia parzystego, ponadto znak współczynnika przy\linebreak najwyższej potędze x jest ujemny.\\ W związku z tym wykres wielomianu zaczyna się od lewej strony powyżej osi OX.\\
Ponadto w punkcie $-1$ wykres odbija się od osi poziomej.\\
A więc $$x \in \{-1\} \cup [5,8].$$
\rozwStop
\odpStart
$x \in \{-1\} \cup [5,8]$
\odpStop
\testStart
A.$x \in \{-1\} \cup [5,8]$\\
B.$x \in \{1\} \cup (5,8)$\\
C.$x \in \{-1\} \cup (5,8]$\\
D.$x \in \{1\} \cup (5,8]$\\
E.$x \in \{-1\} \cup [5,8)$\\
F.$x \in \{1\} \cup [5,8)$\\
G.$x \in \{-1\} \cup (5,8)$\\
H.$x \in \{1\} \cup [5,8]$
\testStop
\kluczStart
A
\kluczStop



\zadStart{Zadanie z Wikieł Z 1.62 c) moja wersja nr 74}

Rozwiązać nierówności $(5-x)(x+1)^{2}(9-x)^{3}\le0$.
\zadStop
\rozwStart{Patryk Wirkus}{Laura Mieczkowska}
Miejsca zerowe naszego wielomianu to: $5, -1, 9$.\\
Wielomian jest stopnia parzystego, ponadto znak współczynnika przy\linebreak najwyższej potędze x jest ujemny.\\ W związku z tym wykres wielomianu zaczyna się od lewej strony powyżej osi OX.\\
Ponadto w punkcie $-1$ wykres odbija się od osi poziomej.\\
A więc $$x \in \{-1\} \cup [5,9].$$
\rozwStop
\odpStart
$x \in \{-1\} \cup [5,9]$
\odpStop
\testStart
A.$x \in \{-1\} \cup [5,9]$\\
B.$x \in \{1\} \cup (5,9)$\\
C.$x \in \{-1\} \cup (5,9]$\\
D.$x \in \{1\} \cup (5,9]$\\
E.$x \in \{-1\} \cup [5,9)$\\
F.$x \in \{1\} \cup [5,9)$\\
G.$x \in \{-1\} \cup (5,9)$\\
H.$x \in \{1\} \cup [5,9]$
\testStop
\kluczStart
A
\kluczStop



\zadStart{Zadanie z Wikieł Z 1.62 c) moja wersja nr 75}

Rozwiązać nierówności $(5-x)(x+1)^{2}(10-x)^{3}\le0$.
\zadStop
\rozwStart{Patryk Wirkus}{Laura Mieczkowska}
Miejsca zerowe naszego wielomianu to: $5, -1, 10$.\\
Wielomian jest stopnia parzystego, ponadto znak współczynnika przy\linebreak najwyższej potędze x jest ujemny.\\ W związku z tym wykres wielomianu zaczyna się od lewej strony powyżej osi OX.\\
Ponadto w punkcie $-1$ wykres odbija się od osi poziomej.\\
A więc $$x \in \{-1\} \cup [5,10].$$
\rozwStop
\odpStart
$x \in \{-1\} \cup [5,10]$
\odpStop
\testStart
A.$x \in \{-1\} \cup [5,10]$\\
B.$x \in \{1\} \cup (5,10)$\\
C.$x \in \{-1\} \cup (5,10]$\\
D.$x \in \{1\} \cup (5,10]$\\
E.$x \in \{-1\} \cup [5,10)$\\
F.$x \in \{1\} \cup [5,10)$\\
G.$x \in \{-1\} \cup (5,10)$\\
H.$x \in \{1\} \cup [5,10]$
\testStop
\kluczStart
A
\kluczStop



\zadStart{Zadanie z Wikieł Z 1.62 c) moja wersja nr 76}

Rozwiązać nierówności $(5-x)(x+1)^{2}(11-x)^{3}\le0$.
\zadStop
\rozwStart{Patryk Wirkus}{Laura Mieczkowska}
Miejsca zerowe naszego wielomianu to: $5, -1, 11$.\\
Wielomian jest stopnia parzystego, ponadto znak współczynnika przy\linebreak najwyższej potędze x jest ujemny.\\ W związku z tym wykres wielomianu zaczyna się od lewej strony powyżej osi OX.\\
Ponadto w punkcie $-1$ wykres odbija się od osi poziomej.\\
A więc $$x \in \{-1\} \cup [5,11].$$
\rozwStop
\odpStart
$x \in \{-1\} \cup [5,11]$
\odpStop
\testStart
A.$x \in \{-1\} \cup [5,11]$\\
B.$x \in \{1\} \cup (5,11)$\\
C.$x \in \{-1\} \cup (5,11]$\\
D.$x \in \{1\} \cup (5,11]$\\
E.$x \in \{-1\} \cup [5,11)$\\
F.$x \in \{1\} \cup [5,11)$\\
G.$x \in \{-1\} \cup (5,11)$\\
H.$x \in \{1\} \cup [5,11]$
\testStop
\kluczStart
A
\kluczStop



\zadStart{Zadanie z Wikieł Z 1.62 c) moja wersja nr 77}

Rozwiązać nierówności $(5-x)(x+1)^{2}(12-x)^{3}\le0$.
\zadStop
\rozwStart{Patryk Wirkus}{Laura Mieczkowska}
Miejsca zerowe naszego wielomianu to: $5, -1, 12$.\\
Wielomian jest stopnia parzystego, ponadto znak współczynnika przy\linebreak najwyższej potędze x jest ujemny.\\ W związku z tym wykres wielomianu zaczyna się od lewej strony powyżej osi OX.\\
Ponadto w punkcie $-1$ wykres odbija się od osi poziomej.\\
A więc $$x \in \{-1\} \cup [5,12].$$
\rozwStop
\odpStart
$x \in \{-1\} \cup [5,12]$
\odpStop
\testStart
A.$x \in \{-1\} \cup [5,12]$\\
B.$x \in \{1\} \cup (5,12)$\\
C.$x \in \{-1\} \cup (5,12]$\\
D.$x \in \{1\} \cup (5,12]$\\
E.$x \in \{-1\} \cup [5,12)$\\
F.$x \in \{1\} \cup [5,12)$\\
G.$x \in \{-1\} \cup (5,12)$\\
H.$x \in \{1\} \cup [5,12]$
\testStop
\kluczStart
A
\kluczStop



\zadStart{Zadanie z Wikieł Z 1.62 c) moja wersja nr 78}

Rozwiązać nierówności $(5-x)(x+1)^{2}(13-x)^{3}\le0$.
\zadStop
\rozwStart{Patryk Wirkus}{Laura Mieczkowska}
Miejsca zerowe naszego wielomianu to: $5, -1, 13$.\\
Wielomian jest stopnia parzystego, ponadto znak współczynnika przy\linebreak najwyższej potędze x jest ujemny.\\ W związku z tym wykres wielomianu zaczyna się od lewej strony powyżej osi OX.\\
Ponadto w punkcie $-1$ wykres odbija się od osi poziomej.\\
A więc $$x \in \{-1\} \cup [5,13].$$
\rozwStop
\odpStart
$x \in \{-1\} \cup [5,13]$
\odpStop
\testStart
A.$x \in \{-1\} \cup [5,13]$\\
B.$x \in \{1\} \cup (5,13)$\\
C.$x \in \{-1\} \cup (5,13]$\\
D.$x \in \{1\} \cup (5,13]$\\
E.$x \in \{-1\} \cup [5,13)$\\
F.$x \in \{1\} \cup [5,13)$\\
G.$x \in \{-1\} \cup (5,13)$\\
H.$x \in \{1\} \cup [5,13]$
\testStop
\kluczStart
A
\kluczStop



\zadStart{Zadanie z Wikieł Z 1.62 c) moja wersja nr 79}

Rozwiązać nierówności $(5-x)(x+1)^{2}(14-x)^{3}\le0$.
\zadStop
\rozwStart{Patryk Wirkus}{Laura Mieczkowska}
Miejsca zerowe naszego wielomianu to: $5, -1, 14$.\\
Wielomian jest stopnia parzystego, ponadto znak współczynnika przy\linebreak najwyższej potędze x jest ujemny.\\ W związku z tym wykres wielomianu zaczyna się od lewej strony powyżej osi OX.\\
Ponadto w punkcie $-1$ wykres odbija się od osi poziomej.\\
A więc $$x \in \{-1\} \cup [5,14].$$
\rozwStop
\odpStart
$x \in \{-1\} \cup [5,14]$
\odpStop
\testStart
A.$x \in \{-1\} \cup [5,14]$\\
B.$x \in \{1\} \cup (5,14)$\\
C.$x \in \{-1\} \cup (5,14]$\\
D.$x \in \{1\} \cup (5,14]$\\
E.$x \in \{-1\} \cup [5,14)$\\
F.$x \in \{1\} \cup [5,14)$\\
G.$x \in \{-1\} \cup (5,14)$\\
H.$x \in \{1\} \cup [5,14]$
\testStop
\kluczStart
A
\kluczStop



\zadStart{Zadanie z Wikieł Z 1.62 c) moja wersja nr 80}

Rozwiązać nierówności $(5-x)(x+1)^{2}(15-x)^{3}\le0$.
\zadStop
\rozwStart{Patryk Wirkus}{Laura Mieczkowska}
Miejsca zerowe naszego wielomianu to: $5, -1, 15$.\\
Wielomian jest stopnia parzystego, ponadto znak współczynnika przy\linebreak najwyższej potędze x jest ujemny.\\ W związku z tym wykres wielomianu zaczyna się od lewej strony powyżej osi OX.\\
Ponadto w punkcie $-1$ wykres odbija się od osi poziomej.\\
A więc $$x \in \{-1\} \cup [5,15].$$
\rozwStop
\odpStart
$x \in \{-1\} \cup [5,15]$
\odpStop
\testStart
A.$x \in \{-1\} \cup [5,15]$\\
B.$x \in \{1\} \cup (5,15)$\\
C.$x \in \{-1\} \cup (5,15]$\\
D.$x \in \{1\} \cup (5,15]$\\
E.$x \in \{-1\} \cup [5,15)$\\
F.$x \in \{1\} \cup [5,15)$\\
G.$x \in \{-1\} \cup (5,15)$\\
H.$x \in \{1\} \cup [5,15]$
\testStop
\kluczStart
A
\kluczStop



\zadStart{Zadanie z Wikieł Z 1.62 c) moja wersja nr 81}

Rozwiązać nierówności $(5-x)(x+2)^{2}(6-x)^{3}\le0$.
\zadStop
\rozwStart{Patryk Wirkus}{Laura Mieczkowska}
Miejsca zerowe naszego wielomianu to: $5, -2, 6$.\\
Wielomian jest stopnia parzystego, ponadto znak współczynnika przy\linebreak najwyższej potędze x jest ujemny.\\ W związku z tym wykres wielomianu zaczyna się od lewej strony powyżej osi OX.\\
Ponadto w punkcie $-2$ wykres odbija się od osi poziomej.\\
A więc $$x \in \{-2\} \cup [5,6].$$
\rozwStop
\odpStart
$x \in \{-2\} \cup [5,6]$
\odpStop
\testStart
A.$x \in \{-2\} \cup [5,6]$\\
B.$x \in \{2\} \cup (5,6)$\\
C.$x \in \{-2\} \cup (5,6]$\\
D.$x \in \{2\} \cup (5,6]$\\
E.$x \in \{-2\} \cup [5,6)$\\
F.$x \in \{2\} \cup [5,6)$\\
G.$x \in \{-2\} \cup (5,6)$\\
H.$x \in \{2\} \cup [5,6]$
\testStop
\kluczStart
A
\kluczStop



\zadStart{Zadanie z Wikieł Z 1.62 c) moja wersja nr 82}

Rozwiązać nierówności $(5-x)(x+2)^{2}(7-x)^{3}\le0$.
\zadStop
\rozwStart{Patryk Wirkus}{Laura Mieczkowska}
Miejsca zerowe naszego wielomianu to: $5, -2, 7$.\\
Wielomian jest stopnia parzystego, ponadto znak współczynnika przy\linebreak najwyższej potędze x jest ujemny.\\ W związku z tym wykres wielomianu zaczyna się od lewej strony powyżej osi OX.\\
Ponadto w punkcie $-2$ wykres odbija się od osi poziomej.\\
A więc $$x \in \{-2\} \cup [5,7].$$
\rozwStop
\odpStart
$x \in \{-2\} \cup [5,7]$
\odpStop
\testStart
A.$x \in \{-2\} \cup [5,7]$\\
B.$x \in \{2\} \cup (5,7)$\\
C.$x \in \{-2\} \cup (5,7]$\\
D.$x \in \{2\} \cup (5,7]$\\
E.$x \in \{-2\} \cup [5,7)$\\
F.$x \in \{2\} \cup [5,7)$\\
G.$x \in \{-2\} \cup (5,7)$\\
H.$x \in \{2\} \cup [5,7]$
\testStop
\kluczStart
A
\kluczStop



\zadStart{Zadanie z Wikieł Z 1.62 c) moja wersja nr 83}

Rozwiązać nierówności $(5-x)(x+2)^{2}(8-x)^{3}\le0$.
\zadStop
\rozwStart{Patryk Wirkus}{Laura Mieczkowska}
Miejsca zerowe naszego wielomianu to: $5, -2, 8$.\\
Wielomian jest stopnia parzystego, ponadto znak współczynnika przy\linebreak najwyższej potędze x jest ujemny.\\ W związku z tym wykres wielomianu zaczyna się od lewej strony powyżej osi OX.\\
Ponadto w punkcie $-2$ wykres odbija się od osi poziomej.\\
A więc $$x \in \{-2\} \cup [5,8].$$
\rozwStop
\odpStart
$x \in \{-2\} \cup [5,8]$
\odpStop
\testStart
A.$x \in \{-2\} \cup [5,8]$\\
B.$x \in \{2\} \cup (5,8)$\\
C.$x \in \{-2\} \cup (5,8]$\\
D.$x \in \{2\} \cup (5,8]$\\
E.$x \in \{-2\} \cup [5,8)$\\
F.$x \in \{2\} \cup [5,8)$\\
G.$x \in \{-2\} \cup (5,8)$\\
H.$x \in \{2\} \cup [5,8]$
\testStop
\kluczStart
A
\kluczStop



\zadStart{Zadanie z Wikieł Z 1.62 c) moja wersja nr 84}

Rozwiązać nierówności $(5-x)(x+2)^{2}(9-x)^{3}\le0$.
\zadStop
\rozwStart{Patryk Wirkus}{Laura Mieczkowska}
Miejsca zerowe naszego wielomianu to: $5, -2, 9$.\\
Wielomian jest stopnia parzystego, ponadto znak współczynnika przy\linebreak najwyższej potędze x jest ujemny.\\ W związku z tym wykres wielomianu zaczyna się od lewej strony powyżej osi OX.\\
Ponadto w punkcie $-2$ wykres odbija się od osi poziomej.\\
A więc $$x \in \{-2\} \cup [5,9].$$
\rozwStop
\odpStart
$x \in \{-2\} \cup [5,9]$
\odpStop
\testStart
A.$x \in \{-2\} \cup [5,9]$\\
B.$x \in \{2\} \cup (5,9)$\\
C.$x \in \{-2\} \cup (5,9]$\\
D.$x \in \{2\} \cup (5,9]$\\
E.$x \in \{-2\} \cup [5,9)$\\
F.$x \in \{2\} \cup [5,9)$\\
G.$x \in \{-2\} \cup (5,9)$\\
H.$x \in \{2\} \cup [5,9]$
\testStop
\kluczStart
A
\kluczStop



\zadStart{Zadanie z Wikieł Z 1.62 c) moja wersja nr 85}

Rozwiązać nierówności $(5-x)(x+2)^{2}(10-x)^{3}\le0$.
\zadStop
\rozwStart{Patryk Wirkus}{Laura Mieczkowska}
Miejsca zerowe naszego wielomianu to: $5, -2, 10$.\\
Wielomian jest stopnia parzystego, ponadto znak współczynnika przy\linebreak najwyższej potędze x jest ujemny.\\ W związku z tym wykres wielomianu zaczyna się od lewej strony powyżej osi OX.\\
Ponadto w punkcie $-2$ wykres odbija się od osi poziomej.\\
A więc $$x \in \{-2\} \cup [5,10].$$
\rozwStop
\odpStart
$x \in \{-2\} \cup [5,10]$
\odpStop
\testStart
A.$x \in \{-2\} \cup [5,10]$\\
B.$x \in \{2\} \cup (5,10)$\\
C.$x \in \{-2\} \cup (5,10]$\\
D.$x \in \{2\} \cup (5,10]$\\
E.$x \in \{-2\} \cup [5,10)$\\
F.$x \in \{2\} \cup [5,10)$\\
G.$x \in \{-2\} \cup (5,10)$\\
H.$x \in \{2\} \cup [5,10]$
\testStop
\kluczStart
A
\kluczStop



\zadStart{Zadanie z Wikieł Z 1.62 c) moja wersja nr 86}

Rozwiązać nierówności $(5-x)(x+2)^{2}(11-x)^{3}\le0$.
\zadStop
\rozwStart{Patryk Wirkus}{Laura Mieczkowska}
Miejsca zerowe naszego wielomianu to: $5, -2, 11$.\\
Wielomian jest stopnia parzystego, ponadto znak współczynnika przy\linebreak najwyższej potędze x jest ujemny.\\ W związku z tym wykres wielomianu zaczyna się od lewej strony powyżej osi OX.\\
Ponadto w punkcie $-2$ wykres odbija się od osi poziomej.\\
A więc $$x \in \{-2\} \cup [5,11].$$
\rozwStop
\odpStart
$x \in \{-2\} \cup [5,11]$
\odpStop
\testStart
A.$x \in \{-2\} \cup [5,11]$\\
B.$x \in \{2\} \cup (5,11)$\\
C.$x \in \{-2\} \cup (5,11]$\\
D.$x \in \{2\} \cup (5,11]$\\
E.$x \in \{-2\} \cup [5,11)$\\
F.$x \in \{2\} \cup [5,11)$\\
G.$x \in \{-2\} \cup (5,11)$\\
H.$x \in \{2\} \cup [5,11]$
\testStop
\kluczStart
A
\kluczStop



\zadStart{Zadanie z Wikieł Z 1.62 c) moja wersja nr 87}

Rozwiązać nierówności $(5-x)(x+2)^{2}(12-x)^{3}\le0$.
\zadStop
\rozwStart{Patryk Wirkus}{Laura Mieczkowska}
Miejsca zerowe naszego wielomianu to: $5, -2, 12$.\\
Wielomian jest stopnia parzystego, ponadto znak współczynnika przy\linebreak najwyższej potędze x jest ujemny.\\ W związku z tym wykres wielomianu zaczyna się od lewej strony powyżej osi OX.\\
Ponadto w punkcie $-2$ wykres odbija się od osi poziomej.\\
A więc $$x \in \{-2\} \cup [5,12].$$
\rozwStop
\odpStart
$x \in \{-2\} \cup [5,12]$
\odpStop
\testStart
A.$x \in \{-2\} \cup [5,12]$\\
B.$x \in \{2\} \cup (5,12)$\\
C.$x \in \{-2\} \cup (5,12]$\\
D.$x \in \{2\} \cup (5,12]$\\
E.$x \in \{-2\} \cup [5,12)$\\
F.$x \in \{2\} \cup [5,12)$\\
G.$x \in \{-2\} \cup (5,12)$\\
H.$x \in \{2\} \cup [5,12]$
\testStop
\kluczStart
A
\kluczStop



\zadStart{Zadanie z Wikieł Z 1.62 c) moja wersja nr 88}

Rozwiązać nierówności $(5-x)(x+2)^{2}(13-x)^{3}\le0$.
\zadStop
\rozwStart{Patryk Wirkus}{Laura Mieczkowska}
Miejsca zerowe naszego wielomianu to: $5, -2, 13$.\\
Wielomian jest stopnia parzystego, ponadto znak współczynnika przy\linebreak najwyższej potędze x jest ujemny.\\ W związku z tym wykres wielomianu zaczyna się od lewej strony powyżej osi OX.\\
Ponadto w punkcie $-2$ wykres odbija się od osi poziomej.\\
A więc $$x \in \{-2\} \cup [5,13].$$
\rozwStop
\odpStart
$x \in \{-2\} \cup [5,13]$
\odpStop
\testStart
A.$x \in \{-2\} \cup [5,13]$\\
B.$x \in \{2\} \cup (5,13)$\\
C.$x \in \{-2\} \cup (5,13]$\\
D.$x \in \{2\} \cup (5,13]$\\
E.$x \in \{-2\} \cup [5,13)$\\
F.$x \in \{2\} \cup [5,13)$\\
G.$x \in \{-2\} \cup (5,13)$\\
H.$x \in \{2\} \cup [5,13]$
\testStop
\kluczStart
A
\kluczStop



\zadStart{Zadanie z Wikieł Z 1.62 c) moja wersja nr 89}

Rozwiązać nierówności $(5-x)(x+2)^{2}(14-x)^{3}\le0$.
\zadStop
\rozwStart{Patryk Wirkus}{Laura Mieczkowska}
Miejsca zerowe naszego wielomianu to: $5, -2, 14$.\\
Wielomian jest stopnia parzystego, ponadto znak współczynnika przy\linebreak najwyższej potędze x jest ujemny.\\ W związku z tym wykres wielomianu zaczyna się od lewej strony powyżej osi OX.\\
Ponadto w punkcie $-2$ wykres odbija się od osi poziomej.\\
A więc $$x \in \{-2\} \cup [5,14].$$
\rozwStop
\odpStart
$x \in \{-2\} \cup [5,14]$
\odpStop
\testStart
A.$x \in \{-2\} \cup [5,14]$\\
B.$x \in \{2\} \cup (5,14)$\\
C.$x \in \{-2\} \cup (5,14]$\\
D.$x \in \{2\} \cup (5,14]$\\
E.$x \in \{-2\} \cup [5,14)$\\
F.$x \in \{2\} \cup [5,14)$\\
G.$x \in \{-2\} \cup (5,14)$\\
H.$x \in \{2\} \cup [5,14]$
\testStop
\kluczStart
A
\kluczStop



\zadStart{Zadanie z Wikieł Z 1.62 c) moja wersja nr 90}

Rozwiązać nierówności $(5-x)(x+2)^{2}(15-x)^{3}\le0$.
\zadStop
\rozwStart{Patryk Wirkus}{Laura Mieczkowska}
Miejsca zerowe naszego wielomianu to: $5, -2, 15$.\\
Wielomian jest stopnia parzystego, ponadto znak współczynnika przy\linebreak najwyższej potędze x jest ujemny.\\ W związku z tym wykres wielomianu zaczyna się od lewej strony powyżej osi OX.\\
Ponadto w punkcie $-2$ wykres odbija się od osi poziomej.\\
A więc $$x \in \{-2\} \cup [5,15].$$
\rozwStop
\odpStart
$x \in \{-2\} \cup [5,15]$
\odpStop
\testStart
A.$x \in \{-2\} \cup [5,15]$\\
B.$x \in \{2\} \cup (5,15)$\\
C.$x \in \{-2\} \cup (5,15]$\\
D.$x \in \{2\} \cup (5,15]$\\
E.$x \in \{-2\} \cup [5,15)$\\
F.$x \in \{2\} \cup [5,15)$\\
G.$x \in \{-2\} \cup (5,15)$\\
H.$x \in \{2\} \cup [5,15]$
\testStop
\kluczStart
A
\kluczStop



\zadStart{Zadanie z Wikieł Z 1.62 c) moja wersja nr 91}

Rozwiązać nierówności $(5-x)(x+3)^{2}(6-x)^{3}\le0$.
\zadStop
\rozwStart{Patryk Wirkus}{Laura Mieczkowska}
Miejsca zerowe naszego wielomianu to: $5, -3, 6$.\\
Wielomian jest stopnia parzystego, ponadto znak współczynnika przy\linebreak najwyższej potędze x jest ujemny.\\ W związku z tym wykres wielomianu zaczyna się od lewej strony powyżej osi OX.\\
Ponadto w punkcie $-3$ wykres odbija się od osi poziomej.\\
A więc $$x \in \{-3\} \cup [5,6].$$
\rozwStop
\odpStart
$x \in \{-3\} \cup [5,6]$
\odpStop
\testStart
A.$x \in \{-3\} \cup [5,6]$\\
B.$x \in \{3\} \cup (5,6)$\\
C.$x \in \{-3\} \cup (5,6]$\\
D.$x \in \{3\} \cup (5,6]$\\
E.$x \in \{-3\} \cup [5,6)$\\
F.$x \in \{3\} \cup [5,6)$\\
G.$x \in \{-3\} \cup (5,6)$\\
H.$x \in \{3\} \cup [5,6]$
\testStop
\kluczStart
A
\kluczStop



\zadStart{Zadanie z Wikieł Z 1.62 c) moja wersja nr 92}

Rozwiązać nierówności $(5-x)(x+3)^{2}(7-x)^{3}\le0$.
\zadStop
\rozwStart{Patryk Wirkus}{Laura Mieczkowska}
Miejsca zerowe naszego wielomianu to: $5, -3, 7$.\\
Wielomian jest stopnia parzystego, ponadto znak współczynnika przy\linebreak najwyższej potędze x jest ujemny.\\ W związku z tym wykres wielomianu zaczyna się od lewej strony powyżej osi OX.\\
Ponadto w punkcie $-3$ wykres odbija się od osi poziomej.\\
A więc $$x \in \{-3\} \cup [5,7].$$
\rozwStop
\odpStart
$x \in \{-3\} \cup [5,7]$
\odpStop
\testStart
A.$x \in \{-3\} \cup [5,7]$\\
B.$x \in \{3\} \cup (5,7)$\\
C.$x \in \{-3\} \cup (5,7]$\\
D.$x \in \{3\} \cup (5,7]$\\
E.$x \in \{-3\} \cup [5,7)$\\
F.$x \in \{3\} \cup [5,7)$\\
G.$x \in \{-3\} \cup (5,7)$\\
H.$x \in \{3\} \cup [5,7]$
\testStop
\kluczStart
A
\kluczStop



\zadStart{Zadanie z Wikieł Z 1.62 c) moja wersja nr 93}

Rozwiązać nierówności $(5-x)(x+3)^{2}(8-x)^{3}\le0$.
\zadStop
\rozwStart{Patryk Wirkus}{Laura Mieczkowska}
Miejsca zerowe naszego wielomianu to: $5, -3, 8$.\\
Wielomian jest stopnia parzystego, ponadto znak współczynnika przy\linebreak najwyższej potędze x jest ujemny.\\ W związku z tym wykres wielomianu zaczyna się od lewej strony powyżej osi OX.\\
Ponadto w punkcie $-3$ wykres odbija się od osi poziomej.\\
A więc $$x \in \{-3\} \cup [5,8].$$
\rozwStop
\odpStart
$x \in \{-3\} \cup [5,8]$
\odpStop
\testStart
A.$x \in \{-3\} \cup [5,8]$\\
B.$x \in \{3\} \cup (5,8)$\\
C.$x \in \{-3\} \cup (5,8]$\\
D.$x \in \{3\} \cup (5,8]$\\
E.$x \in \{-3\} \cup [5,8)$\\
F.$x \in \{3\} \cup [5,8)$\\
G.$x \in \{-3\} \cup (5,8)$\\
H.$x \in \{3\} \cup [5,8]$
\testStop
\kluczStart
A
\kluczStop



\zadStart{Zadanie z Wikieł Z 1.62 c) moja wersja nr 94}

Rozwiązać nierówności $(5-x)(x+3)^{2}(9-x)^{3}\le0$.
\zadStop
\rozwStart{Patryk Wirkus}{Laura Mieczkowska}
Miejsca zerowe naszego wielomianu to: $5, -3, 9$.\\
Wielomian jest stopnia parzystego, ponadto znak współczynnika przy\linebreak najwyższej potędze x jest ujemny.\\ W związku z tym wykres wielomianu zaczyna się od lewej strony powyżej osi OX.\\
Ponadto w punkcie $-3$ wykres odbija się od osi poziomej.\\
A więc $$x \in \{-3\} \cup [5,9].$$
\rozwStop
\odpStart
$x \in \{-3\} \cup [5,9]$
\odpStop
\testStart
A.$x \in \{-3\} \cup [5,9]$\\
B.$x \in \{3\} \cup (5,9)$\\
C.$x \in \{-3\} \cup (5,9]$\\
D.$x \in \{3\} \cup (5,9]$\\
E.$x \in \{-3\} \cup [5,9)$\\
F.$x \in \{3\} \cup [5,9)$\\
G.$x \in \{-3\} \cup (5,9)$\\
H.$x \in \{3\} \cup [5,9]$
\testStop
\kluczStart
A
\kluczStop



\zadStart{Zadanie z Wikieł Z 1.62 c) moja wersja nr 95}

Rozwiązać nierówności $(5-x)(x+3)^{2}(10-x)^{3}\le0$.
\zadStop
\rozwStart{Patryk Wirkus}{Laura Mieczkowska}
Miejsca zerowe naszego wielomianu to: $5, -3, 10$.\\
Wielomian jest stopnia parzystego, ponadto znak współczynnika przy\linebreak najwyższej potędze x jest ujemny.\\ W związku z tym wykres wielomianu zaczyna się od lewej strony powyżej osi OX.\\
Ponadto w punkcie $-3$ wykres odbija się od osi poziomej.\\
A więc $$x \in \{-3\} \cup [5,10].$$
\rozwStop
\odpStart
$x \in \{-3\} \cup [5,10]$
\odpStop
\testStart
A.$x \in \{-3\} \cup [5,10]$\\
B.$x \in \{3\} \cup (5,10)$\\
C.$x \in \{-3\} \cup (5,10]$\\
D.$x \in \{3\} \cup (5,10]$\\
E.$x \in \{-3\} \cup [5,10)$\\
F.$x \in \{3\} \cup [5,10)$\\
G.$x \in \{-3\} \cup (5,10)$\\
H.$x \in \{3\} \cup [5,10]$
\testStop
\kluczStart
A
\kluczStop



\zadStart{Zadanie z Wikieł Z 1.62 c) moja wersja nr 96}

Rozwiązać nierówności $(5-x)(x+3)^{2}(11-x)^{3}\le0$.
\zadStop
\rozwStart{Patryk Wirkus}{Laura Mieczkowska}
Miejsca zerowe naszego wielomianu to: $5, -3, 11$.\\
Wielomian jest stopnia parzystego, ponadto znak współczynnika przy\linebreak najwyższej potędze x jest ujemny.\\ W związku z tym wykres wielomianu zaczyna się od lewej strony powyżej osi OX.\\
Ponadto w punkcie $-3$ wykres odbija się od osi poziomej.\\
A więc $$x \in \{-3\} \cup [5,11].$$
\rozwStop
\odpStart
$x \in \{-3\} \cup [5,11]$
\odpStop
\testStart
A.$x \in \{-3\} \cup [5,11]$\\
B.$x \in \{3\} \cup (5,11)$\\
C.$x \in \{-3\} \cup (5,11]$\\
D.$x \in \{3\} \cup (5,11]$\\
E.$x \in \{-3\} \cup [5,11)$\\
F.$x \in \{3\} \cup [5,11)$\\
G.$x \in \{-3\} \cup (5,11)$\\
H.$x \in \{3\} \cup [5,11]$
\testStop
\kluczStart
A
\kluczStop



\zadStart{Zadanie z Wikieł Z 1.62 c) moja wersja nr 97}

Rozwiązać nierówności $(5-x)(x+3)^{2}(12-x)^{3}\le0$.
\zadStop
\rozwStart{Patryk Wirkus}{Laura Mieczkowska}
Miejsca zerowe naszego wielomianu to: $5, -3, 12$.\\
Wielomian jest stopnia parzystego, ponadto znak współczynnika przy\linebreak najwyższej potędze x jest ujemny.\\ W związku z tym wykres wielomianu zaczyna się od lewej strony powyżej osi OX.\\
Ponadto w punkcie $-3$ wykres odbija się od osi poziomej.\\
A więc $$x \in \{-3\} \cup [5,12].$$
\rozwStop
\odpStart
$x \in \{-3\} \cup [5,12]$
\odpStop
\testStart
A.$x \in \{-3\} \cup [5,12]$\\
B.$x \in \{3\} \cup (5,12)$\\
C.$x \in \{-3\} \cup (5,12]$\\
D.$x \in \{3\} \cup (5,12]$\\
E.$x \in \{-3\} \cup [5,12)$\\
F.$x \in \{3\} \cup [5,12)$\\
G.$x \in \{-3\} \cup (5,12)$\\
H.$x \in \{3\} \cup [5,12]$
\testStop
\kluczStart
A
\kluczStop



\zadStart{Zadanie z Wikieł Z 1.62 c) moja wersja nr 98}

Rozwiązać nierówności $(5-x)(x+3)^{2}(13-x)^{3}\le0$.
\zadStop
\rozwStart{Patryk Wirkus}{Laura Mieczkowska}
Miejsca zerowe naszego wielomianu to: $5, -3, 13$.\\
Wielomian jest stopnia parzystego, ponadto znak współczynnika przy\linebreak najwyższej potędze x jest ujemny.\\ W związku z tym wykres wielomianu zaczyna się od lewej strony powyżej osi OX.\\
Ponadto w punkcie $-3$ wykres odbija się od osi poziomej.\\
A więc $$x \in \{-3\} \cup [5,13].$$
\rozwStop
\odpStart
$x \in \{-3\} \cup [5,13]$
\odpStop
\testStart
A.$x \in \{-3\} \cup [5,13]$\\
B.$x \in \{3\} \cup (5,13)$\\
C.$x \in \{-3\} \cup (5,13]$\\
D.$x \in \{3\} \cup (5,13]$\\
E.$x \in \{-3\} \cup [5,13)$\\
F.$x \in \{3\} \cup [5,13)$\\
G.$x \in \{-3\} \cup (5,13)$\\
H.$x \in \{3\} \cup [5,13]$
\testStop
\kluczStart
A
\kluczStop



\zadStart{Zadanie z Wikieł Z 1.62 c) moja wersja nr 99}

Rozwiązać nierówności $(5-x)(x+3)^{2}(14-x)^{3}\le0$.
\zadStop
\rozwStart{Patryk Wirkus}{Laura Mieczkowska}
Miejsca zerowe naszego wielomianu to: $5, -3, 14$.\\
Wielomian jest stopnia parzystego, ponadto znak współczynnika przy\linebreak najwyższej potędze x jest ujemny.\\ W związku z tym wykres wielomianu zaczyna się od lewej strony powyżej osi OX.\\
Ponadto w punkcie $-3$ wykres odbija się od osi poziomej.\\
A więc $$x \in \{-3\} \cup [5,14].$$
\rozwStop
\odpStart
$x \in \{-3\} \cup [5,14]$
\odpStop
\testStart
A.$x \in \{-3\} \cup [5,14]$\\
B.$x \in \{3\} \cup (5,14)$\\
C.$x \in \{-3\} \cup (5,14]$\\
D.$x \in \{3\} \cup (5,14]$\\
E.$x \in \{-3\} \cup [5,14)$\\
F.$x \in \{3\} \cup [5,14)$\\
G.$x \in \{-3\} \cup (5,14)$\\
H.$x \in \{3\} \cup [5,14]$
\testStop
\kluczStart
A
\kluczStop



\zadStart{Zadanie z Wikieł Z 1.62 c) moja wersja nr 100}

Rozwiązać nierówności $(5-x)(x+3)^{2}(15-x)^{3}\le0$.
\zadStop
\rozwStart{Patryk Wirkus}{Laura Mieczkowska}
Miejsca zerowe naszego wielomianu to: $5, -3, 15$.\\
Wielomian jest stopnia parzystego, ponadto znak współczynnika przy\linebreak najwyższej potędze x jest ujemny.\\ W związku z tym wykres wielomianu zaczyna się od lewej strony powyżej osi OX.\\
Ponadto w punkcie $-3$ wykres odbija się od osi poziomej.\\
A więc $$x \in \{-3\} \cup [5,15].$$
\rozwStop
\odpStart
$x \in \{-3\} \cup [5,15]$
\odpStop
\testStart
A.$x \in \{-3\} \cup [5,15]$\\
B.$x \in \{3\} \cup (5,15)$\\
C.$x \in \{-3\} \cup (5,15]$\\
D.$x \in \{3\} \cup (5,15]$\\
E.$x \in \{-3\} \cup [5,15)$\\
F.$x \in \{3\} \cup [5,15)$\\
G.$x \in \{-3\} \cup (5,15)$\\
H.$x \in \{3\} \cup [5,15]$
\testStop
\kluczStart
A
\kluczStop



\zadStart{Zadanie z Wikieł Z 1.62 c) moja wersja nr 101}

Rozwiązać nierówności $(5-x)(x+4)^{2}(6-x)^{3}\le0$.
\zadStop
\rozwStart{Patryk Wirkus}{Laura Mieczkowska}
Miejsca zerowe naszego wielomianu to: $5, -4, 6$.\\
Wielomian jest stopnia parzystego, ponadto znak współczynnika przy\linebreak najwyższej potędze x jest ujemny.\\ W związku z tym wykres wielomianu zaczyna się od lewej strony powyżej osi OX.\\
Ponadto w punkcie $-4$ wykres odbija się od osi poziomej.\\
A więc $$x \in \{-4\} \cup [5,6].$$
\rozwStop
\odpStart
$x \in \{-4\} \cup [5,6]$
\odpStop
\testStart
A.$x \in \{-4\} \cup [5,6]$\\
B.$x \in \{4\} \cup (5,6)$\\
C.$x \in \{-4\} \cup (5,6]$\\
D.$x \in \{4\} \cup (5,6]$\\
E.$x \in \{-4\} \cup [5,6)$\\
F.$x \in \{4\} \cup [5,6)$\\
G.$x \in \{-4\} \cup (5,6)$\\
H.$x \in \{4\} \cup [5,6]$
\testStop
\kluczStart
A
\kluczStop



\zadStart{Zadanie z Wikieł Z 1.62 c) moja wersja nr 102}

Rozwiązać nierówności $(5-x)(x+4)^{2}(7-x)^{3}\le0$.
\zadStop
\rozwStart{Patryk Wirkus}{Laura Mieczkowska}
Miejsca zerowe naszego wielomianu to: $5, -4, 7$.\\
Wielomian jest stopnia parzystego, ponadto znak współczynnika przy\linebreak najwyższej potędze x jest ujemny.\\ W związku z tym wykres wielomianu zaczyna się od lewej strony powyżej osi OX.\\
Ponadto w punkcie $-4$ wykres odbija się od osi poziomej.\\
A więc $$x \in \{-4\} \cup [5,7].$$
\rozwStop
\odpStart
$x \in \{-4\} \cup [5,7]$
\odpStop
\testStart
A.$x \in \{-4\} \cup [5,7]$\\
B.$x \in \{4\} \cup (5,7)$\\
C.$x \in \{-4\} \cup (5,7]$\\
D.$x \in \{4\} \cup (5,7]$\\
E.$x \in \{-4\} \cup [5,7)$\\
F.$x \in \{4\} \cup [5,7)$\\
G.$x \in \{-4\} \cup (5,7)$\\
H.$x \in \{4\} \cup [5,7]$
\testStop
\kluczStart
A
\kluczStop



\zadStart{Zadanie z Wikieł Z 1.62 c) moja wersja nr 103}

Rozwiązać nierówności $(5-x)(x+4)^{2}(8-x)^{3}\le0$.
\zadStop
\rozwStart{Patryk Wirkus}{Laura Mieczkowska}
Miejsca zerowe naszego wielomianu to: $5, -4, 8$.\\
Wielomian jest stopnia parzystego, ponadto znak współczynnika przy\linebreak najwyższej potędze x jest ujemny.\\ W związku z tym wykres wielomianu zaczyna się od lewej strony powyżej osi OX.\\
Ponadto w punkcie $-4$ wykres odbija się od osi poziomej.\\
A więc $$x \in \{-4\} \cup [5,8].$$
\rozwStop
\odpStart
$x \in \{-4\} \cup [5,8]$
\odpStop
\testStart
A.$x \in \{-4\} \cup [5,8]$\\
B.$x \in \{4\} \cup (5,8)$\\
C.$x \in \{-4\} \cup (5,8]$\\
D.$x \in \{4\} \cup (5,8]$\\
E.$x \in \{-4\} \cup [5,8)$\\
F.$x \in \{4\} \cup [5,8)$\\
G.$x \in \{-4\} \cup (5,8)$\\
H.$x \in \{4\} \cup [5,8]$
\testStop
\kluczStart
A
\kluczStop



\zadStart{Zadanie z Wikieł Z 1.62 c) moja wersja nr 104}

Rozwiązać nierówności $(5-x)(x+4)^{2}(9-x)^{3}\le0$.
\zadStop
\rozwStart{Patryk Wirkus}{Laura Mieczkowska}
Miejsca zerowe naszego wielomianu to: $5, -4, 9$.\\
Wielomian jest stopnia parzystego, ponadto znak współczynnika przy\linebreak najwyższej potędze x jest ujemny.\\ W związku z tym wykres wielomianu zaczyna się od lewej strony powyżej osi OX.\\
Ponadto w punkcie $-4$ wykres odbija się od osi poziomej.\\
A więc $$x \in \{-4\} \cup [5,9].$$
\rozwStop
\odpStart
$x \in \{-4\} \cup [5,9]$
\odpStop
\testStart
A.$x \in \{-4\} \cup [5,9]$\\
B.$x \in \{4\} \cup (5,9)$\\
C.$x \in \{-4\} \cup (5,9]$\\
D.$x \in \{4\} \cup (5,9]$\\
E.$x \in \{-4\} \cup [5,9)$\\
F.$x \in \{4\} \cup [5,9)$\\
G.$x \in \{-4\} \cup (5,9)$\\
H.$x \in \{4\} \cup [5,9]$
\testStop
\kluczStart
A
\kluczStop



\zadStart{Zadanie z Wikieł Z 1.62 c) moja wersja nr 105}

Rozwiązać nierówności $(5-x)(x+4)^{2}(10-x)^{3}\le0$.
\zadStop
\rozwStart{Patryk Wirkus}{Laura Mieczkowska}
Miejsca zerowe naszego wielomianu to: $5, -4, 10$.\\
Wielomian jest stopnia parzystego, ponadto znak współczynnika przy\linebreak najwyższej potędze x jest ujemny.\\ W związku z tym wykres wielomianu zaczyna się od lewej strony powyżej osi OX.\\
Ponadto w punkcie $-4$ wykres odbija się od osi poziomej.\\
A więc $$x \in \{-4\} \cup [5,10].$$
\rozwStop
\odpStart
$x \in \{-4\} \cup [5,10]$
\odpStop
\testStart
A.$x \in \{-4\} \cup [5,10]$\\
B.$x \in \{4\} \cup (5,10)$\\
C.$x \in \{-4\} \cup (5,10]$\\
D.$x \in \{4\} \cup (5,10]$\\
E.$x \in \{-4\} \cup [5,10)$\\
F.$x \in \{4\} \cup [5,10)$\\
G.$x \in \{-4\} \cup (5,10)$\\
H.$x \in \{4\} \cup [5,10]$
\testStop
\kluczStart
A
\kluczStop



\zadStart{Zadanie z Wikieł Z 1.62 c) moja wersja nr 106}

Rozwiązać nierówności $(5-x)(x+4)^{2}(11-x)^{3}\le0$.
\zadStop
\rozwStart{Patryk Wirkus}{Laura Mieczkowska}
Miejsca zerowe naszego wielomianu to: $5, -4, 11$.\\
Wielomian jest stopnia parzystego, ponadto znak współczynnika przy\linebreak najwyższej potędze x jest ujemny.\\ W związku z tym wykres wielomianu zaczyna się od lewej strony powyżej osi OX.\\
Ponadto w punkcie $-4$ wykres odbija się od osi poziomej.\\
A więc $$x \in \{-4\} \cup [5,11].$$
\rozwStop
\odpStart
$x \in \{-4\} \cup [5,11]$
\odpStop
\testStart
A.$x \in \{-4\} \cup [5,11]$\\
B.$x \in \{4\} \cup (5,11)$\\
C.$x \in \{-4\} \cup (5,11]$\\
D.$x \in \{4\} \cup (5,11]$\\
E.$x \in \{-4\} \cup [5,11)$\\
F.$x \in \{4\} \cup [5,11)$\\
G.$x \in \{-4\} \cup (5,11)$\\
H.$x \in \{4\} \cup [5,11]$
\testStop
\kluczStart
A
\kluczStop



\zadStart{Zadanie z Wikieł Z 1.62 c) moja wersja nr 107}

Rozwiązać nierówności $(5-x)(x+4)^{2}(12-x)^{3}\le0$.
\zadStop
\rozwStart{Patryk Wirkus}{Laura Mieczkowska}
Miejsca zerowe naszego wielomianu to: $5, -4, 12$.\\
Wielomian jest stopnia parzystego, ponadto znak współczynnika przy\linebreak najwyższej potędze x jest ujemny.\\ W związku z tym wykres wielomianu zaczyna się od lewej strony powyżej osi OX.\\
Ponadto w punkcie $-4$ wykres odbija się od osi poziomej.\\
A więc $$x \in \{-4\} \cup [5,12].$$
\rozwStop
\odpStart
$x \in \{-4\} \cup [5,12]$
\odpStop
\testStart
A.$x \in \{-4\} \cup [5,12]$\\
B.$x \in \{4\} \cup (5,12)$\\
C.$x \in \{-4\} \cup (5,12]$\\
D.$x \in \{4\} \cup (5,12]$\\
E.$x \in \{-4\} \cup [5,12)$\\
F.$x \in \{4\} \cup [5,12)$\\
G.$x \in \{-4\} \cup (5,12)$\\
H.$x \in \{4\} \cup [5,12]$
\testStop
\kluczStart
A
\kluczStop



\zadStart{Zadanie z Wikieł Z 1.62 c) moja wersja nr 108}

Rozwiązać nierówności $(5-x)(x+4)^{2}(13-x)^{3}\le0$.
\zadStop
\rozwStart{Patryk Wirkus}{Laura Mieczkowska}
Miejsca zerowe naszego wielomianu to: $5, -4, 13$.\\
Wielomian jest stopnia parzystego, ponadto znak współczynnika przy\linebreak najwyższej potędze x jest ujemny.\\ W związku z tym wykres wielomianu zaczyna się od lewej strony powyżej osi OX.\\
Ponadto w punkcie $-4$ wykres odbija się od osi poziomej.\\
A więc $$x \in \{-4\} \cup [5,13].$$
\rozwStop
\odpStart
$x \in \{-4\} \cup [5,13]$
\odpStop
\testStart
A.$x \in \{-4\} \cup [5,13]$\\
B.$x \in \{4\} \cup (5,13)$\\
C.$x \in \{-4\} \cup (5,13]$\\
D.$x \in \{4\} \cup (5,13]$\\
E.$x \in \{-4\} \cup [5,13)$\\
F.$x \in \{4\} \cup [5,13)$\\
G.$x \in \{-4\} \cup (5,13)$\\
H.$x \in \{4\} \cup [5,13]$
\testStop
\kluczStart
A
\kluczStop



\zadStart{Zadanie z Wikieł Z 1.62 c) moja wersja nr 109}

Rozwiązać nierówności $(5-x)(x+4)^{2}(14-x)^{3}\le0$.
\zadStop
\rozwStart{Patryk Wirkus}{Laura Mieczkowska}
Miejsca zerowe naszego wielomianu to: $5, -4, 14$.\\
Wielomian jest stopnia parzystego, ponadto znak współczynnika przy\linebreak najwyższej potędze x jest ujemny.\\ W związku z tym wykres wielomianu zaczyna się od lewej strony powyżej osi OX.\\
Ponadto w punkcie $-4$ wykres odbija się od osi poziomej.\\
A więc $$x \in \{-4\} \cup [5,14].$$
\rozwStop
\odpStart
$x \in \{-4\} \cup [5,14]$
\odpStop
\testStart
A.$x \in \{-4\} \cup [5,14]$\\
B.$x \in \{4\} \cup (5,14)$\\
C.$x \in \{-4\} \cup (5,14]$\\
D.$x \in \{4\} \cup (5,14]$\\
E.$x \in \{-4\} \cup [5,14)$\\
F.$x \in \{4\} \cup [5,14)$\\
G.$x \in \{-4\} \cup (5,14)$\\
H.$x \in \{4\} \cup [5,14]$
\testStop
\kluczStart
A
\kluczStop



\zadStart{Zadanie z Wikieł Z 1.62 c) moja wersja nr 110}

Rozwiązać nierówności $(5-x)(x+4)^{2}(15-x)^{3}\le0$.
\zadStop
\rozwStart{Patryk Wirkus}{Laura Mieczkowska}
Miejsca zerowe naszego wielomianu to: $5, -4, 15$.\\
Wielomian jest stopnia parzystego, ponadto znak współczynnika przy\linebreak najwyższej potędze x jest ujemny.\\ W związku z tym wykres wielomianu zaczyna się od lewej strony powyżej osi OX.\\
Ponadto w punkcie $-4$ wykres odbija się od osi poziomej.\\
A więc $$x \in \{-4\} \cup [5,15].$$
\rozwStop
\odpStart
$x \in \{-4\} \cup [5,15]$
\odpStop
\testStart
A.$x \in \{-4\} \cup [5,15]$\\
B.$x \in \{4\} \cup (5,15)$\\
C.$x \in \{-4\} \cup (5,15]$\\
D.$x \in \{4\} \cup (5,15]$\\
E.$x \in \{-4\} \cup [5,15)$\\
F.$x \in \{4\} \cup [5,15)$\\
G.$x \in \{-4\} \cup (5,15)$\\
H.$x \in \{4\} \cup [5,15]$
\testStop
\kluczStart
A
\kluczStop



\zadStart{Zadanie z Wikieł Z 1.62 c) moja wersja nr 111}

Rozwiązać nierówności $(6-x)(x+1)^{2}(7-x)^{3}\le0$.
\zadStop
\rozwStart{Patryk Wirkus}{Laura Mieczkowska}
Miejsca zerowe naszego wielomianu to: $6, -1, 7$.\\
Wielomian jest stopnia parzystego, ponadto znak współczynnika przy\linebreak najwyższej potędze x jest ujemny.\\ W związku z tym wykres wielomianu zaczyna się od lewej strony powyżej osi OX.\\
Ponadto w punkcie $-1$ wykres odbija się od osi poziomej.\\
A więc $$x \in \{-1\} \cup [6,7].$$
\rozwStop
\odpStart
$x \in \{-1\} \cup [6,7]$
\odpStop
\testStart
A.$x \in \{-1\} \cup [6,7]$\\
B.$x \in \{1\} \cup (6,7)$\\
C.$x \in \{-1\} \cup (6,7]$\\
D.$x \in \{1\} \cup (6,7]$\\
E.$x \in \{-1\} \cup [6,7)$\\
F.$x \in \{1\} \cup [6,7)$\\
G.$x \in \{-1\} \cup (6,7)$\\
H.$x \in \{1\} \cup [6,7]$
\testStop
\kluczStart
A
\kluczStop



\zadStart{Zadanie z Wikieł Z 1.62 c) moja wersja nr 112}

Rozwiązać nierówności $(6-x)(x+1)^{2}(8-x)^{3}\le0$.
\zadStop
\rozwStart{Patryk Wirkus}{Laura Mieczkowska}
Miejsca zerowe naszego wielomianu to: $6, -1, 8$.\\
Wielomian jest stopnia parzystego, ponadto znak współczynnika przy\linebreak najwyższej potędze x jest ujemny.\\ W związku z tym wykres wielomianu zaczyna się od lewej strony powyżej osi OX.\\
Ponadto w punkcie $-1$ wykres odbija się od osi poziomej.\\
A więc $$x \in \{-1\} \cup [6,8].$$
\rozwStop
\odpStart
$x \in \{-1\} \cup [6,8]$
\odpStop
\testStart
A.$x \in \{-1\} \cup [6,8]$\\
B.$x \in \{1\} \cup (6,8)$\\
C.$x \in \{-1\} \cup (6,8]$\\
D.$x \in \{1\} \cup (6,8]$\\
E.$x \in \{-1\} \cup [6,8)$\\
F.$x \in \{1\} \cup [6,8)$\\
G.$x \in \{-1\} \cup (6,8)$\\
H.$x \in \{1\} \cup [6,8]$
\testStop
\kluczStart
A
\kluczStop



\zadStart{Zadanie z Wikieł Z 1.62 c) moja wersja nr 113}

Rozwiązać nierówności $(6-x)(x+1)^{2}(9-x)^{3}\le0$.
\zadStop
\rozwStart{Patryk Wirkus}{Laura Mieczkowska}
Miejsca zerowe naszego wielomianu to: $6, -1, 9$.\\
Wielomian jest stopnia parzystego, ponadto znak współczynnika przy\linebreak najwyższej potędze x jest ujemny.\\ W związku z tym wykres wielomianu zaczyna się od lewej strony powyżej osi OX.\\
Ponadto w punkcie $-1$ wykres odbija się od osi poziomej.\\
A więc $$x \in \{-1\} \cup [6,9].$$
\rozwStop
\odpStart
$x \in \{-1\} \cup [6,9]$
\odpStop
\testStart
A.$x \in \{-1\} \cup [6,9]$\\
B.$x \in \{1\} \cup (6,9)$\\
C.$x \in \{-1\} \cup (6,9]$\\
D.$x \in \{1\} \cup (6,9]$\\
E.$x \in \{-1\} \cup [6,9)$\\
F.$x \in \{1\} \cup [6,9)$\\
G.$x \in \{-1\} \cup (6,9)$\\
H.$x \in \{1\} \cup [6,9]$
\testStop
\kluczStart
A
\kluczStop



\zadStart{Zadanie z Wikieł Z 1.62 c) moja wersja nr 114}

Rozwiązać nierówności $(6-x)(x+1)^{2}(10-x)^{3}\le0$.
\zadStop
\rozwStart{Patryk Wirkus}{Laura Mieczkowska}
Miejsca zerowe naszego wielomianu to: $6, -1, 10$.\\
Wielomian jest stopnia parzystego, ponadto znak współczynnika przy\linebreak najwyższej potędze x jest ujemny.\\ W związku z tym wykres wielomianu zaczyna się od lewej strony powyżej osi OX.\\
Ponadto w punkcie $-1$ wykres odbija się od osi poziomej.\\
A więc $$x \in \{-1\} \cup [6,10].$$
\rozwStop
\odpStart
$x \in \{-1\} \cup [6,10]$
\odpStop
\testStart
A.$x \in \{-1\} \cup [6,10]$\\
B.$x \in \{1\} \cup (6,10)$\\
C.$x \in \{-1\} \cup (6,10]$\\
D.$x \in \{1\} \cup (6,10]$\\
E.$x \in \{-1\} \cup [6,10)$\\
F.$x \in \{1\} \cup [6,10)$\\
G.$x \in \{-1\} \cup (6,10)$\\
H.$x \in \{1\} \cup [6,10]$
\testStop
\kluczStart
A
\kluczStop



\zadStart{Zadanie z Wikieł Z 1.62 c) moja wersja nr 115}

Rozwiązać nierówności $(6-x)(x+1)^{2}(11-x)^{3}\le0$.
\zadStop
\rozwStart{Patryk Wirkus}{Laura Mieczkowska}
Miejsca zerowe naszego wielomianu to: $6, -1, 11$.\\
Wielomian jest stopnia parzystego, ponadto znak współczynnika przy\linebreak najwyższej potędze x jest ujemny.\\ W związku z tym wykres wielomianu zaczyna się od lewej strony powyżej osi OX.\\
Ponadto w punkcie $-1$ wykres odbija się od osi poziomej.\\
A więc $$x \in \{-1\} \cup [6,11].$$
\rozwStop
\odpStart
$x \in \{-1\} \cup [6,11]$
\odpStop
\testStart
A.$x \in \{-1\} \cup [6,11]$\\
B.$x \in \{1\} \cup (6,11)$\\
C.$x \in \{-1\} \cup (6,11]$\\
D.$x \in \{1\} \cup (6,11]$\\
E.$x \in \{-1\} \cup [6,11)$\\
F.$x \in \{1\} \cup [6,11)$\\
G.$x \in \{-1\} \cup (6,11)$\\
H.$x \in \{1\} \cup [6,11]$
\testStop
\kluczStart
A
\kluczStop



\zadStart{Zadanie z Wikieł Z 1.62 c) moja wersja nr 116}

Rozwiązać nierówności $(6-x)(x+1)^{2}(12-x)^{3}\le0$.
\zadStop
\rozwStart{Patryk Wirkus}{Laura Mieczkowska}
Miejsca zerowe naszego wielomianu to: $6, -1, 12$.\\
Wielomian jest stopnia parzystego, ponadto znak współczynnika przy\linebreak najwyższej potędze x jest ujemny.\\ W związku z tym wykres wielomianu zaczyna się od lewej strony powyżej osi OX.\\
Ponadto w punkcie $-1$ wykres odbija się od osi poziomej.\\
A więc $$x \in \{-1\} \cup [6,12].$$
\rozwStop
\odpStart
$x \in \{-1\} \cup [6,12]$
\odpStop
\testStart
A.$x \in \{-1\} \cup [6,12]$\\
B.$x \in \{1\} \cup (6,12)$\\
C.$x \in \{-1\} \cup (6,12]$\\
D.$x \in \{1\} \cup (6,12]$\\
E.$x \in \{-1\} \cup [6,12)$\\
F.$x \in \{1\} \cup [6,12)$\\
G.$x \in \{-1\} \cup (6,12)$\\
H.$x \in \{1\} \cup [6,12]$
\testStop
\kluczStart
A
\kluczStop



\zadStart{Zadanie z Wikieł Z 1.62 c) moja wersja nr 117}

Rozwiązać nierówności $(6-x)(x+1)^{2}(13-x)^{3}\le0$.
\zadStop
\rozwStart{Patryk Wirkus}{Laura Mieczkowska}
Miejsca zerowe naszego wielomianu to: $6, -1, 13$.\\
Wielomian jest stopnia parzystego, ponadto znak współczynnika przy\linebreak najwyższej potędze x jest ujemny.\\ W związku z tym wykres wielomianu zaczyna się od lewej strony powyżej osi OX.\\
Ponadto w punkcie $-1$ wykres odbija się od osi poziomej.\\
A więc $$x \in \{-1\} \cup [6,13].$$
\rozwStop
\odpStart
$x \in \{-1\} \cup [6,13]$
\odpStop
\testStart
A.$x \in \{-1\} \cup [6,13]$\\
B.$x \in \{1\} \cup (6,13)$\\
C.$x \in \{-1\} \cup (6,13]$\\
D.$x \in \{1\} \cup (6,13]$\\
E.$x \in \{-1\} \cup [6,13)$\\
F.$x \in \{1\} \cup [6,13)$\\
G.$x \in \{-1\} \cup (6,13)$\\
H.$x \in \{1\} \cup [6,13]$
\testStop
\kluczStart
A
\kluczStop



\zadStart{Zadanie z Wikieł Z 1.62 c) moja wersja nr 118}

Rozwiązać nierówności $(6-x)(x+1)^{2}(14-x)^{3}\le0$.
\zadStop
\rozwStart{Patryk Wirkus}{Laura Mieczkowska}
Miejsca zerowe naszego wielomianu to: $6, -1, 14$.\\
Wielomian jest stopnia parzystego, ponadto znak współczynnika przy\linebreak najwyższej potędze x jest ujemny.\\ W związku z tym wykres wielomianu zaczyna się od lewej strony powyżej osi OX.\\
Ponadto w punkcie $-1$ wykres odbija się od osi poziomej.\\
A więc $$x \in \{-1\} \cup [6,14].$$
\rozwStop
\odpStart
$x \in \{-1\} \cup [6,14]$
\odpStop
\testStart
A.$x \in \{-1\} \cup [6,14]$\\
B.$x \in \{1\} \cup (6,14)$\\
C.$x \in \{-1\} \cup (6,14]$\\
D.$x \in \{1\} \cup (6,14]$\\
E.$x \in \{-1\} \cup [6,14)$\\
F.$x \in \{1\} \cup [6,14)$\\
G.$x \in \{-1\} \cup (6,14)$\\
H.$x \in \{1\} \cup [6,14]$
\testStop
\kluczStart
A
\kluczStop



\zadStart{Zadanie z Wikieł Z 1.62 c) moja wersja nr 119}

Rozwiązać nierówności $(6-x)(x+1)^{2}(15-x)^{3}\le0$.
\zadStop
\rozwStart{Patryk Wirkus}{Laura Mieczkowska}
Miejsca zerowe naszego wielomianu to: $6, -1, 15$.\\
Wielomian jest stopnia parzystego, ponadto znak współczynnika przy\linebreak najwyższej potędze x jest ujemny.\\ W związku z tym wykres wielomianu zaczyna się od lewej strony powyżej osi OX.\\
Ponadto w punkcie $-1$ wykres odbija się od osi poziomej.\\
A więc $$x \in \{-1\} \cup [6,15].$$
\rozwStop
\odpStart
$x \in \{-1\} \cup [6,15]$
\odpStop
\testStart
A.$x \in \{-1\} \cup [6,15]$\\
B.$x \in \{1\} \cup (6,15)$\\
C.$x \in \{-1\} \cup (6,15]$\\
D.$x \in \{1\} \cup (6,15]$\\
E.$x \in \{-1\} \cup [6,15)$\\
F.$x \in \{1\} \cup [6,15)$\\
G.$x \in \{-1\} \cup (6,15)$\\
H.$x \in \{1\} \cup [6,15]$
\testStop
\kluczStart
A
\kluczStop



\zadStart{Zadanie z Wikieł Z 1.62 c) moja wersja nr 120}

Rozwiązać nierówności $(6-x)(x+2)^{2}(7-x)^{3}\le0$.
\zadStop
\rozwStart{Patryk Wirkus}{Laura Mieczkowska}
Miejsca zerowe naszego wielomianu to: $6, -2, 7$.\\
Wielomian jest stopnia parzystego, ponadto znak współczynnika przy\linebreak najwyższej potędze x jest ujemny.\\ W związku z tym wykres wielomianu zaczyna się od lewej strony powyżej osi OX.\\
Ponadto w punkcie $-2$ wykres odbija się od osi poziomej.\\
A więc $$x \in \{-2\} \cup [6,7].$$
\rozwStop
\odpStart
$x \in \{-2\} \cup [6,7]$
\odpStop
\testStart
A.$x \in \{-2\} \cup [6,7]$\\
B.$x \in \{2\} \cup (6,7)$\\
C.$x \in \{-2\} \cup (6,7]$\\
D.$x \in \{2\} \cup (6,7]$\\
E.$x \in \{-2\} \cup [6,7)$\\
F.$x \in \{2\} \cup [6,7)$\\
G.$x \in \{-2\} \cup (6,7)$\\
H.$x \in \{2\} \cup [6,7]$
\testStop
\kluczStart
A
\kluczStop



\zadStart{Zadanie z Wikieł Z 1.62 c) moja wersja nr 121}

Rozwiązać nierówności $(6-x)(x+2)^{2}(8-x)^{3}\le0$.
\zadStop
\rozwStart{Patryk Wirkus}{Laura Mieczkowska}
Miejsca zerowe naszego wielomianu to: $6, -2, 8$.\\
Wielomian jest stopnia parzystego, ponadto znak współczynnika przy\linebreak najwyższej potędze x jest ujemny.\\ W związku z tym wykres wielomianu zaczyna się od lewej strony powyżej osi OX.\\
Ponadto w punkcie $-2$ wykres odbija się od osi poziomej.\\
A więc $$x \in \{-2\} \cup [6,8].$$
\rozwStop
\odpStart
$x \in \{-2\} \cup [6,8]$
\odpStop
\testStart
A.$x \in \{-2\} \cup [6,8]$\\
B.$x \in \{2\} \cup (6,8)$\\
C.$x \in \{-2\} \cup (6,8]$\\
D.$x \in \{2\} \cup (6,8]$\\
E.$x \in \{-2\} \cup [6,8)$\\
F.$x \in \{2\} \cup [6,8)$\\
G.$x \in \{-2\} \cup (6,8)$\\
H.$x \in \{2\} \cup [6,8]$
\testStop
\kluczStart
A
\kluczStop



\zadStart{Zadanie z Wikieł Z 1.62 c) moja wersja nr 122}

Rozwiązać nierówności $(6-x)(x+2)^{2}(9-x)^{3}\le0$.
\zadStop
\rozwStart{Patryk Wirkus}{Laura Mieczkowska}
Miejsca zerowe naszego wielomianu to: $6, -2, 9$.\\
Wielomian jest stopnia parzystego, ponadto znak współczynnika przy\linebreak najwyższej potędze x jest ujemny.\\ W związku z tym wykres wielomianu zaczyna się od lewej strony powyżej osi OX.\\
Ponadto w punkcie $-2$ wykres odbija się od osi poziomej.\\
A więc $$x \in \{-2\} \cup [6,9].$$
\rozwStop
\odpStart
$x \in \{-2\} \cup [6,9]$
\odpStop
\testStart
A.$x \in \{-2\} \cup [6,9]$\\
B.$x \in \{2\} \cup (6,9)$\\
C.$x \in \{-2\} \cup (6,9]$\\
D.$x \in \{2\} \cup (6,9]$\\
E.$x \in \{-2\} \cup [6,9)$\\
F.$x \in \{2\} \cup [6,9)$\\
G.$x \in \{-2\} \cup (6,9)$\\
H.$x \in \{2\} \cup [6,9]$
\testStop
\kluczStart
A
\kluczStop



\zadStart{Zadanie z Wikieł Z 1.62 c) moja wersja nr 123}

Rozwiązać nierówności $(6-x)(x+2)^{2}(10-x)^{3}\le0$.
\zadStop
\rozwStart{Patryk Wirkus}{Laura Mieczkowska}
Miejsca zerowe naszego wielomianu to: $6, -2, 10$.\\
Wielomian jest stopnia parzystego, ponadto znak współczynnika przy\linebreak najwyższej potędze x jest ujemny.\\ W związku z tym wykres wielomianu zaczyna się od lewej strony powyżej osi OX.\\
Ponadto w punkcie $-2$ wykres odbija się od osi poziomej.\\
A więc $$x \in \{-2\} \cup [6,10].$$
\rozwStop
\odpStart
$x \in \{-2\} \cup [6,10]$
\odpStop
\testStart
A.$x \in \{-2\} \cup [6,10]$\\
B.$x \in \{2\} \cup (6,10)$\\
C.$x \in \{-2\} \cup (6,10]$\\
D.$x \in \{2\} \cup (6,10]$\\
E.$x \in \{-2\} \cup [6,10)$\\
F.$x \in \{2\} \cup [6,10)$\\
G.$x \in \{-2\} \cup (6,10)$\\
H.$x \in \{2\} \cup [6,10]$
\testStop
\kluczStart
A
\kluczStop



\zadStart{Zadanie z Wikieł Z 1.62 c) moja wersja nr 124}

Rozwiązać nierówności $(6-x)(x+2)^{2}(11-x)^{3}\le0$.
\zadStop
\rozwStart{Patryk Wirkus}{Laura Mieczkowska}
Miejsca zerowe naszego wielomianu to: $6, -2, 11$.\\
Wielomian jest stopnia parzystego, ponadto znak współczynnika przy\linebreak najwyższej potędze x jest ujemny.\\ W związku z tym wykres wielomianu zaczyna się od lewej strony powyżej osi OX.\\
Ponadto w punkcie $-2$ wykres odbija się od osi poziomej.\\
A więc $$x \in \{-2\} \cup [6,11].$$
\rozwStop
\odpStart
$x \in \{-2\} \cup [6,11]$
\odpStop
\testStart
A.$x \in \{-2\} \cup [6,11]$\\
B.$x \in \{2\} \cup (6,11)$\\
C.$x \in \{-2\} \cup (6,11]$\\
D.$x \in \{2\} \cup (6,11]$\\
E.$x \in \{-2\} \cup [6,11)$\\
F.$x \in \{2\} \cup [6,11)$\\
G.$x \in \{-2\} \cup (6,11)$\\
H.$x \in \{2\} \cup [6,11]$
\testStop
\kluczStart
A
\kluczStop



\zadStart{Zadanie z Wikieł Z 1.62 c) moja wersja nr 125}

Rozwiązać nierówności $(6-x)(x+2)^{2}(12-x)^{3}\le0$.
\zadStop
\rozwStart{Patryk Wirkus}{Laura Mieczkowska}
Miejsca zerowe naszego wielomianu to: $6, -2, 12$.\\
Wielomian jest stopnia parzystego, ponadto znak współczynnika przy\linebreak najwyższej potędze x jest ujemny.\\ W związku z tym wykres wielomianu zaczyna się od lewej strony powyżej osi OX.\\
Ponadto w punkcie $-2$ wykres odbija się od osi poziomej.\\
A więc $$x \in \{-2\} \cup [6,12].$$
\rozwStop
\odpStart
$x \in \{-2\} \cup [6,12]$
\odpStop
\testStart
A.$x \in \{-2\} \cup [6,12]$\\
B.$x \in \{2\} \cup (6,12)$\\
C.$x \in \{-2\} \cup (6,12]$\\
D.$x \in \{2\} \cup (6,12]$\\
E.$x \in \{-2\} \cup [6,12)$\\
F.$x \in \{2\} \cup [6,12)$\\
G.$x \in \{-2\} \cup (6,12)$\\
H.$x \in \{2\} \cup [6,12]$
\testStop
\kluczStart
A
\kluczStop



\zadStart{Zadanie z Wikieł Z 1.62 c) moja wersja nr 126}

Rozwiązać nierówności $(6-x)(x+2)^{2}(13-x)^{3}\le0$.
\zadStop
\rozwStart{Patryk Wirkus}{Laura Mieczkowska}
Miejsca zerowe naszego wielomianu to: $6, -2, 13$.\\
Wielomian jest stopnia parzystego, ponadto znak współczynnika przy\linebreak najwyższej potędze x jest ujemny.\\ W związku z tym wykres wielomianu zaczyna się od lewej strony powyżej osi OX.\\
Ponadto w punkcie $-2$ wykres odbija się od osi poziomej.\\
A więc $$x \in \{-2\} \cup [6,13].$$
\rozwStop
\odpStart
$x \in \{-2\} \cup [6,13]$
\odpStop
\testStart
A.$x \in \{-2\} \cup [6,13]$\\
B.$x \in \{2\} \cup (6,13)$\\
C.$x \in \{-2\} \cup (6,13]$\\
D.$x \in \{2\} \cup (6,13]$\\
E.$x \in \{-2\} \cup [6,13)$\\
F.$x \in \{2\} \cup [6,13)$\\
G.$x \in \{-2\} \cup (6,13)$\\
H.$x \in \{2\} \cup [6,13]$
\testStop
\kluczStart
A
\kluczStop



\zadStart{Zadanie z Wikieł Z 1.62 c) moja wersja nr 127}

Rozwiązać nierówności $(6-x)(x+2)^{2}(14-x)^{3}\le0$.
\zadStop
\rozwStart{Patryk Wirkus}{Laura Mieczkowska}
Miejsca zerowe naszego wielomianu to: $6, -2, 14$.\\
Wielomian jest stopnia parzystego, ponadto znak współczynnika przy\linebreak najwyższej potędze x jest ujemny.\\ W związku z tym wykres wielomianu zaczyna się od lewej strony powyżej osi OX.\\
Ponadto w punkcie $-2$ wykres odbija się od osi poziomej.\\
A więc $$x \in \{-2\} \cup [6,14].$$
\rozwStop
\odpStart
$x \in \{-2\} \cup [6,14]$
\odpStop
\testStart
A.$x \in \{-2\} \cup [6,14]$\\
B.$x \in \{2\} \cup (6,14)$\\
C.$x \in \{-2\} \cup (6,14]$\\
D.$x \in \{2\} \cup (6,14]$\\
E.$x \in \{-2\} \cup [6,14)$\\
F.$x \in \{2\} \cup [6,14)$\\
G.$x \in \{-2\} \cup (6,14)$\\
H.$x \in \{2\} \cup [6,14]$
\testStop
\kluczStart
A
\kluczStop



\zadStart{Zadanie z Wikieł Z 1.62 c) moja wersja nr 128}

Rozwiązać nierówności $(6-x)(x+2)^{2}(15-x)^{3}\le0$.
\zadStop
\rozwStart{Patryk Wirkus}{Laura Mieczkowska}
Miejsca zerowe naszego wielomianu to: $6, -2, 15$.\\
Wielomian jest stopnia parzystego, ponadto znak współczynnika przy\linebreak najwyższej potędze x jest ujemny.\\ W związku z tym wykres wielomianu zaczyna się od lewej strony powyżej osi OX.\\
Ponadto w punkcie $-2$ wykres odbija się od osi poziomej.\\
A więc $$x \in \{-2\} \cup [6,15].$$
\rozwStop
\odpStart
$x \in \{-2\} \cup [6,15]$
\odpStop
\testStart
A.$x \in \{-2\} \cup [6,15]$\\
B.$x \in \{2\} \cup (6,15)$\\
C.$x \in \{-2\} \cup (6,15]$\\
D.$x \in \{2\} \cup (6,15]$\\
E.$x \in \{-2\} \cup [6,15)$\\
F.$x \in \{2\} \cup [6,15)$\\
G.$x \in \{-2\} \cup (6,15)$\\
H.$x \in \{2\} \cup [6,15]$
\testStop
\kluczStart
A
\kluczStop



\zadStart{Zadanie z Wikieł Z 1.62 c) moja wersja nr 129}

Rozwiązać nierówności $(6-x)(x+3)^{2}(7-x)^{3}\le0$.
\zadStop
\rozwStart{Patryk Wirkus}{Laura Mieczkowska}
Miejsca zerowe naszego wielomianu to: $6, -3, 7$.\\
Wielomian jest stopnia parzystego, ponadto znak współczynnika przy\linebreak najwyższej potędze x jest ujemny.\\ W związku z tym wykres wielomianu zaczyna się od lewej strony powyżej osi OX.\\
Ponadto w punkcie $-3$ wykres odbija się od osi poziomej.\\
A więc $$x \in \{-3\} \cup [6,7].$$
\rozwStop
\odpStart
$x \in \{-3\} \cup [6,7]$
\odpStop
\testStart
A.$x \in \{-3\} \cup [6,7]$\\
B.$x \in \{3\} \cup (6,7)$\\
C.$x \in \{-3\} \cup (6,7]$\\
D.$x \in \{3\} \cup (6,7]$\\
E.$x \in \{-3\} \cup [6,7)$\\
F.$x \in \{3\} \cup [6,7)$\\
G.$x \in \{-3\} \cup (6,7)$\\
H.$x \in \{3\} \cup [6,7]$
\testStop
\kluczStart
A
\kluczStop



\zadStart{Zadanie z Wikieł Z 1.62 c) moja wersja nr 130}

Rozwiązać nierówności $(6-x)(x+3)^{2}(8-x)^{3}\le0$.
\zadStop
\rozwStart{Patryk Wirkus}{Laura Mieczkowska}
Miejsca zerowe naszego wielomianu to: $6, -3, 8$.\\
Wielomian jest stopnia parzystego, ponadto znak współczynnika przy\linebreak najwyższej potędze x jest ujemny.\\ W związku z tym wykres wielomianu zaczyna się od lewej strony powyżej osi OX.\\
Ponadto w punkcie $-3$ wykres odbija się od osi poziomej.\\
A więc $$x \in \{-3\} \cup [6,8].$$
\rozwStop
\odpStart
$x \in \{-3\} \cup [6,8]$
\odpStop
\testStart
A.$x \in \{-3\} \cup [6,8]$\\
B.$x \in \{3\} \cup (6,8)$\\
C.$x \in \{-3\} \cup (6,8]$\\
D.$x \in \{3\} \cup (6,8]$\\
E.$x \in \{-3\} \cup [6,8)$\\
F.$x \in \{3\} \cup [6,8)$\\
G.$x \in \{-3\} \cup (6,8)$\\
H.$x \in \{3\} \cup [6,8]$
\testStop
\kluczStart
A
\kluczStop



\zadStart{Zadanie z Wikieł Z 1.62 c) moja wersja nr 131}

Rozwiązać nierówności $(6-x)(x+3)^{2}(9-x)^{3}\le0$.
\zadStop
\rozwStart{Patryk Wirkus}{Laura Mieczkowska}
Miejsca zerowe naszego wielomianu to: $6, -3, 9$.\\
Wielomian jest stopnia parzystego, ponadto znak współczynnika przy\linebreak najwyższej potędze x jest ujemny.\\ W związku z tym wykres wielomianu zaczyna się od lewej strony powyżej osi OX.\\
Ponadto w punkcie $-3$ wykres odbija się od osi poziomej.\\
A więc $$x \in \{-3\} \cup [6,9].$$
\rozwStop
\odpStart
$x \in \{-3\} \cup [6,9]$
\odpStop
\testStart
A.$x \in \{-3\} \cup [6,9]$\\
B.$x \in \{3\} \cup (6,9)$\\
C.$x \in \{-3\} \cup (6,9]$\\
D.$x \in \{3\} \cup (6,9]$\\
E.$x \in \{-3\} \cup [6,9)$\\
F.$x \in \{3\} \cup [6,9)$\\
G.$x \in \{-3\} \cup (6,9)$\\
H.$x \in \{3\} \cup [6,9]$
\testStop
\kluczStart
A
\kluczStop



\zadStart{Zadanie z Wikieł Z 1.62 c) moja wersja nr 132}

Rozwiązać nierówności $(6-x)(x+3)^{2}(10-x)^{3}\le0$.
\zadStop
\rozwStart{Patryk Wirkus}{Laura Mieczkowska}
Miejsca zerowe naszego wielomianu to: $6, -3, 10$.\\
Wielomian jest stopnia parzystego, ponadto znak współczynnika przy\linebreak najwyższej potędze x jest ujemny.\\ W związku z tym wykres wielomianu zaczyna się od lewej strony powyżej osi OX.\\
Ponadto w punkcie $-3$ wykres odbija się od osi poziomej.\\
A więc $$x \in \{-3\} \cup [6,10].$$
\rozwStop
\odpStart
$x \in \{-3\} \cup [6,10]$
\odpStop
\testStart
A.$x \in \{-3\} \cup [6,10]$\\
B.$x \in \{3\} \cup (6,10)$\\
C.$x \in \{-3\} \cup (6,10]$\\
D.$x \in \{3\} \cup (6,10]$\\
E.$x \in \{-3\} \cup [6,10)$\\
F.$x \in \{3\} \cup [6,10)$\\
G.$x \in \{-3\} \cup (6,10)$\\
H.$x \in \{3\} \cup [6,10]$
\testStop
\kluczStart
A
\kluczStop



\zadStart{Zadanie z Wikieł Z 1.62 c) moja wersja nr 133}

Rozwiązać nierówności $(6-x)(x+3)^{2}(11-x)^{3}\le0$.
\zadStop
\rozwStart{Patryk Wirkus}{Laura Mieczkowska}
Miejsca zerowe naszego wielomianu to: $6, -3, 11$.\\
Wielomian jest stopnia parzystego, ponadto znak współczynnika przy\linebreak najwyższej potędze x jest ujemny.\\ W związku z tym wykres wielomianu zaczyna się od lewej strony powyżej osi OX.\\
Ponadto w punkcie $-3$ wykres odbija się od osi poziomej.\\
A więc $$x \in \{-3\} \cup [6,11].$$
\rozwStop
\odpStart
$x \in \{-3\} \cup [6,11]$
\odpStop
\testStart
A.$x \in \{-3\} \cup [6,11]$\\
B.$x \in \{3\} \cup (6,11)$\\
C.$x \in \{-3\} \cup (6,11]$\\
D.$x \in \{3\} \cup (6,11]$\\
E.$x \in \{-3\} \cup [6,11)$\\
F.$x \in \{3\} \cup [6,11)$\\
G.$x \in \{-3\} \cup (6,11)$\\
H.$x \in \{3\} \cup [6,11]$
\testStop
\kluczStart
A
\kluczStop



\zadStart{Zadanie z Wikieł Z 1.62 c) moja wersja nr 134}

Rozwiązać nierówności $(6-x)(x+3)^{2}(12-x)^{3}\le0$.
\zadStop
\rozwStart{Patryk Wirkus}{Laura Mieczkowska}
Miejsca zerowe naszego wielomianu to: $6, -3, 12$.\\
Wielomian jest stopnia parzystego, ponadto znak współczynnika przy\linebreak najwyższej potędze x jest ujemny.\\ W związku z tym wykres wielomianu zaczyna się od lewej strony powyżej osi OX.\\
Ponadto w punkcie $-3$ wykres odbija się od osi poziomej.\\
A więc $$x \in \{-3\} \cup [6,12].$$
\rozwStop
\odpStart
$x \in \{-3\} \cup [6,12]$
\odpStop
\testStart
A.$x \in \{-3\} \cup [6,12]$\\
B.$x \in \{3\} \cup (6,12)$\\
C.$x \in \{-3\} \cup (6,12]$\\
D.$x \in \{3\} \cup (6,12]$\\
E.$x \in \{-3\} \cup [6,12)$\\
F.$x \in \{3\} \cup [6,12)$\\
G.$x \in \{-3\} \cup (6,12)$\\
H.$x \in \{3\} \cup [6,12]$
\testStop
\kluczStart
A
\kluczStop



\zadStart{Zadanie z Wikieł Z 1.62 c) moja wersja nr 135}

Rozwiązać nierówności $(6-x)(x+3)^{2}(13-x)^{3}\le0$.
\zadStop
\rozwStart{Patryk Wirkus}{Laura Mieczkowska}
Miejsca zerowe naszego wielomianu to: $6, -3, 13$.\\
Wielomian jest stopnia parzystego, ponadto znak współczynnika przy\linebreak najwyższej potędze x jest ujemny.\\ W związku z tym wykres wielomianu zaczyna się od lewej strony powyżej osi OX.\\
Ponadto w punkcie $-3$ wykres odbija się od osi poziomej.\\
A więc $$x \in \{-3\} \cup [6,13].$$
\rozwStop
\odpStart
$x \in \{-3\} \cup [6,13]$
\odpStop
\testStart
A.$x \in \{-3\} \cup [6,13]$\\
B.$x \in \{3\} \cup (6,13)$\\
C.$x \in \{-3\} \cup (6,13]$\\
D.$x \in \{3\} \cup (6,13]$\\
E.$x \in \{-3\} \cup [6,13)$\\
F.$x \in \{3\} \cup [6,13)$\\
G.$x \in \{-3\} \cup (6,13)$\\
H.$x \in \{3\} \cup [6,13]$
\testStop
\kluczStart
A
\kluczStop



\zadStart{Zadanie z Wikieł Z 1.62 c) moja wersja nr 136}

Rozwiązać nierówności $(6-x)(x+3)^{2}(14-x)^{3}\le0$.
\zadStop
\rozwStart{Patryk Wirkus}{Laura Mieczkowska}
Miejsca zerowe naszego wielomianu to: $6, -3, 14$.\\
Wielomian jest stopnia parzystego, ponadto znak współczynnika przy\linebreak najwyższej potędze x jest ujemny.\\ W związku z tym wykres wielomianu zaczyna się od lewej strony powyżej osi OX.\\
Ponadto w punkcie $-3$ wykres odbija się od osi poziomej.\\
A więc $$x \in \{-3\} \cup [6,14].$$
\rozwStop
\odpStart
$x \in \{-3\} \cup [6,14]$
\odpStop
\testStart
A.$x \in \{-3\} \cup [6,14]$\\
B.$x \in \{3\} \cup (6,14)$\\
C.$x \in \{-3\} \cup (6,14]$\\
D.$x \in \{3\} \cup (6,14]$\\
E.$x \in \{-3\} \cup [6,14)$\\
F.$x \in \{3\} \cup [6,14)$\\
G.$x \in \{-3\} \cup (6,14)$\\
H.$x \in \{3\} \cup [6,14]$
\testStop
\kluczStart
A
\kluczStop



\zadStart{Zadanie z Wikieł Z 1.62 c) moja wersja nr 137}

Rozwiązać nierówności $(6-x)(x+3)^{2}(15-x)^{3}\le0$.
\zadStop
\rozwStart{Patryk Wirkus}{Laura Mieczkowska}
Miejsca zerowe naszego wielomianu to: $6, -3, 15$.\\
Wielomian jest stopnia parzystego, ponadto znak współczynnika przy\linebreak najwyższej potędze x jest ujemny.\\ W związku z tym wykres wielomianu zaczyna się od lewej strony powyżej osi OX.\\
Ponadto w punkcie $-3$ wykres odbija się od osi poziomej.\\
A więc $$x \in \{-3\} \cup [6,15].$$
\rozwStop
\odpStart
$x \in \{-3\} \cup [6,15]$
\odpStop
\testStart
A.$x \in \{-3\} \cup [6,15]$\\
B.$x \in \{3\} \cup (6,15)$\\
C.$x \in \{-3\} \cup (6,15]$\\
D.$x \in \{3\} \cup (6,15]$\\
E.$x \in \{-3\} \cup [6,15)$\\
F.$x \in \{3\} \cup [6,15)$\\
G.$x \in \{-3\} \cup (6,15)$\\
H.$x \in \{3\} \cup [6,15]$
\testStop
\kluczStart
A
\kluczStop



\zadStart{Zadanie z Wikieł Z 1.62 c) moja wersja nr 138}

Rozwiązać nierówności $(6-x)(x+4)^{2}(7-x)^{3}\le0$.
\zadStop
\rozwStart{Patryk Wirkus}{Laura Mieczkowska}
Miejsca zerowe naszego wielomianu to: $6, -4, 7$.\\
Wielomian jest stopnia parzystego, ponadto znak współczynnika przy\linebreak najwyższej potędze x jest ujemny.\\ W związku z tym wykres wielomianu zaczyna się od lewej strony powyżej osi OX.\\
Ponadto w punkcie $-4$ wykres odbija się od osi poziomej.\\
A więc $$x \in \{-4\} \cup [6,7].$$
\rozwStop
\odpStart
$x \in \{-4\} \cup [6,7]$
\odpStop
\testStart
A.$x \in \{-4\} \cup [6,7]$\\
B.$x \in \{4\} \cup (6,7)$\\
C.$x \in \{-4\} \cup (6,7]$\\
D.$x \in \{4\} \cup (6,7]$\\
E.$x \in \{-4\} \cup [6,7)$\\
F.$x \in \{4\} \cup [6,7)$\\
G.$x \in \{-4\} \cup (6,7)$\\
H.$x \in \{4\} \cup [6,7]$
\testStop
\kluczStart
A
\kluczStop



\zadStart{Zadanie z Wikieł Z 1.62 c) moja wersja nr 139}

Rozwiązać nierówności $(6-x)(x+4)^{2}(8-x)^{3}\le0$.
\zadStop
\rozwStart{Patryk Wirkus}{Laura Mieczkowska}
Miejsca zerowe naszego wielomianu to: $6, -4, 8$.\\
Wielomian jest stopnia parzystego, ponadto znak współczynnika przy\linebreak najwyższej potędze x jest ujemny.\\ W związku z tym wykres wielomianu zaczyna się od lewej strony powyżej osi OX.\\
Ponadto w punkcie $-4$ wykres odbija się od osi poziomej.\\
A więc $$x \in \{-4\} \cup [6,8].$$
\rozwStop
\odpStart
$x \in \{-4\} \cup [6,8]$
\odpStop
\testStart
A.$x \in \{-4\} \cup [6,8]$\\
B.$x \in \{4\} \cup (6,8)$\\
C.$x \in \{-4\} \cup (6,8]$\\
D.$x \in \{4\} \cup (6,8]$\\
E.$x \in \{-4\} \cup [6,8)$\\
F.$x \in \{4\} \cup [6,8)$\\
G.$x \in \{-4\} \cup (6,8)$\\
H.$x \in \{4\} \cup [6,8]$
\testStop
\kluczStart
A
\kluczStop



\zadStart{Zadanie z Wikieł Z 1.62 c) moja wersja nr 140}

Rozwiązać nierówności $(6-x)(x+4)^{2}(9-x)^{3}\le0$.
\zadStop
\rozwStart{Patryk Wirkus}{Laura Mieczkowska}
Miejsca zerowe naszego wielomianu to: $6, -4, 9$.\\
Wielomian jest stopnia parzystego, ponadto znak współczynnika przy\linebreak najwyższej potędze x jest ujemny.\\ W związku z tym wykres wielomianu zaczyna się od lewej strony powyżej osi OX.\\
Ponadto w punkcie $-4$ wykres odbija się od osi poziomej.\\
A więc $$x \in \{-4\} \cup [6,9].$$
\rozwStop
\odpStart
$x \in \{-4\} \cup [6,9]$
\odpStop
\testStart
A.$x \in \{-4\} \cup [6,9]$\\
B.$x \in \{4\} \cup (6,9)$\\
C.$x \in \{-4\} \cup (6,9]$\\
D.$x \in \{4\} \cup (6,9]$\\
E.$x \in \{-4\} \cup [6,9)$\\
F.$x \in \{4\} \cup [6,9)$\\
G.$x \in \{-4\} \cup (6,9)$\\
H.$x \in \{4\} \cup [6,9]$
\testStop
\kluczStart
A
\kluczStop



\zadStart{Zadanie z Wikieł Z 1.62 c) moja wersja nr 141}

Rozwiązać nierówności $(6-x)(x+4)^{2}(10-x)^{3}\le0$.
\zadStop
\rozwStart{Patryk Wirkus}{Laura Mieczkowska}
Miejsca zerowe naszego wielomianu to: $6, -4, 10$.\\
Wielomian jest stopnia parzystego, ponadto znak współczynnika przy\linebreak najwyższej potędze x jest ujemny.\\ W związku z tym wykres wielomianu zaczyna się od lewej strony powyżej osi OX.\\
Ponadto w punkcie $-4$ wykres odbija się od osi poziomej.\\
A więc $$x \in \{-4\} \cup [6,10].$$
\rozwStop
\odpStart
$x \in \{-4\} \cup [6,10]$
\odpStop
\testStart
A.$x \in \{-4\} \cup [6,10]$\\
B.$x \in \{4\} \cup (6,10)$\\
C.$x \in \{-4\} \cup (6,10]$\\
D.$x \in \{4\} \cup (6,10]$\\
E.$x \in \{-4\} \cup [6,10)$\\
F.$x \in \{4\} \cup [6,10)$\\
G.$x \in \{-4\} \cup (6,10)$\\
H.$x \in \{4\} \cup [6,10]$
\testStop
\kluczStart
A
\kluczStop



\zadStart{Zadanie z Wikieł Z 1.62 c) moja wersja nr 142}

Rozwiązać nierówności $(6-x)(x+4)^{2}(11-x)^{3}\le0$.
\zadStop
\rozwStart{Patryk Wirkus}{Laura Mieczkowska}
Miejsca zerowe naszego wielomianu to: $6, -4, 11$.\\
Wielomian jest stopnia parzystego, ponadto znak współczynnika przy\linebreak najwyższej potędze x jest ujemny.\\ W związku z tym wykres wielomianu zaczyna się od lewej strony powyżej osi OX.\\
Ponadto w punkcie $-4$ wykres odbija się od osi poziomej.\\
A więc $$x \in \{-4\} \cup [6,11].$$
\rozwStop
\odpStart
$x \in \{-4\} \cup [6,11]$
\odpStop
\testStart
A.$x \in \{-4\} \cup [6,11]$\\
B.$x \in \{4\} \cup (6,11)$\\
C.$x \in \{-4\} \cup (6,11]$\\
D.$x \in \{4\} \cup (6,11]$\\
E.$x \in \{-4\} \cup [6,11)$\\
F.$x \in \{4\} \cup [6,11)$\\
G.$x \in \{-4\} \cup (6,11)$\\
H.$x \in \{4\} \cup [6,11]$
\testStop
\kluczStart
A
\kluczStop



\zadStart{Zadanie z Wikieł Z 1.62 c) moja wersja nr 143}

Rozwiązać nierówności $(6-x)(x+4)^{2}(12-x)^{3}\le0$.
\zadStop
\rozwStart{Patryk Wirkus}{Laura Mieczkowska}
Miejsca zerowe naszego wielomianu to: $6, -4, 12$.\\
Wielomian jest stopnia parzystego, ponadto znak współczynnika przy\linebreak najwyższej potędze x jest ujemny.\\ W związku z tym wykres wielomianu zaczyna się od lewej strony powyżej osi OX.\\
Ponadto w punkcie $-4$ wykres odbija się od osi poziomej.\\
A więc $$x \in \{-4\} \cup [6,12].$$
\rozwStop
\odpStart
$x \in \{-4\} \cup [6,12]$
\odpStop
\testStart
A.$x \in \{-4\} \cup [6,12]$\\
B.$x \in \{4\} \cup (6,12)$\\
C.$x \in \{-4\} \cup (6,12]$\\
D.$x \in \{4\} \cup (6,12]$\\
E.$x \in \{-4\} \cup [6,12)$\\
F.$x \in \{4\} \cup [6,12)$\\
G.$x \in \{-4\} \cup (6,12)$\\
H.$x \in \{4\} \cup [6,12]$
\testStop
\kluczStart
A
\kluczStop



\zadStart{Zadanie z Wikieł Z 1.62 c) moja wersja nr 144}

Rozwiązać nierówności $(6-x)(x+4)^{2}(13-x)^{3}\le0$.
\zadStop
\rozwStart{Patryk Wirkus}{Laura Mieczkowska}
Miejsca zerowe naszego wielomianu to: $6, -4, 13$.\\
Wielomian jest stopnia parzystego, ponadto znak współczynnika przy\linebreak najwyższej potędze x jest ujemny.\\ W związku z tym wykres wielomianu zaczyna się od lewej strony powyżej osi OX.\\
Ponadto w punkcie $-4$ wykres odbija się od osi poziomej.\\
A więc $$x \in \{-4\} \cup [6,13].$$
\rozwStop
\odpStart
$x \in \{-4\} \cup [6,13]$
\odpStop
\testStart
A.$x \in \{-4\} \cup [6,13]$\\
B.$x \in \{4\} \cup (6,13)$\\
C.$x \in \{-4\} \cup (6,13]$\\
D.$x \in \{4\} \cup (6,13]$\\
E.$x \in \{-4\} \cup [6,13)$\\
F.$x \in \{4\} \cup [6,13)$\\
G.$x \in \{-4\} \cup (6,13)$\\
H.$x \in \{4\} \cup [6,13]$
\testStop
\kluczStart
A
\kluczStop



\zadStart{Zadanie z Wikieł Z 1.62 c) moja wersja nr 145}

Rozwiązać nierówności $(6-x)(x+4)^{2}(14-x)^{3}\le0$.
\zadStop
\rozwStart{Patryk Wirkus}{Laura Mieczkowska}
Miejsca zerowe naszego wielomianu to: $6, -4, 14$.\\
Wielomian jest stopnia parzystego, ponadto znak współczynnika przy\linebreak najwyższej potędze x jest ujemny.\\ W związku z tym wykres wielomianu zaczyna się od lewej strony powyżej osi OX.\\
Ponadto w punkcie $-4$ wykres odbija się od osi poziomej.\\
A więc $$x \in \{-4\} \cup [6,14].$$
\rozwStop
\odpStart
$x \in \{-4\} \cup [6,14]$
\odpStop
\testStart
A.$x \in \{-4\} \cup [6,14]$\\
B.$x \in \{4\} \cup (6,14)$\\
C.$x \in \{-4\} \cup (6,14]$\\
D.$x \in \{4\} \cup (6,14]$\\
E.$x \in \{-4\} \cup [6,14)$\\
F.$x \in \{4\} \cup [6,14)$\\
G.$x \in \{-4\} \cup (6,14)$\\
H.$x \in \{4\} \cup [6,14]$
\testStop
\kluczStart
A
\kluczStop



\zadStart{Zadanie z Wikieł Z 1.62 c) moja wersja nr 146}

Rozwiązać nierówności $(6-x)(x+4)^{2}(15-x)^{3}\le0$.
\zadStop
\rozwStart{Patryk Wirkus}{Laura Mieczkowska}
Miejsca zerowe naszego wielomianu to: $6, -4, 15$.\\
Wielomian jest stopnia parzystego, ponadto znak współczynnika przy\linebreak najwyższej potędze x jest ujemny.\\ W związku z tym wykres wielomianu zaczyna się od lewej strony powyżej osi OX.\\
Ponadto w punkcie $-4$ wykres odbija się od osi poziomej.\\
A więc $$x \in \{-4\} \cup [6,15].$$
\rozwStop
\odpStart
$x \in \{-4\} \cup [6,15]$
\odpStop
\testStart
A.$x \in \{-4\} \cup [6,15]$\\
B.$x \in \{4\} \cup (6,15)$\\
C.$x \in \{-4\} \cup (6,15]$\\
D.$x \in \{4\} \cup (6,15]$\\
E.$x \in \{-4\} \cup [6,15)$\\
F.$x \in \{4\} \cup [6,15)$\\
G.$x \in \{-4\} \cup (6,15)$\\
H.$x \in \{4\} \cup [6,15]$
\testStop
\kluczStart
A
\kluczStop



\zadStart{Zadanie z Wikieł Z 1.62 c) moja wersja nr 147}

Rozwiązać nierówności $(6-x)(x+5)^{2}(7-x)^{3}\le0$.
\zadStop
\rozwStart{Patryk Wirkus}{Laura Mieczkowska}
Miejsca zerowe naszego wielomianu to: $6, -5, 7$.\\
Wielomian jest stopnia parzystego, ponadto znak współczynnika przy\linebreak najwyższej potędze x jest ujemny.\\ W związku z tym wykres wielomianu zaczyna się od lewej strony powyżej osi OX.\\
Ponadto w punkcie $-5$ wykres odbija się od osi poziomej.\\
A więc $$x \in \{-5\} \cup [6,7].$$
\rozwStop
\odpStart
$x \in \{-5\} \cup [6,7]$
\odpStop
\testStart
A.$x \in \{-5\} \cup [6,7]$\\
B.$x \in \{5\} \cup (6,7)$\\
C.$x \in \{-5\} \cup (6,7]$\\
D.$x \in \{5\} \cup (6,7]$\\
E.$x \in \{-5\} \cup [6,7)$\\
F.$x \in \{5\} \cup [6,7)$\\
G.$x \in \{-5\} \cup (6,7)$\\
H.$x \in \{5\} \cup [6,7]$
\testStop
\kluczStart
A
\kluczStop



\zadStart{Zadanie z Wikieł Z 1.62 c) moja wersja nr 148}

Rozwiązać nierówności $(6-x)(x+5)^{2}(8-x)^{3}\le0$.
\zadStop
\rozwStart{Patryk Wirkus}{Laura Mieczkowska}
Miejsca zerowe naszego wielomianu to: $6, -5, 8$.\\
Wielomian jest stopnia parzystego, ponadto znak współczynnika przy\linebreak najwyższej potędze x jest ujemny.\\ W związku z tym wykres wielomianu zaczyna się od lewej strony powyżej osi OX.\\
Ponadto w punkcie $-5$ wykres odbija się od osi poziomej.\\
A więc $$x \in \{-5\} \cup [6,8].$$
\rozwStop
\odpStart
$x \in \{-5\} \cup [6,8]$
\odpStop
\testStart
A.$x \in \{-5\} \cup [6,8]$\\
B.$x \in \{5\} \cup (6,8)$\\
C.$x \in \{-5\} \cup (6,8]$\\
D.$x \in \{5\} \cup (6,8]$\\
E.$x \in \{-5\} \cup [6,8)$\\
F.$x \in \{5\} \cup [6,8)$\\
G.$x \in \{-5\} \cup (6,8)$\\
H.$x \in \{5\} \cup [6,8]$
\testStop
\kluczStart
A
\kluczStop



\zadStart{Zadanie z Wikieł Z 1.62 c) moja wersja nr 149}

Rozwiązać nierówności $(6-x)(x+5)^{2}(9-x)^{3}\le0$.
\zadStop
\rozwStart{Patryk Wirkus}{Laura Mieczkowska}
Miejsca zerowe naszego wielomianu to: $6, -5, 9$.\\
Wielomian jest stopnia parzystego, ponadto znak współczynnika przy\linebreak najwyższej potędze x jest ujemny.\\ W związku z tym wykres wielomianu zaczyna się od lewej strony powyżej osi OX.\\
Ponadto w punkcie $-5$ wykres odbija się od osi poziomej.\\
A więc $$x \in \{-5\} \cup [6,9].$$
\rozwStop
\odpStart
$x \in \{-5\} \cup [6,9]$
\odpStop
\testStart
A.$x \in \{-5\} \cup [6,9]$\\
B.$x \in \{5\} \cup (6,9)$\\
C.$x \in \{-5\} \cup (6,9]$\\
D.$x \in \{5\} \cup (6,9]$\\
E.$x \in \{-5\} \cup [6,9)$\\
F.$x \in \{5\} \cup [6,9)$\\
G.$x \in \{-5\} \cup (6,9)$\\
H.$x \in \{5\} \cup [6,9]$
\testStop
\kluczStart
A
\kluczStop



\zadStart{Zadanie z Wikieł Z 1.62 c) moja wersja nr 150}

Rozwiązać nierówności $(6-x)(x+5)^{2}(10-x)^{3}\le0$.
\zadStop
\rozwStart{Patryk Wirkus}{Laura Mieczkowska}
Miejsca zerowe naszego wielomianu to: $6, -5, 10$.\\
Wielomian jest stopnia parzystego, ponadto znak współczynnika przy\linebreak najwyższej potędze x jest ujemny.\\ W związku z tym wykres wielomianu zaczyna się od lewej strony powyżej osi OX.\\
Ponadto w punkcie $-5$ wykres odbija się od osi poziomej.\\
A więc $$x \in \{-5\} \cup [6,10].$$
\rozwStop
\odpStart
$x \in \{-5\} \cup [6,10]$
\odpStop
\testStart
A.$x \in \{-5\} \cup [6,10]$\\
B.$x \in \{5\} \cup (6,10)$\\
C.$x \in \{-5\} \cup (6,10]$\\
D.$x \in \{5\} \cup (6,10]$\\
E.$x \in \{-5\} \cup [6,10)$\\
F.$x \in \{5\} \cup [6,10)$\\
G.$x \in \{-5\} \cup (6,10)$\\
H.$x \in \{5\} \cup [6,10]$
\testStop
\kluczStart
A
\kluczStop



\zadStart{Zadanie z Wikieł Z 1.62 c) moja wersja nr 151}

Rozwiązać nierówności $(6-x)(x+5)^{2}(11-x)^{3}\le0$.
\zadStop
\rozwStart{Patryk Wirkus}{Laura Mieczkowska}
Miejsca zerowe naszego wielomianu to: $6, -5, 11$.\\
Wielomian jest stopnia parzystego, ponadto znak współczynnika przy\linebreak najwyższej potędze x jest ujemny.\\ W związku z tym wykres wielomianu zaczyna się od lewej strony powyżej osi OX.\\
Ponadto w punkcie $-5$ wykres odbija się od osi poziomej.\\
A więc $$x \in \{-5\} \cup [6,11].$$
\rozwStop
\odpStart
$x \in \{-5\} \cup [6,11]$
\odpStop
\testStart
A.$x \in \{-5\} \cup [6,11]$\\
B.$x \in \{5\} \cup (6,11)$\\
C.$x \in \{-5\} \cup (6,11]$\\
D.$x \in \{5\} \cup (6,11]$\\
E.$x \in \{-5\} \cup [6,11)$\\
F.$x \in \{5\} \cup [6,11)$\\
G.$x \in \{-5\} \cup (6,11)$\\
H.$x \in \{5\} \cup [6,11]$
\testStop
\kluczStart
A
\kluczStop



\zadStart{Zadanie z Wikieł Z 1.62 c) moja wersja nr 152}

Rozwiązać nierówności $(6-x)(x+5)^{2}(12-x)^{3}\le0$.
\zadStop
\rozwStart{Patryk Wirkus}{Laura Mieczkowska}
Miejsca zerowe naszego wielomianu to: $6, -5, 12$.\\
Wielomian jest stopnia parzystego, ponadto znak współczynnika przy\linebreak najwyższej potędze x jest ujemny.\\ W związku z tym wykres wielomianu zaczyna się od lewej strony powyżej osi OX.\\
Ponadto w punkcie $-5$ wykres odbija się od osi poziomej.\\
A więc $$x \in \{-5\} \cup [6,12].$$
\rozwStop
\odpStart
$x \in \{-5\} \cup [6,12]$
\odpStop
\testStart
A.$x \in \{-5\} \cup [6,12]$\\
B.$x \in \{5\} \cup (6,12)$\\
C.$x \in \{-5\} \cup (6,12]$\\
D.$x \in \{5\} \cup (6,12]$\\
E.$x \in \{-5\} \cup [6,12)$\\
F.$x \in \{5\} \cup [6,12)$\\
G.$x \in \{-5\} \cup (6,12)$\\
H.$x \in \{5\} \cup [6,12]$
\testStop
\kluczStart
A
\kluczStop



\zadStart{Zadanie z Wikieł Z 1.62 c) moja wersja nr 153}

Rozwiązać nierówności $(6-x)(x+5)^{2}(13-x)^{3}\le0$.
\zadStop
\rozwStart{Patryk Wirkus}{Laura Mieczkowska}
Miejsca zerowe naszego wielomianu to: $6, -5, 13$.\\
Wielomian jest stopnia parzystego, ponadto znak współczynnika przy\linebreak najwyższej potędze x jest ujemny.\\ W związku z tym wykres wielomianu zaczyna się od lewej strony powyżej osi OX.\\
Ponadto w punkcie $-5$ wykres odbija się od osi poziomej.\\
A więc $$x \in \{-5\} \cup [6,13].$$
\rozwStop
\odpStart
$x \in \{-5\} \cup [6,13]$
\odpStop
\testStart
A.$x \in \{-5\} \cup [6,13]$\\
B.$x \in \{5\} \cup (6,13)$\\
C.$x \in \{-5\} \cup (6,13]$\\
D.$x \in \{5\} \cup (6,13]$\\
E.$x \in \{-5\} \cup [6,13)$\\
F.$x \in \{5\} \cup [6,13)$\\
G.$x \in \{-5\} \cup (6,13)$\\
H.$x \in \{5\} \cup [6,13]$
\testStop
\kluczStart
A
\kluczStop



\zadStart{Zadanie z Wikieł Z 1.62 c) moja wersja nr 154}

Rozwiązać nierówności $(6-x)(x+5)^{2}(14-x)^{3}\le0$.
\zadStop
\rozwStart{Patryk Wirkus}{Laura Mieczkowska}
Miejsca zerowe naszego wielomianu to: $6, -5, 14$.\\
Wielomian jest stopnia parzystego, ponadto znak współczynnika przy\linebreak najwyższej potędze x jest ujemny.\\ W związku z tym wykres wielomianu zaczyna się od lewej strony powyżej osi OX.\\
Ponadto w punkcie $-5$ wykres odbija się od osi poziomej.\\
A więc $$x \in \{-5\} \cup [6,14].$$
\rozwStop
\odpStart
$x \in \{-5\} \cup [6,14]$
\odpStop
\testStart
A.$x \in \{-5\} \cup [6,14]$\\
B.$x \in \{5\} \cup (6,14)$\\
C.$x \in \{-5\} \cup (6,14]$\\
D.$x \in \{5\} \cup (6,14]$\\
E.$x \in \{-5\} \cup [6,14)$\\
F.$x \in \{5\} \cup [6,14)$\\
G.$x \in \{-5\} \cup (6,14)$\\
H.$x \in \{5\} \cup [6,14]$
\testStop
\kluczStart
A
\kluczStop



\zadStart{Zadanie z Wikieł Z 1.62 c) moja wersja nr 155}

Rozwiązać nierówności $(6-x)(x+5)^{2}(15-x)^{3}\le0$.
\zadStop
\rozwStart{Patryk Wirkus}{Laura Mieczkowska}
Miejsca zerowe naszego wielomianu to: $6, -5, 15$.\\
Wielomian jest stopnia parzystego, ponadto znak współczynnika przy\linebreak najwyższej potędze x jest ujemny.\\ W związku z tym wykres wielomianu zaczyna się od lewej strony powyżej osi OX.\\
Ponadto w punkcie $-5$ wykres odbija się od osi poziomej.\\
A więc $$x \in \{-5\} \cup [6,15].$$
\rozwStop
\odpStart
$x \in \{-5\} \cup [6,15]$
\odpStop
\testStart
A.$x \in \{-5\} \cup [6,15]$\\
B.$x \in \{5\} \cup (6,15)$\\
C.$x \in \{-5\} \cup (6,15]$\\
D.$x \in \{5\} \cup (6,15]$\\
E.$x \in \{-5\} \cup [6,15)$\\
F.$x \in \{5\} \cup [6,15)$\\
G.$x \in \{-5\} \cup (6,15)$\\
H.$x \in \{5\} \cup [6,15]$
\testStop
\kluczStart
A
\kluczStop



\zadStart{Zadanie z Wikieł Z 1.62 c) moja wersja nr 156}

Rozwiązać nierówności $(7-x)(x+1)^{2}(8-x)^{3}\le0$.
\zadStop
\rozwStart{Patryk Wirkus}{Laura Mieczkowska}
Miejsca zerowe naszego wielomianu to: $7, -1, 8$.\\
Wielomian jest stopnia parzystego, ponadto znak współczynnika przy\linebreak najwyższej potędze x jest ujemny.\\ W związku z tym wykres wielomianu zaczyna się od lewej strony powyżej osi OX.\\
Ponadto w punkcie $-1$ wykres odbija się od osi poziomej.\\
A więc $$x \in \{-1\} \cup [7,8].$$
\rozwStop
\odpStart
$x \in \{-1\} \cup [7,8]$
\odpStop
\testStart
A.$x \in \{-1\} \cup [7,8]$\\
B.$x \in \{1\} \cup (7,8)$\\
C.$x \in \{-1\} \cup (7,8]$\\
D.$x \in \{1\} \cup (7,8]$\\
E.$x \in \{-1\} \cup [7,8)$\\
F.$x \in \{1\} \cup [7,8)$\\
G.$x \in \{-1\} \cup (7,8)$\\
H.$x \in \{1\} \cup [7,8]$
\testStop
\kluczStart
A
\kluczStop



\zadStart{Zadanie z Wikieł Z 1.62 c) moja wersja nr 157}

Rozwiązać nierówności $(7-x)(x+1)^{2}(9-x)^{3}\le0$.
\zadStop
\rozwStart{Patryk Wirkus}{Laura Mieczkowska}
Miejsca zerowe naszego wielomianu to: $7, -1, 9$.\\
Wielomian jest stopnia parzystego, ponadto znak współczynnika przy\linebreak najwyższej potędze x jest ujemny.\\ W związku z tym wykres wielomianu zaczyna się od lewej strony powyżej osi OX.\\
Ponadto w punkcie $-1$ wykres odbija się od osi poziomej.\\
A więc $$x \in \{-1\} \cup [7,9].$$
\rozwStop
\odpStart
$x \in \{-1\} \cup [7,9]$
\odpStop
\testStart
A.$x \in \{-1\} \cup [7,9]$\\
B.$x \in \{1\} \cup (7,9)$\\
C.$x \in \{-1\} \cup (7,9]$\\
D.$x \in \{1\} \cup (7,9]$\\
E.$x \in \{-1\} \cup [7,9)$\\
F.$x \in \{1\} \cup [7,9)$\\
G.$x \in \{-1\} \cup (7,9)$\\
H.$x \in \{1\} \cup [7,9]$
\testStop
\kluczStart
A
\kluczStop



\zadStart{Zadanie z Wikieł Z 1.62 c) moja wersja nr 158}

Rozwiązać nierówności $(7-x)(x+1)^{2}(10-x)^{3}\le0$.
\zadStop
\rozwStart{Patryk Wirkus}{Laura Mieczkowska}
Miejsca zerowe naszego wielomianu to: $7, -1, 10$.\\
Wielomian jest stopnia parzystego, ponadto znak współczynnika przy\linebreak najwyższej potędze x jest ujemny.\\ W związku z tym wykres wielomianu zaczyna się od lewej strony powyżej osi OX.\\
Ponadto w punkcie $-1$ wykres odbija się od osi poziomej.\\
A więc $$x \in \{-1\} \cup [7,10].$$
\rozwStop
\odpStart
$x \in \{-1\} \cup [7,10]$
\odpStop
\testStart
A.$x \in \{-1\} \cup [7,10]$\\
B.$x \in \{1\} \cup (7,10)$\\
C.$x \in \{-1\} \cup (7,10]$\\
D.$x \in \{1\} \cup (7,10]$\\
E.$x \in \{-1\} \cup [7,10)$\\
F.$x \in \{1\} \cup [7,10)$\\
G.$x \in \{-1\} \cup (7,10)$\\
H.$x \in \{1\} \cup [7,10]$
\testStop
\kluczStart
A
\kluczStop



\zadStart{Zadanie z Wikieł Z 1.62 c) moja wersja nr 159}

Rozwiązać nierówności $(7-x)(x+1)^{2}(11-x)^{3}\le0$.
\zadStop
\rozwStart{Patryk Wirkus}{Laura Mieczkowska}
Miejsca zerowe naszego wielomianu to: $7, -1, 11$.\\
Wielomian jest stopnia parzystego, ponadto znak współczynnika przy\linebreak najwyższej potędze x jest ujemny.\\ W związku z tym wykres wielomianu zaczyna się od lewej strony powyżej osi OX.\\
Ponadto w punkcie $-1$ wykres odbija się od osi poziomej.\\
A więc $$x \in \{-1\} \cup [7,11].$$
\rozwStop
\odpStart
$x \in \{-1\} \cup [7,11]$
\odpStop
\testStart
A.$x \in \{-1\} \cup [7,11]$\\
B.$x \in \{1\} \cup (7,11)$\\
C.$x \in \{-1\} \cup (7,11]$\\
D.$x \in \{1\} \cup (7,11]$\\
E.$x \in \{-1\} \cup [7,11)$\\
F.$x \in \{1\} \cup [7,11)$\\
G.$x \in \{-1\} \cup (7,11)$\\
H.$x \in \{1\} \cup [7,11]$
\testStop
\kluczStart
A
\kluczStop



\zadStart{Zadanie z Wikieł Z 1.62 c) moja wersja nr 160}

Rozwiązać nierówności $(7-x)(x+1)^{2}(12-x)^{3}\le0$.
\zadStop
\rozwStart{Patryk Wirkus}{Laura Mieczkowska}
Miejsca zerowe naszego wielomianu to: $7, -1, 12$.\\
Wielomian jest stopnia parzystego, ponadto znak współczynnika przy\linebreak najwyższej potędze x jest ujemny.\\ W związku z tym wykres wielomianu zaczyna się od lewej strony powyżej osi OX.\\
Ponadto w punkcie $-1$ wykres odbija się od osi poziomej.\\
A więc $$x \in \{-1\} \cup [7,12].$$
\rozwStop
\odpStart
$x \in \{-1\} \cup [7,12]$
\odpStop
\testStart
A.$x \in \{-1\} \cup [7,12]$\\
B.$x \in \{1\} \cup (7,12)$\\
C.$x \in \{-1\} \cup (7,12]$\\
D.$x \in \{1\} \cup (7,12]$\\
E.$x \in \{-1\} \cup [7,12)$\\
F.$x \in \{1\} \cup [7,12)$\\
G.$x \in \{-1\} \cup (7,12)$\\
H.$x \in \{1\} \cup [7,12]$
\testStop
\kluczStart
A
\kluczStop



\zadStart{Zadanie z Wikieł Z 1.62 c) moja wersja nr 161}

Rozwiązać nierówności $(7-x)(x+1)^{2}(13-x)^{3}\le0$.
\zadStop
\rozwStart{Patryk Wirkus}{Laura Mieczkowska}
Miejsca zerowe naszego wielomianu to: $7, -1, 13$.\\
Wielomian jest stopnia parzystego, ponadto znak współczynnika przy\linebreak najwyższej potędze x jest ujemny.\\ W związku z tym wykres wielomianu zaczyna się od lewej strony powyżej osi OX.\\
Ponadto w punkcie $-1$ wykres odbija się od osi poziomej.\\
A więc $$x \in \{-1\} \cup [7,13].$$
\rozwStop
\odpStart
$x \in \{-1\} \cup [7,13]$
\odpStop
\testStart
A.$x \in \{-1\} \cup [7,13]$\\
B.$x \in \{1\} \cup (7,13)$\\
C.$x \in \{-1\} \cup (7,13]$\\
D.$x \in \{1\} \cup (7,13]$\\
E.$x \in \{-1\} \cup [7,13)$\\
F.$x \in \{1\} \cup [7,13)$\\
G.$x \in \{-1\} \cup (7,13)$\\
H.$x \in \{1\} \cup [7,13]$
\testStop
\kluczStart
A
\kluczStop



\zadStart{Zadanie z Wikieł Z 1.62 c) moja wersja nr 162}

Rozwiązać nierówności $(7-x)(x+1)^{2}(14-x)^{3}\le0$.
\zadStop
\rozwStart{Patryk Wirkus}{Laura Mieczkowska}
Miejsca zerowe naszego wielomianu to: $7, -1, 14$.\\
Wielomian jest stopnia parzystego, ponadto znak współczynnika przy\linebreak najwyższej potędze x jest ujemny.\\ W związku z tym wykres wielomianu zaczyna się od lewej strony powyżej osi OX.\\
Ponadto w punkcie $-1$ wykres odbija się od osi poziomej.\\
A więc $$x \in \{-1\} \cup [7,14].$$
\rozwStop
\odpStart
$x \in \{-1\} \cup [7,14]$
\odpStop
\testStart
A.$x \in \{-1\} \cup [7,14]$\\
B.$x \in \{1\} \cup (7,14)$\\
C.$x \in \{-1\} \cup (7,14]$\\
D.$x \in \{1\} \cup (7,14]$\\
E.$x \in \{-1\} \cup [7,14)$\\
F.$x \in \{1\} \cup [7,14)$\\
G.$x \in \{-1\} \cup (7,14)$\\
H.$x \in \{1\} \cup [7,14]$
\testStop
\kluczStart
A
\kluczStop



\zadStart{Zadanie z Wikieł Z 1.62 c) moja wersja nr 163}

Rozwiązać nierówności $(7-x)(x+1)^{2}(15-x)^{3}\le0$.
\zadStop
\rozwStart{Patryk Wirkus}{Laura Mieczkowska}
Miejsca zerowe naszego wielomianu to: $7, -1, 15$.\\
Wielomian jest stopnia parzystego, ponadto znak współczynnika przy\linebreak najwyższej potędze x jest ujemny.\\ W związku z tym wykres wielomianu zaczyna się od lewej strony powyżej osi OX.\\
Ponadto w punkcie $-1$ wykres odbija się od osi poziomej.\\
A więc $$x \in \{-1\} \cup [7,15].$$
\rozwStop
\odpStart
$x \in \{-1\} \cup [7,15]$
\odpStop
\testStart
A.$x \in \{-1\} \cup [7,15]$\\
B.$x \in \{1\} \cup (7,15)$\\
C.$x \in \{-1\} \cup (7,15]$\\
D.$x \in \{1\} \cup (7,15]$\\
E.$x \in \{-1\} \cup [7,15)$\\
F.$x \in \{1\} \cup [7,15)$\\
G.$x \in \{-1\} \cup (7,15)$\\
H.$x \in \{1\} \cup [7,15]$
\testStop
\kluczStart
A
\kluczStop



\zadStart{Zadanie z Wikieł Z 1.62 c) moja wersja nr 164}

Rozwiązać nierówności $(7-x)(x+2)^{2}(8-x)^{3}\le0$.
\zadStop
\rozwStart{Patryk Wirkus}{Laura Mieczkowska}
Miejsca zerowe naszego wielomianu to: $7, -2, 8$.\\
Wielomian jest stopnia parzystego, ponadto znak współczynnika przy\linebreak najwyższej potędze x jest ujemny.\\ W związku z tym wykres wielomianu zaczyna się od lewej strony powyżej osi OX.\\
Ponadto w punkcie $-2$ wykres odbija się od osi poziomej.\\
A więc $$x \in \{-2\} \cup [7,8].$$
\rozwStop
\odpStart
$x \in \{-2\} \cup [7,8]$
\odpStop
\testStart
A.$x \in \{-2\} \cup [7,8]$\\
B.$x \in \{2\} \cup (7,8)$\\
C.$x \in \{-2\} \cup (7,8]$\\
D.$x \in \{2\} \cup (7,8]$\\
E.$x \in \{-2\} \cup [7,8)$\\
F.$x \in \{2\} \cup [7,8)$\\
G.$x \in \{-2\} \cup (7,8)$\\
H.$x \in \{2\} \cup [7,8]$
\testStop
\kluczStart
A
\kluczStop



\zadStart{Zadanie z Wikieł Z 1.62 c) moja wersja nr 165}

Rozwiązać nierówności $(7-x)(x+2)^{2}(9-x)^{3}\le0$.
\zadStop
\rozwStart{Patryk Wirkus}{Laura Mieczkowska}
Miejsca zerowe naszego wielomianu to: $7, -2, 9$.\\
Wielomian jest stopnia parzystego, ponadto znak współczynnika przy\linebreak najwyższej potędze x jest ujemny.\\ W związku z tym wykres wielomianu zaczyna się od lewej strony powyżej osi OX.\\
Ponadto w punkcie $-2$ wykres odbija się od osi poziomej.\\
A więc $$x \in \{-2\} \cup [7,9].$$
\rozwStop
\odpStart
$x \in \{-2\} \cup [7,9]$
\odpStop
\testStart
A.$x \in \{-2\} \cup [7,9]$\\
B.$x \in \{2\} \cup (7,9)$\\
C.$x \in \{-2\} \cup (7,9]$\\
D.$x \in \{2\} \cup (7,9]$\\
E.$x \in \{-2\} \cup [7,9)$\\
F.$x \in \{2\} \cup [7,9)$\\
G.$x \in \{-2\} \cup (7,9)$\\
H.$x \in \{2\} \cup [7,9]$
\testStop
\kluczStart
A
\kluczStop



\zadStart{Zadanie z Wikieł Z 1.62 c) moja wersja nr 166}

Rozwiązać nierówności $(7-x)(x+2)^{2}(10-x)^{3}\le0$.
\zadStop
\rozwStart{Patryk Wirkus}{Laura Mieczkowska}
Miejsca zerowe naszego wielomianu to: $7, -2, 10$.\\
Wielomian jest stopnia parzystego, ponadto znak współczynnika przy\linebreak najwyższej potędze x jest ujemny.\\ W związku z tym wykres wielomianu zaczyna się od lewej strony powyżej osi OX.\\
Ponadto w punkcie $-2$ wykres odbija się od osi poziomej.\\
A więc $$x \in \{-2\} \cup [7,10].$$
\rozwStop
\odpStart
$x \in \{-2\} \cup [7,10]$
\odpStop
\testStart
A.$x \in \{-2\} \cup [7,10]$\\
B.$x \in \{2\} \cup (7,10)$\\
C.$x \in \{-2\} \cup (7,10]$\\
D.$x \in \{2\} \cup (7,10]$\\
E.$x \in \{-2\} \cup [7,10)$\\
F.$x \in \{2\} \cup [7,10)$\\
G.$x \in \{-2\} \cup (7,10)$\\
H.$x \in \{2\} \cup [7,10]$
\testStop
\kluczStart
A
\kluczStop



\zadStart{Zadanie z Wikieł Z 1.62 c) moja wersja nr 167}

Rozwiązać nierówności $(7-x)(x+2)^{2}(11-x)^{3}\le0$.
\zadStop
\rozwStart{Patryk Wirkus}{Laura Mieczkowska}
Miejsca zerowe naszego wielomianu to: $7, -2, 11$.\\
Wielomian jest stopnia parzystego, ponadto znak współczynnika przy\linebreak najwyższej potędze x jest ujemny.\\ W związku z tym wykres wielomianu zaczyna się od lewej strony powyżej osi OX.\\
Ponadto w punkcie $-2$ wykres odbija się od osi poziomej.\\
A więc $$x \in \{-2\} \cup [7,11].$$
\rozwStop
\odpStart
$x \in \{-2\} \cup [7,11]$
\odpStop
\testStart
A.$x \in \{-2\} \cup [7,11]$\\
B.$x \in \{2\} \cup (7,11)$\\
C.$x \in \{-2\} \cup (7,11]$\\
D.$x \in \{2\} \cup (7,11]$\\
E.$x \in \{-2\} \cup [7,11)$\\
F.$x \in \{2\} \cup [7,11)$\\
G.$x \in \{-2\} \cup (7,11)$\\
H.$x \in \{2\} \cup [7,11]$
\testStop
\kluczStart
A
\kluczStop



\zadStart{Zadanie z Wikieł Z 1.62 c) moja wersja nr 168}

Rozwiązać nierówności $(7-x)(x+2)^{2}(12-x)^{3}\le0$.
\zadStop
\rozwStart{Patryk Wirkus}{Laura Mieczkowska}
Miejsca zerowe naszego wielomianu to: $7, -2, 12$.\\
Wielomian jest stopnia parzystego, ponadto znak współczynnika przy\linebreak najwyższej potędze x jest ujemny.\\ W związku z tym wykres wielomianu zaczyna się od lewej strony powyżej osi OX.\\
Ponadto w punkcie $-2$ wykres odbija się od osi poziomej.\\
A więc $$x \in \{-2\} \cup [7,12].$$
\rozwStop
\odpStart
$x \in \{-2\} \cup [7,12]$
\odpStop
\testStart
A.$x \in \{-2\} \cup [7,12]$\\
B.$x \in \{2\} \cup (7,12)$\\
C.$x \in \{-2\} \cup (7,12]$\\
D.$x \in \{2\} \cup (7,12]$\\
E.$x \in \{-2\} \cup [7,12)$\\
F.$x \in \{2\} \cup [7,12)$\\
G.$x \in \{-2\} \cup (7,12)$\\
H.$x \in \{2\} \cup [7,12]$
\testStop
\kluczStart
A
\kluczStop



\zadStart{Zadanie z Wikieł Z 1.62 c) moja wersja nr 169}

Rozwiązać nierówności $(7-x)(x+2)^{2}(13-x)^{3}\le0$.
\zadStop
\rozwStart{Patryk Wirkus}{Laura Mieczkowska}
Miejsca zerowe naszego wielomianu to: $7, -2, 13$.\\
Wielomian jest stopnia parzystego, ponadto znak współczynnika przy\linebreak najwyższej potędze x jest ujemny.\\ W związku z tym wykres wielomianu zaczyna się od lewej strony powyżej osi OX.\\
Ponadto w punkcie $-2$ wykres odbija się od osi poziomej.\\
A więc $$x \in \{-2\} \cup [7,13].$$
\rozwStop
\odpStart
$x \in \{-2\} \cup [7,13]$
\odpStop
\testStart
A.$x \in \{-2\} \cup [7,13]$\\
B.$x \in \{2\} \cup (7,13)$\\
C.$x \in \{-2\} \cup (7,13]$\\
D.$x \in \{2\} \cup (7,13]$\\
E.$x \in \{-2\} \cup [7,13)$\\
F.$x \in \{2\} \cup [7,13)$\\
G.$x \in \{-2\} \cup (7,13)$\\
H.$x \in \{2\} \cup [7,13]$
\testStop
\kluczStart
A
\kluczStop



\zadStart{Zadanie z Wikieł Z 1.62 c) moja wersja nr 170}

Rozwiązać nierówności $(7-x)(x+2)^{2}(14-x)^{3}\le0$.
\zadStop
\rozwStart{Patryk Wirkus}{Laura Mieczkowska}
Miejsca zerowe naszego wielomianu to: $7, -2, 14$.\\
Wielomian jest stopnia parzystego, ponadto znak współczynnika przy\linebreak najwyższej potędze x jest ujemny.\\ W związku z tym wykres wielomianu zaczyna się od lewej strony powyżej osi OX.\\
Ponadto w punkcie $-2$ wykres odbija się od osi poziomej.\\
A więc $$x \in \{-2\} \cup [7,14].$$
\rozwStop
\odpStart
$x \in \{-2\} \cup [7,14]$
\odpStop
\testStart
A.$x \in \{-2\} \cup [7,14]$\\
B.$x \in \{2\} \cup (7,14)$\\
C.$x \in \{-2\} \cup (7,14]$\\
D.$x \in \{2\} \cup (7,14]$\\
E.$x \in \{-2\} \cup [7,14)$\\
F.$x \in \{2\} \cup [7,14)$\\
G.$x \in \{-2\} \cup (7,14)$\\
H.$x \in \{2\} \cup [7,14]$
\testStop
\kluczStart
A
\kluczStop



\zadStart{Zadanie z Wikieł Z 1.62 c) moja wersja nr 171}

Rozwiązać nierówności $(7-x)(x+2)^{2}(15-x)^{3}\le0$.
\zadStop
\rozwStart{Patryk Wirkus}{Laura Mieczkowska}
Miejsca zerowe naszego wielomianu to: $7, -2, 15$.\\
Wielomian jest stopnia parzystego, ponadto znak współczynnika przy\linebreak najwyższej potędze x jest ujemny.\\ W związku z tym wykres wielomianu zaczyna się od lewej strony powyżej osi OX.\\
Ponadto w punkcie $-2$ wykres odbija się od osi poziomej.\\
A więc $$x \in \{-2\} \cup [7,15].$$
\rozwStop
\odpStart
$x \in \{-2\} \cup [7,15]$
\odpStop
\testStart
A.$x \in \{-2\} \cup [7,15]$\\
B.$x \in \{2\} \cup (7,15)$\\
C.$x \in \{-2\} \cup (7,15]$\\
D.$x \in \{2\} \cup (7,15]$\\
E.$x \in \{-2\} \cup [7,15)$\\
F.$x \in \{2\} \cup [7,15)$\\
G.$x \in \{-2\} \cup (7,15)$\\
H.$x \in \{2\} \cup [7,15]$
\testStop
\kluczStart
A
\kluczStop



\zadStart{Zadanie z Wikieł Z 1.62 c) moja wersja nr 172}

Rozwiązać nierówności $(7-x)(x+3)^{2}(8-x)^{3}\le0$.
\zadStop
\rozwStart{Patryk Wirkus}{Laura Mieczkowska}
Miejsca zerowe naszego wielomianu to: $7, -3, 8$.\\
Wielomian jest stopnia parzystego, ponadto znak współczynnika przy\linebreak najwyższej potędze x jest ujemny.\\ W związku z tym wykres wielomianu zaczyna się od lewej strony powyżej osi OX.\\
Ponadto w punkcie $-3$ wykres odbija się od osi poziomej.\\
A więc $$x \in \{-3\} \cup [7,8].$$
\rozwStop
\odpStart
$x \in \{-3\} \cup [7,8]$
\odpStop
\testStart
A.$x \in \{-3\} \cup [7,8]$\\
B.$x \in \{3\} \cup (7,8)$\\
C.$x \in \{-3\} \cup (7,8]$\\
D.$x \in \{3\} \cup (7,8]$\\
E.$x \in \{-3\} \cup [7,8)$\\
F.$x \in \{3\} \cup [7,8)$\\
G.$x \in \{-3\} \cup (7,8)$\\
H.$x \in \{3\} \cup [7,8]$
\testStop
\kluczStart
A
\kluczStop



\zadStart{Zadanie z Wikieł Z 1.62 c) moja wersja nr 173}

Rozwiązać nierówności $(7-x)(x+3)^{2}(9-x)^{3}\le0$.
\zadStop
\rozwStart{Patryk Wirkus}{Laura Mieczkowska}
Miejsca zerowe naszego wielomianu to: $7, -3, 9$.\\
Wielomian jest stopnia parzystego, ponadto znak współczynnika przy\linebreak najwyższej potędze x jest ujemny.\\ W związku z tym wykres wielomianu zaczyna się od lewej strony powyżej osi OX.\\
Ponadto w punkcie $-3$ wykres odbija się od osi poziomej.\\
A więc $$x \in \{-3\} \cup [7,9].$$
\rozwStop
\odpStart
$x \in \{-3\} \cup [7,9]$
\odpStop
\testStart
A.$x \in \{-3\} \cup [7,9]$\\
B.$x \in \{3\} \cup (7,9)$\\
C.$x \in \{-3\} \cup (7,9]$\\
D.$x \in \{3\} \cup (7,9]$\\
E.$x \in \{-3\} \cup [7,9)$\\
F.$x \in \{3\} \cup [7,9)$\\
G.$x \in \{-3\} \cup (7,9)$\\
H.$x \in \{3\} \cup [7,9]$
\testStop
\kluczStart
A
\kluczStop



\zadStart{Zadanie z Wikieł Z 1.62 c) moja wersja nr 174}

Rozwiązać nierówności $(7-x)(x+3)^{2}(10-x)^{3}\le0$.
\zadStop
\rozwStart{Patryk Wirkus}{Laura Mieczkowska}
Miejsca zerowe naszego wielomianu to: $7, -3, 10$.\\
Wielomian jest stopnia parzystego, ponadto znak współczynnika przy\linebreak najwyższej potędze x jest ujemny.\\ W związku z tym wykres wielomianu zaczyna się od lewej strony powyżej osi OX.\\
Ponadto w punkcie $-3$ wykres odbija się od osi poziomej.\\
A więc $$x \in \{-3\} \cup [7,10].$$
\rozwStop
\odpStart
$x \in \{-3\} \cup [7,10]$
\odpStop
\testStart
A.$x \in \{-3\} \cup [7,10]$\\
B.$x \in \{3\} \cup (7,10)$\\
C.$x \in \{-3\} \cup (7,10]$\\
D.$x \in \{3\} \cup (7,10]$\\
E.$x \in \{-3\} \cup [7,10)$\\
F.$x \in \{3\} \cup [7,10)$\\
G.$x \in \{-3\} \cup (7,10)$\\
H.$x \in \{3\} \cup [7,10]$
\testStop
\kluczStart
A
\kluczStop



\zadStart{Zadanie z Wikieł Z 1.62 c) moja wersja nr 175}

Rozwiązać nierówności $(7-x)(x+3)^{2}(11-x)^{3}\le0$.
\zadStop
\rozwStart{Patryk Wirkus}{Laura Mieczkowska}
Miejsca zerowe naszego wielomianu to: $7, -3, 11$.\\
Wielomian jest stopnia parzystego, ponadto znak współczynnika przy\linebreak najwyższej potędze x jest ujemny.\\ W związku z tym wykres wielomianu zaczyna się od lewej strony powyżej osi OX.\\
Ponadto w punkcie $-3$ wykres odbija się od osi poziomej.\\
A więc $$x \in \{-3\} \cup [7,11].$$
\rozwStop
\odpStart
$x \in \{-3\} \cup [7,11]$
\odpStop
\testStart
A.$x \in \{-3\} \cup [7,11]$\\
B.$x \in \{3\} \cup (7,11)$\\
C.$x \in \{-3\} \cup (7,11]$\\
D.$x \in \{3\} \cup (7,11]$\\
E.$x \in \{-3\} \cup [7,11)$\\
F.$x \in \{3\} \cup [7,11)$\\
G.$x \in \{-3\} \cup (7,11)$\\
H.$x \in \{3\} \cup [7,11]$
\testStop
\kluczStart
A
\kluczStop



\zadStart{Zadanie z Wikieł Z 1.62 c) moja wersja nr 176}

Rozwiązać nierówności $(7-x)(x+3)^{2}(12-x)^{3}\le0$.
\zadStop
\rozwStart{Patryk Wirkus}{Laura Mieczkowska}
Miejsca zerowe naszego wielomianu to: $7, -3, 12$.\\
Wielomian jest stopnia parzystego, ponadto znak współczynnika przy\linebreak najwyższej potędze x jest ujemny.\\ W związku z tym wykres wielomianu zaczyna się od lewej strony powyżej osi OX.\\
Ponadto w punkcie $-3$ wykres odbija się od osi poziomej.\\
A więc $$x \in \{-3\} \cup [7,12].$$
\rozwStop
\odpStart
$x \in \{-3\} \cup [7,12]$
\odpStop
\testStart
A.$x \in \{-3\} \cup [7,12]$\\
B.$x \in \{3\} \cup (7,12)$\\
C.$x \in \{-3\} \cup (7,12]$\\
D.$x \in \{3\} \cup (7,12]$\\
E.$x \in \{-3\} \cup [7,12)$\\
F.$x \in \{3\} \cup [7,12)$\\
G.$x \in \{-3\} \cup (7,12)$\\
H.$x \in \{3\} \cup [7,12]$
\testStop
\kluczStart
A
\kluczStop



\zadStart{Zadanie z Wikieł Z 1.62 c) moja wersja nr 177}

Rozwiązać nierówności $(7-x)(x+3)^{2}(13-x)^{3}\le0$.
\zadStop
\rozwStart{Patryk Wirkus}{Laura Mieczkowska}
Miejsca zerowe naszego wielomianu to: $7, -3, 13$.\\
Wielomian jest stopnia parzystego, ponadto znak współczynnika przy\linebreak najwyższej potędze x jest ujemny.\\ W związku z tym wykres wielomianu zaczyna się od lewej strony powyżej osi OX.\\
Ponadto w punkcie $-3$ wykres odbija się od osi poziomej.\\
A więc $$x \in \{-3\} \cup [7,13].$$
\rozwStop
\odpStart
$x \in \{-3\} \cup [7,13]$
\odpStop
\testStart
A.$x \in \{-3\} \cup [7,13]$\\
B.$x \in \{3\} \cup (7,13)$\\
C.$x \in \{-3\} \cup (7,13]$\\
D.$x \in \{3\} \cup (7,13]$\\
E.$x \in \{-3\} \cup [7,13)$\\
F.$x \in \{3\} \cup [7,13)$\\
G.$x \in \{-3\} \cup (7,13)$\\
H.$x \in \{3\} \cup [7,13]$
\testStop
\kluczStart
A
\kluczStop



\zadStart{Zadanie z Wikieł Z 1.62 c) moja wersja nr 178}

Rozwiązać nierówności $(7-x)(x+3)^{2}(14-x)^{3}\le0$.
\zadStop
\rozwStart{Patryk Wirkus}{Laura Mieczkowska}
Miejsca zerowe naszego wielomianu to: $7, -3, 14$.\\
Wielomian jest stopnia parzystego, ponadto znak współczynnika przy\linebreak najwyższej potędze x jest ujemny.\\ W związku z tym wykres wielomianu zaczyna się od lewej strony powyżej osi OX.\\
Ponadto w punkcie $-3$ wykres odbija się od osi poziomej.\\
A więc $$x \in \{-3\} \cup [7,14].$$
\rozwStop
\odpStart
$x \in \{-3\} \cup [7,14]$
\odpStop
\testStart
A.$x \in \{-3\} \cup [7,14]$\\
B.$x \in \{3\} \cup (7,14)$\\
C.$x \in \{-3\} \cup (7,14]$\\
D.$x \in \{3\} \cup (7,14]$\\
E.$x \in \{-3\} \cup [7,14)$\\
F.$x \in \{3\} \cup [7,14)$\\
G.$x \in \{-3\} \cup (7,14)$\\
H.$x \in \{3\} \cup [7,14]$
\testStop
\kluczStart
A
\kluczStop



\zadStart{Zadanie z Wikieł Z 1.62 c) moja wersja nr 179}

Rozwiązać nierówności $(7-x)(x+3)^{2}(15-x)^{3}\le0$.
\zadStop
\rozwStart{Patryk Wirkus}{Laura Mieczkowska}
Miejsca zerowe naszego wielomianu to: $7, -3, 15$.\\
Wielomian jest stopnia parzystego, ponadto znak współczynnika przy\linebreak najwyższej potędze x jest ujemny.\\ W związku z tym wykres wielomianu zaczyna się od lewej strony powyżej osi OX.\\
Ponadto w punkcie $-3$ wykres odbija się od osi poziomej.\\
A więc $$x \in \{-3\} \cup [7,15].$$
\rozwStop
\odpStart
$x \in \{-3\} \cup [7,15]$
\odpStop
\testStart
A.$x \in \{-3\} \cup [7,15]$\\
B.$x \in \{3\} \cup (7,15)$\\
C.$x \in \{-3\} \cup (7,15]$\\
D.$x \in \{3\} \cup (7,15]$\\
E.$x \in \{-3\} \cup [7,15)$\\
F.$x \in \{3\} \cup [7,15)$\\
G.$x \in \{-3\} \cup (7,15)$\\
H.$x \in \{3\} \cup [7,15]$
\testStop
\kluczStart
A
\kluczStop



\zadStart{Zadanie z Wikieł Z 1.62 c) moja wersja nr 180}

Rozwiązać nierówności $(7-x)(x+4)^{2}(8-x)^{3}\le0$.
\zadStop
\rozwStart{Patryk Wirkus}{Laura Mieczkowska}
Miejsca zerowe naszego wielomianu to: $7, -4, 8$.\\
Wielomian jest stopnia parzystego, ponadto znak współczynnika przy\linebreak najwyższej potędze x jest ujemny.\\ W związku z tym wykres wielomianu zaczyna się od lewej strony powyżej osi OX.\\
Ponadto w punkcie $-4$ wykres odbija się od osi poziomej.\\
A więc $$x \in \{-4\} \cup [7,8].$$
\rozwStop
\odpStart
$x \in \{-4\} \cup [7,8]$
\odpStop
\testStart
A.$x \in \{-4\} \cup [7,8]$\\
B.$x \in \{4\} \cup (7,8)$\\
C.$x \in \{-4\} \cup (7,8]$\\
D.$x \in \{4\} \cup (7,8]$\\
E.$x \in \{-4\} \cup [7,8)$\\
F.$x \in \{4\} \cup [7,8)$\\
G.$x \in \{-4\} \cup (7,8)$\\
H.$x \in \{4\} \cup [7,8]$
\testStop
\kluczStart
A
\kluczStop



\zadStart{Zadanie z Wikieł Z 1.62 c) moja wersja nr 181}

Rozwiązać nierówności $(7-x)(x+4)^{2}(9-x)^{3}\le0$.
\zadStop
\rozwStart{Patryk Wirkus}{Laura Mieczkowska}
Miejsca zerowe naszego wielomianu to: $7, -4, 9$.\\
Wielomian jest stopnia parzystego, ponadto znak współczynnika przy\linebreak najwyższej potędze x jest ujemny.\\ W związku z tym wykres wielomianu zaczyna się od lewej strony powyżej osi OX.\\
Ponadto w punkcie $-4$ wykres odbija się od osi poziomej.\\
A więc $$x \in \{-4\} \cup [7,9].$$
\rozwStop
\odpStart
$x \in \{-4\} \cup [7,9]$
\odpStop
\testStart
A.$x \in \{-4\} \cup [7,9]$\\
B.$x \in \{4\} \cup (7,9)$\\
C.$x \in \{-4\} \cup (7,9]$\\
D.$x \in \{4\} \cup (7,9]$\\
E.$x \in \{-4\} \cup [7,9)$\\
F.$x \in \{4\} \cup [7,9)$\\
G.$x \in \{-4\} \cup (7,9)$\\
H.$x \in \{4\} \cup [7,9]$
\testStop
\kluczStart
A
\kluczStop



\zadStart{Zadanie z Wikieł Z 1.62 c) moja wersja nr 182}

Rozwiązać nierówności $(7-x)(x+4)^{2}(10-x)^{3}\le0$.
\zadStop
\rozwStart{Patryk Wirkus}{Laura Mieczkowska}
Miejsca zerowe naszego wielomianu to: $7, -4, 10$.\\
Wielomian jest stopnia parzystego, ponadto znak współczynnika przy\linebreak najwyższej potędze x jest ujemny.\\ W związku z tym wykres wielomianu zaczyna się od lewej strony powyżej osi OX.\\
Ponadto w punkcie $-4$ wykres odbija się od osi poziomej.\\
A więc $$x \in \{-4\} \cup [7,10].$$
\rozwStop
\odpStart
$x \in \{-4\} \cup [7,10]$
\odpStop
\testStart
A.$x \in \{-4\} \cup [7,10]$\\
B.$x \in \{4\} \cup (7,10)$\\
C.$x \in \{-4\} \cup (7,10]$\\
D.$x \in \{4\} \cup (7,10]$\\
E.$x \in \{-4\} \cup [7,10)$\\
F.$x \in \{4\} \cup [7,10)$\\
G.$x \in \{-4\} \cup (7,10)$\\
H.$x \in \{4\} \cup [7,10]$
\testStop
\kluczStart
A
\kluczStop



\zadStart{Zadanie z Wikieł Z 1.62 c) moja wersja nr 183}

Rozwiązać nierówności $(7-x)(x+4)^{2}(11-x)^{3}\le0$.
\zadStop
\rozwStart{Patryk Wirkus}{Laura Mieczkowska}
Miejsca zerowe naszego wielomianu to: $7, -4, 11$.\\
Wielomian jest stopnia parzystego, ponadto znak współczynnika przy\linebreak najwyższej potędze x jest ujemny.\\ W związku z tym wykres wielomianu zaczyna się od lewej strony powyżej osi OX.\\
Ponadto w punkcie $-4$ wykres odbija się od osi poziomej.\\
A więc $$x \in \{-4\} \cup [7,11].$$
\rozwStop
\odpStart
$x \in \{-4\} \cup [7,11]$
\odpStop
\testStart
A.$x \in \{-4\} \cup [7,11]$\\
B.$x \in \{4\} \cup (7,11)$\\
C.$x \in \{-4\} \cup (7,11]$\\
D.$x \in \{4\} \cup (7,11]$\\
E.$x \in \{-4\} \cup [7,11)$\\
F.$x \in \{4\} \cup [7,11)$\\
G.$x \in \{-4\} \cup (7,11)$\\
H.$x \in \{4\} \cup [7,11]$
\testStop
\kluczStart
A
\kluczStop



\zadStart{Zadanie z Wikieł Z 1.62 c) moja wersja nr 184}

Rozwiązać nierówności $(7-x)(x+4)^{2}(12-x)^{3}\le0$.
\zadStop
\rozwStart{Patryk Wirkus}{Laura Mieczkowska}
Miejsca zerowe naszego wielomianu to: $7, -4, 12$.\\
Wielomian jest stopnia parzystego, ponadto znak współczynnika przy\linebreak najwyższej potędze x jest ujemny.\\ W związku z tym wykres wielomianu zaczyna się od lewej strony powyżej osi OX.\\
Ponadto w punkcie $-4$ wykres odbija się od osi poziomej.\\
A więc $$x \in \{-4\} \cup [7,12].$$
\rozwStop
\odpStart
$x \in \{-4\} \cup [7,12]$
\odpStop
\testStart
A.$x \in \{-4\} \cup [7,12]$\\
B.$x \in \{4\} \cup (7,12)$\\
C.$x \in \{-4\} \cup (7,12]$\\
D.$x \in \{4\} \cup (7,12]$\\
E.$x \in \{-4\} \cup [7,12)$\\
F.$x \in \{4\} \cup [7,12)$\\
G.$x \in \{-4\} \cup (7,12)$\\
H.$x \in \{4\} \cup [7,12]$
\testStop
\kluczStart
A
\kluczStop



\zadStart{Zadanie z Wikieł Z 1.62 c) moja wersja nr 185}

Rozwiązać nierówności $(7-x)(x+4)^{2}(13-x)^{3}\le0$.
\zadStop
\rozwStart{Patryk Wirkus}{Laura Mieczkowska}
Miejsca zerowe naszego wielomianu to: $7, -4, 13$.\\
Wielomian jest stopnia parzystego, ponadto znak współczynnika przy\linebreak najwyższej potędze x jest ujemny.\\ W związku z tym wykres wielomianu zaczyna się od lewej strony powyżej osi OX.\\
Ponadto w punkcie $-4$ wykres odbija się od osi poziomej.\\
A więc $$x \in \{-4\} \cup [7,13].$$
\rozwStop
\odpStart
$x \in \{-4\} \cup [7,13]$
\odpStop
\testStart
A.$x \in \{-4\} \cup [7,13]$\\
B.$x \in \{4\} \cup (7,13)$\\
C.$x \in \{-4\} \cup (7,13]$\\
D.$x \in \{4\} \cup (7,13]$\\
E.$x \in \{-4\} \cup [7,13)$\\
F.$x \in \{4\} \cup [7,13)$\\
G.$x \in \{-4\} \cup (7,13)$\\
H.$x \in \{4\} \cup [7,13]$
\testStop
\kluczStart
A
\kluczStop



\zadStart{Zadanie z Wikieł Z 1.62 c) moja wersja nr 186}

Rozwiązać nierówności $(7-x)(x+4)^{2}(14-x)^{3}\le0$.
\zadStop
\rozwStart{Patryk Wirkus}{Laura Mieczkowska}
Miejsca zerowe naszego wielomianu to: $7, -4, 14$.\\
Wielomian jest stopnia parzystego, ponadto znak współczynnika przy\linebreak najwyższej potędze x jest ujemny.\\ W związku z tym wykres wielomianu zaczyna się od lewej strony powyżej osi OX.\\
Ponadto w punkcie $-4$ wykres odbija się od osi poziomej.\\
A więc $$x \in \{-4\} \cup [7,14].$$
\rozwStop
\odpStart
$x \in \{-4\} \cup [7,14]$
\odpStop
\testStart
A.$x \in \{-4\} \cup [7,14]$\\
B.$x \in \{4\} \cup (7,14)$\\
C.$x \in \{-4\} \cup (7,14]$\\
D.$x \in \{4\} \cup (7,14]$\\
E.$x \in \{-4\} \cup [7,14)$\\
F.$x \in \{4\} \cup [7,14)$\\
G.$x \in \{-4\} \cup (7,14)$\\
H.$x \in \{4\} \cup [7,14]$
\testStop
\kluczStart
A
\kluczStop



\zadStart{Zadanie z Wikieł Z 1.62 c) moja wersja nr 187}

Rozwiązać nierówności $(7-x)(x+4)^{2}(15-x)^{3}\le0$.
\zadStop
\rozwStart{Patryk Wirkus}{Laura Mieczkowska}
Miejsca zerowe naszego wielomianu to: $7, -4, 15$.\\
Wielomian jest stopnia parzystego, ponadto znak współczynnika przy\linebreak najwyższej potędze x jest ujemny.\\ W związku z tym wykres wielomianu zaczyna się od lewej strony powyżej osi OX.\\
Ponadto w punkcie $-4$ wykres odbija się od osi poziomej.\\
A więc $$x \in \{-4\} \cup [7,15].$$
\rozwStop
\odpStart
$x \in \{-4\} \cup [7,15]$
\odpStop
\testStart
A.$x \in \{-4\} \cup [7,15]$\\
B.$x \in \{4\} \cup (7,15)$\\
C.$x \in \{-4\} \cup (7,15]$\\
D.$x \in \{4\} \cup (7,15]$\\
E.$x \in \{-4\} \cup [7,15)$\\
F.$x \in \{4\} \cup [7,15)$\\
G.$x \in \{-4\} \cup (7,15)$\\
H.$x \in \{4\} \cup [7,15]$
\testStop
\kluczStart
A
\kluczStop



\zadStart{Zadanie z Wikieł Z 1.62 c) moja wersja nr 188}

Rozwiązać nierówności $(7-x)(x+5)^{2}(8-x)^{3}\le0$.
\zadStop
\rozwStart{Patryk Wirkus}{Laura Mieczkowska}
Miejsca zerowe naszego wielomianu to: $7, -5, 8$.\\
Wielomian jest stopnia parzystego, ponadto znak współczynnika przy\linebreak najwyższej potędze x jest ujemny.\\ W związku z tym wykres wielomianu zaczyna się od lewej strony powyżej osi OX.\\
Ponadto w punkcie $-5$ wykres odbija się od osi poziomej.\\
A więc $$x \in \{-5\} \cup [7,8].$$
\rozwStop
\odpStart
$x \in \{-5\} \cup [7,8]$
\odpStop
\testStart
A.$x \in \{-5\} \cup [7,8]$\\
B.$x \in \{5\} \cup (7,8)$\\
C.$x \in \{-5\} \cup (7,8]$\\
D.$x \in \{5\} \cup (7,8]$\\
E.$x \in \{-5\} \cup [7,8)$\\
F.$x \in \{5\} \cup [7,8)$\\
G.$x \in \{-5\} \cup (7,8)$\\
H.$x \in \{5\} \cup [7,8]$
\testStop
\kluczStart
A
\kluczStop



\zadStart{Zadanie z Wikieł Z 1.62 c) moja wersja nr 189}

Rozwiązać nierówności $(7-x)(x+5)^{2}(9-x)^{3}\le0$.
\zadStop
\rozwStart{Patryk Wirkus}{Laura Mieczkowska}
Miejsca zerowe naszego wielomianu to: $7, -5, 9$.\\
Wielomian jest stopnia parzystego, ponadto znak współczynnika przy\linebreak najwyższej potędze x jest ujemny.\\ W związku z tym wykres wielomianu zaczyna się od lewej strony powyżej osi OX.\\
Ponadto w punkcie $-5$ wykres odbija się od osi poziomej.\\
A więc $$x \in \{-5\} \cup [7,9].$$
\rozwStop
\odpStart
$x \in \{-5\} \cup [7,9]$
\odpStop
\testStart
A.$x \in \{-5\} \cup [7,9]$\\
B.$x \in \{5\} \cup (7,9)$\\
C.$x \in \{-5\} \cup (7,9]$\\
D.$x \in \{5\} \cup (7,9]$\\
E.$x \in \{-5\} \cup [7,9)$\\
F.$x \in \{5\} \cup [7,9)$\\
G.$x \in \{-5\} \cup (7,9)$\\
H.$x \in \{5\} \cup [7,9]$
\testStop
\kluczStart
A
\kluczStop



\zadStart{Zadanie z Wikieł Z 1.62 c) moja wersja nr 190}

Rozwiązać nierówności $(7-x)(x+5)^{2}(10-x)^{3}\le0$.
\zadStop
\rozwStart{Patryk Wirkus}{Laura Mieczkowska}
Miejsca zerowe naszego wielomianu to: $7, -5, 10$.\\
Wielomian jest stopnia parzystego, ponadto znak współczynnika przy\linebreak najwyższej potędze x jest ujemny.\\ W związku z tym wykres wielomianu zaczyna się od lewej strony powyżej osi OX.\\
Ponadto w punkcie $-5$ wykres odbija się od osi poziomej.\\
A więc $$x \in \{-5\} \cup [7,10].$$
\rozwStop
\odpStart
$x \in \{-5\} \cup [7,10]$
\odpStop
\testStart
A.$x \in \{-5\} \cup [7,10]$\\
B.$x \in \{5\} \cup (7,10)$\\
C.$x \in \{-5\} \cup (7,10]$\\
D.$x \in \{5\} \cup (7,10]$\\
E.$x \in \{-5\} \cup [7,10)$\\
F.$x \in \{5\} \cup [7,10)$\\
G.$x \in \{-5\} \cup (7,10)$\\
H.$x \in \{5\} \cup [7,10]$
\testStop
\kluczStart
A
\kluczStop



\zadStart{Zadanie z Wikieł Z 1.62 c) moja wersja nr 191}

Rozwiązać nierówności $(7-x)(x+5)^{2}(11-x)^{3}\le0$.
\zadStop
\rozwStart{Patryk Wirkus}{Laura Mieczkowska}
Miejsca zerowe naszego wielomianu to: $7, -5, 11$.\\
Wielomian jest stopnia parzystego, ponadto znak współczynnika przy\linebreak najwyższej potędze x jest ujemny.\\ W związku z tym wykres wielomianu zaczyna się od lewej strony powyżej osi OX.\\
Ponadto w punkcie $-5$ wykres odbija się od osi poziomej.\\
A więc $$x \in \{-5\} \cup [7,11].$$
\rozwStop
\odpStart
$x \in \{-5\} \cup [7,11]$
\odpStop
\testStart
A.$x \in \{-5\} \cup [7,11]$\\
B.$x \in \{5\} \cup (7,11)$\\
C.$x \in \{-5\} \cup (7,11]$\\
D.$x \in \{5\} \cup (7,11]$\\
E.$x \in \{-5\} \cup [7,11)$\\
F.$x \in \{5\} \cup [7,11)$\\
G.$x \in \{-5\} \cup (7,11)$\\
H.$x \in \{5\} \cup [7,11]$
\testStop
\kluczStart
A
\kluczStop



\zadStart{Zadanie z Wikieł Z 1.62 c) moja wersja nr 192}

Rozwiązać nierówności $(7-x)(x+5)^{2}(12-x)^{3}\le0$.
\zadStop
\rozwStart{Patryk Wirkus}{Laura Mieczkowska}
Miejsca zerowe naszego wielomianu to: $7, -5, 12$.\\
Wielomian jest stopnia parzystego, ponadto znak współczynnika przy\linebreak najwyższej potędze x jest ujemny.\\ W związku z tym wykres wielomianu zaczyna się od lewej strony powyżej osi OX.\\
Ponadto w punkcie $-5$ wykres odbija się od osi poziomej.\\
A więc $$x \in \{-5\} \cup [7,12].$$
\rozwStop
\odpStart
$x \in \{-5\} \cup [7,12]$
\odpStop
\testStart
A.$x \in \{-5\} \cup [7,12]$\\
B.$x \in \{5\} \cup (7,12)$\\
C.$x \in \{-5\} \cup (7,12]$\\
D.$x \in \{5\} \cup (7,12]$\\
E.$x \in \{-5\} \cup [7,12)$\\
F.$x \in \{5\} \cup [7,12)$\\
G.$x \in \{-5\} \cup (7,12)$\\
H.$x \in \{5\} \cup [7,12]$
\testStop
\kluczStart
A
\kluczStop



\zadStart{Zadanie z Wikieł Z 1.62 c) moja wersja nr 193}

Rozwiązać nierówności $(7-x)(x+5)^{2}(13-x)^{3}\le0$.
\zadStop
\rozwStart{Patryk Wirkus}{Laura Mieczkowska}
Miejsca zerowe naszego wielomianu to: $7, -5, 13$.\\
Wielomian jest stopnia parzystego, ponadto znak współczynnika przy\linebreak najwyższej potędze x jest ujemny.\\ W związku z tym wykres wielomianu zaczyna się od lewej strony powyżej osi OX.\\
Ponadto w punkcie $-5$ wykres odbija się od osi poziomej.\\
A więc $$x \in \{-5\} \cup [7,13].$$
\rozwStop
\odpStart
$x \in \{-5\} \cup [7,13]$
\odpStop
\testStart
A.$x \in \{-5\} \cup [7,13]$\\
B.$x \in \{5\} \cup (7,13)$\\
C.$x \in \{-5\} \cup (7,13]$\\
D.$x \in \{5\} \cup (7,13]$\\
E.$x \in \{-5\} \cup [7,13)$\\
F.$x \in \{5\} \cup [7,13)$\\
G.$x \in \{-5\} \cup (7,13)$\\
H.$x \in \{5\} \cup [7,13]$
\testStop
\kluczStart
A
\kluczStop



\zadStart{Zadanie z Wikieł Z 1.62 c) moja wersja nr 194}

Rozwiązać nierówności $(7-x)(x+5)^{2}(14-x)^{3}\le0$.
\zadStop
\rozwStart{Patryk Wirkus}{Laura Mieczkowska}
Miejsca zerowe naszego wielomianu to: $7, -5, 14$.\\
Wielomian jest stopnia parzystego, ponadto znak współczynnika przy\linebreak najwyższej potędze x jest ujemny.\\ W związku z tym wykres wielomianu zaczyna się od lewej strony powyżej osi OX.\\
Ponadto w punkcie $-5$ wykres odbija się od osi poziomej.\\
A więc $$x \in \{-5\} \cup [7,14].$$
\rozwStop
\odpStart
$x \in \{-5\} \cup [7,14]$
\odpStop
\testStart
A.$x \in \{-5\} \cup [7,14]$\\
B.$x \in \{5\} \cup (7,14)$\\
C.$x \in \{-5\} \cup (7,14]$\\
D.$x \in \{5\} \cup (7,14]$\\
E.$x \in \{-5\} \cup [7,14)$\\
F.$x \in \{5\} \cup [7,14)$\\
G.$x \in \{-5\} \cup (7,14)$\\
H.$x \in \{5\} \cup [7,14]$
\testStop
\kluczStart
A
\kluczStop



\zadStart{Zadanie z Wikieł Z 1.62 c) moja wersja nr 195}

Rozwiązać nierówności $(7-x)(x+5)^{2}(15-x)^{3}\le0$.
\zadStop
\rozwStart{Patryk Wirkus}{Laura Mieczkowska}
Miejsca zerowe naszego wielomianu to: $7, -5, 15$.\\
Wielomian jest stopnia parzystego, ponadto znak współczynnika przy\linebreak najwyższej potędze x jest ujemny.\\ W związku z tym wykres wielomianu zaczyna się od lewej strony powyżej osi OX.\\
Ponadto w punkcie $-5$ wykres odbija się od osi poziomej.\\
A więc $$x \in \{-5\} \cup [7,15].$$
\rozwStop
\odpStart
$x \in \{-5\} \cup [7,15]$
\odpStop
\testStart
A.$x \in \{-5\} \cup [7,15]$\\
B.$x \in \{5\} \cup (7,15)$\\
C.$x \in \{-5\} \cup (7,15]$\\
D.$x \in \{5\} \cup (7,15]$\\
E.$x \in \{-5\} \cup [7,15)$\\
F.$x \in \{5\} \cup [7,15)$\\
G.$x \in \{-5\} \cup (7,15)$\\
H.$x \in \{5\} \cup [7,15]$
\testStop
\kluczStart
A
\kluczStop



\zadStart{Zadanie z Wikieł Z 1.62 c) moja wersja nr 196}

Rozwiązać nierówności $(8-x)(x+1)^{2}(9-x)^{3}\le0$.
\zadStop
\rozwStart{Patryk Wirkus}{Laura Mieczkowska}
Miejsca zerowe naszego wielomianu to: $8, -1, 9$.\\
Wielomian jest stopnia parzystego, ponadto znak współczynnika przy\linebreak najwyższej potędze x jest ujemny.\\ W związku z tym wykres wielomianu zaczyna się od lewej strony powyżej osi OX.\\
Ponadto w punkcie $-1$ wykres odbija się od osi poziomej.\\
A więc $$x \in \{-1\} \cup [8,9].$$
\rozwStop
\odpStart
$x \in \{-1\} \cup [8,9]$
\odpStop
\testStart
A.$x \in \{-1\} \cup [8,9]$\\
B.$x \in \{1\} \cup (8,9)$\\
C.$x \in \{-1\} \cup (8,9]$\\
D.$x \in \{1\} \cup (8,9]$\\
E.$x \in \{-1\} \cup [8,9)$\\
F.$x \in \{1\} \cup [8,9)$\\
G.$x \in \{-1\} \cup (8,9)$\\
H.$x \in \{1\} \cup [8,9]$
\testStop
\kluczStart
A
\kluczStop



\zadStart{Zadanie z Wikieł Z 1.62 c) moja wersja nr 197}

Rozwiązać nierówności $(8-x)(x+1)^{2}(10-x)^{3}\le0$.
\zadStop
\rozwStart{Patryk Wirkus}{Laura Mieczkowska}
Miejsca zerowe naszego wielomianu to: $8, -1, 10$.\\
Wielomian jest stopnia parzystego, ponadto znak współczynnika przy\linebreak najwyższej potędze x jest ujemny.\\ W związku z tym wykres wielomianu zaczyna się od lewej strony powyżej osi OX.\\
Ponadto w punkcie $-1$ wykres odbija się od osi poziomej.\\
A więc $$x \in \{-1\} \cup [8,10].$$
\rozwStop
\odpStart
$x \in \{-1\} \cup [8,10]$
\odpStop
\testStart
A.$x \in \{-1\} \cup [8,10]$\\
B.$x \in \{1\} \cup (8,10)$\\
C.$x \in \{-1\} \cup (8,10]$\\
D.$x \in \{1\} \cup (8,10]$\\
E.$x \in \{-1\} \cup [8,10)$\\
F.$x \in \{1\} \cup [8,10)$\\
G.$x \in \{-1\} \cup (8,10)$\\
H.$x \in \{1\} \cup [8,10]$
\testStop
\kluczStart
A
\kluczStop



\zadStart{Zadanie z Wikieł Z 1.62 c) moja wersja nr 198}

Rozwiązać nierówności $(8-x)(x+1)^{2}(11-x)^{3}\le0$.
\zadStop
\rozwStart{Patryk Wirkus}{Laura Mieczkowska}
Miejsca zerowe naszego wielomianu to: $8, -1, 11$.\\
Wielomian jest stopnia parzystego, ponadto znak współczynnika przy\linebreak najwyższej potędze x jest ujemny.\\ W związku z tym wykres wielomianu zaczyna się od lewej strony powyżej osi OX.\\
Ponadto w punkcie $-1$ wykres odbija się od osi poziomej.\\
A więc $$x \in \{-1\} \cup [8,11].$$
\rozwStop
\odpStart
$x \in \{-1\} \cup [8,11]$
\odpStop
\testStart
A.$x \in \{-1\} \cup [8,11]$\\
B.$x \in \{1\} \cup (8,11)$\\
C.$x \in \{-1\} \cup (8,11]$\\
D.$x \in \{1\} \cup (8,11]$\\
E.$x \in \{-1\} \cup [8,11)$\\
F.$x \in \{1\} \cup [8,11)$\\
G.$x \in \{-1\} \cup (8,11)$\\
H.$x \in \{1\} \cup [8,11]$
\testStop
\kluczStart
A
\kluczStop



\zadStart{Zadanie z Wikieł Z 1.62 c) moja wersja nr 199}

Rozwiązać nierówności $(8-x)(x+1)^{2}(12-x)^{3}\le0$.
\zadStop
\rozwStart{Patryk Wirkus}{Laura Mieczkowska}
Miejsca zerowe naszego wielomianu to: $8, -1, 12$.\\
Wielomian jest stopnia parzystego, ponadto znak współczynnika przy\linebreak najwyższej potędze x jest ujemny.\\ W związku z tym wykres wielomianu zaczyna się od lewej strony powyżej osi OX.\\
Ponadto w punkcie $-1$ wykres odbija się od osi poziomej.\\
A więc $$x \in \{-1\} \cup [8,12].$$
\rozwStop
\odpStart
$x \in \{-1\} \cup [8,12]$
\odpStop
\testStart
A.$x \in \{-1\} \cup [8,12]$\\
B.$x \in \{1\} \cup (8,12)$\\
C.$x \in \{-1\} \cup (8,12]$\\
D.$x \in \{1\} \cup (8,12]$\\
E.$x \in \{-1\} \cup [8,12)$\\
F.$x \in \{1\} \cup [8,12)$\\
G.$x \in \{-1\} \cup (8,12)$\\
H.$x \in \{1\} \cup [8,12]$
\testStop
\kluczStart
A
\kluczStop



\zadStart{Zadanie z Wikieł Z 1.62 c) moja wersja nr 200}

Rozwiązać nierówności $(8-x)(x+1)^{2}(13-x)^{3}\le0$.
\zadStop
\rozwStart{Patryk Wirkus}{Laura Mieczkowska}
Miejsca zerowe naszego wielomianu to: $8, -1, 13$.\\
Wielomian jest stopnia parzystego, ponadto znak współczynnika przy\linebreak najwyższej potędze x jest ujemny.\\ W związku z tym wykres wielomianu zaczyna się od lewej strony powyżej osi OX.\\
Ponadto w punkcie $-1$ wykres odbija się od osi poziomej.\\
A więc $$x \in \{-1\} \cup [8,13].$$
\rozwStop
\odpStart
$x \in \{-1\} \cup [8,13]$
\odpStop
\testStart
A.$x \in \{-1\} \cup [8,13]$\\
B.$x \in \{1\} \cup (8,13)$\\
C.$x \in \{-1\} \cup (8,13]$\\
D.$x \in \{1\} \cup (8,13]$\\
E.$x \in \{-1\} \cup [8,13)$\\
F.$x \in \{1\} \cup [8,13)$\\
G.$x \in \{-1\} \cup (8,13)$\\
H.$x \in \{1\} \cup [8,13]$
\testStop
\kluczStart
A
\kluczStop



\zadStart{Zadanie z Wikieł Z 1.62 c) moja wersja nr 201}

Rozwiązać nierówności $(8-x)(x+1)^{2}(14-x)^{3}\le0$.
\zadStop
\rozwStart{Patryk Wirkus}{Laura Mieczkowska}
Miejsca zerowe naszego wielomianu to: $8, -1, 14$.\\
Wielomian jest stopnia parzystego, ponadto znak współczynnika przy\linebreak najwyższej potędze x jest ujemny.\\ W związku z tym wykres wielomianu zaczyna się od lewej strony powyżej osi OX.\\
Ponadto w punkcie $-1$ wykres odbija się od osi poziomej.\\
A więc $$x \in \{-1\} \cup [8,14].$$
\rozwStop
\odpStart
$x \in \{-1\} \cup [8,14]$
\odpStop
\testStart
A.$x \in \{-1\} \cup [8,14]$\\
B.$x \in \{1\} \cup (8,14)$\\
C.$x \in \{-1\} \cup (8,14]$\\
D.$x \in \{1\} \cup (8,14]$\\
E.$x \in \{-1\} \cup [8,14)$\\
F.$x \in \{1\} \cup [8,14)$\\
G.$x \in \{-1\} \cup (8,14)$\\
H.$x \in \{1\} \cup [8,14]$
\testStop
\kluczStart
A
\kluczStop



\zadStart{Zadanie z Wikieł Z 1.62 c) moja wersja nr 202}

Rozwiązać nierówności $(8-x)(x+1)^{2}(15-x)^{3}\le0$.
\zadStop
\rozwStart{Patryk Wirkus}{Laura Mieczkowska}
Miejsca zerowe naszego wielomianu to: $8, -1, 15$.\\
Wielomian jest stopnia parzystego, ponadto znak współczynnika przy\linebreak najwyższej potędze x jest ujemny.\\ W związku z tym wykres wielomianu zaczyna się od lewej strony powyżej osi OX.\\
Ponadto w punkcie $-1$ wykres odbija się od osi poziomej.\\
A więc $$x \in \{-1\} \cup [8,15].$$
\rozwStop
\odpStart
$x \in \{-1\} \cup [8,15]$
\odpStop
\testStart
A.$x \in \{-1\} \cup [8,15]$\\
B.$x \in \{1\} \cup (8,15)$\\
C.$x \in \{-1\} \cup (8,15]$\\
D.$x \in \{1\} \cup (8,15]$\\
E.$x \in \{-1\} \cup [8,15)$\\
F.$x \in \{1\} \cup [8,15)$\\
G.$x \in \{-1\} \cup (8,15)$\\
H.$x \in \{1\} \cup [8,15]$
\testStop
\kluczStart
A
\kluczStop



\zadStart{Zadanie z Wikieł Z 1.62 c) moja wersja nr 203}

Rozwiązać nierówności $(8-x)(x+2)^{2}(9-x)^{3}\le0$.
\zadStop
\rozwStart{Patryk Wirkus}{Laura Mieczkowska}
Miejsca zerowe naszego wielomianu to: $8, -2, 9$.\\
Wielomian jest stopnia parzystego, ponadto znak współczynnika przy\linebreak najwyższej potędze x jest ujemny.\\ W związku z tym wykres wielomianu zaczyna się od lewej strony powyżej osi OX.\\
Ponadto w punkcie $-2$ wykres odbija się od osi poziomej.\\
A więc $$x \in \{-2\} \cup [8,9].$$
\rozwStop
\odpStart
$x \in \{-2\} \cup [8,9]$
\odpStop
\testStart
A.$x \in \{-2\} \cup [8,9]$\\
B.$x \in \{2\} \cup (8,9)$\\
C.$x \in \{-2\} \cup (8,9]$\\
D.$x \in \{2\} \cup (8,9]$\\
E.$x \in \{-2\} \cup [8,9)$\\
F.$x \in \{2\} \cup [8,9)$\\
G.$x \in \{-2\} \cup (8,9)$\\
H.$x \in \{2\} \cup [8,9]$
\testStop
\kluczStart
A
\kluczStop



\zadStart{Zadanie z Wikieł Z 1.62 c) moja wersja nr 204}

Rozwiązać nierówności $(8-x)(x+2)^{2}(10-x)^{3}\le0$.
\zadStop
\rozwStart{Patryk Wirkus}{Laura Mieczkowska}
Miejsca zerowe naszego wielomianu to: $8, -2, 10$.\\
Wielomian jest stopnia parzystego, ponadto znak współczynnika przy\linebreak najwyższej potędze x jest ujemny.\\ W związku z tym wykres wielomianu zaczyna się od lewej strony powyżej osi OX.\\
Ponadto w punkcie $-2$ wykres odbija się od osi poziomej.\\
A więc $$x \in \{-2\} \cup [8,10].$$
\rozwStop
\odpStart
$x \in \{-2\} \cup [8,10]$
\odpStop
\testStart
A.$x \in \{-2\} \cup [8,10]$\\
B.$x \in \{2\} \cup (8,10)$\\
C.$x \in \{-2\} \cup (8,10]$\\
D.$x \in \{2\} \cup (8,10]$\\
E.$x \in \{-2\} \cup [8,10)$\\
F.$x \in \{2\} \cup [8,10)$\\
G.$x \in \{-2\} \cup (8,10)$\\
H.$x \in \{2\} \cup [8,10]$
\testStop
\kluczStart
A
\kluczStop



\zadStart{Zadanie z Wikieł Z 1.62 c) moja wersja nr 205}

Rozwiązać nierówności $(8-x)(x+2)^{2}(11-x)^{3}\le0$.
\zadStop
\rozwStart{Patryk Wirkus}{Laura Mieczkowska}
Miejsca zerowe naszego wielomianu to: $8, -2, 11$.\\
Wielomian jest stopnia parzystego, ponadto znak współczynnika przy\linebreak najwyższej potędze x jest ujemny.\\ W związku z tym wykres wielomianu zaczyna się od lewej strony powyżej osi OX.\\
Ponadto w punkcie $-2$ wykres odbija się od osi poziomej.\\
A więc $$x \in \{-2\} \cup [8,11].$$
\rozwStop
\odpStart
$x \in \{-2\} \cup [8,11]$
\odpStop
\testStart
A.$x \in \{-2\} \cup [8,11]$\\
B.$x \in \{2\} \cup (8,11)$\\
C.$x \in \{-2\} \cup (8,11]$\\
D.$x \in \{2\} \cup (8,11]$\\
E.$x \in \{-2\} \cup [8,11)$\\
F.$x \in \{2\} \cup [8,11)$\\
G.$x \in \{-2\} \cup (8,11)$\\
H.$x \in \{2\} \cup [8,11]$
\testStop
\kluczStart
A
\kluczStop



\zadStart{Zadanie z Wikieł Z 1.62 c) moja wersja nr 206}

Rozwiązać nierówności $(8-x)(x+2)^{2}(12-x)^{3}\le0$.
\zadStop
\rozwStart{Patryk Wirkus}{Laura Mieczkowska}
Miejsca zerowe naszego wielomianu to: $8, -2, 12$.\\
Wielomian jest stopnia parzystego, ponadto znak współczynnika przy\linebreak najwyższej potędze x jest ujemny.\\ W związku z tym wykres wielomianu zaczyna się od lewej strony powyżej osi OX.\\
Ponadto w punkcie $-2$ wykres odbija się od osi poziomej.\\
A więc $$x \in \{-2\} \cup [8,12].$$
\rozwStop
\odpStart
$x \in \{-2\} \cup [8,12]$
\odpStop
\testStart
A.$x \in \{-2\} \cup [8,12]$\\
B.$x \in \{2\} \cup (8,12)$\\
C.$x \in \{-2\} \cup (8,12]$\\
D.$x \in \{2\} \cup (8,12]$\\
E.$x \in \{-2\} \cup [8,12)$\\
F.$x \in \{2\} \cup [8,12)$\\
G.$x \in \{-2\} \cup (8,12)$\\
H.$x \in \{2\} \cup [8,12]$
\testStop
\kluczStart
A
\kluczStop



\zadStart{Zadanie z Wikieł Z 1.62 c) moja wersja nr 207}

Rozwiązać nierówności $(8-x)(x+2)^{2}(13-x)^{3}\le0$.
\zadStop
\rozwStart{Patryk Wirkus}{Laura Mieczkowska}
Miejsca zerowe naszego wielomianu to: $8, -2, 13$.\\
Wielomian jest stopnia parzystego, ponadto znak współczynnika przy\linebreak najwyższej potędze x jest ujemny.\\ W związku z tym wykres wielomianu zaczyna się od lewej strony powyżej osi OX.\\
Ponadto w punkcie $-2$ wykres odbija się od osi poziomej.\\
A więc $$x \in \{-2\} \cup [8,13].$$
\rozwStop
\odpStart
$x \in \{-2\} \cup [8,13]$
\odpStop
\testStart
A.$x \in \{-2\} \cup [8,13]$\\
B.$x \in \{2\} \cup (8,13)$\\
C.$x \in \{-2\} \cup (8,13]$\\
D.$x \in \{2\} \cup (8,13]$\\
E.$x \in \{-2\} \cup [8,13)$\\
F.$x \in \{2\} \cup [8,13)$\\
G.$x \in \{-2\} \cup (8,13)$\\
H.$x \in \{2\} \cup [8,13]$
\testStop
\kluczStart
A
\kluczStop



\zadStart{Zadanie z Wikieł Z 1.62 c) moja wersja nr 208}

Rozwiązać nierówności $(8-x)(x+2)^{2}(14-x)^{3}\le0$.
\zadStop
\rozwStart{Patryk Wirkus}{Laura Mieczkowska}
Miejsca zerowe naszego wielomianu to: $8, -2, 14$.\\
Wielomian jest stopnia parzystego, ponadto znak współczynnika przy\linebreak najwyższej potędze x jest ujemny.\\ W związku z tym wykres wielomianu zaczyna się od lewej strony powyżej osi OX.\\
Ponadto w punkcie $-2$ wykres odbija się od osi poziomej.\\
A więc $$x \in \{-2\} \cup [8,14].$$
\rozwStop
\odpStart
$x \in \{-2\} \cup [8,14]$
\odpStop
\testStart
A.$x \in \{-2\} \cup [8,14]$\\
B.$x \in \{2\} \cup (8,14)$\\
C.$x \in \{-2\} \cup (8,14]$\\
D.$x \in \{2\} \cup (8,14]$\\
E.$x \in \{-2\} \cup [8,14)$\\
F.$x \in \{2\} \cup [8,14)$\\
G.$x \in \{-2\} \cup (8,14)$\\
H.$x \in \{2\} \cup [8,14]$
\testStop
\kluczStart
A
\kluczStop



\zadStart{Zadanie z Wikieł Z 1.62 c) moja wersja nr 209}

Rozwiązać nierówności $(8-x)(x+2)^{2}(15-x)^{3}\le0$.
\zadStop
\rozwStart{Patryk Wirkus}{Laura Mieczkowska}
Miejsca zerowe naszego wielomianu to: $8, -2, 15$.\\
Wielomian jest stopnia parzystego, ponadto znak współczynnika przy\linebreak najwyższej potędze x jest ujemny.\\ W związku z tym wykres wielomianu zaczyna się od lewej strony powyżej osi OX.\\
Ponadto w punkcie $-2$ wykres odbija się od osi poziomej.\\
A więc $$x \in \{-2\} \cup [8,15].$$
\rozwStop
\odpStart
$x \in \{-2\} \cup [8,15]$
\odpStop
\testStart
A.$x \in \{-2\} \cup [8,15]$\\
B.$x \in \{2\} \cup (8,15)$\\
C.$x \in \{-2\} \cup (8,15]$\\
D.$x \in \{2\} \cup (8,15]$\\
E.$x \in \{-2\} \cup [8,15)$\\
F.$x \in \{2\} \cup [8,15)$\\
G.$x \in \{-2\} \cup (8,15)$\\
H.$x \in \{2\} \cup [8,15]$
\testStop
\kluczStart
A
\kluczStop



\zadStart{Zadanie z Wikieł Z 1.62 c) moja wersja nr 210}

Rozwiązać nierówności $(8-x)(x+3)^{2}(9-x)^{3}\le0$.
\zadStop
\rozwStart{Patryk Wirkus}{Laura Mieczkowska}
Miejsca zerowe naszego wielomianu to: $8, -3, 9$.\\
Wielomian jest stopnia parzystego, ponadto znak współczynnika przy\linebreak najwyższej potędze x jest ujemny.\\ W związku z tym wykres wielomianu zaczyna się od lewej strony powyżej osi OX.\\
Ponadto w punkcie $-3$ wykres odbija się od osi poziomej.\\
A więc $$x \in \{-3\} \cup [8,9].$$
\rozwStop
\odpStart
$x \in \{-3\} \cup [8,9]$
\odpStop
\testStart
A.$x \in \{-3\} \cup [8,9]$\\
B.$x \in \{3\} \cup (8,9)$\\
C.$x \in \{-3\} \cup (8,9]$\\
D.$x \in \{3\} \cup (8,9]$\\
E.$x \in \{-3\} \cup [8,9)$\\
F.$x \in \{3\} \cup [8,9)$\\
G.$x \in \{-3\} \cup (8,9)$\\
H.$x \in \{3\} \cup [8,9]$
\testStop
\kluczStart
A
\kluczStop



\zadStart{Zadanie z Wikieł Z 1.62 c) moja wersja nr 211}

Rozwiązać nierówności $(8-x)(x+3)^{2}(10-x)^{3}\le0$.
\zadStop
\rozwStart{Patryk Wirkus}{Laura Mieczkowska}
Miejsca zerowe naszego wielomianu to: $8, -3, 10$.\\
Wielomian jest stopnia parzystego, ponadto znak współczynnika przy\linebreak najwyższej potędze x jest ujemny.\\ W związku z tym wykres wielomianu zaczyna się od lewej strony powyżej osi OX.\\
Ponadto w punkcie $-3$ wykres odbija się od osi poziomej.\\
A więc $$x \in \{-3\} \cup [8,10].$$
\rozwStop
\odpStart
$x \in \{-3\} \cup [8,10]$
\odpStop
\testStart
A.$x \in \{-3\} \cup [8,10]$\\
B.$x \in \{3\} \cup (8,10)$\\
C.$x \in \{-3\} \cup (8,10]$\\
D.$x \in \{3\} \cup (8,10]$\\
E.$x \in \{-3\} \cup [8,10)$\\
F.$x \in \{3\} \cup [8,10)$\\
G.$x \in \{-3\} \cup (8,10)$\\
H.$x \in \{3\} \cup [8,10]$
\testStop
\kluczStart
A
\kluczStop



\zadStart{Zadanie z Wikieł Z 1.62 c) moja wersja nr 212}

Rozwiązać nierówności $(8-x)(x+3)^{2}(11-x)^{3}\le0$.
\zadStop
\rozwStart{Patryk Wirkus}{Laura Mieczkowska}
Miejsca zerowe naszego wielomianu to: $8, -3, 11$.\\
Wielomian jest stopnia parzystego, ponadto znak współczynnika przy\linebreak najwyższej potędze x jest ujemny.\\ W związku z tym wykres wielomianu zaczyna się od lewej strony powyżej osi OX.\\
Ponadto w punkcie $-3$ wykres odbija się od osi poziomej.\\
A więc $$x \in \{-3\} \cup [8,11].$$
\rozwStop
\odpStart
$x \in \{-3\} \cup [8,11]$
\odpStop
\testStart
A.$x \in \{-3\} \cup [8,11]$\\
B.$x \in \{3\} \cup (8,11)$\\
C.$x \in \{-3\} \cup (8,11]$\\
D.$x \in \{3\} \cup (8,11]$\\
E.$x \in \{-3\} \cup [8,11)$\\
F.$x \in \{3\} \cup [8,11)$\\
G.$x \in \{-3\} \cup (8,11)$\\
H.$x \in \{3\} \cup [8,11]$
\testStop
\kluczStart
A
\kluczStop



\zadStart{Zadanie z Wikieł Z 1.62 c) moja wersja nr 213}

Rozwiązać nierówności $(8-x)(x+3)^{2}(12-x)^{3}\le0$.
\zadStop
\rozwStart{Patryk Wirkus}{Laura Mieczkowska}
Miejsca zerowe naszego wielomianu to: $8, -3, 12$.\\
Wielomian jest stopnia parzystego, ponadto znak współczynnika przy\linebreak najwyższej potędze x jest ujemny.\\ W związku z tym wykres wielomianu zaczyna się od lewej strony powyżej osi OX.\\
Ponadto w punkcie $-3$ wykres odbija się od osi poziomej.\\
A więc $$x \in \{-3\} \cup [8,12].$$
\rozwStop
\odpStart
$x \in \{-3\} \cup [8,12]$
\odpStop
\testStart
A.$x \in \{-3\} \cup [8,12]$\\
B.$x \in \{3\} \cup (8,12)$\\
C.$x \in \{-3\} \cup (8,12]$\\
D.$x \in \{3\} \cup (8,12]$\\
E.$x \in \{-3\} \cup [8,12)$\\
F.$x \in \{3\} \cup [8,12)$\\
G.$x \in \{-3\} \cup (8,12)$\\
H.$x \in \{3\} \cup [8,12]$
\testStop
\kluczStart
A
\kluczStop



\zadStart{Zadanie z Wikieł Z 1.62 c) moja wersja nr 214}

Rozwiązać nierówności $(8-x)(x+3)^{2}(13-x)^{3}\le0$.
\zadStop
\rozwStart{Patryk Wirkus}{Laura Mieczkowska}
Miejsca zerowe naszego wielomianu to: $8, -3, 13$.\\
Wielomian jest stopnia parzystego, ponadto znak współczynnika przy\linebreak najwyższej potędze x jest ujemny.\\ W związku z tym wykres wielomianu zaczyna się od lewej strony powyżej osi OX.\\
Ponadto w punkcie $-3$ wykres odbija się od osi poziomej.\\
A więc $$x \in \{-3\} \cup [8,13].$$
\rozwStop
\odpStart
$x \in \{-3\} \cup [8,13]$
\odpStop
\testStart
A.$x \in \{-3\} \cup [8,13]$\\
B.$x \in \{3\} \cup (8,13)$\\
C.$x \in \{-3\} \cup (8,13]$\\
D.$x \in \{3\} \cup (8,13]$\\
E.$x \in \{-3\} \cup [8,13)$\\
F.$x \in \{3\} \cup [8,13)$\\
G.$x \in \{-3\} \cup (8,13)$\\
H.$x \in \{3\} \cup [8,13]$
\testStop
\kluczStart
A
\kluczStop



\zadStart{Zadanie z Wikieł Z 1.62 c) moja wersja nr 215}

Rozwiązać nierówności $(8-x)(x+3)^{2}(14-x)^{3}\le0$.
\zadStop
\rozwStart{Patryk Wirkus}{Laura Mieczkowska}
Miejsca zerowe naszego wielomianu to: $8, -3, 14$.\\
Wielomian jest stopnia parzystego, ponadto znak współczynnika przy\linebreak najwyższej potędze x jest ujemny.\\ W związku z tym wykres wielomianu zaczyna się od lewej strony powyżej osi OX.\\
Ponadto w punkcie $-3$ wykres odbija się od osi poziomej.\\
A więc $$x \in \{-3\} \cup [8,14].$$
\rozwStop
\odpStart
$x \in \{-3\} \cup [8,14]$
\odpStop
\testStart
A.$x \in \{-3\} \cup [8,14]$\\
B.$x \in \{3\} \cup (8,14)$\\
C.$x \in \{-3\} \cup (8,14]$\\
D.$x \in \{3\} \cup (8,14]$\\
E.$x \in \{-3\} \cup [8,14)$\\
F.$x \in \{3\} \cup [8,14)$\\
G.$x \in \{-3\} \cup (8,14)$\\
H.$x \in \{3\} \cup [8,14]$
\testStop
\kluczStart
A
\kluczStop



\zadStart{Zadanie z Wikieł Z 1.62 c) moja wersja nr 216}

Rozwiązać nierówności $(8-x)(x+3)^{2}(15-x)^{3}\le0$.
\zadStop
\rozwStart{Patryk Wirkus}{Laura Mieczkowska}
Miejsca zerowe naszego wielomianu to: $8, -3, 15$.\\
Wielomian jest stopnia parzystego, ponadto znak współczynnika przy\linebreak najwyższej potędze x jest ujemny.\\ W związku z tym wykres wielomianu zaczyna się od lewej strony powyżej osi OX.\\
Ponadto w punkcie $-3$ wykres odbija się od osi poziomej.\\
A więc $$x \in \{-3\} \cup [8,15].$$
\rozwStop
\odpStart
$x \in \{-3\} \cup [8,15]$
\odpStop
\testStart
A.$x \in \{-3\} \cup [8,15]$\\
B.$x \in \{3\} \cup (8,15)$\\
C.$x \in \{-3\} \cup (8,15]$\\
D.$x \in \{3\} \cup (8,15]$\\
E.$x \in \{-3\} \cup [8,15)$\\
F.$x \in \{3\} \cup [8,15)$\\
G.$x \in \{-3\} \cup (8,15)$\\
H.$x \in \{3\} \cup [8,15]$
\testStop
\kluczStart
A
\kluczStop



\zadStart{Zadanie z Wikieł Z 1.62 c) moja wersja nr 217}

Rozwiązać nierówności $(8-x)(x+4)^{2}(9-x)^{3}\le0$.
\zadStop
\rozwStart{Patryk Wirkus}{Laura Mieczkowska}
Miejsca zerowe naszego wielomianu to: $8, -4, 9$.\\
Wielomian jest stopnia parzystego, ponadto znak współczynnika przy\linebreak najwyższej potędze x jest ujemny.\\ W związku z tym wykres wielomianu zaczyna się od lewej strony powyżej osi OX.\\
Ponadto w punkcie $-4$ wykres odbija się od osi poziomej.\\
A więc $$x \in \{-4\} \cup [8,9].$$
\rozwStop
\odpStart
$x \in \{-4\} \cup [8,9]$
\odpStop
\testStart
A.$x \in \{-4\} \cup [8,9]$\\
B.$x \in \{4\} \cup (8,9)$\\
C.$x \in \{-4\} \cup (8,9]$\\
D.$x \in \{4\} \cup (8,9]$\\
E.$x \in \{-4\} \cup [8,9)$\\
F.$x \in \{4\} \cup [8,9)$\\
G.$x \in \{-4\} \cup (8,9)$\\
H.$x \in \{4\} \cup [8,9]$
\testStop
\kluczStart
A
\kluczStop



\zadStart{Zadanie z Wikieł Z 1.62 c) moja wersja nr 218}

Rozwiązać nierówności $(8-x)(x+4)^{2}(10-x)^{3}\le0$.
\zadStop
\rozwStart{Patryk Wirkus}{Laura Mieczkowska}
Miejsca zerowe naszego wielomianu to: $8, -4, 10$.\\
Wielomian jest stopnia parzystego, ponadto znak współczynnika przy\linebreak najwyższej potędze x jest ujemny.\\ W związku z tym wykres wielomianu zaczyna się od lewej strony powyżej osi OX.\\
Ponadto w punkcie $-4$ wykres odbija się od osi poziomej.\\
A więc $$x \in \{-4\} \cup [8,10].$$
\rozwStop
\odpStart
$x \in \{-4\} \cup [8,10]$
\odpStop
\testStart
A.$x \in \{-4\} \cup [8,10]$\\
B.$x \in \{4\} \cup (8,10)$\\
C.$x \in \{-4\} \cup (8,10]$\\
D.$x \in \{4\} \cup (8,10]$\\
E.$x \in \{-4\} \cup [8,10)$\\
F.$x \in \{4\} \cup [8,10)$\\
G.$x \in \{-4\} \cup (8,10)$\\
H.$x \in \{4\} \cup [8,10]$
\testStop
\kluczStart
A
\kluczStop



\zadStart{Zadanie z Wikieł Z 1.62 c) moja wersja nr 219}

Rozwiązać nierówności $(8-x)(x+4)^{2}(11-x)^{3}\le0$.
\zadStop
\rozwStart{Patryk Wirkus}{Laura Mieczkowska}
Miejsca zerowe naszego wielomianu to: $8, -4, 11$.\\
Wielomian jest stopnia parzystego, ponadto znak współczynnika przy\linebreak najwyższej potędze x jest ujemny.\\ W związku z tym wykres wielomianu zaczyna się od lewej strony powyżej osi OX.\\
Ponadto w punkcie $-4$ wykres odbija się od osi poziomej.\\
A więc $$x \in \{-4\} \cup [8,11].$$
\rozwStop
\odpStart
$x \in \{-4\} \cup [8,11]$
\odpStop
\testStart
A.$x \in \{-4\} \cup [8,11]$\\
B.$x \in \{4\} \cup (8,11)$\\
C.$x \in \{-4\} \cup (8,11]$\\
D.$x \in \{4\} \cup (8,11]$\\
E.$x \in \{-4\} \cup [8,11)$\\
F.$x \in \{4\} \cup [8,11)$\\
G.$x \in \{-4\} \cup (8,11)$\\
H.$x \in \{4\} \cup [8,11]$
\testStop
\kluczStart
A
\kluczStop



\zadStart{Zadanie z Wikieł Z 1.62 c) moja wersja nr 220}

Rozwiązać nierówności $(8-x)(x+4)^{2}(12-x)^{3}\le0$.
\zadStop
\rozwStart{Patryk Wirkus}{Laura Mieczkowska}
Miejsca zerowe naszego wielomianu to: $8, -4, 12$.\\
Wielomian jest stopnia parzystego, ponadto znak współczynnika przy\linebreak najwyższej potędze x jest ujemny.\\ W związku z tym wykres wielomianu zaczyna się od lewej strony powyżej osi OX.\\
Ponadto w punkcie $-4$ wykres odbija się od osi poziomej.\\
A więc $$x \in \{-4\} \cup [8,12].$$
\rozwStop
\odpStart
$x \in \{-4\} \cup [8,12]$
\odpStop
\testStart
A.$x \in \{-4\} \cup [8,12]$\\
B.$x \in \{4\} \cup (8,12)$\\
C.$x \in \{-4\} \cup (8,12]$\\
D.$x \in \{4\} \cup (8,12]$\\
E.$x \in \{-4\} \cup [8,12)$\\
F.$x \in \{4\} \cup [8,12)$\\
G.$x \in \{-4\} \cup (8,12)$\\
H.$x \in \{4\} \cup [8,12]$
\testStop
\kluczStart
A
\kluczStop



\zadStart{Zadanie z Wikieł Z 1.62 c) moja wersja nr 221}

Rozwiązać nierówności $(8-x)(x+4)^{2}(13-x)^{3}\le0$.
\zadStop
\rozwStart{Patryk Wirkus}{Laura Mieczkowska}
Miejsca zerowe naszego wielomianu to: $8, -4, 13$.\\
Wielomian jest stopnia parzystego, ponadto znak współczynnika przy\linebreak najwyższej potędze x jest ujemny.\\ W związku z tym wykres wielomianu zaczyna się od lewej strony powyżej osi OX.\\
Ponadto w punkcie $-4$ wykres odbija się od osi poziomej.\\
A więc $$x \in \{-4\} \cup [8,13].$$
\rozwStop
\odpStart
$x \in \{-4\} \cup [8,13]$
\odpStop
\testStart
A.$x \in \{-4\} \cup [8,13]$\\
B.$x \in \{4\} \cup (8,13)$\\
C.$x \in \{-4\} \cup (8,13]$\\
D.$x \in \{4\} \cup (8,13]$\\
E.$x \in \{-4\} \cup [8,13)$\\
F.$x \in \{4\} \cup [8,13)$\\
G.$x \in \{-4\} \cup (8,13)$\\
H.$x \in \{4\} \cup [8,13]$
\testStop
\kluczStart
A
\kluczStop



\zadStart{Zadanie z Wikieł Z 1.62 c) moja wersja nr 222}

Rozwiązać nierówności $(8-x)(x+4)^{2}(14-x)^{3}\le0$.
\zadStop
\rozwStart{Patryk Wirkus}{Laura Mieczkowska}
Miejsca zerowe naszego wielomianu to: $8, -4, 14$.\\
Wielomian jest stopnia parzystego, ponadto znak współczynnika przy\linebreak najwyższej potędze x jest ujemny.\\ W związku z tym wykres wielomianu zaczyna się od lewej strony powyżej osi OX.\\
Ponadto w punkcie $-4$ wykres odbija się od osi poziomej.\\
A więc $$x \in \{-4\} \cup [8,14].$$
\rozwStop
\odpStart
$x \in \{-4\} \cup [8,14]$
\odpStop
\testStart
A.$x \in \{-4\} \cup [8,14]$\\
B.$x \in \{4\} \cup (8,14)$\\
C.$x \in \{-4\} \cup (8,14]$\\
D.$x \in \{4\} \cup (8,14]$\\
E.$x \in \{-4\} \cup [8,14)$\\
F.$x \in \{4\} \cup [8,14)$\\
G.$x \in \{-4\} \cup (8,14)$\\
H.$x \in \{4\} \cup [8,14]$
\testStop
\kluczStart
A
\kluczStop



\zadStart{Zadanie z Wikieł Z 1.62 c) moja wersja nr 223}

Rozwiązać nierówności $(8-x)(x+4)^{2}(15-x)^{3}\le0$.
\zadStop
\rozwStart{Patryk Wirkus}{Laura Mieczkowska}
Miejsca zerowe naszego wielomianu to: $8, -4, 15$.\\
Wielomian jest stopnia parzystego, ponadto znak współczynnika przy\linebreak najwyższej potędze x jest ujemny.\\ W związku z tym wykres wielomianu zaczyna się od lewej strony powyżej osi OX.\\
Ponadto w punkcie $-4$ wykres odbija się od osi poziomej.\\
A więc $$x \in \{-4\} \cup [8,15].$$
\rozwStop
\odpStart
$x \in \{-4\} \cup [8,15]$
\odpStop
\testStart
A.$x \in \{-4\} \cup [8,15]$\\
B.$x \in \{4\} \cup (8,15)$\\
C.$x \in \{-4\} \cup (8,15]$\\
D.$x \in \{4\} \cup (8,15]$\\
E.$x \in \{-4\} \cup [8,15)$\\
F.$x \in \{4\} \cup [8,15)$\\
G.$x \in \{-4\} \cup (8,15)$\\
H.$x \in \{4\} \cup [8,15]$
\testStop
\kluczStart
A
\kluczStop



\zadStart{Zadanie z Wikieł Z 1.62 c) moja wersja nr 224}

Rozwiązać nierówności $(8-x)(x+5)^{2}(9-x)^{3}\le0$.
\zadStop
\rozwStart{Patryk Wirkus}{Laura Mieczkowska}
Miejsca zerowe naszego wielomianu to: $8, -5, 9$.\\
Wielomian jest stopnia parzystego, ponadto znak współczynnika przy\linebreak najwyższej potędze x jest ujemny.\\ W związku z tym wykres wielomianu zaczyna się od lewej strony powyżej osi OX.\\
Ponadto w punkcie $-5$ wykres odbija się od osi poziomej.\\
A więc $$x \in \{-5\} \cup [8,9].$$
\rozwStop
\odpStart
$x \in \{-5\} \cup [8,9]$
\odpStop
\testStart
A.$x \in \{-5\} \cup [8,9]$\\
B.$x \in \{5\} \cup (8,9)$\\
C.$x \in \{-5\} \cup (8,9]$\\
D.$x \in \{5\} \cup (8,9]$\\
E.$x \in \{-5\} \cup [8,9)$\\
F.$x \in \{5\} \cup [8,9)$\\
G.$x \in \{-5\} \cup (8,9)$\\
H.$x \in \{5\} \cup [8,9]$
\testStop
\kluczStart
A
\kluczStop



\zadStart{Zadanie z Wikieł Z 1.62 c) moja wersja nr 225}

Rozwiązać nierówności $(8-x)(x+5)^{2}(10-x)^{3}\le0$.
\zadStop
\rozwStart{Patryk Wirkus}{Laura Mieczkowska}
Miejsca zerowe naszego wielomianu to: $8, -5, 10$.\\
Wielomian jest stopnia parzystego, ponadto znak współczynnika przy\linebreak najwyższej potędze x jest ujemny.\\ W związku z tym wykres wielomianu zaczyna się od lewej strony powyżej osi OX.\\
Ponadto w punkcie $-5$ wykres odbija się od osi poziomej.\\
A więc $$x \in \{-5\} \cup [8,10].$$
\rozwStop
\odpStart
$x \in \{-5\} \cup [8,10]$
\odpStop
\testStart
A.$x \in \{-5\} \cup [8,10]$\\
B.$x \in \{5\} \cup (8,10)$\\
C.$x \in \{-5\} \cup (8,10]$\\
D.$x \in \{5\} \cup (8,10]$\\
E.$x \in \{-5\} \cup [8,10)$\\
F.$x \in \{5\} \cup [8,10)$\\
G.$x \in \{-5\} \cup (8,10)$\\
H.$x \in \{5\} \cup [8,10]$
\testStop
\kluczStart
A
\kluczStop



\zadStart{Zadanie z Wikieł Z 1.62 c) moja wersja nr 226}

Rozwiązać nierówności $(8-x)(x+5)^{2}(11-x)^{3}\le0$.
\zadStop
\rozwStart{Patryk Wirkus}{Laura Mieczkowska}
Miejsca zerowe naszego wielomianu to: $8, -5, 11$.\\
Wielomian jest stopnia parzystego, ponadto znak współczynnika przy\linebreak najwyższej potędze x jest ujemny.\\ W związku z tym wykres wielomianu zaczyna się od lewej strony powyżej osi OX.\\
Ponadto w punkcie $-5$ wykres odbija się od osi poziomej.\\
A więc $$x \in \{-5\} \cup [8,11].$$
\rozwStop
\odpStart
$x \in \{-5\} \cup [8,11]$
\odpStop
\testStart
A.$x \in \{-5\} \cup [8,11]$\\
B.$x \in \{5\} \cup (8,11)$\\
C.$x \in \{-5\} \cup (8,11]$\\
D.$x \in \{5\} \cup (8,11]$\\
E.$x \in \{-5\} \cup [8,11)$\\
F.$x \in \{5\} \cup [8,11)$\\
G.$x \in \{-5\} \cup (8,11)$\\
H.$x \in \{5\} \cup [8,11]$
\testStop
\kluczStart
A
\kluczStop



\zadStart{Zadanie z Wikieł Z 1.62 c) moja wersja nr 227}

Rozwiązać nierówności $(8-x)(x+5)^{2}(12-x)^{3}\le0$.
\zadStop
\rozwStart{Patryk Wirkus}{Laura Mieczkowska}
Miejsca zerowe naszego wielomianu to: $8, -5, 12$.\\
Wielomian jest stopnia parzystego, ponadto znak współczynnika przy\linebreak najwyższej potędze x jest ujemny.\\ W związku z tym wykres wielomianu zaczyna się od lewej strony powyżej osi OX.\\
Ponadto w punkcie $-5$ wykres odbija się od osi poziomej.\\
A więc $$x \in \{-5\} \cup [8,12].$$
\rozwStop
\odpStart
$x \in \{-5\} \cup [8,12]$
\odpStop
\testStart
A.$x \in \{-5\} \cup [8,12]$\\
B.$x \in \{5\} \cup (8,12)$\\
C.$x \in \{-5\} \cup (8,12]$\\
D.$x \in \{5\} \cup (8,12]$\\
E.$x \in \{-5\} \cup [8,12)$\\
F.$x \in \{5\} \cup [8,12)$\\
G.$x \in \{-5\} \cup (8,12)$\\
H.$x \in \{5\} \cup [8,12]$
\testStop
\kluczStart
A
\kluczStop



\zadStart{Zadanie z Wikieł Z 1.62 c) moja wersja nr 228}

Rozwiązać nierówności $(8-x)(x+5)^{2}(13-x)^{3}\le0$.
\zadStop
\rozwStart{Patryk Wirkus}{Laura Mieczkowska}
Miejsca zerowe naszego wielomianu to: $8, -5, 13$.\\
Wielomian jest stopnia parzystego, ponadto znak współczynnika przy\linebreak najwyższej potędze x jest ujemny.\\ W związku z tym wykres wielomianu zaczyna się od lewej strony powyżej osi OX.\\
Ponadto w punkcie $-5$ wykres odbija się od osi poziomej.\\
A więc $$x \in \{-5\} \cup [8,13].$$
\rozwStop
\odpStart
$x \in \{-5\} \cup [8,13]$
\odpStop
\testStart
A.$x \in \{-5\} \cup [8,13]$\\
B.$x \in \{5\} \cup (8,13)$\\
C.$x \in \{-5\} \cup (8,13]$\\
D.$x \in \{5\} \cup (8,13]$\\
E.$x \in \{-5\} \cup [8,13)$\\
F.$x \in \{5\} \cup [8,13)$\\
G.$x \in \{-5\} \cup (8,13)$\\
H.$x \in \{5\} \cup [8,13]$
\testStop
\kluczStart
A
\kluczStop



\zadStart{Zadanie z Wikieł Z 1.62 c) moja wersja nr 229}

Rozwiązać nierówności $(8-x)(x+5)^{2}(14-x)^{3}\le0$.
\zadStop
\rozwStart{Patryk Wirkus}{Laura Mieczkowska}
Miejsca zerowe naszego wielomianu to: $8, -5, 14$.\\
Wielomian jest stopnia parzystego, ponadto znak współczynnika przy\linebreak najwyższej potędze x jest ujemny.\\ W związku z tym wykres wielomianu zaczyna się od lewej strony powyżej osi OX.\\
Ponadto w punkcie $-5$ wykres odbija się od osi poziomej.\\
A więc $$x \in \{-5\} \cup [8,14].$$
\rozwStop
\odpStart
$x \in \{-5\} \cup [8,14]$
\odpStop
\testStart
A.$x \in \{-5\} \cup [8,14]$\\
B.$x \in \{5\} \cup (8,14)$\\
C.$x \in \{-5\} \cup (8,14]$\\
D.$x \in \{5\} \cup (8,14]$\\
E.$x \in \{-5\} \cup [8,14)$\\
F.$x \in \{5\} \cup [8,14)$\\
G.$x \in \{-5\} \cup (8,14)$\\
H.$x \in \{5\} \cup [8,14]$
\testStop
\kluczStart
A
\kluczStop



\zadStart{Zadanie z Wikieł Z 1.62 c) moja wersja nr 230}

Rozwiązać nierówności $(8-x)(x+5)^{2}(15-x)^{3}\le0$.
\zadStop
\rozwStart{Patryk Wirkus}{Laura Mieczkowska}
Miejsca zerowe naszego wielomianu to: $8, -5, 15$.\\
Wielomian jest stopnia parzystego, ponadto znak współczynnika przy\linebreak najwyższej potędze x jest ujemny.\\ W związku z tym wykres wielomianu zaczyna się od lewej strony powyżej osi OX.\\
Ponadto w punkcie $-5$ wykres odbija się od osi poziomej.\\
A więc $$x \in \{-5\} \cup [8,15].$$
\rozwStop
\odpStart
$x \in \{-5\} \cup [8,15]$
\odpStop
\testStart
A.$x \in \{-5\} \cup [8,15]$\\
B.$x \in \{5\} \cup (8,15)$\\
C.$x \in \{-5\} \cup (8,15]$\\
D.$x \in \{5\} \cup (8,15]$\\
E.$x \in \{-5\} \cup [8,15)$\\
F.$x \in \{5\} \cup [8,15)$\\
G.$x \in \{-5\} \cup (8,15)$\\
H.$x \in \{5\} \cup [8,15]$
\testStop
\kluczStart
A
\kluczStop



\zadStart{Zadanie z Wikieł Z 1.62 c) moja wersja nr 231}

Rozwiązać nierówności $(9-x)(x+1)^{2}(10-x)^{3}\le0$.
\zadStop
\rozwStart{Patryk Wirkus}{Laura Mieczkowska}
Miejsca zerowe naszego wielomianu to: $9, -1, 10$.\\
Wielomian jest stopnia parzystego, ponadto znak współczynnika przy\linebreak najwyższej potędze x jest ujemny.\\ W związku z tym wykres wielomianu zaczyna się od lewej strony powyżej osi OX.\\
Ponadto w punkcie $-1$ wykres odbija się od osi poziomej.\\
A więc $$x \in \{-1\} \cup [9,10].$$
\rozwStop
\odpStart
$x \in \{-1\} \cup [9,10]$
\odpStop
\testStart
A.$x \in \{-1\} \cup [9,10]$\\
B.$x \in \{1\} \cup (9,10)$\\
C.$x \in \{-1\} \cup (9,10]$\\
D.$x \in \{1\} \cup (9,10]$\\
E.$x \in \{-1\} \cup [9,10)$\\
F.$x \in \{1\} \cup [9,10)$\\
G.$x \in \{-1\} \cup (9,10)$\\
H.$x \in \{1\} \cup [9,10]$
\testStop
\kluczStart
A
\kluczStop



\zadStart{Zadanie z Wikieł Z 1.62 c) moja wersja nr 232}

Rozwiązać nierówności $(9-x)(x+1)^{2}(11-x)^{3}\le0$.
\zadStop
\rozwStart{Patryk Wirkus}{Laura Mieczkowska}
Miejsca zerowe naszego wielomianu to: $9, -1, 11$.\\
Wielomian jest stopnia parzystego, ponadto znak współczynnika przy\linebreak najwyższej potędze x jest ujemny.\\ W związku z tym wykres wielomianu zaczyna się od lewej strony powyżej osi OX.\\
Ponadto w punkcie $-1$ wykres odbija się od osi poziomej.\\
A więc $$x \in \{-1\} \cup [9,11].$$
\rozwStop
\odpStart
$x \in \{-1\} \cup [9,11]$
\odpStop
\testStart
A.$x \in \{-1\} \cup [9,11]$\\
B.$x \in \{1\} \cup (9,11)$\\
C.$x \in \{-1\} \cup (9,11]$\\
D.$x \in \{1\} \cup (9,11]$\\
E.$x \in \{-1\} \cup [9,11)$\\
F.$x \in \{1\} \cup [9,11)$\\
G.$x \in \{-1\} \cup (9,11)$\\
H.$x \in \{1\} \cup [9,11]$
\testStop
\kluczStart
A
\kluczStop



\zadStart{Zadanie z Wikieł Z 1.62 c) moja wersja nr 233}

Rozwiązać nierówności $(9-x)(x+1)^{2}(12-x)^{3}\le0$.
\zadStop
\rozwStart{Patryk Wirkus}{Laura Mieczkowska}
Miejsca zerowe naszego wielomianu to: $9, -1, 12$.\\
Wielomian jest stopnia parzystego, ponadto znak współczynnika przy\linebreak najwyższej potędze x jest ujemny.\\ W związku z tym wykres wielomianu zaczyna się od lewej strony powyżej osi OX.\\
Ponadto w punkcie $-1$ wykres odbija się od osi poziomej.\\
A więc $$x \in \{-1\} \cup [9,12].$$
\rozwStop
\odpStart
$x \in \{-1\} \cup [9,12]$
\odpStop
\testStart
A.$x \in \{-1\} \cup [9,12]$\\
B.$x \in \{1\} \cup (9,12)$\\
C.$x \in \{-1\} \cup (9,12]$\\
D.$x \in \{1\} \cup (9,12]$\\
E.$x \in \{-1\} \cup [9,12)$\\
F.$x \in \{1\} \cup [9,12)$\\
G.$x \in \{-1\} \cup (9,12)$\\
H.$x \in \{1\} \cup [9,12]$
\testStop
\kluczStart
A
\kluczStop



\zadStart{Zadanie z Wikieł Z 1.62 c) moja wersja nr 234}

Rozwiązać nierówności $(9-x)(x+1)^{2}(13-x)^{3}\le0$.
\zadStop
\rozwStart{Patryk Wirkus}{Laura Mieczkowska}
Miejsca zerowe naszego wielomianu to: $9, -1, 13$.\\
Wielomian jest stopnia parzystego, ponadto znak współczynnika przy\linebreak najwyższej potędze x jest ujemny.\\ W związku z tym wykres wielomianu zaczyna się od lewej strony powyżej osi OX.\\
Ponadto w punkcie $-1$ wykres odbija się od osi poziomej.\\
A więc $$x \in \{-1\} \cup [9,13].$$
\rozwStop
\odpStart
$x \in \{-1\} \cup [9,13]$
\odpStop
\testStart
A.$x \in \{-1\} \cup [9,13]$\\
B.$x \in \{1\} \cup (9,13)$\\
C.$x \in \{-1\} \cup (9,13]$\\
D.$x \in \{1\} \cup (9,13]$\\
E.$x \in \{-1\} \cup [9,13)$\\
F.$x \in \{1\} \cup [9,13)$\\
G.$x \in \{-1\} \cup (9,13)$\\
H.$x \in \{1\} \cup [9,13]$
\testStop
\kluczStart
A
\kluczStop



\zadStart{Zadanie z Wikieł Z 1.62 c) moja wersja nr 235}

Rozwiązać nierówności $(9-x)(x+1)^{2}(14-x)^{3}\le0$.
\zadStop
\rozwStart{Patryk Wirkus}{Laura Mieczkowska}
Miejsca zerowe naszego wielomianu to: $9, -1, 14$.\\
Wielomian jest stopnia parzystego, ponadto znak współczynnika przy\linebreak najwyższej potędze x jest ujemny.\\ W związku z tym wykres wielomianu zaczyna się od lewej strony powyżej osi OX.\\
Ponadto w punkcie $-1$ wykres odbija się od osi poziomej.\\
A więc $$x \in \{-1\} \cup [9,14].$$
\rozwStop
\odpStart
$x \in \{-1\} \cup [9,14]$
\odpStop
\testStart
A.$x \in \{-1\} \cup [9,14]$\\
B.$x \in \{1\} \cup (9,14)$\\
C.$x \in \{-1\} \cup (9,14]$\\
D.$x \in \{1\} \cup (9,14]$\\
E.$x \in \{-1\} \cup [9,14)$\\
F.$x \in \{1\} \cup [9,14)$\\
G.$x \in \{-1\} \cup (9,14)$\\
H.$x \in \{1\} \cup [9,14]$
\testStop
\kluczStart
A
\kluczStop



\zadStart{Zadanie z Wikieł Z 1.62 c) moja wersja nr 236}

Rozwiązać nierówności $(9-x)(x+1)^{2}(15-x)^{3}\le0$.
\zadStop
\rozwStart{Patryk Wirkus}{Laura Mieczkowska}
Miejsca zerowe naszego wielomianu to: $9, -1, 15$.\\
Wielomian jest stopnia parzystego, ponadto znak współczynnika przy\linebreak najwyższej potędze x jest ujemny.\\ W związku z tym wykres wielomianu zaczyna się od lewej strony powyżej osi OX.\\
Ponadto w punkcie $-1$ wykres odbija się od osi poziomej.\\
A więc $$x \in \{-1\} \cup [9,15].$$
\rozwStop
\odpStart
$x \in \{-1\} \cup [9,15]$
\odpStop
\testStart
A.$x \in \{-1\} \cup [9,15]$\\
B.$x \in \{1\} \cup (9,15)$\\
C.$x \in \{-1\} \cup (9,15]$\\
D.$x \in \{1\} \cup (9,15]$\\
E.$x \in \{-1\} \cup [9,15)$\\
F.$x \in \{1\} \cup [9,15)$\\
G.$x \in \{-1\} \cup (9,15)$\\
H.$x \in \{1\} \cup [9,15]$
\testStop
\kluczStart
A
\kluczStop



\zadStart{Zadanie z Wikieł Z 1.62 c) moja wersja nr 237}

Rozwiązać nierówności $(9-x)(x+2)^{2}(10-x)^{3}\le0$.
\zadStop
\rozwStart{Patryk Wirkus}{Laura Mieczkowska}
Miejsca zerowe naszego wielomianu to: $9, -2, 10$.\\
Wielomian jest stopnia parzystego, ponadto znak współczynnika przy\linebreak najwyższej potędze x jest ujemny.\\ W związku z tym wykres wielomianu zaczyna się od lewej strony powyżej osi OX.\\
Ponadto w punkcie $-2$ wykres odbija się od osi poziomej.\\
A więc $$x \in \{-2\} \cup [9,10].$$
\rozwStop
\odpStart
$x \in \{-2\} \cup [9,10]$
\odpStop
\testStart
A.$x \in \{-2\} \cup [9,10]$\\
B.$x \in \{2\} \cup (9,10)$\\
C.$x \in \{-2\} \cup (9,10]$\\
D.$x \in \{2\} \cup (9,10]$\\
E.$x \in \{-2\} \cup [9,10)$\\
F.$x \in \{2\} \cup [9,10)$\\
G.$x \in \{-2\} \cup (9,10)$\\
H.$x \in \{2\} \cup [9,10]$
\testStop
\kluczStart
A
\kluczStop



\zadStart{Zadanie z Wikieł Z 1.62 c) moja wersja nr 238}

Rozwiązać nierówności $(9-x)(x+2)^{2}(11-x)^{3}\le0$.
\zadStop
\rozwStart{Patryk Wirkus}{Laura Mieczkowska}
Miejsca zerowe naszego wielomianu to: $9, -2, 11$.\\
Wielomian jest stopnia parzystego, ponadto znak współczynnika przy\linebreak najwyższej potędze x jest ujemny.\\ W związku z tym wykres wielomianu zaczyna się od lewej strony powyżej osi OX.\\
Ponadto w punkcie $-2$ wykres odbija się od osi poziomej.\\
A więc $$x \in \{-2\} \cup [9,11].$$
\rozwStop
\odpStart
$x \in \{-2\} \cup [9,11]$
\odpStop
\testStart
A.$x \in \{-2\} \cup [9,11]$\\
B.$x \in \{2\} \cup (9,11)$\\
C.$x \in \{-2\} \cup (9,11]$\\
D.$x \in \{2\} \cup (9,11]$\\
E.$x \in \{-2\} \cup [9,11)$\\
F.$x \in \{2\} \cup [9,11)$\\
G.$x \in \{-2\} \cup (9,11)$\\
H.$x \in \{2\} \cup [9,11]$
\testStop
\kluczStart
A
\kluczStop



\zadStart{Zadanie z Wikieł Z 1.62 c) moja wersja nr 239}

Rozwiązać nierówności $(9-x)(x+2)^{2}(12-x)^{3}\le0$.
\zadStop
\rozwStart{Patryk Wirkus}{Laura Mieczkowska}
Miejsca zerowe naszego wielomianu to: $9, -2, 12$.\\
Wielomian jest stopnia parzystego, ponadto znak współczynnika przy\linebreak najwyższej potędze x jest ujemny.\\ W związku z tym wykres wielomianu zaczyna się od lewej strony powyżej osi OX.\\
Ponadto w punkcie $-2$ wykres odbija się od osi poziomej.\\
A więc $$x \in \{-2\} \cup [9,12].$$
\rozwStop
\odpStart
$x \in \{-2\} \cup [9,12]$
\odpStop
\testStart
A.$x \in \{-2\} \cup [9,12]$\\
B.$x \in \{2\} \cup (9,12)$\\
C.$x \in \{-2\} \cup (9,12]$\\
D.$x \in \{2\} \cup (9,12]$\\
E.$x \in \{-2\} \cup [9,12)$\\
F.$x \in \{2\} \cup [9,12)$\\
G.$x \in \{-2\} \cup (9,12)$\\
H.$x \in \{2\} \cup [9,12]$
\testStop
\kluczStart
A
\kluczStop



\zadStart{Zadanie z Wikieł Z 1.62 c) moja wersja nr 240}

Rozwiązać nierówności $(9-x)(x+2)^{2}(13-x)^{3}\le0$.
\zadStop
\rozwStart{Patryk Wirkus}{Laura Mieczkowska}
Miejsca zerowe naszego wielomianu to: $9, -2, 13$.\\
Wielomian jest stopnia parzystego, ponadto znak współczynnika przy\linebreak najwyższej potędze x jest ujemny.\\ W związku z tym wykres wielomianu zaczyna się od lewej strony powyżej osi OX.\\
Ponadto w punkcie $-2$ wykres odbija się od osi poziomej.\\
A więc $$x \in \{-2\} \cup [9,13].$$
\rozwStop
\odpStart
$x \in \{-2\} \cup [9,13]$
\odpStop
\testStart
A.$x \in \{-2\} \cup [9,13]$\\
B.$x \in \{2\} \cup (9,13)$\\
C.$x \in \{-2\} \cup (9,13]$\\
D.$x \in \{2\} \cup (9,13]$\\
E.$x \in \{-2\} \cup [9,13)$\\
F.$x \in \{2\} \cup [9,13)$\\
G.$x \in \{-2\} \cup (9,13)$\\
H.$x \in \{2\} \cup [9,13]$
\testStop
\kluczStart
A
\kluczStop



\zadStart{Zadanie z Wikieł Z 1.62 c) moja wersja nr 241}

Rozwiązać nierówności $(9-x)(x+2)^{2}(14-x)^{3}\le0$.
\zadStop
\rozwStart{Patryk Wirkus}{Laura Mieczkowska}
Miejsca zerowe naszego wielomianu to: $9, -2, 14$.\\
Wielomian jest stopnia parzystego, ponadto znak współczynnika przy\linebreak najwyższej potędze x jest ujemny.\\ W związku z tym wykres wielomianu zaczyna się od lewej strony powyżej osi OX.\\
Ponadto w punkcie $-2$ wykres odbija się od osi poziomej.\\
A więc $$x \in \{-2\} \cup [9,14].$$
\rozwStop
\odpStart
$x \in \{-2\} \cup [9,14]$
\odpStop
\testStart
A.$x \in \{-2\} \cup [9,14]$\\
B.$x \in \{2\} \cup (9,14)$\\
C.$x \in \{-2\} \cup (9,14]$\\
D.$x \in \{2\} \cup (9,14]$\\
E.$x \in \{-2\} \cup [9,14)$\\
F.$x \in \{2\} \cup [9,14)$\\
G.$x \in \{-2\} \cup (9,14)$\\
H.$x \in \{2\} \cup [9,14]$
\testStop
\kluczStart
A
\kluczStop



\zadStart{Zadanie z Wikieł Z 1.62 c) moja wersja nr 242}

Rozwiązać nierówności $(9-x)(x+2)^{2}(15-x)^{3}\le0$.
\zadStop
\rozwStart{Patryk Wirkus}{Laura Mieczkowska}
Miejsca zerowe naszego wielomianu to: $9, -2, 15$.\\
Wielomian jest stopnia parzystego, ponadto znak współczynnika przy\linebreak najwyższej potędze x jest ujemny.\\ W związku z tym wykres wielomianu zaczyna się od lewej strony powyżej osi OX.\\
Ponadto w punkcie $-2$ wykres odbija się od osi poziomej.\\
A więc $$x \in \{-2\} \cup [9,15].$$
\rozwStop
\odpStart
$x \in \{-2\} \cup [9,15]$
\odpStop
\testStart
A.$x \in \{-2\} \cup [9,15]$\\
B.$x \in \{2\} \cup (9,15)$\\
C.$x \in \{-2\} \cup (9,15]$\\
D.$x \in \{2\} \cup (9,15]$\\
E.$x \in \{-2\} \cup [9,15)$\\
F.$x \in \{2\} \cup [9,15)$\\
G.$x \in \{-2\} \cup (9,15)$\\
H.$x \in \{2\} \cup [9,15]$
\testStop
\kluczStart
A
\kluczStop



\zadStart{Zadanie z Wikieł Z 1.62 c) moja wersja nr 243}

Rozwiązać nierówności $(9-x)(x+3)^{2}(10-x)^{3}\le0$.
\zadStop
\rozwStart{Patryk Wirkus}{Laura Mieczkowska}
Miejsca zerowe naszego wielomianu to: $9, -3, 10$.\\
Wielomian jest stopnia parzystego, ponadto znak współczynnika przy\linebreak najwyższej potędze x jest ujemny.\\ W związku z tym wykres wielomianu zaczyna się od lewej strony powyżej osi OX.\\
Ponadto w punkcie $-3$ wykres odbija się od osi poziomej.\\
A więc $$x \in \{-3\} \cup [9,10].$$
\rozwStop
\odpStart
$x \in \{-3\} \cup [9,10]$
\odpStop
\testStart
A.$x \in \{-3\} \cup [9,10]$\\
B.$x \in \{3\} \cup (9,10)$\\
C.$x \in \{-3\} \cup (9,10]$\\
D.$x \in \{3\} \cup (9,10]$\\
E.$x \in \{-3\} \cup [9,10)$\\
F.$x \in \{3\} \cup [9,10)$\\
G.$x \in \{-3\} \cup (9,10)$\\
H.$x \in \{3\} \cup [9,10]$
\testStop
\kluczStart
A
\kluczStop



\zadStart{Zadanie z Wikieł Z 1.62 c) moja wersja nr 244}

Rozwiązać nierówności $(9-x)(x+3)^{2}(11-x)^{3}\le0$.
\zadStop
\rozwStart{Patryk Wirkus}{Laura Mieczkowska}
Miejsca zerowe naszego wielomianu to: $9, -3, 11$.\\
Wielomian jest stopnia parzystego, ponadto znak współczynnika przy\linebreak najwyższej potędze x jest ujemny.\\ W związku z tym wykres wielomianu zaczyna się od lewej strony powyżej osi OX.\\
Ponadto w punkcie $-3$ wykres odbija się od osi poziomej.\\
A więc $$x \in \{-3\} \cup [9,11].$$
\rozwStop
\odpStart
$x \in \{-3\} \cup [9,11]$
\odpStop
\testStart
A.$x \in \{-3\} \cup [9,11]$\\
B.$x \in \{3\} \cup (9,11)$\\
C.$x \in \{-3\} \cup (9,11]$\\
D.$x \in \{3\} \cup (9,11]$\\
E.$x \in \{-3\} \cup [9,11)$\\
F.$x \in \{3\} \cup [9,11)$\\
G.$x \in \{-3\} \cup (9,11)$\\
H.$x \in \{3\} \cup [9,11]$
\testStop
\kluczStart
A
\kluczStop



\zadStart{Zadanie z Wikieł Z 1.62 c) moja wersja nr 245}

Rozwiązać nierówności $(9-x)(x+3)^{2}(12-x)^{3}\le0$.
\zadStop
\rozwStart{Patryk Wirkus}{Laura Mieczkowska}
Miejsca zerowe naszego wielomianu to: $9, -3, 12$.\\
Wielomian jest stopnia parzystego, ponadto znak współczynnika przy\linebreak najwyższej potędze x jest ujemny.\\ W związku z tym wykres wielomianu zaczyna się od lewej strony powyżej osi OX.\\
Ponadto w punkcie $-3$ wykres odbija się od osi poziomej.\\
A więc $$x \in \{-3\} \cup [9,12].$$
\rozwStop
\odpStart
$x \in \{-3\} \cup [9,12]$
\odpStop
\testStart
A.$x \in \{-3\} \cup [9,12]$\\
B.$x \in \{3\} \cup (9,12)$\\
C.$x \in \{-3\} \cup (9,12]$\\
D.$x \in \{3\} \cup (9,12]$\\
E.$x \in \{-3\} \cup [9,12)$\\
F.$x \in \{3\} \cup [9,12)$\\
G.$x \in \{-3\} \cup (9,12)$\\
H.$x \in \{3\} \cup [9,12]$
\testStop
\kluczStart
A
\kluczStop



\zadStart{Zadanie z Wikieł Z 1.62 c) moja wersja nr 246}

Rozwiązać nierówności $(9-x)(x+3)^{2}(13-x)^{3}\le0$.
\zadStop
\rozwStart{Patryk Wirkus}{Laura Mieczkowska}
Miejsca zerowe naszego wielomianu to: $9, -3, 13$.\\
Wielomian jest stopnia parzystego, ponadto znak współczynnika przy\linebreak najwyższej potędze x jest ujemny.\\ W związku z tym wykres wielomianu zaczyna się od lewej strony powyżej osi OX.\\
Ponadto w punkcie $-3$ wykres odbija się od osi poziomej.\\
A więc $$x \in \{-3\} \cup [9,13].$$
\rozwStop
\odpStart
$x \in \{-3\} \cup [9,13]$
\odpStop
\testStart
A.$x \in \{-3\} \cup [9,13]$\\
B.$x \in \{3\} \cup (9,13)$\\
C.$x \in \{-3\} \cup (9,13]$\\
D.$x \in \{3\} \cup (9,13]$\\
E.$x \in \{-3\} \cup [9,13)$\\
F.$x \in \{3\} \cup [9,13)$\\
G.$x \in \{-3\} \cup (9,13)$\\
H.$x \in \{3\} \cup [9,13]$
\testStop
\kluczStart
A
\kluczStop



\zadStart{Zadanie z Wikieł Z 1.62 c) moja wersja nr 247}

Rozwiązać nierówności $(9-x)(x+3)^{2}(14-x)^{3}\le0$.
\zadStop
\rozwStart{Patryk Wirkus}{Laura Mieczkowska}
Miejsca zerowe naszego wielomianu to: $9, -3, 14$.\\
Wielomian jest stopnia parzystego, ponadto znak współczynnika przy\linebreak najwyższej potędze x jest ujemny.\\ W związku z tym wykres wielomianu zaczyna się od lewej strony powyżej osi OX.\\
Ponadto w punkcie $-3$ wykres odbija się od osi poziomej.\\
A więc $$x \in \{-3\} \cup [9,14].$$
\rozwStop
\odpStart
$x \in \{-3\} \cup [9,14]$
\odpStop
\testStart
A.$x \in \{-3\} \cup [9,14]$\\
B.$x \in \{3\} \cup (9,14)$\\
C.$x \in \{-3\} \cup (9,14]$\\
D.$x \in \{3\} \cup (9,14]$\\
E.$x \in \{-3\} \cup [9,14)$\\
F.$x \in \{3\} \cup [9,14)$\\
G.$x \in \{-3\} \cup (9,14)$\\
H.$x \in \{3\} \cup [9,14]$
\testStop
\kluczStart
A
\kluczStop



\zadStart{Zadanie z Wikieł Z 1.62 c) moja wersja nr 248}

Rozwiązać nierówności $(9-x)(x+3)^{2}(15-x)^{3}\le0$.
\zadStop
\rozwStart{Patryk Wirkus}{Laura Mieczkowska}
Miejsca zerowe naszego wielomianu to: $9, -3, 15$.\\
Wielomian jest stopnia parzystego, ponadto znak współczynnika przy\linebreak najwyższej potędze x jest ujemny.\\ W związku z tym wykres wielomianu zaczyna się od lewej strony powyżej osi OX.\\
Ponadto w punkcie $-3$ wykres odbija się od osi poziomej.\\
A więc $$x \in \{-3\} \cup [9,15].$$
\rozwStop
\odpStart
$x \in \{-3\} \cup [9,15]$
\odpStop
\testStart
A.$x \in \{-3\} \cup [9,15]$\\
B.$x \in \{3\} \cup (9,15)$\\
C.$x \in \{-3\} \cup (9,15]$\\
D.$x \in \{3\} \cup (9,15]$\\
E.$x \in \{-3\} \cup [9,15)$\\
F.$x \in \{3\} \cup [9,15)$\\
G.$x \in \{-3\} \cup (9,15)$\\
H.$x \in \{3\} \cup [9,15]$
\testStop
\kluczStart
A
\kluczStop



\zadStart{Zadanie z Wikieł Z 1.62 c) moja wersja nr 249}

Rozwiązać nierówności $(9-x)(x+4)^{2}(10-x)^{3}\le0$.
\zadStop
\rozwStart{Patryk Wirkus}{Laura Mieczkowska}
Miejsca zerowe naszego wielomianu to: $9, -4, 10$.\\
Wielomian jest stopnia parzystego, ponadto znak współczynnika przy\linebreak najwyższej potędze x jest ujemny.\\ W związku z tym wykres wielomianu zaczyna się od lewej strony powyżej osi OX.\\
Ponadto w punkcie $-4$ wykres odbija się od osi poziomej.\\
A więc $$x \in \{-4\} \cup [9,10].$$
\rozwStop
\odpStart
$x \in \{-4\} \cup [9,10]$
\odpStop
\testStart
A.$x \in \{-4\} \cup [9,10]$\\
B.$x \in \{4\} \cup (9,10)$\\
C.$x \in \{-4\} \cup (9,10]$\\
D.$x \in \{4\} \cup (9,10]$\\
E.$x \in \{-4\} \cup [9,10)$\\
F.$x \in \{4\} \cup [9,10)$\\
G.$x \in \{-4\} \cup (9,10)$\\
H.$x \in \{4\} \cup [9,10]$
\testStop
\kluczStart
A
\kluczStop



\zadStart{Zadanie z Wikieł Z 1.62 c) moja wersja nr 250}

Rozwiązać nierówności $(9-x)(x+4)^{2}(11-x)^{3}\le0$.
\zadStop
\rozwStart{Patryk Wirkus}{Laura Mieczkowska}
Miejsca zerowe naszego wielomianu to: $9, -4, 11$.\\
Wielomian jest stopnia parzystego, ponadto znak współczynnika przy\linebreak najwyższej potędze x jest ujemny.\\ W związku z tym wykres wielomianu zaczyna się od lewej strony powyżej osi OX.\\
Ponadto w punkcie $-4$ wykres odbija się od osi poziomej.\\
A więc $$x \in \{-4\} \cup [9,11].$$
\rozwStop
\odpStart
$x \in \{-4\} \cup [9,11]$
\odpStop
\testStart
A.$x \in \{-4\} \cup [9,11]$\\
B.$x \in \{4\} \cup (9,11)$\\
C.$x \in \{-4\} \cup (9,11]$\\
D.$x \in \{4\} \cup (9,11]$\\
E.$x \in \{-4\} \cup [9,11)$\\
F.$x \in \{4\} \cup [9,11)$\\
G.$x \in \{-4\} \cup (9,11)$\\
H.$x \in \{4\} \cup [9,11]$
\testStop
\kluczStart
A
\kluczStop



\zadStart{Zadanie z Wikieł Z 1.62 c) moja wersja nr 251}

Rozwiązać nierówności $(9-x)(x+4)^{2}(12-x)^{3}\le0$.
\zadStop
\rozwStart{Patryk Wirkus}{Laura Mieczkowska}
Miejsca zerowe naszego wielomianu to: $9, -4, 12$.\\
Wielomian jest stopnia parzystego, ponadto znak współczynnika przy\linebreak najwyższej potędze x jest ujemny.\\ W związku z tym wykres wielomianu zaczyna się od lewej strony powyżej osi OX.\\
Ponadto w punkcie $-4$ wykres odbija się od osi poziomej.\\
A więc $$x \in \{-4\} \cup [9,12].$$
\rozwStop
\odpStart
$x \in \{-4\} \cup [9,12]$
\odpStop
\testStart
A.$x \in \{-4\} \cup [9,12]$\\
B.$x \in \{4\} \cup (9,12)$\\
C.$x \in \{-4\} \cup (9,12]$\\
D.$x \in \{4\} \cup (9,12]$\\
E.$x \in \{-4\} \cup [9,12)$\\
F.$x \in \{4\} \cup [9,12)$\\
G.$x \in \{-4\} \cup (9,12)$\\
H.$x \in \{4\} \cup [9,12]$
\testStop
\kluczStart
A
\kluczStop



\zadStart{Zadanie z Wikieł Z 1.62 c) moja wersja nr 252}

Rozwiązać nierówności $(9-x)(x+4)^{2}(13-x)^{3}\le0$.
\zadStop
\rozwStart{Patryk Wirkus}{Laura Mieczkowska}
Miejsca zerowe naszego wielomianu to: $9, -4, 13$.\\
Wielomian jest stopnia parzystego, ponadto znak współczynnika przy\linebreak najwyższej potędze x jest ujemny.\\ W związku z tym wykres wielomianu zaczyna się od lewej strony powyżej osi OX.\\
Ponadto w punkcie $-4$ wykres odbija się od osi poziomej.\\
A więc $$x \in \{-4\} \cup [9,13].$$
\rozwStop
\odpStart
$x \in \{-4\} \cup [9,13]$
\odpStop
\testStart
A.$x \in \{-4\} \cup [9,13]$\\
B.$x \in \{4\} \cup (9,13)$\\
C.$x \in \{-4\} \cup (9,13]$\\
D.$x \in \{4\} \cup (9,13]$\\
E.$x \in \{-4\} \cup [9,13)$\\
F.$x \in \{4\} \cup [9,13)$\\
G.$x \in \{-4\} \cup (9,13)$\\
H.$x \in \{4\} \cup [9,13]$
\testStop
\kluczStart
A
\kluczStop



\zadStart{Zadanie z Wikieł Z 1.62 c) moja wersja nr 253}

Rozwiązać nierówności $(9-x)(x+4)^{2}(14-x)^{3}\le0$.
\zadStop
\rozwStart{Patryk Wirkus}{Laura Mieczkowska}
Miejsca zerowe naszego wielomianu to: $9, -4, 14$.\\
Wielomian jest stopnia parzystego, ponadto znak współczynnika przy\linebreak najwyższej potędze x jest ujemny.\\ W związku z tym wykres wielomianu zaczyna się od lewej strony powyżej osi OX.\\
Ponadto w punkcie $-4$ wykres odbija się od osi poziomej.\\
A więc $$x \in \{-4\} \cup [9,14].$$
\rozwStop
\odpStart
$x \in \{-4\} \cup [9,14]$
\odpStop
\testStart
A.$x \in \{-4\} \cup [9,14]$\\
B.$x \in \{4\} \cup (9,14)$\\
C.$x \in \{-4\} \cup (9,14]$\\
D.$x \in \{4\} \cup (9,14]$\\
E.$x \in \{-4\} \cup [9,14)$\\
F.$x \in \{4\} \cup [9,14)$\\
G.$x \in \{-4\} \cup (9,14)$\\
H.$x \in \{4\} \cup [9,14]$
\testStop
\kluczStart
A
\kluczStop



\zadStart{Zadanie z Wikieł Z 1.62 c) moja wersja nr 254}

Rozwiązać nierówności $(9-x)(x+4)^{2}(15-x)^{3}\le0$.
\zadStop
\rozwStart{Patryk Wirkus}{Laura Mieczkowska}
Miejsca zerowe naszego wielomianu to: $9, -4, 15$.\\
Wielomian jest stopnia parzystego, ponadto znak współczynnika przy\linebreak najwyższej potędze x jest ujemny.\\ W związku z tym wykres wielomianu zaczyna się od lewej strony powyżej osi OX.\\
Ponadto w punkcie $-4$ wykres odbija się od osi poziomej.\\
A więc $$x \in \{-4\} \cup [9,15].$$
\rozwStop
\odpStart
$x \in \{-4\} \cup [9,15]$
\odpStop
\testStart
A.$x \in \{-4\} \cup [9,15]$\\
B.$x \in \{4\} \cup (9,15)$\\
C.$x \in \{-4\} \cup (9,15]$\\
D.$x \in \{4\} \cup (9,15]$\\
E.$x \in \{-4\} \cup [9,15)$\\
F.$x \in \{4\} \cup [9,15)$\\
G.$x \in \{-4\} \cup (9,15)$\\
H.$x \in \{4\} \cup [9,15]$
\testStop
\kluczStart
A
\kluczStop



\zadStart{Zadanie z Wikieł Z 1.62 c) moja wersja nr 255}

Rozwiązać nierówności $(9-x)(x+5)^{2}(10-x)^{3}\le0$.
\zadStop
\rozwStart{Patryk Wirkus}{Laura Mieczkowska}
Miejsca zerowe naszego wielomianu to: $9, -5, 10$.\\
Wielomian jest stopnia parzystego, ponadto znak współczynnika przy\linebreak najwyższej potędze x jest ujemny.\\ W związku z tym wykres wielomianu zaczyna się od lewej strony powyżej osi OX.\\
Ponadto w punkcie $-5$ wykres odbija się od osi poziomej.\\
A więc $$x \in \{-5\} \cup [9,10].$$
\rozwStop
\odpStart
$x \in \{-5\} \cup [9,10]$
\odpStop
\testStart
A.$x \in \{-5\} \cup [9,10]$\\
B.$x \in \{5\} \cup (9,10)$\\
C.$x \in \{-5\} \cup (9,10]$\\
D.$x \in \{5\} \cup (9,10]$\\
E.$x \in \{-5\} \cup [9,10)$\\
F.$x \in \{5\} \cup [9,10)$\\
G.$x \in \{-5\} \cup (9,10)$\\
H.$x \in \{5\} \cup [9,10]$
\testStop
\kluczStart
A
\kluczStop



\zadStart{Zadanie z Wikieł Z 1.62 c) moja wersja nr 256}

Rozwiązać nierówności $(9-x)(x+5)^{2}(11-x)^{3}\le0$.
\zadStop
\rozwStart{Patryk Wirkus}{Laura Mieczkowska}
Miejsca zerowe naszego wielomianu to: $9, -5, 11$.\\
Wielomian jest stopnia parzystego, ponadto znak współczynnika przy\linebreak najwyższej potędze x jest ujemny.\\ W związku z tym wykres wielomianu zaczyna się od lewej strony powyżej osi OX.\\
Ponadto w punkcie $-5$ wykres odbija się od osi poziomej.\\
A więc $$x \in \{-5\} \cup [9,11].$$
\rozwStop
\odpStart
$x \in \{-5\} \cup [9,11]$
\odpStop
\testStart
A.$x \in \{-5\} \cup [9,11]$\\
B.$x \in \{5\} \cup (9,11)$\\
C.$x \in \{-5\} \cup (9,11]$\\
D.$x \in \{5\} \cup (9,11]$\\
E.$x \in \{-5\} \cup [9,11)$\\
F.$x \in \{5\} \cup [9,11)$\\
G.$x \in \{-5\} \cup (9,11)$\\
H.$x \in \{5\} \cup [9,11]$
\testStop
\kluczStart
A
\kluczStop



\zadStart{Zadanie z Wikieł Z 1.62 c) moja wersja nr 257}

Rozwiązać nierówności $(9-x)(x+5)^{2}(12-x)^{3}\le0$.
\zadStop
\rozwStart{Patryk Wirkus}{Laura Mieczkowska}
Miejsca zerowe naszego wielomianu to: $9, -5, 12$.\\
Wielomian jest stopnia parzystego, ponadto znak współczynnika przy\linebreak najwyższej potędze x jest ujemny.\\ W związku z tym wykres wielomianu zaczyna się od lewej strony powyżej osi OX.\\
Ponadto w punkcie $-5$ wykres odbija się od osi poziomej.\\
A więc $$x \in \{-5\} \cup [9,12].$$
\rozwStop
\odpStart
$x \in \{-5\} \cup [9,12]$
\odpStop
\testStart
A.$x \in \{-5\} \cup [9,12]$\\
B.$x \in \{5\} \cup (9,12)$\\
C.$x \in \{-5\} \cup (9,12]$\\
D.$x \in \{5\} \cup (9,12]$\\
E.$x \in \{-5\} \cup [9,12)$\\
F.$x \in \{5\} \cup [9,12)$\\
G.$x \in \{-5\} \cup (9,12)$\\
H.$x \in \{5\} \cup [9,12]$
\testStop
\kluczStart
A
\kluczStop



\zadStart{Zadanie z Wikieł Z 1.62 c) moja wersja nr 258}

Rozwiązać nierówności $(9-x)(x+5)^{2}(13-x)^{3}\le0$.
\zadStop
\rozwStart{Patryk Wirkus}{Laura Mieczkowska}
Miejsca zerowe naszego wielomianu to: $9, -5, 13$.\\
Wielomian jest stopnia parzystego, ponadto znak współczynnika przy\linebreak najwyższej potędze x jest ujemny.\\ W związku z tym wykres wielomianu zaczyna się od lewej strony powyżej osi OX.\\
Ponadto w punkcie $-5$ wykres odbija się od osi poziomej.\\
A więc $$x \in \{-5\} \cup [9,13].$$
\rozwStop
\odpStart
$x \in \{-5\} \cup [9,13]$
\odpStop
\testStart
A.$x \in \{-5\} \cup [9,13]$\\
B.$x \in \{5\} \cup (9,13)$\\
C.$x \in \{-5\} \cup (9,13]$\\
D.$x \in \{5\} \cup (9,13]$\\
E.$x \in \{-5\} \cup [9,13)$\\
F.$x \in \{5\} \cup [9,13)$\\
G.$x \in \{-5\} \cup (9,13)$\\
H.$x \in \{5\} \cup [9,13]$
\testStop
\kluczStart
A
\kluczStop



\zadStart{Zadanie z Wikieł Z 1.62 c) moja wersja nr 259}

Rozwiązać nierówności $(9-x)(x+5)^{2}(14-x)^{3}\le0$.
\zadStop
\rozwStart{Patryk Wirkus}{Laura Mieczkowska}
Miejsca zerowe naszego wielomianu to: $9, -5, 14$.\\
Wielomian jest stopnia parzystego, ponadto znak współczynnika przy\linebreak najwyższej potędze x jest ujemny.\\ W związku z tym wykres wielomianu zaczyna się od lewej strony powyżej osi OX.\\
Ponadto w punkcie $-5$ wykres odbija się od osi poziomej.\\
A więc $$x \in \{-5\} \cup [9,14].$$
\rozwStop
\odpStart
$x \in \{-5\} \cup [9,14]$
\odpStop
\testStart
A.$x \in \{-5\} \cup [9,14]$\\
B.$x \in \{5\} \cup (9,14)$\\
C.$x \in \{-5\} \cup (9,14]$\\
D.$x \in \{5\} \cup (9,14]$\\
E.$x \in \{-5\} \cup [9,14)$\\
F.$x \in \{5\} \cup [9,14)$\\
G.$x \in \{-5\} \cup (9,14)$\\
H.$x \in \{5\} \cup [9,14]$
\testStop
\kluczStart
A
\kluczStop



\zadStart{Zadanie z Wikieł Z 1.62 c) moja wersja nr 260}

Rozwiązać nierówności $(9-x)(x+5)^{2}(15-x)^{3}\le0$.
\zadStop
\rozwStart{Patryk Wirkus}{Laura Mieczkowska}
Miejsca zerowe naszego wielomianu to: $9, -5, 15$.\\
Wielomian jest stopnia parzystego, ponadto znak współczynnika przy\linebreak najwyższej potędze x jest ujemny.\\ W związku z tym wykres wielomianu zaczyna się od lewej strony powyżej osi OX.\\
Ponadto w punkcie $-5$ wykres odbija się od osi poziomej.\\
A więc $$x \in \{-5\} \cup [9,15].$$
\rozwStop
\odpStart
$x \in \{-5\} \cup [9,15]$
\odpStop
\testStart
A.$x \in \{-5\} \cup [9,15]$\\
B.$x \in \{5\} \cup (9,15)$\\
C.$x \in \{-5\} \cup (9,15]$\\
D.$x \in \{5\} \cup (9,15]$\\
E.$x \in \{-5\} \cup [9,15)$\\
F.$x \in \{5\} \cup [9,15)$\\
G.$x \in \{-5\} \cup (9,15)$\\
H.$x \in \{5\} \cup [9,15]$
\testStop
\kluczStart
A
\kluczStop



\zadStart{Zadanie z Wikieł Z 1.62 c) moja wersja nr 261}

Rozwiązać nierówności $(10-x)(x+1)^{2}(11-x)^{3}\le0$.
\zadStop
\rozwStart{Patryk Wirkus}{Laura Mieczkowska}
Miejsca zerowe naszego wielomianu to: $10, -1, 11$.\\
Wielomian jest stopnia parzystego, ponadto znak współczynnika przy\linebreak najwyższej potędze x jest ujemny.\\ W związku z tym wykres wielomianu zaczyna się od lewej strony powyżej osi OX.\\
Ponadto w punkcie $-1$ wykres odbija się od osi poziomej.\\
A więc $$x \in \{-1\} \cup [10,11].$$
\rozwStop
\odpStart
$x \in \{-1\} \cup [10,11]$
\odpStop
\testStart
A.$x \in \{-1\} \cup [10,11]$\\
B.$x \in \{1\} \cup (10,11)$\\
C.$x \in \{-1\} \cup (10,11]$\\
D.$x \in \{1\} \cup (10,11]$\\
E.$x \in \{-1\} \cup [10,11)$\\
F.$x \in \{1\} \cup [10,11)$\\
G.$x \in \{-1\} \cup (10,11)$\\
H.$x \in \{1\} \cup [10,11]$
\testStop
\kluczStart
A
\kluczStop



\zadStart{Zadanie z Wikieł Z 1.62 c) moja wersja nr 262}

Rozwiązać nierówności $(10-x)(x+1)^{2}(12-x)^{3}\le0$.
\zadStop
\rozwStart{Patryk Wirkus}{Laura Mieczkowska}
Miejsca zerowe naszego wielomianu to: $10, -1, 12$.\\
Wielomian jest stopnia parzystego, ponadto znak współczynnika przy\linebreak najwyższej potędze x jest ujemny.\\ W związku z tym wykres wielomianu zaczyna się od lewej strony powyżej osi OX.\\
Ponadto w punkcie $-1$ wykres odbija się od osi poziomej.\\
A więc $$x \in \{-1\} \cup [10,12].$$
\rozwStop
\odpStart
$x \in \{-1\} \cup [10,12]$
\odpStop
\testStart
A.$x \in \{-1\} \cup [10,12]$\\
B.$x \in \{1\} \cup (10,12)$\\
C.$x \in \{-1\} \cup (10,12]$\\
D.$x \in \{1\} \cup (10,12]$\\
E.$x \in \{-1\} \cup [10,12)$\\
F.$x \in \{1\} \cup [10,12)$\\
G.$x \in \{-1\} \cup (10,12)$\\
H.$x \in \{1\} \cup [10,12]$
\testStop
\kluczStart
A
\kluczStop



\zadStart{Zadanie z Wikieł Z 1.62 c) moja wersja nr 263}

Rozwiązać nierówności $(10-x)(x+1)^{2}(13-x)^{3}\le0$.
\zadStop
\rozwStart{Patryk Wirkus}{Laura Mieczkowska}
Miejsca zerowe naszego wielomianu to: $10, -1, 13$.\\
Wielomian jest stopnia parzystego, ponadto znak współczynnika przy\linebreak najwyższej potędze x jest ujemny.\\ W związku z tym wykres wielomianu zaczyna się od lewej strony powyżej osi OX.\\
Ponadto w punkcie $-1$ wykres odbija się od osi poziomej.\\
A więc $$x \in \{-1\} \cup [10,13].$$
\rozwStop
\odpStart
$x \in \{-1\} \cup [10,13]$
\odpStop
\testStart
A.$x \in \{-1\} \cup [10,13]$\\
B.$x \in \{1\} \cup (10,13)$\\
C.$x \in \{-1\} \cup (10,13]$\\
D.$x \in \{1\} \cup (10,13]$\\
E.$x \in \{-1\} \cup [10,13)$\\
F.$x \in \{1\} \cup [10,13)$\\
G.$x \in \{-1\} \cup (10,13)$\\
H.$x \in \{1\} \cup [10,13]$
\testStop
\kluczStart
A
\kluczStop



\zadStart{Zadanie z Wikieł Z 1.62 c) moja wersja nr 264}

Rozwiązać nierówności $(10-x)(x+1)^{2}(14-x)^{3}\le0$.
\zadStop
\rozwStart{Patryk Wirkus}{Laura Mieczkowska}
Miejsca zerowe naszego wielomianu to: $10, -1, 14$.\\
Wielomian jest stopnia parzystego, ponadto znak współczynnika przy\linebreak najwyższej potędze x jest ujemny.\\ W związku z tym wykres wielomianu zaczyna się od lewej strony powyżej osi OX.\\
Ponadto w punkcie $-1$ wykres odbija się od osi poziomej.\\
A więc $$x \in \{-1\} \cup [10,14].$$
\rozwStop
\odpStart
$x \in \{-1\} \cup [10,14]$
\odpStop
\testStart
A.$x \in \{-1\} \cup [10,14]$\\
B.$x \in \{1\} \cup (10,14)$\\
C.$x \in \{-1\} \cup (10,14]$\\
D.$x \in \{1\} \cup (10,14]$\\
E.$x \in \{-1\} \cup [10,14)$\\
F.$x \in \{1\} \cup [10,14)$\\
G.$x \in \{-1\} \cup (10,14)$\\
H.$x \in \{1\} \cup [10,14]$
\testStop
\kluczStart
A
\kluczStop



\zadStart{Zadanie z Wikieł Z 1.62 c) moja wersja nr 265}

Rozwiązać nierówności $(10-x)(x+1)^{2}(15-x)^{3}\le0$.
\zadStop
\rozwStart{Patryk Wirkus}{Laura Mieczkowska}
Miejsca zerowe naszego wielomianu to: $10, -1, 15$.\\
Wielomian jest stopnia parzystego, ponadto znak współczynnika przy\linebreak najwyższej potędze x jest ujemny.\\ W związku z tym wykres wielomianu zaczyna się od lewej strony powyżej osi OX.\\
Ponadto w punkcie $-1$ wykres odbija się od osi poziomej.\\
A więc $$x \in \{-1\} \cup [10,15].$$
\rozwStop
\odpStart
$x \in \{-1\} \cup [10,15]$
\odpStop
\testStart
A.$x \in \{-1\} \cup [10,15]$\\
B.$x \in \{1\} \cup (10,15)$\\
C.$x \in \{-1\} \cup (10,15]$\\
D.$x \in \{1\} \cup (10,15]$\\
E.$x \in \{-1\} \cup [10,15)$\\
F.$x \in \{1\} \cup [10,15)$\\
G.$x \in \{-1\} \cup (10,15)$\\
H.$x \in \{1\} \cup [10,15]$
\testStop
\kluczStart
A
\kluczStop



\zadStart{Zadanie z Wikieł Z 1.62 c) moja wersja nr 266}

Rozwiązać nierówności $(10-x)(x+2)^{2}(11-x)^{3}\le0$.
\zadStop
\rozwStart{Patryk Wirkus}{Laura Mieczkowska}
Miejsca zerowe naszego wielomianu to: $10, -2, 11$.\\
Wielomian jest stopnia parzystego, ponadto znak współczynnika przy\linebreak najwyższej potędze x jest ujemny.\\ W związku z tym wykres wielomianu zaczyna się od lewej strony powyżej osi OX.\\
Ponadto w punkcie $-2$ wykres odbija się od osi poziomej.\\
A więc $$x \in \{-2\} \cup [10,11].$$
\rozwStop
\odpStart
$x \in \{-2\} \cup [10,11]$
\odpStop
\testStart
A.$x \in \{-2\} \cup [10,11]$\\
B.$x \in \{2\} \cup (10,11)$\\
C.$x \in \{-2\} \cup (10,11]$\\
D.$x \in \{2\} \cup (10,11]$\\
E.$x \in \{-2\} \cup [10,11)$\\
F.$x \in \{2\} \cup [10,11)$\\
G.$x \in \{-2\} \cup (10,11)$\\
H.$x \in \{2\} \cup [10,11]$
\testStop
\kluczStart
A
\kluczStop



\zadStart{Zadanie z Wikieł Z 1.62 c) moja wersja nr 267}

Rozwiązać nierówności $(10-x)(x+2)^{2}(12-x)^{3}\le0$.
\zadStop
\rozwStart{Patryk Wirkus}{Laura Mieczkowska}
Miejsca zerowe naszego wielomianu to: $10, -2, 12$.\\
Wielomian jest stopnia parzystego, ponadto znak współczynnika przy\linebreak najwyższej potędze x jest ujemny.\\ W związku z tym wykres wielomianu zaczyna się od lewej strony powyżej osi OX.\\
Ponadto w punkcie $-2$ wykres odbija się od osi poziomej.\\
A więc $$x \in \{-2\} \cup [10,12].$$
\rozwStop
\odpStart
$x \in \{-2\} \cup [10,12]$
\odpStop
\testStart
A.$x \in \{-2\} \cup [10,12]$\\
B.$x \in \{2\} \cup (10,12)$\\
C.$x \in \{-2\} \cup (10,12]$\\
D.$x \in \{2\} \cup (10,12]$\\
E.$x \in \{-2\} \cup [10,12)$\\
F.$x \in \{2\} \cup [10,12)$\\
G.$x \in \{-2\} \cup (10,12)$\\
H.$x \in \{2\} \cup [10,12]$
\testStop
\kluczStart
A
\kluczStop



\zadStart{Zadanie z Wikieł Z 1.62 c) moja wersja nr 268}

Rozwiązać nierówności $(10-x)(x+2)^{2}(13-x)^{3}\le0$.
\zadStop
\rozwStart{Patryk Wirkus}{Laura Mieczkowska}
Miejsca zerowe naszego wielomianu to: $10, -2, 13$.\\
Wielomian jest stopnia parzystego, ponadto znak współczynnika przy\linebreak najwyższej potędze x jest ujemny.\\ W związku z tym wykres wielomianu zaczyna się od lewej strony powyżej osi OX.\\
Ponadto w punkcie $-2$ wykres odbija się od osi poziomej.\\
A więc $$x \in \{-2\} \cup [10,13].$$
\rozwStop
\odpStart
$x \in \{-2\} \cup [10,13]$
\odpStop
\testStart
A.$x \in \{-2\} \cup [10,13]$\\
B.$x \in \{2\} \cup (10,13)$\\
C.$x \in \{-2\} \cup (10,13]$\\
D.$x \in \{2\} \cup (10,13]$\\
E.$x \in \{-2\} \cup [10,13)$\\
F.$x \in \{2\} \cup [10,13)$\\
G.$x \in \{-2\} \cup (10,13)$\\
H.$x \in \{2\} \cup [10,13]$
\testStop
\kluczStart
A
\kluczStop



\zadStart{Zadanie z Wikieł Z 1.62 c) moja wersja nr 269}

Rozwiązać nierówności $(10-x)(x+2)^{2}(14-x)^{3}\le0$.
\zadStop
\rozwStart{Patryk Wirkus}{Laura Mieczkowska}
Miejsca zerowe naszego wielomianu to: $10, -2, 14$.\\
Wielomian jest stopnia parzystego, ponadto znak współczynnika przy\linebreak najwyższej potędze x jest ujemny.\\ W związku z tym wykres wielomianu zaczyna się od lewej strony powyżej osi OX.\\
Ponadto w punkcie $-2$ wykres odbija się od osi poziomej.\\
A więc $$x \in \{-2\} \cup [10,14].$$
\rozwStop
\odpStart
$x \in \{-2\} \cup [10,14]$
\odpStop
\testStart
A.$x \in \{-2\} \cup [10,14]$\\
B.$x \in \{2\} \cup (10,14)$\\
C.$x \in \{-2\} \cup (10,14]$\\
D.$x \in \{2\} \cup (10,14]$\\
E.$x \in \{-2\} \cup [10,14)$\\
F.$x \in \{2\} \cup [10,14)$\\
G.$x \in \{-2\} \cup (10,14)$\\
H.$x \in \{2\} \cup [10,14]$
\testStop
\kluczStart
A
\kluczStop



\zadStart{Zadanie z Wikieł Z 1.62 c) moja wersja nr 270}

Rozwiązać nierówności $(10-x)(x+2)^{2}(15-x)^{3}\le0$.
\zadStop
\rozwStart{Patryk Wirkus}{Laura Mieczkowska}
Miejsca zerowe naszego wielomianu to: $10, -2, 15$.\\
Wielomian jest stopnia parzystego, ponadto znak współczynnika przy\linebreak najwyższej potędze x jest ujemny.\\ W związku z tym wykres wielomianu zaczyna się od lewej strony powyżej osi OX.\\
Ponadto w punkcie $-2$ wykres odbija się od osi poziomej.\\
A więc $$x \in \{-2\} \cup [10,15].$$
\rozwStop
\odpStart
$x \in \{-2\} \cup [10,15]$
\odpStop
\testStart
A.$x \in \{-2\} \cup [10,15]$\\
B.$x \in \{2\} \cup (10,15)$\\
C.$x \in \{-2\} \cup (10,15]$\\
D.$x \in \{2\} \cup (10,15]$\\
E.$x \in \{-2\} \cup [10,15)$\\
F.$x \in \{2\} \cup [10,15)$\\
G.$x \in \{-2\} \cup (10,15)$\\
H.$x \in \{2\} \cup [10,15]$
\testStop
\kluczStart
A
\kluczStop



\zadStart{Zadanie z Wikieł Z 1.62 c) moja wersja nr 271}

Rozwiązać nierówności $(10-x)(x+3)^{2}(11-x)^{3}\le0$.
\zadStop
\rozwStart{Patryk Wirkus}{Laura Mieczkowska}
Miejsca zerowe naszego wielomianu to: $10, -3, 11$.\\
Wielomian jest stopnia parzystego, ponadto znak współczynnika przy\linebreak najwyższej potędze x jest ujemny.\\ W związku z tym wykres wielomianu zaczyna się od lewej strony powyżej osi OX.\\
Ponadto w punkcie $-3$ wykres odbija się od osi poziomej.\\
A więc $$x \in \{-3\} \cup [10,11].$$
\rozwStop
\odpStart
$x \in \{-3\} \cup [10,11]$
\odpStop
\testStart
A.$x \in \{-3\} \cup [10,11]$\\
B.$x \in \{3\} \cup (10,11)$\\
C.$x \in \{-3\} \cup (10,11]$\\
D.$x \in \{3\} \cup (10,11]$\\
E.$x \in \{-3\} \cup [10,11)$\\
F.$x \in \{3\} \cup [10,11)$\\
G.$x \in \{-3\} \cup (10,11)$\\
H.$x \in \{3\} \cup [10,11]$
\testStop
\kluczStart
A
\kluczStop



\zadStart{Zadanie z Wikieł Z 1.62 c) moja wersja nr 272}

Rozwiązać nierówności $(10-x)(x+3)^{2}(12-x)^{3}\le0$.
\zadStop
\rozwStart{Patryk Wirkus}{Laura Mieczkowska}
Miejsca zerowe naszego wielomianu to: $10, -3, 12$.\\
Wielomian jest stopnia parzystego, ponadto znak współczynnika przy\linebreak najwyższej potędze x jest ujemny.\\ W związku z tym wykres wielomianu zaczyna się od lewej strony powyżej osi OX.\\
Ponadto w punkcie $-3$ wykres odbija się od osi poziomej.\\
A więc $$x \in \{-3\} \cup [10,12].$$
\rozwStop
\odpStart
$x \in \{-3\} \cup [10,12]$
\odpStop
\testStart
A.$x \in \{-3\} \cup [10,12]$\\
B.$x \in \{3\} \cup (10,12)$\\
C.$x \in \{-3\} \cup (10,12]$\\
D.$x \in \{3\} \cup (10,12]$\\
E.$x \in \{-3\} \cup [10,12)$\\
F.$x \in \{3\} \cup [10,12)$\\
G.$x \in \{-3\} \cup (10,12)$\\
H.$x \in \{3\} \cup [10,12]$
\testStop
\kluczStart
A
\kluczStop



\zadStart{Zadanie z Wikieł Z 1.62 c) moja wersja nr 273}

Rozwiązać nierówności $(10-x)(x+3)^{2}(13-x)^{3}\le0$.
\zadStop
\rozwStart{Patryk Wirkus}{Laura Mieczkowska}
Miejsca zerowe naszego wielomianu to: $10, -3, 13$.\\
Wielomian jest stopnia parzystego, ponadto znak współczynnika przy\linebreak najwyższej potędze x jest ujemny.\\ W związku z tym wykres wielomianu zaczyna się od lewej strony powyżej osi OX.\\
Ponadto w punkcie $-3$ wykres odbija się od osi poziomej.\\
A więc $$x \in \{-3\} \cup [10,13].$$
\rozwStop
\odpStart
$x \in \{-3\} \cup [10,13]$
\odpStop
\testStart
A.$x \in \{-3\} \cup [10,13]$\\
B.$x \in \{3\} \cup (10,13)$\\
C.$x \in \{-3\} \cup (10,13]$\\
D.$x \in \{3\} \cup (10,13]$\\
E.$x \in \{-3\} \cup [10,13)$\\
F.$x \in \{3\} \cup [10,13)$\\
G.$x \in \{-3\} \cup (10,13)$\\
H.$x \in \{3\} \cup [10,13]$
\testStop
\kluczStart
A
\kluczStop



\zadStart{Zadanie z Wikieł Z 1.62 c) moja wersja nr 274}

Rozwiązać nierówności $(10-x)(x+3)^{2}(14-x)^{3}\le0$.
\zadStop
\rozwStart{Patryk Wirkus}{Laura Mieczkowska}
Miejsca zerowe naszego wielomianu to: $10, -3, 14$.\\
Wielomian jest stopnia parzystego, ponadto znak współczynnika przy\linebreak najwyższej potędze x jest ujemny.\\ W związku z tym wykres wielomianu zaczyna się od lewej strony powyżej osi OX.\\
Ponadto w punkcie $-3$ wykres odbija się od osi poziomej.\\
A więc $$x \in \{-3\} \cup [10,14].$$
\rozwStop
\odpStart
$x \in \{-3\} \cup [10,14]$
\odpStop
\testStart
A.$x \in \{-3\} \cup [10,14]$\\
B.$x \in \{3\} \cup (10,14)$\\
C.$x \in \{-3\} \cup (10,14]$\\
D.$x \in \{3\} \cup (10,14]$\\
E.$x \in \{-3\} \cup [10,14)$\\
F.$x \in \{3\} \cup [10,14)$\\
G.$x \in \{-3\} \cup (10,14)$\\
H.$x \in \{3\} \cup [10,14]$
\testStop
\kluczStart
A
\kluczStop



\zadStart{Zadanie z Wikieł Z 1.62 c) moja wersja nr 275}

Rozwiązać nierówności $(10-x)(x+3)^{2}(15-x)^{3}\le0$.
\zadStop
\rozwStart{Patryk Wirkus}{Laura Mieczkowska}
Miejsca zerowe naszego wielomianu to: $10, -3, 15$.\\
Wielomian jest stopnia parzystego, ponadto znak współczynnika przy\linebreak najwyższej potędze x jest ujemny.\\ W związku z tym wykres wielomianu zaczyna się od lewej strony powyżej osi OX.\\
Ponadto w punkcie $-3$ wykres odbija się od osi poziomej.\\
A więc $$x \in \{-3\} \cup [10,15].$$
\rozwStop
\odpStart
$x \in \{-3\} \cup [10,15]$
\odpStop
\testStart
A.$x \in \{-3\} \cup [10,15]$\\
B.$x \in \{3\} \cup (10,15)$\\
C.$x \in \{-3\} \cup (10,15]$\\
D.$x \in \{3\} \cup (10,15]$\\
E.$x \in \{-3\} \cup [10,15)$\\
F.$x \in \{3\} \cup [10,15)$\\
G.$x \in \{-3\} \cup (10,15)$\\
H.$x \in \{3\} \cup [10,15]$
\testStop
\kluczStart
A
\kluczStop



\zadStart{Zadanie z Wikieł Z 1.62 c) moja wersja nr 276}

Rozwiązać nierówności $(10-x)(x+4)^{2}(11-x)^{3}\le0$.
\zadStop
\rozwStart{Patryk Wirkus}{Laura Mieczkowska}
Miejsca zerowe naszego wielomianu to: $10, -4, 11$.\\
Wielomian jest stopnia parzystego, ponadto znak współczynnika przy\linebreak najwyższej potędze x jest ujemny.\\ W związku z tym wykres wielomianu zaczyna się od lewej strony powyżej osi OX.\\
Ponadto w punkcie $-4$ wykres odbija się od osi poziomej.\\
A więc $$x \in \{-4\} \cup [10,11].$$
\rozwStop
\odpStart
$x \in \{-4\} \cup [10,11]$
\odpStop
\testStart
A.$x \in \{-4\} \cup [10,11]$\\
B.$x \in \{4\} \cup (10,11)$\\
C.$x \in \{-4\} \cup (10,11]$\\
D.$x \in \{4\} \cup (10,11]$\\
E.$x \in \{-4\} \cup [10,11)$\\
F.$x \in \{4\} \cup [10,11)$\\
G.$x \in \{-4\} \cup (10,11)$\\
H.$x \in \{4\} \cup [10,11]$
\testStop
\kluczStart
A
\kluczStop



\zadStart{Zadanie z Wikieł Z 1.62 c) moja wersja nr 277}

Rozwiązać nierówności $(10-x)(x+4)^{2}(12-x)^{3}\le0$.
\zadStop
\rozwStart{Patryk Wirkus}{Laura Mieczkowska}
Miejsca zerowe naszego wielomianu to: $10, -4, 12$.\\
Wielomian jest stopnia parzystego, ponadto znak współczynnika przy\linebreak najwyższej potędze x jest ujemny.\\ W związku z tym wykres wielomianu zaczyna się od lewej strony powyżej osi OX.\\
Ponadto w punkcie $-4$ wykres odbija się od osi poziomej.\\
A więc $$x \in \{-4\} \cup [10,12].$$
\rozwStop
\odpStart
$x \in \{-4\} \cup [10,12]$
\odpStop
\testStart
A.$x \in \{-4\} \cup [10,12]$\\
B.$x \in \{4\} \cup (10,12)$\\
C.$x \in \{-4\} \cup (10,12]$\\
D.$x \in \{4\} \cup (10,12]$\\
E.$x \in \{-4\} \cup [10,12)$\\
F.$x \in \{4\} \cup [10,12)$\\
G.$x \in \{-4\} \cup (10,12)$\\
H.$x \in \{4\} \cup [10,12]$
\testStop
\kluczStart
A
\kluczStop



\zadStart{Zadanie z Wikieł Z 1.62 c) moja wersja nr 278}

Rozwiązać nierówności $(10-x)(x+4)^{2}(13-x)^{3}\le0$.
\zadStop
\rozwStart{Patryk Wirkus}{Laura Mieczkowska}
Miejsca zerowe naszego wielomianu to: $10, -4, 13$.\\
Wielomian jest stopnia parzystego, ponadto znak współczynnika przy\linebreak najwyższej potędze x jest ujemny.\\ W związku z tym wykres wielomianu zaczyna się od lewej strony powyżej osi OX.\\
Ponadto w punkcie $-4$ wykres odbija się od osi poziomej.\\
A więc $$x \in \{-4\} \cup [10,13].$$
\rozwStop
\odpStart
$x \in \{-4\} \cup [10,13]$
\odpStop
\testStart
A.$x \in \{-4\} \cup [10,13]$\\
B.$x \in \{4\} \cup (10,13)$\\
C.$x \in \{-4\} \cup (10,13]$\\
D.$x \in \{4\} \cup (10,13]$\\
E.$x \in \{-4\} \cup [10,13)$\\
F.$x \in \{4\} \cup [10,13)$\\
G.$x \in \{-4\} \cup (10,13)$\\
H.$x \in \{4\} \cup [10,13]$
\testStop
\kluczStart
A
\kluczStop



\zadStart{Zadanie z Wikieł Z 1.62 c) moja wersja nr 279}

Rozwiązać nierówności $(10-x)(x+4)^{2}(14-x)^{3}\le0$.
\zadStop
\rozwStart{Patryk Wirkus}{Laura Mieczkowska}
Miejsca zerowe naszego wielomianu to: $10, -4, 14$.\\
Wielomian jest stopnia parzystego, ponadto znak współczynnika przy\linebreak najwyższej potędze x jest ujemny.\\ W związku z tym wykres wielomianu zaczyna się od lewej strony powyżej osi OX.\\
Ponadto w punkcie $-4$ wykres odbija się od osi poziomej.\\
A więc $$x \in \{-4\} \cup [10,14].$$
\rozwStop
\odpStart
$x \in \{-4\} \cup [10,14]$
\odpStop
\testStart
A.$x \in \{-4\} \cup [10,14]$\\
B.$x \in \{4\} \cup (10,14)$\\
C.$x \in \{-4\} \cup (10,14]$\\
D.$x \in \{4\} \cup (10,14]$\\
E.$x \in \{-4\} \cup [10,14)$\\
F.$x \in \{4\} \cup [10,14)$\\
G.$x \in \{-4\} \cup (10,14)$\\
H.$x \in \{4\} \cup [10,14]$
\testStop
\kluczStart
A
\kluczStop



\zadStart{Zadanie z Wikieł Z 1.62 c) moja wersja nr 280}

Rozwiązać nierówności $(10-x)(x+4)^{2}(15-x)^{3}\le0$.
\zadStop
\rozwStart{Patryk Wirkus}{Laura Mieczkowska}
Miejsca zerowe naszego wielomianu to: $10, -4, 15$.\\
Wielomian jest stopnia parzystego, ponadto znak współczynnika przy\linebreak najwyższej potędze x jest ujemny.\\ W związku z tym wykres wielomianu zaczyna się od lewej strony powyżej osi OX.\\
Ponadto w punkcie $-4$ wykres odbija się od osi poziomej.\\
A więc $$x \in \{-4\} \cup [10,15].$$
\rozwStop
\odpStart
$x \in \{-4\} \cup [10,15]$
\odpStop
\testStart
A.$x \in \{-4\} \cup [10,15]$\\
B.$x \in \{4\} \cup (10,15)$\\
C.$x \in \{-4\} \cup (10,15]$\\
D.$x \in \{4\} \cup (10,15]$\\
E.$x \in \{-4\} \cup [10,15)$\\
F.$x \in \{4\} \cup [10,15)$\\
G.$x \in \{-4\} \cup (10,15)$\\
H.$x \in \{4\} \cup [10,15]$
\testStop
\kluczStart
A
\kluczStop



\zadStart{Zadanie z Wikieł Z 1.62 c) moja wersja nr 281}

Rozwiązać nierówności $(10-x)(x+5)^{2}(11-x)^{3}\le0$.
\zadStop
\rozwStart{Patryk Wirkus}{Laura Mieczkowska}
Miejsca zerowe naszego wielomianu to: $10, -5, 11$.\\
Wielomian jest stopnia parzystego, ponadto znak współczynnika przy\linebreak najwyższej potędze x jest ujemny.\\ W związku z tym wykres wielomianu zaczyna się od lewej strony powyżej osi OX.\\
Ponadto w punkcie $-5$ wykres odbija się od osi poziomej.\\
A więc $$x \in \{-5\} \cup [10,11].$$
\rozwStop
\odpStart
$x \in \{-5\} \cup [10,11]$
\odpStop
\testStart
A.$x \in \{-5\} \cup [10,11]$\\
B.$x \in \{5\} \cup (10,11)$\\
C.$x \in \{-5\} \cup (10,11]$\\
D.$x \in \{5\} \cup (10,11]$\\
E.$x \in \{-5\} \cup [10,11)$\\
F.$x \in \{5\} \cup [10,11)$\\
G.$x \in \{-5\} \cup (10,11)$\\
H.$x \in \{5\} \cup [10,11]$
\testStop
\kluczStart
A
\kluczStop



\zadStart{Zadanie z Wikieł Z 1.62 c) moja wersja nr 282}

Rozwiązać nierówności $(10-x)(x+5)^{2}(12-x)^{3}\le0$.
\zadStop
\rozwStart{Patryk Wirkus}{Laura Mieczkowska}
Miejsca zerowe naszego wielomianu to: $10, -5, 12$.\\
Wielomian jest stopnia parzystego, ponadto znak współczynnika przy\linebreak najwyższej potędze x jest ujemny.\\ W związku z tym wykres wielomianu zaczyna się od lewej strony powyżej osi OX.\\
Ponadto w punkcie $-5$ wykres odbija się od osi poziomej.\\
A więc $$x \in \{-5\} \cup [10,12].$$
\rozwStop
\odpStart
$x \in \{-5\} \cup [10,12]$
\odpStop
\testStart
A.$x \in \{-5\} \cup [10,12]$\\
B.$x \in \{5\} \cup (10,12)$\\
C.$x \in \{-5\} \cup (10,12]$\\
D.$x \in \{5\} \cup (10,12]$\\
E.$x \in \{-5\} \cup [10,12)$\\
F.$x \in \{5\} \cup [10,12)$\\
G.$x \in \{-5\} \cup (10,12)$\\
H.$x \in \{5\} \cup [10,12]$
\testStop
\kluczStart
A
\kluczStop



\zadStart{Zadanie z Wikieł Z 1.62 c) moja wersja nr 283}

Rozwiązać nierówności $(10-x)(x+5)^{2}(13-x)^{3}\le0$.
\zadStop
\rozwStart{Patryk Wirkus}{Laura Mieczkowska}
Miejsca zerowe naszego wielomianu to: $10, -5, 13$.\\
Wielomian jest stopnia parzystego, ponadto znak współczynnika przy\linebreak najwyższej potędze x jest ujemny.\\ W związku z tym wykres wielomianu zaczyna się od lewej strony powyżej osi OX.\\
Ponadto w punkcie $-5$ wykres odbija się od osi poziomej.\\
A więc $$x \in \{-5\} \cup [10,13].$$
\rozwStop
\odpStart
$x \in \{-5\} \cup [10,13]$
\odpStop
\testStart
A.$x \in \{-5\} \cup [10,13]$\\
B.$x \in \{5\} \cup (10,13)$\\
C.$x \in \{-5\} \cup (10,13]$\\
D.$x \in \{5\} \cup (10,13]$\\
E.$x \in \{-5\} \cup [10,13)$\\
F.$x \in \{5\} \cup [10,13)$\\
G.$x \in \{-5\} \cup (10,13)$\\
H.$x \in \{5\} \cup [10,13]$
\testStop
\kluczStart
A
\kluczStop



\zadStart{Zadanie z Wikieł Z 1.62 c) moja wersja nr 284}

Rozwiązać nierówności $(10-x)(x+5)^{2}(14-x)^{3}\le0$.
\zadStop
\rozwStart{Patryk Wirkus}{Laura Mieczkowska}
Miejsca zerowe naszego wielomianu to: $10, -5, 14$.\\
Wielomian jest stopnia parzystego, ponadto znak współczynnika przy\linebreak najwyższej potędze x jest ujemny.\\ W związku z tym wykres wielomianu zaczyna się od lewej strony powyżej osi OX.\\
Ponadto w punkcie $-5$ wykres odbija się od osi poziomej.\\
A więc $$x \in \{-5\} \cup [10,14].$$
\rozwStop
\odpStart
$x \in \{-5\} \cup [10,14]$
\odpStop
\testStart
A.$x \in \{-5\} \cup [10,14]$\\
B.$x \in \{5\} \cup (10,14)$\\
C.$x \in \{-5\} \cup (10,14]$\\
D.$x \in \{5\} \cup (10,14]$\\
E.$x \in \{-5\} \cup [10,14)$\\
F.$x \in \{5\} \cup [10,14)$\\
G.$x \in \{-5\} \cup (10,14)$\\
H.$x \in \{5\} \cup [10,14]$
\testStop
\kluczStart
A
\kluczStop



\zadStart{Zadanie z Wikieł Z 1.62 c) moja wersja nr 285}

Rozwiązać nierówności $(10-x)(x+5)^{2}(15-x)^{3}\le0$.
\zadStop
\rozwStart{Patryk Wirkus}{Laura Mieczkowska}
Miejsca zerowe naszego wielomianu to: $10, -5, 15$.\\
Wielomian jest stopnia parzystego, ponadto znak współczynnika przy\linebreak najwyższej potędze x jest ujemny.\\ W związku z tym wykres wielomianu zaczyna się od lewej strony powyżej osi OX.\\
Ponadto w punkcie $-5$ wykres odbija się od osi poziomej.\\
A więc $$x \in \{-5\} \cup [10,15].$$
\rozwStop
\odpStart
$x \in \{-5\} \cup [10,15]$
\odpStop
\testStart
A.$x \in \{-5\} \cup [10,15]$\\
B.$x \in \{5\} \cup (10,15)$\\
C.$x \in \{-5\} \cup (10,15]$\\
D.$x \in \{5\} \cup (10,15]$\\
E.$x \in \{-5\} \cup [10,15)$\\
F.$x \in \{5\} \cup [10,15)$\\
G.$x \in \{-5\} \cup (10,15)$\\
H.$x \in \{5\} \cup [10,15]$
\testStop
\kluczStart
A
\kluczStop





\end{document}
