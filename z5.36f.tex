\documentclass[12pt, a4paper]{article}
\usepackage[utf8]{inputenc}
\usepackage{polski}

\usepackage{amsthm}  %pakiet do tworzenia twierdzeń itp.
\usepackage{amsmath} %pakiet do niektórych symboli matematycznych
\usepackage{amssymb} %pakiet do symboli mat., np. \nsubseteq
\usepackage{amsfonts}
\usepackage{graphicx} %obsługa plików graficznych z rozszerzeniem png, jpg
\theoremstyle{definition} %styl dla definicji
\newtheorem{zad}{} 
\title{Multizestaw zadań}
\author{Robert Fidytek}
%\date{\today}
\date{}
\newcounter{liczniksekcji}
\newcommand{\kategoria}[1]{\section{#1}} %olreślamy nazwę kateforii zadań
\newcommand{\zadStart}[1]{\begin{zad}#1\newline} %oznaczenie początku zadania
\newcommand{\zadStop}{\end{zad}}   %oznaczenie końca zadania
%Makra opcjonarne (nie muszą występować):
\newcommand{\rozwStart}[2]{\noindent \textbf{Rozwiązanie (autor #1 , recenzent #2): }\newline} %oznaczenie początku rozwiązania, opcjonarnie można wprowadzić informację o autorze rozwiązania zadania i recenzencie poprawności wykonania rozwiązania zadania
\newcommand{\rozwStop}{\newline}                                            %oznaczenie końca rozwiązania
\newcommand{\odpStart}{\noindent \textbf{Odpowiedź:}\newline}    %oznaczenie początku odpowiedzi końcowej (wypisanie wyniku)
\newcommand{\odpStop}{\newline}                                             %oznaczenie końca odpowiedzi końcowej (wypisanie wyniku)
\newcommand{\testStart}{\noindent \textbf{Test:}\newline} %ewentualne możliwe opcje odpowiedzi testowej: A. ? B. ? C. ? D. ? itd.
\newcommand{\testStop}{\newline} %koniec wprowadzania odpowiedzi testowych
\newcommand{\kluczStart}{\noindent \textbf{Test poprawna odpowiedź:}\newline} %klucz, poprawna odpowiedź pytania testowego (jedna literka): A lub B lub C lub D itd.
\newcommand{\kluczStop}{\newline} %koniec poprawnej odpowiedzi pytania testowego 
\newcommand{\wstawGrafike}[2]{\begin{figure}[h] \centering \includegraphics[scale=#2] {#1} \end{figure}} %gdyby była potrzeba wstawienia obrazka, parametry: nazwa pliku, skala (jak nie wiesz co wpisać, to wpisz 1)

\begin{document}
\maketitle

\kategoria{Wikieł/Z5.36f}

\zadStart{Zadanie z Wikieł Z 5.36 f) moja wersja nr [nrWersji]}
%[a]:[2,3,4,5,6,7,8,9,10,11]
Wyznaczyć przedziały wypukłości i wklęsłości podanej funkcji.
$$y = [a]x^4 e^{-x}$$
\zadStop

\rozwStart{Natalia Danieluk}{}
Postępujemy następująco:
\begin{enumerate}
\item Określamy dziedzinę funkcji: $\quad \mathcal{D}_f=\mathbb{R}$. \\
\item Obliczamy pochodne: 
$$\quad f'(x) = [a]x^3e^{-x}(4-x),\quad f''(x) = [a]x^2e^{-x}(x^2-8x+12)$$
i określamy ich dziedziny: $\quad \mathcal{D}_{f'}=\mathcal{D}_{f''}=\mathbb{R}$.\\
\item Badamy znak $f''$. \\
Zauważmy, że dla każdego $x \in \mathcal{D}_f$ mamy $[a]e^{-x} > 0$. \\
Wystarczy zatem zbadać znak czynnika $x^2(x^2-8x+12)$.
\newpage
$$\Delta = 16, \quad \sqrt{\Delta} = 4, \quad x_1 = 2, \quad x_2 = 6, \quad x_0 = 0 (k=2)$$
\wstawGrafike{wykres_z5_36f.png}{0.75}
	\begin{enumerate}
	\item $f''(x) > 0 \Leftrightarrow x \in (-\infty,0)\cup(0,2)\cup(6,\infty)$ i w tym przedziale wykres funkcji $f$ jest wypukły (wypukły w dół) $ \smile $ \\
	\item $f''(x) < 0 \Leftrightarrow x \in (2,6)$ i w tym przedziale wykres funkcji $f$ jest wklęsły (wypukły w górę) $ \frown $
	\end{enumerate}
\end{enumerate}
.
\rozwStop

\odpStart
Funkcja jest wypukła w $(-\infty,0)\cup(0,2)\cup(6,\infty)$ i wklęsła w $(2,6)$.
\odpStop

\testStart
A. Funkcja jest wypukła w całej dziedzinie.
B. Funkcja jest wklęsła w całej dziedzinie.
C. Funkcja nie jest ani wypukła, ani wklęsła.
D. Funkcja jest wypukła w $(2,6)$ i wklęsła w $(-\infty,0)\cup(0,2)\cup(6,\infty)$.
E. Funkcja jest wypukła w $(-\infty,0)\cup(0,2)\cup(6,\infty)$ i wklęsła w $(2,6)$.
F. Funkcja jest wypukła w $(0,\infty)$ i wklęsła w $(-\infty,0)$.
\testStop

\kluczStart
E
\kluczStop

\end{document}