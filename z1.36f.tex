\documentclass[12pt, a4paper]{article}
\usepackage[utf8]{inputenc}
\usepackage{polski}

\usepackage{amsthm}  %pakiet do tworzenia twierdzeń itp.
\usepackage{amsmath} %pakiet do niektórych symboli matematycznych
\usepackage{amssymb} %pakiet do symboli mat., np. \nsubseteq
\usepackage{amsfonts}
\usepackage{graphicx} %obsługa plików graficznych z rozszerzeniem png, jpg
\theoremstyle{definition} %styl dla definicji
\newtheorem{zad}{} 
\title{Multizestaw zadań}
\author{Laura Mieczkowska}
%\date{\today}
\date{}
\newcounter{liczniksekcji}
\newcommand{\kategoria}[1]{\section{#1}} %olreślamy nazwę kateforii zadań
\newcommand{\zadStart}[1]{\begin{zad}#1\newline} %oznaczenie początku zadania
\newcommand{\zadStop}{\end{zad}}   %oznaczenie końca zadania
%Makra opcjonarne (nie muszą występować):
\newcommand{\rozwStart}[2]{\noindent \textbf{Rozwiązanie (autor #1 , recenzent #2): }\newline} %oznaczenie początku rozwiązania, opcjonarnie można wprowadzić informację o autorze rozwiązania zadania i recenzencie poprawności wykonania rozwiązania zadania
\newcommand{\rozwStop}{\newline}                                            %oznaczenie końca rozwiązania
\newcommand{\odpStart}{\noindent \textbf{Odpowiedź:}\newline}    %oznaczenie początku odpowiedzi końcowej (wypisanie wyniku)
\newcommand{\odpStop}{\newline}                                             %oznaczenie końca odpowiedzi końcowej (wypisanie wyniku)
\newcommand{\testStart}{\noindent \textbf{Test:}\newline} %ewentualne możliwe opcje odpowiedzi testowej: A. ? B. ? C. ? D. ? itd.
\newcommand{\testStop}{\newline} %koniec wprowadzania odpowiedzi testowych
\newcommand{\kluczStart}{\noindent \textbf{Test poprawna odpowiedź:}\newline} %klucz, poprawna odpowiedź pytania testowego (jedna literka): A lub B lub C lub D itd.
\newcommand{\kluczStop}{\newline} %koniec poprawnej odpowiedzi pytania testowego 
\newcommand{\wstawGrafike}[2]{\begin{figure}[h] \includegraphics[scale=#2] {#1} \end{figure}} %gdyby była potrzeba wstawienia obrazka, parametry: nazwa pliku, skala (jak nie wiesz co wpisać, to wpisz 1)

\begin{document}
\maketitle


\kategoria{Wikieł/Z1.36d}
\zadStart{Zadanie z Wikieł Z 1.36 d) moja wersja nr [nrWersji]}
%[a]:[2,3,4,5,6,7,8,9,10,11,12,13,14,15]
%[b]:[2,3,4,5,6,7,8,9,10,11,12,13,14,15]
%[c]:[2,3,4,5,6,7,8,9,10,11,12,13,14,15]
%[4ac]=4*[a]*[c]
%[bb]=[b]*[b]
%[d]=[bb]-[4ac]
%[p]=(pow([d],1/2))
%[p1]=[p].real
%[p2]=int([p1])
%[m]=2*[a]
%[l1]=-[b]-[p2]
%[l2]=-[b]+[p2]
%[dziel1]=math.gcd([l1],[m])
%[dziel2]=math.gcd([l2],[m])
%[licz1]=int([l1]/[dziel1])
%[licz2]=int([l2]/[dziel2])
%[mian1]=int([m]/[dziel1])
%[mian2]=int([m]/[dziel2])
%[ul1]=[licz1]/[mian1]
%[ul2]=[licz2]/[mian2]
%[bb]>[4ac] and [p].is_integer()==True and [ul1].is_integer()==False and [ul2].is_integer()==False
Rozwiązać równanie $[a]x^2+[b]x+[c]=0$.
\zadStop
\rozwStart{Laura Mieczkowska}{}
$$[a]x^2+[b]x+[c]=0 $$ 
$$\triangle=[b]^2-4\cdot[a]\cdot[c] \Leftrightarrow \triangle=[d] \Leftrightarrow \sqrt{\triangle}=[p2]$$
$$x=\frac{-[b]-[p2]}{[m]} \vee x=\frac{-[b]+[p2]}{[m]}$$
$$x=\frac{[l1]}{[m]} \vee x=\frac{[l2]}{[m]}$$
$$x=\frac{[licz1]}{[mian1]} \vee x=\frac{[licz2]}{[mian2]}$$

\odpStart
$x=\frac{[licz1]}{[mian1]} \vee x=\frac{[licz2]}{[mian2]}$
\odpStop
\testStart
A. $x=\frac{[licz1]}{[mian1]} \vee x=\frac{[licz2]}{[mian2]}$ \\
B. $x=\frac{[licz1]}{[mian1]} \vee x=0$ \\
C. $x=1 \vee x=\frac{[licz2]}{[mian2]}$ \\
D. $x=1 \vee x=-1$ 
\testStop
\kluczStart
A
\kluczStop



\end{document}