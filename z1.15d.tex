\documentclass[12pt, a4paper]{article}
\usepackage[utf8]{inputenc}
\usepackage{polski}

\usepackage{amsthm}  %pakiet do tworzenia twierdzeń itp.
\usepackage{amsmath} %pakiet do niektórych symboli matematycznych
\usepackage{amssymb} %pakiet do symboli mat., np. \nsubseteq
\usepackage{amsfonts}
\usepackage{graphicx} %obsługa plików graficznych z rozszerzeniem png, jpg
\theoremstyle{definition} %styl dla definicji
\newtheorem{zad}{} 
\title{Multizestaw zadań}
\author{Robert Fidytek}
%\date{\today}
\date{}
\newcounter{liczniksekcji}
\newcommand{\kategoria}[1]{\section{#1}} %olreślamy nazwę kateforii zadań
\newcommand{\zadStart}[1]{\begin{zad}#1\newline} %oznaczenie początku zadania
\newcommand{\zadStop}{\end{zad}}   %oznaczenie końca zadania
%Makra opcjonarne (nie muszą występować):
\newcommand{\rozwStart}[2]{\noindent \textbf{Rozwiązanie (autor #1 , recenzent #2): }\newline} %oznaczenie początku rozwiązania, opcjonarnie można wprowadzić informację o autorze rozwiązania zadania i recenzencie poprawności wykonania rozwiązania zadania
\newcommand{\rozwStop}{\newline}                                            %oznaczenie końca rozwiązania
\newcommand{\odpStart}{\noindent \textbf{Odpowiedź:}\newline}    %oznaczenie początku odpowiedzi końcowej (wypisanie wyniku)
\newcommand{\odpStop}{\newline}                                             %oznaczenie końca odpowiedzi końcowej (wypisanie wyniku)
\newcommand{\testStart}{\noindent \textbf{Test:}\newline} %ewentualne możliwe opcje odpowiedzi testowej: A. ? B. ? C. ? D. ? itd.
\newcommand{\testStop}{\newline} %koniec wprowadzania odpowiedzi testowych
\newcommand{\kluczStart}{\noindent \textbf{Test poprawna odpowiedź:}\newline} %klucz, poprawna odpowiedź pytania testowego (jedna literka): A lub B lub C lub D itd.
\newcommand{\kluczStop}{\newline} %koniec poprawnej odpowiedzi pytania testowego 
\newcommand{\wstawGrafike}[2]{\begin{figure}[h] \includegraphics[scale=#2] {#1} \end{figure}} %gdyby była potrzeba wstawienia obrazka, parametry: nazwa pliku, skala (jak nie wiesz co wpisać, to wpisz 1)

\begin{document}
\maketitle



\kategoria{Wikieł/Z1.15d}
\zadStart{Zadanie z Wikieł Z 1.15 d) moja wersja nr [nrWersji]}
%[a]:[2,3,4]
%[b]:[2,3,4]
%[d]:[2,3,4]
%[a]=random.randint(2,5)
%[b]=random.randint(1,5)
%[d]=random.randint(1,5)
%[c]=[d]
%[cpd]=[c]+[d]
%[cmd]=[c]-[d]
%[cpdpb]=[cpd]+[b]
%[cpdmb]=[cpd]-[b]
%[cpdpba]=round([cpdpb]/[a],2)
%[cpdmba]=round([cpdmb]/[a],2)
%[cpdpba]!=0 and [cpdmba]!=0
Rozwiązać nierówność $\big||[a]x-[b]|-[c]\big|\leq[d]$
\zadStop
\rozwStart{Pascal Nawrocki}{Jakub Ulrych}
Rozwiązujemy:
$$\big||[a]x-[b]|-[c]\big|\leq[d]$$
Jako, że pierwsza wartość bezwględna jest mniejsza/równa to po rozbiciu na przypadki bierzemy ich koniunkcję:
$$|[a]x-[b]|-[c]\leq[d] \wedge |[a]x-[b]|-[c]\geq-[d]$$
$$|[a]x-[b]|\leq[cpd] \wedge |[a]x-[b]|\geq[cmd]$$
Teraz liczymy te dwa przypadki. Zauważmy, że:
$$|[a]x-[b]|\geq[cmd]$$ Jest zawsze prawdą dla $x\in\mathbb{R}$ zatem tutaj otrzymujemy cały zbiór liczb rzeczywistych.
Zatem naszym rozwiązaniem jest zbiór, który wyznacza nam pierwsza nierówność:
$$|[a]x-[b]|\leq[cpd]$$
$$[a]x-[b]\leq[cpd]\wedge[a]x-[b]\geq-[cpd]$$
$$[a]x\leq[cpdpb]\wedge[a]x\geq-[cpdmb]$$
$$x\leq[cpdpba]\wedge x\geq-[cpdmba]$$
Zatem otrzymujemy zbiór: $x\in[-[cpdmba],[cpdpba]]$
\odpStop
\testStart
A.$x\in[-[cpdmba],[cpdpba]]$
B.$x\in \emptyset$
C.$\infty$
D. $x\in (-\infty,-[a])$
\testStop
\kluczStart
A
\kluczStop


\end{document}