\documentclass[12pt, a4paper]{article}
\usepackage[utf8]{inputenc}
\usepackage{polski}

\usepackage{amsthm}  %pakiet do tworzenia twierdzeń itp.
\usepackage{amsmath} %pakiet do niektórych symboli matematycznych
\usepackage{amssymb} %pakiet do symboli mat., np. \nsubseteq
\usepackage{amsfonts}
\usepackage{graphicx} %obsługa plików graficznych z rozszerzeniem png, jpg
\theoremstyle{definition} %styl dla definicji
\newtheorem{zad}{} 
\title{Multizestaw zadań}
\author{Robert Fidytek}
%\date{\today}
\date{}
\newcounter{liczniksekcji}
\newcommand{\kategoria}[1]{\section{#1}} %olreślamy nazwę kateforii zadań
\newcommand{\zadStart}[1]{\begin{zad}#1\newline} %oznaczenie początku zadania
\newcommand{\zadStop}{\end{zad}}   %oznaczenie końca zadania
%Makra opcjonarne (nie muszą występować):
\newcommand{\rozwStart}[2]{\noindent \textbf{Rozwiązanie (autor #1 , recenzent #2): }\newline} %oznaczenie początku rozwiązania, opcjonarnie można wprowadzić informację o autorze rozwiązania zadania i recenzencie poprawności wykonania rozwiązania zadania
\newcommand{\rozwStop}{\newline}                                            %oznaczenie końca rozwiązania
\newcommand{\odpStart}{\noindent \textbf{Odpowiedź:}\newline}    %oznaczenie początku odpowiedzi końcowej (wypisanie wyniku)
\newcommand{\odpStop}{\newline}                                             %oznaczenie końca odpowiedzi końcowej (wypisanie wyniku)
\newcommand{\testStart}{\noindent \textbf{Test:}\newline} %ewentualne możliwe opcje odpowiedzi testowej: A. ? B. ? C. ? D. ? itd.
\newcommand{\testStop}{\newline} %koniec wprowadzania odpowiedzi testowych
\newcommand{\kluczStart}{\noindent \textbf{Test poprawna odpowiedź:}\newline} %klucz, poprawna odpowiedź pytania testowego (jedna literka): A lub B lub C lub D itd.
\newcommand{\kluczStop}{\newline} %koniec poprawnej odpowiedzi pytania testowego 
\newcommand{\wstawGrafike}[2]{\begin{figure}[h] \includegraphics[scale=#2] {#1} \end{figure}} %gdyby była potrzeba wstawienia obrazka, parametry: nazwa pliku, skala (jak nie wiesz co wpisać, to wpisz 1)

\begin{document}
\maketitle


\kategoria{Wikieł/Z3.10}
\zadStart{Zadanie z Wikieł Z 3.10 ) moja wersja nr [nrWersji]}
%[a]:[1,2,3,4,6,7,8,10,11,12,14]
%[b]:[2,3,4,6,7,8,10,11,12,14]
%[c]:[1,2,3,4,6,7,8,10,11,12,14]
%[p1]=random.randint(1,10)
%[ac]=[a]+[c]
%[p1c]=[p1]+[c]
%math.gcd(([a]+[c]),[b])==[b] and (([a]+[c])/[b])==[c]
Ciąg $(a_{n})$ jest zdefiniowany rekurencyjnie w następujący sposób:
$$
 \left\{ \begin{array}{ll}
a_{1}= [p1] & \\
a_{n+1}=[b]a_{n}+[a] & \mbox{dla }n\geq1
\end{array} \right.
$$
a) Wykazać, że ciąg $(b_{n})$ określony wzorem $b_{n}=a_{n}+[c]$ dla $n \in \mathbb{N}$ jest ciągiem geometrycznym oraz wyznaczyć wyraz ogólny tego ciągu.\\
b) Wyznaczyć wzór na wyraz ogólny ciągu ($a_{n}$).
\zadStop
\rozwStart{Wojciech Przybylski}{}
a) Zauważamy że dla każdego $n \in \mathbb{N}$ zachodzi
$$\frac{b_{n+1}}{b_{n}}=\frac{a_{n+1}+[c]}{a_{n}+[c]}=\frac{[b]a_{n}+[a]+[c]}{a_{n}+[c]}=\frac{[b]a_{n}+[ac]}{a_{n}+[c]}=$$
$$=\frac{[b](a_{n}+[c])}{a_{n}+[c]}=[b] $$
Stąd dla każdego $n \in \mathbb{N}$ mamy $b_{n+1}=[b]b_{n}$, co oznacza, że ciąg $(b_{n})$ jest ciągiem geometrycznym o wyrazie pierwszym $ b_{1}=[p1c]$ oraz ilorazie $q=[b]$.
 Czyli dla każdego $n \in \mathbb{N}$ wyraz ogólny tego ciągu wyraża się wzorem: 
$$ b_{n}=b_{1}q^{n-1}=[p1c]\cdot [b]^{n-1}$$ 
b) Ponieważ dla każdego $n \in \mathbb{N}$ mamy $b_{n}=a_{n}+[c]$, to korzystając z rozwiązania z poprzedniego podpunktu, mamy:
$$a_{n}=b_{n}-[c]=[p1c]\cdot[b]^{n-1}-[c]$$
\rozwStop
\odpStart
a)$b_{n}$ jest ciągiem geometrycznym i $b_{n}=[p1c]\cdot[b]^{n-1}$ \\
b)wzór ogólny ciągu $a_{n}=[p1c]\cdot[b]^{n-1}-[c]$.
\odpStop
\testStart
A. a) $b_{n}$ jest ciągiem geometrycznym i $b_{n}=[p1c]\cdot[b]^{n-1}$ \\
     b) wzór ogólny ciągu $a_{n}=[p1c]\cdot[b]^{n-1}-[c]$.\\
\\
B. a) $b_{n}$ nie jest ciągiem geometrycznym\\
     b) wzór ogólny ciągu $a_{n}=[p1c]\cdot[b]^{n-1}$.\\
\\
C. a) $b_{n}$ jest ciągiem geometrycznym i $b_{n}=[p1c]\cdot[b]^{n}$ \\
     b) wzór ogólny ciągu $a_{n}=[p1c]\cdot[b]^{n}-[c]$.\\
\\
D. a) $b_{n}$ nie jest ciągiem geometrycznym i $b_{n}=[p1c]\cdot[b]^{n}$ \\
     b) wzór ogólny ciągu $a_{n}=[p1c]\cdot[b]^{n}-[c]$.\\
\\
E. a) $b_{n}$ nie jest ciągiem geometrycznym i $b_{n}=[p1c]\cdot[b]^{n-1}$ \\
     b) wzór ogólny ciągu $a_{n}=[b]^{n-1}-[c]$.\\
\testStop
\kluczStart
A
\kluczStop



\end{document}