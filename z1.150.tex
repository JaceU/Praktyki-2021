\documentclass[12pt, a4paper]{article}
\usepackage[utf8]{inputenc}
\usepackage{polski}

\usepackage{amsthm}  %pakiet do tworzenia twierdzeń itp.
\usepackage{amsmath} %pakiet do niektórych symboli matematycznych
\usepackage{amssymb} %pakiet do symboli mat., np. \nsubseteq
\usepackage{amsfonts}
\usepackage{graphicx} %obsługa plików graficznych z rozszerzeniem png, jpg
\theoremstyle{definition} %styl dla definicji
\newtheorem{zad}{} 
\title{Multizestaw zadań}
\author{Robert Fidytek}
%\date{\today}
\date{}\documentclass[12pt, a4paper]{article}
\usepackage[utf8]{inputenc}
\usepackage{polski}

\usepackage{amsthm}  %pakiet do tworzenia twierdzeń itp.
\usepackage{amsmath} %pakiet do niektórych symboli matematycznych
\usepackage{amssymb} %pakiet do symboli mat., np. \nsubseteq
\usepackage{amsfonts}
\usepackage{graphicx} %obsługa plików graficznych z rozszerzeniem png, jpg
\theoremstyle{definition} %styl dla definicji
\newtheorem{zad}{} 
\title{Multizestaw zadań}
\author{Robert Fidytek}
%\date{\today}
\date{}
\newcounter{liczniksekcji}
\newcommand{\kategoria}[1]{\section{#1}} %olreślamy nazwę kateforii zadań
\newcommand{\zadStart}[1]{\begin{zad}#1\newline} %oznaczenie początku zadania
\newcommand{\zadStop}{\end{zad}}   %oznaczenie końca zadania
%Makra opcjonarne (nie muszą występować):
\newcommand{\rozwStart}[2]{\noindent \textbf{Rozwiązanie (autor #1 , recenzent #2): }\newline} %oznaczenie początku rozwiązania, opcjonarnie można wprowadzić informację o autorze rozwiązania zadania i recenzencie poprawności wykonania rozwiązania zadania
\newcommand{\rozwStop}{\newline}                                            %oznaczenie końca rozwiązania
\newcommand{\odpStart}{\noindent \textbf{Odpowiedź:}\newline}    %oznaczenie początku odpowiedzi końcowej (wypisanie wyniku)
\newcommand{\odpStop}{\newline}                                             %oznaczenie końca odpowiedzi końcowej (wypisanie wyniku)
\newcommand{\testStart}{\noindent \textbf{Test:}\newline} %ewentualne możliwe opcje odpowiedzi testowej: A. ? B. ? C. ? D. ? itd.
\newcommand{\testStop}{\newline} %koniec wprowadzania odpowiedzi testowych
\newcommand{\kluczStart}{\noindent \textbf{Test poprawna odpowiedź:}\newline} %klucz, poprawna odpowiedź pytania testowego (jedna literka): A lub B lub C lub D itd.
\newcommand{\kluczStop}{\newline} %koniec poprawnej odpowiedzi pytania testowego 
\newcommand{\wstawGrafike}[2]{\begin{figure}[h] \includegraphics[scale=#2] {#1} \end{figure}} %gdyby była potrzeba wstawienia obrazka, parametry: nazwa pliku, skala (jak nie wiesz co wpisać, to wpisz 1)

\begin{document}
\maketitle


\kategoria{Wikieł/Z1.150}
\zadStart{Zadanie z Wikieł Z 1.150  moja wersja nr [nrWersji]}
%[p1]:[2,3,4,5,6,7,8,9,10]
%[p2]:[2,3,4,5,6,7,8,9,10]
%[p3]:[2,3,4,5,6,7,8,9,10]
%[p1k]=[p1]*[p1]
%[2p1p2]=2*[p1]*[p2]
%[p2k]=[p2]*[p2]
%[a]=-4*[p1k]*[p2k]
%[c]=[2p1p2]*[2p1p2]+[a]
%[b]=2*[2p1p2]
%[bk]=[b]*[b]
%[del]=[bk]
%[pdel]=round(math.sqrt([del]),2)
%[m1]=round(([b]-[pdel])/2,2)
%[m2]=round(([b]+[pdel])/2,2)
%math.gcd([p1],[p2])==1 and [del]>0 


Zbadać ilość rozwiązań równania $\frac{\log(mx)}{\log([p1]x+[p2])}=2 $ w zależności od parametru $m$.
\zadStop

\rozwStart{Maja Szabłowska}{}
Początkowo sprawdzamy dziedzinę rozwiązań.
$$mx>0 \quad \land \quad [p1]x+[p2]>0$$
$$(m>0 \land x>0) \quad \lor \quad  (m<0 \land x<0) \quad \land \quad x>-\frac{[p2]}{[p1]}$$

$$\frac{\log(mx)}{\log([p1]x+[p2])}=2 $$
$$\log(mx)=2\cdot \log([p1]x+[p2])$$
$$mx=([p1]x+[p2])^{2}$$
$$mx=[p1k]x^{2}+[2p1p2]x+[p2k]$$
$$[p1k]x^{2}+([2p1p2]-m)x+[p2k]=0$$
$$\Delta=([2p1p2]-m)^{2}-4\cdot[p1k]\cdot[p2k]=([2p1p2]-m)^{2}[a]$$

\begin{enumerate}
    \item $\Delta<0$, czyli brak rozwiązań w zbiorze liczb rzeczywistych.
    $$([2p1p2]-m)^{2}[a]<0$$
    $$[c]-[b]m+m^{2}<0$$
    $$\Delta=(-[b])^{2}-4\cdot[c]\cdot1=[bk]-0=
    [del] \Rightarrow \sqrt{\Delta}=[pdel]$$
    $$m_{1}=\frac{[b]-[pdel]}{2}=[m1], \quad m_{2}=\frac{[b]+[pdel]}{2}=[m2]$$
    $$m\in([m1],[m2])$$
    
    \item $\Delta=0$, czyli jedno podwójne rozwiązanie w zbiorze liczb rzeczywistych.
    $$([2p1p2]-m)^{2}[a]=0$$
    $$m_{1}=0  \quad \lor \quad m_{2}=[m2]$$
    Gdzie $m_{1}=0$ jest niezgodne z początkowymi założeniami.
    
    \item $\Delta>0$, czyli dwa różne rozwiązania w zbiorze liczb rzeczywistych.
    $$([2p1p2]-m)^{2}[a]>0$$
    $$m\in(-\infty,0)\cup([m2],\infty)$$
\end{enumerate}
\rozwStop
\odpStart
Brak rozwiązań dla $m\in([m1],[m2])$, jedno rozwiązanie dla $m=[m2]$, dwa rozwiązania dla $m\in(-\infty,0)\cup([m2],\infty)$.
\odpStop
\testStart
A.Brak rozwiązań dla $m\in([m1],[m2])$, jedno rozwiązanie dla $m=[m2]$, dwa rozwiązania dla $m\in(-\infty,0)\cup([m2],\infty)$.
B.Brak rozwiązań dla $m\in(-\infty,0)\cup([m2],\infty)$, jedno rozwiązanie dla $m=[m2]$, dwa rozwiązania dla $m\in([m1],[m2])$.
C.Brak rozwiązań dla $m\in\mathbb{R}$.
D.Brak rozwiązań dla $m\in(-[m2],[m2])$, jedno rozwiązanie dla $m=[m2]$, dwa rozwiązania dla $m\in(-\infty,0)\cup([m2],\infty)$.
E.Brak rozwiązań dla $m\in([m1],[m2])$, jedno rozwiązanie dla $m=[m1]$, dwa rozwiązania dla $m\in(-\infty,0)\cup([m2],\infty)$.
F.Brak rozwiązań dla $m\in([m1],[m2])$, jedno rozwiązanie dla $m=[m1]$ oraz  $m=[m2]$, dwa rozwiązania dla $m\in(-\infty,0)\cup([m2],\infty)$.
G.Brak rozwiązań dla $m\in(-[b],[c])$, jedno rozwiązanie dla $m=[m1]$, dwa rozwiązania dla $m\in(-\infty,0)\cup([m2],\infty)$.

\testStop
\kluczStart
A
\kluczStop



\end{document}
