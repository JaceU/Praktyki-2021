\documentclass[12pt, a4paper]{article}
\usepackage[utf8]{inputenc}
\usepackage{polski}

\usepackage{amsthm}  %pakiet do tworzenia twierdzeń itp.
\usepackage{amsmath} %pakiet do niektórych symboli matematycznych
\usepackage{amssymb} %pakiet do symboli mat., np. \nsubseteq
\usepackage{amsfonts}
\usepackage{graphicx} %obsługa plików graficznych z rozszerzeniem png, jpg
\theoremstyle{definition} %styl dla definicji
\newtheorem{zad}{} 
\title{Multizestaw zadań}
\author{Robert Fidytek}
%\date{\today}
\date{}
\newcounter{liczniksekcji}
\newcommand{\kategoria}[1]{\section{#1}} %olreślamy nazwę kateforii zadań
\newcommand{\zadStart}[1]{\begin{zad}#1\newline} %oznaczenie początku zadania
\newcommand{\zadStop}{\end{zad}}   %oznaczenie końca zadania
%Makra opcjonarne (nie muszą występować):
\newcommand{\rozwStart}[2]{\noindent \textbf{Rozwiązanie (autor #1 , recenzent #2): }\newline} %oznaczenie początku rozwiązania, opcjonarnie można wprowadzić informację o autorze rozwiązania zadania i recenzencie poprawności wykonania rozwiązania zadania
\newcommand{\rozwStop}{\newline}                                            %oznaczenie końca rozwiązania
\newcommand{\odpStart}{\noindent \textbf{Odpowiedź:}\newline}    %oznaczenie początku odpowiedzi końcowej (wypisanie wyniku)
\newcommand{\odpStop}{\newline}                                             %oznaczenie końca odpowiedzi końcowej (wypisanie wyniku)
\newcommand{\testStart}{\noindent \textbf{Test:}\newline} %ewentualne możliwe opcje odpowiedzi testowej: A. ? B. ? C. ? D. ? itd.
\newcommand{\testStop}{\newline} %koniec wprowadzania odpowiedzi testowych
\newcommand{\kluczStart}{\noindent \textbf{Test poprawna odpowiedź:}\newline} %klucz, poprawna odpowiedź pytania testowego (jedna literka): A lub B lub C lub D itd.
\newcommand{\kluczStop}{\newline} %koniec poprawnej odpowiedzi pytania testowego 
\newcommand{\wstawGrafike}[2]{\begin{figure}[h] \includegraphics[scale=#2] {#1} \end{figure}} %gdyby była potrzeba wstawienia obrazka, parametry: nazwa pliku, skala (jak nie wiesz co wpisać, to wpisz 1)

\begin{document}
\maketitle


\kategoria{Wikieł/Z4.10f}
\zadStart{Zadanie z Wikieł Z 4.10 f) moja wersja nr [nrWersji]}
%[a]:[2,3,4,5,6,7,8]
%[b]:[2,3,4,5,6,7,8]
%[c]:[2,3,4,5,6,7,8]
%[a]=random.randint(2,16)
%[b]=random.randint(2,16) 
%[c]=random.randint(2,16) 
%[bpc]=[b]+[c]
Obliczyć granicę funkcji $\lim_{x \to \infty}\big(\frac{[a]x+[b]}{[a]x-[c]}\big)^{x^{2}}$.
\zadStop
\rozwStart{Jakub Ulrych}{Pascal Nawrocki}
$$\lim_{x \to \infty}\big(\frac{[a]x+[b]}{[a]x-[c]}\big)^{x^{2}}$$
$$\lim_{x \to \infty}\bigg(\frac{[a]x(1+\frac{[b]}{[a]x})}{[a]x(1-\frac{[c]}{[a]x})}\bigg)^{x^{2}}$$
$$\lim_{x \to \infty}\frac{(1+\frac{[b]}{[a]x})^{x^{2}}}{(1-\frac{[c]}{[a]x})^{x^{2}}}$$
$$\lim_{x \to \infty}\frac{\big((1+\frac{[b]}{[a]x})^{\frac{[a]x}{[b]}}\big)^{\frac{[b]x}{[a]}}}{\big((1-\frac{[c]}{[a]x})^{\frac{-[a]x}{[c]}}\big)^{\frac{-[c]x}{[a]}}}$$
$$\lim_{x \to \infty}\frac{e^{\frac{[b]x}{[a]}}}{e^{-\frac{[c]x}{[a]}}}$$
$$\lim_{x \to \infty}e^{\frac{[b]x}{[a]}+\frac{[c]x}{[a]}}=\lim_{x \to \infty}e^{\frac{[bpc]x}{[a]}}=\infty$$
\rozwStop
\odpStart
$$\infty$$
\odpStop
\testStart
A.$\infty$
B.$e$
C.$-\infty$
D.$e^{2}$
\testStop
\kluczStart
A
\kluczStop



\end{document}