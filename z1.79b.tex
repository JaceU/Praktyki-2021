\documentclass[12pt, a4paper]{article}
\usepackage[utf8]{inputenc}
\usepackage{polski}

\usepackage{amsthm}  %pakiet do tworzenia twierdzeń itp.
\usepackage{amsmath} %pakiet do niektórych symboli matematycznych
\usepackage{amssymb} %pakiet do symboli mat., np. \nsubseteq
\usepackage{amsfonts}
\usepackage{graphicx} %obsługa plików graficznych z rozszerzeniem png, jpg
\theoremstyle{definition} %styl dla definicji
\newtheorem{zad}{} 
\title{Multizestaw zadań}
\author{Robert Fidytek}
%\date{\today}
\date{}
\newcounter{liczniksekcji}
\newcommand{\kategoria}[1]{\section{#1}} %olreślamy nazwę kateforii zadań
\newcommand{\zadStart}[1]{\begin{zad}#1\newline} %oznaczenie początku zadania
\newcommand{\zadStop}{\end{zad}}   %oznaczenie końca zadania
%Makra opcjonarne (nie muszą występować):
\newcommand{\rozwStart}[2]{\noindent \textbf{Rozwiązanie (autor #1 , recenzent #2): }\newline} %oznaczenie początku rozwiązania, opcjonarnie można wprowadzić informację o autorze rozwiązania zadania i recenzencie poprawności wykonania rozwiązania zadania
\newcommand{\rozwStop}{\newline}                                            %oznaczenie końca rozwiązania
\newcommand{\odpStart}{\noindent \textbf{Odpowiedź:}\newline}    %oznaczenie początku odpowiedzi końcowej (wypisanie wyniku)
\newcommand{\odpStop}{\newline}                                             %oznaczenie końca odpowiedzi końcowej (wypisanie wyniku)
\newcommand{\testStart}{\noindent \textbf{Test:}\newline} %ewentualne możliwe opcje odpowiedzi testowej: A. ? B. ? C. ? D. ? itd.
\newcommand{\testStop}{\newline} %koniec wprowadzania odpowiedzi testowych
\newcommand{\kluczStart}{\noindent \textbf{Test poprawna odpowiedź:}\newline} %klucz, poprawna odpowiedź pytania testowego (jedna literka): A lub B lub C lub D itd.
\newcommand{\kluczStop}{\newline} %koniec poprawnej odpowiedzi pytania testowego 
\newcommand{\wstawGrafike}[2]{\begin{figure}[h] \includegraphics[scale=#2] {#1} \end{figure}} %gdyby była potrzeba wstawienia obrazka, parametry: nazwa pliku, skala (jak nie wiesz co wpisać, to wpisz 1)

\begin{document}
\maketitle


\kategoria{Wikieł/Z1.79b}
\zadStart{Zadanie z Wikieł Z 1.79 b) moja wersja nr [nrWersji]}
%[a]:[2,3,4,5,6,7,8,9]
%[b]:[3,4,5,6,7,8,9,10]
%[c]:[2,3,4,5,6,7,8,9]
%[d]=math.gcd([a],[b])
%[a1]=int([a]/[d])
%[b1]=int([b]/[d])
%[a]<[b]
Rozwiązać nierówność
$$\sqrt{[a]-[b]x}<-[c].$$
\zadStop
\rozwStart{Adrianna Stobiecka}{}
Zakładamy $[a]-[b]x\geq0$.
$$[a]-[b]x\geq0\Leftrightarrow[a]\geq[b]x\Leftrightarrow x\leq\frac{[a1]}{[b1]}$$
Zatem $x\leq\frac{[a1]}{[b1]}$. Lewa strona nierówności jest nieujemna, a prawa ujemna, więc nierówność jest sprzeczna. Zatem nierówność nie ma rozwiązań.
\rozwStop
\odpStart
$x\in\emptyset$
\odpStop
\testStart
A.$x\in\{-\frac{[a1]}{[b1]}\}$
B.$x\in\mathbb{R}$
C.$x\in(-\infty,\frac{[a1]}{[b1]})$
D.$x\in\emptyset$
E.$x\in\{\frac{[a1]}{[b1]}\}$
F.$x\in[\frac{[a1]}{[b1]},\infty)$
G.$x\in\mathbb{R}\setminus\{0\}$
H.$x\in(-\infty,\frac{[a1]}{[b1]}]$
I.$x\in\{\frac{[b1]}{[a1]}\}$
\testStop
\kluczStart
D
\kluczStop



\end{document}
