\documentclass[12pt, a4paper]{article}
\usepackage[utf8]{inputenc}
\usepackage{polski}

\usepackage{amsthm}  %pakiet do tworzenia twierdzeń itp.
\usepackage{amsmath} %pakiet do niektórych symboli matematycznych
\usepackage{amssymb} %pakiet do symboli mat., np. \nsubseteq
\usepackage{amsfonts}
\usepackage{graphicx} %obsługa plików graficznych z rozszerzeniem png, jpg
\theoremstyle{definition} %styl dla definicji
\newtheorem{zad}{} 
\title{Multizestaw zadań}
\author{Robert Fidytek}
%\date{\today}
\date{}
\newcounter{liczniksekcji}
\newcommand{\kategoria}[1]{\section{#1}} %olreślamy nazwę kateforii zadań
\newcommand{\zadStart}[1]{\begin{zad}#1\newline} %oznaczenie początku zadania
\newcommand{\zadStop}{\end{zad}}   %oznaczenie końca zadania
%Makra opcjonarne (nie muszą występować):
\newcommand{\rozwStart}[2]{\noindent \textbf{Rozwiązanie (autor #1 , recenzent #2): }\newline} %oznaczenie początku rozwiązania, opcjonarnie można wprowadzić informację o autorze rozwiązania zadania i recenzencie poprawności wykonania rozwiązania zadania
\newcommand{\rozwStop}{\newline}                                            %oznaczenie końca rozwiązania
\newcommand{\odpStart}{\noindent \textbf{Odpowiedź:}\newline}    %oznaczenie początku odpowiedzi końcowej (wypisanie wyniku)
\newcommand{\odpStop}{\newline}                                             %oznaczenie końca odpowiedzi końcowej (wypisanie wyniku)
\newcommand{\testStart}{\noindent \textbf{Test:}\newline} %ewentualne możliwe opcje odpowiedzi testowej: A. ? B. ? C. ? D. ? itd.
\newcommand{\testStop}{\newline} %koniec wprowadzania odpowiedzi testowych
\newcommand{\kluczStart}{\noindent \textbf{Test poprawna odpowiedź:}\newline} %klucz, poprawna odpowiedź pytania testowego (jedna literka): A lub B lub C lub D itd.
\newcommand{\kluczStop}{\newline} %koniec poprawnej odpowiedzi pytania testowego 
\newcommand{\wstawGrafike}[2]{\begin{figure}[h] \includegraphics[scale=#2] {#1} \end{figure}} %gdyby była potrzeba wstawienia obrazka, parametry: nazwa pliku, skala (jak nie wiesz co wpisać, to wpisz 1)

\begin{document}
\maketitle


\kategoria{Wikieł/Z1.25d}
\zadStart{Zadanie z Wikieł Z 1.25 d) moja wersja nr [nrWersji]}
%[a]=random.randint(1,10)
%[b]:[2,3,4,5,6,7,8,9]
%[c]:[2,3,4,5,6,7,8,9]
%[d]=random.randint(1,10)
%[e]=random.randint(1,10)
%[f]:[2,3,4,5,6,7,8,9]
%[g]=random.randint(1,10)
%[h]:[2,3,4,5,6,7,8,9]
%[i]=random.randint(1,10)
%[j]=random.randint(1,10)
%[ed]=-[e]-[d]
%[ed2]=[e]-[d]
%[ij]=[i]-[j]
%[ij2]=[i]+[j]
%math.gcd([a],[b])==1 and math.gcd([g],[f])==1 and math.gcd([ed],[c])==1 and math.gcd([ed2],[c])==1 and math.gcd([ij],[h])==1 and math.gcd([ij2],[h])==1 and ([ed]/[c]) < (-[g]/[f]) < ([ed2]/[c]) < ([ij]/[h]) < ([ij2]/[h]) < ([a]/[b]) 

Wyznacz dziedzinę funkcji określonej podanym wzorem \\ $f(x)=\frac{\sqrt{[a]-[b]x}}{|[c]x+[d]|-[e]} + \frac{\sqrt{[f]x+[g]}}{\sqrt{|[h]x-[i]|-[j]}}$.
\zadStop
\rozwStart{Joanna Świerzbin}{}
$$f(x)=\frac{\sqrt{[a]-[b]x}}{|[c]x+[d]|-[e]} + \frac{\sqrt{[f]x+[g]}}{\sqrt{|[h]x-[i]|-[j]}}$$
\begin{enumerate}
\item $$[a]-[b]x \geq 0 $$
$$[b]x \leq [a] $$
$$x \leq \frac{[a]}{[b]} $$
$$ x \in \left(-\infty, \frac{[a]}{[b]} \right] $$

\item $$ [f]x+[g] \geq 0 $$
$$ [f]x \geq -[g] $$
$$ x \geq -\frac{[g]}{[f]} $$
$$ x \in \left[ -\frac{[g]}{[f]} , \infty \right) $$

\item $$|[c]x+[d]|-[e] \neq 0$$
$$|[c]x+[d]| \neq [e]$$
$$[c]x+[d] \neq -[e] \vee [c]x+[d] \neq [e]$$
$$[c]x \neq [ed] \vee [c]x \neq [ed2]$$
$$x \neq \frac{[ed]}{[c]} \vee x \neq \frac{[ed2]}{[c]}$$
$$x \in \mathbb{R} \backslash \left\{ \frac{[ed]}{[c]}, \frac{[ed2]}{[c]} \right\} $$

\item $$ \sqrt{|[h]x-[i]|-[j]} \neq 0 \land |[h]x-[i]|-[j] \geq 0 $$
$$ |[h]x-[i]|-[j] \neq 0 \land |[h]x-[i]|-[j] \geq 0 $$
$$ |[h]x-[i]|-[j] > 0 $$
$$ |[h]x-[i]| > [j] $$
$$ [h]x-[i] < -[j]  \vee [h]x-[i] >[j]$$
$$ [h]x < [ij]  \vee [h]x >[ij2]$$
$$ x < \frac{[ij]}{[h]}  \vee x >\frac{[ij2]}{[h]}$$
$$ x \in \left( -\infty , \frac{[ij]}{[h]} \right) \cup \left( \frac{[ij2]}{[h]} , \infty \right) $$
\end{enumerate}
$$ x \in \left(-\infty, \frac{[a]}{[b]} \right] \land x \in \left[ -\frac{[g]}{[f]} , \infty \right) \land x \in \mathbb{R} \backslash \left\{ \frac{[ed]}{[c]}, \frac{[ed2]}{[c]} \right\} \land x \in \left( -\infty , \frac{[ij]}{[h]} \right) \cup \left( \frac{[ij2]}{[h]} , \infty \right) $$
$$ x \in \left[ -\frac{[g]}{[f]},  \frac{[ed2]}{[c]} \right) \cup \left( \frac{[ed2]}{[c]}, \frac{[ij]}{[h]} \right) \cup \left( \frac{[ij2]}{[h]} , \frac{[a]}{[b]} \right] $$
\rozwStop
\odpStart
$ x \in \left[ -\frac{[g]}{[f]},  \frac{[ed2]}{[c]} \right) \cup \left( \frac{[ed2]}{[c]}, \frac{[ij]}{[h]} \right) \cup \left( \frac{[ij2]}{[h]} , \frac{[a]}{[b]} \right] $
\odpStop
\testStart
A.$ x \in \left[ -\frac{[g]}{[f]},  \frac{[ed2]}{[c]} \right) \cup \left( \frac{[ed2]}{[c]}, \frac{[ij]}{[h]} \right) \cup \left( \frac{[ij2]}{[h]} , \frac{[a]}{[b]} \right] $\\
B. $ x \in \left[ -\frac{[g]}{[f]},  \frac{[ed2]}{[c]} \right) \cup \left( \frac{[ed2]}{[c]}, \frac{[ij]}{[h]} \right)$ \\
C. $ x \in \left[ -\frac{[g]}{[f]},  \frac{[ed2]}{[c]} \right) \cup  \left( \frac{[ij2]}{[h]} , \frac{[a]}{[b]} \right] $ \\
D. $ x \in  \left( \frac{[ed2]}{[c]}, \frac{[ij]}{[h]} \right) \cup \left( \frac{[ij2]}{[h]} , \frac{[a]}{[b]} \right] $\\
E. $ x \in \mathbb{R} $\\
F. $ x \in \emptyset $
\testStop
\kluczStart
A
\kluczStop



\end{document}