\documentclass[12pt, a4paper]{article}
\usepackage[utf8]{inputenc}
\usepackage{polski}

\usepackage{amsthm}  %pakiet do tworzenia twierdzeń itp.
\usepackage{amsmath} %pakiet do niektórych symboli matematycznych
\usepackage{amssymb} %pakiet do symboli mat., np. \nsubseteq
\usepackage{amsfonts}
\usepackage{graphicx} %obsługa plików graficznych z rozszerzeniem png, jpg
\theoremstyle{definition} %styl dla definicji
\newtheorem{zad}{} 
\title{Multizestaw zadań}
\author{Robert Fidytek}
%\date{\today}
\date{}
\newcounter{liczniksekcji}
\newcommand{\kategoria}[1]{\section{#1}} %olreślamy nazwę kateforii zadań
\newcommand{\zadStart}[1]{\begin{zad}#1\newline} %oznaczenie początku zadania
\newcommand{\zadStop}{\end{zad}}   %oznaczenie końca zadania
%Makra opcjonarne (nie muszą występować):
\newcommand{\rozwStart}[2]{\noindent \textbf{Rozwiązanie (autor #1 , recenzent #2): }\newline} %oznaczenie początku rozwiązania, opcjonarnie można wprowadzić informację o autorze rozwiązania zadania i recenzencie poprawności wykonania rozwiązania zadania
\newcommand{\rozwStop}{\newline}                                            %oznaczenie końca rozwiązania
\newcommand{\odpStart}{\noindent \textbf{Odpowiedź:}\newline}    %oznaczenie początku odpowiedzi końcowej (wypisanie wyniku)
\newcommand{\odpStop}{\newline}                                             %oznaczenie końca odpowiedzi końcowej (wypisanie wyniku)
\newcommand{\testStart}{\noindent \textbf{Test:}\newline} %ewentualne możliwe opcje odpowiedzi testowej: A. ? B. ? C. ? D. ? itd.
\newcommand{\testStop}{\newline} %koniec wprowadzania odpowiedzi testowych
\newcommand{\kluczStart}{\noindent \textbf{Test poprawna odpowiedź:}\newline} %klucz, poprawna odpowiedź pytania testowego (jedna literka): A lub B lub C lub D itd.
\newcommand{\kluczStop}{\newline} %koniec poprawnej odpowiedzi pytania testowego 
\newcommand{\wstawGrafike}[2]{\begin{figure}[h] \includegraphics[scale=#2] {#1} \end{figure}} %gdyby była potrzeba wstawienia obrazka, parametry: nazwa pliku, skala (jak nie wiesz co wpisać, to wpisz 1)

\begin{document}
\maketitle


\kategoria{Dymkowska,Beger/C1.7g}
\zadStart{Zadanie z Dymkowska,Beger C 1.7 g) moja wersja nr [nrWersji]}
%[a]:[2,3,4,5,6,7,8]
%[b]:[2,3,4,5,6,7,8]
%[a]=random.randint(2,20)
%[b]=random.randint(2,20) 
%[bm1]=[b]-1
%[pbm1]=pow([bm1],1/2)
%math.gcd([a],[bm1])==1 and [pbm1].is_integer()==False
Obliczyć całki ułamków prostych.$$\int\frac{[a]}{x^{2}-2x+[b]}dx$$
\zadStop
\rozwStart{Jakub Ulrych}{Pascal Nawrocki}
$$\int\frac{[a]}{x^{2}-2x+[b]}dx$$
$$[a]\int\frac{1}{x^{2}-2x+[b]}dx$$
$$[a]\int\frac{1}{(x-1)^{2}+[bm1]}dx$$
Podstawiamy $t=x-1\Rightarrow dt=dx$
$$[a]\int\frac{1}{(t)^{2}+[bm1]}dt$$
$$[a]\int\frac{1}{(t)^{2}+(\sqrt{[bm1]})^{2}}dt$$
Korzystamy ze wzoru: $\int\frac{dx}{x^{2}+a^{2}}=\frac{1}{a}arctg(\frac{x}{a})+C$
$$\frac{[a]}{\sqrt{[bm1]}}arctg\bigg(\frac{t}{\sqrt{[bm1]}}\bigg)+C=\frac{[a]}{\sqrt{[bm1]}}arctg\bigg(\frac{x-1}{\sqrt{[bm1]}}\bigg)+C$$
\rozwStop
\odpStart
$$\frac{[a]}{\sqrt{[bm1]}}arctg\bigg(\frac{x-1}{\sqrt{[bm1]}}\bigg)+C$$
\odpStop
\testStart
A.$\frac{[a]}{\sqrt{[bm1]}}arctg\big(\frac{x-1}{\sqrt{[bm1]}}\big)+C$
B.$\frac{[a]}{[bm1]}arctg\big(\frac{x-1}{\sqrt{[bm1]}}\big)+C$
C.$arctg\big(\frac{x-1}{\sqrt{[bm1]}}\big)+C$
D.$\frac{[a]}{\sqrt{[bm1]}}tg\big(\frac{x-1}{\sqrt{[bm1]}}\big)+C$
\testStop
\kluczStart
A
\kluczStop



\end{document}