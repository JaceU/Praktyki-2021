\documentclass[12pt, a4paper]{article}
\usepackage[utf8]{inputenc}
\usepackage{polski}

\usepackage{amsthm}  %pakiet do tworzenia twierdzeń itp.
\usepackage{amsmath} %pakiet do niektórych symboli matematycznych
\usepackage{amssymb} %pakiet do symboli mat., np. \nsubseteq
\usepackage{amsfonts}
\usepackage{graphicx} %obsługa plików graficznych z rozszerzeniem png, jpg
\theoremstyle{definition} %styl dla definicji
\newtheorem{zad}{} 
\title{Multizestaw zadań}
\author{Robert Fidytek}
%\date{\today}
\date{}
\newcounter{liczniksekcji}
\newcommand{\kategoria}[1]{\section{#1}} %olreślamy nazwę kateforii zadań
\newcommand{\zadStart}[1]{\begin{zad}#1\newline} %oznaczenie początku zadania
\newcommand{\zadStop}{\end{zad}}   %oznaczenie końca zadania
%Makra opcjonarne (nie muszą występować):
\newcommand{\rozwStart}[2]{\noindent \textbf{Rozwiązanie (autor #1 , recenzent #2): }\newline} %oznaczenie początku rozwiązania, opcjonarnie można wprowadzić informację o autorze rozwiązania zadania i recenzencie poprawności wykonania rozwiązania zadania
\newcommand{\rozwStop}{\newline}                                            %oznaczenie końca rozwiązania
\newcommand{\odpStart}{\noindent \textbf{Odpowiedź:}\newline}    %oznaczenie początku odpowiedzi końcowej (wypisanie wyniku)
\newcommand{\odpStop}{\newline}                                             %oznaczenie końca odpowiedzi końcowej (wypisanie wyniku)
\newcommand{\testStart}{\noindent \textbf{Test:}\newline} %ewentualne możliwe opcje odpowiedzi testowej: A. ? B. ? C. ? D. ? itd.
\newcommand{\testStop}{\newline} %koniec wprowadzania odpowiedzi testowych
\newcommand{\kluczStart}{\noindent \textbf{Test poprawna odpowiedź:}\newline} %klucz, poprawna odpowiedź pytania testowego (jedna literka): A lub B lub C lub D itd.
\newcommand{\kluczStop}{\newline} %koniec poprawnej odpowiedzi pytania testowego 
\newcommand{\wstawGrafike}[2]{\begin{figure}[h] \includegraphics[scale=#2] {#1} \end{figure}} %gdyby była potrzeba wstawienia obrazka, parametry: nazwa pliku, skala (jak nie wiesz co wpisać, to wpisz 1)

\begin{document}
\maketitle


\kategoria{Wikieł/Z5.18 b}
\zadStart{Zadanie z Wikieł Z 5.18 b) moja wersja nr [nrWersji]}
%[b]:[2,3,4,5,6,7,8,9]
%[a]=2*[b]
%[b]!=0
Oblicz granicę $\lim_{x \rightarrow \frac{\pi}{4}} \left(\ctg{(x)} \right)^{\frac{[b]}{x-\frac{\pi}{4}}}$.
\zadStop
\rozwStart{Joanna Świerzbin}{}
$$ \lim_{x \rightarrow \frac{\pi}{4}} \left(\ctg{(x)} \right)^{\frac{[b]}{x-\frac{\pi}{4}}} = \lim_{x \rightarrow \frac{\pi}{4}} e^{\ln \left(\left(\ctg{(x)} \right)^{\frac{[b]}{x-\frac{\pi}{4}}}\right)}= \lim_{x \rightarrow \frac{\pi}{4}} e^{{\frac{[b]}{x-\frac{\pi}{4}}}\ln \left(\ctg{(x)} \right)}=$$
$$=e^{\lim_{x \rightarrow \frac{\pi}{4}}{\frac{[b]}{x-\frac{\pi}{4}}}\ln \left(\ctg{(x)} \right)}$$
Policzmy $ {\lim_{x \rightarrow \frac{\pi}{4}}{\frac{[b]}{x-\frac{\pi}{4}}}\ln \left(\ctg{(x)} \right)}$.
$${\lim_{x \rightarrow \frac{\pi}{4}}{\frac{[b]}{x-\frac{\pi}{4}}}\ln \left(\ctg{(x)} \right)} = \lim_{x \rightarrow \frac{\pi}{4}} \frac{\ln\left(\ctg{(x)} \right)}{\frac{x-\frac{\pi}{4}}{[b]}}$$
Otrzymujemy $ \left[ \frac{0}{0} \right] $ więc możemy skorzystać z twierdzenia de l'Hospitala.
$$\lim_{x \rightarrow \frac{\pi}{4}} \frac{\left(\ln\left(\ctg{(x)} \right)\right)'}{\left(\frac{x-\frac{\pi}{4}}{[b]}\right)'} = \lim_{x \rightarrow \frac{\pi}{4}} \frac{\frac{1}{\ctg{(x)}}\left(\ctg(x)\right)'}{\frac{1}{[b]}} = \lim_{x \rightarrow \frac{\pi}{4}} \frac{\frac{-1}{\ctg{(x)}\cdot sin^2(x)}}{\frac{1}{[b]}}=$$
$$= \lim_{x \rightarrow \frac{\pi}{4}} \frac{-[b]}{\ctg{(x)}\cdot sin^2(x)}= -[b] \lim_{x \rightarrow \frac{\pi}{4}} \frac{1}{\cos{(x)}\cdot sin(x)}$$
$$ =-[b] \lim_{x \rightarrow \frac{\pi}{4}} \frac{1}{\sin( x )}  \lim_{x \rightarrow \frac{\pi}{4}} \frac{1}{\cos{(x)}}=- 2 \cdot [b] = - [a] $$
Podstawmy do początkowego przykładu.
$$e^{\lim_{x \rightarrow \frac{\pi}{4}}{\frac{[b]}{x-\frac{\pi}{4}}}\ln \left(\ctg{(x)} \right)} = e^{-[a]} $$
\rozwStop
\odpStart
$ \lim_{x \rightarrow \frac{\pi}{4}} \left(\ctg{(x)} \right)^{\frac{[b]}{x-\frac{\pi}{4}}} = e^{-[a]} $
\odpStop
\testStart
A. $\lim_{x \rightarrow \frac{\pi}{4}} \left(\ctg{(x)} \right)^{\frac{[b]}{x-\frac{\pi}{4}}} = e^{-[a]}$\\
B. $\lim_{x \rightarrow \frac{\pi}{4}} \left(\ctg{(x)} \right)^{\frac{[b]}{x-\frac{\pi}{4}}} = e^{[a]}$\\
C. $\lim_{x \rightarrow \frac{\pi}{4}} \left(\ctg{(x)} \right)^{\frac{[b]}{x-\frac{\pi}{4}}} = e$\\
D. $\lim_{x \rightarrow \frac{\pi}{4}} \left(\ctg{(x)} \right)^{\frac{[b]}{x-\frac{\pi}{4}}} = 1$\\
E. $\lim_{x \rightarrow \frac{\pi}{4}} \left(\ctg{(x)} \right)^{\frac{[b]}{x-\frac{\pi}{4}}} = 0$\\
F. $\lim_{x \rightarrow \frac{\pi}{4}} \left(\ctg{(x)} \right)^{\frac{[b]}{x-\frac{\pi}{4}}} = \infty$
\testStop
\kluczStart
A
\kluczStop



\end{document}