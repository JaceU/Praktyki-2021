\documentclass[12pt, a4paper]{article}
\usepackage[utf8]{inputenc}
\usepackage{polski}
\usepackage{amsthm}  %pakiet do tworzenia twierdzeń itp.
\usepackage{amsmath} %pakiet do niektórych symboli matematycznych
\usepackage{amssymb} %pakiet do symboli mat., np. \nsubseteq
\usepackage{amsfonts}
\usepackage{graphicx} %obsługa plików graficznych z rozszerzeniem png, jpg
\theoremstyle{definition} %styl dla definicji
\newtheorem{zad}{} 
\title{Multizestaw zadań}
\author{Radosław Grzyb}
%\date{\today}
\date{}
\newcounter{liczniksekcji}
\newcommand{\kategoria}[1]{\section{#1}} %olreślamy nazwę kateforii zadań
\newcommand{\zadStart}[1]{\begin{zad}#1\newline} %oznaczenie początku zadania
\newcommand{\zadStop}{\end{zad}}   %oznaczenie końca zadania
%Makra opcjonarne (nie muszą występować):
\newcommand{\rozwStart}[2]{\noindent \textbf{Rozwiązanie (autor #1 , recenzent #2): }\newline} %oznaczenie początku rozwiązania, opcjonarnie można wprowadzić informację o autorze rozwiązania zadania i recenzencie poprawności wykonania rozwiązania zadania
\newcommand{\rozwStop}{\newline}                                            %oznaczenie końca rozwiązania
\newcommand{\odpStart}{\noindent \textbf{Odpowiedź:}\newline}    %oznaczenie początku odpowiedzi końcowej (wypisanie wyniku)
\newcommand{\odpStop}{\newline}                                             %oznaczenie końca odpowiedzi końcowej (wypisanie wyniku)
\newcommand{\testStart}{\noindent \textbf{Test:}\newline} %ewentualne możliwe opcje odpowiedzi testowej: A. ? B. ? C. ? D. ? itd.
\newcommand{\testStop}{\newline} %koniec wprowadzania odpowiedzi testowych
\newcommand{\kluczStart}{\noindent \textbf{Test poprawna odpowiedź:}\newline} %klucz, poprawna odpowiedź pytania testowego (jedna literka): A lub B lub C lub D itd.
\newcommand{\kluczStop}{\newline} %koniec poprawnej odpowiedzi pytania testowego 
\newcommand{\wstawGrafike}[2]{\begin{figure}[h] \includegraphics[scale=#2] {#1} \end{figure}} %gdyby była potrzeba wstawienia obrazka, parametry: nazwa pliku, skala (jak nie wiesz co wpisać, to wpisz 1)
\begin{document}
\maketitle
\kategoria{Beger/c1.14c}
\zadStart{Zadanie z Beger C 1.14c moja wersja nr [nrWersji]}
%[p1]:[2,4,6,8]
%[p2]:[1,2,3,4,5,6,7,8]
%[p3]:[1,2,3,4,5,6,7,8]
%[dp1]=int([p1]/2)
%[lambda]=[p3]-[dp1]
%[lambda]>1
Metodą współczynników nieoznaczonych obliczyć całki funkcji niewymiernej.
$$\int \frac{[p1]x^{2}+[p2]x+[p3]}{\sqrt{x^{2}+1}} \,dx$$
\zadStop
\rozwStart{Radosław Grzyb}{}
Do obliczenia całki wykorzystamy wzór:
$$\int \frac{W_n(x)}{\sqrt{ax^2+bx+c}} \,dx=W_{n-1}(x)\sqrt{ax^2+bx+c}+\lambda\int \frac{1}{\sqrt{ax^2+bx+c}} \,dx$$
Gdzie $W_n(x)$ oznacza wielomian stopnia $n$.\\
Podstawiając do wzoru otrzymujemy:
$$\int \frac{[p1]x^{2}+[p2]x+[p3]}{\sqrt{x^{2}+1}} \,dx=(Ax+B)\sqrt{x^{2}+1}+\lambda\int \frac{1}{\sqrt{x^{2}+1}} \,dx$$
Następnie różniczkujemy obie strony równania otrzymując:
$$\frac{[p1]x^{2}+[p2]x+[p3]}{\sqrt{x^{2}+1}}=A\sqrt{x^{2}+1}+\frac{Ax+B}{2\sqrt{x^2+1}}\cdot2x+ \frac{\lambda}{\sqrt{x^{2}+1}}$$
$$\frac{[p1]x^{2}+[p2]x+[p3]}{\sqrt{x^{2}+1}}=A\sqrt{x^{2}+1}+\frac{Ax^{2}+Bx}{\sqrt{x^2+1}}+ \frac{\lambda}{\sqrt{x^{2}+1}}$$
Mnożymy obie strony równania przez $\sqrt{x^{2}+1}$:
$$[p1]x^{2}+[p2]x+[p3]=A(x^{2}+1)+Ax^2+Bx+\lambda$$
$$[p1]x^{2}+[p2]x+[p3]=2Ax^{2}+A+Bx+\lambda$$
Przyrównując do siebie odpowiednie współczynniki otrzymujemy do rozwiązania prosty układ równań:
$$\begin{cases} [p1]=2A \implies A=[dp1] \\ [p2]=B \\ [p3]=A+\lambda \implies [p3]=[dp1]+\lambda \implies \lambda=[lambda] \end{cases}$$
Podstawmy otrzymane wartości do naszego wzoru:
$$\int \frac{[p1]x^{2}+[p2]x+[p3]}{\sqrt{x^{2}+1}} \,dx=([dp1]x+[p2])\sqrt{x^{2}+1}+[lambda]\int \frac{1}{\sqrt{x^{2}+1}} \,dx$$
Już prawie gotowe! Pozostaje nam tylko obliczyć całkę $\int \frac{1}{\sqrt{x^{2}+1}} \,dx$. Do tego celu wykorzystamy gotowy już wzór $\int \frac{1}{\sqrt{x^{2}+q}} \,dx = \ln(|x+\sqrt{x^2+q}|)+C$.\\
A więc:
$$\int \frac{1}{\sqrt{x^{2}+1}} \,dx= \ln(|x+\sqrt{x^2+1}|)+C$$
Finalnie otrzymujemy:
$$\int \frac{[p1]x^{2}+[p2]x+[p3]}{\sqrt{x^{2}+1}} \,dx=([dp1]x+[p2])\sqrt{x^{2}+1}+[lambda]\ln(|x+\sqrt{x^2+1}|)+C$$
\rozwStop
\odpStart
$$([dp1]x+[p2])\sqrt{x^{2}+1}+[lambda]\ln(|x+\sqrt{x^2+1}|)+C$$
\odpStop
\testStart
A.$$([dp1]x^{2}+[p2])\sqrt{x^{2}+1}+[lambda]\ln(|x+\sqrt{x^2+1}|)+C$$
B.$$([dp1]x+[p2])\sqrt{x^{2}+1}+[lambda]\ln(|x+\sqrt{x^2+1}|)+C$$
C.$$([dp1]x+[p2])\sqrt{x^{3}+x}+[lambda]\ln(|x+\sqrt{x^2+1}|)+C$$
D.$$([dp1]x-[p2])\sqrt{x^{2}+1}+[lambda]\ln(|x+\sqrt{x^2+1}|)+C$$
\testStop
\kluczStart
B
\kluczStop
\end{document}