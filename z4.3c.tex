\documentclass[12pt, a4paper]{article}
\usepackage[utf8]{inputenc}
\usepackage{polski}
\usepackage{amsthm}  %pakiet do tworzenia twierdzeń itp.
\usepackage{amsmath} %pakiet do niektórych symboli matematycznych
\usepackage{amssymb} %pakiet do symboli mat., np. \nsubseteq
\usepackage{amsfonts}
\usepackage{graphicx} %obsługa plików graficznych z rozszerzeniem png, jpg
\theoremstyle{definition} %styl dla definicji
\newtheorem{zad}{} 
\title{Multizestaw zadań}
\author{Patryk Wirkus}
%\date{\today}
\date{}
\newcommand{\kategoria}[1]{\section{#1}}
\newcommand{\zadStart}[1]{\begin{zad}#1\newline}
\newcommand{\zadStop}{\end{zad}}
\newcommand{\rozwStart}[2]{\noindent \textbf{Rozwiązanie (autor #1 , recenzent #2): }\newline}
\newcommand{\rozwStop}{\newline}                                           
\newcommand{\odpStart}{\noindent \textbf{Odpowiedź:}\newline}
\newcommand{\odpStop}{\newline}
\newcommand{\testStart}{\noindent \textbf{Test:}\newline}
\newcommand{\testStop}{\newline}
\newcommand{\kluczStart}{\noindent \textbf{Test poprawna odpowiedź:}\newline}
\newcommand{\kluczStop}{\newline}
\newcommand{\wstawGrafike}[2]{\begin{figure}[h] \includegraphics[scale=#2] {#1} \end{figure}}

\begin{document}
\maketitle

\kategoria{Wikieł/Z4.3c}


\zadStart{Zadanie z Wikieł Z 4.3 c) moja wersja nr 1}
Obliczyć granicę funkcji $f(x)=\frac{-x-4}{4-\sqrt{-x}}$.
\zadStop
\rozwStart{Patryk Wirkus}{Szymon Tokarski}
$$\frac{-x-4}{4-\sqrt{-x}}=\frac{(-x-4)(4+\sqrt{-x})}{4-(-x)}=-(4+\sqrt{-x})$$
\\
$$\lim\limits_{x\to-1}\frac{-x-4}{4-\sqrt{-x}}=[\frac{0}{0}]=\lim\limits_{x\to-1}-(4+\sqrt{-x}) =-5$$
\rozwStop
\odpStart
$-5$
\odpStop
\testStart
A.$-5$
B.$-3$
C.$3$
D.$5$
E.$\infty$
F.$-\infty$
G.$0$
H.$-4$
I.$4$
\testStop
\kluczStart
A
\kluczStop



\zadStart{Zadanie z Wikieł Z 4.3 c) moja wersja nr 2}
Obliczyć granicę funkcji $f(x)=\frac{-x-6}{6-\sqrt{-x}}$.
\zadStop
\rozwStart{Patryk Wirkus}{Szymon Tokarski}
$$\frac{-x-6}{6-\sqrt{-x}}=\frac{(-x-6)(6+\sqrt{-x})}{6-(-x)}=-(6+\sqrt{-x})$$
\\
$$\lim\limits_{x\to-1}\frac{-x-6}{6-\sqrt{-x}}=[\frac{0}{0}]=\lim\limits_{x\to-1}-(6+\sqrt{-x}) =-7$$
\rozwStop
\odpStart
$-7$
\odpStop
\testStart
A.$-7$
B.$-5$
C.$5$
D.$7$
E.$\infty$
F.$-\infty$
G.$0$
H.$-6$
I.$6$
\testStop
\kluczStart
A
\kluczStop



\zadStart{Zadanie z Wikieł Z 4.3 c) moja wersja nr 3}
Obliczyć granicę funkcji $f(x)=\frac{-x-9}{9-\sqrt{-x}}$.
\zadStop
\rozwStart{Patryk Wirkus}{Szymon Tokarski}
$$\frac{-x-9}{9-\sqrt{-x}}=\frac{(-x-9)(9+\sqrt{-x})}{9-(-x)}=-(9+\sqrt{-x})$$
\\
$$\lim\limits_{x\to-1}\frac{-x-9}{9-\sqrt{-x}}=[\frac{0}{0}]=\lim\limits_{x\to-1}-(9+\sqrt{-x}) =-10$$
\rozwStop
\odpStart
$-10$
\odpStop
\testStart
A.$-10$
B.$-8$
C.$8$
D.$10$
E.$\infty$
F.$-\infty$
G.$0$
H.$-9$
I.$9$
\testStop
\kluczStart
A
\kluczStop



\zadStart{Zadanie z Wikieł Z 4.3 c) moja wersja nr 4}
Obliczyć granicę funkcji $f(x)=\frac{-x-16}{16-\sqrt{-x}}$.
\zadStop
\rozwStart{Patryk Wirkus}{Szymon Tokarski}
$$\frac{-x-16}{16-\sqrt{-x}}=\frac{(-x-16)(16+\sqrt{-x})}{16-(-x)}=-(16+\sqrt{-x})$$
\\
$$\lim\limits_{x\to-1}\frac{-x-16}{16-\sqrt{-x}}=[\frac{0}{0}]=\lim\limits_{x\to-1}-(16+\sqrt{-x}) =-17$$
\rozwStop
\odpStart
$-17$
\odpStop
\testStart
A.$-17$
B.$-15$
C.$15$
D.$17$
E.$\infty$
F.$-\infty$
G.$0$
H.$-16$
I.$16$
\testStop
\kluczStart
A
\kluczStop



\zadStart{Zadanie z Wikieł Z 4.3 c) moja wersja nr 5}
Obliczyć granicę funkcji $f(x)=\frac{-x-25}{25-\sqrt{-x}}$.
\zadStop
\rozwStart{Patryk Wirkus}{Szymon Tokarski}
$$\frac{-x-25}{25-\sqrt{-x}}=\frac{(-x-25)(25+\sqrt{-x})}{25-(-x)}=-(25+\sqrt{-x})$$
\\
$$\lim\limits_{x\to-1}\frac{-x-25}{25-\sqrt{-x}}=[\frac{0}{0}]=\lim\limits_{x\to-1}-(25+\sqrt{-x}) =-26$$
\rozwStop
\odpStart
$-26$
\odpStop
\testStart
A.$-26$
B.$-24$
C.$24$
D.$26$
E.$\infty$
F.$-\infty$
G.$0$
H.$-25$
I.$25$
\testStop
\kluczStart
A
\kluczStop



\zadStart{Zadanie z Wikieł Z 4.3 c) moja wersja nr 6}
Obliczyć granicę funkcji $f(x)=\frac{-x-36}{36-\sqrt{-x}}$.
\zadStop
\rozwStart{Patryk Wirkus}{Szymon Tokarski}
$$\frac{-x-36}{36-\sqrt{-x}}=\frac{(-x-36)(36+\sqrt{-x})}{36-(-x)}=-(36+\sqrt{-x})$$
\\
$$\lim\limits_{x\to-1}\frac{-x-36}{36-\sqrt{-x}}=[\frac{0}{0}]=\lim\limits_{x\to-1}-(36+\sqrt{-x}) =-37$$
\rozwStop
\odpStart
$-37$
\odpStop
\testStart
A.$-37$
B.$-35$
C.$35$
D.$37$
E.$\infty$
F.$-\infty$
G.$0$
H.$-36$
I.$36$
\testStop
\kluczStart
A
\kluczStop



\zadStart{Zadanie z Wikieł Z 4.3 c) moja wersja nr 7}
Obliczyć granicę funkcji $f(x)=\frac{-x-49}{49-\sqrt{-x}}$.
\zadStop
\rozwStart{Patryk Wirkus}{Szymon Tokarski}
$$\frac{-x-49}{49-\sqrt{-x}}=\frac{(-x-49)(49+\sqrt{-x})}{49-(-x)}=-(49+\sqrt{-x})$$
\\
$$\lim\limits_{x\to-1}\frac{-x-49}{49-\sqrt{-x}}=[\frac{0}{0}]=\lim\limits_{x\to-1}-(49+\sqrt{-x}) =-50$$
\rozwStop
\odpStart
$-50$
\odpStop
\testStart
A.$-50$
B.$-48$
C.$48$
D.$50$
E.$\infty$
F.$-\infty$
G.$0$
H.$-49$
I.$49$
\testStop
\kluczStart
A
\kluczStop



\zadStart{Zadanie z Wikieł Z 4.3 c) moja wersja nr 8}
Obliczyć granicę funkcji $f(x)=\frac{-x-64}{64-\sqrt{-x}}$.
\zadStop
\rozwStart{Patryk Wirkus}{Szymon Tokarski}
$$\frac{-x-64}{64-\sqrt{-x}}=\frac{(-x-64)(64+\sqrt{-x})}{64-(-x)}=-(64+\sqrt{-x})$$
\\
$$\lim\limits_{x\to-1}\frac{-x-64}{64-\sqrt{-x}}=[\frac{0}{0}]=\lim\limits_{x\to-1}-(64+\sqrt{-x}) =-65$$
\rozwStop
\odpStart
$-65$
\odpStop
\testStart
A.$-65$
B.$-63$
C.$63$
D.$65$
E.$\infty$
F.$-\infty$
G.$0$
H.$-64$
I.$64$
\testStop
\kluczStart
A
\kluczStop



\zadStart{Zadanie z Wikieł Z 4.3 c) moja wersja nr 9}
Obliczyć granicę funkcji $f(x)=\frac{-x-81}{81-\sqrt{-x}}$.
\zadStop
\rozwStart{Patryk Wirkus}{Szymon Tokarski}
$$\frac{-x-81}{81-\sqrt{-x}}=\frac{(-x-81)(81+\sqrt{-x})}{81-(-x)}=-(81+\sqrt{-x})$$
\\
$$\lim\limits_{x\to-1}\frac{-x-81}{81-\sqrt{-x}}=[\frac{0}{0}]=\lim\limits_{x\to-1}-(81+\sqrt{-x}) =-82$$
\rozwStop
\odpStart
$-82$
\odpStop
\testStart
A.$-82$
B.$-80$
C.$80$
D.$82$
E.$\infty$
F.$-\infty$
G.$0$
H.$-81$
I.$81$
\testStop
\kluczStart
A
\kluczStop



\zadStart{Zadanie z Wikieł Z 4.3 c) moja wersja nr 10}
Obliczyć granicę funkcji $f(x)=\frac{-x-100}{100-\sqrt{-x}}$.
\zadStop
\rozwStart{Patryk Wirkus}{Szymon Tokarski}
$$\frac{-x-100}{100-\sqrt{-x}}=\frac{(-x-100)(100+\sqrt{-x})}{100-(-x)}=-(100+\sqrt{-x})$$
\\
$$\lim\limits_{x\to-1}\frac{-x-100}{100-\sqrt{-x}}=[\frac{0}{0}]=\lim\limits_{x\to-1}-(100+\sqrt{-x}) =-101$$
\rozwStop
\odpStart
$-101$
\odpStop
\testStart
A.$-101$
B.$-99$
C.$99$
D.$101$
E.$\infty$
F.$-\infty$
G.$0$
H.$-100$
I.$100$
\testStop
\kluczStart
A
\kluczStop



\zadStart{Zadanie z Wikieł Z 4.3 c) moja wersja nr 11}
Obliczyć granicę funkcji $f(x)=\frac{-x-121}{121-\sqrt{-x}}$.
\zadStop
\rozwStart{Patryk Wirkus}{Szymon Tokarski}
$$\frac{-x-121}{121-\sqrt{-x}}=\frac{(-x-121)(121+\sqrt{-x})}{121-(-x)}=-(121+\sqrt{-x})$$
\\
$$\lim\limits_{x\to-1}\frac{-x-121}{121-\sqrt{-x}}=[\frac{0}{0}]=\lim\limits_{x\to-1}-(121+\sqrt{-x}) =-122$$
\rozwStop
\odpStart
$-122$
\odpStop
\testStart
A.$-122$
B.$-120$
C.$120$
D.$122$
E.$\infty$
F.$-\infty$
G.$0$
H.$-121$
I.$121$
\testStop
\kluczStart
A
\kluczStop



\zadStart{Zadanie z Wikieł Z 4.3 c) moja wersja nr 12}
Obliczyć granicę funkcji $f(x)=\frac{-x-144}{144-\sqrt{-x}}$.
\zadStop
\rozwStart{Patryk Wirkus}{Szymon Tokarski}
$$\frac{-x-144}{144-\sqrt{-x}}=\frac{(-x-144)(144+\sqrt{-x})}{144-(-x)}=-(144+\sqrt{-x})$$
\\
$$\lim\limits_{x\to-1}\frac{-x-144}{144-\sqrt{-x}}=[\frac{0}{0}]=\lim\limits_{x\to-1}-(144+\sqrt{-x}) =-145$$
\rozwStop
\odpStart
$-145$
\odpStop
\testStart
A.$-145$
B.$-143$
C.$143$
D.$145$
E.$\infty$
F.$-\infty$
G.$0$
H.$-144$
I.$144$
\testStop
\kluczStart
A
\kluczStop



\zadStart{Zadanie z Wikieł Z 4.3 c) moja wersja nr 13}
Obliczyć granicę funkcji $f(x)=\frac{-x-169}{169-\sqrt{-x}}$.
\zadStop
\rozwStart{Patryk Wirkus}{Szymon Tokarski}
$$\frac{-x-169}{169-\sqrt{-x}}=\frac{(-x-169)(169+\sqrt{-x})}{169-(-x)}=-(169+\sqrt{-x})$$
\\
$$\lim\limits_{x\to-1}\frac{-x-169}{169-\sqrt{-x}}=[\frac{0}{0}]=\lim\limits_{x\to-1}-(169+\sqrt{-x}) =-170$$
\rozwStop
\odpStart
$-170$
\odpStop
\testStart
A.$-170$
B.$-168$
C.$168$
D.$170$
E.$\infty$
F.$-\infty$
G.$0$
H.$-169$
I.$169$
\testStop
\kluczStart
A
\kluczStop



\zadStart{Zadanie z Wikieł Z 4.3 c) moja wersja nr 14}
Obliczyć granicę funkcji $f(x)=\frac{-x-196}{196-\sqrt{-x}}$.
\zadStop
\rozwStart{Patryk Wirkus}{Szymon Tokarski}
$$\frac{-x-196}{196-\sqrt{-x}}=\frac{(-x-196)(196+\sqrt{-x})}{196-(-x)}=-(196+\sqrt{-x})$$
\\
$$\lim\limits_{x\to-1}\frac{-x-196}{196-\sqrt{-x}}=[\frac{0}{0}]=\lim\limits_{x\to-1}-(196+\sqrt{-x}) =-197$$
\rozwStop
\odpStart
$-197$
\odpStop
\testStart
A.$-197$
B.$-195$
C.$195$
D.$197$
E.$\infty$
F.$-\infty$
G.$0$
H.$-196$
I.$196$
\testStop
\kluczStart
A
\kluczStop



\zadStart{Zadanie z Wikieł Z 4.3 c) moja wersja nr 15}
Obliczyć granicę funkcji $f(x)=\frac{-x-225}{225-\sqrt{-x}}$.
\zadStop
\rozwStart{Patryk Wirkus}{Szymon Tokarski}
$$\frac{-x-225}{225-\sqrt{-x}}=\frac{(-x-225)(225+\sqrt{-x})}{225-(-x)}=-(225+\sqrt{-x})$$
\\
$$\lim\limits_{x\to-1}\frac{-x-225}{225-\sqrt{-x}}=[\frac{0}{0}]=\lim\limits_{x\to-1}-(225+\sqrt{-x}) =-226$$
\rozwStop
\odpStart
$-226$
\odpStop
\testStart
A.$-226$
B.$-224$
C.$224$
D.$226$
E.$\infty$
F.$-\infty$
G.$0$
H.$-225$
I.$225$
\testStop
\kluczStart
A
\kluczStop



\zadStart{Zadanie z Wikieł Z 4.3 c) moja wersja nr 16}
Obliczyć granicę funkcji $f(x)=\frac{-x-256}{256-\sqrt{-x}}$.
\zadStop
\rozwStart{Patryk Wirkus}{Szymon Tokarski}
$$\frac{-x-256}{256-\sqrt{-x}}=\frac{(-x-256)(256+\sqrt{-x})}{256-(-x)}=-(256+\sqrt{-x})$$
\\
$$\lim\limits_{x\to-1}\frac{-x-256}{256-\sqrt{-x}}=[\frac{0}{0}]=\lim\limits_{x\to-1}-(256+\sqrt{-x}) =-257$$
\rozwStop
\odpStart
$-257$
\odpStop
\testStart
A.$-257$
B.$-255$
C.$255$
D.$257$
E.$\infty$
F.$-\infty$
G.$0$
H.$-256$
I.$256$
\testStop
\kluczStart
A
\kluczStop



\zadStart{Zadanie z Wikieł Z 4.3 c) moja wersja nr 17}
Obliczyć granicę funkcji $f(x)=\frac{-x-289}{289-\sqrt{-x}}$.
\zadStop
\rozwStart{Patryk Wirkus}{Szymon Tokarski}
$$\frac{-x-289}{289-\sqrt{-x}}=\frac{(-x-289)(289+\sqrt{-x})}{289-(-x)}=-(289+\sqrt{-x})$$
\\
$$\lim\limits_{x\to-1}\frac{-x-289}{289-\sqrt{-x}}=[\frac{0}{0}]=\lim\limits_{x\to-1}-(289+\sqrt{-x}) =-290$$
\rozwStop
\odpStart
$-290$
\odpStop
\testStart
A.$-290$
B.$-288$
C.$288$
D.$290$
E.$\infty$
F.$-\infty$
G.$0$
H.$-289$
I.$289$
\testStop
\kluczStart
A
\kluczStop



\zadStart{Zadanie z Wikieł Z 4.3 c) moja wersja nr 18}
Obliczyć granicę funkcji $f(x)=\frac{-x-324}{324-\sqrt{-x}}$.
\zadStop
\rozwStart{Patryk Wirkus}{Szymon Tokarski}
$$\frac{-x-324}{324-\sqrt{-x}}=\frac{(-x-324)(324+\sqrt{-x})}{324-(-x)}=-(324+\sqrt{-x})$$
\\
$$\lim\limits_{x\to-1}\frac{-x-324}{324-\sqrt{-x}}=[\frac{0}{0}]=\lim\limits_{x\to-1}-(324+\sqrt{-x}) =-325$$
\rozwStop
\odpStart
$-325$
\odpStop
\testStart
A.$-325$
B.$-323$
C.$323$
D.$325$
E.$\infty$
F.$-\infty$
G.$0$
H.$-324$
I.$324$
\testStop
\kluczStart
A
\kluczStop



\zadStart{Zadanie z Wikieł Z 4.3 c) moja wersja nr 19}
Obliczyć granicę funkcji $f(x)=\frac{-x-361}{361-\sqrt{-x}}$.
\zadStop
\rozwStart{Patryk Wirkus}{Szymon Tokarski}
$$\frac{-x-361}{361-\sqrt{-x}}=\frac{(-x-361)(361+\sqrt{-x})}{361-(-x)}=-(361+\sqrt{-x})$$
\\
$$\lim\limits_{x\to-1}\frac{-x-361}{361-\sqrt{-x}}=[\frac{0}{0}]=\lim\limits_{x\to-1}-(361+\sqrt{-x}) =-362$$
\rozwStop
\odpStart
$-362$
\odpStop
\testStart
A.$-362$
B.$-360$
C.$360$
D.$362$
E.$\infty$
F.$-\infty$
G.$0$
H.$-361$
I.$361$
\testStop
\kluczStart
A
\kluczStop



\zadStart{Zadanie z Wikieł Z 4.3 c) moja wersja nr 20}
Obliczyć granicę funkcji $f(x)=\frac{-x-400}{400-\sqrt{-x}}$.
\zadStop
\rozwStart{Patryk Wirkus}{Szymon Tokarski}
$$\frac{-x-400}{400-\sqrt{-x}}=\frac{(-x-400)(400+\sqrt{-x})}{400-(-x)}=-(400+\sqrt{-x})$$
\\
$$\lim\limits_{x\to-1}\frac{-x-400}{400-\sqrt{-x}}=[\frac{0}{0}]=\lim\limits_{x\to-1}-(400+\sqrt{-x}) =-401$$
\rozwStop
\odpStart
$-401$
\odpStop
\testStart
A.$-401$
B.$-399$
C.$399$
D.$401$
E.$\infty$
F.$-\infty$
G.$0$
H.$-400$
I.$400$
\testStop
\kluczStart
A
\kluczStop



\zadStart{Zadanie z Wikieł Z 4.3 c) moja wersja nr 21}
Obliczyć granicę funkcji $f(x)=\frac{-x-625}{625-\sqrt{-x}}$.
\zadStop
\rozwStart{Patryk Wirkus}{Szymon Tokarski}
$$\frac{-x-625}{625-\sqrt{-x}}=\frac{(-x-625)(625+\sqrt{-x})}{625-(-x)}=-(625+\sqrt{-x})$$
\\
$$\lim\limits_{x\to-1}\frac{-x-625}{625-\sqrt{-x}}=[\frac{0}{0}]=\lim\limits_{x\to-1}-(625+\sqrt{-x}) =-626$$
\rozwStop
\odpStart
$-626$
\odpStop
\testStart
A.$-626$
B.$-624$
C.$624$
D.$626$
E.$\infty$
F.$-\infty$
G.$0$
H.$-625$
I.$625$
\testStop
\kluczStart
A
\kluczStop



\zadStart{Zadanie z Wikieł Z 4.3 c) moja wersja nr 22}
Obliczyć granicę funkcji $f(x)=\frac{-x-900}{900-\sqrt{-x}}$.
\zadStop
\rozwStart{Patryk Wirkus}{Szymon Tokarski}
$$\frac{-x-900}{900-\sqrt{-x}}=\frac{(-x-900)(900+\sqrt{-x})}{900-(-x)}=-(900+\sqrt{-x})$$
\\
$$\lim\limits_{x\to-1}\frac{-x-900}{900-\sqrt{-x}}=[\frac{0}{0}]=\lim\limits_{x\to-1}-(900+\sqrt{-x}) =-901$$
\rozwStop
\odpStart
$-901$
\odpStop
\testStart
A.$-901$
B.$-899$
C.$899$
D.$901$
E.$\infty$
F.$-\infty$
G.$0$
H.$-900$
I.$900$
\testStop
\kluczStart
A
\kluczStop



\zadStart{Zadanie z Wikieł Z 4.3 c) moja wersja nr 23}
Obliczyć granicę funkcji $f(x)=\frac{-x-1600}{1600-\sqrt{-x}}$.
\zadStop
\rozwStart{Patryk Wirkus}{Szymon Tokarski}
$$\frac{-x-1600}{1600-\sqrt{-x}}=\frac{(-x-1600)(1600+\sqrt{-x})}{1600-(-x)}=-(1600+\sqrt{-x})$$
\\
$$\lim\limits_{x\to-1}\frac{-x-1600}{1600-\sqrt{-x}}=[\frac{0}{0}]=\lim\limits_{x\to-1}-(1600+\sqrt{-x}) =-1601$$
\rozwStop
\odpStart
$-1601$
\odpStop
\testStart
A.$-1601$
B.$-1599$
C.$1599$
D.$1601$
E.$\infty$
F.$-\infty$
G.$0$
H.$-1600$
I.$1600$
\testStop
\kluczStart
A
\kluczStop



\zadStart{Zadanie z Wikieł Z 4.3 c) moja wersja nr 24}
Obliczyć granicę funkcji $f(x)=\frac{-x-2500}{2500-\sqrt{-x}}$.
\zadStop
\rozwStart{Patryk Wirkus}{Szymon Tokarski}
$$\frac{-x-2500}{2500-\sqrt{-x}}=\frac{(-x-2500)(2500+\sqrt{-x})}{2500-(-x)}=-(2500+\sqrt{-x})$$
\\
$$\lim\limits_{x\to-1}\frac{-x-2500}{2500-\sqrt{-x}}=[\frac{0}{0}]=\lim\limits_{x\to-1}-(2500+\sqrt{-x}) =-2501$$
\rozwStop
\odpStart
$-2501$
\odpStop
\testStart
A.$-2501$
B.$-2499$
C.$2499$
D.$2501$
E.$\infty$
F.$-\infty$
G.$0$
H.$-2500$
I.$2500$
\testStop
\kluczStart
A
\kluczStop



\zadStart{Zadanie z Wikieł Z 4.3 c) moja wersja nr 25}
Obliczyć granicę funkcji $f(x)=\frac{-x-3600}{3600-\sqrt{-x}}$.
\zadStop
\rozwStart{Patryk Wirkus}{Szymon Tokarski}
$$\frac{-x-3600}{3600-\sqrt{-x}}=\frac{(-x-3600)(3600+\sqrt{-x})}{3600-(-x)}=-(3600+\sqrt{-x})$$
\\
$$\lim\limits_{x\to-1}\frac{-x-3600}{3600-\sqrt{-x}}=[\frac{0}{0}]=\lim\limits_{x\to-1}-(3600+\sqrt{-x}) =-3601$$
\rozwStop
\odpStart
$-3601$
\odpStop
\testStart
A.$-3601$
B.$-3599$
C.$3599$
D.$3601$
E.$\infty$
F.$-\infty$
G.$0$
H.$-3600$
I.$3600$
\testStop
\kluczStart
A
\kluczStop



\zadStart{Zadanie z Wikieł Z 4.3 c) moja wersja nr 26}
Obliczyć granicę funkcji $f(x)=\frac{-x-4900}{4900-\sqrt{-x}}$.
\zadStop
\rozwStart{Patryk Wirkus}{Szymon Tokarski}
$$\frac{-x-4900}{4900-\sqrt{-x}}=\frac{(-x-4900)(4900+\sqrt{-x})}{4900-(-x)}=-(4900+\sqrt{-x})$$
\\
$$\lim\limits_{x\to-1}\frac{-x-4900}{4900-\sqrt{-x}}=[\frac{0}{0}]=\lim\limits_{x\to-1}-(4900+\sqrt{-x}) =-4901$$
\rozwStop
\odpStart
$-4901$
\odpStop
\testStart
A.$-4901$
B.$-4899$
C.$4899$
D.$4901$
E.$\infty$
F.$-\infty$
G.$0$
H.$-4900$
I.$4900$
\testStop
\kluczStart
A
\kluczStop



\zadStart{Zadanie z Wikieł Z 4.3 c) moja wersja nr 27}
Obliczyć granicę funkcji $f(x)=\frac{-x-6400}{6400-\sqrt{-x}}$.
\zadStop
\rozwStart{Patryk Wirkus}{Szymon Tokarski}
$$\frac{-x-6400}{6400-\sqrt{-x}}=\frac{(-x-6400)(6400+\sqrt{-x})}{6400-(-x)}=-(6400+\sqrt{-x})$$
\\
$$\lim\limits_{x\to-1}\frac{-x-6400}{6400-\sqrt{-x}}=[\frac{0}{0}]=\lim\limits_{x\to-1}-(6400+\sqrt{-x}) =-6401$$
\rozwStop
\odpStart
$-6401$
\odpStop
\testStart
A.$-6401$
B.$-6399$
C.$6399$
D.$6401$
E.$\infty$
F.$-\infty$
G.$0$
H.$-6400$
I.$6400$
\testStop
\kluczStart
A
\kluczStop



\zadStart{Zadanie z Wikieł Z 4.3 c) moja wersja nr 28}
Obliczyć granicę funkcji $f(x)=\frac{-x-8100}{8100-\sqrt{-x}}$.
\zadStop
\rozwStart{Patryk Wirkus}{Szymon Tokarski}
$$\frac{-x-8100}{8100-\sqrt{-x}}=\frac{(-x-8100)(8100+\sqrt{-x})}{8100-(-x)}=-(8100+\sqrt{-x})$$
\\
$$\lim\limits_{x\to-1}\frac{-x-8100}{8100-\sqrt{-x}}=[\frac{0}{0}]=\lim\limits_{x\to-1}-(8100+\sqrt{-x}) =-8101$$
\rozwStop
\odpStart
$-8101$
\odpStop
\testStart
A.$-8101$
B.$-8099$
C.$8099$
D.$8101$
E.$\infty$
F.$-\infty$
G.$0$
H.$-8100$
I.$8100$
\testStop
\kluczStart
A
\kluczStop



\zadStart{Zadanie z Wikieł Z 4.3 c) moja wersja nr 29}
Obliczyć granicę funkcji $f(x)=\frac{-x-10000}{10000-\sqrt{-x}}$.
\zadStop
\rozwStart{Patryk Wirkus}{Szymon Tokarski}
$$\frac{-x-10000}{10000-\sqrt{-x}}=\frac{(-x-10000)(10000+\sqrt{-x})}{10000-(-x)}=-(10000+\sqrt{-x})$$
\\
$$\lim\limits_{x\to-1}\frac{-x-10000}{10000-\sqrt{-x}}=[\frac{0}{0}]=\lim\limits_{x\to-1}-(10000+\sqrt{-x}) =-10001$$
\rozwStop
\odpStart
$-10001$
\odpStop
\testStart
A.$-10001$
B.$-9999$
C.$9999$
D.$10001$
E.$\infty$
F.$-\infty$
G.$0$
H.$-10000$
I.$10000$
\testStop
\kluczStart
A
\kluczStop



\zadStart{Zadanie z Wikieł Z 4.3 c) moja wersja nr 30}
Obliczyć granicę funkcji $f(x)=\frac{-x-2}{2-\sqrt{-x}}$.
\zadStop
\rozwStart{Patryk Wirkus}{Szymon Tokarski}
$$\frac{-x-2}{2-\sqrt{-x}}=\frac{(-x-2)(2+\sqrt{-x})}{2-(-x)}=-(2+\sqrt{-x})$$
\\
$$\lim\limits_{x\to-1}\frac{-x-2}{2-\sqrt{-x}}=[\frac{0}{0}]=\lim\limits_{x\to-1}-(2+\sqrt{-x}) =-3$$
\rozwStop
\odpStart
$-3$
\odpStop
\testStart
A.$-3$
B.$-1$
C.$1$
D.$3$
E.$\infty$
F.$-\infty$
G.$0$
H.$-2$
I.$2$
\testStop
\kluczStart
A
\kluczStop



\zadStart{Zadanie z Wikieł Z 4.3 c) moja wersja nr 31}
Obliczyć granicę funkcji $f(x)=\frac{-x-3}{3-\sqrt{-x}}$.
\zadStop
\rozwStart{Patryk Wirkus}{Szymon Tokarski}
$$\frac{-x-3}{3-\sqrt{-x}}=\frac{(-x-3)(3+\sqrt{-x})}{3-(-x)}=-(3+\sqrt{-x})$$
\\
$$\lim\limits_{x\to-1}\frac{-x-3}{3-\sqrt{-x}}=[\frac{0}{0}]=\lim\limits_{x\to-1}-(3+\sqrt{-x}) =-4$$
\rozwStop
\odpStart
$-4$
\odpStop
\testStart
A.$-4$
B.$-2$
C.$2$
D.$4$
E.$\infty$
F.$-\infty$
G.$0$
H.$-3$
I.$3$
\testStop
\kluczStart
A
\kluczStop



\zadStart{Zadanie z Wikieł Z 4.3 c) moja wersja nr 32}
Obliczyć granicę funkcji $f(x)=\frac{-x-5}{5-\sqrt{-x}}$.
\zadStop
\rozwStart{Patryk Wirkus}{Szymon Tokarski}
$$\frac{-x-5}{5-\sqrt{-x}}=\frac{(-x-5)(5+\sqrt{-x})}{5-(-x)}=-(5+\sqrt{-x})$$
\\
$$\lim\limits_{x\to-1}\frac{-x-5}{5-\sqrt{-x}}=[\frac{0}{0}]=\lim\limits_{x\to-1}-(5+\sqrt{-x}) =-6$$
\rozwStop
\odpStart
$-6$
\odpStop
\testStart
A.$-6$
B.$-4$
C.$4$
D.$6$
E.$\infty$
F.$-\infty$
G.$0$
H.$-5$
I.$5$
\testStop
\kluczStart
A
\kluczStop



\zadStart{Zadanie z Wikieł Z 4.3 c) moja wersja nr 33}
Obliczyć granicę funkcji $f(x)=\frac{-x-7}{7-\sqrt{-x}}$.
\zadStop
\rozwStart{Patryk Wirkus}{Szymon Tokarski}
$$\frac{-x-7}{7-\sqrt{-x}}=\frac{(-x-7)(7+\sqrt{-x})}{7-(-x)}=-(7+\sqrt{-x})$$
\\
$$\lim\limits_{x\to-1}\frac{-x-7}{7-\sqrt{-x}}=[\frac{0}{0}]=\lim\limits_{x\to-1}-(7+\sqrt{-x}) =-8$$
\rozwStop
\odpStart
$-8$
\odpStop
\testStart
A.$-8$
B.$-6$
C.$6$
D.$8$
E.$\infty$
F.$-\infty$
G.$0$
H.$-7$
I.$7$
\testStop
\kluczStart
A
\kluczStop



\zadStart{Zadanie z Wikieł Z 4.3 c) moja wersja nr 34}
Obliczyć granicę funkcji $f(x)=\frac{-x-11}{11-\sqrt{-x}}$.
\zadStop
\rozwStart{Patryk Wirkus}{Szymon Tokarski}
$$\frac{-x-11}{11-\sqrt{-x}}=\frac{(-x-11)(11+\sqrt{-x})}{11-(-x)}=-(11+\sqrt{-x})$$
\\
$$\lim\limits_{x\to-1}\frac{-x-11}{11-\sqrt{-x}}=[\frac{0}{0}]=\lim\limits_{x\to-1}-(11+\sqrt{-x}) =-12$$
\rozwStop
\odpStart
$-12$
\odpStop
\testStart
A.$-12$
B.$-10$
C.$10$
D.$12$
E.$\infty$
F.$-\infty$
G.$0$
H.$-11$
I.$11$
\testStop
\kluczStart
A
\kluczStop



\zadStart{Zadanie z Wikieł Z 4.3 c) moja wersja nr 35}
Obliczyć granicę funkcji $f(x)=\frac{-x-13}{13-\sqrt{-x}}$.
\zadStop
\rozwStart{Patryk Wirkus}{Szymon Tokarski}
$$\frac{-x-13}{13-\sqrt{-x}}=\frac{(-x-13)(13+\sqrt{-x})}{13-(-x)}=-(13+\sqrt{-x})$$
\\
$$\lim\limits_{x\to-1}\frac{-x-13}{13-\sqrt{-x}}=[\frac{0}{0}]=\lim\limits_{x\to-1}-(13+\sqrt{-x}) =-14$$
\rozwStop
\odpStart
$-14$
\odpStop
\testStart
A.$-14$
B.$-12$
C.$12$
D.$14$
E.$\infty$
F.$-\infty$
G.$0$
H.$-13$
I.$13$
\testStop
\kluczStart
A
\kluczStop



\zadStart{Zadanie z Wikieł Z 4.3 c) moja wersja nr 36}
Obliczyć granicę funkcji $f(x)=\frac{-x-17}{17-\sqrt{-x}}$.
\zadStop
\rozwStart{Patryk Wirkus}{Szymon Tokarski}
$$\frac{-x-17}{17-\sqrt{-x}}=\frac{(-x-17)(17+\sqrt{-x})}{17-(-x)}=-(17+\sqrt{-x})$$
\\
$$\lim\limits_{x\to-1}\frac{-x-17}{17-\sqrt{-x}}=[\frac{0}{0}]=\lim\limits_{x\to-1}-(17+\sqrt{-x}) =-18$$
\rozwStop
\odpStart
$-18$
\odpStop
\testStart
A.$-18$
B.$-16$
C.$16$
D.$18$
E.$\infty$
F.$-\infty$
G.$0$
H.$-17$
I.$17$
\testStop
\kluczStart
A
\kluczStop



\zadStart{Zadanie z Wikieł Z 4.3 c) moja wersja nr 37}
Obliczyć granicę funkcji $f(x)=\frac{-x-19}{19-\sqrt{-x}}$.
\zadStop
\rozwStart{Patryk Wirkus}{Szymon Tokarski}
$$\frac{-x-19}{19-\sqrt{-x}}=\frac{(-x-19)(19+\sqrt{-x})}{19-(-x)}=-(19+\sqrt{-x})$$
\\
$$\lim\limits_{x\to-1}\frac{-x-19}{19-\sqrt{-x}}=[\frac{0}{0}]=\lim\limits_{x\to-1}-(19+\sqrt{-x}) =-20$$
\rozwStop
\odpStart
$-20$
\odpStop
\testStart
A.$-20$
B.$-18$
C.$18$
D.$20$
E.$\infty$
F.$-\infty$
G.$0$
H.$-19$
I.$19$
\testStop
\kluczStart
A
\kluczStop



\zadStart{Zadanie z Wikieł Z 4.3 c) moja wersja nr 38}
Obliczyć granicę funkcji $f(x)=\frac{-x-23}{23-\sqrt{-x}}$.
\zadStop
\rozwStart{Patryk Wirkus}{Szymon Tokarski}
$$\frac{-x-23}{23-\sqrt{-x}}=\frac{(-x-23)(23+\sqrt{-x})}{23-(-x)}=-(23+\sqrt{-x})$$
\\
$$\lim\limits_{x\to-1}\frac{-x-23}{23-\sqrt{-x}}=[\frac{0}{0}]=\lim\limits_{x\to-1}-(23+\sqrt{-x}) =-24$$
\rozwStop
\odpStart
$-24$
\odpStop
\testStart
A.$-24$
B.$-22$
C.$22$
D.$24$
E.$\infty$
F.$-\infty$
G.$0$
H.$-23$
I.$23$
\testStop
\kluczStart
A
\kluczStop



\zadStart{Zadanie z Wikieł Z 4.3 c) moja wersja nr 39}
Obliczyć granicę funkcji $f(x)=\frac{-x-29}{29-\sqrt{-x}}$.
\zadStop
\rozwStart{Patryk Wirkus}{Szymon Tokarski}
$$\frac{-x-29}{29-\sqrt{-x}}=\frac{(-x-29)(29+\sqrt{-x})}{29-(-x)}=-(29+\sqrt{-x})$$
\\
$$\lim\limits_{x\to-1}\frac{-x-29}{29-\sqrt{-x}}=[\frac{0}{0}]=\lim\limits_{x\to-1}-(29+\sqrt{-x}) =-30$$
\rozwStop
\odpStart
$-30$
\odpStop
\testStart
A.$-30$
B.$-28$
C.$28$
D.$30$
E.$\infty$
F.$-\infty$
G.$0$
H.$-29$
I.$29$
\testStop
\kluczStart
A
\kluczStop



\zadStart{Zadanie z Wikieł Z 4.3 c) moja wersja nr 40}
Obliczyć granicę funkcji $f(x)=\frac{-x-31}{31-\sqrt{-x}}$.
\zadStop
\rozwStart{Patryk Wirkus}{Szymon Tokarski}
$$\frac{-x-31}{31-\sqrt{-x}}=\frac{(-x-31)(31+\sqrt{-x})}{31-(-x)}=-(31+\sqrt{-x})$$
\\
$$\lim\limits_{x\to-1}\frac{-x-31}{31-\sqrt{-x}}=[\frac{0}{0}]=\lim\limits_{x\to-1}-(31+\sqrt{-x}) =-32$$
\rozwStop
\odpStart
$-32$
\odpStop
\testStart
A.$-32$
B.$-30$
C.$30$
D.$32$
E.$\infty$
F.$-\infty$
G.$0$
H.$-31$
I.$31$
\testStop
\kluczStart
A
\kluczStop



\zadStart{Zadanie z Wikieł Z 4.3 c) moja wersja nr 41}
Obliczyć granicę funkcji $f(x)=\frac{-x-37}{37-\sqrt{-x}}$.
\zadStop
\rozwStart{Patryk Wirkus}{Szymon Tokarski}
$$\frac{-x-37}{37-\sqrt{-x}}=\frac{(-x-37)(37+\sqrt{-x})}{37-(-x)}=-(37+\sqrt{-x})$$
\\
$$\lim\limits_{x\to-1}\frac{-x-37}{37-\sqrt{-x}}=[\frac{0}{0}]=\lim\limits_{x\to-1}-(37+\sqrt{-x}) =-38$$
\rozwStop
\odpStart
$-38$
\odpStop
\testStart
A.$-38$
B.$-36$
C.$36$
D.$38$
E.$\infty$
F.$-\infty$
G.$0$
H.$-37$
I.$37$
\testStop
\kluczStart
A
\kluczStop



\zadStart{Zadanie z Wikieł Z 4.3 c) moja wersja nr 42}
Obliczyć granicę funkcji $f(x)=\frac{-x-41}{41-\sqrt{-x}}$.
\zadStop
\rozwStart{Patryk Wirkus}{Szymon Tokarski}
$$\frac{-x-41}{41-\sqrt{-x}}=\frac{(-x-41)(41+\sqrt{-x})}{41-(-x)}=-(41+\sqrt{-x})$$
\\
$$\lim\limits_{x\to-1}\frac{-x-41}{41-\sqrt{-x}}=[\frac{0}{0}]=\lim\limits_{x\to-1}-(41+\sqrt{-x}) =-42$$
\rozwStop
\odpStart
$-42$
\odpStop
\testStart
A.$-42$
B.$-40$
C.$40$
D.$42$
E.$\infty$
F.$-\infty$
G.$0$
H.$-41$
I.$41$
\testStop
\kluczStart
A
\kluczStop



\zadStart{Zadanie z Wikieł Z 4.3 c) moja wersja nr 43}
Obliczyć granicę funkcji $f(x)=\frac{-x-43}{43-\sqrt{-x}}$.
\zadStop
\rozwStart{Patryk Wirkus}{Szymon Tokarski}
$$\frac{-x-43}{43-\sqrt{-x}}=\frac{(-x-43)(43+\sqrt{-x})}{43-(-x)}=-(43+\sqrt{-x})$$
\\
$$\lim\limits_{x\to-1}\frac{-x-43}{43-\sqrt{-x}}=[\frac{0}{0}]=\lim\limits_{x\to-1}-(43+\sqrt{-x}) =-44$$
\rozwStop
\odpStart
$-44$
\odpStop
\testStart
A.$-44$
B.$-42$
C.$42$
D.$44$
E.$\infty$
F.$-\infty$
G.$0$
H.$-43$
I.$43$
\testStop
\kluczStart
A
\kluczStop



\zadStart{Zadanie z Wikieł Z 4.3 c) moja wersja nr 44}
Obliczyć granicę funkcji $f(x)=\frac{-x-47}{47-\sqrt{-x}}$.
\zadStop
\rozwStart{Patryk Wirkus}{Szymon Tokarski}
$$\frac{-x-47}{47-\sqrt{-x}}=\frac{(-x-47)(47+\sqrt{-x})}{47-(-x)}=-(47+\sqrt{-x})$$
\\
$$\lim\limits_{x\to-1}\frac{-x-47}{47-\sqrt{-x}}=[\frac{0}{0}]=\lim\limits_{x\to-1}-(47+\sqrt{-x}) =-48$$
\rozwStop
\odpStart
$-48$
\odpStop
\testStart
A.$-48$
B.$-46$
C.$46$
D.$48$
E.$\infty$
F.$-\infty$
G.$0$
H.$-47$
I.$47$
\testStop
\kluczStart
A
\kluczStop



\zadStart{Zadanie z Wikieł Z 4.3 c) moja wersja nr 45}
Obliczyć granicę funkcji $f(x)=\frac{-x-53}{53-\sqrt{-x}}$.
\zadStop
\rozwStart{Patryk Wirkus}{Szymon Tokarski}
$$\frac{-x-53}{53-\sqrt{-x}}=\frac{(-x-53)(53+\sqrt{-x})}{53-(-x)}=-(53+\sqrt{-x})$$
\\
$$\lim\limits_{x\to-1}\frac{-x-53}{53-\sqrt{-x}}=[\frac{0}{0}]=\lim\limits_{x\to-1}-(53+\sqrt{-x}) =-54$$
\rozwStop
\odpStart
$-54$
\odpStop
\testStart
A.$-54$
B.$-52$
C.$52$
D.$54$
E.$\infty$
F.$-\infty$
G.$0$
H.$-53$
I.$53$
\testStop
\kluczStart
A
\kluczStop



\zadStart{Zadanie z Wikieł Z 4.3 c) moja wersja nr 46}
Obliczyć granicę funkcji $f(x)=\frac{-x-59}{59-\sqrt{-x}}$.
\zadStop
\rozwStart{Patryk Wirkus}{Szymon Tokarski}
$$\frac{-x-59}{59-\sqrt{-x}}=\frac{(-x-59)(59+\sqrt{-x})}{59-(-x)}=-(59+\sqrt{-x})$$
\\
$$\lim\limits_{x\to-1}\frac{-x-59}{59-\sqrt{-x}}=[\frac{0}{0}]=\lim\limits_{x\to-1}-(59+\sqrt{-x}) =-60$$
\rozwStop
\odpStart
$-60$
\odpStop
\testStart
A.$-60$
B.$-58$
C.$58$
D.$60$
E.$\infty$
F.$-\infty$
G.$0$
H.$-59$
I.$59$
\testStop
\kluczStart
A
\kluczStop



\zadStart{Zadanie z Wikieł Z 4.3 c) moja wersja nr 47}
Obliczyć granicę funkcji $f(x)=\frac{-x-61}{61-\sqrt{-x}}$.
\zadStop
\rozwStart{Patryk Wirkus}{Szymon Tokarski}
$$\frac{-x-61}{61-\sqrt{-x}}=\frac{(-x-61)(61+\sqrt{-x})}{61-(-x)}=-(61+\sqrt{-x})$$
\\
$$\lim\limits_{x\to-1}\frac{-x-61}{61-\sqrt{-x}}=[\frac{0}{0}]=\lim\limits_{x\to-1}-(61+\sqrt{-x}) =-62$$
\rozwStop
\odpStart
$-62$
\odpStop
\testStart
A.$-62$
B.$-60$
C.$60$
D.$62$
E.$\infty$
F.$-\infty$
G.$0$
H.$-61$
I.$61$
\testStop
\kluczStart
A
\kluczStop



\zadStart{Zadanie z Wikieł Z 4.3 c) moja wersja nr 48}
Obliczyć granicę funkcji $f(x)=\frac{-x-67}{67-\sqrt{-x}}$.
\zadStop
\rozwStart{Patryk Wirkus}{Szymon Tokarski}
$$\frac{-x-67}{67-\sqrt{-x}}=\frac{(-x-67)(67+\sqrt{-x})}{67-(-x)}=-(67+\sqrt{-x})$$
\\
$$\lim\limits_{x\to-1}\frac{-x-67}{67-\sqrt{-x}}=[\frac{0}{0}]=\lim\limits_{x\to-1}-(67+\sqrt{-x}) =-68$$
\rozwStop
\odpStart
$-68$
\odpStop
\testStart
A.$-68$
B.$-66$
C.$66$
D.$68$
E.$\infty$
F.$-\infty$
G.$0$
H.$-67$
I.$67$
\testStop
\kluczStart
A
\kluczStop



\zadStart{Zadanie z Wikieł Z 4.3 c) moja wersja nr 49}
Obliczyć granicę funkcji $f(x)=\frac{-x-71}{71-\sqrt{-x}}$.
\zadStop
\rozwStart{Patryk Wirkus}{Szymon Tokarski}
$$\frac{-x-71}{71-\sqrt{-x}}=\frac{(-x-71)(71+\sqrt{-x})}{71-(-x)}=-(71+\sqrt{-x})$$
\\
$$\lim\limits_{x\to-1}\frac{-x-71}{71-\sqrt{-x}}=[\frac{0}{0}]=\lim\limits_{x\to-1}-(71+\sqrt{-x}) =-72$$
\rozwStop
\odpStart
$-72$
\odpStop
\testStart
A.$-72$
B.$-70$
C.$70$
D.$72$
E.$\infty$
F.$-\infty$
G.$0$
H.$-71$
I.$71$
\testStop
\kluczStart
A
\kluczStop



\zadStart{Zadanie z Wikieł Z 4.3 c) moja wersja nr 50}
Obliczyć granicę funkcji $f(x)=\frac{-x-73}{73-\sqrt{-x}}$.
\zadStop
\rozwStart{Patryk Wirkus}{Szymon Tokarski}
$$\frac{-x-73}{73-\sqrt{-x}}=\frac{(-x-73)(73+\sqrt{-x})}{73-(-x)}=-(73+\sqrt{-x})$$
\\
$$\lim\limits_{x\to-1}\frac{-x-73}{73-\sqrt{-x}}=[\frac{0}{0}]=\lim\limits_{x\to-1}-(73+\sqrt{-x}) =-74$$
\rozwStop
\odpStart
$-74$
\odpStop
\testStart
A.$-74$
B.$-72$
C.$72$
D.$74$
E.$\infty$
F.$-\infty$
G.$0$
H.$-73$
I.$73$
\testStop
\kluczStart
A
\kluczStop



\zadStart{Zadanie z Wikieł Z 4.3 c) moja wersja nr 51}
Obliczyć granicę funkcji $f(x)=\frac{-x-79}{79-\sqrt{-x}}$.
\zadStop
\rozwStart{Patryk Wirkus}{Szymon Tokarski}
$$\frac{-x-79}{79-\sqrt{-x}}=\frac{(-x-79)(79+\sqrt{-x})}{79-(-x)}=-(79+\sqrt{-x})$$
\\
$$\lim\limits_{x\to-1}\frac{-x-79}{79-\sqrt{-x}}=[\frac{0}{0}]=\lim\limits_{x\to-1}-(79+\sqrt{-x}) =-80$$
\rozwStop
\odpStart
$-80$
\odpStop
\testStart
A.$-80$
B.$-78$
C.$78$
D.$80$
E.$\infty$
F.$-\infty$
G.$0$
H.$-79$
I.$79$
\testStop
\kluczStart
A
\kluczStop



\zadStart{Zadanie z Wikieł Z 4.3 c) moja wersja nr 52}
Obliczyć granicę funkcji $f(x)=\frac{-x-83}{83-\sqrt{-x}}$.
\zadStop
\rozwStart{Patryk Wirkus}{Szymon Tokarski}
$$\frac{-x-83}{83-\sqrt{-x}}=\frac{(-x-83)(83+\sqrt{-x})}{83-(-x)}=-(83+\sqrt{-x})$$
\\
$$\lim\limits_{x\to-1}\frac{-x-83}{83-\sqrt{-x}}=[\frac{0}{0}]=\lim\limits_{x\to-1}-(83+\sqrt{-x}) =-84$$
\rozwStop
\odpStart
$-84$
\odpStop
\testStart
A.$-84$
B.$-82$
C.$82$
D.$84$
E.$\infty$
F.$-\infty$
G.$0$
H.$-83$
I.$83$
\testStop
\kluczStart
A
\kluczStop



\zadStart{Zadanie z Wikieł Z 4.3 c) moja wersja nr 53}
Obliczyć granicę funkcji $f(x)=\frac{-x-89}{89-\sqrt{-x}}$.
\zadStop
\rozwStart{Patryk Wirkus}{Szymon Tokarski}
$$\frac{-x-89}{89-\sqrt{-x}}=\frac{(-x-89)(89+\sqrt{-x})}{89-(-x)}=-(89+\sqrt{-x})$$
\\
$$\lim\limits_{x\to-1}\frac{-x-89}{89-\sqrt{-x}}=[\frac{0}{0}]=\lim\limits_{x\to-1}-(89+\sqrt{-x}) =-90$$
\rozwStop
\odpStart
$-90$
\odpStop
\testStart
A.$-90$
B.$-88$
C.$88$
D.$90$
E.$\infty$
F.$-\infty$
G.$0$
H.$-89$
I.$89$
\testStop
\kluczStart
A
\kluczStop



\zadStart{Zadanie z Wikieł Z 4.3 c) moja wersja nr 54}
Obliczyć granicę funkcji $f(x)=\frac{-x-97}{97-\sqrt{-x}}$.
\zadStop
\rozwStart{Patryk Wirkus}{Szymon Tokarski}
$$\frac{-x-97}{97-\sqrt{-x}}=\frac{(-x-97)(97+\sqrt{-x})}{97-(-x)}=-(97+\sqrt{-x})$$
\\
$$\lim\limits_{x\to-1}\frac{-x-97}{97-\sqrt{-x}}=[\frac{0}{0}]=\lim\limits_{x\to-1}-(97+\sqrt{-x}) =-98$$
\rozwStop
\odpStart
$-98$
\odpStop
\testStart
A.$-98$
B.$-96$
C.$96$
D.$98$
E.$\infty$
F.$-\infty$
G.$0$
H.$-97$
I.$97$
\testStop
\kluczStart
A
\kluczStop



\zadStart{Zadanie z Wikieł Z 4.3 c) moja wersja nr 55}
Obliczyć granicę funkcji $f(x)=\frac{-x-101}{101-\sqrt{-x}}$.
\zadStop
\rozwStart{Patryk Wirkus}{Szymon Tokarski}
$$\frac{-x-101}{101-\sqrt{-x}}=\frac{(-x-101)(101+\sqrt{-x})}{101-(-x)}=-(101+\sqrt{-x})$$
\\
$$\lim\limits_{x\to-1}\frac{-x-101}{101-\sqrt{-x}}=[\frac{0}{0}]=\lim\limits_{x\to-1}-(101+\sqrt{-x}) =-102$$
\rozwStop
\odpStart
$-102$
\odpStop
\testStart
A.$-102$
B.$-100$
C.$100$
D.$102$
E.$\infty$
F.$-\infty$
G.$0$
H.$-101$
I.$101$
\testStop
\kluczStart
A
\kluczStop



\zadStart{Zadanie z Wikieł Z 4.3 c) moja wersja nr 56}
Obliczyć granicę funkcji $f(x)=\frac{-x-103}{103-\sqrt{-x}}$.
\zadStop
\rozwStart{Patryk Wirkus}{Szymon Tokarski}
$$\frac{-x-103}{103-\sqrt{-x}}=\frac{(-x-103)(103+\sqrt{-x})}{103-(-x)}=-(103+\sqrt{-x})$$
\\
$$\lim\limits_{x\to-1}\frac{-x-103}{103-\sqrt{-x}}=[\frac{0}{0}]=\lim\limits_{x\to-1}-(103+\sqrt{-x}) =-104$$
\rozwStop
\odpStart
$-104$
\odpStop
\testStart
A.$-104$
B.$-102$
C.$102$
D.$104$
E.$\infty$
F.$-\infty$
G.$0$
H.$-103$
I.$103$
\testStop
\kluczStart
A
\kluczStop



\zadStart{Zadanie z Wikieł Z 4.3 c) moja wersja nr 57}
Obliczyć granicę funkcji $f(x)=\frac{-x-107}{107-\sqrt{-x}}$.
\zadStop
\rozwStart{Patryk Wirkus}{Szymon Tokarski}
$$\frac{-x-107}{107-\sqrt{-x}}=\frac{(-x-107)(107+\sqrt{-x})}{107-(-x)}=-(107+\sqrt{-x})$$
\\
$$\lim\limits_{x\to-1}\frac{-x-107}{107-\sqrt{-x}}=[\frac{0}{0}]=\lim\limits_{x\to-1}-(107+\sqrt{-x}) =-108$$
\rozwStop
\odpStart
$-108$
\odpStop
\testStart
A.$-108$
B.$-106$
C.$106$
D.$108$
E.$\infty$
F.$-\infty$
G.$0$
H.$-107$
I.$107$
\testStop
\kluczStart
A
\kluczStop



\zadStart{Zadanie z Wikieł Z 4.3 c) moja wersja nr 58}
Obliczyć granicę funkcji $f(x)=\frac{-x-109}{109-\sqrt{-x}}$.
\zadStop
\rozwStart{Patryk Wirkus}{Szymon Tokarski}
$$\frac{-x-109}{109-\sqrt{-x}}=\frac{(-x-109)(109+\sqrt{-x})}{109-(-x)}=-(109+\sqrt{-x})$$
\\
$$\lim\limits_{x\to-1}\frac{-x-109}{109-\sqrt{-x}}=[\frac{0}{0}]=\lim\limits_{x\to-1}-(109+\sqrt{-x}) =-110$$
\rozwStop
\odpStart
$-110$
\odpStop
\testStart
A.$-110$
B.$-108$
C.$108$
D.$110$
E.$\infty$
F.$-\infty$
G.$0$
H.$-109$
I.$109$
\testStop
\kluczStart
A
\kluczStop



\zadStart{Zadanie z Wikieł Z 4.3 c) moja wersja nr 59}
Obliczyć granicę funkcji $f(x)=\frac{-x-113}{113-\sqrt{-x}}$.
\zadStop
\rozwStart{Patryk Wirkus}{Szymon Tokarski}
$$\frac{-x-113}{113-\sqrt{-x}}=\frac{(-x-113)(113+\sqrt{-x})}{113-(-x)}=-(113+\sqrt{-x})$$
\\
$$\lim\limits_{x\to-1}\frac{-x-113}{113-\sqrt{-x}}=[\frac{0}{0}]=\lim\limits_{x\to-1}-(113+\sqrt{-x}) =-114$$
\rozwStop
\odpStart
$-114$
\odpStop
\testStart
A.$-114$
B.$-112$
C.$112$
D.$114$
E.$\infty$
F.$-\infty$
G.$0$
H.$-113$
I.$113$
\testStop
\kluczStart
A
\kluczStop



\zadStart{Zadanie z Wikieł Z 4.3 c) moja wersja nr 60}
Obliczyć granicę funkcji $f(x)=\frac{-x-127}{127-\sqrt{-x}}$.
\zadStop
\rozwStart{Patryk Wirkus}{Szymon Tokarski}
$$\frac{-x-127}{127-\sqrt{-x}}=\frac{(-x-127)(127+\sqrt{-x})}{127-(-x)}=-(127+\sqrt{-x})$$
\\
$$\lim\limits_{x\to-1}\frac{-x-127}{127-\sqrt{-x}}=[\frac{0}{0}]=\lim\limits_{x\to-1}-(127+\sqrt{-x}) =-128$$
\rozwStop
\odpStart
$-128$
\odpStop
\testStart
A.$-128$
B.$-126$
C.$126$
D.$128$
E.$\infty$
F.$-\infty$
G.$0$
H.$-127$
I.$127$
\testStop
\kluczStart
A
\kluczStop



\zadStart{Zadanie z Wikieł Z 4.3 c) moja wersja nr 61}
Obliczyć granicę funkcji $f(x)=\frac{-x-131}{131-\sqrt{-x}}$.
\zadStop
\rozwStart{Patryk Wirkus}{Szymon Tokarski}
$$\frac{-x-131}{131-\sqrt{-x}}=\frac{(-x-131)(131+\sqrt{-x})}{131-(-x)}=-(131+\sqrt{-x})$$
\\
$$\lim\limits_{x\to-1}\frac{-x-131}{131-\sqrt{-x}}=[\frac{0}{0}]=\lim\limits_{x\to-1}-(131+\sqrt{-x}) =-132$$
\rozwStop
\odpStart
$-132$
\odpStop
\testStart
A.$-132$
B.$-130$
C.$130$
D.$132$
E.$\infty$
F.$-\infty$
G.$0$
H.$-131$
I.$131$
\testStop
\kluczStart
A
\kluczStop



\zadStart{Zadanie z Wikieł Z 4.3 c) moja wersja nr 62}
Obliczyć granicę funkcji $f(x)=\frac{-x-137}{137-\sqrt{-x}}$.
\zadStop
\rozwStart{Patryk Wirkus}{Szymon Tokarski}
$$\frac{-x-137}{137-\sqrt{-x}}=\frac{(-x-137)(137+\sqrt{-x})}{137-(-x)}=-(137+\sqrt{-x})$$
\\
$$\lim\limits_{x\to-1}\frac{-x-137}{137-\sqrt{-x}}=[\frac{0}{0}]=\lim\limits_{x\to-1}-(137+\sqrt{-x}) =-138$$
\rozwStop
\odpStart
$-138$
\odpStop
\testStart
A.$-138$
B.$-136$
C.$136$
D.$138$
E.$\infty$
F.$-\infty$
G.$0$
H.$-137$
I.$137$
\testStop
\kluczStart
A
\kluczStop



\zadStart{Zadanie z Wikieł Z 4.3 c) moja wersja nr 63}
Obliczyć granicę funkcji $f(x)=\frac{-x-139}{139-\sqrt{-x}}$.
\zadStop
\rozwStart{Patryk Wirkus}{Szymon Tokarski}
$$\frac{-x-139}{139-\sqrt{-x}}=\frac{(-x-139)(139+\sqrt{-x})}{139-(-x)}=-(139+\sqrt{-x})$$
\\
$$\lim\limits_{x\to-1}\frac{-x-139}{139-\sqrt{-x}}=[\frac{0}{0}]=\lim\limits_{x\to-1}-(139+\sqrt{-x}) =-140$$
\rozwStop
\odpStart
$-140$
\odpStop
\testStart
A.$-140$
B.$-138$
C.$138$
D.$140$
E.$\infty$
F.$-\infty$
G.$0$
H.$-139$
I.$139$
\testStop
\kluczStart
A
\kluczStop



\zadStart{Zadanie z Wikieł Z 4.3 c) moja wersja nr 64}
Obliczyć granicę funkcji $f(x)=\frac{-x-149}{149-\sqrt{-x}}$.
\zadStop
\rozwStart{Patryk Wirkus}{Szymon Tokarski}
$$\frac{-x-149}{149-\sqrt{-x}}=\frac{(-x-149)(149+\sqrt{-x})}{149-(-x)}=-(149+\sqrt{-x})$$
\\
$$\lim\limits_{x\to-1}\frac{-x-149}{149-\sqrt{-x}}=[\frac{0}{0}]=\lim\limits_{x\to-1}-(149+\sqrt{-x}) =-150$$
\rozwStop
\odpStart
$-150$
\odpStop
\testStart
A.$-150$
B.$-148$
C.$148$
D.$150$
E.$\infty$
F.$-\infty$
G.$0$
H.$-149$
I.$149$
\testStop
\kluczStart
A
\kluczStop



\zadStart{Zadanie z Wikieł Z 4.3 c) moja wersja nr 65}
Obliczyć granicę funkcji $f(x)=\frac{-x-151}{151-\sqrt{-x}}$.
\zadStop
\rozwStart{Patryk Wirkus}{Szymon Tokarski}
$$\frac{-x-151}{151-\sqrt{-x}}=\frac{(-x-151)(151+\sqrt{-x})}{151-(-x)}=-(151+\sqrt{-x})$$
\\
$$\lim\limits_{x\to-1}\frac{-x-151}{151-\sqrt{-x}}=[\frac{0}{0}]=\lim\limits_{x\to-1}-(151+\sqrt{-x}) =-152$$
\rozwStop
\odpStart
$-152$
\odpStop
\testStart
A.$-152$
B.$-150$
C.$150$
D.$152$
E.$\infty$
F.$-\infty$
G.$0$
H.$-151$
I.$151$
\testStop
\kluczStart
A
\kluczStop



\zadStart{Zadanie z Wikieł Z 4.3 c) moja wersja nr 66}
Obliczyć granicę funkcji $f(x)=\frac{-x-157}{157-\sqrt{-x}}$.
\zadStop
\rozwStart{Patryk Wirkus}{Szymon Tokarski}
$$\frac{-x-157}{157-\sqrt{-x}}=\frac{(-x-157)(157+\sqrt{-x})}{157-(-x)}=-(157+\sqrt{-x})$$
\\
$$\lim\limits_{x\to-1}\frac{-x-157}{157-\sqrt{-x}}=[\frac{0}{0}]=\lim\limits_{x\to-1}-(157+\sqrt{-x}) =-158$$
\rozwStop
\odpStart
$-158$
\odpStop
\testStart
A.$-158$
B.$-156$
C.$156$
D.$158$
E.$\infty$
F.$-\infty$
G.$0$
H.$-157$
I.$157$
\testStop
\kluczStart
A
\kluczStop



\zadStart{Zadanie z Wikieł Z 4.3 c) moja wersja nr 67}
Obliczyć granicę funkcji $f(x)=\frac{-x-163}{163-\sqrt{-x}}$.
\zadStop
\rozwStart{Patryk Wirkus}{Szymon Tokarski}
$$\frac{-x-163}{163-\sqrt{-x}}=\frac{(-x-163)(163+\sqrt{-x})}{163-(-x)}=-(163+\sqrt{-x})$$
\\
$$\lim\limits_{x\to-1}\frac{-x-163}{163-\sqrt{-x}}=[\frac{0}{0}]=\lim\limits_{x\to-1}-(163+\sqrt{-x}) =-164$$
\rozwStop
\odpStart
$-164$
\odpStop
\testStart
A.$-164$
B.$-162$
C.$162$
D.$164$
E.$\infty$
F.$-\infty$
G.$0$
H.$-163$
I.$163$
\testStop
\kluczStart
A
\kluczStop



\zadStart{Zadanie z Wikieł Z 4.3 c) moja wersja nr 68}
Obliczyć granicę funkcji $f(x)=\frac{-x-167}{167-\sqrt{-x}}$.
\zadStop
\rozwStart{Patryk Wirkus}{Szymon Tokarski}
$$\frac{-x-167}{167-\sqrt{-x}}=\frac{(-x-167)(167+\sqrt{-x})}{167-(-x)}=-(167+\sqrt{-x})$$
\\
$$\lim\limits_{x\to-1}\frac{-x-167}{167-\sqrt{-x}}=[\frac{0}{0}]=\lim\limits_{x\to-1}-(167+\sqrt{-x}) =-168$$
\rozwStop
\odpStart
$-168$
\odpStop
\testStart
A.$-168$
B.$-166$
C.$166$
D.$168$
E.$\infty$
F.$-\infty$
G.$0$
H.$-167$
I.$167$
\testStop
\kluczStart
A
\kluczStop



\zadStart{Zadanie z Wikieł Z 4.3 c) moja wersja nr 69}
Obliczyć granicę funkcji $f(x)=\frac{-x-173}{173-\sqrt{-x}}$.
\zadStop
\rozwStart{Patryk Wirkus}{Szymon Tokarski}
$$\frac{-x-173}{173-\sqrt{-x}}=\frac{(-x-173)(173+\sqrt{-x})}{173-(-x)}=-(173+\sqrt{-x})$$
\\
$$\lim\limits_{x\to-1}\frac{-x-173}{173-\sqrt{-x}}=[\frac{0}{0}]=\lim\limits_{x\to-1}-(173+\sqrt{-x}) =-174$$
\rozwStop
\odpStart
$-174$
\odpStop
\testStart
A.$-174$
B.$-172$
C.$172$
D.$174$
E.$\infty$
F.$-\infty$
G.$0$
H.$-173$
I.$173$
\testStop
\kluczStart
A
\kluczStop



\zadStart{Zadanie z Wikieł Z 4.3 c) moja wersja nr 70}
Obliczyć granicę funkcji $f(x)=\frac{-x-179}{179-\sqrt{-x}}$.
\zadStop
\rozwStart{Patryk Wirkus}{Szymon Tokarski}
$$\frac{-x-179}{179-\sqrt{-x}}=\frac{(-x-179)(179+\sqrt{-x})}{179-(-x)}=-(179+\sqrt{-x})$$
\\
$$\lim\limits_{x\to-1}\frac{-x-179}{179-\sqrt{-x}}=[\frac{0}{0}]=\lim\limits_{x\to-1}-(179+\sqrt{-x}) =-180$$
\rozwStop
\odpStart
$-180$
\odpStop
\testStart
A.$-180$
B.$-178$
C.$178$
D.$180$
E.$\infty$
F.$-\infty$
G.$0$
H.$-179$
I.$179$
\testStop
\kluczStart
A
\kluczStop



\zadStart{Zadanie z Wikieł Z 4.3 c) moja wersja nr 71}
Obliczyć granicę funkcji $f(x)=\frac{-x-181}{181-\sqrt{-x}}$.
\zadStop
\rozwStart{Patryk Wirkus}{Szymon Tokarski}
$$\frac{-x-181}{181-\sqrt{-x}}=\frac{(-x-181)(181+\sqrt{-x})}{181-(-x)}=-(181+\sqrt{-x})$$
\\
$$\lim\limits_{x\to-1}\frac{-x-181}{181-\sqrt{-x}}=[\frac{0}{0}]=\lim\limits_{x\to-1}-(181+\sqrt{-x}) =-182$$
\rozwStop
\odpStart
$-182$
\odpStop
\testStart
A.$-182$
B.$-180$
C.$180$
D.$182$
E.$\infty$
F.$-\infty$
G.$0$
H.$-181$
I.$181$
\testStop
\kluczStart
A
\kluczStop



\zadStart{Zadanie z Wikieł Z 4.3 c) moja wersja nr 72}
Obliczyć granicę funkcji $f(x)=\frac{-x-191}{191-\sqrt{-x}}$.
\zadStop
\rozwStart{Patryk Wirkus}{Szymon Tokarski}
$$\frac{-x-191}{191-\sqrt{-x}}=\frac{(-x-191)(191+\sqrt{-x})}{191-(-x)}=-(191+\sqrt{-x})$$
\\
$$\lim\limits_{x\to-1}\frac{-x-191}{191-\sqrt{-x}}=[\frac{0}{0}]=\lim\limits_{x\to-1}-(191+\sqrt{-x}) =-192$$
\rozwStop
\odpStart
$-192$
\odpStop
\testStart
A.$-192$
B.$-190$
C.$190$
D.$192$
E.$\infty$
F.$-\infty$
G.$0$
H.$-191$
I.$191$
\testStop
\kluczStart
A
\kluczStop



\zadStart{Zadanie z Wikieł Z 4.3 c) moja wersja nr 73}
Obliczyć granicę funkcji $f(x)=\frac{-x-193}{193-\sqrt{-x}}$.
\zadStop
\rozwStart{Patryk Wirkus}{Szymon Tokarski}
$$\frac{-x-193}{193-\sqrt{-x}}=\frac{(-x-193)(193+\sqrt{-x})}{193-(-x)}=-(193+\sqrt{-x})$$
\\
$$\lim\limits_{x\to-1}\frac{-x-193}{193-\sqrt{-x}}=[\frac{0}{0}]=\lim\limits_{x\to-1}-(193+\sqrt{-x}) =-194$$
\rozwStop
\odpStart
$-194$
\odpStop
\testStart
A.$-194$
B.$-192$
C.$192$
D.$194$
E.$\infty$
F.$-\infty$
G.$0$
H.$-193$
I.$193$
\testStop
\kluczStart
A
\kluczStop



\zadStart{Zadanie z Wikieł Z 4.3 c) moja wersja nr 74}
Obliczyć granicę funkcji $f(x)=\frac{-x-197}{197-\sqrt{-x}}$.
\zadStop
\rozwStart{Patryk Wirkus}{Szymon Tokarski}
$$\frac{-x-197}{197-\sqrt{-x}}=\frac{(-x-197)(197+\sqrt{-x})}{197-(-x)}=-(197+\sqrt{-x})$$
\\
$$\lim\limits_{x\to-1}\frac{-x-197}{197-\sqrt{-x}}=[\frac{0}{0}]=\lim\limits_{x\to-1}-(197+\sqrt{-x}) =-198$$
\rozwStop
\odpStart
$-198$
\odpStop
\testStart
A.$-198$
B.$-196$
C.$196$
D.$198$
E.$\infty$
F.$-\infty$
G.$0$
H.$-197$
I.$197$
\testStop
\kluczStart
A
\kluczStop



\zadStart{Zadanie z Wikieł Z 4.3 c) moja wersja nr 75}
Obliczyć granicę funkcji $f(x)=\frac{-x-199}{199-\sqrt{-x}}$.
\zadStop
\rozwStart{Patryk Wirkus}{Szymon Tokarski}
$$\frac{-x-199}{199-\sqrt{-x}}=\frac{(-x-199)(199+\sqrt{-x})}{199-(-x)}=-(199+\sqrt{-x})$$
\\
$$\lim\limits_{x\to-1}\frac{-x-199}{199-\sqrt{-x}}=[\frac{0}{0}]=\lim\limits_{x\to-1}-(199+\sqrt{-x}) =-200$$
\rozwStop
\odpStart
$-200$
\odpStop
\testStart
A.$-200$
B.$-198$
C.$198$
D.$200$
E.$\infty$
F.$-\infty$
G.$0$
H.$-199$
I.$199$
\testStop
\kluczStart
A
\kluczStop



\zadStart{Zadanie z Wikieł Z 4.3 c) moja wersja nr 76}
Obliczyć granicę funkcji $f(x)=\frac{-x-211}{211-\sqrt{-x}}$.
\zadStop
\rozwStart{Patryk Wirkus}{Szymon Tokarski}
$$\frac{-x-211}{211-\sqrt{-x}}=\frac{(-x-211)(211+\sqrt{-x})}{211-(-x)}=-(211+\sqrt{-x})$$
\\
$$\lim\limits_{x\to-1}\frac{-x-211}{211-\sqrt{-x}}=[\frac{0}{0}]=\lim\limits_{x\to-1}-(211+\sqrt{-x}) =-212$$
\rozwStop
\odpStart
$-212$
\odpStop
\testStart
A.$-212$
B.$-210$
C.$210$
D.$212$
E.$\infty$
F.$-\infty$
G.$0$
H.$-211$
I.$211$
\testStop
\kluczStart
A
\kluczStop



\zadStart{Zadanie z Wikieł Z 4.3 c) moja wersja nr 77}
Obliczyć granicę funkcji $f(x)=\frac{-x-223}{223-\sqrt{-x}}$.
\zadStop
\rozwStart{Patryk Wirkus}{Szymon Tokarski}
$$\frac{-x-223}{223-\sqrt{-x}}=\frac{(-x-223)(223+\sqrt{-x})}{223-(-x)}=-(223+\sqrt{-x})$$
\\
$$\lim\limits_{x\to-1}\frac{-x-223}{223-\sqrt{-x}}=[\frac{0}{0}]=\lim\limits_{x\to-1}-(223+\sqrt{-x}) =-224$$
\rozwStop
\odpStart
$-224$
\odpStop
\testStart
A.$-224$
B.$-222$
C.$222$
D.$224$
E.$\infty$
F.$-\infty$
G.$0$
H.$-223$
I.$223$
\testStop
\kluczStart
A
\kluczStop



\zadStart{Zadanie z Wikieł Z 4.3 c) moja wersja nr 78}
Obliczyć granicę funkcji $f(x)=\frac{-x-227}{227-\sqrt{-x}}$.
\zadStop
\rozwStart{Patryk Wirkus}{Szymon Tokarski}
$$\frac{-x-227}{227-\sqrt{-x}}=\frac{(-x-227)(227+\sqrt{-x})}{227-(-x)}=-(227+\sqrt{-x})$$
\\
$$\lim\limits_{x\to-1}\frac{-x-227}{227-\sqrt{-x}}=[\frac{0}{0}]=\lim\limits_{x\to-1}-(227+\sqrt{-x}) =-228$$
\rozwStop
\odpStart
$-228$
\odpStop
\testStart
A.$-228$
B.$-226$
C.$226$
D.$228$
E.$\infty$
F.$-\infty$
G.$0$
H.$-227$
I.$227$
\testStop
\kluczStart
A
\kluczStop



\zadStart{Zadanie z Wikieł Z 4.3 c) moja wersja nr 79}
Obliczyć granicę funkcji $f(x)=\frac{-x-229}{229-\sqrt{-x}}$.
\zadStop
\rozwStart{Patryk Wirkus}{Szymon Tokarski}
$$\frac{-x-229}{229-\sqrt{-x}}=\frac{(-x-229)(229+\sqrt{-x})}{229-(-x)}=-(229+\sqrt{-x})$$
\\
$$\lim\limits_{x\to-1}\frac{-x-229}{229-\sqrt{-x}}=[\frac{0}{0}]=\lim\limits_{x\to-1}-(229+\sqrt{-x}) =-230$$
\rozwStop
\odpStart
$-230$
\odpStop
\testStart
A.$-230$
B.$-228$
C.$228$
D.$230$
E.$\infty$
F.$-\infty$
G.$0$
H.$-229$
I.$229$
\testStop
\kluczStart
A
\kluczStop



\zadStart{Zadanie z Wikieł Z 4.3 c) moja wersja nr 80}
Obliczyć granicę funkcji $f(x)=\frac{-x-233}{233-\sqrt{-x}}$.
\zadStop
\rozwStart{Patryk Wirkus}{Szymon Tokarski}
$$\frac{-x-233}{233-\sqrt{-x}}=\frac{(-x-233)(233+\sqrt{-x})}{233-(-x)}=-(233+\sqrt{-x})$$
\\
$$\lim\limits_{x\to-1}\frac{-x-233}{233-\sqrt{-x}}=[\frac{0}{0}]=\lim\limits_{x\to-1}-(233+\sqrt{-x}) =-234$$
\rozwStop
\odpStart
$-234$
\odpStop
\testStart
A.$-234$
B.$-232$
C.$232$
D.$234$
E.$\infty$
F.$-\infty$
G.$0$
H.$-233$
I.$233$
\testStop
\kluczStart
A
\kluczStop



\zadStart{Zadanie z Wikieł Z 4.3 c) moja wersja nr 81}
Obliczyć granicę funkcji $f(x)=\frac{-x-239}{239-\sqrt{-x}}$.
\zadStop
\rozwStart{Patryk Wirkus}{Szymon Tokarski}
$$\frac{-x-239}{239-\sqrt{-x}}=\frac{(-x-239)(239+\sqrt{-x})}{239-(-x)}=-(239+\sqrt{-x})$$
\\
$$\lim\limits_{x\to-1}\frac{-x-239}{239-\sqrt{-x}}=[\frac{0}{0}]=\lim\limits_{x\to-1}-(239+\sqrt{-x}) =-240$$
\rozwStop
\odpStart
$-240$
\odpStop
\testStart
A.$-240$
B.$-238$
C.$238$
D.$240$
E.$\infty$
F.$-\infty$
G.$0$
H.$-239$
I.$239$
\testStop
\kluczStart
A
\kluczStop



\zadStart{Zadanie z Wikieł Z 4.3 c) moja wersja nr 82}
Obliczyć granicę funkcji $f(x)=\frac{-x-241}{241-\sqrt{-x}}$.
\zadStop
\rozwStart{Patryk Wirkus}{Szymon Tokarski}
$$\frac{-x-241}{241-\sqrt{-x}}=\frac{(-x-241)(241+\sqrt{-x})}{241-(-x)}=-(241+\sqrt{-x})$$
\\
$$\lim\limits_{x\to-1}\frac{-x-241}{241-\sqrt{-x}}=[\frac{0}{0}]=\lim\limits_{x\to-1}-(241+\sqrt{-x}) =-242$$
\rozwStop
\odpStart
$-242$
\odpStop
\testStart
A.$-242$
B.$-240$
C.$240$
D.$242$
E.$\infty$
F.$-\infty$
G.$0$
H.$-241$
I.$241$
\testStop
\kluczStart
A
\kluczStop



\zadStart{Zadanie z Wikieł Z 4.3 c) moja wersja nr 83}
Obliczyć granicę funkcji $f(x)=\frac{-x-251}{251-\sqrt{-x}}$.
\zadStop
\rozwStart{Patryk Wirkus}{Szymon Tokarski}
$$\frac{-x-251}{251-\sqrt{-x}}=\frac{(-x-251)(251+\sqrt{-x})}{251-(-x)}=-(251+\sqrt{-x})$$
\\
$$\lim\limits_{x\to-1}\frac{-x-251}{251-\sqrt{-x}}=[\frac{0}{0}]=\lim\limits_{x\to-1}-(251+\sqrt{-x}) =-252$$
\rozwStop
\odpStart
$-252$
\odpStop
\testStart
A.$-252$
B.$-250$
C.$250$
D.$252$
E.$\infty$
F.$-\infty$
G.$0$
H.$-251$
I.$251$
\testStop
\kluczStart
A
\kluczStop



\zadStart{Zadanie z Wikieł Z 4.3 c) moja wersja nr 84}
Obliczyć granicę funkcji $f(x)=\frac{-x-257}{257-\sqrt{-x}}$.
\zadStop
\rozwStart{Patryk Wirkus}{Szymon Tokarski}
$$\frac{-x-257}{257-\sqrt{-x}}=\frac{(-x-257)(257+\sqrt{-x})}{257-(-x)}=-(257+\sqrt{-x})$$
\\
$$\lim\limits_{x\to-1}\frac{-x-257}{257-\sqrt{-x}}=[\frac{0}{0}]=\lim\limits_{x\to-1}-(257+\sqrt{-x}) =-258$$
\rozwStop
\odpStart
$-258$
\odpStop
\testStart
A.$-258$
B.$-256$
C.$256$
D.$258$
E.$\infty$
F.$-\infty$
G.$0$
H.$-257$
I.$257$
\testStop
\kluczStart
A
\kluczStop



\zadStart{Zadanie z Wikieł Z 4.3 c) moja wersja nr 85}
Obliczyć granicę funkcji $f(x)=\frac{-x-263}{263-\sqrt{-x}}$.
\zadStop
\rozwStart{Patryk Wirkus}{Szymon Tokarski}
$$\frac{-x-263}{263-\sqrt{-x}}=\frac{(-x-263)(263+\sqrt{-x})}{263-(-x)}=-(263+\sqrt{-x})$$
\\
$$\lim\limits_{x\to-1}\frac{-x-263}{263-\sqrt{-x}}=[\frac{0}{0}]=\lim\limits_{x\to-1}-(263+\sqrt{-x}) =-264$$
\rozwStop
\odpStart
$-264$
\odpStop
\testStart
A.$-264$
B.$-262$
C.$262$
D.$264$
E.$\infty$
F.$-\infty$
G.$0$
H.$-263$
I.$263$
\testStop
\kluczStart
A
\kluczStop



\zadStart{Zadanie z Wikieł Z 4.3 c) moja wersja nr 86}
Obliczyć granicę funkcji $f(x)=\frac{-x-269}{269-\sqrt{-x}}$.
\zadStop
\rozwStart{Patryk Wirkus}{Szymon Tokarski}
$$\frac{-x-269}{269-\sqrt{-x}}=\frac{(-x-269)(269+\sqrt{-x})}{269-(-x)}=-(269+\sqrt{-x})$$
\\
$$\lim\limits_{x\to-1}\frac{-x-269}{269-\sqrt{-x}}=[\frac{0}{0}]=\lim\limits_{x\to-1}-(269+\sqrt{-x}) =-270$$
\rozwStop
\odpStart
$-270$
\odpStop
\testStart
A.$-270$
B.$-268$
C.$268$
D.$270$
E.$\infty$
F.$-\infty$
G.$0$
H.$-269$
I.$269$
\testStop
\kluczStart
A
\kluczStop



\zadStart{Zadanie z Wikieł Z 4.3 c) moja wersja nr 87}
Obliczyć granicę funkcji $f(x)=\frac{-x-271}{271-\sqrt{-x}}$.
\zadStop
\rozwStart{Patryk Wirkus}{Szymon Tokarski}
$$\frac{-x-271}{271-\sqrt{-x}}=\frac{(-x-271)(271+\sqrt{-x})}{271-(-x)}=-(271+\sqrt{-x})$$
\\
$$\lim\limits_{x\to-1}\frac{-x-271}{271-\sqrt{-x}}=[\frac{0}{0}]=\lim\limits_{x\to-1}-(271+\sqrt{-x}) =-272$$
\rozwStop
\odpStart
$-272$
\odpStop
\testStart
A.$-272$
B.$-270$
C.$270$
D.$272$
E.$\infty$
F.$-\infty$
G.$0$
H.$-271$
I.$271$
\testStop
\kluczStart
A
\kluczStop



\zadStart{Zadanie z Wikieł Z 4.3 c) moja wersja nr 88}
Obliczyć granicę funkcji $f(x)=\frac{-x-22}{22-\sqrt{-x}}$.
\zadStop
\rozwStart{Patryk Wirkus}{Szymon Tokarski}
$$\frac{-x-22}{22-\sqrt{-x}}=\frac{(-x-22)(22+\sqrt{-x})}{22-(-x)}=-(22+\sqrt{-x})$$
\\
$$\lim\limits_{x\to-1}\frac{-x-22}{22-\sqrt{-x}}=[\frac{0}{0}]=\lim\limits_{x\to-1}-(22+\sqrt{-x}) =-23$$
\rozwStop
\odpStart
$-23$
\odpStop
\testStart
A.$-23$
B.$-21$
C.$21$
D.$23$
E.$\infty$
F.$-\infty$
G.$0$
H.$-22$
I.$22$
\testStop
\kluczStart
A
\kluczStop



\zadStart{Zadanie z Wikieł Z 4.3 c) moja wersja nr 89}
Obliczyć granicę funkcji $f(x)=\frac{-x-33}{33-\sqrt{-x}}$.
\zadStop
\rozwStart{Patryk Wirkus}{Szymon Tokarski}
$$\frac{-x-33}{33-\sqrt{-x}}=\frac{(-x-33)(33+\sqrt{-x})}{33-(-x)}=-(33+\sqrt{-x})$$
\\
$$\lim\limits_{x\to-1}\frac{-x-33}{33-\sqrt{-x}}=[\frac{0}{0}]=\lim\limits_{x\to-1}-(33+\sqrt{-x}) =-34$$
\rozwStop
\odpStart
$-34$
\odpStop
\testStart
A.$-34$
B.$-32$
C.$32$
D.$34$
E.$\infty$
F.$-\infty$
G.$0$
H.$-33$
I.$33$
\testStop
\kluczStart
A
\kluczStop



\zadStart{Zadanie z Wikieł Z 4.3 c) moja wersja nr 90}
Obliczyć granicę funkcji $f(x)=\frac{-x-44}{44-\sqrt{-x}}$.
\zadStop
\rozwStart{Patryk Wirkus}{Szymon Tokarski}
$$\frac{-x-44}{44-\sqrt{-x}}=\frac{(-x-44)(44+\sqrt{-x})}{44-(-x)}=-(44+\sqrt{-x})$$
\\
$$\lim\limits_{x\to-1}\frac{-x-44}{44-\sqrt{-x}}=[\frac{0}{0}]=\lim\limits_{x\to-1}-(44+\sqrt{-x}) =-45$$
\rozwStop
\odpStart
$-45$
\odpStop
\testStart
A.$-45$
B.$-43$
C.$43$
D.$45$
E.$\infty$
F.$-\infty$
G.$0$
H.$-44$
I.$44$
\testStop
\kluczStart
A
\kluczStop



\zadStart{Zadanie z Wikieł Z 4.3 c) moja wersja nr 91}
Obliczyć granicę funkcji $f(x)=\frac{-x-55}{55-\sqrt{-x}}$.
\zadStop
\rozwStart{Patryk Wirkus}{Szymon Tokarski}
$$\frac{-x-55}{55-\sqrt{-x}}=\frac{(-x-55)(55+\sqrt{-x})}{55-(-x)}=-(55+\sqrt{-x})$$
\\
$$\lim\limits_{x\to-1}\frac{-x-55}{55-\sqrt{-x}}=[\frac{0}{0}]=\lim\limits_{x\to-1}-(55+\sqrt{-x}) =-56$$
\rozwStop
\odpStart
$-56$
\odpStop
\testStart
A.$-56$
B.$-54$
C.$54$
D.$56$
E.$\infty$
F.$-\infty$
G.$0$
H.$-55$
I.$55$
\testStop
\kluczStart
A
\kluczStop



\zadStart{Zadanie z Wikieł Z 4.3 c) moja wersja nr 92}
Obliczyć granicę funkcji $f(x)=\frac{-x-66}{66-\sqrt{-x}}$.
\zadStop
\rozwStart{Patryk Wirkus}{Szymon Tokarski}
$$\frac{-x-66}{66-\sqrt{-x}}=\frac{(-x-66)(66+\sqrt{-x})}{66-(-x)}=-(66+\sqrt{-x})$$
\\
$$\lim\limits_{x\to-1}\frac{-x-66}{66-\sqrt{-x}}=[\frac{0}{0}]=\lim\limits_{x\to-1}-(66+\sqrt{-x}) =-67$$
\rozwStop
\odpStart
$-67$
\odpStop
\testStart
A.$-67$
B.$-65$
C.$65$
D.$67$
E.$\infty$
F.$-\infty$
G.$0$
H.$-66$
I.$66$
\testStop
\kluczStart
A
\kluczStop



\zadStart{Zadanie z Wikieł Z 4.3 c) moja wersja nr 93}
Obliczyć granicę funkcji $f(x)=\frac{-x-77}{77-\sqrt{-x}}$.
\zadStop
\rozwStart{Patryk Wirkus}{Szymon Tokarski}
$$\frac{-x-77}{77-\sqrt{-x}}=\frac{(-x-77)(77+\sqrt{-x})}{77-(-x)}=-(77+\sqrt{-x})$$
\\
$$\lim\limits_{x\to-1}\frac{-x-77}{77-\sqrt{-x}}=[\frac{0}{0}]=\lim\limits_{x\to-1}-(77+\sqrt{-x}) =-78$$
\rozwStop
\odpStart
$-78$
\odpStop
\testStart
A.$-78$
B.$-76$
C.$76$
D.$78$
E.$\infty$
F.$-\infty$
G.$0$
H.$-77$
I.$77$
\testStop
\kluczStart
A
\kluczStop



\zadStart{Zadanie z Wikieł Z 4.3 c) moja wersja nr 94}
Obliczyć granicę funkcji $f(x)=\frac{-x-88}{88-\sqrt{-x}}$.
\zadStop
\rozwStart{Patryk Wirkus}{Szymon Tokarski}
$$\frac{-x-88}{88-\sqrt{-x}}=\frac{(-x-88)(88+\sqrt{-x})}{88-(-x)}=-(88+\sqrt{-x})$$
\\
$$\lim\limits_{x\to-1}\frac{-x-88}{88-\sqrt{-x}}=[\frac{0}{0}]=\lim\limits_{x\to-1}-(88+\sqrt{-x}) =-89$$
\rozwStop
\odpStart
$-89$
\odpStop
\testStart
A.$-89$
B.$-87$
C.$87$
D.$89$
E.$\infty$
F.$-\infty$
G.$0$
H.$-88$
I.$88$
\testStop
\kluczStart
A
\kluczStop



\zadStart{Zadanie z Wikieł Z 4.3 c) moja wersja nr 95}
Obliczyć granicę funkcji $f(x)=\frac{-x-99}{99-\sqrt{-x}}$.
\zadStop
\rozwStart{Patryk Wirkus}{Szymon Tokarski}
$$\frac{-x-99}{99-\sqrt{-x}}=\frac{(-x-99)(99+\sqrt{-x})}{99-(-x)}=-(99+\sqrt{-x})$$
\\
$$\lim\limits_{x\to-1}\frac{-x-99}{99-\sqrt{-x}}=[\frac{0}{0}]=\lim\limits_{x\to-1}-(99+\sqrt{-x}) =-100$$
\rozwStop
\odpStart
$-100$
\odpStop
\testStart
A.$-100$
B.$-98$
C.$98$
D.$100$
E.$\infty$
F.$-\infty$
G.$0$
H.$-99$
I.$99$
\testStop
\kluczStart
A
\kluczStop



\zadStart{Zadanie z Wikieł Z 4.3 c) moja wersja nr 96}
Obliczyć granicę funkcji $f(x)=\frac{-x-999}{999-\sqrt{-x}}$.
\zadStop
\rozwStart{Patryk Wirkus}{Szymon Tokarski}
$$\frac{-x-999}{999-\sqrt{-x}}=\frac{(-x-999)(999+\sqrt{-x})}{999-(-x)}=-(999+\sqrt{-x})$$
\\
$$\lim\limits_{x\to-1}\frac{-x-999}{999-\sqrt{-x}}=[\frac{0}{0}]=\lim\limits_{x\to-1}-(999+\sqrt{-x}) =-1000$$
\rozwStop
\odpStart
$-1000$
\odpStop
\testStart
A.$-1000$
B.$-998$
C.$998$
D.$1000$
E.$\infty$
F.$-\infty$
G.$0$
H.$-999$
I.$999$
\testStop
\kluczStart
A
\kluczStop



\zadStart{Zadanie z Wikieł Z 4.3 c) moja wersja nr 97}
Obliczyć granicę funkcji $f(x)=\frac{-x-888}{888-\sqrt{-x}}$.
\zadStop
\rozwStart{Patryk Wirkus}{Szymon Tokarski}
$$\frac{-x-888}{888-\sqrt{-x}}=\frac{(-x-888)(888+\sqrt{-x})}{888-(-x)}=-(888+\sqrt{-x})$$
\\
$$\lim\limits_{x\to-1}\frac{-x-888}{888-\sqrt{-x}}=[\frac{0}{0}]=\lim\limits_{x\to-1}-(888+\sqrt{-x}) =-889$$
\rozwStop
\odpStart
$-889$
\odpStop
\testStart
A.$-889$
B.$-887$
C.$887$
D.$889$
E.$\infty$
F.$-\infty$
G.$0$
H.$-888$
I.$888$
\testStop
\kluczStart
A
\kluczStop



\zadStart{Zadanie z Wikieł Z 4.3 c) moja wersja nr 98}
Obliczyć granicę funkcji $f(x)=\frac{-x-777}{777-\sqrt{-x}}$.
\zadStop
\rozwStart{Patryk Wirkus}{Szymon Tokarski}
$$\frac{-x-777}{777-\sqrt{-x}}=\frac{(-x-777)(777+\sqrt{-x})}{777-(-x)}=-(777+\sqrt{-x})$$
\\
$$\lim\limits_{x\to-1}\frac{-x-777}{777-\sqrt{-x}}=[\frac{0}{0}]=\lim\limits_{x\to-1}-(777+\sqrt{-x}) =-778$$
\rozwStop
\odpStart
$-778$
\odpStop
\testStart
A.$-778$
B.$-776$
C.$776$
D.$778$
E.$\infty$
F.$-\infty$
G.$0$
H.$-777$
I.$777$
\testStop
\kluczStart
A
\kluczStop



\zadStart{Zadanie z Wikieł Z 4.3 c) moja wersja nr 99}
Obliczyć granicę funkcji $f(x)=\frac{-x-666}{666-\sqrt{-x}}$.
\zadStop
\rozwStart{Patryk Wirkus}{Szymon Tokarski}
$$\frac{-x-666}{666-\sqrt{-x}}=\frac{(-x-666)(666+\sqrt{-x})}{666-(-x)}=-(666+\sqrt{-x})$$
\\
$$\lim\limits_{x\to-1}\frac{-x-666}{666-\sqrt{-x}}=[\frac{0}{0}]=\lim\limits_{x\to-1}-(666+\sqrt{-x}) =-667$$
\rozwStop
\odpStart
$-667$
\odpStop
\testStart
A.$-667$
B.$-665$
C.$665$
D.$667$
E.$\infty$
F.$-\infty$
G.$0$
H.$-666$
I.$666$
\testStop
\kluczStart
A
\kluczStop



\zadStart{Zadanie z Wikieł Z 4.3 c) moja wersja nr 100}
Obliczyć granicę funkcji $f(x)=\frac{-x-555}{555-\sqrt{-x}}$.
\zadStop
\rozwStart{Patryk Wirkus}{Szymon Tokarski}
$$\frac{-x-555}{555-\sqrt{-x}}=\frac{(-x-555)(555+\sqrt{-x})}{555-(-x)}=-(555+\sqrt{-x})$$
\\
$$\lim\limits_{x\to-1}\frac{-x-555}{555-\sqrt{-x}}=[\frac{0}{0}]=\lim\limits_{x\to-1}-(555+\sqrt{-x}) =-556$$
\rozwStop
\odpStart
$-556$
\odpStop
\testStart
A.$-556$
B.$-554$
C.$554$
D.$556$
E.$\infty$
F.$-\infty$
G.$0$
H.$-555$
I.$555$
\testStop
\kluczStart
A
\kluczStop



\zadStart{Zadanie z Wikieł Z 4.3 c) moja wersja nr 101}
Obliczyć granicę funkcji $f(x)=\frac{-x-444}{444-\sqrt{-x}}$.
\zadStop
\rozwStart{Patryk Wirkus}{Szymon Tokarski}
$$\frac{-x-444}{444-\sqrt{-x}}=\frac{(-x-444)(444+\sqrt{-x})}{444-(-x)}=-(444+\sqrt{-x})$$
\\
$$\lim\limits_{x\to-1}\frac{-x-444}{444-\sqrt{-x}}=[\frac{0}{0}]=\lim\limits_{x\to-1}-(444+\sqrt{-x}) =-445$$
\rozwStop
\odpStart
$-445$
\odpStop
\testStart
A.$-445$
B.$-443$
C.$443$
D.$445$
E.$\infty$
F.$-\infty$
G.$0$
H.$-444$
I.$444$
\testStop
\kluczStart
A
\kluczStop



\zadStart{Zadanie z Wikieł Z 4.3 c) moja wersja nr 102}
Obliczyć granicę funkcji $f(x)=\frac{-x-333}{333-\sqrt{-x}}$.
\zadStop
\rozwStart{Patryk Wirkus}{Szymon Tokarski}
$$\frac{-x-333}{333-\sqrt{-x}}=\frac{(-x-333)(333+\sqrt{-x})}{333-(-x)}=-(333+\sqrt{-x})$$
\\
$$\lim\limits_{x\to-1}\frac{-x-333}{333-\sqrt{-x}}=[\frac{0}{0}]=\lim\limits_{x\to-1}-(333+\sqrt{-x}) =-334$$
\rozwStop
\odpStart
$-334$
\odpStop
\testStart
A.$-334$
B.$-332$
C.$332$
D.$334$
E.$\infty$
F.$-\infty$
G.$0$
H.$-333$
I.$333$
\testStop
\kluczStart
A
\kluczStop



\zadStart{Zadanie z Wikieł Z 4.3 c) moja wersja nr 103}
Obliczyć granicę funkcji $f(x)=\frac{-x-222}{222-\sqrt{-x}}$.
\zadStop
\rozwStart{Patryk Wirkus}{Szymon Tokarski}
$$\frac{-x-222}{222-\sqrt{-x}}=\frac{(-x-222)(222+\sqrt{-x})}{222-(-x)}=-(222+\sqrt{-x})$$
\\
$$\lim\limits_{x\to-1}\frac{-x-222}{222-\sqrt{-x}}=[\frac{0}{0}]=\lim\limits_{x\to-1}-(222+\sqrt{-x}) =-223$$
\rozwStop
\odpStart
$-223$
\odpStop
\testStart
A.$-223$
B.$-221$
C.$221$
D.$223$
E.$\infty$
F.$-\infty$
G.$0$
H.$-222$
I.$222$
\testStop
\kluczStart
A
\kluczStop



\zadStart{Zadanie z Wikieł Z 4.3 c) moja wersja nr 104}
Obliczyć granicę funkcji $f(x)=\frac{-x-111}{111-\sqrt{-x}}$.
\zadStop
\rozwStart{Patryk Wirkus}{Szymon Tokarski}
$$\frac{-x-111}{111-\sqrt{-x}}=\frac{(-x-111)(111+\sqrt{-x})}{111-(-x)}=-(111+\sqrt{-x})$$
\\
$$\lim\limits_{x\to-1}\frac{-x-111}{111-\sqrt{-x}}=[\frac{0}{0}]=\lim\limits_{x\to-1}-(111+\sqrt{-x}) =-112$$
\rozwStop
\odpStart
$-112$
\odpStop
\testStart
A.$-112$
B.$-110$
C.$110$
D.$112$
E.$\infty$
F.$-\infty$
G.$0$
H.$-111$
I.$111$
\testStop
\kluczStart
A
\kluczStop





\end{document}
