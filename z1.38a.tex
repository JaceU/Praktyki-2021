\documentclass[12pt, a4paper]{article}
\usepackage[utf8]{inputenc}
\usepackage{polski}

\usepackage{amsthm}  %pakiet do tworzenia twierdzeń itp.
\usepackage{amsmath} %pakiet do niektórych symboli matematycznych
\usepackage{amssymb} %pakiet do symboli mat., np. \nsubseteq
\usepackage{amsfonts}
\usepackage{graphicx} %obsługa plików graficznych z rozszerzeniem png, jpg
\theoremstyle{definition} %styl dla definicji
\newtheorem{zad}{} 
\title{Multizestaw zadań}
\author{Mirella Narewska}
%\date{\today}
\date{}
\newcounter{liczniksekcji}
\newcommand{\kategoria}[1]{\section{#1}} %olreślamy nazwę kateforii zadań
\newcommand{\zadStart}[1]{\begin{zad}#1\newline} %oznaczenie początku zadania
\newcommand{\zadStop}{\end{zad}}   %oznaczenie końca zadania
%Makra opcjonarne (nie muszą występować):
\newcommand{\rozwStart}[2]{\noindent \textbf{Rozwiązanie (autor #1 , recenzent #2): }\newline} %oznaczenie początku rozwiązania, opcjonarnie można wprowadzić informację o autorze rozwiązania zadania i recenzencie poprawności wykonania rozwiązania zadania
\newcommand{\rozwStop}{\newline}                                            %oznaczenie końca rozwiązania
\newcommand{\odpStart}{\noindent \textbf{Odpowiedź:}\newline}    %oznaczenie początku odpowiedzi końcowej (wypisanie wyniku)
\newcommand{\odpStop}{\newline}                                             %oznaczenie końca odpowiedzi końcowej (wypisanie wyniku)
\newcommand{\testStart}{\noindent \textbf{Test:}\newline} %ewentualne możliwe opcje odpowiedzi testowej: A. ? B. ? C. ? D. ? itd.
\newcommand{\testStop}{\newline} %koniec wprowadzania odpowiedzi testowych
\newcommand{\kluczStart}{\noindent \textbf{Test poprawna odpowiedź:}\newline} %klucz, poprawna odpowiedź pytania testowego (jedna literka): A lub B lub C lub D itd.
\newcommand{\kluczStop}{\newline} %koniec poprawnej odpowiedzi pytania testowego 
\newcommand{\wstawGrafike}[2]{\begin{figure}[h] \includegraphics[scale=#2] {#1} \end{figure}} %gdyby była potrzeba wstawienia obrazka, parametry: nazwa pliku, skala (jak nie wiesz co wpisać, to wpisz 1)

\begin{document}
\maketitle


\kategoria{Wikieł/Z1.58n}
\zadStart{Zadanie z Wikieł Z 1.58 n) moja wersja nr [nrWersji]}
%[a]:[2,3,4,5,6,7,8,9,10,11,12,13,14,15,16,17,18,19,20,21]
%[b]:[2,3,4,5,6,7,8,9,10,11,12,13,14,15,16,17,18,19,20]
%[c]:[2,3,4,5,6,7,8]
%[e]=[b]*[c]
%[k]=2*[b]*[e]
%[m]=4*[a]
%[h]=[k]-[m]
%[g]=[e]*[e]
%[z]=[b]*[b]-4
%[k] == [m] and [z] == 0
Wyznaczyć wszystkie wartości parametru m, dla których nierówność $([a]m+1)x^2-[b]([c]m+1)x+1>0$ jest prawdziwa dla każdego x $\in \mathbb{R}.$
\zadStop
\rozwStart{Mirella Narewska}{}
Aby nierówność $([a]m+1)x^2-[b]([c]m+1)x+1>0$ była prawdziwa dla każdego  x $\in \mathbb{R}$, muszą zostać spełnione 2 warunki
$$1) ([a]m+1)>0$$
$$2)  \triangle<0$$
\\
(1)
\\
$$[a]m+1>0$$
$$[a]m>-1$$
$$m>-1/[a]$$
\\
(2)
\\
$$\triangle=((-[e]m -[b])^2-4\cdot ([a]m+1)\cdot 1= [g]m^2 + [h]m +[z] \Rightarrow [g]m^2<0$$
$$[g]m^2<0$$
$$m^2<0 \Rightarrow sprzecznosc \Rightarrow m \in \emptyset$$
\odpStart
$m \in \emptyset$
\odpStop
\testStart
A. $m \in \emptyset$ \\
B. $m=1$ \\
C. $m \in (-[a];[b])$ \\
D. $m \in (0,[a])$ 
\testStop
\kluczStart
A
\kluczStop
\end{document}