\documentclass[12pt, a4paper]{article}
\usepackage[utf8]{inputenc}
\usepackage{polski}
\usepackage{amsthm}  %pakiet do tworzenia twierdzeń itp.
\usepackage{amsmath} %pakiet do niektórych symboli matematycznych
\usepackage{amssymb} %pakiet do symboli mat., np. \nsubseteq
\usepackage{amsfonts}
\usepackage{graphicx} %obsługa plików graficznych z rozszerzeniem png, jpg
\theoremstyle{definition} %styl dla definicji
\newtheorem{zad}{} 
\title{Multizestaw zadań}
\author{Robert Fidytek}
%\date{\today}
\date{}
\newcounter{liczniksekcji}
\newcommand{\kategoria}[1]{\section{#1}} %olreślamy nazwę kateforii zadań
\newcommand{\zadStart}[1]{\begin{zad}#1\newline} %oznaczenie początku zadania
\newcommand{\zadStop}{\end{zad}}   %oznaczenie końca zadania
%Makra opcjonarne (nie muszą występować):
\newcommand{\rozwStart}[2]{\noindent \textbf{Rozwiązanie (autor #1 , recenzent #2): }\newline} %oznaczenie początku rozwiązania, opcjonarnie można wprowadzić informację o autorze rozwiązania zadania i recenzencie poprawności wykonania rozwiązania zadania
\newcommand{\rozwStop}{\newline}                                            %oznaczenie końca rozwiązania
\newcommand{\odpStart}{\noindent \textbf{Odpowiedź:}\newline}    %oznaczenie początku odpowiedzi końcowej (wypisanie wyniku)
\newcommand{\odpStop}{\newline}                                             %oznaczenie końca odpowiedzi końcowej (wypisanie wyniku)
\newcommand{\testStart}{\noindent \textbf{Test:}\newline} %ewentualne możliwe opcje odpowiedzi testowej: A. ? B. ? C. ? D. ? itd.
\newcommand{\testStop}{\newline} %koniec wprowadzania odpowiedzi testowych
\newcommand{\kluczStart}{\noindent \textbf{Test poprawna odpowiedź:}\newline} %klucz, poprawna odpowiedź pytania testowego (jedna literka): A lub B lub C lub D itd.
\newcommand{\kluczStop}{\newline} %koniec poprawnej odpowiedzi pytania testowego 
\newcommand{\wstawGrafike}[2]{\begin{figure}[h] \includegraphics[scale=#2] {#1} \end{figure}} %gdyby była potrzeba wstawienia obrazka, parametry: nazwa pliku, skala (jak nie wiesz co wpisać, to wpisz 1)

\begin{document}
\maketitle

\kategoria{Wikieł/Z1.92c}
\zadStart{Zadanie z Wikieł Z1.92 c) moja wersja nr [nrWersji]}
%[z]:[6,12,18,24,30]
%[t]=int([z]/3)
%[o]=int([t]*8)
%[dz]=math.gcd(8,[z])
%[g]=int(8/[dz])
%[d]=int([z]/[dz])
Rozwiąż równania.\\
Podane równanie: $ 3 \cdot 8^{x - 1} = [z]^{x + 1} $
\zadStop
\rozwStart{Martyna Czarnobaj}{}
\begin{center}
	$ 3 \cdot 8^{x - 1} = [z]^{x + 1} $\\
	$ 3 \cdot 8^{x - 1} = [z]^{x} \cdot [z] |:3 $\\
	$ 8^{x - 1} = [z]^{x} \cdot [t] $\\
	$ \frac{8^{x}}{8} = [z]^{x} \cdot [t] | \cdot 8 $\\
	$ 8^{x} = [z]^{x} \cdot [o] |:[z]^{x} $\\
	$ (\frac{8}{[z]})^{x} = [o] $\\
	$ (\frac{[g]}{[d]})^{x} = [o] $\\
	$ x = \log_{\frac{[g]}{[d]}} [o] $\\
\end{center}
Koniec rozwiązania.\\
\rozwStop
\odpStart
 $ x = \log_{\frac{[g]}{[d]}} [o] $\\
\odpStop
\testStart
A.$ x = \log_{\frac{[g]}{[d]}} [o] $\\\
B.$ x = \log_{\frac{[g]}{[d]}} [z] $\\
C.$ x = \log_{\frac{[g]}{[d]}} [g] $\\\\
\testStop
\kluczStart
A
\kluczStop



\end{document}