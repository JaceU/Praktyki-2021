\documentclass[12pt, a4paper]{article}
\usepackage[utf8]{inputenc}
\usepackage{polski}

\usepackage{amsthm}  %pakiet do tworzenia twierdzeń itp.
\usepackage{amsmath} %pakiet do niektórych symboli matematycznych
\usepackage{amssymb} %pakiet do symboli mat., np. \nsubseteq
\usepackage{amsfonts}
\usepackage{graphicx} %obsługa plików graficznych z rozszerzeniem png, jpg
\theoremstyle{definition} %styl dla definicji
\newtheorem{zad}{} 
\title{Multizestaw zadań}
\author{Robert Fidytek}
%\date{\today}
\date{}
\newcounter{liczniksekcji}
\newcommand{\kategoria}[1]{\section{#1}} %olreślamy nazwę kateforii zadań
\newcommand{\zadStart}[1]{\begin{zad}#1\newline} %oznaczenie początku zadania
\newcommand{\zadStop}{\end{zad}}   %oznaczenie końca zadania
%Makra opcjonarne (nie muszą występować):
\newcommand{\rozwStart}[2]{\noindent \textbf{Rozwiązanie (autor #1 , recenzent #2): }\newline} %oznaczenie początku rozwiązania, opcjonarnie można wprowadzić informację o autorze rozwiązania zadania i recenzencie poprawności wykonania rozwiązania zadania
\newcommand{\rozwStop}{\newline}                                            %oznaczenie końca rozwiązania
\newcommand{\odpStart}{\noindent \textbf{Odpowiedź:}\newline}    %oznaczenie początku odpowiedzi końcowej (wypisanie wyniku)
\newcommand{\odpStop}{\newline}                                             %oznaczenie końca odpowiedzi końcowej (wypisanie wyniku)
\newcommand{\testStart}{\noindent \textbf{Test:}\newline} %ewentualne możliwe opcje odpowiedzi testowej: A. ? B. ? C. ? D. ? itd.
\newcommand{\testStop}{\newline} %koniec wprowadzania odpowiedzi testowych
\newcommand{\kluczStart}{\noindent \textbf{Test poprawna odpowiedź:}\newline} %klucz, poprawna odpowiedź pytania testowego (jedna literka): A lub B lub C lub D itd.
\newcommand{\kluczStop}{\newline} %koniec poprawnej odpowiedzi pytania testowego 
\newcommand{\wstawGrafike}[2]{\begin{figure}[h] \includegraphics[scale=#2] {#1} \end{figure}} %gdyby była potrzeba wstawienia obrazka, parametry: nazwa pliku, skala (jak nie wiesz co wpisać, to wpisz 1)

\begin{document}
\maketitle


\kategoria{Wikieł/Z3.13e}
\zadStart{Zadanie z Wikieł Z 3.13 e) moja wersja nr [nrWersji]}
%[a]:[2,3,4,5,6,7,8,9]
%[b]:[9,16,25,36,49,64,81,100]
%[c]:[4,9,16,25,36,49,64,81]
%[a1]=[a]+1
%[a2]=[a]+2
%[cs]=int(math.sqrt([c]))
%[bs]=int(math.sqrt([b]))
%[calosci]=[bs]//[cs]
%[reszta]=[bs]%[cs]
%math.gcd([cs],[bs])==1 and [cs]<[bs]
Obliczyć granicę ciągu 
$$a_n=\frac{\sqrt{[b]n+[a2]}-\sqrt{[b]n+[a1]}}{\sqrt{[c]n+[a1]}-\sqrt{[c]n+[a]}}.$$
\zadStop
\rozwStart{Adrianna Stobiecka}{}
$$\lim_{n\to\infty}\frac{\sqrt{[b]n+[a2]}-\sqrt{[b]n+[a1]}}{\sqrt{[c]n+[a1]}-\sqrt{[c]n+[a]}}$$
$$=\lim_{n\to\infty}\frac{(\sqrt{[b]n+[a2]}-\sqrt{[b]n+[a1]})(\sqrt{[b]n+[a2]}+\sqrt{[b]n+[a1]})}{\sqrt{[b]n+[a2]}+\sqrt{[b]n+[a1]}}$$
$$\cdot\frac{\sqrt{[c]n+[a1]}+\sqrt{[c]n+[a]}}{(\sqrt{[c]n+[a1]}-\sqrt{[c]n+[a]})(\sqrt{[c]n+[a1]}+\sqrt{[c]n+[a]})}$$
$$=\lim_{n\to\infty}\frac{[b]n+[a2]-[b]n-[a1]}{\sqrt{[b]n+[a2]}+\sqrt{[b]n+[a1]}}\cdot\frac{\sqrt{[c]n+[a1]}+\sqrt{[c]n+[a]}}{[c]n+[a1]-[c]n-[a]}$$
$$=\lim_{n\to\infty}\frac{1}{\sqrt{[b]n+[a2]}+\sqrt{[b]n+[a1]}}\cdot\frac{\sqrt{[c]n+[a1]}+\sqrt{[c]n+[a]}}{1}$$
$$=\lim_{n\to\infty}\frac{\sqrt{[c]n+[a1]}+\sqrt{[c]n+[a]}}{\sqrt{[b]n+[a2]}+\sqrt{[b]n+[a1]}}=\lim_{n\to\infty}\frac{\sqrt{n([c]+\frac{[a1]}{n})}+\sqrt{n([c]+\frac{[a]}{n})}}{\sqrt{n([b]+\frac{[a2]}{n})}+\sqrt{n([b]+\frac{[a1]}{n})}}$$
$$=\lim_{n\to\infty}\frac{\sqrt{[c]+\frac{[a1]}{n}}+\sqrt{[c]+\frac{[a]}{n}}}{\sqrt{[b]+\frac{[a2]}{n}}+\sqrt{[b]+\frac{[a1]}{n}}}=\frac{\sqrt{[c]}+\sqrt{[c]}}{\sqrt{[b]}+\sqrt{[b]}}=\frac{[cs]}{[bs]}$$
\rozwStop
\odpStart
$\frac{[cs]}{[bs]}$
\odpStop
\testStart
A.$\infty$
B.$-\infty$
C.$[bs]$
D.$[calosci]\frac{[reszta]}{[cs]}$
E.$0$
F.$-\frac{[cs]}{[bs]}$
G.$-[cs]$
H.$\frac{[cs]}{[bs]}$
I.$-[calosci]\frac{[reszta]}{[cs]}$
\testStop
\kluczStart
H
\kluczStop



\end{document}
