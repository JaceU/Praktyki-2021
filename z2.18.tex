\documentclass[12pt, a4paper]{article}
\usepackage[utf8]{inputenc}
\usepackage{polski}

\usepackage{amsthm}  %pakiet do tworzenia twierdzeń itp.
\usepackage{amsmath} %pakiet do niektórych symboli matematycznych
\usepackage{amssymb} %pakiet do symboli mat., np. \nsubseteq
\usepackage{amsfonts}
\usepackage{graphicx} %obsługa plików graficznych z rozszerzeniem png, jpg
\theoremstyle{definition} %styl dla definicji
\newtheorem{zad}{} 
\title{Multizestaw zadań}
\author{Robert Fidytek}
%\date{\today}
\date{}
\newcounter{liczniksekcji}
\newcommand{\kategoria}[1]{\section{#1}} %olreślamy nazwę kateforii zadań
\newcommand{\zadStart}[1]{\begin{zad}#1\newline} %oznaczenie początku zadania
\newcommand{\zadStop}{\end{zad}}   %oznaczenie końca zadania
%Makra opcjonarne (nie muszą występować):
\newcommand{\rozwStart}[2]{\noindent \textbf{Rozwiązanie (autor #1 , recenzent #2): }\newline} %oznaczenie początku rozwiązania, opcjonarnie można wprowadzić informację o autorze rozwiązania zadania i recenzencie poprawności wykonania rozwiązania zadania
\newcommand{\rozwStop}{\newline}                                            %oznaczenie końca rozwiązania
\newcommand{\odpStart}{\noindent \textbf{Odpowiedź:}\newline}    %oznaczenie początku odpowiedzi końcowej (wypisanie wyniku)
\newcommand{\odpStop}{\newline}                                             %oznaczenie końca odpowiedzi końcowej (wypisanie wyniku)
\newcommand{\testStart}{\noindent \textbf{Test:}\newline} %ewentualne możliwe opcje odpowiedzi testowej: A. ? B. ? C. ? D. ? itd.
\newcommand{\testStop}{\newline} %koniec wprowadzania odpowiedzi testowych
\newcommand{\kluczStart}{\noindent \textbf{Test poprawna odpowiedź:}\newline} %klucz, poprawna odpowiedź pytania testowego (jedna literka): A lub B lub C lub D itd.
\newcommand{\kluczStop}{\newline} %koniec poprawnej odpowiedzi pytania testowego 
\newcommand{\wstawGrafike}[2]{\begin{figure}[h] \includegraphics[scale=#2] {#1} \end{figure}} %gdyby była potrzeba wstawienia obrazka, parametry: nazwa pliku, skala (jak nie wiesz co wpisać, to wpisz 1)

\begin{document}
\maketitle


\kategoria{Wikieł/Z2.18}
\zadStart{Zadanie z Wikieł Z 2.18 moja wersja nr [nrWersji]}
%[a1]:[-1,-2,-3,-4,-5,-6]
%[a2]:[1,2,3,4,5]
%[b1]:[1,2,3]
%[b2]:[1,4]
%[c1]:[1,2,3,8]
%[c2]:[-3,-4,-5,-6,-7,-8]
%[a]:[3,4,5]
%[b]:[3,4,5]
%[c]:[1,2,3,4,5,6,7,8,9,10]
%[licznik]=([a]*[c1]+[b]*[c2]-[c])
%[wb]=abs([licznik])
%[d]=(([a]**2)+([b]**2))
%[pr2]=(pow([d],(1/2)))
%[pr1]=[pr2].real
%[mian]=int([pr1])
%[pr2].is_integer()==True and (([a]*[a1])+([b]*[a2])-[c])==0 and (([a]*[b1])+([b]*[b2])-[c])==0 and (([a]*[c1])+([b]*[c2])-[c])!=0 and math.gcd([wb],[mian])==1
Dane są punkty A([a1],[a2]), B([b1],[b2]), C([c1],[c2]). Jeden z tych punktów nie leży na prostej $k: [a]x+[b]y-[c]=0$. \\Obliczyć jego odległość od prostej $k$.
\zadStop
\rozwStart{Małgorzata Ugowska}{}
Sprawdzamy, który punkt nie należy do prostej $k$:
$$[a] \cdot [a1]+[b] \cdot [a2]-[c]=0 \quad \Longrightarrow \quad A \in k$$
$$[a] \cdot [b1]+[b] \cdot [b2]-[c]=0 \quad \Longrightarrow \quad B \in k$$
$$[a] \cdot [c1]+[b] \cdot ([c2])-[c] \ne 0 \quad \Longrightarrow \quad C \notin k$$
Obliczamy odległo\'sc punktu C od prostej $k$ zgodnie ze wzorem:
$$d_{C,k}=\frac{|A x_0+ B y_0 +C|}{\sqrt{A^2+B^2}} = \frac{|[a] \cdot [c1]+ [b] \cdot [c2] - [c]|}{\sqrt{[a]^2+[b]^2}} = \frac{|[licznik]|}{\sqrt{[d]}} = \frac{[wb]}{[mian]}$$
\rozwStop
\odpStart
Odległo\'sć jest równa $\frac{[wb]}{[mian]}$.
\odpStop
\testStart
A. $\frac{[wb]}{[mian]}$\\
B. $\sqrt{13}$\\
C. $\frac{21}{2}$\\
D. $\frac{-[wb]}{[mian]}$\\
E. $3$
\testStop
\kluczStart
A
\kluczStop



\end{document}