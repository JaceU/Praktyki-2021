\documentclass[12pt, a4paper]{article}
\usepackage[utf8]{inputenc}
\usepackage{polski}

\usepackage{amsthm}  %pakiet do tworzenia twierdzeń itp.
\usepackage{amsmath} %pakiet do niektórych symboli matematycznych
\usepackage{amssymb} %pakiet do symboli mat., np. \nsubseteq
\usepackage{amsfonts}
\usepackage{graphicx} %obsługa plików graficznych z rozszerzeniem png, jpg
\theoremstyle{definition} %styl dla definicji
\newtheorem{zad}{} 
\title{Multizestaw zadań}
\author{Robert Fidytek}
%\date{\today}
\date{}
\newcounter{liczniksekcji}
\newcommand{\kategoria}[1]{\section{#1}} %olreślamy nazwę kateforii zadań
\newcommand{\zadStart}[1]{\begin{zad}#1\newline} %oznaczenie początku zadania
\newcommand{\zadStop}{\end{zad}}   %oznaczenie końca zadania
%Makra opcjonarne (nie muszą występować):
\newcommand{\rozwStart}[2]{\noindent \textbf{Rozwiązanie (autor #1 , recenzent #2): }\newline} %oznaczenie początku rozwiązania, opcjonarnie można wprowadzić informację o autorze rozwiązania zadania i recenzencie poprawności wykonania rozwiązania zadania
\newcommand{\rozwStop}{\newline}                                            %oznaczenie końca rozwiązania
\newcommand{\odpStart}{\noindent \textbf{Odpowiedź:}\newline}    %oznaczenie początku odpowiedzi końcowej (wypisanie wyniku)
\newcommand{\odpStop}{\newline}                                             %oznaczenie końca odpowiedzi końcowej (wypisanie wyniku)
\newcommand{\testStart}{\noindent \textbf{Test:}\newline} %ewentualne możliwe opcje odpowiedzi testowej: A. ? B. ? C. ? D. ? itd.
\newcommand{\testStop}{\newline} %koniec wprowadzania odpowiedzi testowych
\newcommand{\kluczStart}{\noindent \textbf{Test poprawna odpowiedź:}\newline} %klucz, poprawna odpowiedź pytania testowego (jedna literka): A lub B lub C lub D itd.
\newcommand{\kluczStop}{\newline} %koniec poprawnej odpowiedzi pytania testowego 
\newcommand{\wstawGrafike}[2]{\begin{figure}[h] \includegraphics[scale=#2] {#1} \end{figure}} %gdyby była potrzeba wstawienia obrazka, parametry: nazwa pliku, skala (jak nie wiesz co wpisać, to wpisz 1)

\begin{document}
\maketitle


\kategoria{Wikieł/Z1.46}
\zadStart{Zadanie z Wikieł Z 1.46 moja wersja nr [nrWersji]}
%[p1]:[2,3,4,5,6,7,8,9]
%[p2]:[2,3,4,5,6,7,8,9]
%[p3]=random.randint(2,10)
%[2p2]=2*[p2]
%[kp2]=[p2]*[p2]
%[4p1]=4*[p1]
%[4p1p3]=4*[p1]*[p3]
%[m4p1]=1-[4p1]
%[r]=[kp2]+[4p1p3]
%[del]=[2p2]*[2p2]-4*[m4p1]*[r]
%[pdel]=round(math.sqrt(abs([del])),2)
%[2a]=2*[m4p1]
%[m1]=round((-[2p2]-[pdel])/[2a],2)
%[m2]=round((-[2p2]+[pdel])/[2a],2)
%[2p1]=2*[p1]
%[2p1p2]=[2p1]-[p2]
%[m2p21]=-1-[2p2]
%[r2]=[2p1p2]-[r]
%[2p21]=-1+[2p2]
%[r3]=[2p1p2]+[r]
%[dm4p1]=-[m4p1]
%[del2]=[m2p21]*[m2p21]-4*[dm4p1]*[r2]
%[pdel2]=round(math.sqrt(abs([del2])),2)
%[del3]=[2p21]*[2p21]-4*[m4p1]*[r3]
%[pdel3]=round(math.sqrt(abs([del3])),2)
%[2dm4p1]=2*[dm4p1]
%[2m4p1]=2*[m4p1]
%[m11]=round((-[m2p21]-[pdel2])/[2dm4p1],2)
%[m12]=round((-[m2p21]+[pdel2])/[2dm4p1],2)
%[m21]=round((-[2p21]-[pdel3])/[2m4p1],2)
%[m22]=round((-[2p21]+[pdel3])/[2m4p1],2)
%[del]>0 and [del2]>0 and [del3]>0 and [m22]<[m11] and [m11]>[m2] and [m21]<[m1]

Wyznaczyć wartość parametru $m$, dla których pierwiastki rzeczywiste $x_{1}, x_{2}$ równania $[p1]x^{2}-(m+[p2])x+m^{2}-[p3]=0$ spełniają warunek $x_{1}<1<x_{2}.$
\zadStop

\rozwStart{Maja Szabłowska}{}
Aby równanie posiadało dwa różne pierwiastki powinien być spełniony warunek $\Delta>0.$
$$\Delta=(m+[p2])^{2}-4\cdot[p1]\cdot(m^{2}-[p3])=m^{2}+[2p2]m+[kp2]-[4p1]m^{2}+[4p1p3]$$
$$=[m4p1]m^{2}+[2p2]m+[r]>0$$

$$[m4p1]m^{2}+[2p2]m+[r]>0$$
$$\Delta=[2p2]^{2}-4\cdot([m4p1])\cdot[r]=[del] \Rightarrow \sqrt{\Delta}=[pdel]$$
$$m_{1}=\frac{-[2p2]-[pdel]}{[2a]}=[m1], \quad m_{2}=\frac{-[2p2]+[pdel]}{[2a]}=[m2]$$
Zatem $m\in([m2], [m1])$

Przejdźmy do warunku:
$$x_{1}<1<x_{2}$$
$$\frac{m+[p2]-\sqrt{[m4p1]m^{2}+[2p2]m+[r]}}{[2p1]}<1<\frac{m+[p2]+\sqrt{[m4p1]m^{2}+[2p2]m+[r]}}{[2p1]}$$
$$m+[p2]-\sqrt{[m4p1]m^{2}+[2p2]m+[r]}<[2p1]<m+[p2]+\sqrt{[m4p1]m^{2}+[2p2]m+[r]}$$
$$-\sqrt{[m4p1]m^{2}+[2p2]m+[r]}<[2p1]-m-[p2]<\sqrt{[m4p1]m^{2}+[2p2]m+[r]}$$
$$-\sqrt{[m4p1]m^{2}+[2p2]m+[r]}<[2p1p2]-m<\sqrt{[m4p1]m^{2}+[2p2]m+[r]}$$

$$([2p1p2]-m-([m4p1]m^{2}+[2p2]m+[r]))([2p1p2]-m+[m4p1]m^{2}+[2p2]m+[r])<0$$
$$([dm4p1]m^{2}+[m2p21]m+[r2])([m4p1]m^{2}+[2p21]m+[r3])<0$$
$$\Delta_{1}=([m2p21])^{2}-4\cdot[dm4p1]\cdot([r2])=[del2], \Rightarrow \sqrt{\Delta_{1}}=[pdel2]$$
$$m_{1}=\frac{-([m2p21])-[pdel2]}{[2dm4p1]}=[m11], \quad m_{2}=\frac{-([m2p21])+[pdel2]}{[2dm4p1]}=[m12]$$

$$\Delta_{2}=[2p21]^{2}-4\cdot([m4p1])\cdot[r3]=[del3], \Rightarrow \sqrt{\Delta_{2}}=[pdel3]$$
$$m_{1}=\frac{-[2p21]-[pdel3]}{[2m4p1]}=[m21], \quad m_{2}=\frac{-[2p21]+[pdel3]}{[2m4p1]}=[m22]$$
Zatem $m\in(-\infty,[m22])\cup([m11],[m21])\cup([m12],\infty).$

Odpowiedzią jest część wspólna wyznaczonych przedziałów:
$$m\in([m11],[m21]).$$
\rozwStop


\odpStart
$m\in([m11],[m21])$
\odpStop
\testStart
A.$m\in([m11],[m21])$
B.$m\in([m2],[m1])$
C.$m\in(-\infty,[m22])$
D.$m\in([m21],[m12])$
E.$m\in([m22],[m12])$
F.$m\in[[m1],\infty)$
\testStop
\kluczStart
A
\kluczStop



\end{document}
