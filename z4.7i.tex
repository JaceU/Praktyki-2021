\documentclass[12pt, a4paper]{article}
\usepackage[utf8]{inputenc}
\usepackage{polski}

\usepackage{amsthm}  %pakiet do tworzenia twierdzeń itp.
\usepackage{amsmath} %pakiet do niektórych symboli matematycznych
\usepackage{amssymb} %pakiet do symboli mat., np. \nsubseteq
\usepackage{amsfonts}
\usepackage{graphicx} %obsługa plików graficznych z rozszerzeniem png, jpg
\theoremstyle{definition} %styl dla definicji
\newtheorem{zad}{} 
\title{Multizestaw zadań}
\author{Robert Fidytek}
%\date{\today}
\date{}
\newcounter{liczniksekcji}
\newcommand{\kategoria}[1]{\section{#1}} %olreślamy nazwę kateforii zadań
\newcommand{\zadStart}[1]{\begin{zad}#1\newline} %oznaczenie początku zadania
\newcommand{\zadStop}{\end{zad}}   %oznaczenie końca zadania
%Makra opcjonarne (nie muszą występować):
\newcommand{\rozwStart}[2]{\noindent \textbf{Rozwiązanie (autor #1 , recenzent #2): }\newline} %oznaczenie początku rozwiązania, opcjonarnie można wprowadzić informację o autorze rozwiązania zadania i recenzencie poprawności wykonania rozwiązania zadania
\newcommand{\rozwStop}{\newline}                                            %oznaczenie końca rozwiązania
\newcommand{\odpStart}{\noindent \textbf{Odpowiedź:}\newline}    %oznaczenie początku odpowiedzi końcowej (wypisanie wyniku)
\newcommand{\odpStop}{\newline}                                             %oznaczenie końca odpowiedzi końcowej (wypisanie wyniku)
\newcommand{\testStart}{\noindent \textbf{Test:}\newline} %ewentualne możliwe opcje odpowiedzi testowej: A. ? B. ? C. ? D. ? itd.
\newcommand{\testStop}{\newline} %koniec wprowadzania odpowiedzi testowych
\newcommand{\kluczStart}{\noindent \textbf{Test poprawna odpowiedź:}\newline} %klucz, poprawna odpowiedź pytania testowego (jedna literka): A lub B lub C lub D itd.
\newcommand{\kluczStop}{\newline} %koniec poprawnej odpowiedzi pytania testowego 
\newcommand{\wstawGrafike}[2]{\begin{figure}[h] \includegraphics[scale=#2] {#1} \end{figure}} %gdyby była potrzeba wstawienia obrazka, parametry: nazwa pliku, skala (jak nie wiesz co wpisać, to wpisz 1)

\begin{document}
\maketitle


\kategoria{Wikieł/Z4.7i}
\zadStart{Zadanie z Wikieł Z 4.7 i) moja wersja nr [nrWersji]}
%[a]:[2,3,4,5]
%[b]:[2,3,4,5]
%[a]=random.randint(2,5)
%[b]=random.randint(1,5)
%[2b]=2*[b]
Obliczyć granice funkcji $\displaystyle{\lim_{x \to \infty}}\big(\sqrt{e^{[a]x}+[b]}-\sqrt{e^{[a]x}-[b]}\big)$
\zadStop
\rozwStart{Pascal Nawrocki}{Jakub Ulrych}
Korzystamy ze sprzężenia:
$$\displaystyle{\lim_{x \to \infty}}\bigg(\big(\sqrt{e^{[a]x}+[b]}-\sqrt{e^{[a]x}-[b]}\big)\cdot\frac{\sqrt{e^{[a]x}+[b]}+\sqrt{e^{[a]x}-[b]}}{\sqrt{e^{[a]x}+[b]}+\sqrt{e^{[a]x}-[b]}}\bigg)$$
Stosujemy wzór skróconego mnożenia $(a-b)(a+b)=a^2-b^2$:
$$\displaystyle{\lim_{x \to \infty}}\big(\frac{e^{[a]x}+[b]-e^{[a]x}+[b]}{\sqrt{e^{[a]x}+[b]}+\sqrt{e^{[a]x}-[b]}}\big)$$
$$\displaystyle{\lim_{x \to \infty}}\big(\frac{[2b]}{\sqrt{e^{[a]x}+[b]}+\sqrt{e^{[a]x}-[b]}}\big)$$
W mianowniku mamy nieskończoności, a więc:
$$\frac{[2b]}{\infty}=0$$
\rozwStop
\odpStart
$0$
\odpStop
\testStart
A.$0$
B.$\infty$
C.$-\infty$
D.$1$
\testStop
\kluczStart
A
\kluczStop
\end{document}