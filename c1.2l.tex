\documentclass[12pt, a4paper]{article}
\usepackage[utf8]{inputenc}
\usepackage{polski}

\usepackage{amsthm}  %pakiet do tworzenia twierdzeń itp.
\usepackage{amsmath} %pakiet do niektórych symboli matematycznych
\usepackage{amssymb} %pakiet do symboli mat., np. \nsubseteq
\usepackage{amsfonts}
\usepackage{graphicx} %obsługa plików graficznych z rozszerzeniem png, jpg
\theoremstyle{definition} %styl dla definicji
\newtheorem{zad}{} 
\title{Multizestaw zadań}
\author{Robert Fidytek}
%\date{\today}
\date{}
\newcounter{liczniksekcji}
\newcommand{\kategoria}[1]{\section{#1}} %olreślamy nazwę kateforii zadań
\newcommand{\zadStart}[1]{\begin{zad}#1\newline} %oznaczenie początku zadania
\newcommand{\zadStop}{\end{zad}}   %oznaczenie końca zadania
%Makra opcjonarne (nie muszą występować):
\newcommand{\rozwStart}[2]{\noindent \textbf{Rozwiązanie (autor #1 , recenzent #2): }\newline} %oznaczenie początku rozwiązania, opcjonarnie można wprowadzić informację o autorze rozwiązania zadania i recenzencie poprawności wykonania rozwiązania zadania
\newcommand{\rozwStop}{\newline}                                            %oznaczenie końca rozwiązania
\newcommand{\odpStart}{\noindent \textbf{Odpowiedź:}\newline}    %oznaczenie początku odpowiedzi końcowej (wypisanie wyniku)
\newcommand{\odpStop}{\newline}                                             %oznaczenie końca odpowiedzi końcowej (wypisanie wyniku)
\newcommand{\testStart}{\noindent \textbf{Test:}\newline} %ewentualne możliwe opcje odpowiedzi testowej: A. ? B. ? C. ? D. ? itd.
\newcommand{\testStop}{\newline} %koniec wprowadzania odpowiedzi testowych
\newcommand{\kluczStart}{\noindent \textbf{Test poprawna odpowiedź:}\newline} %klucz, poprawna odpowiedź pytania testowego (jedna literka): A lub B lub C lub D itd.
\newcommand{\kluczStop}{\newline} %koniec poprawnej odpowiedzi pytania testowego 
\newcommand{\wstawGrafike}[2]{\begin{figure}[h] \includegraphics[scale=#2] {#1} \end{figure}} %gdyby była potrzeba wstawienia obrazka, parametry: nazwa pliku, skala (jak nie wiesz co wpisać, to wpisz 1)

\begin{document}
\maketitle
\kategoria{Wikieł/C1.2l}
\zadStart{Zadanie z Wikieł C 1.2l moja wersja nr [nrWersji]}
%[a]:[2,3,4,5,6,7,8,9,10,11,12,13,14,15,16,17,18,19,20]
Oblicz całkę $$\int \frac{[a]-sin^{3}(x)}{sin^{2}(x)} dx.$$
\zadStop
\rozwStart{Justyna Chojecka}{}
$$\int \frac{[a]-sin^{3}(x)}{sin^{2}(x)} dx=\int \left(\frac{[a]}{sin^{2}(x)}-\frac{sin^{3}(x)}{sin^{2}(x)}\right)dx$$$$=\int \frac{[a]}{sin^{2}(x)}dx-\int\frac{sin^{3}(x)}{sin^{2}(x)}dx=[a]\int \frac{1}{sin^{2}(x)}dx-\int sin(x)dx$$$$=[a]\left(-ctg(x)\right)-\left(-cos(x)\right)+C=-[a]ctg(x)+cos(x)+C$$
\rozwStop
\odpStart
$-[a]ctg(x)+cos(x)+C$
\odpStop
\testStart
A.$-[a]ctg(x)+cos(x)+C$\\
B.$-ctg(x)+[a]cos(x)+C$\\
C.$ctg(x)-[a]cos(x)+C$\\
D.$[a]ctg(x)-cos(x)+C$\\
E.$-ctg(x)+cos(x)+C$\\
F.$[a]ctg(x)+cos(x)+C$\\
G.$ctg(x)+[a]cos(x)+C$\\
H.$-ctg(x)-[a]cos(x)+C$\\
I.$-[a]ctg(x)-cos(x)+C$\\
\testStop
\kluczStart
A
\kluczStop



\end{document}