\documentclass[12pt, a4paper]{article}
\usepackage[utf8]{inputenc}
\usepackage{polski}

\usepackage{amsthm}  %pakiet do tworzenia twierdzeń itp.
\usepackage{amsmath} %pakiet do niektórych symboli matematycznych
\usepackage{amssymb} %pakiet do symboli mat., np. \nsubseteq
\usepackage{amsfonts}
\usepackage{graphicx} %obsługa plików graficznych z rozszerzeniem png, jpg
\theoremstyle{definition} %styl dla definicji
\newtheorem{zad}{} 
\title{Multizestaw zadań}
\author{Robert Fidytek}
%\date{\today}
\date{}
\newcounter{liczniksekcji}
\newcommand{\kategoria}[1]{\section{#1}} %olreślamy nazwę kateforii zadań
\newcommand{\zadStart}[1]{\begin{zad}#1\newline} %oznaczenie początku zadania
\newcommand{\zadStop}{\end{zad}}   %oznaczenie końca zadania
%Makra opcjonarne (nie muszą występować):
\newcommand{\rozwStart}[2]{\noindent \textbf{Rozwiązanie (autor #1 , recenzent #2): }\newline} %oznaczenie początku rozwiązania, opcjonarnie można wprowadzić informację o autorze rozwiązania zadania i recenzencie poprawności wykonania rozwiązania zadania
\newcommand{\rozwStop}{\newline}                                            %oznaczenie końca rozwiązania
\newcommand{\odpStart}{\noindent \textbf{Odpowiedź:}\newline}    %oznaczenie początku odpowiedzi końcowej (wypisanie wyniku)
\newcommand{\odpStop}{\newline}                                             %oznaczenie końca odpowiedzi końcowej (wypisanie wyniku)
\newcommand{\testStart}{\noindent \textbf{Test:}\newline} %ewentualne możliwe opcje odpowiedzi testowej: A. ? B. ? C. ? D. ? itd.
\newcommand{\testStop}{\newline} %koniec wprowadzania odpowiedzi testowych
\newcommand{\kluczStart}{\noindent \textbf{Test poprawna odpowiedź:}\newline} %klucz, poprawna odpowiedź pytania testowego (jedna literka): A lub B lub C lub D itd.
\newcommand{\kluczStop}{\newline} %koniec poprawnej odpowiedzi pytania testowego 
\newcommand{\wstawGrafike}[2]{\begin{figure}[h] \centering \includegraphics[scale=#2] {#1} \end{figure}} %gdyby była potrzeba wstawienia obrazka, parametry: nazwa pliku, skala (jak nie wiesz co wpisać, to wpisz 1)

\begin{document}
\maketitle

\kategoria{Wikieł/Z5.43g}

\zadStart{Zadanie z Wikieł Z 5.43 g) moja wersja nr [nrWersji]}
%[a]:[1,2,3,4,5,6,7,8,9,10,11]
%[b]:[2,3,4,5,6,7,8,9,10,11]
%[c]=2*[a]
%[d]=math.gcd([c],[b])
%[e]=int([c]/[d])
%[f]=int([b]/[d])
%math.gcd([a],[b])==1 and [f]!=1
Obliczyć i przedstawić w najprostszej postaci pochodną funkcji f.
$$f(x) = \frac{[a]}{[b]}x\sqrt{1 - x^2} + \frac{[a]}{[b]}\arcsin x$$
\zadStop

\rozwStart{Natalia Danieluk}{}
Dziedzina: $\quad \mathcal{D}_f=\langle -1, 1 \rangle$
$$f'(x) = \Big(\frac{[a]}{[b]}x'\sqrt{1 - x^2} + (\sqrt{1 - x^2})'\frac{[a]}{[b]}x\Big) + \Big(\frac{[a]}{[b]}\arcsin x\Big)' 
\mathrel{\stackrel{\makebox[0pt]{\mbox{\normalfont\scriptsize\textbf{(*)}}}}{=}}$$ 
$$= \frac{[a]}{[b]}\sqrt{1 - x^2} + \Big(\frac{1}{2} \cdot \frac{1}{\sqrt{1 - x^2}} \cdot (-2x) \cdot \frac{[a]}{[b]}x\Big) + \frac{[a]}{[b]}\cdot \frac{1}{\sqrt{1 - x^2}} =$$
$$= \frac{\frac{[a]}{[b]}(1 - x^2) - \frac{[a]}{[b]}x^2 + \frac{[a]}{[b]}}{\sqrt{1 - x^2}} = \frac{\frac{[a]}{[b]} - \frac{[a]}{[b]}x^2 - \frac{[a]}{[b]}x^2 + \frac{[a]}{[b]}}{\sqrt{1 - x^2}} =$$
$$= \frac{\frac{[c]}{[b]}(1 - x^2)}{\sqrt{1 - x^2}} = \frac{\frac{[e]}{[f]}(1 - x^2)\sqrt{1 - x^2}}{1 - x^2} = \frac{[e]}{[f]}\sqrt{1 - x^2}$$
{\normalfont\scriptsize\textbf{(*)}\\
$y = \arcsin x \Leftrightarrow x = \sin y$\\
$(\arcsin x)' = \frac{1}{(\sin y)'} = \frac{1}{\cos y} = \frac{1}{\sqrt{1 - \sin^2 y}} = \frac{1}{\sqrt{1 - \sin^2 (\arcsin x)}} = \frac{1}{\sqrt{1 - x^2}}$}
\rozwStop

\odpStart
$f'(x) = \frac{[e]}{[f]}\sqrt{1 - x^2}$
\odpStop

\testStart
A. $f'(x) = \frac{[a]}{[b]}x\sqrt{1 - x^2} + \frac{[a]}{[b]}\arcsin x$\\
B. $f'(x) = \frac{[a]}{[b]}\sqrt{1 - x^2} + \Big(\frac{1}{2} \cdot \frac{1}{\sqrt{1 - x^2}} \cdot (-2x) \cdot \frac{[a]}{[b]}x\Big) + \frac{[a]}{[b]}\cdot \frac{1}{\sqrt{1 - x^2}}$\\
C. $f'(x) = \frac{[a]}{[b]}\sqrt{1 - x^2}$\\
D. $f'(x) = \frac{[e]}{[f]}\sqrt{1 - x^2}$
\testStop

\kluczStart
D
\kluczStop

\end{document}