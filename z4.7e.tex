\documentclass[12pt, a4paper]{article}
\usepackage[utf8]{inputenc}
\usepackage{polski}

\usepackage{amsthm}  %pakiet do tworzenia twierdzeń itp.
\usepackage{amsmath} %pakiet do niektórych symboli matematycznych
\usepackage{amssymb} %pakiet do symboli mat., np. \nsubseteq
\usepackage{amsfonts}
\usepackage{graphicx} %obsługa plików graficznych z rozszerzeniem png, jpg
\theoremstyle{definition} %styl dla definicji
\newtheorem{zad}{} 
\title{Multizestaw zadań}
\author{Robert Fidytek}
%\date{\today}
\date{}
\newcounter{liczniksekcji}
\newcommand{\kategoria}[1]{\section{#1}} %olreślamy nazwę kateforii zadań
\newcommand{\zadStart}[1]{\begin{zad}#1\newline} %oznaczenie początku zadania
\newcommand{\zadStop}{\end{zad}}   %oznaczenie końca zadania
%Makra opcjonarne (nie muszą występować):
\newcommand{\rozwStart}[2]{\noindent \textbf{Rozwiązanie (autor #1 , recenzent #2): }\newline} %oznaczenie początku rozwiązania, opcjonarnie można wprowadzić informację o autorze rozwiązania zadania i recenzencie poprawności wykonania rozwiązania zadania
\newcommand{\rozwStop}{\newline}                                            %oznaczenie końca rozwiązania
\newcommand{\odpStart}{\noindent \textbf{Odpowiedź:}\newline}    %oznaczenie początku odpowiedzi końcowej (wypisanie wyniku)
\newcommand{\odpStop}{\newline}                                             %oznaczenie końca odpowiedzi końcowej (wypisanie wyniku)
\newcommand{\testStart}{\noindent \textbf{Test:}\newline} %ewentualne możliwe opcje odpowiedzi testowej: A. ? B. ? C. ? D. ? itd.
\newcommand{\testStop}{\newline} %koniec wprowadzania odpowiedzi testowych
\newcommand{\kluczStart}{\noindent \textbf{Test poprawna odpowiedź:}\newline} %klucz, poprawna odpowiedź pytania testowego (jedna literka): A lub B lub C lub D itd.
\newcommand{\kluczStop}{\newline} %koniec poprawnej odpowiedzi pytania testowego 
\newcommand{\wstawGrafike}[2]{\begin{figure}[h] \includegraphics[scale=#2] {#1} \end{figure}} %gdyby była potrzeba wstawienia obrazka, parametry: nazwa pliku, skala (jak nie wiesz co wpisać, to wpisz 1)

\begin{document}
\maketitle


\kategoria{Wikieł/Z4.7e}
\zadStart{Zadanie z Wikieł Z 4.7 e) moja wersja nr [nrWersji]}
%[a]:[2,3,5,7]
%[b]:[2,3,4,5]
%math.gcd([b],2)==1
Obliczyć granice funkcji $\displaystyle{\lim_{x \to \infty}}\sqrt{[a]x}\big(\sqrt{x+[b]}-\sqrt{x}\big)$
\zadStop
\rozwStart{Pascal Nawrocki}{Jakub Ulrych}
Mnożymy sobie nawias przez sprzężenie:
$$\displaystyle{\lim_{x \to \infty}}\sqrt{[a]x}\big((\sqrt{x+[b]}-\sqrt{x})\cdot\frac{\sqrt{x+[b]}+\sqrt{x}}{\sqrt{x+[b]}+\sqrt{x}}\big)$$
Stosujemy wzór skróconego mnożenia $(a-b)(a+b)=a^2-b^2$:
$$\displaystyle{\lim_{x \to \infty}}\sqrt{[a]x}\bigg(\frac{x+[b]-x}{\sqrt{x+[b]}+\sqrt{x}}\bigg)$$
$$\displaystyle{\lim_{x \to \infty}}\sqrt{[a]x}\bigg(\frac{[b]}{\sqrt{x+[b]}+\sqrt{x}}\bigg)$$
$$\displaystyle{\lim_{x \to \infty}}\bigg(\frac{[b]\cdot\sqrt{[a]x}}{\sqrt{x+[b]}+\sqrt{x}}\bigg)$$
Stosujemy wzór $\sqrt{ab}=\sqrt{a}*\sqrt{b}$ oraz $\sqrt{a+b}=\sqrt{a}\cdot\sqrt{1+\frac{b}{a}}$
$$\displaystyle{\lim_{x \to \infty}}\bigg(\frac{\sqrt{x}\cdot[b]\cdot\sqrt{[a]}}{\sqrt{x}(\sqrt{1+\frac{[b]}{x}}+1)}\bigg)$$
$$\displaystyle{\lim_{x \to \infty}}\bigg(\frac{[b]\sqrt{[a]}}{(\sqrt{1+\frac{[b]}{x}}+1)}\bigg)=\frac{[b]\sqrt{[a]}}{2}$$
\rozwStop
\odpStart
$\frac{[b]\sqrt{[a]}}{2}$
\odpStop
\testStart
A.$\frac{[b]\sqrt{[a]}}{2}$
B.$\infty$
C.$-\infty$
D.$1$
\testStop
\kluczStart
A
\kluczStop
\end{document}