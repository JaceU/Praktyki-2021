\documentclass[12pt, a4paper]{article}
\usepackage[utf8]{inputenc}
\usepackage{polski}

\usepackage{amsthm}  %pakiet do tworzenia twierdzeń itp.
\usepackage{amsmath} %pakiet do niektórych symboli matematycznych
\usepackage{amssymb} %pakiet do symboli mat., np. \nsubseteq
\usepackage{amsfonts}
\usepackage{graphicx} %obsługa plików graficznych z rozszerzeniem png, jpg
\theoremstyle{definition} %styl dla definicji
\newtheorem{zad}{} 
\title{Multizestaw zadań}
\author{Robert Fidytek}
%\date{\today}
\date{}
\newcounter{liczniksekcji}
\newcommand{\kategoria}[1]{\section{#1}} %olreślamy nazwę kateforii zadań
\newcommand{\zadStart}[1]{\begin{zad}#1\newline} %oznaczenie początku zadania
\newcommand{\zadStop}{\end{zad}}   %oznaczenie końca zadania
%Makra opcjonarne (nie muszą występować):
\newcommand{\rozwStart}[2]{\noindent \textbf{Rozwiązanie (autor #1 , recenzent #2): }\newline} %oznaczenie początku rozwiązania, opcjonarnie można wprowadzić informację o autorze rozwiązania zadania i recenzencie poprawności wykonania rozwiązania zadania
\newcommand{\rozwStop}{\newline}                                            %oznaczenie końca rozwiązania
\newcommand{\odpStart}{\noindent \textbf{Odpowiedź:}\newline}    %oznaczenie początku odpowiedzi końcowej (wypisanie wyniku)
\newcommand{\odpStop}{\newline}                                             %oznaczenie końca odpowiedzi końcowej (wypisanie wyniku)
\newcommand{\testStart}{\noindent \textbf{Test:}\newline} %ewentualne możliwe opcje odpowiedzi testowej: A. ? B. ? C. ? D. ? itd.
\newcommand{\testStop}{\newline} %koniec wprowadzania odpowiedzi testowych
\newcommand{\kluczStart}{\noindent \textbf{Test poprawna odpowiedź:}\newline} %klucz, poprawna odpowiedź pytania testowego (jedna literka): A lub B lub C lub D itd.
\newcommand{\kluczStop}{\newline} %koniec poprawnej odpowiedzi pytania testowego 
\newcommand{\wstawGrafike}[2]{\begin{figure}[h] \includegraphics[scale=#2] {#1} \end{figure}} %gdyby była potrzeba wstawienia obrazka, parametry: nazwa pliku, skala (jak nie wiesz co wpisać, to wpisz 1)

\begin{document}
\maketitle


\kategoria{Wikieł/Z4.2e}
\zadStart{Zadanie z Wikieł Z 4.2e) moja wersja nr [nrWersji]}
%[p1]:[2,3,4,5,6,7,8,9]
%[p2]:[2,3,4,5,6,7,8,9]
%[p3]:[2,3,4,5,6,7,8,9]
%[b]=[p1]-[p2]
%[c]=-[p1]*[p2]
%[d2]=-[p1]*[p3]
%[up]=-[p1]-[p2]
%[down]=[p1]*[p1]-[p3]
%[up1]=abs([up])
%[down1]=abs([down])
%[b]<-1 and [down]!=0 and math.gcd([up],[down])==1 and [down]!=1 and [down]!=-1 and [up]*[down]<0
Obliczyć granicę funkcji $$\lim_{x \to -[p1]} \frac{x^2 [b]x [c]}{x^3 + [p1]x^2 - [p3]x [d2]}$$
\zadStop
\rozwStart{Jakub Janik}{Martyna Czarnobaj}
$$\lim_{x \to -[p1]} \frac{x^2 [b]x [c]}{x^3 + [p1]x^2 - [p3]x [d2]}=\lim_{x \to -[p1]} \frac{(x-[p2])(x+[p1])}{(x^2-[p3])(x+[p1])}=$$
$$\lim_{x \to -[p1]} \frac{(x-[p2])}{(x^2-[p3])}=-\frac{[up1]}{[down1]}$$
\rozwStop
\odpStart
$-\frac{[up1]}{[down1]}$
\odpStop
\testStart
A.$-\frac{[up1]}{[down1]}$
B.$0$
C.$-\frac{[down1]}{[up1]}$
D.$\infty$
\testStop
\kluczStart
A
\kluczStop



\end{document}