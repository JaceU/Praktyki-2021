\documentclass[12pt, a4paper]{article}
\usepackage[utf8]{inputenc}
\usepackage{polski}

\usepackage{amsthm}  %pakiet do tworzenia twierdzeń itp.
\usepackage{amsmath} %pakiet do niektórych symboli matematycznych
\usepackage{amssymb} %pakiet do symboli mat., np. \nsubseteq
\usepackage{amsfonts}
\usepackage{graphicx} %obsługa plików graficznych z rozszerzeniem png, jpg
\theoremstyle{definition} %styl dla definicji
\newtheorem{zad}{} 
\title{Multizestaw zadań}
\author{Robert Fidytek}
%\date{\today}
\date{}
\newcounter{liczniksekcji}
\newcommand{\kategoria}[1]{\section{#1}} %olreślamy nazwę kateforii zadań
\newcommand{\zadStart}[1]{\begin{zad}#1\newline} %oznaczenie początku zadania
\newcommand{\zadStop}{\end{zad}}   %oznaczenie końca zadania
%Makra opcjonarne (nie muszą występować):
\newcommand{\rozwStart}[2]{\noindent \textbf{Rozwiązanie (autor #1 , recenzent #2): }\newline} %oznaczenie początku rozwiązania, opcjonarnie można wprowadzić informację o autorze rozwiązania zadania i recenzencie poprawności wykonania rozwiązania zadania
\newcommand{\rozwStop}{\newline}                                            %oznaczenie końca rozwiązania
\newcommand{\odpStart}{\noindent \textbf{Odpowiedź:}\newline}    %oznaczenie początku odpowiedzi końcowej (wypisanie wyniku)
\newcommand{\odpStop}{\newline}                                             %oznaczenie końca odpowiedzi końcowej (wypisanie wyniku)
\newcommand{\testStart}{\noindent \textbf{Test:}\newline} %ewentualne możliwe opcje odpowiedzi testowej: A. ? B. ? C. ? D. ? itd.
\newcommand{\testStop}{\newline} %koniec wprowadzania odpowiedzi testowych
\newcommand{\kluczStart}{\noindent \textbf{Test poprawna odpowiedź:}\newline} %klucz, poprawna odpowiedź pytania testowego (jedna literka): A lub B lub C lub D itd.
\newcommand{\kluczStop}{\newline} %koniec poprawnej odpowiedzi pytania testowego 
\newcommand{\wstawGrafike}[2]{\begin{figure}[h] \includegraphics[scale=#2] {#1} \end{figure}} %gdyby była potrzeba wstawienia obrazka, parametry: nazwa pliku, skala (jak nie wiesz co wpisać, to wpisz 1)

\begin{document}
\maketitle


\kategoria{Wikieł/C1.1e}
\zadStart{Zadanie z Wikieł C 1.1e moja wersja nr [nrWersji]}
%[a]:[2,3,4,5,6,7,8,9,10,11,12,13,14,15,16,17,18,19,20,21,22,23,24,25,26,27,28,29,30,31,32,33,34,35,36,37,38,39,40]
%[aa]=[a]*[a]
%[a2]=[a]*2
Oblicz całkę $$\int \frac{([a]+x)^2}{x^2}dx.$$
\zadStop
\rozwStart{Aleksandra Pasińska}{}
$$\int \frac{([a]+x)^2}{x^2}dx=\int \frac{[aa]+[a2]x+x^2}{x^2}dx=\int \frac{[aa]}{x^2}+\frac{[a2]}{x}+1dx=$$
$$=[aa]\int\frac{1}{x^2}dx+[a2]\int\frac{1}{x}dx+\int1dx= $$
$$=-\frac{[aa]}{x}+[a2] ln(|x|)+x+C$$
\rozwStop
\odpStart
$-\frac{[aa]}{x}+[a2] ln(|x|)+x+C$\\
\odpStop
\testStart
A.$-\frac{[aa]}{x}+[a2] ln(|x|)+x+C$
B.$-\frac{2x^5\sqrt{x}}{11}+C$
C.$\frac{[a2]x\sqrt{x}}{19}-2+C$
D.$\frac{[a2]x\sqrt{x}}{13}+C$
E.$\frac{2x^5\sqrt{x}}{11}+C$
F.$\frac{[a2]x^3\sqrt{x}}{7}+\frac{2x^5\sqrt{x}}{11}+C$
G.$-\frac{[a2]x^3\sqrt{x}}{7}+\frac{2x^5\sqrt{x}}{11}+C$
H.$\frac{-x\sqrt{x}}{3}+\frac{2x^5\sqrt{x}}{11}+C$
I.$\frac{x\sqrt{x}}{3}-\frac{[a2]x^3\sqrt{x}}{7}+C$
\testStop
\kluczStart
A
\kluczStop



\end{document}