\documentclass[12pt, a4paper]{article}
\usepackage[utf8]{inputenc}
\usepackage{polski}
\usepackage{amsthm}  %pakiet do tworzenia twierdzeń itp.
\usepackage{amsmath} %pakiet do niektórych symboli matematycznych
\usepackage{amssymb} %pakiet do symboli mat., np. \nsubseteq
\usepackage{amsfonts}
\usepackage{graphicx} %obsługa plików graficznych z rozszerzeniem png, jpg
\theoremstyle{definition} %styl dla definicji
\newtheorem{zad}{} 
\title{Multizestaw zadań}
\author{Radosław Grzyb}
%\date{\today}
\date{}
\newcounter{liczniksekcji}
\newcommand{\kategoria}[1]{\section{#1}} %olreślamy nazwę kateforii zadań
\newcommand{\zadStart}[1]{\begin{zad}#1\newline} %oznaczenie początku zadania
\newcommand{\zadStop}{\end{zad}}   %oznaczenie końca zadania
%Makra opcjonarne (nie muszą występować):
\newcommand{\rozwStart}[2]{\noindent \textbf{Rozwiązanie (autor #1 , recenzent #2): }\newline} %oznaczenie początku rozwiązania, opcjonarnie można wprowadzić informację o autorze rozwiązania zadania i recenzencie poprawności wykonania rozwiązania zadania
\newcommand{\rozwStop}{\newline}                                            %oznaczenie końca rozwiązania
\newcommand{\odpStart}{\noindent \textbf{Odpowiedź:}\newline}    %oznaczenie początku odpowiedzi końcowej (wypisanie wyniku)
\newcommand{\odpStop}{\newline}                                             %oznaczenie końca odpowiedzi końcowej (wypisanie wyniku)
\newcommand{\testStart}{\noindent \textbf{Test:}\newline} %ewentualne możliwe opcje odpowiedzi testowej: A. ? B. ? C. ? D. ? itd.
\newcommand{\testStop}{\newline} %koniec wprowadzania odpowiedzi testowych
\newcommand{\kluczStart}{\noindent \textbf{Test poprawna odpowiedź:}\newline} %klucz, poprawna odpowiedź pytania testowego (jedna literka): A lub B lub C lub D itd.
\newcommand{\kluczStop}{\newline} %koniec poprawnej odpowiedzi pytania testowego 
\newcommand{\wstawGrafike}[2]{\begin{figure}[h] \includegraphics[scale=#2] {#1} \end{figure}} %gdyby była potrzeba wstawienia obrazka, parametry: nazwa pliku, skala (jak nie wiesz co wpisać, to wpisz 1)
\begin{document}
\maketitle
\kategoria{Wikieł/Z1.84r}
\zadStart{Zadanie z Wikieł Z 1.84 r moja wersja nr [nrWersji]}
%[p2]:[1,2,3,4,5,6,7,8,9,10,11,12,13,14,15]
%[p3]:[1,2,3]
%[p4]:[1,2,3,4,5]
%[c1]=4**[p3]
%[c2]=2**[p4]
%[l1]=1-[c1]
%[g1]=[c2]*[l1]
%[g2]=[p2]*[c1]*[c2]
%[Delta]=int(math.pow([c1],2)-4*[g1]*[g2])
%[tDelta]=math.sqrt([Delta])
%[sDelta]=int([tDelta])
%[t1]=(-[c1]-[sDelta])/(2*[g1])
%[t2]=(-[c1]+[sDelta])/(2*[g1])
%[f1]=int(-[c1]-[sDelta])
%[f2]=int(-[c1]+[sDelta])
%[f11]=2*[g1]
%[gcd1]=math.gcd([f1],[f11])
%[gcd2]=math.gcd([f2],[f11])
%[wg1]=-int([f1]/[gcd1])
%[wg2]=int([f2]/[gcd2])
%[wgf1]=-int([f11]/[gcd1])
%[wgf2]=int([f11]/[gcd2])
%[Delta]>0 and ([tDelta]).is_integer() is True and ([t1]>0 or [t2]>0) and [wgf1]>1
Rozwiązać równanie:
$$4^{x-1}+0.5^{1-x}=0.25^{-x}-[p2]$$
\zadStop
\rozwStart{Radosław Grzyb}{}
$$4^{x-[p3]}+2^{x-[p4]}=4^{x}-[p2]$$
$$4^{x}:4^{[p3]}+2^{x}:2^{[p4]}=4^{x}-[p2]$$
$$4^{x}\cdot(\frac{1}{4})^{[p3]}+2^{x}\cdot(\frac{1}{2})^{[p4]}=4^{x}-[p2]$$
$$4^{x}\cdot\frac{1}{[c1]}+2^{x}\cdot\frac{1}{[c2]}-4^{x}+[p2]=0$$
$$\frac{[l1]}{[c1]}\cdot4^{x}+2^{x}\cdot\frac{1}{[c2]}+[p2]=0$$
$$\frac{[l1]}{[c1]}\cdot2^{2x}+\frac{1}{[c2]}\cdot2^{x}+[p2]=0 //\cdot[c1]$$
$$[l1]\cdot2^{2x}+\frac{[c1]}{[c2]}\cdot2^{x}+[c1]\cdot[p2]=0 //\cdot[c2]$$
$$[g1]\cdot2^{2x}+[c1]\cdot2^{x}+[g2]=0$$
Zauważamy, że nasze równanie wykładnicze da się rozwiązać deltą podstawiając $t=2^{x}$:
$$[g1]\cdot t^{2}+[c1]\cdot t+[g2]=0$$
Czas policzyć deltę:
$$\Delta_{t}=[c1]^{2}-4\cdot[g1]\cdot[g2]=[Delta]$$
$$\sqrt{\Delta_{t}}=[sDelta]$$
Czas znaleźć miejsca zerowe:
$$t_{1}=\frac{-[c1]-[sDeltam]}{2\cdot[g1]}=\frac{[f1]}{[f11]}=\frac{[wg1]}{[wgf1]}$$
$$t_{2}=\frac{-[c1]+[sDeltam]}{2\cdot[g1]}=\frac{[f2]}{[f11]}=\frac{[wg2]}{[wgf2]}$$
Z racji tego, że $2^{x}$ musi być dodatnie odrzucamy drugi wynik, czyli $t_{2}$.\\
A więc:
$$t_{1}=2^{x}=\frac{[wg1]}{[wgf1]}\implies x=\frac{\log\frac{[wg1]}{[wgf1]}}{\log2}$$
\rozwStop
\odpStart
$$x=\frac{\log\frac{[wg1]}{[wgf1]}}{\log2}$$
\odpStop
\testStart
A. $$x=\frac{\log\frac{[wg1]}{[wgf1]}}{\log4}$$
B. $$x=\frac{[wg1]}{[wgf1]}$$
C. $$x=[wg1]^{[wgf1]}$$
D. $$x=\frac{\log\frac{[wg1]}{[wgf1]}}{\log2}$$
\testStop
\kluczStart
D
\kluczStop
\end{document}