\documentclass[12pt, a4paper]{article}
\usepackage[utf8]{inputenc}
\usepackage{polski}

\usepackage{amsthm}  %pakiet do tworzenia twierdzeń itp.
\usepackage{amsmath} %pakiet do niektórych symboli matematycznych
\usepackage{amssymb} %pakiet do symboli mat., np. \nsubseteq
\usepackage{amsfonts}
\usepackage{graphicx} %obsługa plików graficznych z rozszerzeniem png, jpg
\theoremstyle{definition} %styl dla definicji
\newtheorem{zad}{} 
\title{Multizestaw zadań}
\author{Mirella Narewska}
%\date{\today}
\date{}
\newcounter{liczniksekcji}
\newcommand{\kategoria}[1]{\section{#1}} %olreślamy nazwę kateforii zadań
\newcommand{\zadStart}[1]{\begin{zad}#1\newline} %oznaczenie początku zadania
\newcommand{\zadStop}{\end{zad}}   %oznaczenie końca zadania
%Makra opcjonarne (nie muszą występować):
\newcommand{\rozwStart}[2]{\noindent \textbf{Rozwiązanie (autor #1 , recenzent #2): }\newline} %oznaczenie początku rozwiązania, opcjonarnie można wprowadzić informację o autorze rozwiązania zadania i recenzencie poprawności wykonania rozwiązania zadania
\newcommand{\rozwStop}{\newline}                                            %oznaczenie końca rozwiązania
\newcommand{\odpStart}{\noindent \textbf{Odpowiedź:}\newline}    %oznaczenie początku odpowiedzi końcowej (wypisanie wyniku)
\newcommand{\odpStop}{\newline}                                             %oznaczenie końca odpowiedzi końcowej (wypisanie wyniku)
\newcommand{\testStart}{\noindent \textbf{Test:}\newline} %ewentualne możliwe opcje odpowiedzi testowej: A. ? B. ? C. ? D. ? itd.
\newcommand{\testStop}{\newline} %koniec wprowadzania odpowiedzi testowych
\newcommand{\kluczStart}{\noindent \textbf{Test poprawna odpowiedź:}\newline} %klucz, poprawna odpowiedź pytania testowego (jedna literka): A lub B lub C lub D itd.
\newcommand{\kluczStop}{\newline} %koniec poprawnej odpowiedzi pytania testowego 
\newcommand{\wstawGrafike}[2]{\begin{figure}[h] \includegraphics[scale=#2] {#1} \end{figure}} %gdyby była potrzeba wstawienia obrazka, parametry: nazwa pliku, skala (jak nie wiesz co wpisać, to wpisz 1)

\begin{document}
\maketitle


\kategoria{Wikieł/z1.14m}
\zadStart{Zadanie z Wikieł z1.14m  moja wersja nr [nrWersji]}
%[a]:[2,3,4,5,6,7,8,9,10,11,12]
%[b]:[2,3,4,5,6,7,8]
%[c]:[2,3,4,5]
%[d]:[3,4,5,6,7]
%[e]=[d]+[c]
%[f]=[c]-[d]
%[s]=round([e]/[a],2)
%[p]=round([f]/[a],2)
%[z]=round([s]+[b],2)
%[r]=round([b]-[s],2)
%[c]>[d]
%[t]=round([p]+[b],2)
%[v]=round([b]-[p],2)
Rozwiązać równanie $|[a]|x-[b]|-[c]|=[d]$
\zadStop
\rozwStart{Mirella Narewska}{}
$$|[a] |x-[b]|-[c]|=[d] \Leftrightarrow$$
$$[a]|x-[b]|-[c]=[d] \vee [a] |x-[b]|-[c]=-[d]$$
$$[a]|x-[b]|=[e] \vee [a]|x-[b]|=[f]$$
$$|x-[b]|=[s] \vee |x-[b]|=[p]$$
$$x-[b]=[s] \vee x-[b]=-[s] \vee x-[b]=[p] \vee x-[b]=-[p]$$
$$x=[z] \vee x=[r] \vee x=[t] \vee x=[v]$$
\rozwStop
\odpStart
$x=[z] \vee x=[r] \vee x=[t] \vee x=[v]$
\odpStop
\testStart
A.$x=[z] \vee x=[r] \vee x=[t] \vee x=[v]$
\\
B.$x=0$
\\
C.$x=[z] \vee x=[r]$
\\
D.$x=[v]$
\testStop
\kluczStart
A
\kluczStop



\end{document}