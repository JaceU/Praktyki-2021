\documentclass[12pt, a4paper]{article}
\usepackage[utf8]{inputenc}
\usepackage{polski}

\usepackage{amsthm}  %pakiet do tworzenia twierdzeń itp.
\usepackage{amsmath} %pakiet do niektórych symboli matematycznych
\usepackage{amssymb} %pakiet do symboli mat., np. \nsubseteq
\usepackage{amsfonts}
\usepackage{graphicx} %obsługa plików graficznych z rozszerzeniem png, jpg
\theoremstyle{definition} %styl dla definicji
\newtheorem{zad}{} 
\title{Multizestaw zadań}
\author{Jacek Jabłoński}
%\date{\today}
\date{}
\newcounter{liczniksekcji}
\newcommand{\kategoria}[1]{\section{#1}} %olreślamy nazwę kateforii zadań
\newcommand{\zadStart}[1]{\begin{zad}#1\newline} %oznaczenie początku zadania
\newcommand{\zadStop}{\end{zad}}   %oznaczenie końca zadania
%Makra opcjonarne (nie muszą występować):
\newcommand{\rozwStart}[2]{\noindent \textbf{Rozwiązanie (autor #1 , recenzent #2): }\newline} %oznaczenie początku rozwiązania, opcjonarnie można wprowadzić informację o autorze rozwiązania zadania i recenzencie poprawności wykonania rozwiązania zadania
\newcommand{\rozwStop}{\newline}                                            %oznaczenie końca rozwiązania
\newcommand{\odpStart}{\noindent \textbf{Odpowiedź:}\newline}    %oznaczenie początku odpowiedzi końcowej (wypisanie wyniku)
\newcommand{\odpStop}{\newline}                                             %oznaczenie końca odpowiedzi końcowej (wypisanie wyniku)
\newcommand{\testStart}{\noindent \textbf{Test:}\newline} %ewentualne możliwe opcje odpowiedzi testowej: A. ? B. ? C. ? D. ? itd.
\newcommand{\testStop}{\newline} %koniec wprowadzania odpowiedzi testowych
\newcommand{\kluczStart}{\noindent \textbf{Test poprawna odpowiedź:}\newline} %klucz, poprawna odpowiedź pytania testowego (jedna literka): A lub B lub C lub D itd.
\newcommand{\kluczStop}{\newline} %koniec poprawnej odpowiedzi pytania testowego 
\newcommand{\wstawGrafike}[2]{\begin{figure}[h] \includegraphics[scale=#2] {#1} \end{figure}} %gdyby była potrzeba wstawienia obrazka, parametry: nazwa pliku, skala (jak nie wiesz co wpisać, to wpisz 1)

\begin{document}
\maketitle


\kategoria{Wikieł/z1.84l}
\zadStart{Zadanie z Wikieł z1.84l) moja wersja nr [nrWersji]}
%[p1]:[2,3]
%[p2]:[2,4,8,16]
%[p3]:[2]
%[p4]:[2,4]
%[p5]:[2,3,4,5,6,7,8,9,10,11,12]
%[r1]=int(math.pow([p1],[p4]))
%[r2]=int([p2]/2)
%[r3]=[p4]+[p5]
%[r4]=2*[r2]
%[r5]=int([r3]/[r4])
%[c1]=math.sqrt([r3])
%[c2]=math.isqrt([r3])
%[c3]=math.sqrt([r4])
%[c4]=math.isqrt([r4])
%[c5]=math.gcd([r3],2)
%[f1]=[r5]+1
%[f2]=[r5]+2
%[f3]=[r5]+3
%[f4]=[r5]+4
%[f5]=[r5]+5
%[f6]=[r5]+6
%[f7]=[r5]+7
%[f8]=[r5]+8
%not([c1]!=[c2]) and not([c3]!=[c4]) and [c5]!=1 and [r5]>0
Rozwiązać równanie:
l) $(\sqrt{[p1]})^{[p2]x^{[p3]}-[p5]} = \sqrt{[r1]}$
\zadStop
\rozwStart{Jacek Jabłoński}{}
$$(\sqrt{[p1]})^{[p2]x^{[p3]}-[p5]} = \sqrt{[r1]}$$
$$[p1]^{\frac{1}{2}([p2]x^{[p3]}-[p5])} = ([p1]^{[p4]})^{\frac{1}{2}}$$
$$[p1]^{[r2]x^{[p3]}-\frac{[p5]}{2}} = [p1]^{\frac{[p4]}{2}}$$
$$[r2]x^{[p3]}-\frac{[p5]}{2} = \frac{[p4]}{2} $$
$$[r2]x^{[p3]} = \frac{[p4]}{2} + \frac{[p5]}{2} $$
$$x^{[p3]} = \frac{[r3]}{2} \cdot \frac{1}{[r2]}$$
$$x^{[p3]} =\frac{[r3]}{[r4]}$$
$$x^{[p3]} = [r5]$$
$$x=\sqrt{[r5]} \ \ lub \ \ x=-\sqrt{[r5]}$$
\rozwStop
\odpStart
$$x=\sqrt{[r5]} \ \ lub \ \ x=-\sqrt{[r5]}$$
\odpStop
\testStart
A. $$x=\sqrt{[r5]} \ \ lub \ \ x=-\sqrt{[r5]}$$
B. $$x=\sqrt{[f1]} \ \ lub \ \ x=-\sqrt{[f1]}$$
C. $$x=\sqrt{[f2]} \ \ lub \ \ x=-\sqrt{[f2]}$$
D. $$x=\sqrt{[f3]} \ \ lub \ \ x=-\sqrt{[f3]}$$
E. $$x=\sqrt{[f4]} \ \ lub \ \ x=-\sqrt{[f4]}$$
F. $$x=\sqrt{[f5]} \ \ lub \ \ x=-\sqrt{[f5]}$$
G. $$x=\sqrt{[f6]} \ \ lub \ \ x=-\sqrt{[f6]}$$
H. $$x=\sqrt{[f7]} \ \ lub \ \ x=-\sqrt{[f7]}$$
I. $$x=\sqrt{[f8]} \ \ lub \ \ x=-\sqrt{[f8]}$$
\testStop
\kluczStart
A
\kluczStop



\end{document}