\documentclass[12pt, a4paper]{article}
\usepackage[utf8]{inputenc}
\usepackage{polski}

\usepackage{amsthm}  %pakiet do tworzenia twierdzeń itp.
\usepackage{amsmath} %pakiet do niektórych symboli matematycznych
\usepackage{amssymb} %pakiet do symboli mat., np. \nsubseteq
\usepackage{amsfonts}
\usepackage{graphicx} %obsługa plików graficznych z rozszerzeniem png, jpg
\theoremstyle{definition} %styl dla definicji
\newtheorem{zad}{} 
\title{Multizestaw zadań}
\author{Robert Fidytek}
%\date{\today}
\date{}
\newcounter{liczniksekcji}
\newcommand{\kategoria}[1]{\section{#1}} %olreślamy nazwę kateforii zadań
\newcommand{\zadStart}[1]{\begin{zad}#1\newline} %oznaczenie początku zadania
\newcommand{\zadStop}{\end{zad}}   %oznaczenie końca zadania
%Makra opcjonarne (nie muszą występować):
\newcommand{\rozwStart}[2]{\noindent \textbf{Rozwiązanie (autor #1 , recenzent #2): }\newline} %oznaczenie początku rozwiązania, opcjonarnie można wprowadzić informację o autorze rozwiązania zadania i recenzencie poprawności wykonania rozwiązania zadania
\newcommand{\rozwStop}{\newline}                                            %oznaczenie końca rozwiązania
\newcommand{\odpStart}{\noindent \textbf{Odpowiedź:}\newline}    %oznaczenie początku odpowiedzi końcowej (wypisanie wyniku)
\newcommand{\odpStop}{\newline}                                             %oznaczenie końca odpowiedzi końcowej (wypisanie wyniku)
\newcommand{\testStart}{\noindent \textbf{Test:}\newline} %ewentualne możliwe opcje odpowiedzi testowej: A. ? B. ? C. ? D. ? itd.
\newcommand{\testStop}{\newline} %koniec wprowadzania odpowiedzi testowych
\newcommand{\kluczStart}{\noindent \textbf{Test poprawna odpowiedź:}\newline} %klucz, poprawna odpowiedź pytania testowego (jedna literka): A lub B lub C lub D itd.
\newcommand{\kluczStop}{\newline} %koniec poprawnej odpowiedzi pytania testowego 
\newcommand{\wstawGrafike}[2]{\begin{figure}[h] \includegraphics[scale=#2] {#1} \end{figure}} %gdyby była potrzeba wstawienia obrazka, parametry: nazwa pliku, skala (jak nie wiesz co wpisać, to wpisz 1)

\begin{document}
\maketitle


\kategoria{Wikieł/Z1.70e}
\zadStart{Zadanie z Wikieł Z 1.70 e) moja wersja nr [nrWersji]}
%[a]:[2, 3, 4, 5, 6, 7, 8, 9, 10, 11, 12, 13, 14, 15, 16, 17, 18, 19, 20, 21, 22, 23, 24, 25, 26, 27, 28, 29, 30,31,32,33,34,35,36,37,38,39,40]
%[b]:[2, 3, 4, 5, 6, 7, 8, 9, 10, 11, 12, 13, 14, 15, 16, 17, 18, 19, 20, 21, 22, 23, 24, 25, 26, 27, 28, 29, 30,31,32,33,34,35,36,37,38,39,40]
%[1a]=(-1-[a])
%[-1a]=(1+[a])
%[apb]=[a]+[b]
%[4apb]=4*[apb]
%[delta]=pow([1a],2)-[4apb]
%[x0]=([-1a]/2)
%[x01]=int([x0])
%pow([1a],2)==[4apb]
Rozwiązać nierówność $x\leq[a]-\frac{[b]}{x-1}$
\zadStop
\rozwStart{Jakub Ulrych}{Pascal Nawrocki}
założenie: $$x-1\neq0$$
$$x\neq1$$
dziedzina:$$x\in \mathbb{R}-\{1\}$$
rozwiązanie:$$x\leq[a]-\frac{[b]}{x-1}$$
$$x-[a]+\frac{[b]}{x-1}\leq0$$
$$\frac{x(x-1)-[a](x-1)+[b]}{x-1}\leq0\Leftrightarrow (x^{2}+([1a])x+[apb])\cdot(x-1)\leq0$$
Podzielimy rozwiązanie na 2 przypadki:
$$\textbf{1)}x^{2}+([1a])x+[apb]\geq0 \land (x-1)\leq0 \vee \textbf{2)}x^{2}+([1a])x+[apb]\leq0 \land (x-1)\geq0$$
Pierwszy przypadek:\\
\textbf{1)}
$$\Delta=[delta] \land x\leq1$$
$$x_{0}=\frac{[-1a]}{2}=[x01]\Rightarrow(x-[x01])^{2}\geq0\Rightarrow x\in \mathbb{R} \land x\leq1 \land x\neq1\Rightarrow x\in(-\infty,1)$$
Drugi przypadek:\\
\textbf{2)}
$$(x-[x01])^{2}\leq0 \land (x-1)\geq0$$
$$x=[x01] \land x\geq1 \land x\neq1\Rightarrow x=[x01]$$
Suma obu przypadków:
$$\textbf{1)}\vee\textbf{2)}\Rightarrow x\in(-\infty,1)\cup\{[x01]\}$$
\rozwStop
\odpStart
$$x\in(-\infty,1)\cup\{[x01]\}$$
\odpStop
\testStart
A.$x\in(-\infty,1)\cup\{[x01]\}$
B.$x\in(-\infty,1)$
C.$x\in(-\infty,-1)\cup\{[x01]\}$\\
D.$x=[x01]$
\testStop
\kluczStart
A
\kluczStop
\end{document}