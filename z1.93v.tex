\documentclass[12pt, a4paper]{article}
\usepackage[utf8]{inputenc}
\usepackage{polski}

\usepackage{amsthm}  %pakiet do tworzenia twierdzeń itp.
\usepackage{amsmath} %pakiet do niektórych symboli matematycznych
\usepackage{amssymb} %pakiet do symboli mat., np. \nsubseteq
\usepackage{amsfonts}
\usepackage{graphicx} %obsługa plików graficznych z rozszerzeniem png, jpg
\theoremstyle{definition} %styl dla definicji
\newtheorem{zad}{} 
\title{Multizestaw zadań}
\author{Robert Fidytek}
%\date{\today}
\date{}
\newcounter{liczniksekcji}
\newcommand{\kategoria}[1]{\section{#1}} %olreślamy nazwę kateforii zadań
\newcommand{\zadStart}[1]{\begin{zad}#1\newline} %oznaczenie początku zadania
\newcommand{\zadStop}{\end{zad}}   %oznaczenie końca zadania
%Makra opcjonarne (nie muszą występować):
\newcommand{\rozwStart}[2]{\noindent \textbf{Rozwiązanie (autor #1 , recenzent #2): }\newline} %oznaczenie początku rozwiązania, opcjonarnie można wprowadzić informację o autorze rozwiązania zadania i recenzencie poprawności wykonania rozwiązania zadania
\newcommand{\rozwStop}{\newline}                                            %oznaczenie końca rozwiązania
\newcommand{\odpStart}{\noindent \textbf{Odpowiedź:}\newline}    %oznaczenie początku odpowiedzi końcowej (wypisanie wyniku)
\newcommand{\odpStop}{\newline}                                             %oznaczenie końca odpowiedzi końcowej (wypisanie wyniku)
\newcommand{\testStart}{\noindent \textbf{Test:}\newline} %ewentualne możliwe opcje odpowiedzi testowej: A. ? B. ? C. ? D. ? itd.
\newcommand{\testStop}{\newline} %koniec wprowadzania odpowiedzi testowych
\newcommand{\kluczStart}{\noindent \textbf{Test poprawna odpowiedź:}\newline} %klucz, poprawna odpowiedź pytania testowego (jedna literka): A lub B lub C lub D itd.
\newcommand{\kluczStop}{\newline} %koniec poprawnej odpowiedzi pytania testowego 
\newcommand{\wstawGrafike}[2]{\begin{figure}[h] \includegraphics[scale=#2] {#1} \end{figure}} %gdyby była potrzeba wstawienia obrazka, parametry: nazwa pliku, skala (jak nie wiesz co wpisać, to wpisz 1)

\begin{document}
\maketitle


\kategoria{Wikieł/Z1.93v}
\zadStart{Zadanie z Wikieł Z 1.93 v) moja wersja nr [nrWersji]}
%[a]:[2,3,4,5,6,7,8,9]
%[c]:[1,2,3,4,5,6,7,8,9,10,11,12,13,14,15]
%[delta]=1+8*[c]
%[pr2]=(pow([delta],(1/2)))
%[pr1]=[pr2].real
%[pr]=int([pr1])
%[zz1]=(-1-[pr])
%[zz2]=(-1+[pr])
%[zzz1]=[zz1]/4
%[zzz2]=[zz2]/4
%[mm1]=math.gcd([zz1],4)
%[mm2]=math.gcd([zz2],4)
%[z1]=int([zz1]/[mm1])
%[z2]=int([zz2]/[mm2])
%[m1]=int(4/[mm1])
%[m2]=int(4/[mm2])
%[x1]=round(pow([a],[zzz1]),2)
%[x2]=round(pow([a],[zzz2]),2)
%[xx1]=pow([a],[z1])
%[xx2]=pow([a],[z2])
%[delta]>0 and [pr2].is_integer()==True
Rozwiązać równanie $\log_{x}{([a]x^2)} \cdot \log^2_{[a]}{x} = [c]$
\zadStop
\rozwStart{Małgorzata Ugowska}{}
Dziedzina: $x \in (0, \infty)$
$$ \log_{x}{([a]x^2)} \cdot \log^2_{[a]}{x} = [c] \quad \Longleftrightarrow \quad (\log_{x}{(x^2)}+\log_{x}{[a]}) \cdot \log^2_{[a]}{x} = [c] $$
$$\Longleftrightarrow \quad (2+\frac{1}{\log_{[a]}{x}}) \cdot \log^2_{[a]}{x} = [c] $$
Podstawiamy $y=\log_{[a]}{x}$ i mamy:
$$\Big(2+\frac{1}{y}\Big)y^2=[c] \quad \Longleftrightarrow \quad 2y^2 +y -[c] = 0$$
$$ \bigtriangleup = 1^2 + 4 \cdot 2 \cdot [c] = [delta] \quad  \Longrightarrow \quad \sqrt{\bigtriangleup} = [pr]$$
$$y_1=\frac{-1-\sqrt{\bigtriangleup}}{2\cdot 2} = \frac{[z1]}{[m1]} = [zzz1] \quad \land \quad y_2=\frac{-1+\sqrt{\bigtriangleup}}{2\cdot 2} = \frac{[z2]}{[m2]} = [zzz2]$$
dla $y=[zzz1]$:
$$\log_{[a]}{x} = [zzz1] \quad  \Longrightarrow \quad x = [a]^{[zzz1]} = [x1]$$
dla $y=[zzz2]$:
$$\log_{[a]}{x} = [zzz2] \quad  \Longrightarrow \quad x = [a]^{[zzz2]} = [x2]$$
\rozwStop
\odpStart
$x \in \{[x1],[x2]\}$
\odpStop
\testStart
A. $x \in \{[delta], [c]\}$\\
B. $x \in \{[x1],[x2]\}$\\
$x \in \{\sqrt{[z1]}, [z2]\}$
D. $x \in \{0, 2\}$\\
E. $x \in \{-4, 5\}$
\testStop
\kluczStart
B
\kluczStop



\end{document}