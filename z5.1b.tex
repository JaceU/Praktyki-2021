\documentclass[12pt, a4paper]{article}
\usepackage[utf8]{inputenc}
\usepackage{polski}

\usepackage{amsthm}  %pakiet do tworzenia twierdzeń itp.
\usepackage{amsmath} %pakiet do niektórych symboli matematycznych
\usepackage{amssymb} %pakiet do symboli mat., np. \nsubseteq
\usepackage{amsfonts}
\usepackage{graphicx} %obsługa plików graficznych z rozszerzeniem png, jpg
\theoremstyle{definition} %styl dla definicji
\newtheorem{zad}{} 
\title{Multizestaw zadań}
\author{Robert Fidytek}
%\date{\today}
\date{}
\newcounter{liczniksekcji}
\newcommand{\kategoria}[1]{\section{#1}} %olreślamy nazwę kateforii zadań
\newcommand{\zadStart}[1]{\begin{zad}#1\newline} %oznaczenie początku zadania
\newcommand{\zadStop}{\end{zad}}   %oznaczenie końca zadania
%Makra opcjonarne (nie muszą występować):
\newcommand{\rozwStart}[2]{\noindent \textbf{Rozwiązanie (autor #1 , recenzent #2): }\newline} %oznaczenie początku rozwiązania, opcjonarnie można wprowadzić informację o autorze rozwiązania zadania i recenzencie poprawności wykonania rozwiązania zadania
\newcommand{\rozwStop}{\newline}                                            %oznaczenie końca rozwiązania
\newcommand{\odpStart}{\noindent \textbf{Odpowiedź:}\newline}    %oznaczenie początku odpowiedzi końcowej (wypisanie wyniku)
\newcommand{\odpStop}{\newline}                                             %oznaczenie końca odpowiedzi końcowej (wypisanie wyniku)
\newcommand{\testStart}{\noindent \textbf{Test:}\newline} %ewentualne możliwe opcje odpowiedzi testowej: A. ? B. ? C. ? D. ? itd.
\newcommand{\testStop}{\newline} %koniec wprowadzania odpowiedzi testowych
\newcommand{\kluczStart}{\noindent \textbf{Test poprawna odpowiedź:}\newline} %klucz, poprawna odpowiedź pytania testowego (jedna literka): A lub B lub C lub D itd.
\newcommand{\kluczStop}{\newline} %koniec poprawnej odpowiedzi pytania testowego 
\newcommand{\wstawGrafike}[2]{\begin{figure}[h] \includegraphics[scale=#2] {#1} \end{figure}} %gdyby była potrzeba wstawienia obrazka, parametry: nazwa pliku, skala (jak nie wiesz co wpisać, to wpisz 1)

\begin{document}
\maketitle


\kategoria{Wikieł/Z5.1b}
\zadStart{Zadanie z Wikieł Z 5.1 b) moja wersja nr [nrWersji]}
%[a]:[2,3,4,5,6,7,8,9]
%[b]=2*[a]
%[a]!=[b]
Korzystając z definicji pochodnej, oblicz wartość pochodnej \\ $f(x)=\frac{[a]}{x^2} $.
\zadStop
\rozwStart{Joanna Świerzbin}{}
$$f(x)=\frac{[a]}{x^2}$$
$$f'(x)=\lim_{\Delta x \rightarrow 0} \frac{f(x+\Delta x)-f(x)}{\Delta x} = \lim_{\Delta x \rightarrow 0} \frac{\frac{[a]}{(x+\Delta x)^2}-\frac{[a]}{x^2}}{\Delta x}= $$
$$= \lim_{\Delta x \rightarrow 0} \frac{\frac{[a]x^2}{(x+\Delta x)^2 x^2}-\frac{[a](x+\Delta x)^2}{(x+\Delta x)^2 x^2}}{\Delta x}= 
 \lim_{\Delta x \rightarrow 0} \frac{\frac{[a]x^2-[a](x^2+2x\Delta x + (\Delta x)^2)}{(x+\Delta x)^2 x^2}}{\Delta x}= $$
$$= \lim_{\Delta x \rightarrow 0} \frac{\frac{[a]x^2 -[a]x^2-2\cdot[a]x\Delta x -[a] (\Delta x)^2}{(x+\Delta x)^2 x^2}}{\Delta x}= 
\lim_{\Delta x \rightarrow 0} \frac{-2\cdot[a]x \Delta x -[a](\Delta x)^2}{(x+\Delta x)^2 x^2 \Delta x}= $$
$$= \lim_{\Delta x \rightarrow 0} \frac{-2\cdot[a]x-[a]\Delta x}{x^4 +2x^3\Delta x +x^2(\Delta x)^2}=  -\frac{[b]}{x^3}$$
\rozwStop
\odpStart
$f'(x)= -\frac{[b]}{x^3}$
\odpStop
\testStart
A.$f'(x)= -\frac{[b]}{x^3}$\\
B. $f'(x)= -\frac{[a]}{x^3}$ \\
C. $f'(x)= -\frac{[b]}{x^2}$  \\
D. $f'(x)= -\frac{[a]}{x^2}$\\
E. $f'(x)= -\frac{1}{x^3}$\\
F. $f'(x)= 0 $
\testStop
\kluczStart
A
\kluczStop



\end{document}