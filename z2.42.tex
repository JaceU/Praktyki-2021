\documentclass[12pt, a4paper]{article}
\usepackage[utf8]{inputenc}
\usepackage{polski}

\usepackage{amsthm}  %pakiet do tworzenia twierdzeń itp.
\usepackage{amsmath} %pakiet do niektórych symboli matematycznych
\usepackage{amssymb} %pakiet do symboli mat., np. \nsubseteq
\usepackage{amsfonts}
\usepackage{graphicx} %obsługa plików graficznych z rozszerzeniem png, jpg
\theoremstyle{definition} %styl dla definicji
\newtheorem{zad}{} 
\title{Multizestaw zadań}
\author{Robert Fidytek}
%\date{\today}
\date{}
\newcounter{liczniksekcji}
\newcommand{\kategoria}[1]{\section{#1}} %olreślamy nazwę kateforii zadań
\newcommand{\zadStart}[1]{\begin{zad}#1\newline} %oznaczenie początku zadania
\newcommand{\zadStop}{\end{zad}}   %oznaczenie końca zadania
%Makra opcjonarne (nie muszą występować):
\newcommand{\rozwStart}[2]{\noindent \textbf{Rozwiązanie (autor #1 , recenzent #2): }\newline} %oznaczenie początku rozwiązania, opcjonarnie można wprowadzić informację o autorze rozwiązania zadania i recenzencie poprawności wykonania rozwiązania zadania
\newcommand{\rozwStop}{\newline}                                            %oznaczenie końca rozwiązania
\newcommand{\odpStart}{\noindent \textbf{Odpowiedź:}\newline}    %oznaczenie początku odpowiedzi końcowej (wypisanie wyniku)
\newcommand{\odpStop}{\newline}                                             %oznaczenie końca odpowiedzi końcowej (wypisanie wyniku)
\newcommand{\testStart}{\noindent \textbf{Test:}\newline} %ewentualne możliwe opcje odpowiedzi testowej: A. ? B. ? C. ? D. ? itd.
\newcommand{\testStop}{\newline} %koniec wprowadzania odpowiedzi testowych
\newcommand{\kluczStart}{\noindent \textbf{Test poprawna odpowiedź:}\newline} %klucz, poprawna odpowiedź pytania testowego (jedna literka): A lub B lub C lub D itd.
\newcommand{\kluczStop}{\newline} %koniec poprawnej odpowiedzi pytania testowego 
\newcommand{\wstawGrafike}[2]{\begin{figure}[h] \includegraphics[scale=#2] {#1} \end{figure}} %gdyby była potrzeba wstawienia obrazka, parametry: nazwa pliku, skala (jak nie wiesz co wpisać, to wpisz 1)

\begin{document}
\maketitle


\kategoria{Wikieł/Z2.42}
\zadStart{Zadanie z Wikieł Z 2.42 ) moja wersja nr [nrWersji]}
%[p1]:[2,3,4,5,6,10,11]
%[p2]:[2,3,4,7,8,9,10]
%[p3]:[2,3,4,5,6,7,8,9]
%[p4]:[2,3,4,5,6,7,11]
%[a]=random.randint(2,20)
%[b]=random.randint(1,10)
%[c]=random.randint(1,10)
%[w]=[p1]*[p4]-[p3]*[p2]
%[wxm]=[p4]+[p2]
%[wx]=-[p4]*[a]+[p2]*[b]
%[wym]=-[p1]-[p3]
%[wy]=-[b]*[p1]+[p3]*[a]
%[x1]=round(-[wx]/[wxm],2)
%[x2]=round(-[wy]/[wym],2)
%[abswym]=abs([wym])
%[m2]=round((([c]*[w])+[wx]+[wy])/(-[wxm]+[abswym]-[w]+0.00001),2)
%[m1]=round(([wy]-[wx]+([c]*[w]))/([wxm]+[abswym]-[w]+0.000001),2)
%[m3]=round((([c]*[w])+[wx]-[wy])/(-[wxm]-[abswym]-[w]+0.00001),2)
%([p1]*[p4]-[p3]*[p2])<0 and (-[p4]*[a]+[p2]*[b])>0 and (-[b]*[p1]+[p3]*[a])>0 and (-[wxm]-[wym]+[w])!=0 and ([m1]<[x1] and ([m2]>[x2] or [m2]<[x1]) and [m3]>[x2])
Podać, dla jakich wartości parametru $m$ rozwiązaniem układu równań:
$$
 \left\{ \begin{array}{ll}
[p1]\cdot x+[p2]\cdot y=m-[a] & \\
{[p3]}\cdot x+[p4]\cdot y=-m-[b]  & 
\end{array} \right.
$$
jest para $x$, $y$ spełniająca warunek $|x|+|y|=m+[c]$.
\zadStop
\rozwStart{Wojciech Przybylski}{Maja Szabłowska}
$$
W =
\left| \begin{array}{ccc}
[p1] & [p2]  \\
{[p3]} & [p4]  \\
\end{array} \right| =[p1]\cdot[p4]-[p3]\cdot[p2]=[w]
$$
$$
W_{x} =
\left| \begin{array}{ccc}
m-[a] & [p2]  \\
-m-[b] & [p4]  \\
\end{array} \right| =(m-[a])\cdot[p4]-(-m-[b])\cdot[p2]=[wxm]m+[wx]
$$
$$
W_{y} =
\left| \begin{array}{ccc}
[p1] & m-[a] \\
{[p3]} & -m-[b]  \\
\end{array} \right| =[p1]\cdot(-m-[b])-[p3]\cdot(m-[a])=[wym]m+[wy]
$$
W $\neq 0$, więc układ ma dokładnie jedno rozwiązanie
$$x=\frac{W_{x}}{W}=\frac{[wxm]m+[wx]}{[w]},$$
$$y=\frac{W_{y}}{W}=\frac{[wym]m+[wy]}{[w]}$$
Wstawiając $x$ i $y$ do nierówności $|x|+|y|=m+[c]$, otrzymujemy równanie
$|\frac{[wxm]m+[wx]}{[w]}|+|\frac{[wym]m+[wy]}{[w]}|=m+[c]$
z niewiadomą $m$. Rozpatrujemy trzy przypadki:
$$
1. \left\{ \begin{array}{ll}
m\in(-\infty,[x1]] & \\
{[wxm]}m+[wx]+[abswym]m-[wy]=(m+[c])\cdot[w]  
\end{array} \right. \Leftrightarrow m=[m1]\in(-\infty,[x1]] 
$$
$$
2. \left\{ \begin{array}{ll}
m\in([x1],[x2]] & \\
-[wxm]m-[wx]+[abswym]m-[wy]=(m+[c])\cdot[w]  
\end{array} \right. \Leftrightarrow m=[m2]\notin([x1],[x2]]
$$
$$
3. \left\{ \begin{array}{ll}
m\in([x2],\infty) & \\
-[wxm]m-[wx]-[abswym]m+[wy]=(m+[c])\cdot[w]  
\end{array} \right. \Leftrightarrow m=[m3]\in([x2],\infty)
$$
\rozwStop
\odpStart
$m=[m1]\wedge m=[m3]$
\odpStop
\testStart
A. $m=[m1]\wedge m=[m3]$\\
B. $m=[m2]\wedge m=[m3]$\\
C. $m=[m1]\wedge m=[m2]$\\
D. $m=[m1]\wedge m=-[m3]$\\
E. $m=0\wedge m=1$\\
F. nie ma takiego m 
\testStop
\kluczStart
A
\kluczStop



\end{document}