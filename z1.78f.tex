\documentclass[12pt, a4paper]{article}
\usepackage[utf8]{inputenc}
\usepackage{polski}

\usepackage{amsthm}  %pakiet do tworzenia twierdzeń itp.
\usepackage{amsmath} %pakiet do niektórych symboli matematycznych
\usepackage{amssymb} %pakiet do symboli mat., np. \nsubseteq
\usepackage{amsfonts}
\usepackage{graphicx} %obsługa plików graficznych z rozszerzeniem png, jpg
\theoremstyle{definition} %styl dla definicji
\newtheorem{zad}{} 
\title{Multizestaw zadań}
\author{Robert Fidytek}
%\date{\today}
\date{}
\newcounter{liczniksekcji}
\newcommand{\kategoria}[1]{\section{#1}} %olreślamy nazwę kateforii zadań
\newcommand{\zadStart}[1]{\begin{zad}#1\newline} %oznaczenie początku zadania
\newcommand{\zadStop}{\end{zad}}   %oznaczenie końca zadania
%Makra opcjonarne (nie muszą występować):
\newcommand{\rozwStart}[2]{\noindent \textbf{Rozwiązanie (autor #1 , recenzent #2): }\newline} %oznaczenie początku rozwiązania, opcjonarnie można wprowadzić informację o autorze rozwiązania zadania i recenzencie poprawności wykonania rozwiązania zadania
\newcommand{\rozwStop}{\newline}                                            %oznaczenie końca rozwiązania
\newcommand{\odpStart}{\noindent \textbf{Odpowiedź:}\newline}    %oznaczenie początku odpowiedzi końcowej (wypisanie wyniku)
\newcommand{\odpStop}{\newline}                                             %oznaczenie końca odpowiedzi końcowej (wypisanie wyniku)
\newcommand{\testStart}{\noindent \textbf{Test:}\newline} %ewentualne możliwe opcje odpowiedzi testowej: A. ? B. ? C. ? D. ? itd.
\newcommand{\testStop}{\newline} %koniec wprowadzania odpowiedzi testowych
\newcommand{\kluczStart}{\noindent \textbf{Test poprawna odpowiedź:}\newline} %klucz, poprawna odpowiedź pytania testowego (jedna literka): A lub B lub C lub D itd.
\newcommand{\kluczStop}{\newline} %koniec poprawnej odpowiedzi pytania testowego 
\newcommand{\wstawGrafike}[2]{\begin{figure}[h] \includegraphics[scale=#2] {#1} \end{figure}} %gdyby była potrzeba wstawienia obrazka, parametry: nazwa pliku, skala (jak nie wiesz co wpisać, to wpisz 1)

\begin{document}
\maketitle


\kategoria{Wikieł/Z1.78f}
\zadStart{Zadanie z Wikieł Z 1.78 f) moja wersja nr [nrWersji]}
%[b]:[6,8,10,12,16,18,24,30,32,40,42,48,60,70,72,80,96]
%[c]:[2,3,4,5,6,7,8,9,10,11,12,13,14,15,16,17,18,19,20]
%[a]=[c]**2
%[delta1]=[b]**2
%[delta2]=4*[a]
%[delta]=[delta1]+[delta2]
%[pdelta]=int(math.sqrt(abs([delta])))
%[lx1]=-[b]-[pdelta]
%[lx2]=-[b]+[pdelta]
%[x1]=int([lx1]/(-2))
%[x2]=int([lx2]/(-2))
%[2c]=2*[c]
%[cc]=[c]**2
%[2cb]=[2c]+[b]
%[cb]=int([2cb]/2)
%[x22]=-[x2]
%[delta]>0 and [pdelta]**2==[delta] and math.gcd([lx1],-2)==2 and math.gcd([lx2],-2)==2 and math.gcd([2cb],2)==2 and [cb]>[x2] and [cb]<[x1]
Rozwiązać równanie
$$\sqrt{[a]+[b]x-x^2}=x-[c].$$
\zadStop
\rozwStart{Adrianna Stobiecka}{}
Zakładamy, że $[a]+[b]x-x^2\geq0$. Zatem:
$$\Delta=[b]^2-4\cdot[a]\cdot(-1)=[delta1]+[delta2]=[delta]\qquad\Rightarrow\qquad\sqrt{\Delta}=[pdelta]$$
$$x_1=\frac{-[b]-[pdelta]}{-2}=\frac{[lx1]}{-2}=[x1],\qquad x_2=\frac{-[b]+[pdelta]}{-2}=\frac{[lx2]}{-2}=[x2]$$
Funkcja $[a]+[b]x-x^2$ jest parabolą z ramionami skierowanymi w dół i posiada miejsca zerowe w $[x2]$ oraz $[x1]$.Zatem $x\in[[x2],[x1]]$.
\\Lewa strona rozważanego równania jest nieujemna, żeby zatem równanie nie było sprzeczne, musimy dodatkowo założyć, że $x-[c]\geq0$, czyli $x\geq[c]$.
Mamy zatem:
$$x\in[[x2],[x1]]\qquad\land\qquad x\in[[c],\infty)$$
Otrzymujemy stąd założenie, że $x\in[[c],[x1]]$.
\\Wiemy, że dla $a\geq0$, $b\geq0$ oraz $n\in\mathbb{N}$ zachodzi własność
$$a=b\Leftrightarrow a^n=b^n.$$ 
Korzystając z tej własności otrzymujemy:
$$[a]+[b]x-x^2=x^2-[2c]x+[cc]$$
$$x^2+x^2-[2c]x-[b]x+[cc]-[a]=0$$
$$2x^2-[2cb]x=0$$
$$2x(x-[cb])=0$$
$$x_1=0,\qquad x_2=[cb]$$
Uwzględniając założenie $x\in[[c],[x1]]$, otrzymujemy, że rozwiązaniem równania jest $x=[cb].$
\rozwStop
\odpStart
$x=[cb]$
\odpStop
\testStart
A.$[x22]$
B.$x=-[x1]$
C.$x=[x1]$
D.$\frac{1}{2}$
E.$x\in\emptyset$
F.$-\frac{1}{2}$
G.$x=[cb]$
H.$x=-[cb]$
I.$x=[x2]$
\testStop
\kluczStart
G
\kluczStop



\end{document}
