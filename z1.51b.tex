\documentclass[12pt, a4paper]{article}
\usepackage[utf8]{inputenc}
\usepackage{polski}

\usepackage{amsthm}  %pakiet do tworzenia twierdzeń itp.
\usepackage{amsmath} %pakiet do niektórych symboli matematycznych
\usepackage{amssymb} %pakiet do symboli mat., np. \nsubseteq
\usepackage{amsfonts}
\usepackage{graphicx} %obsługa plików graficznych z rozszerzeniem png, jpg
\theoremstyle{definition} %styl dla definicji
\newtheorem{zad}{} 
\title{Multizestaw zadań}
\author{Robert Fidytek}
%\date{\today}
\date{}
\newcounter{liczniksekcji}
\newcommand{\kategoria}[1]{\section{#1}} %olreślamy nazwę kateforii zadań
\newcommand{\zadStart}[1]{\begin{zad}#1\newline} %oznaczenie początku zadania
\newcommand{\zadStop}{\end{zad}}   %oznaczenie końca zadania
%Makra opcjonarne (nie muszą występować):
\newcommand{\rozwStart}[2]{\noindent \textbf{Rozwiązanie (autor #1 , recenzent #2): }\newline} %oznaczenie początku rozwiązania, opcjonarnie można wprowadzić informację o autorze rozwiązania zadania i recenzencie poprawności wykonania rozwiązania zadania
\newcommand{\rozwStop}{\newline}                                            %oznaczenie końca rozwiązania
\newcommand{\odpStart}{\noindent \textbf{Odpowiedź:}\newline}    %oznaczenie początku odpowiedzi końcowej (wypisanie wyniku)
\newcommand{\odpStop}{\newline}                                             %oznaczenie końca odpowiedzi końcowej (wypisanie wyniku)
\newcommand{\testStart}{\noindent \textbf{Test:}\newline} %ewentualne możliwe opcje odpowiedzi testowej: A. ? B. ? C. ? D. ? itd.
\newcommand{\testStop}{\newline} %koniec wprowadzania odpowiedzi testowych
\newcommand{\kluczStart}{\noindent \textbf{Test poprawna odpowiedź:}\newline} %klucz, poprawna odpowiedź pytania testowego (jedna literka): A lub B lub C lub D itd.
\newcommand{\kluczStop}{\newline} %koniec poprawnej odpowiedzi pytania testowego 
\newcommand{\wstawGrafike}[2]{\begin{figure}[h] \includegraphics[scale=#2] {#1} \end{figure}} %gdyby była potrzeba wstawienia obrazka, parametry: nazwa pliku, skala (jak nie wiesz co wpisać, to wpisz 1)

\begin{document}
\maketitle


\kategoria{Wikieł/Z1.51b}
\zadStart{Zadanie z Wikieł Z 1.51 b) moja wersja nr [nrWersji]}
%[a]:[2,3,4,5,6]
%[b]:[2,3,5,6,7,8,10]
%[c]=random.randint(1,10)
%[aar]=[a]*[a]
%[acr]=[a]*[c]
%[bcr]=[b]*[c]
%[ccr]=[c]*[c]
%[x2]=[acr]+[acr]-[b]
%[x2]>0 and [x2]!=-1
Obliczyć iloczyn wielomianów $([a]x^{2}+\sqrt{[b]}x+[c])([a]x^{2}-\sqrt{[b]}x+[c])$.
\zadStop
\rozwStart{Wojciech Przybylski}{Laura Mieczkowska}
$$([a]x^{2}+\sqrt{[b]}x+[c])([a]x^{2}-\sqrt{[b]}x+[c])=$$
$$=[aar]x^{4}-[a]\sqrt{[b]}x^{3}+[acr]x^{2}+[a]\sqrt{[b]}x^{3}-[b]x^{2}+$$
$$+[c]\sqrt{[b]}x+[acr]x^{2}-[c]\sqrt{[b]}x+[ccr]=$$
$$=[aar]x^{4}+[x2]x^{2}+[ccr]$$
\rozwStop
\odpStart
$[aar]x^{4}+[x2]x^{2}+[ccr]$
\odpStop
\testStart
A. $[aar]x^{4}+[x2]x^{2}+[ccr]$\\
B. $[aar]x^{3}+[x2]x^{2}+[ccr]x$\\
C. $[aar]x^{3}+[x2]x^{2}+[ccr]$\\
D. $[a]x^{4}+[x2]x^{2}+[ccr]$\\
E. $[a]x^{3}+[x2]x^{2}+[ccr]x$\\
F. $[a]x^{4}+[x2]x^{3}+[ccr]x^{2}$
\testStop
\kluczStart
A
\kluczStop



\end{document}