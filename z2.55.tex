\documentclass[12pt, a4paper]{article}
\usepackage[utf8]{inputenc}
\usepackage{polski}

\usepackage{amsthm}  %pakiet do tworzenia twierdzeń itp.
\usepackage{amsmath} %pakiet do niektórych symboli matematycznych
\usepackage{amssymb} %pakiet do symboli mat., np. \nsubseteq
\usepackage{amsfonts}
\usepackage{graphicx} %obsługa plików graficznych z rozszerzeniem png, jpg
\theoremstyle{definition} %styl dla definicji
\newtheorem{zad}{} 
\title{Multizestaw zadań}
\author{Robert Fidytek}
%\date{\today}
\date{}
\newcounter{liczniksekcji}
\newcommand{\kategoria}[1]{\section{#1}} %olreślamy nazwę kateforii zadań
\newcommand{\zadStart}[1]{\begin{zad}#1\newline} %oznaczenie początku zadania
\newcommand{\zadStop}{\end{zad}}   %oznaczenie końca zadania
%Makra opcjonarne (nie muszą występować):
\newcommand{\rozwStart}[2]{\noindent \textbf{Rozwiązanie (autor #1 , recenzent #2): }\newline} %oznaczenie początku rozwiązania, opcjonarnie można wprowadzić informację o autorze rozwiązania zadania i recenzencie poprawności wykonania rozwiązania zadania
\newcommand{\rozwStop}{\newline}                                            %oznaczenie końca rozwiązania
\newcommand{\odpStart}{\noindent \textbf{Odpowiedź:}\newline}    %oznaczenie początku odpowiedzi końcowej (wypisanie wyniku)
\newcommand{\odpStop}{\newline}                                             %oznaczenie końca odpowiedzi końcowej (wypisanie wyniku)
\newcommand{\testStart}{\noindent \textbf{Test:}\newline} %ewentualne możliwe opcje odpowiedzi testowej: A. ? B. ? C. ? D. ? itd.
\newcommand{\testStop}{\newline} %koniec wprowadzania odpowiedzi testowych
\newcommand{\kluczStart}{\noindent \textbf{Test poprawna odpowiedź:}\newline} %klucz, poprawna odpowiedź pytania testowego (jedna literka): A lub B lub C lub D itd.
\newcommand{\kluczStop}{\newline} %koniec poprawnej odpowiedzi pytania testowego 
\newcommand{\wstawGrafike}[2]{\begin{figure}[h] \includegraphics[scale=#2] {#1} \end{figure}} %gdyby była potrzeba wstawienia obrazka, parametry: nazwa pliku, skala (jak nie wiesz co wpisać, to wpisz 1)

\begin{document}
\maketitle


\kategoria{Wikieł/Z2.55}
\zadStart{Zadanie z Wikieł Z 2.55 moja wersja nr [nrWersji]}
%[a1]:[2,3,4,5,6,7,8,9,10,11,12,13,14,15]
%[a2]:[1,2,3,4,5,6,7,8,9,10,11,12,13,14,15]
%[b1]=[a1]+1
%[b2]=[a2]+1
%[c1]=[b1]+1
%[c2]=[b2]+1
%[2a1]=2*[a1]
%[2a2]=2*[a2]
%[2b1]=2*[b1]
%[2b2]=2*[b2]
%[2c1]=2*[c1]
%[2c2]=2*[c2]
%[ka1]=[a1]*[a1]
%[ka2]=[a2]*[a2]
%[kb1]=[b1]*[b1]
%[kb2]=[b2]*[b2]
%[kc1]=[c1]*[c1]
%[kc2]=[c2]*[c2]
%[aa12]=[ka1]+[ka2]
%[bb12]=[kb1]+[kb2]
%[cc12]=[kc1]+[kc2]
%[2bb]=[2b1]-[2a1]
%[2b]=[2b2]-[2a2]
%[b12]=[bb12]-[aa12]
%[b]=[b12]/2
%[cb]=int([b])
%[2cc]=[2c1]-[2b1]
%[2c]=[2c2]-[2b2]
%[c12]=[cc12]-[bb12]
%[x]=[2cc]*[2b]
%[xx]=[x]-[2c]
%[y]=[2cc]*[cb]
%[yy]=[c12]-[y]
%[bb]=[yy]/2
%[cbb]=int([bb])
%[a]=[2b]*[cbb]
%[aa]=[cb]+[a]
%[q]=[aa]-[a1]
%[p]=[a2]+[cbb]
%[s]=[p]*[aa]
%[t]=[cbb]*[q]
%[u]=[s]-[t]
%[2bb]>0 and [2b]>0 and [b12]>0 and [2cc]>0 and [2c]>0 and [c12]>0 and [aa]>0 and math.gcd([p],[q])==1 and math.gcd([u],[q])==1
Podać równanie jednej z prostych, do której należy środek okręgu opisanego na trójkącie o wierzchołkach A([a1],[a2]),B([b1],[b2]),C([c1],[c2]).
\zadStop
\rozwStart{Aleksandra Pasińska}{}
1) $$([a1]-a)^2+([a2]-b)^2=r^2$$
$$a^2+b^2-[2a1]a-[2a2]b+[aa12]=r^2$$
2) $$([b1]-a)^2+([b2]-b)^2=r^2$$
$$a^2+b^2-[2b1]a-[2b2]b+[bb12]=r^2$$
3) $$([c1]-a)^2+([c2]-b)^2=r^2$$
$$a^2+b^2-[2c1]a-[2c2]b+[cc12]=r^2$$
1)-2)$$[2bb]a+[2b]b-[b12]=0$$
$$a=-[2b]b+[cb]$$
2)-3)$$[2cc]a+[2c]b-[c12]=0$$
$$[2cc](-[2b]b+[cb])+[2c]b-[c12]=0$$
$$b=-[cbb], a=[aa]$$
$$S([aa],-[cbb])$$
$$\left\{ \begin{array}{ll}
-[cbb]=[aa]a+b\\ 
.[a2]=[a1]a+b 
\end{array} \right.$$
$$b=-[cbb]-[aa]a$$
$$[a2]=[a1]a-[cbb]-[aa]a$$
$$[q]a=-[p]$$
$$a=-\frac{[p]}{[q]}, b=\frac{[u]}{[q]}$$
$$y=-\frac{[p]}{[q]}x+\frac{[u]}{[q]}$$
\rozwStop
\odpStart
$y=-\frac{[p]}{[q]}x+\frac{[u]}{[q]}$\\
\odpStop
\testStart
A.$y=2x+\frac{[u]}{[q]}$
B.$y=-\frac{[p]}{[q]}x-1$
C.$y=x+\frac{[u]}{[q]}$
D.$y=-x+\frac{[u]}{[q]}$
E.$y=\frac{[p]}{[q]}x+\frac{[u]}{[q]}$
F.$y=-\frac{[p]}{[q]}x-\frac{[u]}{[q]}$
G.$y=\frac{[u]}{[q]}$
H.$ y=-\frac{[u]}{[q]}$
I.$y=-\frac{[p]}{[q]}x$
\testStop
\kluczStart
A
\kluczStop



\end{document}