\documentclass[12pt, a4paper]{article}
\usepackage[utf8]{inputenc}
\usepackage{polski}

\usepackage{amsthm}  %pakiet do tworzenia twierdzeń itp.
\usepackage{amsmath} %pakiet do niektórych symboli matematycznych
\usepackage{amssymb} %pakiet do symboli mat., np. \nsubseteq
\usepackage{amsfonts}
\usepackage{graphicx} %obsługa plików graficznych z rozszerzeniem png, jpg
\theoremstyle{definition} %styl dla definicji
\newtheorem{zad}{} 
\title{Multizestaw zadań}
\author{Robert Fidytek}
%\date{\today}
\date{}
\newcounter{liczniksekcji}
\newcommand{\kategoria}[1]{\section{#1}} %olreślamy nazwę kateforii zadań
\newcommand{\zadStart}[1]{\begin{zad}#1\newline} %oznaczenie początku zadania
\newcommand{\zadStop}{\end{zad}}   %oznaczenie końca zadania
%Makra opcjonarne (nie muszą występować):
\newcommand{\rozwStart}[2]{\noindent \textbf{Rozwiązanie (autor #1 , recenzent #2): }\newline} %oznaczenie początku rozwiązania, opcjonarnie można wprowadzić informację o autorze rozwiązania zadania i recenzencie poprawności wykonania rozwiązania zadania
\newcommand{\rozwStop}{\newline}                                            %oznaczenie końca rozwiązania
\newcommand{\odpStart}{\noindent \textbf{Odpowiedź:}\newline}    %oznaczenie początku odpowiedzi końcowej (wypisanie wyniku)
\newcommand{\odpStop}{\newline}                                             %oznaczenie końca odpowiedzi końcowej (wypisanie wyniku)
\newcommand{\testStart}{\noindent \textbf{Test:}\newline} %ewentualne możliwe opcje odpowiedzi testowej: A. ? B. ? C. ? D. ? itd.
\newcommand{\testStop}{\newline} %koniec wprowadzania odpowiedzi testowych
\newcommand{\kluczStart}{\noindent \textbf{Test poprawna odpowiedź:}\newline} %klucz, poprawna odpowiedź pytania testowego (jedna literka): A lub B lub C lub D itd.
\newcommand{\kluczStop}{\newline} %koniec poprawnej odpowiedzi pytania testowego 
\newcommand{\wstawGrafike}[2]{\begin{figure}[h] \includegraphics[scale=#2] {#1} \end{figure}} %gdyby była potrzeba wstawienia obrazka, parametry: nazwa pliku, skala (jak nie wiesz co wpisać, to wpisz 1)

\begin{document}
\maketitle


\kategoria{Wikieł/Z5.19g}
\zadStart{Zadanie z Wikieł Z 5.19 g) moja wersja nr [nrWersji]}
%[a]:[2,3,4,5,6,7,8,9,10,11,12,13,14,15,16,17,18,19,20,21,22,23,24,25,26,27,28,29,30,31,32,33,34,35,36,37,38,39,40,41]
%math.gcd([a],2)==1
Obliczyć granicę $\lim_{x \to 1}\big(\frac{[a]x}{x-1}-\frac{[a]}{lnx}\big)$.
\zadStop
\rozwStart{Jakub Ulrych}{Pascal Nawrocki}
$$\lim_{x \to 1}\big(\frac{[a]x}{x-1}-\frac{[a]}{lnx}\big)$$
$$\lim_{x \to 1}\big(\frac{[a]xlnx-[a](x-1)}{(x-1)lnx}\big)$$
otrzymujemy $\big[\frac{0}{0}\big]$, więc można zastosować twierdzenie de l'Hospitala
$$\lim_{x \to 1}\big(\frac{[a]lnx}{lnx+\frac{x-1}{x}}\big)$$
$$\lim_{x \to 1}\big(\frac{x[a]lnx}{xlnx+x-1}\big)$$
znowu otrzymujemy $\big[\frac{0}{0}\big]$, więc kolejny raz używamy twierdzenia de l'Hospitala
$$\lim_{x \to 1}\big(\frac{[a](lnx+1)}{lnx+2}\big)$$
$$\lim_{x \to 1}\big(\frac{[a]lnx+[a]}{lnx+2}\big)$$
$$\frac{[a]}{2}$$
\rozwStop
\odpStart
$$\frac{[a]}{2}$$
\odpStop
\testStart
A.$\frac{[a]}{2}$
B.$\infty$
C.$-\infty$
D.$[a]$
\testStop
\kluczStart
A
\kluczStop



\end{document}