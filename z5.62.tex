\documentclass[12pt, a4paper]{article}
\usepackage[utf8]{inputenc}
\usepackage{polski}

\usepackage{amsthm}  %pakiet do tworzenia twierdzeń itp.
\usepackage{amsmath} %pakiet do niektórych symboli matematycznych
\usepackage{amssymb} %pakiet do symboli mat., np. \nsubseteq
\usepackage{amsfonts}
\usepackage{graphicx} %obsługa plików graficznych z rozszerzeniem png, jpg
\theoremstyle{definition} %styl dla definicji
\newtheorem{zad}{} 
\title{Multizestaw zadań}
\author{Robert Fidytek}
%\date{\today}
\date{}
\newcounter{liczniksekcji}
\newcommand{\kategoria}[1]{\section{#1}} %olreślamy nazwę kateforii zadań
\newcommand{\zadStart}[1]{\begin{zad}#1\newline} %oznaczenie początku zadania
\newcommand{\zadStop}{\end{zad}}   %oznaczenie końca zadania
%Makra opcjonarne (nie muszą występować):
\newcommand{\rozwStart}[2]{\noindent \textbf{Rozwiązanie (autor #1 , recenzent #2): }\newline} %oznaczenie początku rozwiązania, opcjonarnie można wprowadzić informację o autorze rozwiązania zadania i recenzencie poprawności wykonania rozwiązania zadania
\newcommand{\rozwStop}{\newline}                                            %oznaczenie końca rozwiązania
\newcommand{\odpStart}{\noindent \textbf{Odpowiedź:}\newline}    %oznaczenie początku odpowiedzi końcowej (wypisanie wyniku)
\newcommand{\odpStop}{\newline}                                             %oznaczenie końca odpowiedzi końcowej (wypisanie wyniku)
\newcommand{\testStart}{\noindent \textbf{Test:}\newline} %ewentualne możliwe opcje odpowiedzi testowej: A. ? B. ? C. ? D. ? itd.
\newcommand{\testStop}{\newline} %koniec wprowadzania odpowiedzi testowych
\newcommand{\kluczStart}{\noindent \textbf{Test poprawna odpowiedź:}\newline} %klucz, poprawna odpowiedź pytania testowego (jedna literka): A lub B lub C lub D itd.
\newcommand{\kluczStop}{\newline} %koniec poprawnej odpowiedzi pytania testowego 
\newcommand{\wstawGrafike}[2]{\begin{figure}[h] \includegraphics[scale=#2] {#1} \end{figure}} %gdyby była potrzeba wstawienia obrazka, parametry: nazwa pliku, skala (jak nie wiesz co wpisać, to wpisz 1)

\begin{document}
\maketitle


\kategoria{Wikieł/Z5.62}
\zadStart{Zadanie z Wikieł Z 5.62 ) moja wersja nr [nrWersji]}
%[a]:[4,5,6,7,8,9,10,11,12,13,14,15,16,17,18,19,20]
%[a2]=[a]*2
%[ak]=[a]*[a]
%[x]=int([ak]/[a2])
%[y]=[a]-[x]
%[x1]=[x]-1
%[x2]=[x]+1
%[f1]=round(-1/([x1]**(2))+1/(([a]-[x1])**(2)),3)
%[f2]=round(-1/([x2]**(2))+1/(([a]-[x2])**(2)),3)
%[minimum]=round(1/[x]+1/[y],2)
%[zx]=[x]-2
%[zy]=[y]+2
%[f1]<0 and [f2]>0 and math.gcd([ak],[a2])==[a2] and [x]>0 and [y]>0
Liczbę $[a]$ przedstawić w postaci sumy dwóch składników dodatnich tak, aby suma ich odwrotności była najmniejsza.
\zadStop
\rozwStart{Wojciech Przybylski}{Pascal Nawrocki}
$$x+y=[a] \Rightarrow y=[a]-x$$
$$\frac{1}{x}+\frac{1}{y}=\mbox{minimum}\hspace{5mm} x,y>0,\hspace{3mm}x,y<[a]$$
$$f(x)=\frac{1}{x}+\frac{1}{[a]-x} \Rightarrow f'(x)=\frac{-1}{x^{2}}+\frac{1}{([a]-x)^{2}}$$
$$f'(x)=\frac{-[ak]+[a2]x}{x^{2}\cdot([a]-x)^{2}}$$
Cały mianownik jest większy od 0.
$$-[ak]+[a2]x=0 \Rightarrow x=\frac{[ak]}{[a2]}=[x]$$
$$x_{1}=[x]-1=[x1], \hspace{3mm} x_{2}=[x]+1=[x2]$$
$$f'([x1])=\frac{-1}{[x1]^{2}}+\frac{1}{([a]-[x1])^{2}}=[f1]<0$$
$$f'([x2])=\frac{-1}{[x2]^{2}}+\frac{1}{([a]-[x2])^{2}}=[f2]>0$$
Stąd wiemy, że $x=[x]$ jest ektremum minimum, więc $y=[a]-[x]=[y]$
$$\frac{1}{x}+\frac{1}{y}=\mbox{minimum}=\frac{1}{[x]}+\frac{1}{[y]}=[minimum]$$
\rozwStop
\odpStart
$x=[x], \hspace{3mm} y=[y].$
\odpStop
\testStart
A.$x=[x], \hspace{3mm} y=[y]$.\\
B. $x=[zx], \hspace{3mm} y=[y]$.\\
C. $x=[zx], \hspace{3mm} y=[zy]$.\\
D.$x=[x], \hspace{3mm} y=[zy]$.\\
E. $x=[a2], \hspace{3mm} y=[a2]$.\\
F. Nie istnieją takie składniki spełniające to zadanie.
\testStop
\kluczStart
A
\kluczStop



\end{document}