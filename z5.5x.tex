\documentclass[12pt, a4paper]{article}
\usepackage[utf8]{inputenc}
\usepackage{polski}

\usepackage{amsthm}  %pakiet do tworzenia twierdzeń itp.
\usepackage{amsmath} %pakiet do niektórych symboli matematycznych
\usepackage{amssymb} %pakiet do symboli mat., np. \nsubseteq
\usepackage{amsfonts}
\usepackage{graphicx} %obsługa plików graficznych z rozszerzeniem png, jpg
\theoremstyle{definition} %styl dla definicji
\newtheorem{zad}{} 
\title{Multizestaw zadań}
\author{Robert Fidytek}
%\date{\today}
\date{}
\newcounter{liczniksekcji}
\newcommand{\kategoria}[1]{\section{#1}} %olreślamy nazwę kateforii zadań
\newcommand{\zadStart}[1]{\begin{zad}#1\newline} %oznaczenie początku zadania
\newcommand{\zadStop}{\end{zad}}   %oznaczenie końca zadania
%Makra opcjonarne (nie muszą występować):
\newcommand{\rozwStart}[2]{\noindent \textbf{Rozwiązanie (autor #1 , recenzent #2): }\newline} %oznaczenie początku rozwiązania, opcjonarnie można wprowadzić informację o autorze rozwiązania zadania i recenzencie poprawności wykonania rozwiązania zadania
\newcommand{\rozwStop}{\newline}                                            %oznaczenie końca rozwiązania
\newcommand{\odpStart}{\noindent \textbf{Odpowiedź:}\newline}    %oznaczenie początku odpowiedzi końcowej (wypisanie wyniku)
\newcommand{\odpStop}{\newline}                                             %oznaczenie końca odpowiedzi końcowej (wypisanie wyniku)
\newcommand{\testStart}{\noindent \textbf{Test:}\newline} %ewentualne możliwe opcje odpowiedzi testowej: A. ? B. ? C. ? D. ? itd.
\newcommand{\testStop}{\newline} %koniec wprowadzania odpowiedzi testowych
\newcommand{\kluczStart}{\noindent \textbf{Test poprawna odpowiedź:}\newline} %klucz, poprawna odpowiedź pytania testowego (jedna literka): A lub B lub C lub D itd.
\newcommand{\kluczStop}{\newline} %koniec poprawnej odpowiedzi pytania testowego 
\newcommand{\wstawGrafike}[2]{\begin{figure}[h] \includegraphics[scale=#2] {#1} \end{figure}} %gdyby była potrzeba wstawienia obrazka, parametry: nazwa pliku, skala (jak nie wiesz co wpisać, to wpisz 1)

\begin{document}
\maketitle


\kategoria{Wikieł/Z5.5x}
\zadStart{Zadanie z Wikieł Z 5.5 x) moja wersja nr [nrWersji]}
%[a]:[2,3,4,5,6,7,8,9]
%[b]:[2,3,4,5,6,7,8,9]
%[c]=random.randint(2,10)
%[d]=random.randint(2,10)
%[e]=random.randint(2,10)
%[f]=random.randint(2,10)
%[ae]=[a]*[e]
%[bd]=[b]*[d]
%[af]=2*[a]*[f]
%[cd]=2*[c]*[d]
%[bf]=[b]*[f]
%[ce]=[c]*[e]
%[bdae]=[bd]-[ae]
%[afcd]=-1*([af]-[cd])
%[cebf]=[ce]-[bf]
%[bdae]>1 and [afcd]>1 and [cebf]>0
Wyznacz pochodną funkcji \\ $f(x)=\sqrt{\frac{[a]x^2-[b]x+[c]}{[d]x^2-[e]x+[f]}} $.
\zadStop
\rozwStart{Joanna Świerzbin}{}
$$ f(x)=\sqrt{\frac{[a]x^2-[b]x+[c]}{[d]x^2-[e]x+[f]}}  $$
$$ f'(x)=\left(\sqrt{\frac{[a]x^2-[b]x+[c]}{[d]x^2-[e]x+[f]}}\right)' =  $$
$$ = \frac{1}{2\sqrt{\frac{[a]x^2-[b]x+[c]}{[d]x^2-[e]x+[f]}}} \left(\frac{[a]x^2-[b]x+[c]}{[d]x^2-[e]x+[f]}\right)' = $$
$$ = \frac{\sqrt{[d]x^2-[e]x+[f]}}{2\sqrt{[a]x^2-[b]x+[c]}} \frac{(2\cdot[a]x-[b])([d]x^2-[e]x+[f])-([a]x^2-[b]x+[c])(2\cdot[d]x-[e])}{([d]x^2-[e]x+[f])^2} = $$
$$ = \frac{\sqrt{[d]x^2-[e]x+[f]} ((2\cdot[a]x-[b])([d]x^2-[e]x+[f])-([a]x^2-[b]x+[c])(2\cdot[d]x-[e]))}{2\sqrt{[a]x^2-[b]x+[c]} ([d]x^2-[e]x+[f])^2} = $$
$$ = \frac{\sqrt{[d]x^2-[e]x+[f]} (x^2(-[a]\cdot[e]+[b]\cdot[d])+x(2\cdot[a]\cdot[f]-2\cdot[c]\cdot[d])-[b]\cdot[f]+[c]\cdot[e])}{2\sqrt{[a]x^2-[b]x+[c]} ([d]x^2-[e]x+[f])^2} = $$
$$ = \frac{\sqrt{[d]x^2-[e]x+[f]} (x^2(-[ae]+[bd])+x([af]-[cd])-[bf]+[ce])}{2\sqrt{[a]x^2-[b]x+[c]} ([d]x^2-[e]x+[f])^2} = $$
$$ = \frac{\sqrt{[d]x^2-[e]x+[f]} ([bdae]x^2-[afcd]x+[cebf])}{2\sqrt{[a]x^2-[b]x+[c]} ([d]x^2-[e]x+[f])^2}$$
\rozwStop
\odpStart
$ f'(x) = \frac{\sqrt{[d]x^2-[e]x+[f]} ([bdae]x^2-[afcd]x+[cebf])}{2\sqrt{[a]x^2-[b]x+[c]} ([d]x^2-[e]x+[f])^2}$
\odpStop
\testStart
A. $ f'(x) = \frac{\sqrt{[d]x^2-[e]x+[f]} ([bdae]x^2-[afcd]x+[cebf])}{2\sqrt{[a]x^2-[b]x+[c]} ([d]x^2-[e]x+[f])^2}$\\
B. $ f'(x) = \frac{1}{2\sqrt{[a]x^2-[b]x+[c]} ([d]x^2-[e]x+[f])^2}$\\
C. $ f'(x) = \frac{\sqrt{[d]x^2-[e]x+[f]} }{2\sqrt{[a]x^2-[b]x+[c]} ([d]x^2-[e]x+[f])^2}$\\
D. $ f'(x) = \frac{\sqrt{[d]x^2-[e]x+[f]} ([bdae]x^2-[afcd]x+[cebf])}{\sqrt{[a]x^2-[b]x+[c]} ([d]x^2-[e]x+[f])^2}$\\
E. $ f'(x) = \frac{\sqrt{[d]x^2-[e]x+[f]} ([bdae]x^2-[afcd]x+[cebf])}{2\sqrt{[a]x^2-[b]x+[c]} }$\\
F. $ f'(x) = \frac{\sqrt{[d]x^2-[e]x+[f]} ([bdae]x^2-[afcd]x+[cebf])}{2\sqrt{[a]x^2-[b]x+[c]} ([d]x^2-[e]x+[f])}$
\testStop
\kluczStart
A
\kluczStop
\end{document}