\documentclass[12pt, a4paper]{article}
\usepackage[utf8]{inputenc}
\usepackage{polski}

\usepackage{amsthm}  %pakiet do tworzenia twierdzeń itp.
\usepackage{amsmath} %pakiet do niektórych symboli matematycznych
\usepackage{amssymb} %pakiet do symboli mat., np. \nsubseteq
\usepackage{amsfonts}
\usepackage{graphicx} %obsługa plików graficznych z rozszerzeniem png, jpg
\theoremstyle{definition} %styl dla definicji
\newtheorem{zad}{} 
\title{Multizestaw zadań}
\author{Robert Fidytek}
%\date{\today}
\date{}
\newcounter{liczniksekcji}
\newcommand{\kategoria}[1]{\section{#1}} %olreślamy nazwę kateforii zadań
\newcommand{\zadStart}[1]{\begin{zad}#1\newline} %oznaczenie początku zadania
\newcommand{\zadStop}{\end{zad}}   %oznaczenie końca zadania
%Makra opcjonarne (nie muszą występować):
\newcommand{\rozwStart}[2]{\noindent \textbf{Rozwiązanie (autor #1 , recenzent #2): }\newline} %oznaczenie początku rozwiązania, opcjonarnie można wprowadzić informację o autorze rozwiązania zadania i recenzencie poprawności wykonania rozwiązania zadania
\newcommand{\rozwStop}{\newline}                                            %oznaczenie końca rozwiązania
\newcommand{\odpStart}{\noindent \textbf{Odpowiedź:}\newline}    %oznaczenie początku odpowiedzi końcowej (wypisanie wyniku)
\newcommand{\odpStop}{\newline}                                             %oznaczenie końca odpowiedzi końcowej (wypisanie wyniku)
\newcommand{\testStart}{\noindent \textbf{Test:}\newline} %ewentualne możliwe opcje odpowiedzi testowej: A. ? B. ? C. ? D. ? itd.
\newcommand{\testStop}{\newline} %koniec wprowadzania odpowiedzi testowych
\newcommand{\kluczStart}{\noindent \textbf{Test poprawna odpowiedź:}\newline} %klucz, poprawna odpowiedź pytania testowego (jedna literka): A lub B lub C lub D itd.
\newcommand{\kluczStop}{\newline} %koniec poprawnej odpowiedzi pytania testowego 
\newcommand{\wstawGrafike}[2]{\begin{figure}[h] \includegraphics[scale=#2] {#1} \end{figure}} %gdyby była potrzeba wstawienia obrazka, parametry: nazwa pliku, skala (jak nie wiesz co wpisać, to wpisz 1)

\begin{document}
\maketitle


\kategoria{Wikieł/Z1.69}
\zadStart{Zadanie z Wikieł Z 1.69) moja wersja nr [nrWersji]}
%[a]:[2,3,4,5,6,7,8,9]
%[b]:[2,3,4,5,6,7,8,9]
%[a]=random.randint(2,15)
%[b]=random.randint(2,15)
%[bma]=[b]-[a]
%[ab]=[a]*[b]
%[ab1]=[ab]+1
%[delta]=pow([bma],2)-(4*(-[ab1]))
%[sdelta]=pow([delta],1/2)
%[w1]=(-[bma]-[sdelta])/(2)
%[w2]=(-[bma]+[sdelta])/(2)
%[p1]=-7/2
%[p2]=5/2
%[b]>[a] and [w1]<[p1] and [w2]>[p2] and [sdelta].is_integer()==False
Sprawdzić, czy zbiór $A=\big(-\frac{7}{2},\frac{5}{2}\big)$ zawiera zbiór rozwiązań poniższej\\ nierówności.
$$\frac{x-[a]}{x+[b]}\leq\frac{1}{(x+[b])^{2}}$$
\zadStop
\rozwStart{Jakub Ulrych}{Pascal Nawrocki}
dziedzina:$$x+[b]\neq0\Rightarrow x\in \mathbb{R}-\{-[b]\}$$
rozwiązanie:$$\frac{x-[a]}{x+[b]}\leq\frac{1}{(x+[b])^{2}}$$
$$\frac{(x-[a])(x+[b])}{(x+[b])^{2}}\leq\frac{1}{(x+[b])^{2}}$$
$$(x-[a])(x+[b])\leq1$$
$$x^{2}+[bma]x-[ab1]\leq0$$
$$\Delta=[delta]\Rightarrow \sqrt{\Delta}=\sqrt{[delta]}$$
$$x_{1}=\frac{-[bma]+\sqrt{[delta]}}{2} \vee x_{2}=\frac{-[bma]-\sqrt{[delta]}}{2}$$
$$\big(x-\frac{-[bma]+\sqrt{[delta]}}{2}\big)\cdot\big(x-\frac{-[bma]-\sqrt{[delta]}}{2}\big)\leq0$$
$$\textbf{1)}\big(x-\frac{-[bma]+\sqrt{[delta]}}{2}\big)\geq0 \land \big(x-\frac{-[bma]-\sqrt{[delta]}}{2}\big)\leq0$$ $$\vee$$ $$\textbf{2)}\big(x-\frac{-[bma]+\sqrt{[delta]}}{2}\big)\leq0 \land \big(x-\frac{-[bma]-\sqrt{[delta]}}{2}\big)\geq0$$
$$\textbf{1)}x\in\emptyset \vee \textbf{2)}x\in\bigg[\frac{-[bma]-\sqrt{[delta]}}{2},\frac{-[bma]+\sqrt{[delta]}}{2}\bigg]\land \text{ dziedzina: }x\in \mathbb{R}-\{-[b]\}$$
$$\textbf{ 1) }\vee\textbf{ 2) }\land\text{ dziedzina }\notin A$$
\rozwStop
\odpStart
$$\text{nie zawiera}$$
\odpStop
\testStart
A.$\text{Nie zawiera}$
B.$\text{Zawiera}$
\testStop
\kluczStart
A
\kluczStop
\end{document}