\documentclass[12pt, a4paper]{article}
\usepackage[utf8]{inputenc}
\usepackage{polski}

\usepackage{amsthm}  %pakiet do tworzenia twierdzeń itp.
\usepackage{amsmath} %pakiet do niektórych symboli matematycznych
\usepackage{amssymb} %pakiet do symboli mat., np. \nsubseteq
\usepackage{amsfonts}
\usepackage{graphicx} %obsługa plików graficznych z rozszerzeniem png, jpg
\theoremstyle{definition} %styl dla definicji
\newtheorem{zad}{} 
\title{Multizestaw zadań}
\author{Jacek Jabłoński}
%\date{\today}
\date{}
\newcounter{liczniksekcji}
\newcommand{\kategoria}[1]{\section{#1}} %olreślamy nazwę kateforii zadań
\newcommand{\zadStart}[1]{\begin{zad}#1\newline} %oznaczenie początku zadania
\newcommand{\zadStop}{\end{zad}}   %oznaczenie końca zadania
%Makra opcjonarne (nie muszą występować):
\newcommand{\rozwStart}[2]{\noindent \textbf{Rozwiązanie (autor #1 , recenzent #2): }\newline} %oznaczenie początku rozwiązania, opcjonarnie można wprowadzić informację o autorze rozwiązania zadania i recenzencie poprawności wykonania rozwiązania zadania
\newcommand{\rozwStop}{\newline}                                            %oznaczenie końca rozwiązania
\newcommand{\odpStart}{\noindent \textbf{Odpowiedź:}\newline}    %oznaczenie początku odpowiedzi końcowej (wypisanie wyniku)
\newcommand{\odpStop}{\newline}                                             %oznaczenie końca odpowiedzi końcowej (wypisanie wyniku)
\newcommand{\testStart}{\noindent \textbf{Test:}\newline} %ewentualne możliwe opcje odpowiedzi testowej: A. ? B. ? C. ? D. ? itd.
\newcommand{\testStop}{\newline} %koniec wprowadzania odpowiedzi testowych
\newcommand{\kluczStart}{\noindent \textbf{Test poprawna odpowiedź:}\newline} %klucz, poprawna odpowiedź pytania testowego (jedna literka): A lub B lub C lub D itd.
\newcommand{\kluczStop}{\newline} %koniec poprawnej odpowiedzi pytania testowego 
\newcommand{\wstawGrafike}[2]{\begin{figure}[h] \includegraphics[scale=#2] {#1} \end{figure}} %gdyby była potrzeba wstawienia obrazka, parametry: nazwa pliku, skala (jak nie wiesz co wpisać, to wpisz 1)

\begin{document}
\maketitle


\kategoria{Wikieł/z1.81d}
\zadStart{Zadanie z Wikieł z.1.81d) moja wersja nr [nrWersji]}
%[p1]:[1,2,3,4,5,6]
%[p2]:[2,3,4,5,6,7,8]
%[p3]:[2,3,4,5,6,7,8]
%[Delta]=1+(4*[p3])
%[PDelta]=int(math.pow([Delta],(1/2)))
%[x1]=int((-1-[PDelta])/2)
%[x2]=int((-1+[PDelta])/2)
%[c1]=math.sqrt([Delta])
%[c2]=math.isqrt([Delta])
%[f1]=[x1]+1
%[f2]=[x1]+2
%[f3]=[x1]+3
%[f4]=[x1]+4
%[f5]=[x2]+1
%[f6]=[x2]+2
%[f7]=[x2]+3
%[f8]=[x2]+4
%[p1]<[p2] and [Delta]>0 and not([c1]!=[c2])
Wyznaczyć wartości zmiennej x, dla których funkcje f i g mają równe wartości.
a)$f(x)=(\frac{[p1]}{[p2]})^{x^2} \ \ \ \ g(x)=(\frac{[p1]}{[p2]})^{[p3]-x}$
\zadStop
\rozwStart{Jacek Jabłoński}{}
$$f(x)=f(g)$$
$$(\frac{[p1]}{[p2]})^{x^2} = (\frac{[p1]}{[p2]})^{[p3]-x}$$
$$x^2 = [p3]-x$$
$$x^2 + x - [p3] = 0$$
$$\Delta = [Delta] $$
$$\sqrt{\Delta} = [PDelta] $$
$$x_1 = \frac{-1-[PDelta]}{2} = [x1] $$
$$x_2 = \frac{-1+[PDelta]}{2} = [x2] $$
$$x = [x1] \ \ lub \ \ x = [x2]$$
\rozwStop
\odpStart
$$x = [x1] \ \ lub \ \ x = [x2]$$
\odpStop
\testStart
A. $$x = [x1] \ \ lub \ \ x = [x2]$$
B. $$x = [f1] \ \ lub \ \ x = [f5]$$
C. $$x = [f2] \ \ lub \ \ x = [f6]$$
D. $$x = [f3] \ \ lub \ \ x = [f7]$$
E. $$x = [f4] \ \ lub \ \ x = [f8]$$
F. $$x = [x1] \ \ lub \ \ x = [f5]$$
G. $$x = [x1] \ \ lub \ \ x = [f6]$$
H. $$x = [x1]$$
I. $$x = [x2]$$
\testStop
\kluczStart
A
\kluczStop



\end{document}