\documentclass[12pt, a4paper]{article}
\usepackage[utf8]{inputenc}
\usepackage{polski}
\usepackage{amsthm}  %pakiet do tworzenia twierdzeń itp.
\usepackage{amsmath} %pakiet do niektórych symboli matematycznych
\usepackage{amssymb} %pakiet do symboli mat., np. \nsubseteq
\usepackage{amsfonts}
\usepackage{graphicx} %obsługa plików graficznych z rozszerzeniem png, jpg
\theoremstyle{definition} %styl dla definicji
\newtheorem{zad}{} 
\title{Multizestaw zadań}
\author{Radosław Grzyb}
%\date{\today}
\date{}
\newcounter{liczniksekcji}
\newcommand{\kategoria}[1]{\section{#1}} %olreślamy nazwę kateforii zadań
\newcommand{\zadStart}[1]{\begin{zad}#1\newline} %oznaczenie początku zadania
\newcommand{\zadStop}{\end{zad}}   %oznaczenie końca zadania
%Makra opcjonarne (nie muszą występować):
\newcommand{\rozwStart}[2]{\noindent \textbf{Rozwiązanie (autor #1 , recenzent #2): }\newline} %oznaczenie początku rozwiązania, opcjonarnie można wprowadzić informację o autorze rozwiązania zadania i recenzencie poprawności wykonania rozwiązania zadania
\newcommand{\rozwStop}{\newline}                                            %oznaczenie końca rozwiązania
\newcommand{\odpStart}{\noindent \textbf{Odpowiedź:}\newline}    %oznaczenie początku odpowiedzi końcowej (wypisanie wyniku)
\newcommand{\odpStop}{\newline}                                             %oznaczenie końca odpowiedzi końcowej (wypisanie wyniku)
\newcommand{\testStart}{\noindent \textbf{Test:}\newline} %ewentualne możliwe opcje odpowiedzi testowej: A. ? B. ? C. ? D. ? itd.
\newcommand{\testStop}{\newline} %koniec wprowadzania odpowiedzi testowych
\newcommand{\kluczStart}{\noindent \textbf{Test poprawna odpowiedź:}\newline} %klucz, poprawna odpowiedź pytania testowego (jedna literka): A lub B lub C lub D itd.
\newcommand{\kluczStop}{\newline} %koniec poprawnej odpowiedzi pytania testowego 
\newcommand{\wstawGrafike}[2]{\begin{figure}[h] \includegraphics[scale=#2] {#1} \end{figure}} %gdyby była potrzeba wstawienia obrazka, parametry: nazwa pliku, skala (jak nie wiesz co wpisać, to wpisz 1)
\begin{document}
\maketitle
\kategoria{Wikieł/Z1.95}
\zadStart{Zadanie z Wikieł Z 1.95 moja wersja nr [nrWersji]}
%[p1]:[2,4,5,10,20,50,100]
%[wp1]=int(100/[p1])
%[wp2]=int(1000/[p1])
Wyznaczyć, dla jakich wartości $a$ prawdziwa jest poniższa równość.
$$\log([p1]a)^2=(\log[p1]a)^2$$
\zadStop
\rozwStart{Radosław Grzyb}{}
Zauważmy, że równość, będzie zachodzić dla $[p1]a=1 \implies a=\frac{1}{[p1]}$. Wówczas:
$$\log([p1]\cdot\frac{1}{[p1]})^2=(\log[p1]\cdot\frac{1}{[p1]})^2$$
$$\log(1)^2=(\log1)^2$$
$$0=0$$
Ale to nie jest jedyne rozwiązanie... Przekształćmy nieco naszą lewą stronę równania.
$$\log([p1]a)^2=2\cdot\log([p1]a)=\log100\cdot \log([p1]a)$$
Otrzymane wyrażenie będzie równe $(\log[p1]a)^2$ o ile $\log100=\log([p1]a)$
$$\log([p1]a)=\log100\implies[p1]a=100\implies a=[wp1]$$
\rozwStop
\odpStart
$$a=\frac{1}{[p1]}\lor[wp1]$$
\odpStop
\testStart
A.$$a=\frac{1}{[p1]}\lor[wp1]$$
B.$$a=\frac{1}{[p1]}$$
C.$$a=[wp1]$$
D.$$a=\frac{1}{[p1]}\lor[wp2]$$
\testStop
\kluczStart
A
\kluczStop
\end{document}