\documentclass[12pt, a4paper]{article}
\usepackage[utf8]{inputenc}
\usepackage{polski}

\usepackage{amsthm}  %pakiet do tworzenia twierdzeń itp.
\usepackage{amsmath} %pakiet do niektórych symboli matematycznych
\usepackage{amssymb} %pakiet do symboli mat., np. \nsubseteq
\usepackage{amsfonts}
\usepackage{graphicx} %obsługa plików graficznych z rozszerzeniem png, jpg
\theoremstyle{definition} %styl dla definicji
\newtheorem{zad}{} 
\title{Multizestaw zadań}
\author{Robert Fidytek}
%\date{\today}
\date{}
\newcounter{liczniksekcji}
\newcommand{\kategoria}[1]{\section{#1}} %olreślamy nazwę kateforii zadań
\newcommand{\zadStart}[1]{\begin{zad}#1\newline} %oznaczenie początku zadania
\newcommand{\zadStop}{\end{zad}}   %oznaczenie końca zadania
%Makra opcjonarne (nie muszą występować):
\newcommand{\rozwStart}[2]{\noindent \textbf{Rozwiązanie (autor #1 , recenzent #2): }\newline} %oznaczenie początku rozwiązania, opcjonarnie można wprowadzić informację o autorze rozwiązania zadania i recenzencie poprawności wykonania rozwiązania zadania
\newcommand{\rozwStop}{\newline}                                            %oznaczenie końca rozwiązania
\newcommand{\odpStart}{\noindent \textbf{Odpowiedź:}\newline}    %oznaczenie początku odpowiedzi końcowej (wypisanie wyniku)
\newcommand{\odpStop}{\newline}                                             %oznaczenie końca odpowiedzi końcowej (wypisanie wyniku)
\newcommand{\testStart}{\noindent \textbf{Test:}\newline} %ewentualne możliwe opcje odpowiedzi testowej: A. ? B. ? C. ? D. ? itd.
\newcommand{\testStop}{\newline} %koniec wprowadzania odpowiedzi testowych
\newcommand{\kluczStart}{\noindent \textbf{Test poprawna odpowiedź:}\newline} %klucz, poprawna odpowiedź pytania testowego (jedna literka): A lub B lub C lub D itd.
\newcommand{\kluczStop}{\newline} %koniec poprawnej odpowiedzi pytania testowego 
\newcommand{\wstawGrafike}[2]{\begin{figure}[h] \includegraphics[scale=#2] {#1} \end{figure}} %gdyby była potrzeba wstawienia obrazka, parametry: nazwa pliku, skala (jak nie wiesz co wpisać, to wpisz 1)

\begin{document}
\maketitle


\kategoria{Wikieł/Z5.46}
\zadStart{Zadanie z Wikieł Z 5.46 ) moja wersja nr [nrWersji]}
%[a]:[2,3,4,5,6,7,8]
%[b]:[2,3,4,5,6,7,8]
%[c]:[1,2,3,4,5,6,7,8]
%[d]:[2,3,4,5,6,7,8]
%[e]=random.randint(1,10)
%[a3]=[a]*3
%[d2]=[d]*[d]
%[a1]=[a3]*[d2]-[b]
%[b1]=[e]-[a1]*[d]
%[absb1]=abs([b1])
%[b2]=round([e]+[d]*(1/[a1]),2)
%[a1]>0 and [b1]<0 and [a1]!=1 
Napisać równanie stycznej i normalnej do krzywej $y=[a]x^{3}-[b]x+[c]$ \\w punkcie $M([d],[e])$.
\zadStop
\rozwStart{Wojciech Przybylski}{Maja Szabłowska}
$$ y=[a]x^{3}-[b]x+[c] \hspace{5mm} M([d],[e])$$
$$ f'(x)=[a3]x^{2}-[b]$$
$$y=a_{1}x+b_{1} \mbox{ - prosta styczna}$$
$$y=a_{2}x+b_{2} \mbox{ - prosta normalna }$$
$$a_{1}=-\frac{1}{a_{2}}$$
$$a_{1}=f'([d])=[a3]\cdot[d2]-[b]=[a1]$$
$$a_{2}=-\frac{1}{[a1]}$$
$$b_{1}=[e]-[a1]\cdot[d]=[b1]$$
$$b_{2}=[e]+\frac{1}{[a1]}\cdot[d]=[b2]$$
$$y=[a1]x-[absb1] \mbox{ - prosta styczna, }y=-\frac{1}{[a1]}\cdot x+[b2] \mbox{ - prosta normalna }$$
\rozwStop
\odpStart
$y=[a1]x-[absb1] \mbox{ - prosta styczna, }y=-\frac{1}{[a1]}\cdot x+[b2] \mbox{ - prosta normalna }$
\odpStop
\testStart
A. $y=[a1]x-[absb1] \mbox{ - prosta styczna, }y=-\frac{1}{[a1]}\cdot x+[b2] \mbox{ - prosta normalna }$\\
B. $y=[a1]x^{2}+[a1] \mbox{ - prosta styczna, }y=-\frac{1}{[a1]}\cdot x+\frac{1}{[a1]} \mbox{ - prosta normalna }$\\
C. $y=[a1]x+[a1] \mbox{ - prosta styczna, }y=-\frac{1}{[a1]}\cdot x \mbox{ - prosta normalna }$\\
D. $y=[a1]x \mbox{ - prosta styczna, }y=\frac{1}{[a1]}\cdot x+[b2] \mbox{ - prosta normalna }$\\
E. Prosta styczna i prosta normalna do krzywej $f(x)$ nie istnieją.
\testStop
\kluczStart
A
\kluczStop



\end{document}