\documentclass[12pt, a4paper]{article}
\usepackage[utf8]{inputenc}
\usepackage{polski}

\usepackage{amsthm}  %pakiet do tworzenia twierdzeń itp.
\usepackage{amsmath} %pakiet do niektórych symboli matematycznych
\usepackage{amssymb} %pakiet do symboli mat., np. \nsubseteq
\usepackage{amsfonts}
\usepackage{graphicx} %obsługa plików graficznych z rozszerzeniem png, jpg
\theoremstyle{definition} %styl dla definicji
\newtheorem{zad}{} 
\title{Multizestaw zadań}
\author{Robert Fidytek}
%\date{\today}
\date{}
\newcounter{liczniksekcji}
\newcommand{\kategoria}[1]{\section{#1}} %olreślamy nazwę kateforii zadań
\newcommand{\zadStart}[1]{\begin{zad}#1\newline} %oznaczenie początku zadania
\newcommand{\zadStop}{\end{zad}}   %oznaczenie końca zadania
%Makra opcjonarne (nie muszą występować):
\newcommand{\rozwStart}[2]{\noindent \textbf{Rozwiązanie (autor #1 , recenzent #2): }\newline} %oznaczenie początku rozwiązania, opcjonarnie można wprowadzić informację o autorze rozwiązania zadania i recenzencie poprawności wykonania rozwiązania zadania
\newcommand{\rozwStop}{\newline}                                            %oznaczenie końca rozwiązania
\newcommand{\odpStart}{\noindent \textbf{Odpowiedź:}\newline}    %oznaczenie początku odpowiedzi końcowej (wypisanie wyniku)
\newcommand{\odpStop}{\newline}                                             %oznaczenie końca odpowiedzi końcowej (wypisanie wyniku)
\newcommand{\testStart}{\noindent \textbf{Test:}\newline} %ewentualne możliwe opcje odpowiedzi testowej: A. ? B. ? C. ? D. ? itd.
\newcommand{\testStop}{\newline} %koniec wprowadzania odpowiedzi testowych
\newcommand{\kluczStart}{\noindent \textbf{Test poprawna odpowiedź:}\newline} %klucz, poprawna odpowiedź pytania testowego (jedna literka): A lub B lub C lub D itd.
\newcommand{\kluczStop}{\newline} %koniec poprawnej odpowiedzi pytania testowego 
\newcommand{\wstawGrafike}[2]{\begin{figure}[h] \includegraphics[scale=#2] {#1} \end{figure}} %gdyby była potrzeba wstawienia obrazka, parametry: nazwa pliku, skala (jak nie wiesz co wpisać, to wpisz 1)

\begin{document}
\maketitle


\kategoria{Wikieł/Z5.18 i}
\zadStart{Zadanie z Wikieł Z 5.18 i) moja wersja nr [nrWersji]}
%[a]:[2,3,4,5,6,7,8,9]
%[b]:[2,3,4,5,6,7,8,9]
%[c]=random.randint(3,10)
%[b]!=0
Oblicz granicę $\lim_{x \rightarrow \infty} \left(\frac{[a]}{\pi} tg^{-1}([b]x) \right)^{x^{[c]}}$.
\zadStop
\rozwStart{Joanna Świerzbin}{}
$$ \lim_{x \rightarrow \infty} \left(\frac{[a]}{\pi} tg^{-1}([b]x) \right)^{x^{[c]}} =\lim_{x \rightarrow \infty}  e^{\ln \left(\frac{[a]}{\pi} tg^{-1}([b]x) \right)^{x^{[c]}}} = \lim_{x \rightarrow \infty}  e^{ {x^{[c]}} \ln \left(\frac{[a]}{\pi} tg^{-1}([b]x) \right)} = $$
$$=  e^{\lim_{x \rightarrow \infty} {x^{[c]}} \ln \left(\frac{[a]}{\pi} tg^{-1}([b]x) \right)} $$
Policzmy $ \lim_{x \rightarrow \infty} {x^{[c]}} \ln \left(\frac{[a]}{\pi} tg^{-1}([b]x) \right)$.
$$\lim_{x \rightarrow \infty} {x^{[c]}} \ln \left(\frac{[a]}{\pi} tg^{-1}([b]x) \right)= \lim_{x \rightarrow \infty} \frac{\ln \left(\frac{[a]}{\pi} tg^{-1}([b]x) \right)}{\frac{1}{x^{[c]}}}$$
Otrzymujemy $ \left[ \frac{0}{0} \right] $ więc możemy skorzystać z twierdzenia de l'Hospitala.
$$\lim_{x \rightarrow \infty} \frac{\left(\ln \left(\frac{[a]}{\pi} tg^{-1}([b]x) \right)\right)'}{\left(\frac{1}{x^{[c]}}\right)'}= \lim_{x \rightarrow \infty} \frac{\frac{1}{\frac{[a]}{\pi} tg^{-1}([b]x)}\left( \frac{[a]}{\pi} tg^{-1}([b]x) \right)'}{\frac{-[c]}{x^{[c]+1}}}=$$
$$= \lim_{x \rightarrow \infty} \frac{\frac{\pi \frac{[a]}{\pi} \frac{1}{x^2+1}([b]x)'}{[a] tg^{-1}([b]x)}}{\frac{-[c]}{x^{[c]+1}}}=
 \lim_{x \rightarrow \infty} \frac{\frac{[b]}{(x^2+1) tg^{-1}([b]x)}}{\frac{-[c]}{x^{[c]+1}}}=
\lim_{x \rightarrow \infty}\frac{-[b]x^{[c]+1}}{[c](x^2+1) tg^{-1}([b]x)}=$$
$$=-\frac{[b]}{[c]}\lim_{x \rightarrow \infty}\frac{x^{[c]+1}}{x^2 tg^{-1}([b]x) + tg^{-1}([b]x)}=
=-\frac{[b]}{[c]}\lim_{x \rightarrow \infty}\frac{x^{[c]-1}}{tg^{-1}([b]x) + \frac{ tg^{-1}([b]x)}{x^2}} = -\infty$$
Podstawmy do początkowego przykładu.
$$e^{\lim_{x \rightarrow \infty} {x^{[c]}} \ln \left(\frac{[a]}{\pi} tg^{-1}([b]x) \right)} = e^{-\infty} =0 $$
\rozwStop
\odpStart
$ \lim_{x \rightarrow \infty} \left(\frac{[a]}{\pi} tg^{-1}([b]x) \right)^{x^{[c]}}= 0 $
\odpStop
\testStart
A. $ \lim_{x \rightarrow \infty} \left(\frac{[a]}{\pi} tg^{-1}([b]x) \right)^{x^{[c]}}= 0 $\\
B. $ \lim_{x \rightarrow \infty} \left(\frac{[a]}{\pi} tg^{-1}([b]x) \right)^{x^{[c]}}= 1 $\\
C. $ \lim_{x \rightarrow \infty} \left(\frac{[a]}{\pi} tg^{-1}([b]x) \right)^{x^{[c]}}= e $\\
D. $ \lim_{x \rightarrow \infty} \left(\frac{[a]}{\pi} tg^{-1}([b]x) \right)^{x^{[c]}}= [a] $\\
E. $ \lim_{x \rightarrow \infty} \left(\frac{[a]}{\pi} tg^{-1}([b]x) \right)^{x^{[c]}}= \infty $\\
F. $ \lim_{x \rightarrow \infty} \left(\frac{[a]}{\pi} tg^{-1}([b]x) \right)^{x^{[c]}}= -\infty $
\testStop
\kluczStart
A
\kluczStop



\end{document}