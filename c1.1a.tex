\documentclass[12pt, a4paper]{article}
\usepackage[utf8]{inputenc}
\usepackage{polski}

\usepackage{amsthm}  %pakiet do tworzenia twierdzeń itp.
\usepackage{amsmath} %pakiet do niektórych symboli matematycznych
\usepackage{amssymb} %pakiet do symboli mat., np. \nsubseteq
\usepackage{amsfonts}
\usepackage{graphicx} %obsługa plików graficznych z rozszerzeniem png, jpg
\theoremstyle{definition} %styl dla definicji
\newtheorem{zad}{} 
\title{Multizestaw zadań}
\author{Robert Fidytek}
%\date{\today}
\date{}
\newcounter{liczniksekcji}
\newcommand{\kategoria}[1]{\section{#1}} %olreślamy nazwę kateforii zadań
\newcommand{\zadStart}[1]{\begin{zad}#1\newline} %oznaczenie początku zadania
\newcommand{\zadStop}{\end{zad}}   %oznaczenie końca zadania
%Makra opcjonarne (nie muszą występować):
\newcommand{\rozwStart}[2]{\noindent \textbf{Rozwiązanie (autor #1 , recenzent #2): }\newline} %oznaczenie początku rozwiązania, opcjonarnie można wprowadzić informację o autorze rozwiązania zadania i recenzencie poprawności wykonania rozwiązania zadania
\newcommand{\rozwStop}{\newline}                                            %oznaczenie końca rozwiązania
\newcommand{\odpStart}{\noindent \textbf{Odpowiedź:}\newline}    %oznaczenie początku odpowiedzi końcowej (wypisanie wyniku)
\newcommand{\odpStop}{\newline}                                             %oznaczenie końca odpowiedzi końcowej (wypisanie wyniku)
\newcommand{\testStart}{\noindent \textbf{Test:}\newline} %ewentualne możliwe opcje odpowiedzi testowej: A. ? B. ? C. ? D. ? itd.
\newcommand{\testStop}{\newline} %koniec wprowadzania odpowiedzi testowych
\newcommand{\kluczStart}{\noindent \textbf{Test poprawna odpowiedź:}\newline} %klucz, poprawna odpowiedź pytania testowego (jedna literka): A lub B lub C lub D itd.
\newcommand{\kluczStop}{\newline} %koniec poprawnej odpowiedzi pytania testowego 
\newcommand{\wstawGrafike}[2]{\begin{figure}[h] \includegraphics[scale=#2] {#1} \end{figure}} %gdyby była potrzeba wstawienia obrazka, parametry: nazwa pliku, skala (jak nie wiesz co wpisać, to wpisz 1)

\begin{document}
\maketitle


\kategoria{Wikieł/C1.1a}
\zadStart{Zadanie z Wikieł C 1.1a moja wersja nr [nrWersji]}
%[a]:[2,3,4,5,6,7,8,9,10,11,12,13,14,15,16,17,18,19,20,21,22,23,24,25,26,27,28,29,30,31,32,33,34,35,36,37,38,39,40]
%[aa]=[a]*[a]
%[a2]=[a]*2
%[aa2]=[aa]*2
%[a22]=[a2]*2
%math.gcd([aa2],3)==1 and math.gcd([a22],7)==1
Oblicz całkę $$\int \sqrt{x}([a]-x^2)^2dx.$$
\zadStop
\rozwStart{Aleksandra Pasińska}{}
$$\int \sqrt{x}([a]-x^2)^2dx=\int \sqrt{x}([aa]-[a2]x^2+x^4)dx=$$
$$=\int[aa]\sqrt{x}-[a2]x^2\sqrt{x}+x^4\sqrt{x}dx=\int[aa]x^{\frac{1}{2}}-[a2]x^2\cdot x^{\frac{1}{2}}+x^4\cdot x^{\frac{1}{2}}dx=$$ 
$$=\int[aa]x^{\frac{1}{2}}-[a2]x^{\frac{5}{2}}+x^{\frac{9}{2}}dx=[aa]\int x^\frac{1}{2}dx-[a2]\int x^\frac{5}{2}dx+\int x^\frac{9}{2}dx=$$
$$=[aa]\cdot \frac{2x^{\frac{3}{2}}}{3}-[a2]\cdot \frac{2x^{\frac{7}{2}}}{7}+\frac{2x^{\frac{11}{2}}}{11}=[aa]\cdot \frac{2x\sqrt{x}}{3}-[a2]\cdot \frac{2x^3\sqrt{x}}{7}+\frac{2x^5\sqrt{x}}{11}+C=$$
$$= \frac{[aa2]x\sqrt{x}}{3}-\frac{[a22]x^3\sqrt{x}}{7}+\frac{2x^5\sqrt{x}}{11}+C$$
\rozwStop
\odpStart
$\frac{[aa2]x\sqrt{x}}{3}-\frac{[a22]x^3\sqrt{x}}{7}+\frac{2x^5\sqrt{x}}{11}+C$\\
\odpStop
\testStart
A.$\frac{[aa2]x\sqrt{x}}{3}-\frac{[a22]x^3\sqrt{x}}{7}+\frac{2x^5\sqrt{x}}{11}+C$
B.$-\frac{2x^5\sqrt{x}}{11}+C$
C.$\frac{[aa2]x\sqrt{x}}{3}-2+C$
D.$\frac{[aa2]x\sqrt{x}}{3}+C$
E.$\frac{2x^5\sqrt{x}}{11}+C$
F.$\frac{[a22]x^3\sqrt{x}}{7}+\frac{2x^5\sqrt{x}}{11}+C$
G.$-\frac{[a22]x^3\sqrt{x}}{7}+\frac{2x^5\sqrt{x}}{11}+C$
H.$\frac{[aa2]x\sqrt{x}}{3}+\frac{2x^5\sqrt{x}}{11}+C$
I.$\frac{[aa2]x\sqrt{x}}{3}-\frac{[a22]x^3\sqrt{x}}{7}+C$
\testStop
\kluczStart
A
\kluczStop



\end{document}