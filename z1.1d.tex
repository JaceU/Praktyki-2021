\documentclass[12pt, a4paper]{article}
\usepackage[utf8]{inputenc}
\usepackage{polski}
\usepackage{amsthm}  %pakiet do tworzenia twierdzeń itp.
\usepackage{amsmath} %pakiet do niektórych symboli matematycznych
\usepackage{amssymb} %pakiet do symboli mat., np. \nsubseteq
\usepackage{amsfonts}
\usepackage{graphicx} %obsługa plików graficznych z rozszerzeniem png, jpg
\theoremstyle{definition} %styl dla definicji
\newtheorem{zad}{} 
\title{Multizestaw zadań}
\author{Patryk Wirkus}
%\date{\today}
\date{}
\newcommand{\kategoria}[1]{\section{#1}}
\newcommand{\zadStart}[1]{\begin{zad}#1\newline}
\newcommand{\zadStop}{\end{zad}}
\newcommand{\rozwStart}[2]{\noindent \textbf{Rozwiązanie (autor #1 , recenzent #2): }\newline}
\newcommand{\rozwStop}{\newline}                                           
\newcommand{\odpStart}{\noindent \textbf{Odpowiedź:}\newline}
\newcommand{\odpStop}{\newline}
\newcommand{\testStart}{\noindent \textbf{Test:}\newline}
\newcommand{\testStop}{\newline}
\newcommand{\kluczStart}{\noindent \textbf{Test poprawna odpowiedź:}\newline}
\newcommand{\kluczStop}{\newline}
\newcommand{\wstawGrafike}[2]{\begin{figure}[h] \includegraphics[scale=#2] {#1} \end{figure}}

\begin{document}
\maketitle

\kategoria{Wikieł/Z1.1d}


\zadStart{Zadanie z Wikieł Z 1.1 d) moja wersja nr 1}

Obliczyć wartość wyrażenia $(\frac{149}{103})^{2} \cdot (\frac{103}{149})^{2} \cdot \pi^{0}$.
\zadStop
\rozwStart{Patryk Wirkus}{Martyna Czarnobaj}
$$(\frac{149}{103})^{2} \cdot (\frac{103}{149})^{2} \cdot \pi^{0} = (\frac{149}{103} \cdot \frac{103}{149})^{2} \cdot 1 = 1^{2} \cdot 1 = 1$$
\rozwStop
\odpStart
$1$
\odpStop
\testStart
A.$1$ B.$\pi$ C.$0$ D.$\frac{149}{103}$ E.$\frac{103}{149}$
F.$-\frac{149}{103}$ G.$-1$
H.$(\frac{149}{103})^{2}$
I.$(\frac{103}{149})^{2}$
\testStop
\kluczStart
A
\kluczStop



\zadStart{Zadanie z Wikieł Z 1.1 d) moja wersja nr 2}

Obliczyć wartość wyrażenia $(\frac{149}{107})^{2} \cdot (\frac{107}{149})^{2} \cdot \pi^{0}$.
\zadStop
\rozwStart{Patryk Wirkus}{Martyna Czarnobaj}
$$(\frac{149}{107})^{2} \cdot (\frac{107}{149})^{2} \cdot \pi^{0} = (\frac{149}{107} \cdot \frac{107}{149})^{2} \cdot 1 = 1^{2} \cdot 1 = 1$$
\rozwStop
\odpStart
$1$
\odpStop
\testStart
A.$1$ B.$\pi$ C.$0$ D.$\frac{149}{107}$ E.$\frac{107}{149}$
F.$-\frac{149}{107}$ G.$-1$
H.$(\frac{149}{107})^{2}$
I.$(\frac{107}{149})^{2}$
\testStop
\kluczStart
A
\kluczStop



\zadStart{Zadanie z Wikieł Z 1.1 d) moja wersja nr 3}

Obliczyć wartość wyrażenia $(\frac{149}{109})^{2} \cdot (\frac{109}{149})^{2} \cdot \pi^{0}$.
\zadStop
\rozwStart{Patryk Wirkus}{Martyna Czarnobaj}
$$(\frac{149}{109})^{2} \cdot (\frac{109}{149})^{2} \cdot \pi^{0} = (\frac{149}{109} \cdot \frac{109}{149})^{2} \cdot 1 = 1^{2} \cdot 1 = 1$$
\rozwStop
\odpStart
$1$
\odpStop
\testStart
A.$1$ B.$\pi$ C.$0$ D.$\frac{149}{109}$ E.$\frac{109}{149}$
F.$-\frac{149}{109}$ G.$-1$
H.$(\frac{149}{109})^{2}$
I.$(\frac{109}{149})^{2}$
\testStop
\kluczStart
A
\kluczStop



\zadStart{Zadanie z Wikieł Z 1.1 d) moja wersja nr 4}

Obliczyć wartość wyrażenia $(\frac{149}{113})^{2} \cdot (\frac{113}{149})^{2} \cdot \pi^{0}$.
\zadStop
\rozwStart{Patryk Wirkus}{Martyna Czarnobaj}
$$(\frac{149}{113})^{2} \cdot (\frac{113}{149})^{2} \cdot \pi^{0} = (\frac{149}{113} \cdot \frac{113}{149})^{2} \cdot 1 = 1^{2} \cdot 1 = 1$$
\rozwStop
\odpStart
$1$
\odpStop
\testStart
A.$1$ B.$\pi$ C.$0$ D.$\frac{149}{113}$ E.$\frac{113}{149}$
F.$-\frac{149}{113}$ G.$-1$
H.$(\frac{149}{113})^{2}$
I.$(\frac{113}{149})^{2}$
\testStop
\kluczStart
A
\kluczStop



\zadStart{Zadanie z Wikieł Z 1.1 d) moja wersja nr 5}

Obliczyć wartość wyrażenia $(\frac{149}{127})^{2} \cdot (\frac{127}{149})^{2} \cdot \pi^{0}$.
\zadStop
\rozwStart{Patryk Wirkus}{Martyna Czarnobaj}
$$(\frac{149}{127})^{2} \cdot (\frac{127}{149})^{2} \cdot \pi^{0} = (\frac{149}{127} \cdot \frac{127}{149})^{2} \cdot 1 = 1^{2} \cdot 1 = 1$$
\rozwStop
\odpStart
$1$
\odpStop
\testStart
A.$1$ B.$\pi$ C.$0$ D.$\frac{149}{127}$ E.$\frac{127}{149}$
F.$-\frac{149}{127}$ G.$-1$
H.$(\frac{149}{127})^{2}$
I.$(\frac{127}{149})^{2}$
\testStop
\kluczStart
A
\kluczStop



\zadStart{Zadanie z Wikieł Z 1.1 d) moja wersja nr 6}

Obliczyć wartość wyrażenia $(\frac{149}{131})^{2} \cdot (\frac{131}{149})^{2} \cdot \pi^{0}$.
\zadStop
\rozwStart{Patryk Wirkus}{Martyna Czarnobaj}
$$(\frac{149}{131})^{2} \cdot (\frac{131}{149})^{2} \cdot \pi^{0} = (\frac{149}{131} \cdot \frac{131}{149})^{2} \cdot 1 = 1^{2} \cdot 1 = 1$$
\rozwStop
\odpStart
$1$
\odpStop
\testStart
A.$1$ B.$\pi$ C.$0$ D.$\frac{149}{131}$ E.$\frac{131}{149}$
F.$-\frac{149}{131}$ G.$-1$
H.$(\frac{149}{131})^{2}$
I.$(\frac{131}{149})^{2}$
\testStop
\kluczStart
A
\kluczStop



\zadStart{Zadanie z Wikieł Z 1.1 d) moja wersja nr 7}

Obliczyć wartość wyrażenia $(\frac{149}{137})^{2} \cdot (\frac{137}{149})^{2} \cdot \pi^{0}$.
\zadStop
\rozwStart{Patryk Wirkus}{Martyna Czarnobaj}
$$(\frac{149}{137})^{2} \cdot (\frac{137}{149})^{2} \cdot \pi^{0} = (\frac{149}{137} \cdot \frac{137}{149})^{2} \cdot 1 = 1^{2} \cdot 1 = 1$$
\rozwStop
\odpStart
$1$
\odpStop
\testStart
A.$1$ B.$\pi$ C.$0$ D.$\frac{149}{137}$ E.$\frac{137}{149}$
F.$-\frac{149}{137}$ G.$-1$
H.$(\frac{149}{137})^{2}$
I.$(\frac{137}{149})^{2}$
\testStop
\kluczStart
A
\kluczStop



\zadStart{Zadanie z Wikieł Z 1.1 d) moja wersja nr 8}

Obliczyć wartość wyrażenia $(\frac{149}{139})^{2} \cdot (\frac{139}{149})^{2} \cdot \pi^{0}$.
\zadStop
\rozwStart{Patryk Wirkus}{Martyna Czarnobaj}
$$(\frac{149}{139})^{2} \cdot (\frac{139}{149})^{2} \cdot \pi^{0} = (\frac{149}{139} \cdot \frac{139}{149})^{2} \cdot 1 = 1^{2} \cdot 1 = 1$$
\rozwStop
\odpStart
$1$
\odpStop
\testStart
A.$1$ B.$\pi$ C.$0$ D.$\frac{149}{139}$ E.$\frac{139}{149}$
F.$-\frac{149}{139}$ G.$-1$
H.$(\frac{149}{139})^{2}$
I.$(\frac{139}{149})^{2}$
\testStop
\kluczStart
A
\kluczStop



\zadStart{Zadanie z Wikieł Z 1.1 d) moja wersja nr 9}

Obliczyć wartość wyrażenia $(\frac{151}{103})^{2} \cdot (\frac{103}{151})^{2} \cdot \pi^{0}$.
\zadStop
\rozwStart{Patryk Wirkus}{Martyna Czarnobaj}
$$(\frac{151}{103})^{2} \cdot (\frac{103}{151})^{2} \cdot \pi^{0} = (\frac{151}{103} \cdot \frac{103}{151})^{2} \cdot 1 = 1^{2} \cdot 1 = 1$$
\rozwStop
\odpStart
$1$
\odpStop
\testStart
A.$1$ B.$\pi$ C.$0$ D.$\frac{151}{103}$ E.$\frac{103}{151}$
F.$-\frac{151}{103}$ G.$-1$
H.$(\frac{151}{103})^{2}$
I.$(\frac{103}{151})^{2}$
\testStop
\kluczStart
A
\kluczStop



\zadStart{Zadanie z Wikieł Z 1.1 d) moja wersja nr 10}

Obliczyć wartość wyrażenia $(\frac{151}{107})^{2} \cdot (\frac{107}{151})^{2} \cdot \pi^{0}$.
\zadStop
\rozwStart{Patryk Wirkus}{Martyna Czarnobaj}
$$(\frac{151}{107})^{2} \cdot (\frac{107}{151})^{2} \cdot \pi^{0} = (\frac{151}{107} \cdot \frac{107}{151})^{2} \cdot 1 = 1^{2} \cdot 1 = 1$$
\rozwStop
\odpStart
$1$
\odpStop
\testStart
A.$1$ B.$\pi$ C.$0$ D.$\frac{151}{107}$ E.$\frac{107}{151}$
F.$-\frac{151}{107}$ G.$-1$
H.$(\frac{151}{107})^{2}$
I.$(\frac{107}{151})^{2}$
\testStop
\kluczStart
A
\kluczStop



\zadStart{Zadanie z Wikieł Z 1.1 d) moja wersja nr 11}

Obliczyć wartość wyrażenia $(\frac{151}{109})^{2} \cdot (\frac{109}{151})^{2} \cdot \pi^{0}$.
\zadStop
\rozwStart{Patryk Wirkus}{Martyna Czarnobaj}
$$(\frac{151}{109})^{2} \cdot (\frac{109}{151})^{2} \cdot \pi^{0} = (\frac{151}{109} \cdot \frac{109}{151})^{2} \cdot 1 = 1^{2} \cdot 1 = 1$$
\rozwStop
\odpStart
$1$
\odpStop
\testStart
A.$1$ B.$\pi$ C.$0$ D.$\frac{151}{109}$ E.$\frac{109}{151}$
F.$-\frac{151}{109}$ G.$-1$
H.$(\frac{151}{109})^{2}$
I.$(\frac{109}{151})^{2}$
\testStop
\kluczStart
A
\kluczStop



\zadStart{Zadanie z Wikieł Z 1.1 d) moja wersja nr 12}

Obliczyć wartość wyrażenia $(\frac{151}{113})^{2} \cdot (\frac{113}{151})^{2} \cdot \pi^{0}$.
\zadStop
\rozwStart{Patryk Wirkus}{Martyna Czarnobaj}
$$(\frac{151}{113})^{2} \cdot (\frac{113}{151})^{2} \cdot \pi^{0} = (\frac{151}{113} \cdot \frac{113}{151})^{2} \cdot 1 = 1^{2} \cdot 1 = 1$$
\rozwStop
\odpStart
$1$
\odpStop
\testStart
A.$1$ B.$\pi$ C.$0$ D.$\frac{151}{113}$ E.$\frac{113}{151}$
F.$-\frac{151}{113}$ G.$-1$
H.$(\frac{151}{113})^{2}$
I.$(\frac{113}{151})^{2}$
\testStop
\kluczStart
A
\kluczStop



\zadStart{Zadanie z Wikieł Z 1.1 d) moja wersja nr 13}

Obliczyć wartość wyrażenia $(\frac{151}{127})^{2} \cdot (\frac{127}{151})^{2} \cdot \pi^{0}$.
\zadStop
\rozwStart{Patryk Wirkus}{Martyna Czarnobaj}
$$(\frac{151}{127})^{2} \cdot (\frac{127}{151})^{2} \cdot \pi^{0} = (\frac{151}{127} \cdot \frac{127}{151})^{2} \cdot 1 = 1^{2} \cdot 1 = 1$$
\rozwStop
\odpStart
$1$
\odpStop
\testStart
A.$1$ B.$\pi$ C.$0$ D.$\frac{151}{127}$ E.$\frac{127}{151}$
F.$-\frac{151}{127}$ G.$-1$
H.$(\frac{151}{127})^{2}$
I.$(\frac{127}{151})^{2}$
\testStop
\kluczStart
A
\kluczStop



\zadStart{Zadanie z Wikieł Z 1.1 d) moja wersja nr 14}

Obliczyć wartość wyrażenia $(\frac{151}{131})^{2} \cdot (\frac{131}{151})^{2} \cdot \pi^{0}$.
\zadStop
\rozwStart{Patryk Wirkus}{Martyna Czarnobaj}
$$(\frac{151}{131})^{2} \cdot (\frac{131}{151})^{2} \cdot \pi^{0} = (\frac{151}{131} \cdot \frac{131}{151})^{2} \cdot 1 = 1^{2} \cdot 1 = 1$$
\rozwStop
\odpStart
$1$
\odpStop
\testStart
A.$1$ B.$\pi$ C.$0$ D.$\frac{151}{131}$ E.$\frac{131}{151}$
F.$-\frac{151}{131}$ G.$-1$
H.$(\frac{151}{131})^{2}$
I.$(\frac{131}{151})^{2}$
\testStop
\kluczStart
A
\kluczStop



\zadStart{Zadanie z Wikieł Z 1.1 d) moja wersja nr 15}

Obliczyć wartość wyrażenia $(\frac{151}{137})^{2} \cdot (\frac{137}{151})^{2} \cdot \pi^{0}$.
\zadStop
\rozwStart{Patryk Wirkus}{Martyna Czarnobaj}
$$(\frac{151}{137})^{2} \cdot (\frac{137}{151})^{2} \cdot \pi^{0} = (\frac{151}{137} \cdot \frac{137}{151})^{2} \cdot 1 = 1^{2} \cdot 1 = 1$$
\rozwStop
\odpStart
$1$
\odpStop
\testStart
A.$1$ B.$\pi$ C.$0$ D.$\frac{151}{137}$ E.$\frac{137}{151}$
F.$-\frac{151}{137}$ G.$-1$
H.$(\frac{151}{137})^{2}$
I.$(\frac{137}{151})^{2}$
\testStop
\kluczStart
A
\kluczStop



\zadStart{Zadanie z Wikieł Z 1.1 d) moja wersja nr 16}

Obliczyć wartość wyrażenia $(\frac{151}{139})^{2} \cdot (\frac{139}{151})^{2} \cdot \pi^{0}$.
\zadStop
\rozwStart{Patryk Wirkus}{Martyna Czarnobaj}
$$(\frac{151}{139})^{2} \cdot (\frac{139}{151})^{2} \cdot \pi^{0} = (\frac{151}{139} \cdot \frac{139}{151})^{2} \cdot 1 = 1^{2} \cdot 1 = 1$$
\rozwStop
\odpStart
$1$
\odpStop
\testStart
A.$1$ B.$\pi$ C.$0$ D.$\frac{151}{139}$ E.$\frac{139}{151}$
F.$-\frac{151}{139}$ G.$-1$
H.$(\frac{151}{139})^{2}$
I.$(\frac{139}{151})^{2}$
\testStop
\kluczStart
A
\kluczStop



\zadStart{Zadanie z Wikieł Z 1.1 d) moja wersja nr 17}

Obliczyć wartość wyrażenia $(\frac{157}{103})^{2} \cdot (\frac{103}{157})^{2} \cdot \pi^{0}$.
\zadStop
\rozwStart{Patryk Wirkus}{Martyna Czarnobaj}
$$(\frac{157}{103})^{2} \cdot (\frac{103}{157})^{2} \cdot \pi^{0} = (\frac{157}{103} \cdot \frac{103}{157})^{2} \cdot 1 = 1^{2} \cdot 1 = 1$$
\rozwStop
\odpStart
$1$
\odpStop
\testStart
A.$1$ B.$\pi$ C.$0$ D.$\frac{157}{103}$ E.$\frac{103}{157}$
F.$-\frac{157}{103}$ G.$-1$
H.$(\frac{157}{103})^{2}$
I.$(\frac{103}{157})^{2}$
\testStop
\kluczStart
A
\kluczStop



\zadStart{Zadanie z Wikieł Z 1.1 d) moja wersja nr 18}

Obliczyć wartość wyrażenia $(\frac{157}{107})^{2} \cdot (\frac{107}{157})^{2} \cdot \pi^{0}$.
\zadStop
\rozwStart{Patryk Wirkus}{Martyna Czarnobaj}
$$(\frac{157}{107})^{2} \cdot (\frac{107}{157})^{2} \cdot \pi^{0} = (\frac{157}{107} \cdot \frac{107}{157})^{2} \cdot 1 = 1^{2} \cdot 1 = 1$$
\rozwStop
\odpStart
$1$
\odpStop
\testStart
A.$1$ B.$\pi$ C.$0$ D.$\frac{157}{107}$ E.$\frac{107}{157}$
F.$-\frac{157}{107}$ G.$-1$
H.$(\frac{157}{107})^{2}$
I.$(\frac{107}{157})^{2}$
\testStop
\kluczStart
A
\kluczStop



\zadStart{Zadanie z Wikieł Z 1.1 d) moja wersja nr 19}

Obliczyć wartość wyrażenia $(\frac{157}{109})^{2} \cdot (\frac{109}{157})^{2} \cdot \pi^{0}$.
\zadStop
\rozwStart{Patryk Wirkus}{Martyna Czarnobaj}
$$(\frac{157}{109})^{2} \cdot (\frac{109}{157})^{2} \cdot \pi^{0} = (\frac{157}{109} \cdot \frac{109}{157})^{2} \cdot 1 = 1^{2} \cdot 1 = 1$$
\rozwStop
\odpStart
$1$
\odpStop
\testStart
A.$1$ B.$\pi$ C.$0$ D.$\frac{157}{109}$ E.$\frac{109}{157}$
F.$-\frac{157}{109}$ G.$-1$
H.$(\frac{157}{109})^{2}$
I.$(\frac{109}{157})^{2}$
\testStop
\kluczStart
A
\kluczStop



\zadStart{Zadanie z Wikieł Z 1.1 d) moja wersja nr 20}

Obliczyć wartość wyrażenia $(\frac{157}{113})^{2} \cdot (\frac{113}{157})^{2} \cdot \pi^{0}$.
\zadStop
\rozwStart{Patryk Wirkus}{Martyna Czarnobaj}
$$(\frac{157}{113})^{2} \cdot (\frac{113}{157})^{2} \cdot \pi^{0} = (\frac{157}{113} \cdot \frac{113}{157})^{2} \cdot 1 = 1^{2} \cdot 1 = 1$$
\rozwStop
\odpStart
$1$
\odpStop
\testStart
A.$1$ B.$\pi$ C.$0$ D.$\frac{157}{113}$ E.$\frac{113}{157}$
F.$-\frac{157}{113}$ G.$-1$
H.$(\frac{157}{113})^{2}$
I.$(\frac{113}{157})^{2}$
\testStop
\kluczStart
A
\kluczStop



\zadStart{Zadanie z Wikieł Z 1.1 d) moja wersja nr 21}

Obliczyć wartość wyrażenia $(\frac{157}{127})^{2} \cdot (\frac{127}{157})^{2} \cdot \pi^{0}$.
\zadStop
\rozwStart{Patryk Wirkus}{Martyna Czarnobaj}
$$(\frac{157}{127})^{2} \cdot (\frac{127}{157})^{2} \cdot \pi^{0} = (\frac{157}{127} \cdot \frac{127}{157})^{2} \cdot 1 = 1^{2} \cdot 1 = 1$$
\rozwStop
\odpStart
$1$
\odpStop
\testStart
A.$1$ B.$\pi$ C.$0$ D.$\frac{157}{127}$ E.$\frac{127}{157}$
F.$-\frac{157}{127}$ G.$-1$
H.$(\frac{157}{127})^{2}$
I.$(\frac{127}{157})^{2}$
\testStop
\kluczStart
A
\kluczStop



\zadStart{Zadanie z Wikieł Z 1.1 d) moja wersja nr 22}

Obliczyć wartość wyrażenia $(\frac{157}{131})^{2} \cdot (\frac{131}{157})^{2} \cdot \pi^{0}$.
\zadStop
\rozwStart{Patryk Wirkus}{Martyna Czarnobaj}
$$(\frac{157}{131})^{2} \cdot (\frac{131}{157})^{2} \cdot \pi^{0} = (\frac{157}{131} \cdot \frac{131}{157})^{2} \cdot 1 = 1^{2} \cdot 1 = 1$$
\rozwStop
\odpStart
$1$
\odpStop
\testStart
A.$1$ B.$\pi$ C.$0$ D.$\frac{157}{131}$ E.$\frac{131}{157}$
F.$-\frac{157}{131}$ G.$-1$
H.$(\frac{157}{131})^{2}$
I.$(\frac{131}{157})^{2}$
\testStop
\kluczStart
A
\kluczStop



\zadStart{Zadanie z Wikieł Z 1.1 d) moja wersja nr 23}

Obliczyć wartość wyrażenia $(\frac{157}{137})^{2} \cdot (\frac{137}{157})^{2} \cdot \pi^{0}$.
\zadStop
\rozwStart{Patryk Wirkus}{Martyna Czarnobaj}
$$(\frac{157}{137})^{2} \cdot (\frac{137}{157})^{2} \cdot \pi^{0} = (\frac{157}{137} \cdot \frac{137}{157})^{2} \cdot 1 = 1^{2} \cdot 1 = 1$$
\rozwStop
\odpStart
$1$
\odpStop
\testStart
A.$1$ B.$\pi$ C.$0$ D.$\frac{157}{137}$ E.$\frac{137}{157}$
F.$-\frac{157}{137}$ G.$-1$
H.$(\frac{157}{137})^{2}$
I.$(\frac{137}{157})^{2}$
\testStop
\kluczStart
A
\kluczStop



\zadStart{Zadanie z Wikieł Z 1.1 d) moja wersja nr 24}

Obliczyć wartość wyrażenia $(\frac{157}{139})^{2} \cdot (\frac{139}{157})^{2} \cdot \pi^{0}$.
\zadStop
\rozwStart{Patryk Wirkus}{Martyna Czarnobaj}
$$(\frac{157}{139})^{2} \cdot (\frac{139}{157})^{2} \cdot \pi^{0} = (\frac{157}{139} \cdot \frac{139}{157})^{2} \cdot 1 = 1^{2} \cdot 1 = 1$$
\rozwStop
\odpStart
$1$
\odpStop
\testStart
A.$1$ B.$\pi$ C.$0$ D.$\frac{157}{139}$ E.$\frac{139}{157}$
F.$-\frac{157}{139}$ G.$-1$
H.$(\frac{157}{139})^{2}$
I.$(\frac{139}{157})^{2}$
\testStop
\kluczStart
A
\kluczStop



\zadStart{Zadanie z Wikieł Z 1.1 d) moja wersja nr 25}

Obliczyć wartość wyrażenia $(\frac{163}{103})^{2} \cdot (\frac{103}{163})^{2} \cdot \pi^{0}$.
\zadStop
\rozwStart{Patryk Wirkus}{Martyna Czarnobaj}
$$(\frac{163}{103})^{2} \cdot (\frac{103}{163})^{2} \cdot \pi^{0} = (\frac{163}{103} \cdot \frac{103}{163})^{2} \cdot 1 = 1^{2} \cdot 1 = 1$$
\rozwStop
\odpStart
$1$
\odpStop
\testStart
A.$1$ B.$\pi$ C.$0$ D.$\frac{163}{103}$ E.$\frac{103}{163}$
F.$-\frac{163}{103}$ G.$-1$
H.$(\frac{163}{103})^{2}$
I.$(\frac{103}{163})^{2}$
\testStop
\kluczStart
A
\kluczStop



\zadStart{Zadanie z Wikieł Z 1.1 d) moja wersja nr 26}

Obliczyć wartość wyrażenia $(\frac{163}{107})^{2} \cdot (\frac{107}{163})^{2} \cdot \pi^{0}$.
\zadStop
\rozwStart{Patryk Wirkus}{Martyna Czarnobaj}
$$(\frac{163}{107})^{2} \cdot (\frac{107}{163})^{2} \cdot \pi^{0} = (\frac{163}{107} \cdot \frac{107}{163})^{2} \cdot 1 = 1^{2} \cdot 1 = 1$$
\rozwStop
\odpStart
$1$
\odpStop
\testStart
A.$1$ B.$\pi$ C.$0$ D.$\frac{163}{107}$ E.$\frac{107}{163}$
F.$-\frac{163}{107}$ G.$-1$
H.$(\frac{163}{107})^{2}$
I.$(\frac{107}{163})^{2}$
\testStop
\kluczStart
A
\kluczStop



\zadStart{Zadanie z Wikieł Z 1.1 d) moja wersja nr 27}

Obliczyć wartość wyrażenia $(\frac{163}{109})^{2} \cdot (\frac{109}{163})^{2} \cdot \pi^{0}$.
\zadStop
\rozwStart{Patryk Wirkus}{Martyna Czarnobaj}
$$(\frac{163}{109})^{2} \cdot (\frac{109}{163})^{2} \cdot \pi^{0} = (\frac{163}{109} \cdot \frac{109}{163})^{2} \cdot 1 = 1^{2} \cdot 1 = 1$$
\rozwStop
\odpStart
$1$
\odpStop
\testStart
A.$1$ B.$\pi$ C.$0$ D.$\frac{163}{109}$ E.$\frac{109}{163}$
F.$-\frac{163}{109}$ G.$-1$
H.$(\frac{163}{109})^{2}$
I.$(\frac{109}{163})^{2}$
\testStop
\kluczStart
A
\kluczStop



\zadStart{Zadanie z Wikieł Z 1.1 d) moja wersja nr 28}

Obliczyć wartość wyrażenia $(\frac{163}{113})^{2} \cdot (\frac{113}{163})^{2} \cdot \pi^{0}$.
\zadStop
\rozwStart{Patryk Wirkus}{Martyna Czarnobaj}
$$(\frac{163}{113})^{2} \cdot (\frac{113}{163})^{2} \cdot \pi^{0} = (\frac{163}{113} \cdot \frac{113}{163})^{2} \cdot 1 = 1^{2} \cdot 1 = 1$$
\rozwStop
\odpStart
$1$
\odpStop
\testStart
A.$1$ B.$\pi$ C.$0$ D.$\frac{163}{113}$ E.$\frac{113}{163}$
F.$-\frac{163}{113}$ G.$-1$
H.$(\frac{163}{113})^{2}$
I.$(\frac{113}{163})^{2}$
\testStop
\kluczStart
A
\kluczStop



\zadStart{Zadanie z Wikieł Z 1.1 d) moja wersja nr 29}

Obliczyć wartość wyrażenia $(\frac{163}{127})^{2} \cdot (\frac{127}{163})^{2} \cdot \pi^{0}$.
\zadStop
\rozwStart{Patryk Wirkus}{Martyna Czarnobaj}
$$(\frac{163}{127})^{2} \cdot (\frac{127}{163})^{2} \cdot \pi^{0} = (\frac{163}{127} \cdot \frac{127}{163})^{2} \cdot 1 = 1^{2} \cdot 1 = 1$$
\rozwStop
\odpStart
$1$
\odpStop
\testStart
A.$1$ B.$\pi$ C.$0$ D.$\frac{163}{127}$ E.$\frac{127}{163}$
F.$-\frac{163}{127}$ G.$-1$
H.$(\frac{163}{127})^{2}$
I.$(\frac{127}{163})^{2}$
\testStop
\kluczStart
A
\kluczStop



\zadStart{Zadanie z Wikieł Z 1.1 d) moja wersja nr 30}

Obliczyć wartość wyrażenia $(\frac{163}{131})^{2} \cdot (\frac{131}{163})^{2} \cdot \pi^{0}$.
\zadStop
\rozwStart{Patryk Wirkus}{Martyna Czarnobaj}
$$(\frac{163}{131})^{2} \cdot (\frac{131}{163})^{2} \cdot \pi^{0} = (\frac{163}{131} \cdot \frac{131}{163})^{2} \cdot 1 = 1^{2} \cdot 1 = 1$$
\rozwStop
\odpStart
$1$
\odpStop
\testStart
A.$1$ B.$\pi$ C.$0$ D.$\frac{163}{131}$ E.$\frac{131}{163}$
F.$-\frac{163}{131}$ G.$-1$
H.$(\frac{163}{131})^{2}$
I.$(\frac{131}{163})^{2}$
\testStop
\kluczStart
A
\kluczStop



\zadStart{Zadanie z Wikieł Z 1.1 d) moja wersja nr 31}

Obliczyć wartość wyrażenia $(\frac{163}{137})^{2} \cdot (\frac{137}{163})^{2} \cdot \pi^{0}$.
\zadStop
\rozwStart{Patryk Wirkus}{Martyna Czarnobaj}
$$(\frac{163}{137})^{2} \cdot (\frac{137}{163})^{2} \cdot \pi^{0} = (\frac{163}{137} \cdot \frac{137}{163})^{2} \cdot 1 = 1^{2} \cdot 1 = 1$$
\rozwStop
\odpStart
$1$
\odpStop
\testStart
A.$1$ B.$\pi$ C.$0$ D.$\frac{163}{137}$ E.$\frac{137}{163}$
F.$-\frac{163}{137}$ G.$-1$
H.$(\frac{163}{137})^{2}$
I.$(\frac{137}{163})^{2}$
\testStop
\kluczStart
A
\kluczStop



\zadStart{Zadanie z Wikieł Z 1.1 d) moja wersja nr 32}

Obliczyć wartość wyrażenia $(\frac{163}{139})^{2} \cdot (\frac{139}{163})^{2} \cdot \pi^{0}$.
\zadStop
\rozwStart{Patryk Wirkus}{Martyna Czarnobaj}
$$(\frac{163}{139})^{2} \cdot (\frac{139}{163})^{2} \cdot \pi^{0} = (\frac{163}{139} \cdot \frac{139}{163})^{2} \cdot 1 = 1^{2} \cdot 1 = 1$$
\rozwStop
\odpStart
$1$
\odpStop
\testStart
A.$1$ B.$\pi$ C.$0$ D.$\frac{163}{139}$ E.$\frac{139}{163}$
F.$-\frac{163}{139}$ G.$-1$
H.$(\frac{163}{139})^{2}$
I.$(\frac{139}{163})^{2}$
\testStop
\kluczStart
A
\kluczStop



\zadStart{Zadanie z Wikieł Z 1.1 d) moja wersja nr 33}

Obliczyć wartość wyrażenia $(\frac{167}{103})^{2} \cdot (\frac{103}{167})^{2} \cdot \pi^{0}$.
\zadStop
\rozwStart{Patryk Wirkus}{Martyna Czarnobaj}
$$(\frac{167}{103})^{2} \cdot (\frac{103}{167})^{2} \cdot \pi^{0} = (\frac{167}{103} \cdot \frac{103}{167})^{2} \cdot 1 = 1^{2} \cdot 1 = 1$$
\rozwStop
\odpStart
$1$
\odpStop
\testStart
A.$1$ B.$\pi$ C.$0$ D.$\frac{167}{103}$ E.$\frac{103}{167}$
F.$-\frac{167}{103}$ G.$-1$
H.$(\frac{167}{103})^{2}$
I.$(\frac{103}{167})^{2}$
\testStop
\kluczStart
A
\kluczStop



\zadStart{Zadanie z Wikieł Z 1.1 d) moja wersja nr 34}

Obliczyć wartość wyrażenia $(\frac{167}{107})^{2} \cdot (\frac{107}{167})^{2} \cdot \pi^{0}$.
\zadStop
\rozwStart{Patryk Wirkus}{Martyna Czarnobaj}
$$(\frac{167}{107})^{2} \cdot (\frac{107}{167})^{2} \cdot \pi^{0} = (\frac{167}{107} \cdot \frac{107}{167})^{2} \cdot 1 = 1^{2} \cdot 1 = 1$$
\rozwStop
\odpStart
$1$
\odpStop
\testStart
A.$1$ B.$\pi$ C.$0$ D.$\frac{167}{107}$ E.$\frac{107}{167}$
F.$-\frac{167}{107}$ G.$-1$
H.$(\frac{167}{107})^{2}$
I.$(\frac{107}{167})^{2}$
\testStop
\kluczStart
A
\kluczStop



\zadStart{Zadanie z Wikieł Z 1.1 d) moja wersja nr 35}

Obliczyć wartość wyrażenia $(\frac{167}{109})^{2} \cdot (\frac{109}{167})^{2} \cdot \pi^{0}$.
\zadStop
\rozwStart{Patryk Wirkus}{Martyna Czarnobaj}
$$(\frac{167}{109})^{2} \cdot (\frac{109}{167})^{2} \cdot \pi^{0} = (\frac{167}{109} \cdot \frac{109}{167})^{2} \cdot 1 = 1^{2} \cdot 1 = 1$$
\rozwStop
\odpStart
$1$
\odpStop
\testStart
A.$1$ B.$\pi$ C.$0$ D.$\frac{167}{109}$ E.$\frac{109}{167}$
F.$-\frac{167}{109}$ G.$-1$
H.$(\frac{167}{109})^{2}$
I.$(\frac{109}{167})^{2}$
\testStop
\kluczStart
A
\kluczStop



\zadStart{Zadanie z Wikieł Z 1.1 d) moja wersja nr 36}

Obliczyć wartość wyrażenia $(\frac{167}{113})^{2} \cdot (\frac{113}{167})^{2} \cdot \pi^{0}$.
\zadStop
\rozwStart{Patryk Wirkus}{Martyna Czarnobaj}
$$(\frac{167}{113})^{2} \cdot (\frac{113}{167})^{2} \cdot \pi^{0} = (\frac{167}{113} \cdot \frac{113}{167})^{2} \cdot 1 = 1^{2} \cdot 1 = 1$$
\rozwStop
\odpStart
$1$
\odpStop
\testStart
A.$1$ B.$\pi$ C.$0$ D.$\frac{167}{113}$ E.$\frac{113}{167}$
F.$-\frac{167}{113}$ G.$-1$
H.$(\frac{167}{113})^{2}$
I.$(\frac{113}{167})^{2}$
\testStop
\kluczStart
A
\kluczStop



\zadStart{Zadanie z Wikieł Z 1.1 d) moja wersja nr 37}

Obliczyć wartość wyrażenia $(\frac{167}{127})^{2} \cdot (\frac{127}{167})^{2} \cdot \pi^{0}$.
\zadStop
\rozwStart{Patryk Wirkus}{Martyna Czarnobaj}
$$(\frac{167}{127})^{2} \cdot (\frac{127}{167})^{2} \cdot \pi^{0} = (\frac{167}{127} \cdot \frac{127}{167})^{2} \cdot 1 = 1^{2} \cdot 1 = 1$$
\rozwStop
\odpStart
$1$
\odpStop
\testStart
A.$1$ B.$\pi$ C.$0$ D.$\frac{167}{127}$ E.$\frac{127}{167}$
F.$-\frac{167}{127}$ G.$-1$
H.$(\frac{167}{127})^{2}$
I.$(\frac{127}{167})^{2}$
\testStop
\kluczStart
A
\kluczStop



\zadStart{Zadanie z Wikieł Z 1.1 d) moja wersja nr 38}

Obliczyć wartość wyrażenia $(\frac{167}{131})^{2} \cdot (\frac{131}{167})^{2} \cdot \pi^{0}$.
\zadStop
\rozwStart{Patryk Wirkus}{Martyna Czarnobaj}
$$(\frac{167}{131})^{2} \cdot (\frac{131}{167})^{2} \cdot \pi^{0} = (\frac{167}{131} \cdot \frac{131}{167})^{2} \cdot 1 = 1^{2} \cdot 1 = 1$$
\rozwStop
\odpStart
$1$
\odpStop
\testStart
A.$1$ B.$\pi$ C.$0$ D.$\frac{167}{131}$ E.$\frac{131}{167}$
F.$-\frac{167}{131}$ G.$-1$
H.$(\frac{167}{131})^{2}$
I.$(\frac{131}{167})^{2}$
\testStop
\kluczStart
A
\kluczStop



\zadStart{Zadanie z Wikieł Z 1.1 d) moja wersja nr 39}

Obliczyć wartość wyrażenia $(\frac{167}{137})^{2} \cdot (\frac{137}{167})^{2} \cdot \pi^{0}$.
\zadStop
\rozwStart{Patryk Wirkus}{Martyna Czarnobaj}
$$(\frac{167}{137})^{2} \cdot (\frac{137}{167})^{2} \cdot \pi^{0} = (\frac{167}{137} \cdot \frac{137}{167})^{2} \cdot 1 = 1^{2} \cdot 1 = 1$$
\rozwStop
\odpStart
$1$
\odpStop
\testStart
A.$1$ B.$\pi$ C.$0$ D.$\frac{167}{137}$ E.$\frac{137}{167}$
F.$-\frac{167}{137}$ G.$-1$
H.$(\frac{167}{137})^{2}$
I.$(\frac{137}{167})^{2}$
\testStop
\kluczStart
A
\kluczStop



\zadStart{Zadanie z Wikieł Z 1.1 d) moja wersja nr 40}

Obliczyć wartość wyrażenia $(\frac{167}{139})^{2} \cdot (\frac{139}{167})^{2} \cdot \pi^{0}$.
\zadStop
\rozwStart{Patryk Wirkus}{Martyna Czarnobaj}
$$(\frac{167}{139})^{2} \cdot (\frac{139}{167})^{2} \cdot \pi^{0} = (\frac{167}{139} \cdot \frac{139}{167})^{2} \cdot 1 = 1^{2} \cdot 1 = 1$$
\rozwStop
\odpStart
$1$
\odpStop
\testStart
A.$1$ B.$\pi$ C.$0$ D.$\frac{167}{139}$ E.$\frac{139}{167}$
F.$-\frac{167}{139}$ G.$-1$
H.$(\frac{167}{139})^{2}$
I.$(\frac{139}{167})^{2}$
\testStop
\kluczStart
A
\kluczStop



\zadStart{Zadanie z Wikieł Z 1.1 d) moja wersja nr 41}

Obliczyć wartość wyrażenia $(\frac{173}{103})^{2} \cdot (\frac{103}{173})^{2} \cdot \pi^{0}$.
\zadStop
\rozwStart{Patryk Wirkus}{Martyna Czarnobaj}
$$(\frac{173}{103})^{2} \cdot (\frac{103}{173})^{2} \cdot \pi^{0} = (\frac{173}{103} \cdot \frac{103}{173})^{2} \cdot 1 = 1^{2} \cdot 1 = 1$$
\rozwStop
\odpStart
$1$
\odpStop
\testStart
A.$1$ B.$\pi$ C.$0$ D.$\frac{173}{103}$ E.$\frac{103}{173}$
F.$-\frac{173}{103}$ G.$-1$
H.$(\frac{173}{103})^{2}$
I.$(\frac{103}{173})^{2}$
\testStop
\kluczStart
A
\kluczStop



\zadStart{Zadanie z Wikieł Z 1.1 d) moja wersja nr 42}

Obliczyć wartość wyrażenia $(\frac{173}{107})^{2} \cdot (\frac{107}{173})^{2} \cdot \pi^{0}$.
\zadStop
\rozwStart{Patryk Wirkus}{Martyna Czarnobaj}
$$(\frac{173}{107})^{2} \cdot (\frac{107}{173})^{2} \cdot \pi^{0} = (\frac{173}{107} \cdot \frac{107}{173})^{2} \cdot 1 = 1^{2} \cdot 1 = 1$$
\rozwStop
\odpStart
$1$
\odpStop
\testStart
A.$1$ B.$\pi$ C.$0$ D.$\frac{173}{107}$ E.$\frac{107}{173}$
F.$-\frac{173}{107}$ G.$-1$
H.$(\frac{173}{107})^{2}$
I.$(\frac{107}{173})^{2}$
\testStop
\kluczStart
A
\kluczStop



\zadStart{Zadanie z Wikieł Z 1.1 d) moja wersja nr 43}

Obliczyć wartość wyrażenia $(\frac{173}{109})^{2} \cdot (\frac{109}{173})^{2} \cdot \pi^{0}$.
\zadStop
\rozwStart{Patryk Wirkus}{Martyna Czarnobaj}
$$(\frac{173}{109})^{2} \cdot (\frac{109}{173})^{2} \cdot \pi^{0} = (\frac{173}{109} \cdot \frac{109}{173})^{2} \cdot 1 = 1^{2} \cdot 1 = 1$$
\rozwStop
\odpStart
$1$
\odpStop
\testStart
A.$1$ B.$\pi$ C.$0$ D.$\frac{173}{109}$ E.$\frac{109}{173}$
F.$-\frac{173}{109}$ G.$-1$
H.$(\frac{173}{109})^{2}$
I.$(\frac{109}{173})^{2}$
\testStop
\kluczStart
A
\kluczStop



\zadStart{Zadanie z Wikieł Z 1.1 d) moja wersja nr 44}

Obliczyć wartość wyrażenia $(\frac{173}{113})^{2} \cdot (\frac{113}{173})^{2} \cdot \pi^{0}$.
\zadStop
\rozwStart{Patryk Wirkus}{Martyna Czarnobaj}
$$(\frac{173}{113})^{2} \cdot (\frac{113}{173})^{2} \cdot \pi^{0} = (\frac{173}{113} \cdot \frac{113}{173})^{2} \cdot 1 = 1^{2} \cdot 1 = 1$$
\rozwStop
\odpStart
$1$
\odpStop
\testStart
A.$1$ B.$\pi$ C.$0$ D.$\frac{173}{113}$ E.$\frac{113}{173}$
F.$-\frac{173}{113}$ G.$-1$
H.$(\frac{173}{113})^{2}$
I.$(\frac{113}{173})^{2}$
\testStop
\kluczStart
A
\kluczStop



\zadStart{Zadanie z Wikieł Z 1.1 d) moja wersja nr 45}

Obliczyć wartość wyrażenia $(\frac{173}{127})^{2} \cdot (\frac{127}{173})^{2} \cdot \pi^{0}$.
\zadStop
\rozwStart{Patryk Wirkus}{Martyna Czarnobaj}
$$(\frac{173}{127})^{2} \cdot (\frac{127}{173})^{2} \cdot \pi^{0} = (\frac{173}{127} \cdot \frac{127}{173})^{2} \cdot 1 = 1^{2} \cdot 1 = 1$$
\rozwStop
\odpStart
$1$
\odpStop
\testStart
A.$1$ B.$\pi$ C.$0$ D.$\frac{173}{127}$ E.$\frac{127}{173}$
F.$-\frac{173}{127}$ G.$-1$
H.$(\frac{173}{127})^{2}$
I.$(\frac{127}{173})^{2}$
\testStop
\kluczStart
A
\kluczStop



\zadStart{Zadanie z Wikieł Z 1.1 d) moja wersja nr 46}

Obliczyć wartość wyrażenia $(\frac{173}{131})^{2} \cdot (\frac{131}{173})^{2} \cdot \pi^{0}$.
\zadStop
\rozwStart{Patryk Wirkus}{Martyna Czarnobaj}
$$(\frac{173}{131})^{2} \cdot (\frac{131}{173})^{2} \cdot \pi^{0} = (\frac{173}{131} \cdot \frac{131}{173})^{2} \cdot 1 = 1^{2} \cdot 1 = 1$$
\rozwStop
\odpStart
$1$
\odpStop
\testStart
A.$1$ B.$\pi$ C.$0$ D.$\frac{173}{131}$ E.$\frac{131}{173}$
F.$-\frac{173}{131}$ G.$-1$
H.$(\frac{173}{131})^{2}$
I.$(\frac{131}{173})^{2}$
\testStop
\kluczStart
A
\kluczStop



\zadStart{Zadanie z Wikieł Z 1.1 d) moja wersja nr 47}

Obliczyć wartość wyrażenia $(\frac{173}{137})^{2} \cdot (\frac{137}{173})^{2} \cdot \pi^{0}$.
\zadStop
\rozwStart{Patryk Wirkus}{Martyna Czarnobaj}
$$(\frac{173}{137})^{2} \cdot (\frac{137}{173})^{2} \cdot \pi^{0} = (\frac{173}{137} \cdot \frac{137}{173})^{2} \cdot 1 = 1^{2} \cdot 1 = 1$$
\rozwStop
\odpStart
$1$
\odpStop
\testStart
A.$1$ B.$\pi$ C.$0$ D.$\frac{173}{137}$ E.$\frac{137}{173}$
F.$-\frac{173}{137}$ G.$-1$
H.$(\frac{173}{137})^{2}$
I.$(\frac{137}{173})^{2}$
\testStop
\kluczStart
A
\kluczStop



\zadStart{Zadanie z Wikieł Z 1.1 d) moja wersja nr 48}

Obliczyć wartość wyrażenia $(\frac{173}{139})^{2} \cdot (\frac{139}{173})^{2} \cdot \pi^{0}$.
\zadStop
\rozwStart{Patryk Wirkus}{Martyna Czarnobaj}
$$(\frac{173}{139})^{2} \cdot (\frac{139}{173})^{2} \cdot \pi^{0} = (\frac{173}{139} \cdot \frac{139}{173})^{2} \cdot 1 = 1^{2} \cdot 1 = 1$$
\rozwStop
\odpStart
$1$
\odpStop
\testStart
A.$1$ B.$\pi$ C.$0$ D.$\frac{173}{139}$ E.$\frac{139}{173}$
F.$-\frac{173}{139}$ G.$-1$
H.$(\frac{173}{139})^{2}$
I.$(\frac{139}{173})^{2}$
\testStop
\kluczStart
A
\kluczStop



\zadStart{Zadanie z Wikieł Z 1.1 d) moja wersja nr 49}

Obliczyć wartość wyrażenia $(\frac{179}{103})^{2} \cdot (\frac{103}{179})^{2} \cdot \pi^{0}$.
\zadStop
\rozwStart{Patryk Wirkus}{Martyna Czarnobaj}
$$(\frac{179}{103})^{2} \cdot (\frac{103}{179})^{2} \cdot \pi^{0} = (\frac{179}{103} \cdot \frac{103}{179})^{2} \cdot 1 = 1^{2} \cdot 1 = 1$$
\rozwStop
\odpStart
$1$
\odpStop
\testStart
A.$1$ B.$\pi$ C.$0$ D.$\frac{179}{103}$ E.$\frac{103}{179}$
F.$-\frac{179}{103}$ G.$-1$
H.$(\frac{179}{103})^{2}$
I.$(\frac{103}{179})^{2}$
\testStop
\kluczStart
A
\kluczStop



\zadStart{Zadanie z Wikieł Z 1.1 d) moja wersja nr 50}

Obliczyć wartość wyrażenia $(\frac{179}{107})^{2} \cdot (\frac{107}{179})^{2} \cdot \pi^{0}$.
\zadStop
\rozwStart{Patryk Wirkus}{Martyna Czarnobaj}
$$(\frac{179}{107})^{2} \cdot (\frac{107}{179})^{2} \cdot \pi^{0} = (\frac{179}{107} \cdot \frac{107}{179})^{2} \cdot 1 = 1^{2} \cdot 1 = 1$$
\rozwStop
\odpStart
$1$
\odpStop
\testStart
A.$1$ B.$\pi$ C.$0$ D.$\frac{179}{107}$ E.$\frac{107}{179}$
F.$-\frac{179}{107}$ G.$-1$
H.$(\frac{179}{107})^{2}$
I.$(\frac{107}{179})^{2}$
\testStop
\kluczStart
A
\kluczStop



\zadStart{Zadanie z Wikieł Z 1.1 d) moja wersja nr 51}

Obliczyć wartość wyrażenia $(\frac{179}{109})^{2} \cdot (\frac{109}{179})^{2} \cdot \pi^{0}$.
\zadStop
\rozwStart{Patryk Wirkus}{Martyna Czarnobaj}
$$(\frac{179}{109})^{2} \cdot (\frac{109}{179})^{2} \cdot \pi^{0} = (\frac{179}{109} \cdot \frac{109}{179})^{2} \cdot 1 = 1^{2} \cdot 1 = 1$$
\rozwStop
\odpStart
$1$
\odpStop
\testStart
A.$1$ B.$\pi$ C.$0$ D.$\frac{179}{109}$ E.$\frac{109}{179}$
F.$-\frac{179}{109}$ G.$-1$
H.$(\frac{179}{109})^{2}$
I.$(\frac{109}{179})^{2}$
\testStop
\kluczStart
A
\kluczStop



\zadStart{Zadanie z Wikieł Z 1.1 d) moja wersja nr 52}

Obliczyć wartość wyrażenia $(\frac{179}{113})^{2} \cdot (\frac{113}{179})^{2} \cdot \pi^{0}$.
\zadStop
\rozwStart{Patryk Wirkus}{Martyna Czarnobaj}
$$(\frac{179}{113})^{2} \cdot (\frac{113}{179})^{2} \cdot \pi^{0} = (\frac{179}{113} \cdot \frac{113}{179})^{2} \cdot 1 = 1^{2} \cdot 1 = 1$$
\rozwStop
\odpStart
$1$
\odpStop
\testStart
A.$1$ B.$\pi$ C.$0$ D.$\frac{179}{113}$ E.$\frac{113}{179}$
F.$-\frac{179}{113}$ G.$-1$
H.$(\frac{179}{113})^{2}$
I.$(\frac{113}{179})^{2}$
\testStop
\kluczStart
A
\kluczStop



\zadStart{Zadanie z Wikieł Z 1.1 d) moja wersja nr 53}

Obliczyć wartość wyrażenia $(\frac{179}{127})^{2} \cdot (\frac{127}{179})^{2} \cdot \pi^{0}$.
\zadStop
\rozwStart{Patryk Wirkus}{Martyna Czarnobaj}
$$(\frac{179}{127})^{2} \cdot (\frac{127}{179})^{2} \cdot \pi^{0} = (\frac{179}{127} \cdot \frac{127}{179})^{2} \cdot 1 = 1^{2} \cdot 1 = 1$$
\rozwStop
\odpStart
$1$
\odpStop
\testStart
A.$1$ B.$\pi$ C.$0$ D.$\frac{179}{127}$ E.$\frac{127}{179}$
F.$-\frac{179}{127}$ G.$-1$
H.$(\frac{179}{127})^{2}$
I.$(\frac{127}{179})^{2}$
\testStop
\kluczStart
A
\kluczStop



\zadStart{Zadanie z Wikieł Z 1.1 d) moja wersja nr 54}

Obliczyć wartość wyrażenia $(\frac{179}{131})^{2} \cdot (\frac{131}{179})^{2} \cdot \pi^{0}$.
\zadStop
\rozwStart{Patryk Wirkus}{Martyna Czarnobaj}
$$(\frac{179}{131})^{2} \cdot (\frac{131}{179})^{2} \cdot \pi^{0} = (\frac{179}{131} \cdot \frac{131}{179})^{2} \cdot 1 = 1^{2} \cdot 1 = 1$$
\rozwStop
\odpStart
$1$
\odpStop
\testStart
A.$1$ B.$\pi$ C.$0$ D.$\frac{179}{131}$ E.$\frac{131}{179}$
F.$-\frac{179}{131}$ G.$-1$
H.$(\frac{179}{131})^{2}$
I.$(\frac{131}{179})^{2}$
\testStop
\kluczStart
A
\kluczStop



\zadStart{Zadanie z Wikieł Z 1.1 d) moja wersja nr 55}

Obliczyć wartość wyrażenia $(\frac{179}{137})^{2} \cdot (\frac{137}{179})^{2} \cdot \pi^{0}$.
\zadStop
\rozwStart{Patryk Wirkus}{Martyna Czarnobaj}
$$(\frac{179}{137})^{2} \cdot (\frac{137}{179})^{2} \cdot \pi^{0} = (\frac{179}{137} \cdot \frac{137}{179})^{2} \cdot 1 = 1^{2} \cdot 1 = 1$$
\rozwStop
\odpStart
$1$
\odpStop
\testStart
A.$1$ B.$\pi$ C.$0$ D.$\frac{179}{137}$ E.$\frac{137}{179}$
F.$-\frac{179}{137}$ G.$-1$
H.$(\frac{179}{137})^{2}$
I.$(\frac{137}{179})^{2}$
\testStop
\kluczStart
A
\kluczStop



\zadStart{Zadanie z Wikieł Z 1.1 d) moja wersja nr 56}

Obliczyć wartość wyrażenia $(\frac{179}{139})^{2} \cdot (\frac{139}{179})^{2} \cdot \pi^{0}$.
\zadStop
\rozwStart{Patryk Wirkus}{Martyna Czarnobaj}
$$(\frac{179}{139})^{2} \cdot (\frac{139}{179})^{2} \cdot \pi^{0} = (\frac{179}{139} \cdot \frac{139}{179})^{2} \cdot 1 = 1^{2} \cdot 1 = 1$$
\rozwStop
\odpStart
$1$
\odpStop
\testStart
A.$1$ B.$\pi$ C.$0$ D.$\frac{179}{139}$ E.$\frac{139}{179}$
F.$-\frac{179}{139}$ G.$-1$
H.$(\frac{179}{139})^{2}$
I.$(\frac{139}{179})^{2}$
\testStop
\kluczStart
A
\kluczStop



\zadStart{Zadanie z Wikieł Z 1.1 d) moja wersja nr 57}

Obliczyć wartość wyrażenia $(\frac{251}{103})^{2} \cdot (\frac{103}{251})^{2} \cdot \pi^{0}$.
\zadStop
\rozwStart{Patryk Wirkus}{Martyna Czarnobaj}
$$(\frac{251}{103})^{2} \cdot (\frac{103}{251})^{2} \cdot \pi^{0} = (\frac{251}{103} \cdot \frac{103}{251})^{2} \cdot 1 = 1^{2} \cdot 1 = 1$$
\rozwStop
\odpStart
$1$
\odpStop
\testStart
A.$1$ B.$\pi$ C.$0$ D.$\frac{251}{103}$ E.$\frac{103}{251}$
F.$-\frac{251}{103}$ G.$-1$
H.$(\frac{251}{103})^{2}$
I.$(\frac{103}{251})^{2}$
\testStop
\kluczStart
A
\kluczStop



\zadStart{Zadanie z Wikieł Z 1.1 d) moja wersja nr 58}

Obliczyć wartość wyrażenia $(\frac{251}{107})^{2} \cdot (\frac{107}{251})^{2} \cdot \pi^{0}$.
\zadStop
\rozwStart{Patryk Wirkus}{Martyna Czarnobaj}
$$(\frac{251}{107})^{2} \cdot (\frac{107}{251})^{2} \cdot \pi^{0} = (\frac{251}{107} \cdot \frac{107}{251})^{2} \cdot 1 = 1^{2} \cdot 1 = 1$$
\rozwStop
\odpStart
$1$
\odpStop
\testStart
A.$1$ B.$\pi$ C.$0$ D.$\frac{251}{107}$ E.$\frac{107}{251}$
F.$-\frac{251}{107}$ G.$-1$
H.$(\frac{251}{107})^{2}$
I.$(\frac{107}{251})^{2}$
\testStop
\kluczStart
A
\kluczStop



\zadStart{Zadanie z Wikieł Z 1.1 d) moja wersja nr 59}

Obliczyć wartość wyrażenia $(\frac{251}{109})^{2} \cdot (\frac{109}{251})^{2} \cdot \pi^{0}$.
\zadStop
\rozwStart{Patryk Wirkus}{Martyna Czarnobaj}
$$(\frac{251}{109})^{2} \cdot (\frac{109}{251})^{2} \cdot \pi^{0} = (\frac{251}{109} \cdot \frac{109}{251})^{2} \cdot 1 = 1^{2} \cdot 1 = 1$$
\rozwStop
\odpStart
$1$
\odpStop
\testStart
A.$1$ B.$\pi$ C.$0$ D.$\frac{251}{109}$ E.$\frac{109}{251}$
F.$-\frac{251}{109}$ G.$-1$
H.$(\frac{251}{109})^{2}$
I.$(\frac{109}{251})^{2}$
\testStop
\kluczStart
A
\kluczStop



\zadStart{Zadanie z Wikieł Z 1.1 d) moja wersja nr 60}

Obliczyć wartość wyrażenia $(\frac{251}{113})^{2} \cdot (\frac{113}{251})^{2} \cdot \pi^{0}$.
\zadStop
\rozwStart{Patryk Wirkus}{Martyna Czarnobaj}
$$(\frac{251}{113})^{2} \cdot (\frac{113}{251})^{2} \cdot \pi^{0} = (\frac{251}{113} \cdot \frac{113}{251})^{2} \cdot 1 = 1^{2} \cdot 1 = 1$$
\rozwStop
\odpStart
$1$
\odpStop
\testStart
A.$1$ B.$\pi$ C.$0$ D.$\frac{251}{113}$ E.$\frac{113}{251}$
F.$-\frac{251}{113}$ G.$-1$
H.$(\frac{251}{113})^{2}$
I.$(\frac{113}{251})^{2}$
\testStop
\kluczStart
A
\kluczStop



\zadStart{Zadanie z Wikieł Z 1.1 d) moja wersja nr 61}

Obliczyć wartość wyrażenia $(\frac{251}{127})^{2} \cdot (\frac{127}{251})^{2} \cdot \pi^{0}$.
\zadStop
\rozwStart{Patryk Wirkus}{Martyna Czarnobaj}
$$(\frac{251}{127})^{2} \cdot (\frac{127}{251})^{2} \cdot \pi^{0} = (\frac{251}{127} \cdot \frac{127}{251})^{2} \cdot 1 = 1^{2} \cdot 1 = 1$$
\rozwStop
\odpStart
$1$
\odpStop
\testStart
A.$1$ B.$\pi$ C.$0$ D.$\frac{251}{127}$ E.$\frac{127}{251}$
F.$-\frac{251}{127}$ G.$-1$
H.$(\frac{251}{127})^{2}$
I.$(\frac{127}{251})^{2}$
\testStop
\kluczStart
A
\kluczStop



\zadStart{Zadanie z Wikieł Z 1.1 d) moja wersja nr 62}

Obliczyć wartość wyrażenia $(\frac{251}{131})^{2} \cdot (\frac{131}{251})^{2} \cdot \pi^{0}$.
\zadStop
\rozwStart{Patryk Wirkus}{Martyna Czarnobaj}
$$(\frac{251}{131})^{2} \cdot (\frac{131}{251})^{2} \cdot \pi^{0} = (\frac{251}{131} \cdot \frac{131}{251})^{2} \cdot 1 = 1^{2} \cdot 1 = 1$$
\rozwStop
\odpStart
$1$
\odpStop
\testStart
A.$1$ B.$\pi$ C.$0$ D.$\frac{251}{131}$ E.$\frac{131}{251}$
F.$-\frac{251}{131}$ G.$-1$
H.$(\frac{251}{131})^{2}$
I.$(\frac{131}{251})^{2}$
\testStop
\kluczStart
A
\kluczStop



\zadStart{Zadanie z Wikieł Z 1.1 d) moja wersja nr 63}

Obliczyć wartość wyrażenia $(\frac{251}{137})^{2} \cdot (\frac{137}{251})^{2} \cdot \pi^{0}$.
\zadStop
\rozwStart{Patryk Wirkus}{Martyna Czarnobaj}
$$(\frac{251}{137})^{2} \cdot (\frac{137}{251})^{2} \cdot \pi^{0} = (\frac{251}{137} \cdot \frac{137}{251})^{2} \cdot 1 = 1^{2} \cdot 1 = 1$$
\rozwStop
\odpStart
$1$
\odpStop
\testStart
A.$1$ B.$\pi$ C.$0$ D.$\frac{251}{137}$ E.$\frac{137}{251}$
F.$-\frac{251}{137}$ G.$-1$
H.$(\frac{251}{137})^{2}$
I.$(\frac{137}{251})^{2}$
\testStop
\kluczStart
A
\kluczStop



\zadStart{Zadanie z Wikieł Z 1.1 d) moja wersja nr 64}

Obliczyć wartość wyrażenia $(\frac{251}{139})^{2} \cdot (\frac{139}{251})^{2} \cdot \pi^{0}$.
\zadStop
\rozwStart{Patryk Wirkus}{Martyna Czarnobaj}
$$(\frac{251}{139})^{2} \cdot (\frac{139}{251})^{2} \cdot \pi^{0} = (\frac{251}{139} \cdot \frac{139}{251})^{2} \cdot 1 = 1^{2} \cdot 1 = 1$$
\rozwStop
\odpStart
$1$
\odpStop
\testStart
A.$1$ B.$\pi$ C.$0$ D.$\frac{251}{139}$ E.$\frac{139}{251}$
F.$-\frac{251}{139}$ G.$-1$
H.$(\frac{251}{139})^{2}$
I.$(\frac{139}{251})^{2}$
\testStop
\kluczStart
A
\kluczStop



\zadStart{Zadanie z Wikieł Z 1.1 d) moja wersja nr 65}

Obliczyć wartość wyrażenia $(\frac{257}{103})^{2} \cdot (\frac{103}{257})^{2} \cdot \pi^{0}$.
\zadStop
\rozwStart{Patryk Wirkus}{Martyna Czarnobaj}
$$(\frac{257}{103})^{2} \cdot (\frac{103}{257})^{2} \cdot \pi^{0} = (\frac{257}{103} \cdot \frac{103}{257})^{2} \cdot 1 = 1^{2} \cdot 1 = 1$$
\rozwStop
\odpStart
$1$
\odpStop
\testStart
A.$1$ B.$\pi$ C.$0$ D.$\frac{257}{103}$ E.$\frac{103}{257}$
F.$-\frac{257}{103}$ G.$-1$
H.$(\frac{257}{103})^{2}$
I.$(\frac{103}{257})^{2}$
\testStop
\kluczStart
A
\kluczStop



\zadStart{Zadanie z Wikieł Z 1.1 d) moja wersja nr 66}

Obliczyć wartość wyrażenia $(\frac{257}{107})^{2} \cdot (\frac{107}{257})^{2} \cdot \pi^{0}$.
\zadStop
\rozwStart{Patryk Wirkus}{Martyna Czarnobaj}
$$(\frac{257}{107})^{2} \cdot (\frac{107}{257})^{2} \cdot \pi^{0} = (\frac{257}{107} \cdot \frac{107}{257})^{2} \cdot 1 = 1^{2} \cdot 1 = 1$$
\rozwStop
\odpStart
$1$
\odpStop
\testStart
A.$1$ B.$\pi$ C.$0$ D.$\frac{257}{107}$ E.$\frac{107}{257}$
F.$-\frac{257}{107}$ G.$-1$
H.$(\frac{257}{107})^{2}$
I.$(\frac{107}{257})^{2}$
\testStop
\kluczStart
A
\kluczStop



\zadStart{Zadanie z Wikieł Z 1.1 d) moja wersja nr 67}

Obliczyć wartość wyrażenia $(\frac{257}{109})^{2} \cdot (\frac{109}{257})^{2} \cdot \pi^{0}$.
\zadStop
\rozwStart{Patryk Wirkus}{Martyna Czarnobaj}
$$(\frac{257}{109})^{2} \cdot (\frac{109}{257})^{2} \cdot \pi^{0} = (\frac{257}{109} \cdot \frac{109}{257})^{2} \cdot 1 = 1^{2} \cdot 1 = 1$$
\rozwStop
\odpStart
$1$
\odpStop
\testStart
A.$1$ B.$\pi$ C.$0$ D.$\frac{257}{109}$ E.$\frac{109}{257}$
F.$-\frac{257}{109}$ G.$-1$
H.$(\frac{257}{109})^{2}$
I.$(\frac{109}{257})^{2}$
\testStop
\kluczStart
A
\kluczStop



\zadStart{Zadanie z Wikieł Z 1.1 d) moja wersja nr 68}

Obliczyć wartość wyrażenia $(\frac{257}{113})^{2} \cdot (\frac{113}{257})^{2} \cdot \pi^{0}$.
\zadStop
\rozwStart{Patryk Wirkus}{Martyna Czarnobaj}
$$(\frac{257}{113})^{2} \cdot (\frac{113}{257})^{2} \cdot \pi^{0} = (\frac{257}{113} \cdot \frac{113}{257})^{2} \cdot 1 = 1^{2} \cdot 1 = 1$$
\rozwStop
\odpStart
$1$
\odpStop
\testStart
A.$1$ B.$\pi$ C.$0$ D.$\frac{257}{113}$ E.$\frac{113}{257}$
F.$-\frac{257}{113}$ G.$-1$
H.$(\frac{257}{113})^{2}$
I.$(\frac{113}{257})^{2}$
\testStop
\kluczStart
A
\kluczStop



\zadStart{Zadanie z Wikieł Z 1.1 d) moja wersja nr 69}

Obliczyć wartość wyrażenia $(\frac{257}{127})^{2} \cdot (\frac{127}{257})^{2} \cdot \pi^{0}$.
\zadStop
\rozwStart{Patryk Wirkus}{Martyna Czarnobaj}
$$(\frac{257}{127})^{2} \cdot (\frac{127}{257})^{2} \cdot \pi^{0} = (\frac{257}{127} \cdot \frac{127}{257})^{2} \cdot 1 = 1^{2} \cdot 1 = 1$$
\rozwStop
\odpStart
$1$
\odpStop
\testStart
A.$1$ B.$\pi$ C.$0$ D.$\frac{257}{127}$ E.$\frac{127}{257}$
F.$-\frac{257}{127}$ G.$-1$
H.$(\frac{257}{127})^{2}$
I.$(\frac{127}{257})^{2}$
\testStop
\kluczStart
A
\kluczStop



\zadStart{Zadanie z Wikieł Z 1.1 d) moja wersja nr 70}

Obliczyć wartość wyrażenia $(\frac{257}{131})^{2} \cdot (\frac{131}{257})^{2} \cdot \pi^{0}$.
\zadStop
\rozwStart{Patryk Wirkus}{Martyna Czarnobaj}
$$(\frac{257}{131})^{2} \cdot (\frac{131}{257})^{2} \cdot \pi^{0} = (\frac{257}{131} \cdot \frac{131}{257})^{2} \cdot 1 = 1^{2} \cdot 1 = 1$$
\rozwStop
\odpStart
$1$
\odpStop
\testStart
A.$1$ B.$\pi$ C.$0$ D.$\frac{257}{131}$ E.$\frac{131}{257}$
F.$-\frac{257}{131}$ G.$-1$
H.$(\frac{257}{131})^{2}$
I.$(\frac{131}{257})^{2}$
\testStop
\kluczStart
A
\kluczStop



\zadStart{Zadanie z Wikieł Z 1.1 d) moja wersja nr 71}

Obliczyć wartość wyrażenia $(\frac{257}{137})^{2} \cdot (\frac{137}{257})^{2} \cdot \pi^{0}$.
\zadStop
\rozwStart{Patryk Wirkus}{Martyna Czarnobaj}
$$(\frac{257}{137})^{2} \cdot (\frac{137}{257})^{2} \cdot \pi^{0} = (\frac{257}{137} \cdot \frac{137}{257})^{2} \cdot 1 = 1^{2} \cdot 1 = 1$$
\rozwStop
\odpStart
$1$
\odpStop
\testStart
A.$1$ B.$\pi$ C.$0$ D.$\frac{257}{137}$ E.$\frac{137}{257}$
F.$-\frac{257}{137}$ G.$-1$
H.$(\frac{257}{137})^{2}$
I.$(\frac{137}{257})^{2}$
\testStop
\kluczStart
A
\kluczStop



\zadStart{Zadanie z Wikieł Z 1.1 d) moja wersja nr 72}

Obliczyć wartość wyrażenia $(\frac{257}{139})^{2} \cdot (\frac{139}{257})^{2} \cdot \pi^{0}$.
\zadStop
\rozwStart{Patryk Wirkus}{Martyna Czarnobaj}
$$(\frac{257}{139})^{2} \cdot (\frac{139}{257})^{2} \cdot \pi^{0} = (\frac{257}{139} \cdot \frac{139}{257})^{2} \cdot 1 = 1^{2} \cdot 1 = 1$$
\rozwStop
\odpStart
$1$
\odpStop
\testStart
A.$1$ B.$\pi$ C.$0$ D.$\frac{257}{139}$ E.$\frac{139}{257}$
F.$-\frac{257}{139}$ G.$-1$
H.$(\frac{257}{139})^{2}$
I.$(\frac{139}{257})^{2}$
\testStop
\kluczStart
A
\kluczStop



\zadStart{Zadanie z Wikieł Z 1.1 d) moja wersja nr 73}

Obliczyć wartość wyrażenia $(\frac{263}{103})^{2} \cdot (\frac{103}{263})^{2} \cdot \pi^{0}$.
\zadStop
\rozwStart{Patryk Wirkus}{Martyna Czarnobaj}
$$(\frac{263}{103})^{2} \cdot (\frac{103}{263})^{2} \cdot \pi^{0} = (\frac{263}{103} \cdot \frac{103}{263})^{2} \cdot 1 = 1^{2} \cdot 1 = 1$$
\rozwStop
\odpStart
$1$
\odpStop
\testStart
A.$1$ B.$\pi$ C.$0$ D.$\frac{263}{103}$ E.$\frac{103}{263}$
F.$-\frac{263}{103}$ G.$-1$
H.$(\frac{263}{103})^{2}$
I.$(\frac{103}{263})^{2}$
\testStop
\kluczStart
A
\kluczStop



\zadStart{Zadanie z Wikieł Z 1.1 d) moja wersja nr 74}

Obliczyć wartość wyrażenia $(\frac{263}{107})^{2} \cdot (\frac{107}{263})^{2} \cdot \pi^{0}$.
\zadStop
\rozwStart{Patryk Wirkus}{Martyna Czarnobaj}
$$(\frac{263}{107})^{2} \cdot (\frac{107}{263})^{2} \cdot \pi^{0} = (\frac{263}{107} \cdot \frac{107}{263})^{2} \cdot 1 = 1^{2} \cdot 1 = 1$$
\rozwStop
\odpStart
$1$
\odpStop
\testStart
A.$1$ B.$\pi$ C.$0$ D.$\frac{263}{107}$ E.$\frac{107}{263}$
F.$-\frac{263}{107}$ G.$-1$
H.$(\frac{263}{107})^{2}$
I.$(\frac{107}{263})^{2}$
\testStop
\kluczStart
A
\kluczStop



\zadStart{Zadanie z Wikieł Z 1.1 d) moja wersja nr 75}

Obliczyć wartość wyrażenia $(\frac{263}{109})^{2} \cdot (\frac{109}{263})^{2} \cdot \pi^{0}$.
\zadStop
\rozwStart{Patryk Wirkus}{Martyna Czarnobaj}
$$(\frac{263}{109})^{2} \cdot (\frac{109}{263})^{2} \cdot \pi^{0} = (\frac{263}{109} \cdot \frac{109}{263})^{2} \cdot 1 = 1^{2} \cdot 1 = 1$$
\rozwStop
\odpStart
$1$
\odpStop
\testStart
A.$1$ B.$\pi$ C.$0$ D.$\frac{263}{109}$ E.$\frac{109}{263}$
F.$-\frac{263}{109}$ G.$-1$
H.$(\frac{263}{109})^{2}$
I.$(\frac{109}{263})^{2}$
\testStop
\kluczStart
A
\kluczStop



\zadStart{Zadanie z Wikieł Z 1.1 d) moja wersja nr 76}

Obliczyć wartość wyrażenia $(\frac{263}{113})^{2} \cdot (\frac{113}{263})^{2} \cdot \pi^{0}$.
\zadStop
\rozwStart{Patryk Wirkus}{Martyna Czarnobaj}
$$(\frac{263}{113})^{2} \cdot (\frac{113}{263})^{2} \cdot \pi^{0} = (\frac{263}{113} \cdot \frac{113}{263})^{2} \cdot 1 = 1^{2} \cdot 1 = 1$$
\rozwStop
\odpStart
$1$
\odpStop
\testStart
A.$1$ B.$\pi$ C.$0$ D.$\frac{263}{113}$ E.$\frac{113}{263}$
F.$-\frac{263}{113}$ G.$-1$
H.$(\frac{263}{113})^{2}$
I.$(\frac{113}{263})^{2}$
\testStop
\kluczStart
A
\kluczStop



\zadStart{Zadanie z Wikieł Z 1.1 d) moja wersja nr 77}

Obliczyć wartość wyrażenia $(\frac{263}{127})^{2} \cdot (\frac{127}{263})^{2} \cdot \pi^{0}$.
\zadStop
\rozwStart{Patryk Wirkus}{Martyna Czarnobaj}
$$(\frac{263}{127})^{2} \cdot (\frac{127}{263})^{2} \cdot \pi^{0} = (\frac{263}{127} \cdot \frac{127}{263})^{2} \cdot 1 = 1^{2} \cdot 1 = 1$$
\rozwStop
\odpStart
$1$
\odpStop
\testStart
A.$1$ B.$\pi$ C.$0$ D.$\frac{263}{127}$ E.$\frac{127}{263}$
F.$-\frac{263}{127}$ G.$-1$
H.$(\frac{263}{127})^{2}$
I.$(\frac{127}{263})^{2}$
\testStop
\kluczStart
A
\kluczStop



\zadStart{Zadanie z Wikieł Z 1.1 d) moja wersja nr 78}

Obliczyć wartość wyrażenia $(\frac{263}{131})^{2} \cdot (\frac{131}{263})^{2} \cdot \pi^{0}$.
\zadStop
\rozwStart{Patryk Wirkus}{Martyna Czarnobaj}
$$(\frac{263}{131})^{2} \cdot (\frac{131}{263})^{2} \cdot \pi^{0} = (\frac{263}{131} \cdot \frac{131}{263})^{2} \cdot 1 = 1^{2} \cdot 1 = 1$$
\rozwStop
\odpStart
$1$
\odpStop
\testStart
A.$1$ B.$\pi$ C.$0$ D.$\frac{263}{131}$ E.$\frac{131}{263}$
F.$-\frac{263}{131}$ G.$-1$
H.$(\frac{263}{131})^{2}$
I.$(\frac{131}{263})^{2}$
\testStop
\kluczStart
A
\kluczStop



\zadStart{Zadanie z Wikieł Z 1.1 d) moja wersja nr 79}

Obliczyć wartość wyrażenia $(\frac{263}{137})^{2} \cdot (\frac{137}{263})^{2} \cdot \pi^{0}$.
\zadStop
\rozwStart{Patryk Wirkus}{Martyna Czarnobaj}
$$(\frac{263}{137})^{2} \cdot (\frac{137}{263})^{2} \cdot \pi^{0} = (\frac{263}{137} \cdot \frac{137}{263})^{2} \cdot 1 = 1^{2} \cdot 1 = 1$$
\rozwStop
\odpStart
$1$
\odpStop
\testStart
A.$1$ B.$\pi$ C.$0$ D.$\frac{263}{137}$ E.$\frac{137}{263}$
F.$-\frac{263}{137}$ G.$-1$
H.$(\frac{263}{137})^{2}$
I.$(\frac{137}{263})^{2}$
\testStop
\kluczStart
A
\kluczStop



\zadStart{Zadanie z Wikieł Z 1.1 d) moja wersja nr 80}

Obliczyć wartość wyrażenia $(\frac{263}{139})^{2} \cdot (\frac{139}{263})^{2} \cdot \pi^{0}$.
\zadStop
\rozwStart{Patryk Wirkus}{Martyna Czarnobaj}
$$(\frac{263}{139})^{2} \cdot (\frac{139}{263})^{2} \cdot \pi^{0} = (\frac{263}{139} \cdot \frac{139}{263})^{2} \cdot 1 = 1^{2} \cdot 1 = 1$$
\rozwStop
\odpStart
$1$
\odpStop
\testStart
A.$1$ B.$\pi$ C.$0$ D.$\frac{263}{139}$ E.$\frac{139}{263}$
F.$-\frac{263}{139}$ G.$-1$
H.$(\frac{263}{139})^{2}$
I.$(\frac{139}{263})^{2}$
\testStop
\kluczStart
A
\kluczStop



\zadStart{Zadanie z Wikieł Z 1.1 d) moja wersja nr 81}

Obliczyć wartość wyrażenia $(\frac{269}{103})^{2} \cdot (\frac{103}{269})^{2} \cdot \pi^{0}$.
\zadStop
\rozwStart{Patryk Wirkus}{Martyna Czarnobaj}
$$(\frac{269}{103})^{2} \cdot (\frac{103}{269})^{2} \cdot \pi^{0} = (\frac{269}{103} \cdot \frac{103}{269})^{2} \cdot 1 = 1^{2} \cdot 1 = 1$$
\rozwStop
\odpStart
$1$
\odpStop
\testStart
A.$1$ B.$\pi$ C.$0$ D.$\frac{269}{103}$ E.$\frac{103}{269}$
F.$-\frac{269}{103}$ G.$-1$
H.$(\frac{269}{103})^{2}$
I.$(\frac{103}{269})^{2}$
\testStop
\kluczStart
A
\kluczStop



\zadStart{Zadanie z Wikieł Z 1.1 d) moja wersja nr 82}

Obliczyć wartość wyrażenia $(\frac{269}{107})^{2} \cdot (\frac{107}{269})^{2} \cdot \pi^{0}$.
\zadStop
\rozwStart{Patryk Wirkus}{Martyna Czarnobaj}
$$(\frac{269}{107})^{2} \cdot (\frac{107}{269})^{2} \cdot \pi^{0} = (\frac{269}{107} \cdot \frac{107}{269})^{2} \cdot 1 = 1^{2} \cdot 1 = 1$$
\rozwStop
\odpStart
$1$
\odpStop
\testStart
A.$1$ B.$\pi$ C.$0$ D.$\frac{269}{107}$ E.$\frac{107}{269}$
F.$-\frac{269}{107}$ G.$-1$
H.$(\frac{269}{107})^{2}$
I.$(\frac{107}{269})^{2}$
\testStop
\kluczStart
A
\kluczStop



\zadStart{Zadanie z Wikieł Z 1.1 d) moja wersja nr 83}

Obliczyć wartość wyrażenia $(\frac{269}{109})^{2} \cdot (\frac{109}{269})^{2} \cdot \pi^{0}$.
\zadStop
\rozwStart{Patryk Wirkus}{Martyna Czarnobaj}
$$(\frac{269}{109})^{2} \cdot (\frac{109}{269})^{2} \cdot \pi^{0} = (\frac{269}{109} \cdot \frac{109}{269})^{2} \cdot 1 = 1^{2} \cdot 1 = 1$$
\rozwStop
\odpStart
$1$
\odpStop
\testStart
A.$1$ B.$\pi$ C.$0$ D.$\frac{269}{109}$ E.$\frac{109}{269}$
F.$-\frac{269}{109}$ G.$-1$
H.$(\frac{269}{109})^{2}$
I.$(\frac{109}{269})^{2}$
\testStop
\kluczStart
A
\kluczStop



\zadStart{Zadanie z Wikieł Z 1.1 d) moja wersja nr 84}

Obliczyć wartość wyrażenia $(\frac{269}{113})^{2} \cdot (\frac{113}{269})^{2} \cdot \pi^{0}$.
\zadStop
\rozwStart{Patryk Wirkus}{Martyna Czarnobaj}
$$(\frac{269}{113})^{2} \cdot (\frac{113}{269})^{2} \cdot \pi^{0} = (\frac{269}{113} \cdot \frac{113}{269})^{2} \cdot 1 = 1^{2} \cdot 1 = 1$$
\rozwStop
\odpStart
$1$
\odpStop
\testStart
A.$1$ B.$\pi$ C.$0$ D.$\frac{269}{113}$ E.$\frac{113}{269}$
F.$-\frac{269}{113}$ G.$-1$
H.$(\frac{269}{113})^{2}$
I.$(\frac{113}{269})^{2}$
\testStop
\kluczStart
A
\kluczStop



\zadStart{Zadanie z Wikieł Z 1.1 d) moja wersja nr 85}

Obliczyć wartość wyrażenia $(\frac{269}{127})^{2} \cdot (\frac{127}{269})^{2} \cdot \pi^{0}$.
\zadStop
\rozwStart{Patryk Wirkus}{Martyna Czarnobaj}
$$(\frac{269}{127})^{2} \cdot (\frac{127}{269})^{2} \cdot \pi^{0} = (\frac{269}{127} \cdot \frac{127}{269})^{2} \cdot 1 = 1^{2} \cdot 1 = 1$$
\rozwStop
\odpStart
$1$
\odpStop
\testStart
A.$1$ B.$\pi$ C.$0$ D.$\frac{269}{127}$ E.$\frac{127}{269}$
F.$-\frac{269}{127}$ G.$-1$
H.$(\frac{269}{127})^{2}$
I.$(\frac{127}{269})^{2}$
\testStop
\kluczStart
A
\kluczStop



\zadStart{Zadanie z Wikieł Z 1.1 d) moja wersja nr 86}

Obliczyć wartość wyrażenia $(\frac{269}{131})^{2} \cdot (\frac{131}{269})^{2} \cdot \pi^{0}$.
\zadStop
\rozwStart{Patryk Wirkus}{Martyna Czarnobaj}
$$(\frac{269}{131})^{2} \cdot (\frac{131}{269})^{2} \cdot \pi^{0} = (\frac{269}{131} \cdot \frac{131}{269})^{2} \cdot 1 = 1^{2} \cdot 1 = 1$$
\rozwStop
\odpStart
$1$
\odpStop
\testStart
A.$1$ B.$\pi$ C.$0$ D.$\frac{269}{131}$ E.$\frac{131}{269}$
F.$-\frac{269}{131}$ G.$-1$
H.$(\frac{269}{131})^{2}$
I.$(\frac{131}{269})^{2}$
\testStop
\kluczStart
A
\kluczStop



\zadStart{Zadanie z Wikieł Z 1.1 d) moja wersja nr 87}

Obliczyć wartość wyrażenia $(\frac{269}{137})^{2} \cdot (\frac{137}{269})^{2} \cdot \pi^{0}$.
\zadStop
\rozwStart{Patryk Wirkus}{Martyna Czarnobaj}
$$(\frac{269}{137})^{2} \cdot (\frac{137}{269})^{2} \cdot \pi^{0} = (\frac{269}{137} \cdot \frac{137}{269})^{2} \cdot 1 = 1^{2} \cdot 1 = 1$$
\rozwStop
\odpStart
$1$
\odpStop
\testStart
A.$1$ B.$\pi$ C.$0$ D.$\frac{269}{137}$ E.$\frac{137}{269}$
F.$-\frac{269}{137}$ G.$-1$
H.$(\frac{269}{137})^{2}$
I.$(\frac{137}{269})^{2}$
\testStop
\kluczStart
A
\kluczStop



\zadStart{Zadanie z Wikieł Z 1.1 d) moja wersja nr 88}

Obliczyć wartość wyrażenia $(\frac{269}{139})^{2} \cdot (\frac{139}{269})^{2} \cdot \pi^{0}$.
\zadStop
\rozwStart{Patryk Wirkus}{Martyna Czarnobaj}
$$(\frac{269}{139})^{2} \cdot (\frac{139}{269})^{2} \cdot \pi^{0} = (\frac{269}{139} \cdot \frac{139}{269})^{2} \cdot 1 = 1^{2} \cdot 1 = 1$$
\rozwStop
\odpStart
$1$
\odpStop
\testStart
A.$1$ B.$\pi$ C.$0$ D.$\frac{269}{139}$ E.$\frac{139}{269}$
F.$-\frac{269}{139}$ G.$-1$
H.$(\frac{269}{139})^{2}$
I.$(\frac{139}{269})^{2}$
\testStop
\kluczStart
A
\kluczStop



\zadStart{Zadanie z Wikieł Z 1.1 d) moja wersja nr 89}

Obliczyć wartość wyrażenia $(\frac{271}{103})^{2} \cdot (\frac{103}{271})^{2} \cdot \pi^{0}$.
\zadStop
\rozwStart{Patryk Wirkus}{Martyna Czarnobaj}
$$(\frac{271}{103})^{2} \cdot (\frac{103}{271})^{2} \cdot \pi^{0} = (\frac{271}{103} \cdot \frac{103}{271})^{2} \cdot 1 = 1^{2} \cdot 1 = 1$$
\rozwStop
\odpStart
$1$
\odpStop
\testStart
A.$1$ B.$\pi$ C.$0$ D.$\frac{271}{103}$ E.$\frac{103}{271}$
F.$-\frac{271}{103}$ G.$-1$
H.$(\frac{271}{103})^{2}$
I.$(\frac{103}{271})^{2}$
\testStop
\kluczStart
A
\kluczStop



\zadStart{Zadanie z Wikieł Z 1.1 d) moja wersja nr 90}

Obliczyć wartość wyrażenia $(\frac{271}{107})^{2} \cdot (\frac{107}{271})^{2} \cdot \pi^{0}$.
\zadStop
\rozwStart{Patryk Wirkus}{Martyna Czarnobaj}
$$(\frac{271}{107})^{2} \cdot (\frac{107}{271})^{2} \cdot \pi^{0} = (\frac{271}{107} \cdot \frac{107}{271})^{2} \cdot 1 = 1^{2} \cdot 1 = 1$$
\rozwStop
\odpStart
$1$
\odpStop
\testStart
A.$1$ B.$\pi$ C.$0$ D.$\frac{271}{107}$ E.$\frac{107}{271}$
F.$-\frac{271}{107}$ G.$-1$
H.$(\frac{271}{107})^{2}$
I.$(\frac{107}{271})^{2}$
\testStop
\kluczStart
A
\kluczStop



\zadStart{Zadanie z Wikieł Z 1.1 d) moja wersja nr 91}

Obliczyć wartość wyrażenia $(\frac{271}{109})^{2} \cdot (\frac{109}{271})^{2} \cdot \pi^{0}$.
\zadStop
\rozwStart{Patryk Wirkus}{Martyna Czarnobaj}
$$(\frac{271}{109})^{2} \cdot (\frac{109}{271})^{2} \cdot \pi^{0} = (\frac{271}{109} \cdot \frac{109}{271})^{2} \cdot 1 = 1^{2} \cdot 1 = 1$$
\rozwStop
\odpStart
$1$
\odpStop
\testStart
A.$1$ B.$\pi$ C.$0$ D.$\frac{271}{109}$ E.$\frac{109}{271}$
F.$-\frac{271}{109}$ G.$-1$
H.$(\frac{271}{109})^{2}$
I.$(\frac{109}{271})^{2}$
\testStop
\kluczStart
A
\kluczStop



\zadStart{Zadanie z Wikieł Z 1.1 d) moja wersja nr 92}

Obliczyć wartość wyrażenia $(\frac{271}{113})^{2} \cdot (\frac{113}{271})^{2} \cdot \pi^{0}$.
\zadStop
\rozwStart{Patryk Wirkus}{Martyna Czarnobaj}
$$(\frac{271}{113})^{2} \cdot (\frac{113}{271})^{2} \cdot \pi^{0} = (\frac{271}{113} \cdot \frac{113}{271})^{2} \cdot 1 = 1^{2} \cdot 1 = 1$$
\rozwStop
\odpStart
$1$
\odpStop
\testStart
A.$1$ B.$\pi$ C.$0$ D.$\frac{271}{113}$ E.$\frac{113}{271}$
F.$-\frac{271}{113}$ G.$-1$
H.$(\frac{271}{113})^{2}$
I.$(\frac{113}{271})^{2}$
\testStop
\kluczStart
A
\kluczStop



\zadStart{Zadanie z Wikieł Z 1.1 d) moja wersja nr 93}

Obliczyć wartość wyrażenia $(\frac{271}{127})^{2} \cdot (\frac{127}{271})^{2} \cdot \pi^{0}$.
\zadStop
\rozwStart{Patryk Wirkus}{Martyna Czarnobaj}
$$(\frac{271}{127})^{2} \cdot (\frac{127}{271})^{2} \cdot \pi^{0} = (\frac{271}{127} \cdot \frac{127}{271})^{2} \cdot 1 = 1^{2} \cdot 1 = 1$$
\rozwStop
\odpStart
$1$
\odpStop
\testStart
A.$1$ B.$\pi$ C.$0$ D.$\frac{271}{127}$ E.$\frac{127}{271}$
F.$-\frac{271}{127}$ G.$-1$
H.$(\frac{271}{127})^{2}$
I.$(\frac{127}{271})^{2}$
\testStop
\kluczStart
A
\kluczStop



\zadStart{Zadanie z Wikieł Z 1.1 d) moja wersja nr 94}

Obliczyć wartość wyrażenia $(\frac{271}{131})^{2} \cdot (\frac{131}{271})^{2} \cdot \pi^{0}$.
\zadStop
\rozwStart{Patryk Wirkus}{Martyna Czarnobaj}
$$(\frac{271}{131})^{2} \cdot (\frac{131}{271})^{2} \cdot \pi^{0} = (\frac{271}{131} \cdot \frac{131}{271})^{2} \cdot 1 = 1^{2} \cdot 1 = 1$$
\rozwStop
\odpStart
$1$
\odpStop
\testStart
A.$1$ B.$\pi$ C.$0$ D.$\frac{271}{131}$ E.$\frac{131}{271}$
F.$-\frac{271}{131}$ G.$-1$
H.$(\frac{271}{131})^{2}$
I.$(\frac{131}{271})^{2}$
\testStop
\kluczStart
A
\kluczStop



\zadStart{Zadanie z Wikieł Z 1.1 d) moja wersja nr 95}

Obliczyć wartość wyrażenia $(\frac{271}{137})^{2} \cdot (\frac{137}{271})^{2} \cdot \pi^{0}$.
\zadStop
\rozwStart{Patryk Wirkus}{Martyna Czarnobaj}
$$(\frac{271}{137})^{2} \cdot (\frac{137}{271})^{2} \cdot \pi^{0} = (\frac{271}{137} \cdot \frac{137}{271})^{2} \cdot 1 = 1^{2} \cdot 1 = 1$$
\rozwStop
\odpStart
$1$
\odpStop
\testStart
A.$1$ B.$\pi$ C.$0$ D.$\frac{271}{137}$ E.$\frac{137}{271}$
F.$-\frac{271}{137}$ G.$-1$
H.$(\frac{271}{137})^{2}$
I.$(\frac{137}{271})^{2}$
\testStop
\kluczStart
A
\kluczStop



\zadStart{Zadanie z Wikieł Z 1.1 d) moja wersja nr 96}

Obliczyć wartość wyrażenia $(\frac{271}{139})^{2} \cdot (\frac{139}{271})^{2} \cdot \pi^{0}$.
\zadStop
\rozwStart{Patryk Wirkus}{Martyna Czarnobaj}
$$(\frac{271}{139})^{2} \cdot (\frac{139}{271})^{2} \cdot \pi^{0} = (\frac{271}{139} \cdot \frac{139}{271})^{2} \cdot 1 = 1^{2} \cdot 1 = 1$$
\rozwStop
\odpStart
$1$
\odpStop
\testStart
A.$1$ B.$\pi$ C.$0$ D.$\frac{271}{139}$ E.$\frac{139}{271}$
F.$-\frac{271}{139}$ G.$-1$
H.$(\frac{271}{139})^{2}$
I.$(\frac{139}{271})^{2}$
\testStop
\kluczStart
A
\kluczStop



\zadStart{Zadanie z Wikieł Z 1.1 d) moja wersja nr 97}

Obliczyć wartość wyrażenia $(\frac{277}{103})^{2} \cdot (\frac{103}{277})^{2} \cdot \pi^{0}$.
\zadStop
\rozwStart{Patryk Wirkus}{Martyna Czarnobaj}
$$(\frac{277}{103})^{2} \cdot (\frac{103}{277})^{2} \cdot \pi^{0} = (\frac{277}{103} \cdot \frac{103}{277})^{2} \cdot 1 = 1^{2} \cdot 1 = 1$$
\rozwStop
\odpStart
$1$
\odpStop
\testStart
A.$1$ B.$\pi$ C.$0$ D.$\frac{277}{103}$ E.$\frac{103}{277}$
F.$-\frac{277}{103}$ G.$-1$
H.$(\frac{277}{103})^{2}$
I.$(\frac{103}{277})^{2}$
\testStop
\kluczStart
A
\kluczStop



\zadStart{Zadanie z Wikieł Z 1.1 d) moja wersja nr 98}

Obliczyć wartość wyrażenia $(\frac{277}{107})^{2} \cdot (\frac{107}{277})^{2} \cdot \pi^{0}$.
\zadStop
\rozwStart{Patryk Wirkus}{Martyna Czarnobaj}
$$(\frac{277}{107})^{2} \cdot (\frac{107}{277})^{2} \cdot \pi^{0} = (\frac{277}{107} \cdot \frac{107}{277})^{2} \cdot 1 = 1^{2} \cdot 1 = 1$$
\rozwStop
\odpStart
$1$
\odpStop
\testStart
A.$1$ B.$\pi$ C.$0$ D.$\frac{277}{107}$ E.$\frac{107}{277}$
F.$-\frac{277}{107}$ G.$-1$
H.$(\frac{277}{107})^{2}$
I.$(\frac{107}{277})^{2}$
\testStop
\kluczStart
A
\kluczStop



\zadStart{Zadanie z Wikieł Z 1.1 d) moja wersja nr 99}

Obliczyć wartość wyrażenia $(\frac{277}{109})^{2} \cdot (\frac{109}{277})^{2} \cdot \pi^{0}$.
\zadStop
\rozwStart{Patryk Wirkus}{Martyna Czarnobaj}
$$(\frac{277}{109})^{2} \cdot (\frac{109}{277})^{2} \cdot \pi^{0} = (\frac{277}{109} \cdot \frac{109}{277})^{2} \cdot 1 = 1^{2} \cdot 1 = 1$$
\rozwStop
\odpStart
$1$
\odpStop
\testStart
A.$1$ B.$\pi$ C.$0$ D.$\frac{277}{109}$ E.$\frac{109}{277}$
F.$-\frac{277}{109}$ G.$-1$
H.$(\frac{277}{109})^{2}$
I.$(\frac{109}{277})^{2}$
\testStop
\kluczStart
A
\kluczStop



\zadStart{Zadanie z Wikieł Z 1.1 d) moja wersja nr 100}

Obliczyć wartość wyrażenia $(\frac{277}{113})^{2} \cdot (\frac{113}{277})^{2} \cdot \pi^{0}$.
\zadStop
\rozwStart{Patryk Wirkus}{Martyna Czarnobaj}
$$(\frac{277}{113})^{2} \cdot (\frac{113}{277})^{2} \cdot \pi^{0} = (\frac{277}{113} \cdot \frac{113}{277})^{2} \cdot 1 = 1^{2} \cdot 1 = 1$$
\rozwStop
\odpStart
$1$
\odpStop
\testStart
A.$1$ B.$\pi$ C.$0$ D.$\frac{277}{113}$ E.$\frac{113}{277}$
F.$-\frac{277}{113}$ G.$-1$
H.$(\frac{277}{113})^{2}$
I.$(\frac{113}{277})^{2}$
\testStop
\kluczStart
A
\kluczStop



\zadStart{Zadanie z Wikieł Z 1.1 d) moja wersja nr 101}

Obliczyć wartość wyrażenia $(\frac{277}{127})^{2} \cdot (\frac{127}{277})^{2} \cdot \pi^{0}$.
\zadStop
\rozwStart{Patryk Wirkus}{Martyna Czarnobaj}
$$(\frac{277}{127})^{2} \cdot (\frac{127}{277})^{2} \cdot \pi^{0} = (\frac{277}{127} \cdot \frac{127}{277})^{2} \cdot 1 = 1^{2} \cdot 1 = 1$$
\rozwStop
\odpStart
$1$
\odpStop
\testStart
A.$1$ B.$\pi$ C.$0$ D.$\frac{277}{127}$ E.$\frac{127}{277}$
F.$-\frac{277}{127}$ G.$-1$
H.$(\frac{277}{127})^{2}$
I.$(\frac{127}{277})^{2}$
\testStop
\kluczStart
A
\kluczStop



\zadStart{Zadanie z Wikieł Z 1.1 d) moja wersja nr 102}

Obliczyć wartość wyrażenia $(\frac{277}{131})^{2} \cdot (\frac{131}{277})^{2} \cdot \pi^{0}$.
\zadStop
\rozwStart{Patryk Wirkus}{Martyna Czarnobaj}
$$(\frac{277}{131})^{2} \cdot (\frac{131}{277})^{2} \cdot \pi^{0} = (\frac{277}{131} \cdot \frac{131}{277})^{2} \cdot 1 = 1^{2} \cdot 1 = 1$$
\rozwStop
\odpStart
$1$
\odpStop
\testStart
A.$1$ B.$\pi$ C.$0$ D.$\frac{277}{131}$ E.$\frac{131}{277}$
F.$-\frac{277}{131}$ G.$-1$
H.$(\frac{277}{131})^{2}$
I.$(\frac{131}{277})^{2}$
\testStop
\kluczStart
A
\kluczStop



\zadStart{Zadanie z Wikieł Z 1.1 d) moja wersja nr 103}

Obliczyć wartość wyrażenia $(\frac{277}{137})^{2} \cdot (\frac{137}{277})^{2} \cdot \pi^{0}$.
\zadStop
\rozwStart{Patryk Wirkus}{Martyna Czarnobaj}
$$(\frac{277}{137})^{2} \cdot (\frac{137}{277})^{2} \cdot \pi^{0} = (\frac{277}{137} \cdot \frac{137}{277})^{2} \cdot 1 = 1^{2} \cdot 1 = 1$$
\rozwStop
\odpStart
$1$
\odpStop
\testStart
A.$1$ B.$\pi$ C.$0$ D.$\frac{277}{137}$ E.$\frac{137}{277}$
F.$-\frac{277}{137}$ G.$-1$
H.$(\frac{277}{137})^{2}$
I.$(\frac{137}{277})^{2}$
\testStop
\kluczStart
A
\kluczStop



\zadStart{Zadanie z Wikieł Z 1.1 d) moja wersja nr 104}

Obliczyć wartość wyrażenia $(\frac{277}{139})^{2} \cdot (\frac{139}{277})^{2} \cdot \pi^{0}$.
\zadStop
\rozwStart{Patryk Wirkus}{Martyna Czarnobaj}
$$(\frac{277}{139})^{2} \cdot (\frac{139}{277})^{2} \cdot \pi^{0} = (\frac{277}{139} \cdot \frac{139}{277})^{2} \cdot 1 = 1^{2} \cdot 1 = 1$$
\rozwStop
\odpStart
$1$
\odpStop
\testStart
A.$1$ B.$\pi$ C.$0$ D.$\frac{277}{139}$ E.$\frac{139}{277}$
F.$-\frac{277}{139}$ G.$-1$
H.$(\frac{277}{139})^{2}$
I.$(\frac{139}{277})^{2}$
\testStop
\kluczStart
A
\kluczStop



\zadStart{Zadanie z Wikieł Z 1.1 d) moja wersja nr 105}

Obliczyć wartość wyrażenia $(\frac{149}{103})^{3} \cdot (\frac{103}{149})^{3} \cdot \pi^{0}$.
\zadStop
\rozwStart{Patryk Wirkus}{Martyna Czarnobaj}
$$(\frac{149}{103})^{3} \cdot (\frac{103}{149})^{3} \cdot \pi^{0} = (\frac{149}{103} \cdot \frac{103}{149})^{3} \cdot 1 = 1^{3} \cdot 1 = 1$$
\rozwStop
\odpStart
$1$
\odpStop
\testStart
A.$1$ B.$\pi$ C.$0$ D.$\frac{149}{103}$ E.$\frac{103}{149}$
F.$-\frac{149}{103}$ G.$-1$
H.$(\frac{149}{103})^{3}$
I.$(\frac{103}{149})^{3}$
\testStop
\kluczStart
A
\kluczStop



\zadStart{Zadanie z Wikieł Z 1.1 d) moja wersja nr 106}

Obliczyć wartość wyrażenia $(\frac{149}{107})^{3} \cdot (\frac{107}{149})^{3} \cdot \pi^{0}$.
\zadStop
\rozwStart{Patryk Wirkus}{Martyna Czarnobaj}
$$(\frac{149}{107})^{3} \cdot (\frac{107}{149})^{3} \cdot \pi^{0} = (\frac{149}{107} \cdot \frac{107}{149})^{3} \cdot 1 = 1^{3} \cdot 1 = 1$$
\rozwStop
\odpStart
$1$
\odpStop
\testStart
A.$1$ B.$\pi$ C.$0$ D.$\frac{149}{107}$ E.$\frac{107}{149}$
F.$-\frac{149}{107}$ G.$-1$
H.$(\frac{149}{107})^{3}$
I.$(\frac{107}{149})^{3}$
\testStop
\kluczStart
A
\kluczStop



\zadStart{Zadanie z Wikieł Z 1.1 d) moja wersja nr 107}

Obliczyć wartość wyrażenia $(\frac{149}{109})^{3} \cdot (\frac{109}{149})^{3} \cdot \pi^{0}$.
\zadStop
\rozwStart{Patryk Wirkus}{Martyna Czarnobaj}
$$(\frac{149}{109})^{3} \cdot (\frac{109}{149})^{3} \cdot \pi^{0} = (\frac{149}{109} \cdot \frac{109}{149})^{3} \cdot 1 = 1^{3} \cdot 1 = 1$$
\rozwStop
\odpStart
$1$
\odpStop
\testStart
A.$1$ B.$\pi$ C.$0$ D.$\frac{149}{109}$ E.$\frac{109}{149}$
F.$-\frac{149}{109}$ G.$-1$
H.$(\frac{149}{109})^{3}$
I.$(\frac{109}{149})^{3}$
\testStop
\kluczStart
A
\kluczStop



\zadStart{Zadanie z Wikieł Z 1.1 d) moja wersja nr 108}

Obliczyć wartość wyrażenia $(\frac{149}{113})^{3} \cdot (\frac{113}{149})^{3} \cdot \pi^{0}$.
\zadStop
\rozwStart{Patryk Wirkus}{Martyna Czarnobaj}
$$(\frac{149}{113})^{3} \cdot (\frac{113}{149})^{3} \cdot \pi^{0} = (\frac{149}{113} \cdot \frac{113}{149})^{3} \cdot 1 = 1^{3} \cdot 1 = 1$$
\rozwStop
\odpStart
$1$
\odpStop
\testStart
A.$1$ B.$\pi$ C.$0$ D.$\frac{149}{113}$ E.$\frac{113}{149}$
F.$-\frac{149}{113}$ G.$-1$
H.$(\frac{149}{113})^{3}$
I.$(\frac{113}{149})^{3}$
\testStop
\kluczStart
A
\kluczStop



\zadStart{Zadanie z Wikieł Z 1.1 d) moja wersja nr 109}

Obliczyć wartość wyrażenia $(\frac{149}{127})^{3} \cdot (\frac{127}{149})^{3} \cdot \pi^{0}$.
\zadStop
\rozwStart{Patryk Wirkus}{Martyna Czarnobaj}
$$(\frac{149}{127})^{3} \cdot (\frac{127}{149})^{3} \cdot \pi^{0} = (\frac{149}{127} \cdot \frac{127}{149})^{3} \cdot 1 = 1^{3} \cdot 1 = 1$$
\rozwStop
\odpStart
$1$
\odpStop
\testStart
A.$1$ B.$\pi$ C.$0$ D.$\frac{149}{127}$ E.$\frac{127}{149}$
F.$-\frac{149}{127}$ G.$-1$
H.$(\frac{149}{127})^{3}$
I.$(\frac{127}{149})^{3}$
\testStop
\kluczStart
A
\kluczStop



\zadStart{Zadanie z Wikieł Z 1.1 d) moja wersja nr 110}

Obliczyć wartość wyrażenia $(\frac{149}{131})^{3} \cdot (\frac{131}{149})^{3} \cdot \pi^{0}$.
\zadStop
\rozwStart{Patryk Wirkus}{Martyna Czarnobaj}
$$(\frac{149}{131})^{3} \cdot (\frac{131}{149})^{3} \cdot \pi^{0} = (\frac{149}{131} \cdot \frac{131}{149})^{3} \cdot 1 = 1^{3} \cdot 1 = 1$$
\rozwStop
\odpStart
$1$
\odpStop
\testStart
A.$1$ B.$\pi$ C.$0$ D.$\frac{149}{131}$ E.$\frac{131}{149}$
F.$-\frac{149}{131}$ G.$-1$
H.$(\frac{149}{131})^{3}$
I.$(\frac{131}{149})^{3}$
\testStop
\kluczStart
A
\kluczStop



\zadStart{Zadanie z Wikieł Z 1.1 d) moja wersja nr 111}

Obliczyć wartość wyrażenia $(\frac{149}{137})^{3} \cdot (\frac{137}{149})^{3} \cdot \pi^{0}$.
\zadStop
\rozwStart{Patryk Wirkus}{Martyna Czarnobaj}
$$(\frac{149}{137})^{3} \cdot (\frac{137}{149})^{3} \cdot \pi^{0} = (\frac{149}{137} \cdot \frac{137}{149})^{3} \cdot 1 = 1^{3} \cdot 1 = 1$$
\rozwStop
\odpStart
$1$
\odpStop
\testStart
A.$1$ B.$\pi$ C.$0$ D.$\frac{149}{137}$ E.$\frac{137}{149}$
F.$-\frac{149}{137}$ G.$-1$
H.$(\frac{149}{137})^{3}$
I.$(\frac{137}{149})^{3}$
\testStop
\kluczStart
A
\kluczStop



\zadStart{Zadanie z Wikieł Z 1.1 d) moja wersja nr 112}

Obliczyć wartość wyrażenia $(\frac{149}{139})^{3} \cdot (\frac{139}{149})^{3} \cdot \pi^{0}$.
\zadStop
\rozwStart{Patryk Wirkus}{Martyna Czarnobaj}
$$(\frac{149}{139})^{3} \cdot (\frac{139}{149})^{3} \cdot \pi^{0} = (\frac{149}{139} \cdot \frac{139}{149})^{3} \cdot 1 = 1^{3} \cdot 1 = 1$$
\rozwStop
\odpStart
$1$
\odpStop
\testStart
A.$1$ B.$\pi$ C.$0$ D.$\frac{149}{139}$ E.$\frac{139}{149}$
F.$-\frac{149}{139}$ G.$-1$
H.$(\frac{149}{139})^{3}$
I.$(\frac{139}{149})^{3}$
\testStop
\kluczStart
A
\kluczStop



\zadStart{Zadanie z Wikieł Z 1.1 d) moja wersja nr 113}

Obliczyć wartość wyrażenia $(\frac{151}{103})^{3} \cdot (\frac{103}{151})^{3} \cdot \pi^{0}$.
\zadStop
\rozwStart{Patryk Wirkus}{Martyna Czarnobaj}
$$(\frac{151}{103})^{3} \cdot (\frac{103}{151})^{3} \cdot \pi^{0} = (\frac{151}{103} \cdot \frac{103}{151})^{3} \cdot 1 = 1^{3} \cdot 1 = 1$$
\rozwStop
\odpStart
$1$
\odpStop
\testStart
A.$1$ B.$\pi$ C.$0$ D.$\frac{151}{103}$ E.$\frac{103}{151}$
F.$-\frac{151}{103}$ G.$-1$
H.$(\frac{151}{103})^{3}$
I.$(\frac{103}{151})^{3}$
\testStop
\kluczStart
A
\kluczStop



\zadStart{Zadanie z Wikieł Z 1.1 d) moja wersja nr 114}

Obliczyć wartość wyrażenia $(\frac{151}{107})^{3} \cdot (\frac{107}{151})^{3} \cdot \pi^{0}$.
\zadStop
\rozwStart{Patryk Wirkus}{Martyna Czarnobaj}
$$(\frac{151}{107})^{3} \cdot (\frac{107}{151})^{3} \cdot \pi^{0} = (\frac{151}{107} \cdot \frac{107}{151})^{3} \cdot 1 = 1^{3} \cdot 1 = 1$$
\rozwStop
\odpStart
$1$
\odpStop
\testStart
A.$1$ B.$\pi$ C.$0$ D.$\frac{151}{107}$ E.$\frac{107}{151}$
F.$-\frac{151}{107}$ G.$-1$
H.$(\frac{151}{107})^{3}$
I.$(\frac{107}{151})^{3}$
\testStop
\kluczStart
A
\kluczStop



\zadStart{Zadanie z Wikieł Z 1.1 d) moja wersja nr 115}

Obliczyć wartość wyrażenia $(\frac{151}{109})^{3} \cdot (\frac{109}{151})^{3} \cdot \pi^{0}$.
\zadStop
\rozwStart{Patryk Wirkus}{Martyna Czarnobaj}
$$(\frac{151}{109})^{3} \cdot (\frac{109}{151})^{3} \cdot \pi^{0} = (\frac{151}{109} \cdot \frac{109}{151})^{3} \cdot 1 = 1^{3} \cdot 1 = 1$$
\rozwStop
\odpStart
$1$
\odpStop
\testStart
A.$1$ B.$\pi$ C.$0$ D.$\frac{151}{109}$ E.$\frac{109}{151}$
F.$-\frac{151}{109}$ G.$-1$
H.$(\frac{151}{109})^{3}$
I.$(\frac{109}{151})^{3}$
\testStop
\kluczStart
A
\kluczStop



\zadStart{Zadanie z Wikieł Z 1.1 d) moja wersja nr 116}

Obliczyć wartość wyrażenia $(\frac{151}{113})^{3} \cdot (\frac{113}{151})^{3} \cdot \pi^{0}$.
\zadStop
\rozwStart{Patryk Wirkus}{Martyna Czarnobaj}
$$(\frac{151}{113})^{3} \cdot (\frac{113}{151})^{3} \cdot \pi^{0} = (\frac{151}{113} \cdot \frac{113}{151})^{3} \cdot 1 = 1^{3} \cdot 1 = 1$$
\rozwStop
\odpStart
$1$
\odpStop
\testStart
A.$1$ B.$\pi$ C.$0$ D.$\frac{151}{113}$ E.$\frac{113}{151}$
F.$-\frac{151}{113}$ G.$-1$
H.$(\frac{151}{113})^{3}$
I.$(\frac{113}{151})^{3}$
\testStop
\kluczStart
A
\kluczStop



\zadStart{Zadanie z Wikieł Z 1.1 d) moja wersja nr 117}

Obliczyć wartość wyrażenia $(\frac{151}{127})^{3} \cdot (\frac{127}{151})^{3} \cdot \pi^{0}$.
\zadStop
\rozwStart{Patryk Wirkus}{Martyna Czarnobaj}
$$(\frac{151}{127})^{3} \cdot (\frac{127}{151})^{3} \cdot \pi^{0} = (\frac{151}{127} \cdot \frac{127}{151})^{3} \cdot 1 = 1^{3} \cdot 1 = 1$$
\rozwStop
\odpStart
$1$
\odpStop
\testStart
A.$1$ B.$\pi$ C.$0$ D.$\frac{151}{127}$ E.$\frac{127}{151}$
F.$-\frac{151}{127}$ G.$-1$
H.$(\frac{151}{127})^{3}$
I.$(\frac{127}{151})^{3}$
\testStop
\kluczStart
A
\kluczStop



\zadStart{Zadanie z Wikieł Z 1.1 d) moja wersja nr 118}

Obliczyć wartość wyrażenia $(\frac{151}{131})^{3} \cdot (\frac{131}{151})^{3} \cdot \pi^{0}$.
\zadStop
\rozwStart{Patryk Wirkus}{Martyna Czarnobaj}
$$(\frac{151}{131})^{3} \cdot (\frac{131}{151})^{3} \cdot \pi^{0} = (\frac{151}{131} \cdot \frac{131}{151})^{3} \cdot 1 = 1^{3} \cdot 1 = 1$$
\rozwStop
\odpStart
$1$
\odpStop
\testStart
A.$1$ B.$\pi$ C.$0$ D.$\frac{151}{131}$ E.$\frac{131}{151}$
F.$-\frac{151}{131}$ G.$-1$
H.$(\frac{151}{131})^{3}$
I.$(\frac{131}{151})^{3}$
\testStop
\kluczStart
A
\kluczStop



\zadStart{Zadanie z Wikieł Z 1.1 d) moja wersja nr 119}

Obliczyć wartość wyrażenia $(\frac{151}{137})^{3} \cdot (\frac{137}{151})^{3} \cdot \pi^{0}$.
\zadStop
\rozwStart{Patryk Wirkus}{Martyna Czarnobaj}
$$(\frac{151}{137})^{3} \cdot (\frac{137}{151})^{3} \cdot \pi^{0} = (\frac{151}{137} \cdot \frac{137}{151})^{3} \cdot 1 = 1^{3} \cdot 1 = 1$$
\rozwStop
\odpStart
$1$
\odpStop
\testStart
A.$1$ B.$\pi$ C.$0$ D.$\frac{151}{137}$ E.$\frac{137}{151}$
F.$-\frac{151}{137}$ G.$-1$
H.$(\frac{151}{137})^{3}$
I.$(\frac{137}{151})^{3}$
\testStop
\kluczStart
A
\kluczStop



\zadStart{Zadanie z Wikieł Z 1.1 d) moja wersja nr 120}

Obliczyć wartość wyrażenia $(\frac{151}{139})^{3} \cdot (\frac{139}{151})^{3} \cdot \pi^{0}$.
\zadStop
\rozwStart{Patryk Wirkus}{Martyna Czarnobaj}
$$(\frac{151}{139})^{3} \cdot (\frac{139}{151})^{3} \cdot \pi^{0} = (\frac{151}{139} \cdot \frac{139}{151})^{3} \cdot 1 = 1^{3} \cdot 1 = 1$$
\rozwStop
\odpStart
$1$
\odpStop
\testStart
A.$1$ B.$\pi$ C.$0$ D.$\frac{151}{139}$ E.$\frac{139}{151}$
F.$-\frac{151}{139}$ G.$-1$
H.$(\frac{151}{139})^{3}$
I.$(\frac{139}{151})^{3}$
\testStop
\kluczStart
A
\kluczStop



\zadStart{Zadanie z Wikieł Z 1.1 d) moja wersja nr 121}

Obliczyć wartość wyrażenia $(\frac{157}{103})^{3} \cdot (\frac{103}{157})^{3} \cdot \pi^{0}$.
\zadStop
\rozwStart{Patryk Wirkus}{Martyna Czarnobaj}
$$(\frac{157}{103})^{3} \cdot (\frac{103}{157})^{3} \cdot \pi^{0} = (\frac{157}{103} \cdot \frac{103}{157})^{3} \cdot 1 = 1^{3} \cdot 1 = 1$$
\rozwStop
\odpStart
$1$
\odpStop
\testStart
A.$1$ B.$\pi$ C.$0$ D.$\frac{157}{103}$ E.$\frac{103}{157}$
F.$-\frac{157}{103}$ G.$-1$
H.$(\frac{157}{103})^{3}$
I.$(\frac{103}{157})^{3}$
\testStop
\kluczStart
A
\kluczStop



\zadStart{Zadanie z Wikieł Z 1.1 d) moja wersja nr 122}

Obliczyć wartość wyrażenia $(\frac{157}{107})^{3} \cdot (\frac{107}{157})^{3} \cdot \pi^{0}$.
\zadStop
\rozwStart{Patryk Wirkus}{Martyna Czarnobaj}
$$(\frac{157}{107})^{3} \cdot (\frac{107}{157})^{3} \cdot \pi^{0} = (\frac{157}{107} \cdot \frac{107}{157})^{3} \cdot 1 = 1^{3} \cdot 1 = 1$$
\rozwStop
\odpStart
$1$
\odpStop
\testStart
A.$1$ B.$\pi$ C.$0$ D.$\frac{157}{107}$ E.$\frac{107}{157}$
F.$-\frac{157}{107}$ G.$-1$
H.$(\frac{157}{107})^{3}$
I.$(\frac{107}{157})^{3}$
\testStop
\kluczStart
A
\kluczStop



\zadStart{Zadanie z Wikieł Z 1.1 d) moja wersja nr 123}

Obliczyć wartość wyrażenia $(\frac{157}{109})^{3} \cdot (\frac{109}{157})^{3} \cdot \pi^{0}$.
\zadStop
\rozwStart{Patryk Wirkus}{Martyna Czarnobaj}
$$(\frac{157}{109})^{3} \cdot (\frac{109}{157})^{3} \cdot \pi^{0} = (\frac{157}{109} \cdot \frac{109}{157})^{3} \cdot 1 = 1^{3} \cdot 1 = 1$$
\rozwStop
\odpStart
$1$
\odpStop
\testStart
A.$1$ B.$\pi$ C.$0$ D.$\frac{157}{109}$ E.$\frac{109}{157}$
F.$-\frac{157}{109}$ G.$-1$
H.$(\frac{157}{109})^{3}$
I.$(\frac{109}{157})^{3}$
\testStop
\kluczStart
A
\kluczStop



\zadStart{Zadanie z Wikieł Z 1.1 d) moja wersja nr 124}

Obliczyć wartość wyrażenia $(\frac{157}{113})^{3} \cdot (\frac{113}{157})^{3} \cdot \pi^{0}$.
\zadStop
\rozwStart{Patryk Wirkus}{Martyna Czarnobaj}
$$(\frac{157}{113})^{3} \cdot (\frac{113}{157})^{3} \cdot \pi^{0} = (\frac{157}{113} \cdot \frac{113}{157})^{3} \cdot 1 = 1^{3} \cdot 1 = 1$$
\rozwStop
\odpStart
$1$
\odpStop
\testStart
A.$1$ B.$\pi$ C.$0$ D.$\frac{157}{113}$ E.$\frac{113}{157}$
F.$-\frac{157}{113}$ G.$-1$
H.$(\frac{157}{113})^{3}$
I.$(\frac{113}{157})^{3}$
\testStop
\kluczStart
A
\kluczStop



\zadStart{Zadanie z Wikieł Z 1.1 d) moja wersja nr 125}

Obliczyć wartość wyrażenia $(\frac{157}{127})^{3} \cdot (\frac{127}{157})^{3} \cdot \pi^{0}$.
\zadStop
\rozwStart{Patryk Wirkus}{Martyna Czarnobaj}
$$(\frac{157}{127})^{3} \cdot (\frac{127}{157})^{3} \cdot \pi^{0} = (\frac{157}{127} \cdot \frac{127}{157})^{3} \cdot 1 = 1^{3} \cdot 1 = 1$$
\rozwStop
\odpStart
$1$
\odpStop
\testStart
A.$1$ B.$\pi$ C.$0$ D.$\frac{157}{127}$ E.$\frac{127}{157}$
F.$-\frac{157}{127}$ G.$-1$
H.$(\frac{157}{127})^{3}$
I.$(\frac{127}{157})^{3}$
\testStop
\kluczStart
A
\kluczStop



\zadStart{Zadanie z Wikieł Z 1.1 d) moja wersja nr 126}

Obliczyć wartość wyrażenia $(\frac{157}{131})^{3} \cdot (\frac{131}{157})^{3} \cdot \pi^{0}$.
\zadStop
\rozwStart{Patryk Wirkus}{Martyna Czarnobaj}
$$(\frac{157}{131})^{3} \cdot (\frac{131}{157})^{3} \cdot \pi^{0} = (\frac{157}{131} \cdot \frac{131}{157})^{3} \cdot 1 = 1^{3} \cdot 1 = 1$$
\rozwStop
\odpStart
$1$
\odpStop
\testStart
A.$1$ B.$\pi$ C.$0$ D.$\frac{157}{131}$ E.$\frac{131}{157}$
F.$-\frac{157}{131}$ G.$-1$
H.$(\frac{157}{131})^{3}$
I.$(\frac{131}{157})^{3}$
\testStop
\kluczStart
A
\kluczStop



\zadStart{Zadanie z Wikieł Z 1.1 d) moja wersja nr 127}

Obliczyć wartość wyrażenia $(\frac{157}{137})^{3} \cdot (\frac{137}{157})^{3} \cdot \pi^{0}$.
\zadStop
\rozwStart{Patryk Wirkus}{Martyna Czarnobaj}
$$(\frac{157}{137})^{3} \cdot (\frac{137}{157})^{3} \cdot \pi^{0} = (\frac{157}{137} \cdot \frac{137}{157})^{3} \cdot 1 = 1^{3} \cdot 1 = 1$$
\rozwStop
\odpStart
$1$
\odpStop
\testStart
A.$1$ B.$\pi$ C.$0$ D.$\frac{157}{137}$ E.$\frac{137}{157}$
F.$-\frac{157}{137}$ G.$-1$
H.$(\frac{157}{137})^{3}$
I.$(\frac{137}{157})^{3}$
\testStop
\kluczStart
A
\kluczStop



\zadStart{Zadanie z Wikieł Z 1.1 d) moja wersja nr 128}

Obliczyć wartość wyrażenia $(\frac{157}{139})^{3} \cdot (\frac{139}{157})^{3} \cdot \pi^{0}$.
\zadStop
\rozwStart{Patryk Wirkus}{Martyna Czarnobaj}
$$(\frac{157}{139})^{3} \cdot (\frac{139}{157})^{3} \cdot \pi^{0} = (\frac{157}{139} \cdot \frac{139}{157})^{3} \cdot 1 = 1^{3} \cdot 1 = 1$$
\rozwStop
\odpStart
$1$
\odpStop
\testStart
A.$1$ B.$\pi$ C.$0$ D.$\frac{157}{139}$ E.$\frac{139}{157}$
F.$-\frac{157}{139}$ G.$-1$
H.$(\frac{157}{139})^{3}$
I.$(\frac{139}{157})^{3}$
\testStop
\kluczStart
A
\kluczStop



\zadStart{Zadanie z Wikieł Z 1.1 d) moja wersja nr 129}

Obliczyć wartość wyrażenia $(\frac{163}{103})^{3} \cdot (\frac{103}{163})^{3} \cdot \pi^{0}$.
\zadStop
\rozwStart{Patryk Wirkus}{Martyna Czarnobaj}
$$(\frac{163}{103})^{3} \cdot (\frac{103}{163})^{3} \cdot \pi^{0} = (\frac{163}{103} \cdot \frac{103}{163})^{3} \cdot 1 = 1^{3} \cdot 1 = 1$$
\rozwStop
\odpStart
$1$
\odpStop
\testStart
A.$1$ B.$\pi$ C.$0$ D.$\frac{163}{103}$ E.$\frac{103}{163}$
F.$-\frac{163}{103}$ G.$-1$
H.$(\frac{163}{103})^{3}$
I.$(\frac{103}{163})^{3}$
\testStop
\kluczStart
A
\kluczStop



\zadStart{Zadanie z Wikieł Z 1.1 d) moja wersja nr 130}

Obliczyć wartość wyrażenia $(\frac{163}{107})^{3} \cdot (\frac{107}{163})^{3} \cdot \pi^{0}$.
\zadStop
\rozwStart{Patryk Wirkus}{Martyna Czarnobaj}
$$(\frac{163}{107})^{3} \cdot (\frac{107}{163})^{3} \cdot \pi^{0} = (\frac{163}{107} \cdot \frac{107}{163})^{3} \cdot 1 = 1^{3} \cdot 1 = 1$$
\rozwStop
\odpStart
$1$
\odpStop
\testStart
A.$1$ B.$\pi$ C.$0$ D.$\frac{163}{107}$ E.$\frac{107}{163}$
F.$-\frac{163}{107}$ G.$-1$
H.$(\frac{163}{107})^{3}$
I.$(\frac{107}{163})^{3}$
\testStop
\kluczStart
A
\kluczStop



\zadStart{Zadanie z Wikieł Z 1.1 d) moja wersja nr 131}

Obliczyć wartość wyrażenia $(\frac{163}{109})^{3} \cdot (\frac{109}{163})^{3} \cdot \pi^{0}$.
\zadStop
\rozwStart{Patryk Wirkus}{Martyna Czarnobaj}
$$(\frac{163}{109})^{3} \cdot (\frac{109}{163})^{3} \cdot \pi^{0} = (\frac{163}{109} \cdot \frac{109}{163})^{3} \cdot 1 = 1^{3} \cdot 1 = 1$$
\rozwStop
\odpStart
$1$
\odpStop
\testStart
A.$1$ B.$\pi$ C.$0$ D.$\frac{163}{109}$ E.$\frac{109}{163}$
F.$-\frac{163}{109}$ G.$-1$
H.$(\frac{163}{109})^{3}$
I.$(\frac{109}{163})^{3}$
\testStop
\kluczStart
A
\kluczStop



\zadStart{Zadanie z Wikieł Z 1.1 d) moja wersja nr 132}

Obliczyć wartość wyrażenia $(\frac{163}{113})^{3} \cdot (\frac{113}{163})^{3} \cdot \pi^{0}$.
\zadStop
\rozwStart{Patryk Wirkus}{Martyna Czarnobaj}
$$(\frac{163}{113})^{3} \cdot (\frac{113}{163})^{3} \cdot \pi^{0} = (\frac{163}{113} \cdot \frac{113}{163})^{3} \cdot 1 = 1^{3} \cdot 1 = 1$$
\rozwStop
\odpStart
$1$
\odpStop
\testStart
A.$1$ B.$\pi$ C.$0$ D.$\frac{163}{113}$ E.$\frac{113}{163}$
F.$-\frac{163}{113}$ G.$-1$
H.$(\frac{163}{113})^{3}$
I.$(\frac{113}{163})^{3}$
\testStop
\kluczStart
A
\kluczStop



\zadStart{Zadanie z Wikieł Z 1.1 d) moja wersja nr 133}

Obliczyć wartość wyrażenia $(\frac{163}{127})^{3} \cdot (\frac{127}{163})^{3} \cdot \pi^{0}$.
\zadStop
\rozwStart{Patryk Wirkus}{Martyna Czarnobaj}
$$(\frac{163}{127})^{3} \cdot (\frac{127}{163})^{3} \cdot \pi^{0} = (\frac{163}{127} \cdot \frac{127}{163})^{3} \cdot 1 = 1^{3} \cdot 1 = 1$$
\rozwStop
\odpStart
$1$
\odpStop
\testStart
A.$1$ B.$\pi$ C.$0$ D.$\frac{163}{127}$ E.$\frac{127}{163}$
F.$-\frac{163}{127}$ G.$-1$
H.$(\frac{163}{127})^{3}$
I.$(\frac{127}{163})^{3}$
\testStop
\kluczStart
A
\kluczStop



\zadStart{Zadanie z Wikieł Z 1.1 d) moja wersja nr 134}

Obliczyć wartość wyrażenia $(\frac{163}{131})^{3} \cdot (\frac{131}{163})^{3} \cdot \pi^{0}$.
\zadStop
\rozwStart{Patryk Wirkus}{Martyna Czarnobaj}
$$(\frac{163}{131})^{3} \cdot (\frac{131}{163})^{3} \cdot \pi^{0} = (\frac{163}{131} \cdot \frac{131}{163})^{3} \cdot 1 = 1^{3} \cdot 1 = 1$$
\rozwStop
\odpStart
$1$
\odpStop
\testStart
A.$1$ B.$\pi$ C.$0$ D.$\frac{163}{131}$ E.$\frac{131}{163}$
F.$-\frac{163}{131}$ G.$-1$
H.$(\frac{163}{131})^{3}$
I.$(\frac{131}{163})^{3}$
\testStop
\kluczStart
A
\kluczStop



\zadStart{Zadanie z Wikieł Z 1.1 d) moja wersja nr 135}

Obliczyć wartość wyrażenia $(\frac{163}{137})^{3} \cdot (\frac{137}{163})^{3} \cdot \pi^{0}$.
\zadStop
\rozwStart{Patryk Wirkus}{Martyna Czarnobaj}
$$(\frac{163}{137})^{3} \cdot (\frac{137}{163})^{3} \cdot \pi^{0} = (\frac{163}{137} \cdot \frac{137}{163})^{3} \cdot 1 = 1^{3} \cdot 1 = 1$$
\rozwStop
\odpStart
$1$
\odpStop
\testStart
A.$1$ B.$\pi$ C.$0$ D.$\frac{163}{137}$ E.$\frac{137}{163}$
F.$-\frac{163}{137}$ G.$-1$
H.$(\frac{163}{137})^{3}$
I.$(\frac{137}{163})^{3}$
\testStop
\kluczStart
A
\kluczStop



\zadStart{Zadanie z Wikieł Z 1.1 d) moja wersja nr 136}

Obliczyć wartość wyrażenia $(\frac{163}{139})^{3} \cdot (\frac{139}{163})^{3} \cdot \pi^{0}$.
\zadStop
\rozwStart{Patryk Wirkus}{Martyna Czarnobaj}
$$(\frac{163}{139})^{3} \cdot (\frac{139}{163})^{3} \cdot \pi^{0} = (\frac{163}{139} \cdot \frac{139}{163})^{3} \cdot 1 = 1^{3} \cdot 1 = 1$$
\rozwStop
\odpStart
$1$
\odpStop
\testStart
A.$1$ B.$\pi$ C.$0$ D.$\frac{163}{139}$ E.$\frac{139}{163}$
F.$-\frac{163}{139}$ G.$-1$
H.$(\frac{163}{139})^{3}$
I.$(\frac{139}{163})^{3}$
\testStop
\kluczStart
A
\kluczStop



\zadStart{Zadanie z Wikieł Z 1.1 d) moja wersja nr 137}

Obliczyć wartość wyrażenia $(\frac{167}{103})^{3} \cdot (\frac{103}{167})^{3} \cdot \pi^{0}$.
\zadStop
\rozwStart{Patryk Wirkus}{Martyna Czarnobaj}
$$(\frac{167}{103})^{3} \cdot (\frac{103}{167})^{3} \cdot \pi^{0} = (\frac{167}{103} \cdot \frac{103}{167})^{3} \cdot 1 = 1^{3} \cdot 1 = 1$$
\rozwStop
\odpStart
$1$
\odpStop
\testStart
A.$1$ B.$\pi$ C.$0$ D.$\frac{167}{103}$ E.$\frac{103}{167}$
F.$-\frac{167}{103}$ G.$-1$
H.$(\frac{167}{103})^{3}$
I.$(\frac{103}{167})^{3}$
\testStop
\kluczStart
A
\kluczStop



\zadStart{Zadanie z Wikieł Z 1.1 d) moja wersja nr 138}

Obliczyć wartość wyrażenia $(\frac{167}{107})^{3} \cdot (\frac{107}{167})^{3} \cdot \pi^{0}$.
\zadStop
\rozwStart{Patryk Wirkus}{Martyna Czarnobaj}
$$(\frac{167}{107})^{3} \cdot (\frac{107}{167})^{3} \cdot \pi^{0} = (\frac{167}{107} \cdot \frac{107}{167})^{3} \cdot 1 = 1^{3} \cdot 1 = 1$$
\rozwStop
\odpStart
$1$
\odpStop
\testStart
A.$1$ B.$\pi$ C.$0$ D.$\frac{167}{107}$ E.$\frac{107}{167}$
F.$-\frac{167}{107}$ G.$-1$
H.$(\frac{167}{107})^{3}$
I.$(\frac{107}{167})^{3}$
\testStop
\kluczStart
A
\kluczStop



\zadStart{Zadanie z Wikieł Z 1.1 d) moja wersja nr 139}

Obliczyć wartość wyrażenia $(\frac{167}{109})^{3} \cdot (\frac{109}{167})^{3} \cdot \pi^{0}$.
\zadStop
\rozwStart{Patryk Wirkus}{Martyna Czarnobaj}
$$(\frac{167}{109})^{3} \cdot (\frac{109}{167})^{3} \cdot \pi^{0} = (\frac{167}{109} \cdot \frac{109}{167})^{3} \cdot 1 = 1^{3} \cdot 1 = 1$$
\rozwStop
\odpStart
$1$
\odpStop
\testStart
A.$1$ B.$\pi$ C.$0$ D.$\frac{167}{109}$ E.$\frac{109}{167}$
F.$-\frac{167}{109}$ G.$-1$
H.$(\frac{167}{109})^{3}$
I.$(\frac{109}{167})^{3}$
\testStop
\kluczStart
A
\kluczStop



\zadStart{Zadanie z Wikieł Z 1.1 d) moja wersja nr 140}

Obliczyć wartość wyrażenia $(\frac{167}{113})^{3} \cdot (\frac{113}{167})^{3} \cdot \pi^{0}$.
\zadStop
\rozwStart{Patryk Wirkus}{Martyna Czarnobaj}
$$(\frac{167}{113})^{3} \cdot (\frac{113}{167})^{3} \cdot \pi^{0} = (\frac{167}{113} \cdot \frac{113}{167})^{3} \cdot 1 = 1^{3} \cdot 1 = 1$$
\rozwStop
\odpStart
$1$
\odpStop
\testStart
A.$1$ B.$\pi$ C.$0$ D.$\frac{167}{113}$ E.$\frac{113}{167}$
F.$-\frac{167}{113}$ G.$-1$
H.$(\frac{167}{113})^{3}$
I.$(\frac{113}{167})^{3}$
\testStop
\kluczStart
A
\kluczStop



\zadStart{Zadanie z Wikieł Z 1.1 d) moja wersja nr 141}

Obliczyć wartość wyrażenia $(\frac{167}{127})^{3} \cdot (\frac{127}{167})^{3} \cdot \pi^{0}$.
\zadStop
\rozwStart{Patryk Wirkus}{Martyna Czarnobaj}
$$(\frac{167}{127})^{3} \cdot (\frac{127}{167})^{3} \cdot \pi^{0} = (\frac{167}{127} \cdot \frac{127}{167})^{3} \cdot 1 = 1^{3} \cdot 1 = 1$$
\rozwStop
\odpStart
$1$
\odpStop
\testStart
A.$1$ B.$\pi$ C.$0$ D.$\frac{167}{127}$ E.$\frac{127}{167}$
F.$-\frac{167}{127}$ G.$-1$
H.$(\frac{167}{127})^{3}$
I.$(\frac{127}{167})^{3}$
\testStop
\kluczStart
A
\kluczStop



\zadStart{Zadanie z Wikieł Z 1.1 d) moja wersja nr 142}

Obliczyć wartość wyrażenia $(\frac{167}{131})^{3} \cdot (\frac{131}{167})^{3} \cdot \pi^{0}$.
\zadStop
\rozwStart{Patryk Wirkus}{Martyna Czarnobaj}
$$(\frac{167}{131})^{3} \cdot (\frac{131}{167})^{3} \cdot \pi^{0} = (\frac{167}{131} \cdot \frac{131}{167})^{3} \cdot 1 = 1^{3} \cdot 1 = 1$$
\rozwStop
\odpStart
$1$
\odpStop
\testStart
A.$1$ B.$\pi$ C.$0$ D.$\frac{167}{131}$ E.$\frac{131}{167}$
F.$-\frac{167}{131}$ G.$-1$
H.$(\frac{167}{131})^{3}$
I.$(\frac{131}{167})^{3}$
\testStop
\kluczStart
A
\kluczStop



\zadStart{Zadanie z Wikieł Z 1.1 d) moja wersja nr 143}

Obliczyć wartość wyrażenia $(\frac{167}{137})^{3} \cdot (\frac{137}{167})^{3} \cdot \pi^{0}$.
\zadStop
\rozwStart{Patryk Wirkus}{Martyna Czarnobaj}
$$(\frac{167}{137})^{3} \cdot (\frac{137}{167})^{3} \cdot \pi^{0} = (\frac{167}{137} \cdot \frac{137}{167})^{3} \cdot 1 = 1^{3} \cdot 1 = 1$$
\rozwStop
\odpStart
$1$
\odpStop
\testStart
A.$1$ B.$\pi$ C.$0$ D.$\frac{167}{137}$ E.$\frac{137}{167}$
F.$-\frac{167}{137}$ G.$-1$
H.$(\frac{167}{137})^{3}$
I.$(\frac{137}{167})^{3}$
\testStop
\kluczStart
A
\kluczStop



\zadStart{Zadanie z Wikieł Z 1.1 d) moja wersja nr 144}

Obliczyć wartość wyrażenia $(\frac{167}{139})^{3} \cdot (\frac{139}{167})^{3} \cdot \pi^{0}$.
\zadStop
\rozwStart{Patryk Wirkus}{Martyna Czarnobaj}
$$(\frac{167}{139})^{3} \cdot (\frac{139}{167})^{3} \cdot \pi^{0} = (\frac{167}{139} \cdot \frac{139}{167})^{3} \cdot 1 = 1^{3} \cdot 1 = 1$$
\rozwStop
\odpStart
$1$
\odpStop
\testStart
A.$1$ B.$\pi$ C.$0$ D.$\frac{167}{139}$ E.$\frac{139}{167}$
F.$-\frac{167}{139}$ G.$-1$
H.$(\frac{167}{139})^{3}$
I.$(\frac{139}{167})^{3}$
\testStop
\kluczStart
A
\kluczStop



\zadStart{Zadanie z Wikieł Z 1.1 d) moja wersja nr 145}

Obliczyć wartość wyrażenia $(\frac{173}{103})^{3} \cdot (\frac{103}{173})^{3} \cdot \pi^{0}$.
\zadStop
\rozwStart{Patryk Wirkus}{Martyna Czarnobaj}
$$(\frac{173}{103})^{3} \cdot (\frac{103}{173})^{3} \cdot \pi^{0} = (\frac{173}{103} \cdot \frac{103}{173})^{3} \cdot 1 = 1^{3} \cdot 1 = 1$$
\rozwStop
\odpStart
$1$
\odpStop
\testStart
A.$1$ B.$\pi$ C.$0$ D.$\frac{173}{103}$ E.$\frac{103}{173}$
F.$-\frac{173}{103}$ G.$-1$
H.$(\frac{173}{103})^{3}$
I.$(\frac{103}{173})^{3}$
\testStop
\kluczStart
A
\kluczStop



\zadStart{Zadanie z Wikieł Z 1.1 d) moja wersja nr 146}

Obliczyć wartość wyrażenia $(\frac{173}{107})^{3} \cdot (\frac{107}{173})^{3} \cdot \pi^{0}$.
\zadStop
\rozwStart{Patryk Wirkus}{Martyna Czarnobaj}
$$(\frac{173}{107})^{3} \cdot (\frac{107}{173})^{3} \cdot \pi^{0} = (\frac{173}{107} \cdot \frac{107}{173})^{3} \cdot 1 = 1^{3} \cdot 1 = 1$$
\rozwStop
\odpStart
$1$
\odpStop
\testStart
A.$1$ B.$\pi$ C.$0$ D.$\frac{173}{107}$ E.$\frac{107}{173}$
F.$-\frac{173}{107}$ G.$-1$
H.$(\frac{173}{107})^{3}$
I.$(\frac{107}{173})^{3}$
\testStop
\kluczStart
A
\kluczStop



\zadStart{Zadanie z Wikieł Z 1.1 d) moja wersja nr 147}

Obliczyć wartość wyrażenia $(\frac{173}{109})^{3} \cdot (\frac{109}{173})^{3} \cdot \pi^{0}$.
\zadStop
\rozwStart{Patryk Wirkus}{Martyna Czarnobaj}
$$(\frac{173}{109})^{3} \cdot (\frac{109}{173})^{3} \cdot \pi^{0} = (\frac{173}{109} \cdot \frac{109}{173})^{3} \cdot 1 = 1^{3} \cdot 1 = 1$$
\rozwStop
\odpStart
$1$
\odpStop
\testStart
A.$1$ B.$\pi$ C.$0$ D.$\frac{173}{109}$ E.$\frac{109}{173}$
F.$-\frac{173}{109}$ G.$-1$
H.$(\frac{173}{109})^{3}$
I.$(\frac{109}{173})^{3}$
\testStop
\kluczStart
A
\kluczStop



\zadStart{Zadanie z Wikieł Z 1.1 d) moja wersja nr 148}

Obliczyć wartość wyrażenia $(\frac{173}{113})^{3} \cdot (\frac{113}{173})^{3} \cdot \pi^{0}$.
\zadStop
\rozwStart{Patryk Wirkus}{Martyna Czarnobaj}
$$(\frac{173}{113})^{3} \cdot (\frac{113}{173})^{3} \cdot \pi^{0} = (\frac{173}{113} \cdot \frac{113}{173})^{3} \cdot 1 = 1^{3} \cdot 1 = 1$$
\rozwStop
\odpStart
$1$
\odpStop
\testStart
A.$1$ B.$\pi$ C.$0$ D.$\frac{173}{113}$ E.$\frac{113}{173}$
F.$-\frac{173}{113}$ G.$-1$
H.$(\frac{173}{113})^{3}$
I.$(\frac{113}{173})^{3}$
\testStop
\kluczStart
A
\kluczStop



\zadStart{Zadanie z Wikieł Z 1.1 d) moja wersja nr 149}

Obliczyć wartość wyrażenia $(\frac{173}{127})^{3} \cdot (\frac{127}{173})^{3} \cdot \pi^{0}$.
\zadStop
\rozwStart{Patryk Wirkus}{Martyna Czarnobaj}
$$(\frac{173}{127})^{3} \cdot (\frac{127}{173})^{3} \cdot \pi^{0} = (\frac{173}{127} \cdot \frac{127}{173})^{3} \cdot 1 = 1^{3} \cdot 1 = 1$$
\rozwStop
\odpStart
$1$
\odpStop
\testStart
A.$1$ B.$\pi$ C.$0$ D.$\frac{173}{127}$ E.$\frac{127}{173}$
F.$-\frac{173}{127}$ G.$-1$
H.$(\frac{173}{127})^{3}$
I.$(\frac{127}{173})^{3}$
\testStop
\kluczStart
A
\kluczStop



\zadStart{Zadanie z Wikieł Z 1.1 d) moja wersja nr 150}

Obliczyć wartość wyrażenia $(\frac{173}{131})^{3} \cdot (\frac{131}{173})^{3} \cdot \pi^{0}$.
\zadStop
\rozwStart{Patryk Wirkus}{Martyna Czarnobaj}
$$(\frac{173}{131})^{3} \cdot (\frac{131}{173})^{3} \cdot \pi^{0} = (\frac{173}{131} \cdot \frac{131}{173})^{3} \cdot 1 = 1^{3} \cdot 1 = 1$$
\rozwStop
\odpStart
$1$
\odpStop
\testStart
A.$1$ B.$\pi$ C.$0$ D.$\frac{173}{131}$ E.$\frac{131}{173}$
F.$-\frac{173}{131}$ G.$-1$
H.$(\frac{173}{131})^{3}$
I.$(\frac{131}{173})^{3}$
\testStop
\kluczStart
A
\kluczStop



\zadStart{Zadanie z Wikieł Z 1.1 d) moja wersja nr 151}

Obliczyć wartość wyrażenia $(\frac{173}{137})^{3} \cdot (\frac{137}{173})^{3} \cdot \pi^{0}$.
\zadStop
\rozwStart{Patryk Wirkus}{Martyna Czarnobaj}
$$(\frac{173}{137})^{3} \cdot (\frac{137}{173})^{3} \cdot \pi^{0} = (\frac{173}{137} \cdot \frac{137}{173})^{3} \cdot 1 = 1^{3} \cdot 1 = 1$$
\rozwStop
\odpStart
$1$
\odpStop
\testStart
A.$1$ B.$\pi$ C.$0$ D.$\frac{173}{137}$ E.$\frac{137}{173}$
F.$-\frac{173}{137}$ G.$-1$
H.$(\frac{173}{137})^{3}$
I.$(\frac{137}{173})^{3}$
\testStop
\kluczStart
A
\kluczStop



\zadStart{Zadanie z Wikieł Z 1.1 d) moja wersja nr 152}

Obliczyć wartość wyrażenia $(\frac{173}{139})^{3} \cdot (\frac{139}{173})^{3} \cdot \pi^{0}$.
\zadStop
\rozwStart{Patryk Wirkus}{Martyna Czarnobaj}
$$(\frac{173}{139})^{3} \cdot (\frac{139}{173})^{3} \cdot \pi^{0} = (\frac{173}{139} \cdot \frac{139}{173})^{3} \cdot 1 = 1^{3} \cdot 1 = 1$$
\rozwStop
\odpStart
$1$
\odpStop
\testStart
A.$1$ B.$\pi$ C.$0$ D.$\frac{173}{139}$ E.$\frac{139}{173}$
F.$-\frac{173}{139}$ G.$-1$
H.$(\frac{173}{139})^{3}$
I.$(\frac{139}{173})^{3}$
\testStop
\kluczStart
A
\kluczStop



\zadStart{Zadanie z Wikieł Z 1.1 d) moja wersja nr 153}

Obliczyć wartość wyrażenia $(\frac{179}{103})^{3} \cdot (\frac{103}{179})^{3} \cdot \pi^{0}$.
\zadStop
\rozwStart{Patryk Wirkus}{Martyna Czarnobaj}
$$(\frac{179}{103})^{3} \cdot (\frac{103}{179})^{3} \cdot \pi^{0} = (\frac{179}{103} \cdot \frac{103}{179})^{3} \cdot 1 = 1^{3} \cdot 1 = 1$$
\rozwStop
\odpStart
$1$
\odpStop
\testStart
A.$1$ B.$\pi$ C.$0$ D.$\frac{179}{103}$ E.$\frac{103}{179}$
F.$-\frac{179}{103}$ G.$-1$
H.$(\frac{179}{103})^{3}$
I.$(\frac{103}{179})^{3}$
\testStop
\kluczStart
A
\kluczStop



\zadStart{Zadanie z Wikieł Z 1.1 d) moja wersja nr 154}

Obliczyć wartość wyrażenia $(\frac{179}{107})^{3} \cdot (\frac{107}{179})^{3} \cdot \pi^{0}$.
\zadStop
\rozwStart{Patryk Wirkus}{Martyna Czarnobaj}
$$(\frac{179}{107})^{3} \cdot (\frac{107}{179})^{3} \cdot \pi^{0} = (\frac{179}{107} \cdot \frac{107}{179})^{3} \cdot 1 = 1^{3} \cdot 1 = 1$$
\rozwStop
\odpStart
$1$
\odpStop
\testStart
A.$1$ B.$\pi$ C.$0$ D.$\frac{179}{107}$ E.$\frac{107}{179}$
F.$-\frac{179}{107}$ G.$-1$
H.$(\frac{179}{107})^{3}$
I.$(\frac{107}{179})^{3}$
\testStop
\kluczStart
A
\kluczStop



\zadStart{Zadanie z Wikieł Z 1.1 d) moja wersja nr 155}

Obliczyć wartość wyrażenia $(\frac{179}{109})^{3} \cdot (\frac{109}{179})^{3} \cdot \pi^{0}$.
\zadStop
\rozwStart{Patryk Wirkus}{Martyna Czarnobaj}
$$(\frac{179}{109})^{3} \cdot (\frac{109}{179})^{3} \cdot \pi^{0} = (\frac{179}{109} \cdot \frac{109}{179})^{3} \cdot 1 = 1^{3} \cdot 1 = 1$$
\rozwStop
\odpStart
$1$
\odpStop
\testStart
A.$1$ B.$\pi$ C.$0$ D.$\frac{179}{109}$ E.$\frac{109}{179}$
F.$-\frac{179}{109}$ G.$-1$
H.$(\frac{179}{109})^{3}$
I.$(\frac{109}{179})^{3}$
\testStop
\kluczStart
A
\kluczStop



\zadStart{Zadanie z Wikieł Z 1.1 d) moja wersja nr 156}

Obliczyć wartość wyrażenia $(\frac{179}{113})^{3} \cdot (\frac{113}{179})^{3} \cdot \pi^{0}$.
\zadStop
\rozwStart{Patryk Wirkus}{Martyna Czarnobaj}
$$(\frac{179}{113})^{3} \cdot (\frac{113}{179})^{3} \cdot \pi^{0} = (\frac{179}{113} \cdot \frac{113}{179})^{3} \cdot 1 = 1^{3} \cdot 1 = 1$$
\rozwStop
\odpStart
$1$
\odpStop
\testStart
A.$1$ B.$\pi$ C.$0$ D.$\frac{179}{113}$ E.$\frac{113}{179}$
F.$-\frac{179}{113}$ G.$-1$
H.$(\frac{179}{113})^{3}$
I.$(\frac{113}{179})^{3}$
\testStop
\kluczStart
A
\kluczStop



\zadStart{Zadanie z Wikieł Z 1.1 d) moja wersja nr 157}

Obliczyć wartość wyrażenia $(\frac{179}{127})^{3} \cdot (\frac{127}{179})^{3} \cdot \pi^{0}$.
\zadStop
\rozwStart{Patryk Wirkus}{Martyna Czarnobaj}
$$(\frac{179}{127})^{3} \cdot (\frac{127}{179})^{3} \cdot \pi^{0} = (\frac{179}{127} \cdot \frac{127}{179})^{3} \cdot 1 = 1^{3} \cdot 1 = 1$$
\rozwStop
\odpStart
$1$
\odpStop
\testStart
A.$1$ B.$\pi$ C.$0$ D.$\frac{179}{127}$ E.$\frac{127}{179}$
F.$-\frac{179}{127}$ G.$-1$
H.$(\frac{179}{127})^{3}$
I.$(\frac{127}{179})^{3}$
\testStop
\kluczStart
A
\kluczStop



\zadStart{Zadanie z Wikieł Z 1.1 d) moja wersja nr 158}

Obliczyć wartość wyrażenia $(\frac{179}{131})^{3} \cdot (\frac{131}{179})^{3} \cdot \pi^{0}$.
\zadStop
\rozwStart{Patryk Wirkus}{Martyna Czarnobaj}
$$(\frac{179}{131})^{3} \cdot (\frac{131}{179})^{3} \cdot \pi^{0} = (\frac{179}{131} \cdot \frac{131}{179})^{3} \cdot 1 = 1^{3} \cdot 1 = 1$$
\rozwStop
\odpStart
$1$
\odpStop
\testStart
A.$1$ B.$\pi$ C.$0$ D.$\frac{179}{131}$ E.$\frac{131}{179}$
F.$-\frac{179}{131}$ G.$-1$
H.$(\frac{179}{131})^{3}$
I.$(\frac{131}{179})^{3}$
\testStop
\kluczStart
A
\kluczStop



\zadStart{Zadanie z Wikieł Z 1.1 d) moja wersja nr 159}

Obliczyć wartość wyrażenia $(\frac{179}{137})^{3} \cdot (\frac{137}{179})^{3} \cdot \pi^{0}$.
\zadStop
\rozwStart{Patryk Wirkus}{Martyna Czarnobaj}
$$(\frac{179}{137})^{3} \cdot (\frac{137}{179})^{3} \cdot \pi^{0} = (\frac{179}{137} \cdot \frac{137}{179})^{3} \cdot 1 = 1^{3} \cdot 1 = 1$$
\rozwStop
\odpStart
$1$
\odpStop
\testStart
A.$1$ B.$\pi$ C.$0$ D.$\frac{179}{137}$ E.$\frac{137}{179}$
F.$-\frac{179}{137}$ G.$-1$
H.$(\frac{179}{137})^{3}$
I.$(\frac{137}{179})^{3}$
\testStop
\kluczStart
A
\kluczStop



\zadStart{Zadanie z Wikieł Z 1.1 d) moja wersja nr 160}

Obliczyć wartość wyrażenia $(\frac{179}{139})^{3} \cdot (\frac{139}{179})^{3} \cdot \pi^{0}$.
\zadStop
\rozwStart{Patryk Wirkus}{Martyna Czarnobaj}
$$(\frac{179}{139})^{3} \cdot (\frac{139}{179})^{3} \cdot \pi^{0} = (\frac{179}{139} \cdot \frac{139}{179})^{3} \cdot 1 = 1^{3} \cdot 1 = 1$$
\rozwStop
\odpStart
$1$
\odpStop
\testStart
A.$1$ B.$\pi$ C.$0$ D.$\frac{179}{139}$ E.$\frac{139}{179}$
F.$-\frac{179}{139}$ G.$-1$
H.$(\frac{179}{139})^{3}$
I.$(\frac{139}{179})^{3}$
\testStop
\kluczStart
A
\kluczStop



\zadStart{Zadanie z Wikieł Z 1.1 d) moja wersja nr 161}

Obliczyć wartość wyrażenia $(\frac{251}{103})^{3} \cdot (\frac{103}{251})^{3} \cdot \pi^{0}$.
\zadStop
\rozwStart{Patryk Wirkus}{Martyna Czarnobaj}
$$(\frac{251}{103})^{3} \cdot (\frac{103}{251})^{3} \cdot \pi^{0} = (\frac{251}{103} \cdot \frac{103}{251})^{3} \cdot 1 = 1^{3} \cdot 1 = 1$$
\rozwStop
\odpStart
$1$
\odpStop
\testStart
A.$1$ B.$\pi$ C.$0$ D.$\frac{251}{103}$ E.$\frac{103}{251}$
F.$-\frac{251}{103}$ G.$-1$
H.$(\frac{251}{103})^{3}$
I.$(\frac{103}{251})^{3}$
\testStop
\kluczStart
A
\kluczStop



\zadStart{Zadanie z Wikieł Z 1.1 d) moja wersja nr 162}

Obliczyć wartość wyrażenia $(\frac{251}{107})^{3} \cdot (\frac{107}{251})^{3} \cdot \pi^{0}$.
\zadStop
\rozwStart{Patryk Wirkus}{Martyna Czarnobaj}
$$(\frac{251}{107})^{3} \cdot (\frac{107}{251})^{3} \cdot \pi^{0} = (\frac{251}{107} \cdot \frac{107}{251})^{3} \cdot 1 = 1^{3} \cdot 1 = 1$$
\rozwStop
\odpStart
$1$
\odpStop
\testStart
A.$1$ B.$\pi$ C.$0$ D.$\frac{251}{107}$ E.$\frac{107}{251}$
F.$-\frac{251}{107}$ G.$-1$
H.$(\frac{251}{107})^{3}$
I.$(\frac{107}{251})^{3}$
\testStop
\kluczStart
A
\kluczStop



\zadStart{Zadanie z Wikieł Z 1.1 d) moja wersja nr 163}

Obliczyć wartość wyrażenia $(\frac{251}{109})^{3} \cdot (\frac{109}{251})^{3} \cdot \pi^{0}$.
\zadStop
\rozwStart{Patryk Wirkus}{Martyna Czarnobaj}
$$(\frac{251}{109})^{3} \cdot (\frac{109}{251})^{3} \cdot \pi^{0} = (\frac{251}{109} \cdot \frac{109}{251})^{3} \cdot 1 = 1^{3} \cdot 1 = 1$$
\rozwStop
\odpStart
$1$
\odpStop
\testStart
A.$1$ B.$\pi$ C.$0$ D.$\frac{251}{109}$ E.$\frac{109}{251}$
F.$-\frac{251}{109}$ G.$-1$
H.$(\frac{251}{109})^{3}$
I.$(\frac{109}{251})^{3}$
\testStop
\kluczStart
A
\kluczStop



\zadStart{Zadanie z Wikieł Z 1.1 d) moja wersja nr 164}

Obliczyć wartość wyrażenia $(\frac{251}{113})^{3} \cdot (\frac{113}{251})^{3} \cdot \pi^{0}$.
\zadStop
\rozwStart{Patryk Wirkus}{Martyna Czarnobaj}
$$(\frac{251}{113})^{3} \cdot (\frac{113}{251})^{3} \cdot \pi^{0} = (\frac{251}{113} \cdot \frac{113}{251})^{3} \cdot 1 = 1^{3} \cdot 1 = 1$$
\rozwStop
\odpStart
$1$
\odpStop
\testStart
A.$1$ B.$\pi$ C.$0$ D.$\frac{251}{113}$ E.$\frac{113}{251}$
F.$-\frac{251}{113}$ G.$-1$
H.$(\frac{251}{113})^{3}$
I.$(\frac{113}{251})^{3}$
\testStop
\kluczStart
A
\kluczStop



\zadStart{Zadanie z Wikieł Z 1.1 d) moja wersja nr 165}

Obliczyć wartość wyrażenia $(\frac{251}{127})^{3} \cdot (\frac{127}{251})^{3} \cdot \pi^{0}$.
\zadStop
\rozwStart{Patryk Wirkus}{Martyna Czarnobaj}
$$(\frac{251}{127})^{3} \cdot (\frac{127}{251})^{3} \cdot \pi^{0} = (\frac{251}{127} \cdot \frac{127}{251})^{3} \cdot 1 = 1^{3} \cdot 1 = 1$$
\rozwStop
\odpStart
$1$
\odpStop
\testStart
A.$1$ B.$\pi$ C.$0$ D.$\frac{251}{127}$ E.$\frac{127}{251}$
F.$-\frac{251}{127}$ G.$-1$
H.$(\frac{251}{127})^{3}$
I.$(\frac{127}{251})^{3}$
\testStop
\kluczStart
A
\kluczStop



\zadStart{Zadanie z Wikieł Z 1.1 d) moja wersja nr 166}

Obliczyć wartość wyrażenia $(\frac{251}{131})^{3} \cdot (\frac{131}{251})^{3} \cdot \pi^{0}$.
\zadStop
\rozwStart{Patryk Wirkus}{Martyna Czarnobaj}
$$(\frac{251}{131})^{3} \cdot (\frac{131}{251})^{3} \cdot \pi^{0} = (\frac{251}{131} \cdot \frac{131}{251})^{3} \cdot 1 = 1^{3} \cdot 1 = 1$$
\rozwStop
\odpStart
$1$
\odpStop
\testStart
A.$1$ B.$\pi$ C.$0$ D.$\frac{251}{131}$ E.$\frac{131}{251}$
F.$-\frac{251}{131}$ G.$-1$
H.$(\frac{251}{131})^{3}$
I.$(\frac{131}{251})^{3}$
\testStop
\kluczStart
A
\kluczStop



\zadStart{Zadanie z Wikieł Z 1.1 d) moja wersja nr 167}

Obliczyć wartość wyrażenia $(\frac{251}{137})^{3} \cdot (\frac{137}{251})^{3} \cdot \pi^{0}$.
\zadStop
\rozwStart{Patryk Wirkus}{Martyna Czarnobaj}
$$(\frac{251}{137})^{3} \cdot (\frac{137}{251})^{3} \cdot \pi^{0} = (\frac{251}{137} \cdot \frac{137}{251})^{3} \cdot 1 = 1^{3} \cdot 1 = 1$$
\rozwStop
\odpStart
$1$
\odpStop
\testStart
A.$1$ B.$\pi$ C.$0$ D.$\frac{251}{137}$ E.$\frac{137}{251}$
F.$-\frac{251}{137}$ G.$-1$
H.$(\frac{251}{137})^{3}$
I.$(\frac{137}{251})^{3}$
\testStop
\kluczStart
A
\kluczStop



\zadStart{Zadanie z Wikieł Z 1.1 d) moja wersja nr 168}

Obliczyć wartość wyrażenia $(\frac{251}{139})^{3} \cdot (\frac{139}{251})^{3} \cdot \pi^{0}$.
\zadStop
\rozwStart{Patryk Wirkus}{Martyna Czarnobaj}
$$(\frac{251}{139})^{3} \cdot (\frac{139}{251})^{3} \cdot \pi^{0} = (\frac{251}{139} \cdot \frac{139}{251})^{3} \cdot 1 = 1^{3} \cdot 1 = 1$$
\rozwStop
\odpStart
$1$
\odpStop
\testStart
A.$1$ B.$\pi$ C.$0$ D.$\frac{251}{139}$ E.$\frac{139}{251}$
F.$-\frac{251}{139}$ G.$-1$
H.$(\frac{251}{139})^{3}$
I.$(\frac{139}{251})^{3}$
\testStop
\kluczStart
A
\kluczStop



\zadStart{Zadanie z Wikieł Z 1.1 d) moja wersja nr 169}

Obliczyć wartość wyrażenia $(\frac{257}{103})^{3} \cdot (\frac{103}{257})^{3} \cdot \pi^{0}$.
\zadStop
\rozwStart{Patryk Wirkus}{Martyna Czarnobaj}
$$(\frac{257}{103})^{3} \cdot (\frac{103}{257})^{3} \cdot \pi^{0} = (\frac{257}{103} \cdot \frac{103}{257})^{3} \cdot 1 = 1^{3} \cdot 1 = 1$$
\rozwStop
\odpStart
$1$
\odpStop
\testStart
A.$1$ B.$\pi$ C.$0$ D.$\frac{257}{103}$ E.$\frac{103}{257}$
F.$-\frac{257}{103}$ G.$-1$
H.$(\frac{257}{103})^{3}$
I.$(\frac{103}{257})^{3}$
\testStop
\kluczStart
A
\kluczStop



\zadStart{Zadanie z Wikieł Z 1.1 d) moja wersja nr 170}

Obliczyć wartość wyrażenia $(\frac{257}{107})^{3} \cdot (\frac{107}{257})^{3} \cdot \pi^{0}$.
\zadStop
\rozwStart{Patryk Wirkus}{Martyna Czarnobaj}
$$(\frac{257}{107})^{3} \cdot (\frac{107}{257})^{3} \cdot \pi^{0} = (\frac{257}{107} \cdot \frac{107}{257})^{3} \cdot 1 = 1^{3} \cdot 1 = 1$$
\rozwStop
\odpStart
$1$
\odpStop
\testStart
A.$1$ B.$\pi$ C.$0$ D.$\frac{257}{107}$ E.$\frac{107}{257}$
F.$-\frac{257}{107}$ G.$-1$
H.$(\frac{257}{107})^{3}$
I.$(\frac{107}{257})^{3}$
\testStop
\kluczStart
A
\kluczStop



\zadStart{Zadanie z Wikieł Z 1.1 d) moja wersja nr 171}

Obliczyć wartość wyrażenia $(\frac{257}{109})^{3} \cdot (\frac{109}{257})^{3} \cdot \pi^{0}$.
\zadStop
\rozwStart{Patryk Wirkus}{Martyna Czarnobaj}
$$(\frac{257}{109})^{3} \cdot (\frac{109}{257})^{3} \cdot \pi^{0} = (\frac{257}{109} \cdot \frac{109}{257})^{3} \cdot 1 = 1^{3} \cdot 1 = 1$$
\rozwStop
\odpStart
$1$
\odpStop
\testStart
A.$1$ B.$\pi$ C.$0$ D.$\frac{257}{109}$ E.$\frac{109}{257}$
F.$-\frac{257}{109}$ G.$-1$
H.$(\frac{257}{109})^{3}$
I.$(\frac{109}{257})^{3}$
\testStop
\kluczStart
A
\kluczStop



\zadStart{Zadanie z Wikieł Z 1.1 d) moja wersja nr 172}

Obliczyć wartość wyrażenia $(\frac{257}{113})^{3} \cdot (\frac{113}{257})^{3} \cdot \pi^{0}$.
\zadStop
\rozwStart{Patryk Wirkus}{Martyna Czarnobaj}
$$(\frac{257}{113})^{3} \cdot (\frac{113}{257})^{3} \cdot \pi^{0} = (\frac{257}{113} \cdot \frac{113}{257})^{3} \cdot 1 = 1^{3} \cdot 1 = 1$$
\rozwStop
\odpStart
$1$
\odpStop
\testStart
A.$1$ B.$\pi$ C.$0$ D.$\frac{257}{113}$ E.$\frac{113}{257}$
F.$-\frac{257}{113}$ G.$-1$
H.$(\frac{257}{113})^{3}$
I.$(\frac{113}{257})^{3}$
\testStop
\kluczStart
A
\kluczStop



\zadStart{Zadanie z Wikieł Z 1.1 d) moja wersja nr 173}

Obliczyć wartość wyrażenia $(\frac{257}{127})^{3} \cdot (\frac{127}{257})^{3} \cdot \pi^{0}$.
\zadStop
\rozwStart{Patryk Wirkus}{Martyna Czarnobaj}
$$(\frac{257}{127})^{3} \cdot (\frac{127}{257})^{3} \cdot \pi^{0} = (\frac{257}{127} \cdot \frac{127}{257})^{3} \cdot 1 = 1^{3} \cdot 1 = 1$$
\rozwStop
\odpStart
$1$
\odpStop
\testStart
A.$1$ B.$\pi$ C.$0$ D.$\frac{257}{127}$ E.$\frac{127}{257}$
F.$-\frac{257}{127}$ G.$-1$
H.$(\frac{257}{127})^{3}$
I.$(\frac{127}{257})^{3}$
\testStop
\kluczStart
A
\kluczStop



\zadStart{Zadanie z Wikieł Z 1.1 d) moja wersja nr 174}

Obliczyć wartość wyrażenia $(\frac{257}{131})^{3} \cdot (\frac{131}{257})^{3} \cdot \pi^{0}$.
\zadStop
\rozwStart{Patryk Wirkus}{Martyna Czarnobaj}
$$(\frac{257}{131})^{3} \cdot (\frac{131}{257})^{3} \cdot \pi^{0} = (\frac{257}{131} \cdot \frac{131}{257})^{3} \cdot 1 = 1^{3} \cdot 1 = 1$$
\rozwStop
\odpStart
$1$
\odpStop
\testStart
A.$1$ B.$\pi$ C.$0$ D.$\frac{257}{131}$ E.$\frac{131}{257}$
F.$-\frac{257}{131}$ G.$-1$
H.$(\frac{257}{131})^{3}$
I.$(\frac{131}{257})^{3}$
\testStop
\kluczStart
A
\kluczStop



\zadStart{Zadanie z Wikieł Z 1.1 d) moja wersja nr 175}

Obliczyć wartość wyrażenia $(\frac{257}{137})^{3} \cdot (\frac{137}{257})^{3} \cdot \pi^{0}$.
\zadStop
\rozwStart{Patryk Wirkus}{Martyna Czarnobaj}
$$(\frac{257}{137})^{3} \cdot (\frac{137}{257})^{3} \cdot \pi^{0} = (\frac{257}{137} \cdot \frac{137}{257})^{3} \cdot 1 = 1^{3} \cdot 1 = 1$$
\rozwStop
\odpStart
$1$
\odpStop
\testStart
A.$1$ B.$\pi$ C.$0$ D.$\frac{257}{137}$ E.$\frac{137}{257}$
F.$-\frac{257}{137}$ G.$-1$
H.$(\frac{257}{137})^{3}$
I.$(\frac{137}{257})^{3}$
\testStop
\kluczStart
A
\kluczStop



\zadStart{Zadanie z Wikieł Z 1.1 d) moja wersja nr 176}

Obliczyć wartość wyrażenia $(\frac{257}{139})^{3} \cdot (\frac{139}{257})^{3} \cdot \pi^{0}$.
\zadStop
\rozwStart{Patryk Wirkus}{Martyna Czarnobaj}
$$(\frac{257}{139})^{3} \cdot (\frac{139}{257})^{3} \cdot \pi^{0} = (\frac{257}{139} \cdot \frac{139}{257})^{3} \cdot 1 = 1^{3} \cdot 1 = 1$$
\rozwStop
\odpStart
$1$
\odpStop
\testStart
A.$1$ B.$\pi$ C.$0$ D.$\frac{257}{139}$ E.$\frac{139}{257}$
F.$-\frac{257}{139}$ G.$-1$
H.$(\frac{257}{139})^{3}$
I.$(\frac{139}{257})^{3}$
\testStop
\kluczStart
A
\kluczStop



\zadStart{Zadanie z Wikieł Z 1.1 d) moja wersja nr 177}

Obliczyć wartość wyrażenia $(\frac{263}{103})^{3} \cdot (\frac{103}{263})^{3} \cdot \pi^{0}$.
\zadStop
\rozwStart{Patryk Wirkus}{Martyna Czarnobaj}
$$(\frac{263}{103})^{3} \cdot (\frac{103}{263})^{3} \cdot \pi^{0} = (\frac{263}{103} \cdot \frac{103}{263})^{3} \cdot 1 = 1^{3} \cdot 1 = 1$$
\rozwStop
\odpStart
$1$
\odpStop
\testStart
A.$1$ B.$\pi$ C.$0$ D.$\frac{263}{103}$ E.$\frac{103}{263}$
F.$-\frac{263}{103}$ G.$-1$
H.$(\frac{263}{103})^{3}$
I.$(\frac{103}{263})^{3}$
\testStop
\kluczStart
A
\kluczStop



\zadStart{Zadanie z Wikieł Z 1.1 d) moja wersja nr 178}

Obliczyć wartość wyrażenia $(\frac{263}{107})^{3} \cdot (\frac{107}{263})^{3} \cdot \pi^{0}$.
\zadStop
\rozwStart{Patryk Wirkus}{Martyna Czarnobaj}
$$(\frac{263}{107})^{3} \cdot (\frac{107}{263})^{3} \cdot \pi^{0} = (\frac{263}{107} \cdot \frac{107}{263})^{3} \cdot 1 = 1^{3} \cdot 1 = 1$$
\rozwStop
\odpStart
$1$
\odpStop
\testStart
A.$1$ B.$\pi$ C.$0$ D.$\frac{263}{107}$ E.$\frac{107}{263}$
F.$-\frac{263}{107}$ G.$-1$
H.$(\frac{263}{107})^{3}$
I.$(\frac{107}{263})^{3}$
\testStop
\kluczStart
A
\kluczStop



\zadStart{Zadanie z Wikieł Z 1.1 d) moja wersja nr 179}

Obliczyć wartość wyrażenia $(\frac{263}{109})^{3} \cdot (\frac{109}{263})^{3} \cdot \pi^{0}$.
\zadStop
\rozwStart{Patryk Wirkus}{Martyna Czarnobaj}
$$(\frac{263}{109})^{3} \cdot (\frac{109}{263})^{3} \cdot \pi^{0} = (\frac{263}{109} \cdot \frac{109}{263})^{3} \cdot 1 = 1^{3} \cdot 1 = 1$$
\rozwStop
\odpStart
$1$
\odpStop
\testStart
A.$1$ B.$\pi$ C.$0$ D.$\frac{263}{109}$ E.$\frac{109}{263}$
F.$-\frac{263}{109}$ G.$-1$
H.$(\frac{263}{109})^{3}$
I.$(\frac{109}{263})^{3}$
\testStop
\kluczStart
A
\kluczStop



\zadStart{Zadanie z Wikieł Z 1.1 d) moja wersja nr 180}

Obliczyć wartość wyrażenia $(\frac{263}{113})^{3} \cdot (\frac{113}{263})^{3} \cdot \pi^{0}$.
\zadStop
\rozwStart{Patryk Wirkus}{Martyna Czarnobaj}
$$(\frac{263}{113})^{3} \cdot (\frac{113}{263})^{3} \cdot \pi^{0} = (\frac{263}{113} \cdot \frac{113}{263})^{3} \cdot 1 = 1^{3} \cdot 1 = 1$$
\rozwStop
\odpStart
$1$
\odpStop
\testStart
A.$1$ B.$\pi$ C.$0$ D.$\frac{263}{113}$ E.$\frac{113}{263}$
F.$-\frac{263}{113}$ G.$-1$
H.$(\frac{263}{113})^{3}$
I.$(\frac{113}{263})^{3}$
\testStop
\kluczStart
A
\kluczStop



\zadStart{Zadanie z Wikieł Z 1.1 d) moja wersja nr 181}

Obliczyć wartość wyrażenia $(\frac{263}{127})^{3} \cdot (\frac{127}{263})^{3} \cdot \pi^{0}$.
\zadStop
\rozwStart{Patryk Wirkus}{Martyna Czarnobaj}
$$(\frac{263}{127})^{3} \cdot (\frac{127}{263})^{3} \cdot \pi^{0} = (\frac{263}{127} \cdot \frac{127}{263})^{3} \cdot 1 = 1^{3} \cdot 1 = 1$$
\rozwStop
\odpStart
$1$
\odpStop
\testStart
A.$1$ B.$\pi$ C.$0$ D.$\frac{263}{127}$ E.$\frac{127}{263}$
F.$-\frac{263}{127}$ G.$-1$
H.$(\frac{263}{127})^{3}$
I.$(\frac{127}{263})^{3}$
\testStop
\kluczStart
A
\kluczStop



\zadStart{Zadanie z Wikieł Z 1.1 d) moja wersja nr 182}

Obliczyć wartość wyrażenia $(\frac{263}{131})^{3} \cdot (\frac{131}{263})^{3} \cdot \pi^{0}$.
\zadStop
\rozwStart{Patryk Wirkus}{Martyna Czarnobaj}
$$(\frac{263}{131})^{3} \cdot (\frac{131}{263})^{3} \cdot \pi^{0} = (\frac{263}{131} \cdot \frac{131}{263})^{3} \cdot 1 = 1^{3} \cdot 1 = 1$$
\rozwStop
\odpStart
$1$
\odpStop
\testStart
A.$1$ B.$\pi$ C.$0$ D.$\frac{263}{131}$ E.$\frac{131}{263}$
F.$-\frac{263}{131}$ G.$-1$
H.$(\frac{263}{131})^{3}$
I.$(\frac{131}{263})^{3}$
\testStop
\kluczStart
A
\kluczStop



\zadStart{Zadanie z Wikieł Z 1.1 d) moja wersja nr 183}

Obliczyć wartość wyrażenia $(\frac{263}{137})^{3} \cdot (\frac{137}{263})^{3} \cdot \pi^{0}$.
\zadStop
\rozwStart{Patryk Wirkus}{Martyna Czarnobaj}
$$(\frac{263}{137})^{3} \cdot (\frac{137}{263})^{3} \cdot \pi^{0} = (\frac{263}{137} \cdot \frac{137}{263})^{3} \cdot 1 = 1^{3} \cdot 1 = 1$$
\rozwStop
\odpStart
$1$
\odpStop
\testStart
A.$1$ B.$\pi$ C.$0$ D.$\frac{263}{137}$ E.$\frac{137}{263}$
F.$-\frac{263}{137}$ G.$-1$
H.$(\frac{263}{137})^{3}$
I.$(\frac{137}{263})^{3}$
\testStop
\kluczStart
A
\kluczStop



\zadStart{Zadanie z Wikieł Z 1.1 d) moja wersja nr 184}

Obliczyć wartość wyrażenia $(\frac{263}{139})^{3} \cdot (\frac{139}{263})^{3} \cdot \pi^{0}$.
\zadStop
\rozwStart{Patryk Wirkus}{Martyna Czarnobaj}
$$(\frac{263}{139})^{3} \cdot (\frac{139}{263})^{3} \cdot \pi^{0} = (\frac{263}{139} \cdot \frac{139}{263})^{3} \cdot 1 = 1^{3} \cdot 1 = 1$$
\rozwStop
\odpStart
$1$
\odpStop
\testStart
A.$1$ B.$\pi$ C.$0$ D.$\frac{263}{139}$ E.$\frac{139}{263}$
F.$-\frac{263}{139}$ G.$-1$
H.$(\frac{263}{139})^{3}$
I.$(\frac{139}{263})^{3}$
\testStop
\kluczStart
A
\kluczStop



\zadStart{Zadanie z Wikieł Z 1.1 d) moja wersja nr 185}

Obliczyć wartość wyrażenia $(\frac{269}{103})^{3} \cdot (\frac{103}{269})^{3} \cdot \pi^{0}$.
\zadStop
\rozwStart{Patryk Wirkus}{Martyna Czarnobaj}
$$(\frac{269}{103})^{3} \cdot (\frac{103}{269})^{3} \cdot \pi^{0} = (\frac{269}{103} \cdot \frac{103}{269})^{3} \cdot 1 = 1^{3} \cdot 1 = 1$$
\rozwStop
\odpStart
$1$
\odpStop
\testStart
A.$1$ B.$\pi$ C.$0$ D.$\frac{269}{103}$ E.$\frac{103}{269}$
F.$-\frac{269}{103}$ G.$-1$
H.$(\frac{269}{103})^{3}$
I.$(\frac{103}{269})^{3}$
\testStop
\kluczStart
A
\kluczStop



\zadStart{Zadanie z Wikieł Z 1.1 d) moja wersja nr 186}

Obliczyć wartość wyrażenia $(\frac{269}{107})^{3} \cdot (\frac{107}{269})^{3} \cdot \pi^{0}$.
\zadStop
\rozwStart{Patryk Wirkus}{Martyna Czarnobaj}
$$(\frac{269}{107})^{3} \cdot (\frac{107}{269})^{3} \cdot \pi^{0} = (\frac{269}{107} \cdot \frac{107}{269})^{3} \cdot 1 = 1^{3} \cdot 1 = 1$$
\rozwStop
\odpStart
$1$
\odpStop
\testStart
A.$1$ B.$\pi$ C.$0$ D.$\frac{269}{107}$ E.$\frac{107}{269}$
F.$-\frac{269}{107}$ G.$-1$
H.$(\frac{269}{107})^{3}$
I.$(\frac{107}{269})^{3}$
\testStop
\kluczStart
A
\kluczStop



\zadStart{Zadanie z Wikieł Z 1.1 d) moja wersja nr 187}

Obliczyć wartość wyrażenia $(\frac{269}{109})^{3} \cdot (\frac{109}{269})^{3} \cdot \pi^{0}$.
\zadStop
\rozwStart{Patryk Wirkus}{Martyna Czarnobaj}
$$(\frac{269}{109})^{3} \cdot (\frac{109}{269})^{3} \cdot \pi^{0} = (\frac{269}{109} \cdot \frac{109}{269})^{3} \cdot 1 = 1^{3} \cdot 1 = 1$$
\rozwStop
\odpStart
$1$
\odpStop
\testStart
A.$1$ B.$\pi$ C.$0$ D.$\frac{269}{109}$ E.$\frac{109}{269}$
F.$-\frac{269}{109}$ G.$-1$
H.$(\frac{269}{109})^{3}$
I.$(\frac{109}{269})^{3}$
\testStop
\kluczStart
A
\kluczStop



\zadStart{Zadanie z Wikieł Z 1.1 d) moja wersja nr 188}

Obliczyć wartość wyrażenia $(\frac{269}{113})^{3} \cdot (\frac{113}{269})^{3} \cdot \pi^{0}$.
\zadStop
\rozwStart{Patryk Wirkus}{Martyna Czarnobaj}
$$(\frac{269}{113})^{3} \cdot (\frac{113}{269})^{3} \cdot \pi^{0} = (\frac{269}{113} \cdot \frac{113}{269})^{3} \cdot 1 = 1^{3} \cdot 1 = 1$$
\rozwStop
\odpStart
$1$
\odpStop
\testStart
A.$1$ B.$\pi$ C.$0$ D.$\frac{269}{113}$ E.$\frac{113}{269}$
F.$-\frac{269}{113}$ G.$-1$
H.$(\frac{269}{113})^{3}$
I.$(\frac{113}{269})^{3}$
\testStop
\kluczStart
A
\kluczStop



\zadStart{Zadanie z Wikieł Z 1.1 d) moja wersja nr 189}

Obliczyć wartość wyrażenia $(\frac{269}{127})^{3} \cdot (\frac{127}{269})^{3} \cdot \pi^{0}$.
\zadStop
\rozwStart{Patryk Wirkus}{Martyna Czarnobaj}
$$(\frac{269}{127})^{3} \cdot (\frac{127}{269})^{3} \cdot \pi^{0} = (\frac{269}{127} \cdot \frac{127}{269})^{3} \cdot 1 = 1^{3} \cdot 1 = 1$$
\rozwStop
\odpStart
$1$
\odpStop
\testStart
A.$1$ B.$\pi$ C.$0$ D.$\frac{269}{127}$ E.$\frac{127}{269}$
F.$-\frac{269}{127}$ G.$-1$
H.$(\frac{269}{127})^{3}$
I.$(\frac{127}{269})^{3}$
\testStop
\kluczStart
A
\kluczStop



\zadStart{Zadanie z Wikieł Z 1.1 d) moja wersja nr 190}

Obliczyć wartość wyrażenia $(\frac{269}{131})^{3} \cdot (\frac{131}{269})^{3} \cdot \pi^{0}$.
\zadStop
\rozwStart{Patryk Wirkus}{Martyna Czarnobaj}
$$(\frac{269}{131})^{3} \cdot (\frac{131}{269})^{3} \cdot \pi^{0} = (\frac{269}{131} \cdot \frac{131}{269})^{3} \cdot 1 = 1^{3} \cdot 1 = 1$$
\rozwStop
\odpStart
$1$
\odpStop
\testStart
A.$1$ B.$\pi$ C.$0$ D.$\frac{269}{131}$ E.$\frac{131}{269}$
F.$-\frac{269}{131}$ G.$-1$
H.$(\frac{269}{131})^{3}$
I.$(\frac{131}{269})^{3}$
\testStop
\kluczStart
A
\kluczStop



\zadStart{Zadanie z Wikieł Z 1.1 d) moja wersja nr 191}

Obliczyć wartość wyrażenia $(\frac{269}{137})^{3} \cdot (\frac{137}{269})^{3} \cdot \pi^{0}$.
\zadStop
\rozwStart{Patryk Wirkus}{Martyna Czarnobaj}
$$(\frac{269}{137})^{3} \cdot (\frac{137}{269})^{3} \cdot \pi^{0} = (\frac{269}{137} \cdot \frac{137}{269})^{3} \cdot 1 = 1^{3} \cdot 1 = 1$$
\rozwStop
\odpStart
$1$
\odpStop
\testStart
A.$1$ B.$\pi$ C.$0$ D.$\frac{269}{137}$ E.$\frac{137}{269}$
F.$-\frac{269}{137}$ G.$-1$
H.$(\frac{269}{137})^{3}$
I.$(\frac{137}{269})^{3}$
\testStop
\kluczStart
A
\kluczStop



\zadStart{Zadanie z Wikieł Z 1.1 d) moja wersja nr 192}

Obliczyć wartość wyrażenia $(\frac{269}{139})^{3} \cdot (\frac{139}{269})^{3} \cdot \pi^{0}$.
\zadStop
\rozwStart{Patryk Wirkus}{Martyna Czarnobaj}
$$(\frac{269}{139})^{3} \cdot (\frac{139}{269})^{3} \cdot \pi^{0} = (\frac{269}{139} \cdot \frac{139}{269})^{3} \cdot 1 = 1^{3} \cdot 1 = 1$$
\rozwStop
\odpStart
$1$
\odpStop
\testStart
A.$1$ B.$\pi$ C.$0$ D.$\frac{269}{139}$ E.$\frac{139}{269}$
F.$-\frac{269}{139}$ G.$-1$
H.$(\frac{269}{139})^{3}$
I.$(\frac{139}{269})^{3}$
\testStop
\kluczStart
A
\kluczStop



\zadStart{Zadanie z Wikieł Z 1.1 d) moja wersja nr 193}

Obliczyć wartość wyrażenia $(\frac{271}{103})^{3} \cdot (\frac{103}{271})^{3} \cdot \pi^{0}$.
\zadStop
\rozwStart{Patryk Wirkus}{Martyna Czarnobaj}
$$(\frac{271}{103})^{3} \cdot (\frac{103}{271})^{3} \cdot \pi^{0} = (\frac{271}{103} \cdot \frac{103}{271})^{3} \cdot 1 = 1^{3} \cdot 1 = 1$$
\rozwStop
\odpStart
$1$
\odpStop
\testStart
A.$1$ B.$\pi$ C.$0$ D.$\frac{271}{103}$ E.$\frac{103}{271}$
F.$-\frac{271}{103}$ G.$-1$
H.$(\frac{271}{103})^{3}$
I.$(\frac{103}{271})^{3}$
\testStop
\kluczStart
A
\kluczStop



\zadStart{Zadanie z Wikieł Z 1.1 d) moja wersja nr 194}

Obliczyć wartość wyrażenia $(\frac{271}{107})^{3} \cdot (\frac{107}{271})^{3} \cdot \pi^{0}$.
\zadStop
\rozwStart{Patryk Wirkus}{Martyna Czarnobaj}
$$(\frac{271}{107})^{3} \cdot (\frac{107}{271})^{3} \cdot \pi^{0} = (\frac{271}{107} \cdot \frac{107}{271})^{3} \cdot 1 = 1^{3} \cdot 1 = 1$$
\rozwStop
\odpStart
$1$
\odpStop
\testStart
A.$1$ B.$\pi$ C.$0$ D.$\frac{271}{107}$ E.$\frac{107}{271}$
F.$-\frac{271}{107}$ G.$-1$
H.$(\frac{271}{107})^{3}$
I.$(\frac{107}{271})^{3}$
\testStop
\kluczStart
A
\kluczStop



\zadStart{Zadanie z Wikieł Z 1.1 d) moja wersja nr 195}

Obliczyć wartość wyrażenia $(\frac{271}{109})^{3} \cdot (\frac{109}{271})^{3} \cdot \pi^{0}$.
\zadStop
\rozwStart{Patryk Wirkus}{Martyna Czarnobaj}
$$(\frac{271}{109})^{3} \cdot (\frac{109}{271})^{3} \cdot \pi^{0} = (\frac{271}{109} \cdot \frac{109}{271})^{3} \cdot 1 = 1^{3} \cdot 1 = 1$$
\rozwStop
\odpStart
$1$
\odpStop
\testStart
A.$1$ B.$\pi$ C.$0$ D.$\frac{271}{109}$ E.$\frac{109}{271}$
F.$-\frac{271}{109}$ G.$-1$
H.$(\frac{271}{109})^{3}$
I.$(\frac{109}{271})^{3}$
\testStop
\kluczStart
A
\kluczStop



\zadStart{Zadanie z Wikieł Z 1.1 d) moja wersja nr 196}

Obliczyć wartość wyrażenia $(\frac{271}{113})^{3} \cdot (\frac{113}{271})^{3} \cdot \pi^{0}$.
\zadStop
\rozwStart{Patryk Wirkus}{Martyna Czarnobaj}
$$(\frac{271}{113})^{3} \cdot (\frac{113}{271})^{3} \cdot \pi^{0} = (\frac{271}{113} \cdot \frac{113}{271})^{3} \cdot 1 = 1^{3} \cdot 1 = 1$$
\rozwStop
\odpStart
$1$
\odpStop
\testStart
A.$1$ B.$\pi$ C.$0$ D.$\frac{271}{113}$ E.$\frac{113}{271}$
F.$-\frac{271}{113}$ G.$-1$
H.$(\frac{271}{113})^{3}$
I.$(\frac{113}{271})^{3}$
\testStop
\kluczStart
A
\kluczStop



\zadStart{Zadanie z Wikieł Z 1.1 d) moja wersja nr 197}

Obliczyć wartość wyrażenia $(\frac{271}{127})^{3} \cdot (\frac{127}{271})^{3} \cdot \pi^{0}$.
\zadStop
\rozwStart{Patryk Wirkus}{Martyna Czarnobaj}
$$(\frac{271}{127})^{3} \cdot (\frac{127}{271})^{3} \cdot \pi^{0} = (\frac{271}{127} \cdot \frac{127}{271})^{3} \cdot 1 = 1^{3} \cdot 1 = 1$$
\rozwStop
\odpStart
$1$
\odpStop
\testStart
A.$1$ B.$\pi$ C.$0$ D.$\frac{271}{127}$ E.$\frac{127}{271}$
F.$-\frac{271}{127}$ G.$-1$
H.$(\frac{271}{127})^{3}$
I.$(\frac{127}{271})^{3}$
\testStop
\kluczStart
A
\kluczStop



\zadStart{Zadanie z Wikieł Z 1.1 d) moja wersja nr 198}

Obliczyć wartość wyrażenia $(\frac{271}{131})^{3} \cdot (\frac{131}{271})^{3} \cdot \pi^{0}$.
\zadStop
\rozwStart{Patryk Wirkus}{Martyna Czarnobaj}
$$(\frac{271}{131})^{3} \cdot (\frac{131}{271})^{3} \cdot \pi^{0} = (\frac{271}{131} \cdot \frac{131}{271})^{3} \cdot 1 = 1^{3} \cdot 1 = 1$$
\rozwStop
\odpStart
$1$
\odpStop
\testStart
A.$1$ B.$\pi$ C.$0$ D.$\frac{271}{131}$ E.$\frac{131}{271}$
F.$-\frac{271}{131}$ G.$-1$
H.$(\frac{271}{131})^{3}$
I.$(\frac{131}{271})^{3}$
\testStop
\kluczStart
A
\kluczStop



\zadStart{Zadanie z Wikieł Z 1.1 d) moja wersja nr 199}

Obliczyć wartość wyrażenia $(\frac{271}{137})^{3} \cdot (\frac{137}{271})^{3} \cdot \pi^{0}$.
\zadStop
\rozwStart{Patryk Wirkus}{Martyna Czarnobaj}
$$(\frac{271}{137})^{3} \cdot (\frac{137}{271})^{3} \cdot \pi^{0} = (\frac{271}{137} \cdot \frac{137}{271})^{3} \cdot 1 = 1^{3} \cdot 1 = 1$$
\rozwStop
\odpStart
$1$
\odpStop
\testStart
A.$1$ B.$\pi$ C.$0$ D.$\frac{271}{137}$ E.$\frac{137}{271}$
F.$-\frac{271}{137}$ G.$-1$
H.$(\frac{271}{137})^{3}$
I.$(\frac{137}{271})^{3}$
\testStop
\kluczStart
A
\kluczStop



\zadStart{Zadanie z Wikieł Z 1.1 d) moja wersja nr 200}

Obliczyć wartość wyrażenia $(\frac{271}{139})^{3} \cdot (\frac{139}{271})^{3} \cdot \pi^{0}$.
\zadStop
\rozwStart{Patryk Wirkus}{Martyna Czarnobaj}
$$(\frac{271}{139})^{3} \cdot (\frac{139}{271})^{3} \cdot \pi^{0} = (\frac{271}{139} \cdot \frac{139}{271})^{3} \cdot 1 = 1^{3} \cdot 1 = 1$$
\rozwStop
\odpStart
$1$
\odpStop
\testStart
A.$1$ B.$\pi$ C.$0$ D.$\frac{271}{139}$ E.$\frac{139}{271}$
F.$-\frac{271}{139}$ G.$-1$
H.$(\frac{271}{139})^{3}$
I.$(\frac{139}{271})^{3}$
\testStop
\kluczStart
A
\kluczStop



\zadStart{Zadanie z Wikieł Z 1.1 d) moja wersja nr 201}

Obliczyć wartość wyrażenia $(\frac{277}{103})^{3} \cdot (\frac{103}{277})^{3} \cdot \pi^{0}$.
\zadStop
\rozwStart{Patryk Wirkus}{Martyna Czarnobaj}
$$(\frac{277}{103})^{3} \cdot (\frac{103}{277})^{3} \cdot \pi^{0} = (\frac{277}{103} \cdot \frac{103}{277})^{3} \cdot 1 = 1^{3} \cdot 1 = 1$$
\rozwStop
\odpStart
$1$
\odpStop
\testStart
A.$1$ B.$\pi$ C.$0$ D.$\frac{277}{103}$ E.$\frac{103}{277}$
F.$-\frac{277}{103}$ G.$-1$
H.$(\frac{277}{103})^{3}$
I.$(\frac{103}{277})^{3}$
\testStop
\kluczStart
A
\kluczStop



\zadStart{Zadanie z Wikieł Z 1.1 d) moja wersja nr 202}

Obliczyć wartość wyrażenia $(\frac{277}{107})^{3} \cdot (\frac{107}{277})^{3} \cdot \pi^{0}$.
\zadStop
\rozwStart{Patryk Wirkus}{Martyna Czarnobaj}
$$(\frac{277}{107})^{3} \cdot (\frac{107}{277})^{3} \cdot \pi^{0} = (\frac{277}{107} \cdot \frac{107}{277})^{3} \cdot 1 = 1^{3} \cdot 1 = 1$$
\rozwStop
\odpStart
$1$
\odpStop
\testStart
A.$1$ B.$\pi$ C.$0$ D.$\frac{277}{107}$ E.$\frac{107}{277}$
F.$-\frac{277}{107}$ G.$-1$
H.$(\frac{277}{107})^{3}$
I.$(\frac{107}{277})^{3}$
\testStop
\kluczStart
A
\kluczStop



\zadStart{Zadanie z Wikieł Z 1.1 d) moja wersja nr 203}

Obliczyć wartość wyrażenia $(\frac{277}{109})^{3} \cdot (\frac{109}{277})^{3} \cdot \pi^{0}$.
\zadStop
\rozwStart{Patryk Wirkus}{Martyna Czarnobaj}
$$(\frac{277}{109})^{3} \cdot (\frac{109}{277})^{3} \cdot \pi^{0} = (\frac{277}{109} \cdot \frac{109}{277})^{3} \cdot 1 = 1^{3} \cdot 1 = 1$$
\rozwStop
\odpStart
$1$
\odpStop
\testStart
A.$1$ B.$\pi$ C.$0$ D.$\frac{277}{109}$ E.$\frac{109}{277}$
F.$-\frac{277}{109}$ G.$-1$
H.$(\frac{277}{109})^{3}$
I.$(\frac{109}{277})^{3}$
\testStop
\kluczStart
A
\kluczStop



\zadStart{Zadanie z Wikieł Z 1.1 d) moja wersja nr 204}

Obliczyć wartość wyrażenia $(\frac{277}{113})^{3} \cdot (\frac{113}{277})^{3} \cdot \pi^{0}$.
\zadStop
\rozwStart{Patryk Wirkus}{Martyna Czarnobaj}
$$(\frac{277}{113})^{3} \cdot (\frac{113}{277})^{3} \cdot \pi^{0} = (\frac{277}{113} \cdot \frac{113}{277})^{3} \cdot 1 = 1^{3} \cdot 1 = 1$$
\rozwStop
\odpStart
$1$
\odpStop
\testStart
A.$1$ B.$\pi$ C.$0$ D.$\frac{277}{113}$ E.$\frac{113}{277}$
F.$-\frac{277}{113}$ G.$-1$
H.$(\frac{277}{113})^{3}$
I.$(\frac{113}{277})^{3}$
\testStop
\kluczStart
A
\kluczStop



\zadStart{Zadanie z Wikieł Z 1.1 d) moja wersja nr 205}

Obliczyć wartość wyrażenia $(\frac{277}{127})^{3} \cdot (\frac{127}{277})^{3} \cdot \pi^{0}$.
\zadStop
\rozwStart{Patryk Wirkus}{Martyna Czarnobaj}
$$(\frac{277}{127})^{3} \cdot (\frac{127}{277})^{3} \cdot \pi^{0} = (\frac{277}{127} \cdot \frac{127}{277})^{3} \cdot 1 = 1^{3} \cdot 1 = 1$$
\rozwStop
\odpStart
$1$
\odpStop
\testStart
A.$1$ B.$\pi$ C.$0$ D.$\frac{277}{127}$ E.$\frac{127}{277}$
F.$-\frac{277}{127}$ G.$-1$
H.$(\frac{277}{127})^{3}$
I.$(\frac{127}{277})^{3}$
\testStop
\kluczStart
A
\kluczStop



\zadStart{Zadanie z Wikieł Z 1.1 d) moja wersja nr 206}

Obliczyć wartość wyrażenia $(\frac{277}{131})^{3} \cdot (\frac{131}{277})^{3} \cdot \pi^{0}$.
\zadStop
\rozwStart{Patryk Wirkus}{Martyna Czarnobaj}
$$(\frac{277}{131})^{3} \cdot (\frac{131}{277})^{3} \cdot \pi^{0} = (\frac{277}{131} \cdot \frac{131}{277})^{3} \cdot 1 = 1^{3} \cdot 1 = 1$$
\rozwStop
\odpStart
$1$
\odpStop
\testStart
A.$1$ B.$\pi$ C.$0$ D.$\frac{277}{131}$ E.$\frac{131}{277}$
F.$-\frac{277}{131}$ G.$-1$
H.$(\frac{277}{131})^{3}$
I.$(\frac{131}{277})^{3}$
\testStop
\kluczStart
A
\kluczStop



\zadStart{Zadanie z Wikieł Z 1.1 d) moja wersja nr 207}

Obliczyć wartość wyrażenia $(\frac{277}{137})^{3} \cdot (\frac{137}{277})^{3} \cdot \pi^{0}$.
\zadStop
\rozwStart{Patryk Wirkus}{Martyna Czarnobaj}
$$(\frac{277}{137})^{3} \cdot (\frac{137}{277})^{3} \cdot \pi^{0} = (\frac{277}{137} \cdot \frac{137}{277})^{3} \cdot 1 = 1^{3} \cdot 1 = 1$$
\rozwStop
\odpStart
$1$
\odpStop
\testStart
A.$1$ B.$\pi$ C.$0$ D.$\frac{277}{137}$ E.$\frac{137}{277}$
F.$-\frac{277}{137}$ G.$-1$
H.$(\frac{277}{137})^{3}$
I.$(\frac{137}{277})^{3}$
\testStop
\kluczStart
A
\kluczStop



\zadStart{Zadanie z Wikieł Z 1.1 d) moja wersja nr 208}

Obliczyć wartość wyrażenia $(\frac{277}{139})^{3} \cdot (\frac{139}{277})^{3} \cdot \pi^{0}$.
\zadStop
\rozwStart{Patryk Wirkus}{Martyna Czarnobaj}
$$(\frac{277}{139})^{3} \cdot (\frac{139}{277})^{3} \cdot \pi^{0} = (\frac{277}{139} \cdot \frac{139}{277})^{3} \cdot 1 = 1^{3} \cdot 1 = 1$$
\rozwStop
\odpStart
$1$
\odpStop
\testStart
A.$1$ B.$\pi$ C.$0$ D.$\frac{277}{139}$ E.$\frac{139}{277}$
F.$-\frac{277}{139}$ G.$-1$
H.$(\frac{277}{139})^{3}$
I.$(\frac{139}{277})^{3}$
\testStop
\kluczStart
A
\kluczStop



\zadStart{Zadanie z Wikieł Z 1.1 d) moja wersja nr 209}

Obliczyć wartość wyrażenia $(\frac{149}{103})^{4} \cdot (\frac{103}{149})^{4} \cdot \pi^{0}$.
\zadStop
\rozwStart{Patryk Wirkus}{Martyna Czarnobaj}
$$(\frac{149}{103})^{4} \cdot (\frac{103}{149})^{4} \cdot \pi^{0} = (\frac{149}{103} \cdot \frac{103}{149})^{4} \cdot 1 = 1^{4} \cdot 1 = 1$$
\rozwStop
\odpStart
$1$
\odpStop
\testStart
A.$1$ B.$\pi$ C.$0$ D.$\frac{149}{103}$ E.$\frac{103}{149}$
F.$-\frac{149}{103}$ G.$-1$
H.$(\frac{149}{103})^{4}$
I.$(\frac{103}{149})^{4}$
\testStop
\kluczStart
A
\kluczStop



\zadStart{Zadanie z Wikieł Z 1.1 d) moja wersja nr 210}

Obliczyć wartość wyrażenia $(\frac{149}{107})^{4} \cdot (\frac{107}{149})^{4} \cdot \pi^{0}$.
\zadStop
\rozwStart{Patryk Wirkus}{Martyna Czarnobaj}
$$(\frac{149}{107})^{4} \cdot (\frac{107}{149})^{4} \cdot \pi^{0} = (\frac{149}{107} \cdot \frac{107}{149})^{4} \cdot 1 = 1^{4} \cdot 1 = 1$$
\rozwStop
\odpStart
$1$
\odpStop
\testStart
A.$1$ B.$\pi$ C.$0$ D.$\frac{149}{107}$ E.$\frac{107}{149}$
F.$-\frac{149}{107}$ G.$-1$
H.$(\frac{149}{107})^{4}$
I.$(\frac{107}{149})^{4}$
\testStop
\kluczStart
A
\kluczStop



\zadStart{Zadanie z Wikieł Z 1.1 d) moja wersja nr 211}

Obliczyć wartość wyrażenia $(\frac{149}{109})^{4} \cdot (\frac{109}{149})^{4} \cdot \pi^{0}$.
\zadStop
\rozwStart{Patryk Wirkus}{Martyna Czarnobaj}
$$(\frac{149}{109})^{4} \cdot (\frac{109}{149})^{4} \cdot \pi^{0} = (\frac{149}{109} \cdot \frac{109}{149})^{4} \cdot 1 = 1^{4} \cdot 1 = 1$$
\rozwStop
\odpStart
$1$
\odpStop
\testStart
A.$1$ B.$\pi$ C.$0$ D.$\frac{149}{109}$ E.$\frac{109}{149}$
F.$-\frac{149}{109}$ G.$-1$
H.$(\frac{149}{109})^{4}$
I.$(\frac{109}{149})^{4}$
\testStop
\kluczStart
A
\kluczStop



\zadStart{Zadanie z Wikieł Z 1.1 d) moja wersja nr 212}

Obliczyć wartość wyrażenia $(\frac{149}{113})^{4} \cdot (\frac{113}{149})^{4} \cdot \pi^{0}$.
\zadStop
\rozwStart{Patryk Wirkus}{Martyna Czarnobaj}
$$(\frac{149}{113})^{4} \cdot (\frac{113}{149})^{4} \cdot \pi^{0} = (\frac{149}{113} \cdot \frac{113}{149})^{4} \cdot 1 = 1^{4} \cdot 1 = 1$$
\rozwStop
\odpStart
$1$
\odpStop
\testStart
A.$1$ B.$\pi$ C.$0$ D.$\frac{149}{113}$ E.$\frac{113}{149}$
F.$-\frac{149}{113}$ G.$-1$
H.$(\frac{149}{113})^{4}$
I.$(\frac{113}{149})^{4}$
\testStop
\kluczStart
A
\kluczStop



\zadStart{Zadanie z Wikieł Z 1.1 d) moja wersja nr 213}

Obliczyć wartość wyrażenia $(\frac{149}{127})^{4} \cdot (\frac{127}{149})^{4} \cdot \pi^{0}$.
\zadStop
\rozwStart{Patryk Wirkus}{Martyna Czarnobaj}
$$(\frac{149}{127})^{4} \cdot (\frac{127}{149})^{4} \cdot \pi^{0} = (\frac{149}{127} \cdot \frac{127}{149})^{4} \cdot 1 = 1^{4} \cdot 1 = 1$$
\rozwStop
\odpStart
$1$
\odpStop
\testStart
A.$1$ B.$\pi$ C.$0$ D.$\frac{149}{127}$ E.$\frac{127}{149}$
F.$-\frac{149}{127}$ G.$-1$
H.$(\frac{149}{127})^{4}$
I.$(\frac{127}{149})^{4}$
\testStop
\kluczStart
A
\kluczStop



\zadStart{Zadanie z Wikieł Z 1.1 d) moja wersja nr 214}

Obliczyć wartość wyrażenia $(\frac{149}{131})^{4} \cdot (\frac{131}{149})^{4} \cdot \pi^{0}$.
\zadStop
\rozwStart{Patryk Wirkus}{Martyna Czarnobaj}
$$(\frac{149}{131})^{4} \cdot (\frac{131}{149})^{4} \cdot \pi^{0} = (\frac{149}{131} \cdot \frac{131}{149})^{4} \cdot 1 = 1^{4} \cdot 1 = 1$$
\rozwStop
\odpStart
$1$
\odpStop
\testStart
A.$1$ B.$\pi$ C.$0$ D.$\frac{149}{131}$ E.$\frac{131}{149}$
F.$-\frac{149}{131}$ G.$-1$
H.$(\frac{149}{131})^{4}$
I.$(\frac{131}{149})^{4}$
\testStop
\kluczStart
A
\kluczStop



\zadStart{Zadanie z Wikieł Z 1.1 d) moja wersja nr 215}

Obliczyć wartość wyrażenia $(\frac{149}{137})^{4} \cdot (\frac{137}{149})^{4} \cdot \pi^{0}$.
\zadStop
\rozwStart{Patryk Wirkus}{Martyna Czarnobaj}
$$(\frac{149}{137})^{4} \cdot (\frac{137}{149})^{4} \cdot \pi^{0} = (\frac{149}{137} \cdot \frac{137}{149})^{4} \cdot 1 = 1^{4} \cdot 1 = 1$$
\rozwStop
\odpStart
$1$
\odpStop
\testStart
A.$1$ B.$\pi$ C.$0$ D.$\frac{149}{137}$ E.$\frac{137}{149}$
F.$-\frac{149}{137}$ G.$-1$
H.$(\frac{149}{137})^{4}$
I.$(\frac{137}{149})^{4}$
\testStop
\kluczStart
A
\kluczStop



\zadStart{Zadanie z Wikieł Z 1.1 d) moja wersja nr 216}

Obliczyć wartość wyrażenia $(\frac{149}{139})^{4} \cdot (\frac{139}{149})^{4} \cdot \pi^{0}$.
\zadStop
\rozwStart{Patryk Wirkus}{Martyna Czarnobaj}
$$(\frac{149}{139})^{4} \cdot (\frac{139}{149})^{4} \cdot \pi^{0} = (\frac{149}{139} \cdot \frac{139}{149})^{4} \cdot 1 = 1^{4} \cdot 1 = 1$$
\rozwStop
\odpStart
$1$
\odpStop
\testStart
A.$1$ B.$\pi$ C.$0$ D.$\frac{149}{139}$ E.$\frac{139}{149}$
F.$-\frac{149}{139}$ G.$-1$
H.$(\frac{149}{139})^{4}$
I.$(\frac{139}{149})^{4}$
\testStop
\kluczStart
A
\kluczStop



\zadStart{Zadanie z Wikieł Z 1.1 d) moja wersja nr 217}

Obliczyć wartość wyrażenia $(\frac{151}{103})^{4} \cdot (\frac{103}{151})^{4} \cdot \pi^{0}$.
\zadStop
\rozwStart{Patryk Wirkus}{Martyna Czarnobaj}
$$(\frac{151}{103})^{4} \cdot (\frac{103}{151})^{4} \cdot \pi^{0} = (\frac{151}{103} \cdot \frac{103}{151})^{4} \cdot 1 = 1^{4} \cdot 1 = 1$$
\rozwStop
\odpStart
$1$
\odpStop
\testStart
A.$1$ B.$\pi$ C.$0$ D.$\frac{151}{103}$ E.$\frac{103}{151}$
F.$-\frac{151}{103}$ G.$-1$
H.$(\frac{151}{103})^{4}$
I.$(\frac{103}{151})^{4}$
\testStop
\kluczStart
A
\kluczStop



\zadStart{Zadanie z Wikieł Z 1.1 d) moja wersja nr 218}

Obliczyć wartość wyrażenia $(\frac{151}{107})^{4} \cdot (\frac{107}{151})^{4} \cdot \pi^{0}$.
\zadStop
\rozwStart{Patryk Wirkus}{Martyna Czarnobaj}
$$(\frac{151}{107})^{4} \cdot (\frac{107}{151})^{4} \cdot \pi^{0} = (\frac{151}{107} \cdot \frac{107}{151})^{4} \cdot 1 = 1^{4} \cdot 1 = 1$$
\rozwStop
\odpStart
$1$
\odpStop
\testStart
A.$1$ B.$\pi$ C.$0$ D.$\frac{151}{107}$ E.$\frac{107}{151}$
F.$-\frac{151}{107}$ G.$-1$
H.$(\frac{151}{107})^{4}$
I.$(\frac{107}{151})^{4}$
\testStop
\kluczStart
A
\kluczStop



\zadStart{Zadanie z Wikieł Z 1.1 d) moja wersja nr 219}

Obliczyć wartość wyrażenia $(\frac{151}{109})^{4} \cdot (\frac{109}{151})^{4} \cdot \pi^{0}$.
\zadStop
\rozwStart{Patryk Wirkus}{Martyna Czarnobaj}
$$(\frac{151}{109})^{4} \cdot (\frac{109}{151})^{4} \cdot \pi^{0} = (\frac{151}{109} \cdot \frac{109}{151})^{4} \cdot 1 = 1^{4} \cdot 1 = 1$$
\rozwStop
\odpStart
$1$
\odpStop
\testStart
A.$1$ B.$\pi$ C.$0$ D.$\frac{151}{109}$ E.$\frac{109}{151}$
F.$-\frac{151}{109}$ G.$-1$
H.$(\frac{151}{109})^{4}$
I.$(\frac{109}{151})^{4}$
\testStop
\kluczStart
A
\kluczStop



\zadStart{Zadanie z Wikieł Z 1.1 d) moja wersja nr 220}

Obliczyć wartość wyrażenia $(\frac{151}{113})^{4} \cdot (\frac{113}{151})^{4} \cdot \pi^{0}$.
\zadStop
\rozwStart{Patryk Wirkus}{Martyna Czarnobaj}
$$(\frac{151}{113})^{4} \cdot (\frac{113}{151})^{4} \cdot \pi^{0} = (\frac{151}{113} \cdot \frac{113}{151})^{4} \cdot 1 = 1^{4} \cdot 1 = 1$$
\rozwStop
\odpStart
$1$
\odpStop
\testStart
A.$1$ B.$\pi$ C.$0$ D.$\frac{151}{113}$ E.$\frac{113}{151}$
F.$-\frac{151}{113}$ G.$-1$
H.$(\frac{151}{113})^{4}$
I.$(\frac{113}{151})^{4}$
\testStop
\kluczStart
A
\kluczStop



\zadStart{Zadanie z Wikieł Z 1.1 d) moja wersja nr 221}

Obliczyć wartość wyrażenia $(\frac{151}{127})^{4} \cdot (\frac{127}{151})^{4} \cdot \pi^{0}$.
\zadStop
\rozwStart{Patryk Wirkus}{Martyna Czarnobaj}
$$(\frac{151}{127})^{4} \cdot (\frac{127}{151})^{4} \cdot \pi^{0} = (\frac{151}{127} \cdot \frac{127}{151})^{4} \cdot 1 = 1^{4} \cdot 1 = 1$$
\rozwStop
\odpStart
$1$
\odpStop
\testStart
A.$1$ B.$\pi$ C.$0$ D.$\frac{151}{127}$ E.$\frac{127}{151}$
F.$-\frac{151}{127}$ G.$-1$
H.$(\frac{151}{127})^{4}$
I.$(\frac{127}{151})^{4}$
\testStop
\kluczStart
A
\kluczStop



\zadStart{Zadanie z Wikieł Z 1.1 d) moja wersja nr 222}

Obliczyć wartość wyrażenia $(\frac{151}{131})^{4} \cdot (\frac{131}{151})^{4} \cdot \pi^{0}$.
\zadStop
\rozwStart{Patryk Wirkus}{Martyna Czarnobaj}
$$(\frac{151}{131})^{4} \cdot (\frac{131}{151})^{4} \cdot \pi^{0} = (\frac{151}{131} \cdot \frac{131}{151})^{4} \cdot 1 = 1^{4} \cdot 1 = 1$$
\rozwStop
\odpStart
$1$
\odpStop
\testStart
A.$1$ B.$\pi$ C.$0$ D.$\frac{151}{131}$ E.$\frac{131}{151}$
F.$-\frac{151}{131}$ G.$-1$
H.$(\frac{151}{131})^{4}$
I.$(\frac{131}{151})^{4}$
\testStop
\kluczStart
A
\kluczStop



\zadStart{Zadanie z Wikieł Z 1.1 d) moja wersja nr 223}

Obliczyć wartość wyrażenia $(\frac{151}{137})^{4} \cdot (\frac{137}{151})^{4} \cdot \pi^{0}$.
\zadStop
\rozwStart{Patryk Wirkus}{Martyna Czarnobaj}
$$(\frac{151}{137})^{4} \cdot (\frac{137}{151})^{4} \cdot \pi^{0} = (\frac{151}{137} \cdot \frac{137}{151})^{4} \cdot 1 = 1^{4} \cdot 1 = 1$$
\rozwStop
\odpStart
$1$
\odpStop
\testStart
A.$1$ B.$\pi$ C.$0$ D.$\frac{151}{137}$ E.$\frac{137}{151}$
F.$-\frac{151}{137}$ G.$-1$
H.$(\frac{151}{137})^{4}$
I.$(\frac{137}{151})^{4}$
\testStop
\kluczStart
A
\kluczStop



\zadStart{Zadanie z Wikieł Z 1.1 d) moja wersja nr 224}

Obliczyć wartość wyrażenia $(\frac{151}{139})^{4} \cdot (\frac{139}{151})^{4} \cdot \pi^{0}$.
\zadStop
\rozwStart{Patryk Wirkus}{Martyna Czarnobaj}
$$(\frac{151}{139})^{4} \cdot (\frac{139}{151})^{4} \cdot \pi^{0} = (\frac{151}{139} \cdot \frac{139}{151})^{4} \cdot 1 = 1^{4} \cdot 1 = 1$$
\rozwStop
\odpStart
$1$
\odpStop
\testStart
A.$1$ B.$\pi$ C.$0$ D.$\frac{151}{139}$ E.$\frac{139}{151}$
F.$-\frac{151}{139}$ G.$-1$
H.$(\frac{151}{139})^{4}$
I.$(\frac{139}{151})^{4}$
\testStop
\kluczStart
A
\kluczStop



\zadStart{Zadanie z Wikieł Z 1.1 d) moja wersja nr 225}

Obliczyć wartość wyrażenia $(\frac{157}{103})^{4} \cdot (\frac{103}{157})^{4} \cdot \pi^{0}$.
\zadStop
\rozwStart{Patryk Wirkus}{Martyna Czarnobaj}
$$(\frac{157}{103})^{4} \cdot (\frac{103}{157})^{4} \cdot \pi^{0} = (\frac{157}{103} \cdot \frac{103}{157})^{4} \cdot 1 = 1^{4} \cdot 1 = 1$$
\rozwStop
\odpStart
$1$
\odpStop
\testStart
A.$1$ B.$\pi$ C.$0$ D.$\frac{157}{103}$ E.$\frac{103}{157}$
F.$-\frac{157}{103}$ G.$-1$
H.$(\frac{157}{103})^{4}$
I.$(\frac{103}{157})^{4}$
\testStop
\kluczStart
A
\kluczStop



\zadStart{Zadanie z Wikieł Z 1.1 d) moja wersja nr 226}

Obliczyć wartość wyrażenia $(\frac{157}{107})^{4} \cdot (\frac{107}{157})^{4} \cdot \pi^{0}$.
\zadStop
\rozwStart{Patryk Wirkus}{Martyna Czarnobaj}
$$(\frac{157}{107})^{4} \cdot (\frac{107}{157})^{4} \cdot \pi^{0} = (\frac{157}{107} \cdot \frac{107}{157})^{4} \cdot 1 = 1^{4} \cdot 1 = 1$$
\rozwStop
\odpStart
$1$
\odpStop
\testStart
A.$1$ B.$\pi$ C.$0$ D.$\frac{157}{107}$ E.$\frac{107}{157}$
F.$-\frac{157}{107}$ G.$-1$
H.$(\frac{157}{107})^{4}$
I.$(\frac{107}{157})^{4}$
\testStop
\kluczStart
A
\kluczStop



\zadStart{Zadanie z Wikieł Z 1.1 d) moja wersja nr 227}

Obliczyć wartość wyrażenia $(\frac{157}{109})^{4} \cdot (\frac{109}{157})^{4} \cdot \pi^{0}$.
\zadStop
\rozwStart{Patryk Wirkus}{Martyna Czarnobaj}
$$(\frac{157}{109})^{4} \cdot (\frac{109}{157})^{4} \cdot \pi^{0} = (\frac{157}{109} \cdot \frac{109}{157})^{4} \cdot 1 = 1^{4} \cdot 1 = 1$$
\rozwStop
\odpStart
$1$
\odpStop
\testStart
A.$1$ B.$\pi$ C.$0$ D.$\frac{157}{109}$ E.$\frac{109}{157}$
F.$-\frac{157}{109}$ G.$-1$
H.$(\frac{157}{109})^{4}$
I.$(\frac{109}{157})^{4}$
\testStop
\kluczStart
A
\kluczStop



\zadStart{Zadanie z Wikieł Z 1.1 d) moja wersja nr 228}

Obliczyć wartość wyrażenia $(\frac{157}{113})^{4} \cdot (\frac{113}{157})^{4} \cdot \pi^{0}$.
\zadStop
\rozwStart{Patryk Wirkus}{Martyna Czarnobaj}
$$(\frac{157}{113})^{4} \cdot (\frac{113}{157})^{4} \cdot \pi^{0} = (\frac{157}{113} \cdot \frac{113}{157})^{4} \cdot 1 = 1^{4} \cdot 1 = 1$$
\rozwStop
\odpStart
$1$
\odpStop
\testStart
A.$1$ B.$\pi$ C.$0$ D.$\frac{157}{113}$ E.$\frac{113}{157}$
F.$-\frac{157}{113}$ G.$-1$
H.$(\frac{157}{113})^{4}$
I.$(\frac{113}{157})^{4}$
\testStop
\kluczStart
A
\kluczStop



\zadStart{Zadanie z Wikieł Z 1.1 d) moja wersja nr 229}

Obliczyć wartość wyrażenia $(\frac{157}{127})^{4} \cdot (\frac{127}{157})^{4} \cdot \pi^{0}$.
\zadStop
\rozwStart{Patryk Wirkus}{Martyna Czarnobaj}
$$(\frac{157}{127})^{4} \cdot (\frac{127}{157})^{4} \cdot \pi^{0} = (\frac{157}{127} \cdot \frac{127}{157})^{4} \cdot 1 = 1^{4} \cdot 1 = 1$$
\rozwStop
\odpStart
$1$
\odpStop
\testStart
A.$1$ B.$\pi$ C.$0$ D.$\frac{157}{127}$ E.$\frac{127}{157}$
F.$-\frac{157}{127}$ G.$-1$
H.$(\frac{157}{127})^{4}$
I.$(\frac{127}{157})^{4}$
\testStop
\kluczStart
A
\kluczStop



\zadStart{Zadanie z Wikieł Z 1.1 d) moja wersja nr 230}

Obliczyć wartość wyrażenia $(\frac{157}{131})^{4} \cdot (\frac{131}{157})^{4} \cdot \pi^{0}$.
\zadStop
\rozwStart{Patryk Wirkus}{Martyna Czarnobaj}
$$(\frac{157}{131})^{4} \cdot (\frac{131}{157})^{4} \cdot \pi^{0} = (\frac{157}{131} \cdot \frac{131}{157})^{4} \cdot 1 = 1^{4} \cdot 1 = 1$$
\rozwStop
\odpStart
$1$
\odpStop
\testStart
A.$1$ B.$\pi$ C.$0$ D.$\frac{157}{131}$ E.$\frac{131}{157}$
F.$-\frac{157}{131}$ G.$-1$
H.$(\frac{157}{131})^{4}$
I.$(\frac{131}{157})^{4}$
\testStop
\kluczStart
A
\kluczStop



\zadStart{Zadanie z Wikieł Z 1.1 d) moja wersja nr 231}

Obliczyć wartość wyrażenia $(\frac{157}{137})^{4} \cdot (\frac{137}{157})^{4} \cdot \pi^{0}$.
\zadStop
\rozwStart{Patryk Wirkus}{Martyna Czarnobaj}
$$(\frac{157}{137})^{4} \cdot (\frac{137}{157})^{4} \cdot \pi^{0} = (\frac{157}{137} \cdot \frac{137}{157})^{4} \cdot 1 = 1^{4} \cdot 1 = 1$$
\rozwStop
\odpStart
$1$
\odpStop
\testStart
A.$1$ B.$\pi$ C.$0$ D.$\frac{157}{137}$ E.$\frac{137}{157}$
F.$-\frac{157}{137}$ G.$-1$
H.$(\frac{157}{137})^{4}$
I.$(\frac{137}{157})^{4}$
\testStop
\kluczStart
A
\kluczStop



\zadStart{Zadanie z Wikieł Z 1.1 d) moja wersja nr 232}

Obliczyć wartość wyrażenia $(\frac{157}{139})^{4} \cdot (\frac{139}{157})^{4} \cdot \pi^{0}$.
\zadStop
\rozwStart{Patryk Wirkus}{Martyna Czarnobaj}
$$(\frac{157}{139})^{4} \cdot (\frac{139}{157})^{4} \cdot \pi^{0} = (\frac{157}{139} \cdot \frac{139}{157})^{4} \cdot 1 = 1^{4} \cdot 1 = 1$$
\rozwStop
\odpStart
$1$
\odpStop
\testStart
A.$1$ B.$\pi$ C.$0$ D.$\frac{157}{139}$ E.$\frac{139}{157}$
F.$-\frac{157}{139}$ G.$-1$
H.$(\frac{157}{139})^{4}$
I.$(\frac{139}{157})^{4}$
\testStop
\kluczStart
A
\kluczStop



\zadStart{Zadanie z Wikieł Z 1.1 d) moja wersja nr 233}

Obliczyć wartość wyrażenia $(\frac{163}{103})^{4} \cdot (\frac{103}{163})^{4} \cdot \pi^{0}$.
\zadStop
\rozwStart{Patryk Wirkus}{Martyna Czarnobaj}
$$(\frac{163}{103})^{4} \cdot (\frac{103}{163})^{4} \cdot \pi^{0} = (\frac{163}{103} \cdot \frac{103}{163})^{4} \cdot 1 = 1^{4} \cdot 1 = 1$$
\rozwStop
\odpStart
$1$
\odpStop
\testStart
A.$1$ B.$\pi$ C.$0$ D.$\frac{163}{103}$ E.$\frac{103}{163}$
F.$-\frac{163}{103}$ G.$-1$
H.$(\frac{163}{103})^{4}$
I.$(\frac{103}{163})^{4}$
\testStop
\kluczStart
A
\kluczStop



\zadStart{Zadanie z Wikieł Z 1.1 d) moja wersja nr 234}

Obliczyć wartość wyrażenia $(\frac{163}{107})^{4} \cdot (\frac{107}{163})^{4} \cdot \pi^{0}$.
\zadStop
\rozwStart{Patryk Wirkus}{Martyna Czarnobaj}
$$(\frac{163}{107})^{4} \cdot (\frac{107}{163})^{4} \cdot \pi^{0} = (\frac{163}{107} \cdot \frac{107}{163})^{4} \cdot 1 = 1^{4} \cdot 1 = 1$$
\rozwStop
\odpStart
$1$
\odpStop
\testStart
A.$1$ B.$\pi$ C.$0$ D.$\frac{163}{107}$ E.$\frac{107}{163}$
F.$-\frac{163}{107}$ G.$-1$
H.$(\frac{163}{107})^{4}$
I.$(\frac{107}{163})^{4}$
\testStop
\kluczStart
A
\kluczStop



\zadStart{Zadanie z Wikieł Z 1.1 d) moja wersja nr 235}

Obliczyć wartość wyrażenia $(\frac{163}{109})^{4} \cdot (\frac{109}{163})^{4} \cdot \pi^{0}$.
\zadStop
\rozwStart{Patryk Wirkus}{Martyna Czarnobaj}
$$(\frac{163}{109})^{4} \cdot (\frac{109}{163})^{4} \cdot \pi^{0} = (\frac{163}{109} \cdot \frac{109}{163})^{4} \cdot 1 = 1^{4} \cdot 1 = 1$$
\rozwStop
\odpStart
$1$
\odpStop
\testStart
A.$1$ B.$\pi$ C.$0$ D.$\frac{163}{109}$ E.$\frac{109}{163}$
F.$-\frac{163}{109}$ G.$-1$
H.$(\frac{163}{109})^{4}$
I.$(\frac{109}{163})^{4}$
\testStop
\kluczStart
A
\kluczStop



\zadStart{Zadanie z Wikieł Z 1.1 d) moja wersja nr 236}

Obliczyć wartość wyrażenia $(\frac{163}{113})^{4} \cdot (\frac{113}{163})^{4} \cdot \pi^{0}$.
\zadStop
\rozwStart{Patryk Wirkus}{Martyna Czarnobaj}
$$(\frac{163}{113})^{4} \cdot (\frac{113}{163})^{4} \cdot \pi^{0} = (\frac{163}{113} \cdot \frac{113}{163})^{4} \cdot 1 = 1^{4} \cdot 1 = 1$$
\rozwStop
\odpStart
$1$
\odpStop
\testStart
A.$1$ B.$\pi$ C.$0$ D.$\frac{163}{113}$ E.$\frac{113}{163}$
F.$-\frac{163}{113}$ G.$-1$
H.$(\frac{163}{113})^{4}$
I.$(\frac{113}{163})^{4}$
\testStop
\kluczStart
A
\kluczStop



\zadStart{Zadanie z Wikieł Z 1.1 d) moja wersja nr 237}

Obliczyć wartość wyrażenia $(\frac{163}{127})^{4} \cdot (\frac{127}{163})^{4} \cdot \pi^{0}$.
\zadStop
\rozwStart{Patryk Wirkus}{Martyna Czarnobaj}
$$(\frac{163}{127})^{4} \cdot (\frac{127}{163})^{4} \cdot \pi^{0} = (\frac{163}{127} \cdot \frac{127}{163})^{4} \cdot 1 = 1^{4} \cdot 1 = 1$$
\rozwStop
\odpStart
$1$
\odpStop
\testStart
A.$1$ B.$\pi$ C.$0$ D.$\frac{163}{127}$ E.$\frac{127}{163}$
F.$-\frac{163}{127}$ G.$-1$
H.$(\frac{163}{127})^{4}$
I.$(\frac{127}{163})^{4}$
\testStop
\kluczStart
A
\kluczStop



\zadStart{Zadanie z Wikieł Z 1.1 d) moja wersja nr 238}

Obliczyć wartość wyrażenia $(\frac{163}{131})^{4} \cdot (\frac{131}{163})^{4} \cdot \pi^{0}$.
\zadStop
\rozwStart{Patryk Wirkus}{Martyna Czarnobaj}
$$(\frac{163}{131})^{4} \cdot (\frac{131}{163})^{4} \cdot \pi^{0} = (\frac{163}{131} \cdot \frac{131}{163})^{4} \cdot 1 = 1^{4} \cdot 1 = 1$$
\rozwStop
\odpStart
$1$
\odpStop
\testStart
A.$1$ B.$\pi$ C.$0$ D.$\frac{163}{131}$ E.$\frac{131}{163}$
F.$-\frac{163}{131}$ G.$-1$
H.$(\frac{163}{131})^{4}$
I.$(\frac{131}{163})^{4}$
\testStop
\kluczStart
A
\kluczStop



\zadStart{Zadanie z Wikieł Z 1.1 d) moja wersja nr 239}

Obliczyć wartość wyrażenia $(\frac{163}{137})^{4} \cdot (\frac{137}{163})^{4} \cdot \pi^{0}$.
\zadStop
\rozwStart{Patryk Wirkus}{Martyna Czarnobaj}
$$(\frac{163}{137})^{4} \cdot (\frac{137}{163})^{4} \cdot \pi^{0} = (\frac{163}{137} \cdot \frac{137}{163})^{4} \cdot 1 = 1^{4} \cdot 1 = 1$$
\rozwStop
\odpStart
$1$
\odpStop
\testStart
A.$1$ B.$\pi$ C.$0$ D.$\frac{163}{137}$ E.$\frac{137}{163}$
F.$-\frac{163}{137}$ G.$-1$
H.$(\frac{163}{137})^{4}$
I.$(\frac{137}{163})^{4}$
\testStop
\kluczStart
A
\kluczStop



\zadStart{Zadanie z Wikieł Z 1.1 d) moja wersja nr 240}

Obliczyć wartość wyrażenia $(\frac{163}{139})^{4} \cdot (\frac{139}{163})^{4} \cdot \pi^{0}$.
\zadStop
\rozwStart{Patryk Wirkus}{Martyna Czarnobaj}
$$(\frac{163}{139})^{4} \cdot (\frac{139}{163})^{4} \cdot \pi^{0} = (\frac{163}{139} \cdot \frac{139}{163})^{4} \cdot 1 = 1^{4} \cdot 1 = 1$$
\rozwStop
\odpStart
$1$
\odpStop
\testStart
A.$1$ B.$\pi$ C.$0$ D.$\frac{163}{139}$ E.$\frac{139}{163}$
F.$-\frac{163}{139}$ G.$-1$
H.$(\frac{163}{139})^{4}$
I.$(\frac{139}{163})^{4}$
\testStop
\kluczStart
A
\kluczStop



\zadStart{Zadanie z Wikieł Z 1.1 d) moja wersja nr 241}

Obliczyć wartość wyrażenia $(\frac{167}{103})^{4} \cdot (\frac{103}{167})^{4} \cdot \pi^{0}$.
\zadStop
\rozwStart{Patryk Wirkus}{Martyna Czarnobaj}
$$(\frac{167}{103})^{4} \cdot (\frac{103}{167})^{4} \cdot \pi^{0} = (\frac{167}{103} \cdot \frac{103}{167})^{4} \cdot 1 = 1^{4} \cdot 1 = 1$$
\rozwStop
\odpStart
$1$
\odpStop
\testStart
A.$1$ B.$\pi$ C.$0$ D.$\frac{167}{103}$ E.$\frac{103}{167}$
F.$-\frac{167}{103}$ G.$-1$
H.$(\frac{167}{103})^{4}$
I.$(\frac{103}{167})^{4}$
\testStop
\kluczStart
A
\kluczStop



\zadStart{Zadanie z Wikieł Z 1.1 d) moja wersja nr 242}

Obliczyć wartość wyrażenia $(\frac{167}{107})^{4} \cdot (\frac{107}{167})^{4} \cdot \pi^{0}$.
\zadStop
\rozwStart{Patryk Wirkus}{Martyna Czarnobaj}
$$(\frac{167}{107})^{4} \cdot (\frac{107}{167})^{4} \cdot \pi^{0} = (\frac{167}{107} \cdot \frac{107}{167})^{4} \cdot 1 = 1^{4} \cdot 1 = 1$$
\rozwStop
\odpStart
$1$
\odpStop
\testStart
A.$1$ B.$\pi$ C.$0$ D.$\frac{167}{107}$ E.$\frac{107}{167}$
F.$-\frac{167}{107}$ G.$-1$
H.$(\frac{167}{107})^{4}$
I.$(\frac{107}{167})^{4}$
\testStop
\kluczStart
A
\kluczStop



\zadStart{Zadanie z Wikieł Z 1.1 d) moja wersja nr 243}

Obliczyć wartość wyrażenia $(\frac{167}{109})^{4} \cdot (\frac{109}{167})^{4} \cdot \pi^{0}$.
\zadStop
\rozwStart{Patryk Wirkus}{Martyna Czarnobaj}
$$(\frac{167}{109})^{4} \cdot (\frac{109}{167})^{4} \cdot \pi^{0} = (\frac{167}{109} \cdot \frac{109}{167})^{4} \cdot 1 = 1^{4} \cdot 1 = 1$$
\rozwStop
\odpStart
$1$
\odpStop
\testStart
A.$1$ B.$\pi$ C.$0$ D.$\frac{167}{109}$ E.$\frac{109}{167}$
F.$-\frac{167}{109}$ G.$-1$
H.$(\frac{167}{109})^{4}$
I.$(\frac{109}{167})^{4}$
\testStop
\kluczStart
A
\kluczStop



\zadStart{Zadanie z Wikieł Z 1.1 d) moja wersja nr 244}

Obliczyć wartość wyrażenia $(\frac{167}{113})^{4} \cdot (\frac{113}{167})^{4} \cdot \pi^{0}$.
\zadStop
\rozwStart{Patryk Wirkus}{Martyna Czarnobaj}
$$(\frac{167}{113})^{4} \cdot (\frac{113}{167})^{4} \cdot \pi^{0} = (\frac{167}{113} \cdot \frac{113}{167})^{4} \cdot 1 = 1^{4} \cdot 1 = 1$$
\rozwStop
\odpStart
$1$
\odpStop
\testStart
A.$1$ B.$\pi$ C.$0$ D.$\frac{167}{113}$ E.$\frac{113}{167}$
F.$-\frac{167}{113}$ G.$-1$
H.$(\frac{167}{113})^{4}$
I.$(\frac{113}{167})^{4}$
\testStop
\kluczStart
A
\kluczStop



\zadStart{Zadanie z Wikieł Z 1.1 d) moja wersja nr 245}

Obliczyć wartość wyrażenia $(\frac{167}{127})^{4} \cdot (\frac{127}{167})^{4} \cdot \pi^{0}$.
\zadStop
\rozwStart{Patryk Wirkus}{Martyna Czarnobaj}
$$(\frac{167}{127})^{4} \cdot (\frac{127}{167})^{4} \cdot \pi^{0} = (\frac{167}{127} \cdot \frac{127}{167})^{4} \cdot 1 = 1^{4} \cdot 1 = 1$$
\rozwStop
\odpStart
$1$
\odpStop
\testStart
A.$1$ B.$\pi$ C.$0$ D.$\frac{167}{127}$ E.$\frac{127}{167}$
F.$-\frac{167}{127}$ G.$-1$
H.$(\frac{167}{127})^{4}$
I.$(\frac{127}{167})^{4}$
\testStop
\kluczStart
A
\kluczStop



\zadStart{Zadanie z Wikieł Z 1.1 d) moja wersja nr 246}

Obliczyć wartość wyrażenia $(\frac{167}{131})^{4} \cdot (\frac{131}{167})^{4} \cdot \pi^{0}$.
\zadStop
\rozwStart{Patryk Wirkus}{Martyna Czarnobaj}
$$(\frac{167}{131})^{4} \cdot (\frac{131}{167})^{4} \cdot \pi^{0} = (\frac{167}{131} \cdot \frac{131}{167})^{4} \cdot 1 = 1^{4} \cdot 1 = 1$$
\rozwStop
\odpStart
$1$
\odpStop
\testStart
A.$1$ B.$\pi$ C.$0$ D.$\frac{167}{131}$ E.$\frac{131}{167}$
F.$-\frac{167}{131}$ G.$-1$
H.$(\frac{167}{131})^{4}$
I.$(\frac{131}{167})^{4}$
\testStop
\kluczStart
A
\kluczStop



\zadStart{Zadanie z Wikieł Z 1.1 d) moja wersja nr 247}

Obliczyć wartość wyrażenia $(\frac{167}{137})^{4} \cdot (\frac{137}{167})^{4} \cdot \pi^{0}$.
\zadStop
\rozwStart{Patryk Wirkus}{Martyna Czarnobaj}
$$(\frac{167}{137})^{4} \cdot (\frac{137}{167})^{4} \cdot \pi^{0} = (\frac{167}{137} \cdot \frac{137}{167})^{4} \cdot 1 = 1^{4} \cdot 1 = 1$$
\rozwStop
\odpStart
$1$
\odpStop
\testStart
A.$1$ B.$\pi$ C.$0$ D.$\frac{167}{137}$ E.$\frac{137}{167}$
F.$-\frac{167}{137}$ G.$-1$
H.$(\frac{167}{137})^{4}$
I.$(\frac{137}{167})^{4}$
\testStop
\kluczStart
A
\kluczStop



\zadStart{Zadanie z Wikieł Z 1.1 d) moja wersja nr 248}

Obliczyć wartość wyrażenia $(\frac{167}{139})^{4} \cdot (\frac{139}{167})^{4} \cdot \pi^{0}$.
\zadStop
\rozwStart{Patryk Wirkus}{Martyna Czarnobaj}
$$(\frac{167}{139})^{4} \cdot (\frac{139}{167})^{4} \cdot \pi^{0} = (\frac{167}{139} \cdot \frac{139}{167})^{4} \cdot 1 = 1^{4} \cdot 1 = 1$$
\rozwStop
\odpStart
$1$
\odpStop
\testStart
A.$1$ B.$\pi$ C.$0$ D.$\frac{167}{139}$ E.$\frac{139}{167}$
F.$-\frac{167}{139}$ G.$-1$
H.$(\frac{167}{139})^{4}$
I.$(\frac{139}{167})^{4}$
\testStop
\kluczStart
A
\kluczStop



\zadStart{Zadanie z Wikieł Z 1.1 d) moja wersja nr 249}

Obliczyć wartość wyrażenia $(\frac{173}{103})^{4} \cdot (\frac{103}{173})^{4} \cdot \pi^{0}$.
\zadStop
\rozwStart{Patryk Wirkus}{Martyna Czarnobaj}
$$(\frac{173}{103})^{4} \cdot (\frac{103}{173})^{4} \cdot \pi^{0} = (\frac{173}{103} \cdot \frac{103}{173})^{4} \cdot 1 = 1^{4} \cdot 1 = 1$$
\rozwStop
\odpStart
$1$
\odpStop
\testStart
A.$1$ B.$\pi$ C.$0$ D.$\frac{173}{103}$ E.$\frac{103}{173}$
F.$-\frac{173}{103}$ G.$-1$
H.$(\frac{173}{103})^{4}$
I.$(\frac{103}{173})^{4}$
\testStop
\kluczStart
A
\kluczStop



\zadStart{Zadanie z Wikieł Z 1.1 d) moja wersja nr 250}

Obliczyć wartość wyrażenia $(\frac{173}{107})^{4} \cdot (\frac{107}{173})^{4} \cdot \pi^{0}$.
\zadStop
\rozwStart{Patryk Wirkus}{Martyna Czarnobaj}
$$(\frac{173}{107})^{4} \cdot (\frac{107}{173})^{4} \cdot \pi^{0} = (\frac{173}{107} \cdot \frac{107}{173})^{4} \cdot 1 = 1^{4} \cdot 1 = 1$$
\rozwStop
\odpStart
$1$
\odpStop
\testStart
A.$1$ B.$\pi$ C.$0$ D.$\frac{173}{107}$ E.$\frac{107}{173}$
F.$-\frac{173}{107}$ G.$-1$
H.$(\frac{173}{107})^{4}$
I.$(\frac{107}{173})^{4}$
\testStop
\kluczStart
A
\kluczStop



\zadStart{Zadanie z Wikieł Z 1.1 d) moja wersja nr 251}

Obliczyć wartość wyrażenia $(\frac{173}{109})^{4} \cdot (\frac{109}{173})^{4} \cdot \pi^{0}$.
\zadStop
\rozwStart{Patryk Wirkus}{Martyna Czarnobaj}
$$(\frac{173}{109})^{4} \cdot (\frac{109}{173})^{4} \cdot \pi^{0} = (\frac{173}{109} \cdot \frac{109}{173})^{4} \cdot 1 = 1^{4} \cdot 1 = 1$$
\rozwStop
\odpStart
$1$
\odpStop
\testStart
A.$1$ B.$\pi$ C.$0$ D.$\frac{173}{109}$ E.$\frac{109}{173}$
F.$-\frac{173}{109}$ G.$-1$
H.$(\frac{173}{109})^{4}$
I.$(\frac{109}{173})^{4}$
\testStop
\kluczStart
A
\kluczStop



\zadStart{Zadanie z Wikieł Z 1.1 d) moja wersja nr 252}

Obliczyć wartość wyrażenia $(\frac{173}{113})^{4} \cdot (\frac{113}{173})^{4} \cdot \pi^{0}$.
\zadStop
\rozwStart{Patryk Wirkus}{Martyna Czarnobaj}
$$(\frac{173}{113})^{4} \cdot (\frac{113}{173})^{4} \cdot \pi^{0} = (\frac{173}{113} \cdot \frac{113}{173})^{4} \cdot 1 = 1^{4} \cdot 1 = 1$$
\rozwStop
\odpStart
$1$
\odpStop
\testStart
A.$1$ B.$\pi$ C.$0$ D.$\frac{173}{113}$ E.$\frac{113}{173}$
F.$-\frac{173}{113}$ G.$-1$
H.$(\frac{173}{113})^{4}$
I.$(\frac{113}{173})^{4}$
\testStop
\kluczStart
A
\kluczStop



\zadStart{Zadanie z Wikieł Z 1.1 d) moja wersja nr 253}

Obliczyć wartość wyrażenia $(\frac{173}{127})^{4} \cdot (\frac{127}{173})^{4} \cdot \pi^{0}$.
\zadStop
\rozwStart{Patryk Wirkus}{Martyna Czarnobaj}
$$(\frac{173}{127})^{4} \cdot (\frac{127}{173})^{4} \cdot \pi^{0} = (\frac{173}{127} \cdot \frac{127}{173})^{4} \cdot 1 = 1^{4} \cdot 1 = 1$$
\rozwStop
\odpStart
$1$
\odpStop
\testStart
A.$1$ B.$\pi$ C.$0$ D.$\frac{173}{127}$ E.$\frac{127}{173}$
F.$-\frac{173}{127}$ G.$-1$
H.$(\frac{173}{127})^{4}$
I.$(\frac{127}{173})^{4}$
\testStop
\kluczStart
A
\kluczStop



\zadStart{Zadanie z Wikieł Z 1.1 d) moja wersja nr 254}

Obliczyć wartość wyrażenia $(\frac{173}{131})^{4} \cdot (\frac{131}{173})^{4} \cdot \pi^{0}$.
\zadStop
\rozwStart{Patryk Wirkus}{Martyna Czarnobaj}
$$(\frac{173}{131})^{4} \cdot (\frac{131}{173})^{4} \cdot \pi^{0} = (\frac{173}{131} \cdot \frac{131}{173})^{4} \cdot 1 = 1^{4} \cdot 1 = 1$$
\rozwStop
\odpStart
$1$
\odpStop
\testStart
A.$1$ B.$\pi$ C.$0$ D.$\frac{173}{131}$ E.$\frac{131}{173}$
F.$-\frac{173}{131}$ G.$-1$
H.$(\frac{173}{131})^{4}$
I.$(\frac{131}{173})^{4}$
\testStop
\kluczStart
A
\kluczStop



\zadStart{Zadanie z Wikieł Z 1.1 d) moja wersja nr 255}

Obliczyć wartość wyrażenia $(\frac{173}{137})^{4} \cdot (\frac{137}{173})^{4} \cdot \pi^{0}$.
\zadStop
\rozwStart{Patryk Wirkus}{Martyna Czarnobaj}
$$(\frac{173}{137})^{4} \cdot (\frac{137}{173})^{4} \cdot \pi^{0} = (\frac{173}{137} \cdot \frac{137}{173})^{4} \cdot 1 = 1^{4} \cdot 1 = 1$$
\rozwStop
\odpStart
$1$
\odpStop
\testStart
A.$1$ B.$\pi$ C.$0$ D.$\frac{173}{137}$ E.$\frac{137}{173}$
F.$-\frac{173}{137}$ G.$-1$
H.$(\frac{173}{137})^{4}$
I.$(\frac{137}{173})^{4}$
\testStop
\kluczStart
A
\kluczStop



\zadStart{Zadanie z Wikieł Z 1.1 d) moja wersja nr 256}

Obliczyć wartość wyrażenia $(\frac{173}{139})^{4} \cdot (\frac{139}{173})^{4} \cdot \pi^{0}$.
\zadStop
\rozwStart{Patryk Wirkus}{Martyna Czarnobaj}
$$(\frac{173}{139})^{4} \cdot (\frac{139}{173})^{4} \cdot \pi^{0} = (\frac{173}{139} \cdot \frac{139}{173})^{4} \cdot 1 = 1^{4} \cdot 1 = 1$$
\rozwStop
\odpStart
$1$
\odpStop
\testStart
A.$1$ B.$\pi$ C.$0$ D.$\frac{173}{139}$ E.$\frac{139}{173}$
F.$-\frac{173}{139}$ G.$-1$
H.$(\frac{173}{139})^{4}$
I.$(\frac{139}{173})^{4}$
\testStop
\kluczStart
A
\kluczStop



\zadStart{Zadanie z Wikieł Z 1.1 d) moja wersja nr 257}

Obliczyć wartość wyrażenia $(\frac{179}{103})^{4} \cdot (\frac{103}{179})^{4} \cdot \pi^{0}$.
\zadStop
\rozwStart{Patryk Wirkus}{Martyna Czarnobaj}
$$(\frac{179}{103})^{4} \cdot (\frac{103}{179})^{4} \cdot \pi^{0} = (\frac{179}{103} \cdot \frac{103}{179})^{4} \cdot 1 = 1^{4} \cdot 1 = 1$$
\rozwStop
\odpStart
$1$
\odpStop
\testStart
A.$1$ B.$\pi$ C.$0$ D.$\frac{179}{103}$ E.$\frac{103}{179}$
F.$-\frac{179}{103}$ G.$-1$
H.$(\frac{179}{103})^{4}$
I.$(\frac{103}{179})^{4}$
\testStop
\kluczStart
A
\kluczStop



\zadStart{Zadanie z Wikieł Z 1.1 d) moja wersja nr 258}

Obliczyć wartość wyrażenia $(\frac{179}{107})^{4} \cdot (\frac{107}{179})^{4} \cdot \pi^{0}$.
\zadStop
\rozwStart{Patryk Wirkus}{Martyna Czarnobaj}
$$(\frac{179}{107})^{4} \cdot (\frac{107}{179})^{4} \cdot \pi^{0} = (\frac{179}{107} \cdot \frac{107}{179})^{4} \cdot 1 = 1^{4} \cdot 1 = 1$$
\rozwStop
\odpStart
$1$
\odpStop
\testStart
A.$1$ B.$\pi$ C.$0$ D.$\frac{179}{107}$ E.$\frac{107}{179}$
F.$-\frac{179}{107}$ G.$-1$
H.$(\frac{179}{107})^{4}$
I.$(\frac{107}{179})^{4}$
\testStop
\kluczStart
A
\kluczStop



\zadStart{Zadanie z Wikieł Z 1.1 d) moja wersja nr 259}

Obliczyć wartość wyrażenia $(\frac{179}{109})^{4} \cdot (\frac{109}{179})^{4} \cdot \pi^{0}$.
\zadStop
\rozwStart{Patryk Wirkus}{Martyna Czarnobaj}
$$(\frac{179}{109})^{4} \cdot (\frac{109}{179})^{4} \cdot \pi^{0} = (\frac{179}{109} \cdot \frac{109}{179})^{4} \cdot 1 = 1^{4} \cdot 1 = 1$$
\rozwStop
\odpStart
$1$
\odpStop
\testStart
A.$1$ B.$\pi$ C.$0$ D.$\frac{179}{109}$ E.$\frac{109}{179}$
F.$-\frac{179}{109}$ G.$-1$
H.$(\frac{179}{109})^{4}$
I.$(\frac{109}{179})^{4}$
\testStop
\kluczStart
A
\kluczStop



\zadStart{Zadanie z Wikieł Z 1.1 d) moja wersja nr 260}

Obliczyć wartość wyrażenia $(\frac{179}{113})^{4} \cdot (\frac{113}{179})^{4} \cdot \pi^{0}$.
\zadStop
\rozwStart{Patryk Wirkus}{Martyna Czarnobaj}
$$(\frac{179}{113})^{4} \cdot (\frac{113}{179})^{4} \cdot \pi^{0} = (\frac{179}{113} \cdot \frac{113}{179})^{4} \cdot 1 = 1^{4} \cdot 1 = 1$$
\rozwStop
\odpStart
$1$
\odpStop
\testStart
A.$1$ B.$\pi$ C.$0$ D.$\frac{179}{113}$ E.$\frac{113}{179}$
F.$-\frac{179}{113}$ G.$-1$
H.$(\frac{179}{113})^{4}$
I.$(\frac{113}{179})^{4}$
\testStop
\kluczStart
A
\kluczStop



\zadStart{Zadanie z Wikieł Z 1.1 d) moja wersja nr 261}

Obliczyć wartość wyrażenia $(\frac{179}{127})^{4} \cdot (\frac{127}{179})^{4} \cdot \pi^{0}$.
\zadStop
\rozwStart{Patryk Wirkus}{Martyna Czarnobaj}
$$(\frac{179}{127})^{4} \cdot (\frac{127}{179})^{4} \cdot \pi^{0} = (\frac{179}{127} \cdot \frac{127}{179})^{4} \cdot 1 = 1^{4} \cdot 1 = 1$$
\rozwStop
\odpStart
$1$
\odpStop
\testStart
A.$1$ B.$\pi$ C.$0$ D.$\frac{179}{127}$ E.$\frac{127}{179}$
F.$-\frac{179}{127}$ G.$-1$
H.$(\frac{179}{127})^{4}$
I.$(\frac{127}{179})^{4}$
\testStop
\kluczStart
A
\kluczStop



\zadStart{Zadanie z Wikieł Z 1.1 d) moja wersja nr 262}

Obliczyć wartość wyrażenia $(\frac{179}{131})^{4} \cdot (\frac{131}{179})^{4} \cdot \pi^{0}$.
\zadStop
\rozwStart{Patryk Wirkus}{Martyna Czarnobaj}
$$(\frac{179}{131})^{4} \cdot (\frac{131}{179})^{4} \cdot \pi^{0} = (\frac{179}{131} \cdot \frac{131}{179})^{4} \cdot 1 = 1^{4} \cdot 1 = 1$$
\rozwStop
\odpStart
$1$
\odpStop
\testStart
A.$1$ B.$\pi$ C.$0$ D.$\frac{179}{131}$ E.$\frac{131}{179}$
F.$-\frac{179}{131}$ G.$-1$
H.$(\frac{179}{131})^{4}$
I.$(\frac{131}{179})^{4}$
\testStop
\kluczStart
A
\kluczStop



\zadStart{Zadanie z Wikieł Z 1.1 d) moja wersja nr 263}

Obliczyć wartość wyrażenia $(\frac{179}{137})^{4} \cdot (\frac{137}{179})^{4} \cdot \pi^{0}$.
\zadStop
\rozwStart{Patryk Wirkus}{Martyna Czarnobaj}
$$(\frac{179}{137})^{4} \cdot (\frac{137}{179})^{4} \cdot \pi^{0} = (\frac{179}{137} \cdot \frac{137}{179})^{4} \cdot 1 = 1^{4} \cdot 1 = 1$$
\rozwStop
\odpStart
$1$
\odpStop
\testStart
A.$1$ B.$\pi$ C.$0$ D.$\frac{179}{137}$ E.$\frac{137}{179}$
F.$-\frac{179}{137}$ G.$-1$
H.$(\frac{179}{137})^{4}$
I.$(\frac{137}{179})^{4}$
\testStop
\kluczStart
A
\kluczStop



\zadStart{Zadanie z Wikieł Z 1.1 d) moja wersja nr 264}

Obliczyć wartość wyrażenia $(\frac{179}{139})^{4} \cdot (\frac{139}{179})^{4} \cdot \pi^{0}$.
\zadStop
\rozwStart{Patryk Wirkus}{Martyna Czarnobaj}
$$(\frac{179}{139})^{4} \cdot (\frac{139}{179})^{4} \cdot \pi^{0} = (\frac{179}{139} \cdot \frac{139}{179})^{4} \cdot 1 = 1^{4} \cdot 1 = 1$$
\rozwStop
\odpStart
$1$
\odpStop
\testStart
A.$1$ B.$\pi$ C.$0$ D.$\frac{179}{139}$ E.$\frac{139}{179}$
F.$-\frac{179}{139}$ G.$-1$
H.$(\frac{179}{139})^{4}$
I.$(\frac{139}{179})^{4}$
\testStop
\kluczStart
A
\kluczStop



\zadStart{Zadanie z Wikieł Z 1.1 d) moja wersja nr 265}

Obliczyć wartość wyrażenia $(\frac{251}{103})^{4} \cdot (\frac{103}{251})^{4} \cdot \pi^{0}$.
\zadStop
\rozwStart{Patryk Wirkus}{Martyna Czarnobaj}
$$(\frac{251}{103})^{4} \cdot (\frac{103}{251})^{4} \cdot \pi^{0} = (\frac{251}{103} \cdot \frac{103}{251})^{4} \cdot 1 = 1^{4} \cdot 1 = 1$$
\rozwStop
\odpStart
$1$
\odpStop
\testStart
A.$1$ B.$\pi$ C.$0$ D.$\frac{251}{103}$ E.$\frac{103}{251}$
F.$-\frac{251}{103}$ G.$-1$
H.$(\frac{251}{103})^{4}$
I.$(\frac{103}{251})^{4}$
\testStop
\kluczStart
A
\kluczStop



\zadStart{Zadanie z Wikieł Z 1.1 d) moja wersja nr 266}

Obliczyć wartość wyrażenia $(\frac{251}{107})^{4} \cdot (\frac{107}{251})^{4} \cdot \pi^{0}$.
\zadStop
\rozwStart{Patryk Wirkus}{Martyna Czarnobaj}
$$(\frac{251}{107})^{4} \cdot (\frac{107}{251})^{4} \cdot \pi^{0} = (\frac{251}{107} \cdot \frac{107}{251})^{4} \cdot 1 = 1^{4} \cdot 1 = 1$$
\rozwStop
\odpStart
$1$
\odpStop
\testStart
A.$1$ B.$\pi$ C.$0$ D.$\frac{251}{107}$ E.$\frac{107}{251}$
F.$-\frac{251}{107}$ G.$-1$
H.$(\frac{251}{107})^{4}$
I.$(\frac{107}{251})^{4}$
\testStop
\kluczStart
A
\kluczStop



\zadStart{Zadanie z Wikieł Z 1.1 d) moja wersja nr 267}

Obliczyć wartość wyrażenia $(\frac{251}{109})^{4} \cdot (\frac{109}{251})^{4} \cdot \pi^{0}$.
\zadStop
\rozwStart{Patryk Wirkus}{Martyna Czarnobaj}
$$(\frac{251}{109})^{4} \cdot (\frac{109}{251})^{4} \cdot \pi^{0} = (\frac{251}{109} \cdot \frac{109}{251})^{4} \cdot 1 = 1^{4} \cdot 1 = 1$$
\rozwStop
\odpStart
$1$
\odpStop
\testStart
A.$1$ B.$\pi$ C.$0$ D.$\frac{251}{109}$ E.$\frac{109}{251}$
F.$-\frac{251}{109}$ G.$-1$
H.$(\frac{251}{109})^{4}$
I.$(\frac{109}{251})^{4}$
\testStop
\kluczStart
A
\kluczStop



\zadStart{Zadanie z Wikieł Z 1.1 d) moja wersja nr 268}

Obliczyć wartość wyrażenia $(\frac{251}{113})^{4} \cdot (\frac{113}{251})^{4} \cdot \pi^{0}$.
\zadStop
\rozwStart{Patryk Wirkus}{Martyna Czarnobaj}
$$(\frac{251}{113})^{4} \cdot (\frac{113}{251})^{4} \cdot \pi^{0} = (\frac{251}{113} \cdot \frac{113}{251})^{4} \cdot 1 = 1^{4} \cdot 1 = 1$$
\rozwStop
\odpStart
$1$
\odpStop
\testStart
A.$1$ B.$\pi$ C.$0$ D.$\frac{251}{113}$ E.$\frac{113}{251}$
F.$-\frac{251}{113}$ G.$-1$
H.$(\frac{251}{113})^{4}$
I.$(\frac{113}{251})^{4}$
\testStop
\kluczStart
A
\kluczStop



\zadStart{Zadanie z Wikieł Z 1.1 d) moja wersja nr 269}

Obliczyć wartość wyrażenia $(\frac{251}{127})^{4} \cdot (\frac{127}{251})^{4} \cdot \pi^{0}$.
\zadStop
\rozwStart{Patryk Wirkus}{Martyna Czarnobaj}
$$(\frac{251}{127})^{4} \cdot (\frac{127}{251})^{4} \cdot \pi^{0} = (\frac{251}{127} \cdot \frac{127}{251})^{4} \cdot 1 = 1^{4} \cdot 1 = 1$$
\rozwStop
\odpStart
$1$
\odpStop
\testStart
A.$1$ B.$\pi$ C.$0$ D.$\frac{251}{127}$ E.$\frac{127}{251}$
F.$-\frac{251}{127}$ G.$-1$
H.$(\frac{251}{127})^{4}$
I.$(\frac{127}{251})^{4}$
\testStop
\kluczStart
A
\kluczStop



\zadStart{Zadanie z Wikieł Z 1.1 d) moja wersja nr 270}

Obliczyć wartość wyrażenia $(\frac{251}{131})^{4} \cdot (\frac{131}{251})^{4} \cdot \pi^{0}$.
\zadStop
\rozwStart{Patryk Wirkus}{Martyna Czarnobaj}
$$(\frac{251}{131})^{4} \cdot (\frac{131}{251})^{4} \cdot \pi^{0} = (\frac{251}{131} \cdot \frac{131}{251})^{4} \cdot 1 = 1^{4} \cdot 1 = 1$$
\rozwStop
\odpStart
$1$
\odpStop
\testStart
A.$1$ B.$\pi$ C.$0$ D.$\frac{251}{131}$ E.$\frac{131}{251}$
F.$-\frac{251}{131}$ G.$-1$
H.$(\frac{251}{131})^{4}$
I.$(\frac{131}{251})^{4}$
\testStop
\kluczStart
A
\kluczStop



\zadStart{Zadanie z Wikieł Z 1.1 d) moja wersja nr 271}

Obliczyć wartość wyrażenia $(\frac{251}{137})^{4} \cdot (\frac{137}{251})^{4} \cdot \pi^{0}$.
\zadStop
\rozwStart{Patryk Wirkus}{Martyna Czarnobaj}
$$(\frac{251}{137})^{4} \cdot (\frac{137}{251})^{4} \cdot \pi^{0} = (\frac{251}{137} \cdot \frac{137}{251})^{4} \cdot 1 = 1^{4} \cdot 1 = 1$$
\rozwStop
\odpStart
$1$
\odpStop
\testStart
A.$1$ B.$\pi$ C.$0$ D.$\frac{251}{137}$ E.$\frac{137}{251}$
F.$-\frac{251}{137}$ G.$-1$
H.$(\frac{251}{137})^{4}$
I.$(\frac{137}{251})^{4}$
\testStop
\kluczStart
A
\kluczStop



\zadStart{Zadanie z Wikieł Z 1.1 d) moja wersja nr 272}

Obliczyć wartość wyrażenia $(\frac{251}{139})^{4} \cdot (\frac{139}{251})^{4} \cdot \pi^{0}$.
\zadStop
\rozwStart{Patryk Wirkus}{Martyna Czarnobaj}
$$(\frac{251}{139})^{4} \cdot (\frac{139}{251})^{4} \cdot \pi^{0} = (\frac{251}{139} \cdot \frac{139}{251})^{4} \cdot 1 = 1^{4} \cdot 1 = 1$$
\rozwStop
\odpStart
$1$
\odpStop
\testStart
A.$1$ B.$\pi$ C.$0$ D.$\frac{251}{139}$ E.$\frac{139}{251}$
F.$-\frac{251}{139}$ G.$-1$
H.$(\frac{251}{139})^{4}$
I.$(\frac{139}{251})^{4}$
\testStop
\kluczStart
A
\kluczStop



\zadStart{Zadanie z Wikieł Z 1.1 d) moja wersja nr 273}

Obliczyć wartość wyrażenia $(\frac{257}{103})^{4} \cdot (\frac{103}{257})^{4} \cdot \pi^{0}$.
\zadStop
\rozwStart{Patryk Wirkus}{Martyna Czarnobaj}
$$(\frac{257}{103})^{4} \cdot (\frac{103}{257})^{4} \cdot \pi^{0} = (\frac{257}{103} \cdot \frac{103}{257})^{4} \cdot 1 = 1^{4} \cdot 1 = 1$$
\rozwStop
\odpStart
$1$
\odpStop
\testStart
A.$1$ B.$\pi$ C.$0$ D.$\frac{257}{103}$ E.$\frac{103}{257}$
F.$-\frac{257}{103}$ G.$-1$
H.$(\frac{257}{103})^{4}$
I.$(\frac{103}{257})^{4}$
\testStop
\kluczStart
A
\kluczStop



\zadStart{Zadanie z Wikieł Z 1.1 d) moja wersja nr 274}

Obliczyć wartość wyrażenia $(\frac{257}{107})^{4} \cdot (\frac{107}{257})^{4} \cdot \pi^{0}$.
\zadStop
\rozwStart{Patryk Wirkus}{Martyna Czarnobaj}
$$(\frac{257}{107})^{4} \cdot (\frac{107}{257})^{4} \cdot \pi^{0} = (\frac{257}{107} \cdot \frac{107}{257})^{4} \cdot 1 = 1^{4} \cdot 1 = 1$$
\rozwStop
\odpStart
$1$
\odpStop
\testStart
A.$1$ B.$\pi$ C.$0$ D.$\frac{257}{107}$ E.$\frac{107}{257}$
F.$-\frac{257}{107}$ G.$-1$
H.$(\frac{257}{107})^{4}$
I.$(\frac{107}{257})^{4}$
\testStop
\kluczStart
A
\kluczStop



\zadStart{Zadanie z Wikieł Z 1.1 d) moja wersja nr 275}

Obliczyć wartość wyrażenia $(\frac{257}{109})^{4} \cdot (\frac{109}{257})^{4} \cdot \pi^{0}$.
\zadStop
\rozwStart{Patryk Wirkus}{Martyna Czarnobaj}
$$(\frac{257}{109})^{4} \cdot (\frac{109}{257})^{4} \cdot \pi^{0} = (\frac{257}{109} \cdot \frac{109}{257})^{4} \cdot 1 = 1^{4} \cdot 1 = 1$$
\rozwStop
\odpStart
$1$
\odpStop
\testStart
A.$1$ B.$\pi$ C.$0$ D.$\frac{257}{109}$ E.$\frac{109}{257}$
F.$-\frac{257}{109}$ G.$-1$
H.$(\frac{257}{109})^{4}$
I.$(\frac{109}{257})^{4}$
\testStop
\kluczStart
A
\kluczStop



\zadStart{Zadanie z Wikieł Z 1.1 d) moja wersja nr 276}

Obliczyć wartość wyrażenia $(\frac{257}{113})^{4} \cdot (\frac{113}{257})^{4} \cdot \pi^{0}$.
\zadStop
\rozwStart{Patryk Wirkus}{Martyna Czarnobaj}
$$(\frac{257}{113})^{4} \cdot (\frac{113}{257})^{4} \cdot \pi^{0} = (\frac{257}{113} \cdot \frac{113}{257})^{4} \cdot 1 = 1^{4} \cdot 1 = 1$$
\rozwStop
\odpStart
$1$
\odpStop
\testStart
A.$1$ B.$\pi$ C.$0$ D.$\frac{257}{113}$ E.$\frac{113}{257}$
F.$-\frac{257}{113}$ G.$-1$
H.$(\frac{257}{113})^{4}$
I.$(\frac{113}{257})^{4}$
\testStop
\kluczStart
A
\kluczStop



\zadStart{Zadanie z Wikieł Z 1.1 d) moja wersja nr 277}

Obliczyć wartość wyrażenia $(\frac{257}{127})^{4} \cdot (\frac{127}{257})^{4} \cdot \pi^{0}$.
\zadStop
\rozwStart{Patryk Wirkus}{Martyna Czarnobaj}
$$(\frac{257}{127})^{4} \cdot (\frac{127}{257})^{4} \cdot \pi^{0} = (\frac{257}{127} \cdot \frac{127}{257})^{4} \cdot 1 = 1^{4} \cdot 1 = 1$$
\rozwStop
\odpStart
$1$
\odpStop
\testStart
A.$1$ B.$\pi$ C.$0$ D.$\frac{257}{127}$ E.$\frac{127}{257}$
F.$-\frac{257}{127}$ G.$-1$
H.$(\frac{257}{127})^{4}$
I.$(\frac{127}{257})^{4}$
\testStop
\kluczStart
A
\kluczStop



\zadStart{Zadanie z Wikieł Z 1.1 d) moja wersja nr 278}

Obliczyć wartość wyrażenia $(\frac{257}{131})^{4} \cdot (\frac{131}{257})^{4} \cdot \pi^{0}$.
\zadStop
\rozwStart{Patryk Wirkus}{Martyna Czarnobaj}
$$(\frac{257}{131})^{4} \cdot (\frac{131}{257})^{4} \cdot \pi^{0} = (\frac{257}{131} \cdot \frac{131}{257})^{4} \cdot 1 = 1^{4} \cdot 1 = 1$$
\rozwStop
\odpStart
$1$
\odpStop
\testStart
A.$1$ B.$\pi$ C.$0$ D.$\frac{257}{131}$ E.$\frac{131}{257}$
F.$-\frac{257}{131}$ G.$-1$
H.$(\frac{257}{131})^{4}$
I.$(\frac{131}{257})^{4}$
\testStop
\kluczStart
A
\kluczStop



\zadStart{Zadanie z Wikieł Z 1.1 d) moja wersja nr 279}

Obliczyć wartość wyrażenia $(\frac{257}{137})^{4} \cdot (\frac{137}{257})^{4} \cdot \pi^{0}$.
\zadStop
\rozwStart{Patryk Wirkus}{Martyna Czarnobaj}
$$(\frac{257}{137})^{4} \cdot (\frac{137}{257})^{4} \cdot \pi^{0} = (\frac{257}{137} \cdot \frac{137}{257})^{4} \cdot 1 = 1^{4} \cdot 1 = 1$$
\rozwStop
\odpStart
$1$
\odpStop
\testStart
A.$1$ B.$\pi$ C.$0$ D.$\frac{257}{137}$ E.$\frac{137}{257}$
F.$-\frac{257}{137}$ G.$-1$
H.$(\frac{257}{137})^{4}$
I.$(\frac{137}{257})^{4}$
\testStop
\kluczStart
A
\kluczStop



\zadStart{Zadanie z Wikieł Z 1.1 d) moja wersja nr 280}

Obliczyć wartość wyrażenia $(\frac{257}{139})^{4} \cdot (\frac{139}{257})^{4} \cdot \pi^{0}$.
\zadStop
\rozwStart{Patryk Wirkus}{Martyna Czarnobaj}
$$(\frac{257}{139})^{4} \cdot (\frac{139}{257})^{4} \cdot \pi^{0} = (\frac{257}{139} \cdot \frac{139}{257})^{4} \cdot 1 = 1^{4} \cdot 1 = 1$$
\rozwStop
\odpStart
$1$
\odpStop
\testStart
A.$1$ B.$\pi$ C.$0$ D.$\frac{257}{139}$ E.$\frac{139}{257}$
F.$-\frac{257}{139}$ G.$-1$
H.$(\frac{257}{139})^{4}$
I.$(\frac{139}{257})^{4}$
\testStop
\kluczStart
A
\kluczStop



\zadStart{Zadanie z Wikieł Z 1.1 d) moja wersja nr 281}

Obliczyć wartość wyrażenia $(\frac{263}{103})^{4} \cdot (\frac{103}{263})^{4} \cdot \pi^{0}$.
\zadStop
\rozwStart{Patryk Wirkus}{Martyna Czarnobaj}
$$(\frac{263}{103})^{4} \cdot (\frac{103}{263})^{4} \cdot \pi^{0} = (\frac{263}{103} \cdot \frac{103}{263})^{4} \cdot 1 = 1^{4} \cdot 1 = 1$$
\rozwStop
\odpStart
$1$
\odpStop
\testStart
A.$1$ B.$\pi$ C.$0$ D.$\frac{263}{103}$ E.$\frac{103}{263}$
F.$-\frac{263}{103}$ G.$-1$
H.$(\frac{263}{103})^{4}$
I.$(\frac{103}{263})^{4}$
\testStop
\kluczStart
A
\kluczStop



\zadStart{Zadanie z Wikieł Z 1.1 d) moja wersja nr 282}

Obliczyć wartość wyrażenia $(\frac{263}{107})^{4} \cdot (\frac{107}{263})^{4} \cdot \pi^{0}$.
\zadStop
\rozwStart{Patryk Wirkus}{Martyna Czarnobaj}
$$(\frac{263}{107})^{4} \cdot (\frac{107}{263})^{4} \cdot \pi^{0} = (\frac{263}{107} \cdot \frac{107}{263})^{4} \cdot 1 = 1^{4} \cdot 1 = 1$$
\rozwStop
\odpStart
$1$
\odpStop
\testStart
A.$1$ B.$\pi$ C.$0$ D.$\frac{263}{107}$ E.$\frac{107}{263}$
F.$-\frac{263}{107}$ G.$-1$
H.$(\frac{263}{107})^{4}$
I.$(\frac{107}{263})^{4}$
\testStop
\kluczStart
A
\kluczStop



\zadStart{Zadanie z Wikieł Z 1.1 d) moja wersja nr 283}

Obliczyć wartość wyrażenia $(\frac{263}{109})^{4} \cdot (\frac{109}{263})^{4} \cdot \pi^{0}$.
\zadStop
\rozwStart{Patryk Wirkus}{Martyna Czarnobaj}
$$(\frac{263}{109})^{4} \cdot (\frac{109}{263})^{4} \cdot \pi^{0} = (\frac{263}{109} \cdot \frac{109}{263})^{4} \cdot 1 = 1^{4} \cdot 1 = 1$$
\rozwStop
\odpStart
$1$
\odpStop
\testStart
A.$1$ B.$\pi$ C.$0$ D.$\frac{263}{109}$ E.$\frac{109}{263}$
F.$-\frac{263}{109}$ G.$-1$
H.$(\frac{263}{109})^{4}$
I.$(\frac{109}{263})^{4}$
\testStop
\kluczStart
A
\kluczStop



\zadStart{Zadanie z Wikieł Z 1.1 d) moja wersja nr 284}

Obliczyć wartość wyrażenia $(\frac{263}{113})^{4} \cdot (\frac{113}{263})^{4} \cdot \pi^{0}$.
\zadStop
\rozwStart{Patryk Wirkus}{Martyna Czarnobaj}
$$(\frac{263}{113})^{4} \cdot (\frac{113}{263})^{4} \cdot \pi^{0} = (\frac{263}{113} \cdot \frac{113}{263})^{4} \cdot 1 = 1^{4} \cdot 1 = 1$$
\rozwStop
\odpStart
$1$
\odpStop
\testStart
A.$1$ B.$\pi$ C.$0$ D.$\frac{263}{113}$ E.$\frac{113}{263}$
F.$-\frac{263}{113}$ G.$-1$
H.$(\frac{263}{113})^{4}$
I.$(\frac{113}{263})^{4}$
\testStop
\kluczStart
A
\kluczStop



\zadStart{Zadanie z Wikieł Z 1.1 d) moja wersja nr 285}

Obliczyć wartość wyrażenia $(\frac{263}{127})^{4} \cdot (\frac{127}{263})^{4} \cdot \pi^{0}$.
\zadStop
\rozwStart{Patryk Wirkus}{Martyna Czarnobaj}
$$(\frac{263}{127})^{4} \cdot (\frac{127}{263})^{4} \cdot \pi^{0} = (\frac{263}{127} \cdot \frac{127}{263})^{4} \cdot 1 = 1^{4} \cdot 1 = 1$$
\rozwStop
\odpStart
$1$
\odpStop
\testStart
A.$1$ B.$\pi$ C.$0$ D.$\frac{263}{127}$ E.$\frac{127}{263}$
F.$-\frac{263}{127}$ G.$-1$
H.$(\frac{263}{127})^{4}$
I.$(\frac{127}{263})^{4}$
\testStop
\kluczStart
A
\kluczStop



\zadStart{Zadanie z Wikieł Z 1.1 d) moja wersja nr 286}

Obliczyć wartość wyrażenia $(\frac{263}{131})^{4} \cdot (\frac{131}{263})^{4} \cdot \pi^{0}$.
\zadStop
\rozwStart{Patryk Wirkus}{Martyna Czarnobaj}
$$(\frac{263}{131})^{4} \cdot (\frac{131}{263})^{4} \cdot \pi^{0} = (\frac{263}{131} \cdot \frac{131}{263})^{4} \cdot 1 = 1^{4} \cdot 1 = 1$$
\rozwStop
\odpStart
$1$
\odpStop
\testStart
A.$1$ B.$\pi$ C.$0$ D.$\frac{263}{131}$ E.$\frac{131}{263}$
F.$-\frac{263}{131}$ G.$-1$
H.$(\frac{263}{131})^{4}$
I.$(\frac{131}{263})^{4}$
\testStop
\kluczStart
A
\kluczStop



\zadStart{Zadanie z Wikieł Z 1.1 d) moja wersja nr 287}

Obliczyć wartość wyrażenia $(\frac{263}{137})^{4} \cdot (\frac{137}{263})^{4} \cdot \pi^{0}$.
\zadStop
\rozwStart{Patryk Wirkus}{Martyna Czarnobaj}
$$(\frac{263}{137})^{4} \cdot (\frac{137}{263})^{4} \cdot \pi^{0} = (\frac{263}{137} \cdot \frac{137}{263})^{4} \cdot 1 = 1^{4} \cdot 1 = 1$$
\rozwStop
\odpStart
$1$
\odpStop
\testStart
A.$1$ B.$\pi$ C.$0$ D.$\frac{263}{137}$ E.$\frac{137}{263}$
F.$-\frac{263}{137}$ G.$-1$
H.$(\frac{263}{137})^{4}$
I.$(\frac{137}{263})^{4}$
\testStop
\kluczStart
A
\kluczStop



\zadStart{Zadanie z Wikieł Z 1.1 d) moja wersja nr 288}

Obliczyć wartość wyrażenia $(\frac{263}{139})^{4} \cdot (\frac{139}{263})^{4} \cdot \pi^{0}$.
\zadStop
\rozwStart{Patryk Wirkus}{Martyna Czarnobaj}
$$(\frac{263}{139})^{4} \cdot (\frac{139}{263})^{4} \cdot \pi^{0} = (\frac{263}{139} \cdot \frac{139}{263})^{4} \cdot 1 = 1^{4} \cdot 1 = 1$$
\rozwStop
\odpStart
$1$
\odpStop
\testStart
A.$1$ B.$\pi$ C.$0$ D.$\frac{263}{139}$ E.$\frac{139}{263}$
F.$-\frac{263}{139}$ G.$-1$
H.$(\frac{263}{139})^{4}$
I.$(\frac{139}{263})^{4}$
\testStop
\kluczStart
A
\kluczStop



\zadStart{Zadanie z Wikieł Z 1.1 d) moja wersja nr 289}

Obliczyć wartość wyrażenia $(\frac{269}{103})^{4} \cdot (\frac{103}{269})^{4} \cdot \pi^{0}$.
\zadStop
\rozwStart{Patryk Wirkus}{Martyna Czarnobaj}
$$(\frac{269}{103})^{4} \cdot (\frac{103}{269})^{4} \cdot \pi^{0} = (\frac{269}{103} \cdot \frac{103}{269})^{4} \cdot 1 = 1^{4} \cdot 1 = 1$$
\rozwStop
\odpStart
$1$
\odpStop
\testStart
A.$1$ B.$\pi$ C.$0$ D.$\frac{269}{103}$ E.$\frac{103}{269}$
F.$-\frac{269}{103}$ G.$-1$
H.$(\frac{269}{103})^{4}$
I.$(\frac{103}{269})^{4}$
\testStop
\kluczStart
A
\kluczStop



\zadStart{Zadanie z Wikieł Z 1.1 d) moja wersja nr 290}

Obliczyć wartość wyrażenia $(\frac{269}{107})^{4} \cdot (\frac{107}{269})^{4} \cdot \pi^{0}$.
\zadStop
\rozwStart{Patryk Wirkus}{Martyna Czarnobaj}
$$(\frac{269}{107})^{4} \cdot (\frac{107}{269})^{4} \cdot \pi^{0} = (\frac{269}{107} \cdot \frac{107}{269})^{4} \cdot 1 = 1^{4} \cdot 1 = 1$$
\rozwStop
\odpStart
$1$
\odpStop
\testStart
A.$1$ B.$\pi$ C.$0$ D.$\frac{269}{107}$ E.$\frac{107}{269}$
F.$-\frac{269}{107}$ G.$-1$
H.$(\frac{269}{107})^{4}$
I.$(\frac{107}{269})^{4}$
\testStop
\kluczStart
A
\kluczStop



\zadStart{Zadanie z Wikieł Z 1.1 d) moja wersja nr 291}

Obliczyć wartość wyrażenia $(\frac{269}{109})^{4} \cdot (\frac{109}{269})^{4} \cdot \pi^{0}$.
\zadStop
\rozwStart{Patryk Wirkus}{Martyna Czarnobaj}
$$(\frac{269}{109})^{4} \cdot (\frac{109}{269})^{4} \cdot \pi^{0} = (\frac{269}{109} \cdot \frac{109}{269})^{4} \cdot 1 = 1^{4} \cdot 1 = 1$$
\rozwStop
\odpStart
$1$
\odpStop
\testStart
A.$1$ B.$\pi$ C.$0$ D.$\frac{269}{109}$ E.$\frac{109}{269}$
F.$-\frac{269}{109}$ G.$-1$
H.$(\frac{269}{109})^{4}$
I.$(\frac{109}{269})^{4}$
\testStop
\kluczStart
A
\kluczStop



\zadStart{Zadanie z Wikieł Z 1.1 d) moja wersja nr 292}

Obliczyć wartość wyrażenia $(\frac{269}{113})^{4} \cdot (\frac{113}{269})^{4} \cdot \pi^{0}$.
\zadStop
\rozwStart{Patryk Wirkus}{Martyna Czarnobaj}
$$(\frac{269}{113})^{4} \cdot (\frac{113}{269})^{4} \cdot \pi^{0} = (\frac{269}{113} \cdot \frac{113}{269})^{4} \cdot 1 = 1^{4} \cdot 1 = 1$$
\rozwStop
\odpStart
$1$
\odpStop
\testStart
A.$1$ B.$\pi$ C.$0$ D.$\frac{269}{113}$ E.$\frac{113}{269}$
F.$-\frac{269}{113}$ G.$-1$
H.$(\frac{269}{113})^{4}$
I.$(\frac{113}{269})^{4}$
\testStop
\kluczStart
A
\kluczStop



\zadStart{Zadanie z Wikieł Z 1.1 d) moja wersja nr 293}

Obliczyć wartość wyrażenia $(\frac{269}{127})^{4} \cdot (\frac{127}{269})^{4} \cdot \pi^{0}$.
\zadStop
\rozwStart{Patryk Wirkus}{Martyna Czarnobaj}
$$(\frac{269}{127})^{4} \cdot (\frac{127}{269})^{4} \cdot \pi^{0} = (\frac{269}{127} \cdot \frac{127}{269})^{4} \cdot 1 = 1^{4} \cdot 1 = 1$$
\rozwStop
\odpStart
$1$
\odpStop
\testStart
A.$1$ B.$\pi$ C.$0$ D.$\frac{269}{127}$ E.$\frac{127}{269}$
F.$-\frac{269}{127}$ G.$-1$
H.$(\frac{269}{127})^{4}$
I.$(\frac{127}{269})^{4}$
\testStop
\kluczStart
A
\kluczStop



\zadStart{Zadanie z Wikieł Z 1.1 d) moja wersja nr 294}

Obliczyć wartość wyrażenia $(\frac{269}{131})^{4} \cdot (\frac{131}{269})^{4} \cdot \pi^{0}$.
\zadStop
\rozwStart{Patryk Wirkus}{Martyna Czarnobaj}
$$(\frac{269}{131})^{4} \cdot (\frac{131}{269})^{4} \cdot \pi^{0} = (\frac{269}{131} \cdot \frac{131}{269})^{4} \cdot 1 = 1^{4} \cdot 1 = 1$$
\rozwStop
\odpStart
$1$
\odpStop
\testStart
A.$1$ B.$\pi$ C.$0$ D.$\frac{269}{131}$ E.$\frac{131}{269}$
F.$-\frac{269}{131}$ G.$-1$
H.$(\frac{269}{131})^{4}$
I.$(\frac{131}{269})^{4}$
\testStop
\kluczStart
A
\kluczStop



\zadStart{Zadanie z Wikieł Z 1.1 d) moja wersja nr 295}

Obliczyć wartość wyrażenia $(\frac{269}{137})^{4} \cdot (\frac{137}{269})^{4} \cdot \pi^{0}$.
\zadStop
\rozwStart{Patryk Wirkus}{Martyna Czarnobaj}
$$(\frac{269}{137})^{4} \cdot (\frac{137}{269})^{4} \cdot \pi^{0} = (\frac{269}{137} \cdot \frac{137}{269})^{4} \cdot 1 = 1^{4} \cdot 1 = 1$$
\rozwStop
\odpStart
$1$
\odpStop
\testStart
A.$1$ B.$\pi$ C.$0$ D.$\frac{269}{137}$ E.$\frac{137}{269}$
F.$-\frac{269}{137}$ G.$-1$
H.$(\frac{269}{137})^{4}$
I.$(\frac{137}{269})^{4}$
\testStop
\kluczStart
A
\kluczStop



\zadStart{Zadanie z Wikieł Z 1.1 d) moja wersja nr 296}

Obliczyć wartość wyrażenia $(\frac{269}{139})^{4} \cdot (\frac{139}{269})^{4} \cdot \pi^{0}$.
\zadStop
\rozwStart{Patryk Wirkus}{Martyna Czarnobaj}
$$(\frac{269}{139})^{4} \cdot (\frac{139}{269})^{4} \cdot \pi^{0} = (\frac{269}{139} \cdot \frac{139}{269})^{4} \cdot 1 = 1^{4} \cdot 1 = 1$$
\rozwStop
\odpStart
$1$
\odpStop
\testStart
A.$1$ B.$\pi$ C.$0$ D.$\frac{269}{139}$ E.$\frac{139}{269}$
F.$-\frac{269}{139}$ G.$-1$
H.$(\frac{269}{139})^{4}$
I.$(\frac{139}{269})^{4}$
\testStop
\kluczStart
A
\kluczStop



\zadStart{Zadanie z Wikieł Z 1.1 d) moja wersja nr 297}

Obliczyć wartość wyrażenia $(\frac{271}{103})^{4} \cdot (\frac{103}{271})^{4} \cdot \pi^{0}$.
\zadStop
\rozwStart{Patryk Wirkus}{Martyna Czarnobaj}
$$(\frac{271}{103})^{4} \cdot (\frac{103}{271})^{4} \cdot \pi^{0} = (\frac{271}{103} \cdot \frac{103}{271})^{4} \cdot 1 = 1^{4} \cdot 1 = 1$$
\rozwStop
\odpStart
$1$
\odpStop
\testStart
A.$1$ B.$\pi$ C.$0$ D.$\frac{271}{103}$ E.$\frac{103}{271}$
F.$-\frac{271}{103}$ G.$-1$
H.$(\frac{271}{103})^{4}$
I.$(\frac{103}{271})^{4}$
\testStop
\kluczStart
A
\kluczStop



\zadStart{Zadanie z Wikieł Z 1.1 d) moja wersja nr 298}

Obliczyć wartość wyrażenia $(\frac{271}{107})^{4} \cdot (\frac{107}{271})^{4} \cdot \pi^{0}$.
\zadStop
\rozwStart{Patryk Wirkus}{Martyna Czarnobaj}
$$(\frac{271}{107})^{4} \cdot (\frac{107}{271})^{4} \cdot \pi^{0} = (\frac{271}{107} \cdot \frac{107}{271})^{4} \cdot 1 = 1^{4} \cdot 1 = 1$$
\rozwStop
\odpStart
$1$
\odpStop
\testStart
A.$1$ B.$\pi$ C.$0$ D.$\frac{271}{107}$ E.$\frac{107}{271}$
F.$-\frac{271}{107}$ G.$-1$
H.$(\frac{271}{107})^{4}$
I.$(\frac{107}{271})^{4}$
\testStop
\kluczStart
A
\kluczStop



\zadStart{Zadanie z Wikieł Z 1.1 d) moja wersja nr 299}

Obliczyć wartość wyrażenia $(\frac{271}{109})^{4} \cdot (\frac{109}{271})^{4} \cdot \pi^{0}$.
\zadStop
\rozwStart{Patryk Wirkus}{Martyna Czarnobaj}
$$(\frac{271}{109})^{4} \cdot (\frac{109}{271})^{4} \cdot \pi^{0} = (\frac{271}{109} \cdot \frac{109}{271})^{4} \cdot 1 = 1^{4} \cdot 1 = 1$$
\rozwStop
\odpStart
$1$
\odpStop
\testStart
A.$1$ B.$\pi$ C.$0$ D.$\frac{271}{109}$ E.$\frac{109}{271}$
F.$-\frac{271}{109}$ G.$-1$
H.$(\frac{271}{109})^{4}$
I.$(\frac{109}{271})^{4}$
\testStop
\kluczStart
A
\kluczStop



\zadStart{Zadanie z Wikieł Z 1.1 d) moja wersja nr 300}

Obliczyć wartość wyrażenia $(\frac{271}{113})^{4} \cdot (\frac{113}{271})^{4} \cdot \pi^{0}$.
\zadStop
\rozwStart{Patryk Wirkus}{Martyna Czarnobaj}
$$(\frac{271}{113})^{4} \cdot (\frac{113}{271})^{4} \cdot \pi^{0} = (\frac{271}{113} \cdot \frac{113}{271})^{4} \cdot 1 = 1^{4} \cdot 1 = 1$$
\rozwStop
\odpStart
$1$
\odpStop
\testStart
A.$1$ B.$\pi$ C.$0$ D.$\frac{271}{113}$ E.$\frac{113}{271}$
F.$-\frac{271}{113}$ G.$-1$
H.$(\frac{271}{113})^{4}$
I.$(\frac{113}{271})^{4}$
\testStop
\kluczStart
A
\kluczStop



\zadStart{Zadanie z Wikieł Z 1.1 d) moja wersja nr 301}

Obliczyć wartość wyrażenia $(\frac{271}{127})^{4} \cdot (\frac{127}{271})^{4} \cdot \pi^{0}$.
\zadStop
\rozwStart{Patryk Wirkus}{Martyna Czarnobaj}
$$(\frac{271}{127})^{4} \cdot (\frac{127}{271})^{4} \cdot \pi^{0} = (\frac{271}{127} \cdot \frac{127}{271})^{4} \cdot 1 = 1^{4} \cdot 1 = 1$$
\rozwStop
\odpStart
$1$
\odpStop
\testStart
A.$1$ B.$\pi$ C.$0$ D.$\frac{271}{127}$ E.$\frac{127}{271}$
F.$-\frac{271}{127}$ G.$-1$
H.$(\frac{271}{127})^{4}$
I.$(\frac{127}{271})^{4}$
\testStop
\kluczStart
A
\kluczStop



\zadStart{Zadanie z Wikieł Z 1.1 d) moja wersja nr 302}

Obliczyć wartość wyrażenia $(\frac{271}{131})^{4} \cdot (\frac{131}{271})^{4} \cdot \pi^{0}$.
\zadStop
\rozwStart{Patryk Wirkus}{Martyna Czarnobaj}
$$(\frac{271}{131})^{4} \cdot (\frac{131}{271})^{4} \cdot \pi^{0} = (\frac{271}{131} \cdot \frac{131}{271})^{4} \cdot 1 = 1^{4} \cdot 1 = 1$$
\rozwStop
\odpStart
$1$
\odpStop
\testStart
A.$1$ B.$\pi$ C.$0$ D.$\frac{271}{131}$ E.$\frac{131}{271}$
F.$-\frac{271}{131}$ G.$-1$
H.$(\frac{271}{131})^{4}$
I.$(\frac{131}{271})^{4}$
\testStop
\kluczStart
A
\kluczStop



\zadStart{Zadanie z Wikieł Z 1.1 d) moja wersja nr 303}

Obliczyć wartość wyrażenia $(\frac{271}{137})^{4} \cdot (\frac{137}{271})^{4} \cdot \pi^{0}$.
\zadStop
\rozwStart{Patryk Wirkus}{Martyna Czarnobaj}
$$(\frac{271}{137})^{4} \cdot (\frac{137}{271})^{4} \cdot \pi^{0} = (\frac{271}{137} \cdot \frac{137}{271})^{4} \cdot 1 = 1^{4} \cdot 1 = 1$$
\rozwStop
\odpStart
$1$
\odpStop
\testStart
A.$1$ B.$\pi$ C.$0$ D.$\frac{271}{137}$ E.$\frac{137}{271}$
F.$-\frac{271}{137}$ G.$-1$
H.$(\frac{271}{137})^{4}$
I.$(\frac{137}{271})^{4}$
\testStop
\kluczStart
A
\kluczStop



\zadStart{Zadanie z Wikieł Z 1.1 d) moja wersja nr 304}

Obliczyć wartość wyrażenia $(\frac{271}{139})^{4} \cdot (\frac{139}{271})^{4} \cdot \pi^{0}$.
\zadStop
\rozwStart{Patryk Wirkus}{Martyna Czarnobaj}
$$(\frac{271}{139})^{4} \cdot (\frac{139}{271})^{4} \cdot \pi^{0} = (\frac{271}{139} \cdot \frac{139}{271})^{4} \cdot 1 = 1^{4} \cdot 1 = 1$$
\rozwStop
\odpStart
$1$
\odpStop
\testStart
A.$1$ B.$\pi$ C.$0$ D.$\frac{271}{139}$ E.$\frac{139}{271}$
F.$-\frac{271}{139}$ G.$-1$
H.$(\frac{271}{139})^{4}$
I.$(\frac{139}{271})^{4}$
\testStop
\kluczStart
A
\kluczStop



\zadStart{Zadanie z Wikieł Z 1.1 d) moja wersja nr 305}

Obliczyć wartość wyrażenia $(\frac{277}{103})^{4} \cdot (\frac{103}{277})^{4} \cdot \pi^{0}$.
\zadStop
\rozwStart{Patryk Wirkus}{Martyna Czarnobaj}
$$(\frac{277}{103})^{4} \cdot (\frac{103}{277})^{4} \cdot \pi^{0} = (\frac{277}{103} \cdot \frac{103}{277})^{4} \cdot 1 = 1^{4} \cdot 1 = 1$$
\rozwStop
\odpStart
$1$
\odpStop
\testStart
A.$1$ B.$\pi$ C.$0$ D.$\frac{277}{103}$ E.$\frac{103}{277}$
F.$-\frac{277}{103}$ G.$-1$
H.$(\frac{277}{103})^{4}$
I.$(\frac{103}{277})^{4}$
\testStop
\kluczStart
A
\kluczStop



\zadStart{Zadanie z Wikieł Z 1.1 d) moja wersja nr 306}

Obliczyć wartość wyrażenia $(\frac{277}{107})^{4} \cdot (\frac{107}{277})^{4} \cdot \pi^{0}$.
\zadStop
\rozwStart{Patryk Wirkus}{Martyna Czarnobaj}
$$(\frac{277}{107})^{4} \cdot (\frac{107}{277})^{4} \cdot \pi^{0} = (\frac{277}{107} \cdot \frac{107}{277})^{4} \cdot 1 = 1^{4} \cdot 1 = 1$$
\rozwStop
\odpStart
$1$
\odpStop
\testStart
A.$1$ B.$\pi$ C.$0$ D.$\frac{277}{107}$ E.$\frac{107}{277}$
F.$-\frac{277}{107}$ G.$-1$
H.$(\frac{277}{107})^{4}$
I.$(\frac{107}{277})^{4}$
\testStop
\kluczStart
A
\kluczStop



\zadStart{Zadanie z Wikieł Z 1.1 d) moja wersja nr 307}

Obliczyć wartość wyrażenia $(\frac{277}{109})^{4} \cdot (\frac{109}{277})^{4} \cdot \pi^{0}$.
\zadStop
\rozwStart{Patryk Wirkus}{Martyna Czarnobaj}
$$(\frac{277}{109})^{4} \cdot (\frac{109}{277})^{4} \cdot \pi^{0} = (\frac{277}{109} \cdot \frac{109}{277})^{4} \cdot 1 = 1^{4} \cdot 1 = 1$$
\rozwStop
\odpStart
$1$
\odpStop
\testStart
A.$1$ B.$\pi$ C.$0$ D.$\frac{277}{109}$ E.$\frac{109}{277}$
F.$-\frac{277}{109}$ G.$-1$
H.$(\frac{277}{109})^{4}$
I.$(\frac{109}{277})^{4}$
\testStop
\kluczStart
A
\kluczStop



\zadStart{Zadanie z Wikieł Z 1.1 d) moja wersja nr 308}

Obliczyć wartość wyrażenia $(\frac{277}{113})^{4} \cdot (\frac{113}{277})^{4} \cdot \pi^{0}$.
\zadStop
\rozwStart{Patryk Wirkus}{Martyna Czarnobaj}
$$(\frac{277}{113})^{4} \cdot (\frac{113}{277})^{4} \cdot \pi^{0} = (\frac{277}{113} \cdot \frac{113}{277})^{4} \cdot 1 = 1^{4} \cdot 1 = 1$$
\rozwStop
\odpStart
$1$
\odpStop
\testStart
A.$1$ B.$\pi$ C.$0$ D.$\frac{277}{113}$ E.$\frac{113}{277}$
F.$-\frac{277}{113}$ G.$-1$
H.$(\frac{277}{113})^{4}$
I.$(\frac{113}{277})^{4}$
\testStop
\kluczStart
A
\kluczStop



\zadStart{Zadanie z Wikieł Z 1.1 d) moja wersja nr 309}

Obliczyć wartość wyrażenia $(\frac{277}{127})^{4} \cdot (\frac{127}{277})^{4} \cdot \pi^{0}$.
\zadStop
\rozwStart{Patryk Wirkus}{Martyna Czarnobaj}
$$(\frac{277}{127})^{4} \cdot (\frac{127}{277})^{4} \cdot \pi^{0} = (\frac{277}{127} \cdot \frac{127}{277})^{4} \cdot 1 = 1^{4} \cdot 1 = 1$$
\rozwStop
\odpStart
$1$
\odpStop
\testStart
A.$1$ B.$\pi$ C.$0$ D.$\frac{277}{127}$ E.$\frac{127}{277}$
F.$-\frac{277}{127}$ G.$-1$
H.$(\frac{277}{127})^{4}$
I.$(\frac{127}{277})^{4}$
\testStop
\kluczStart
A
\kluczStop



\zadStart{Zadanie z Wikieł Z 1.1 d) moja wersja nr 310}

Obliczyć wartość wyrażenia $(\frac{277}{131})^{4} \cdot (\frac{131}{277})^{4} \cdot \pi^{0}$.
\zadStop
\rozwStart{Patryk Wirkus}{Martyna Czarnobaj}
$$(\frac{277}{131})^{4} \cdot (\frac{131}{277})^{4} \cdot \pi^{0} = (\frac{277}{131} \cdot \frac{131}{277})^{4} \cdot 1 = 1^{4} \cdot 1 = 1$$
\rozwStop
\odpStart
$1$
\odpStop
\testStart
A.$1$ B.$\pi$ C.$0$ D.$\frac{277}{131}$ E.$\frac{131}{277}$
F.$-\frac{277}{131}$ G.$-1$
H.$(\frac{277}{131})^{4}$
I.$(\frac{131}{277})^{4}$
\testStop
\kluczStart
A
\kluczStop



\zadStart{Zadanie z Wikieł Z 1.1 d) moja wersja nr 311}

Obliczyć wartość wyrażenia $(\frac{277}{137})^{4} \cdot (\frac{137}{277})^{4} \cdot \pi^{0}$.
\zadStop
\rozwStart{Patryk Wirkus}{Martyna Czarnobaj}
$$(\frac{277}{137})^{4} \cdot (\frac{137}{277})^{4} \cdot \pi^{0} = (\frac{277}{137} \cdot \frac{137}{277})^{4} \cdot 1 = 1^{4} \cdot 1 = 1$$
\rozwStop
\odpStart
$1$
\odpStop
\testStart
A.$1$ B.$\pi$ C.$0$ D.$\frac{277}{137}$ E.$\frac{137}{277}$
F.$-\frac{277}{137}$ G.$-1$
H.$(\frac{277}{137})^{4}$
I.$(\frac{137}{277})^{4}$
\testStop
\kluczStart
A
\kluczStop



\zadStart{Zadanie z Wikieł Z 1.1 d) moja wersja nr 312}

Obliczyć wartość wyrażenia $(\frac{277}{139})^{4} \cdot (\frac{139}{277})^{4} \cdot \pi^{0}$.
\zadStop
\rozwStart{Patryk Wirkus}{Martyna Czarnobaj}
$$(\frac{277}{139})^{4} \cdot (\frac{139}{277})^{4} \cdot \pi^{0} = (\frac{277}{139} \cdot \frac{139}{277})^{4} \cdot 1 = 1^{4} \cdot 1 = 1$$
\rozwStop
\odpStart
$1$
\odpStop
\testStart
A.$1$ B.$\pi$ C.$0$ D.$\frac{277}{139}$ E.$\frac{139}{277}$
F.$-\frac{277}{139}$ G.$-1$
H.$(\frac{277}{139})^{4}$
I.$(\frac{139}{277})^{4}$
\testStop
\kluczStart
A
\kluczStop



\zadStart{Zadanie z Wikieł Z 1.1 d) moja wersja nr 313}

Obliczyć wartość wyrażenia $(\frac{149}{103})^{5} \cdot (\frac{103}{149})^{5} \cdot \pi^{0}$.
\zadStop
\rozwStart{Patryk Wirkus}{Martyna Czarnobaj}
$$(\frac{149}{103})^{5} \cdot (\frac{103}{149})^{5} \cdot \pi^{0} = (\frac{149}{103} \cdot \frac{103}{149})^{5} \cdot 1 = 1^{5} \cdot 1 = 1$$
\rozwStop
\odpStart
$1$
\odpStop
\testStart
A.$1$ B.$\pi$ C.$0$ D.$\frac{149}{103}$ E.$\frac{103}{149}$
F.$-\frac{149}{103}$ G.$-1$
H.$(\frac{149}{103})^{5}$
I.$(\frac{103}{149})^{5}$
\testStop
\kluczStart
A
\kluczStop



\zadStart{Zadanie z Wikieł Z 1.1 d) moja wersja nr 314}

Obliczyć wartość wyrażenia $(\frac{149}{107})^{5} \cdot (\frac{107}{149})^{5} \cdot \pi^{0}$.
\zadStop
\rozwStart{Patryk Wirkus}{Martyna Czarnobaj}
$$(\frac{149}{107})^{5} \cdot (\frac{107}{149})^{5} \cdot \pi^{0} = (\frac{149}{107} \cdot \frac{107}{149})^{5} \cdot 1 = 1^{5} \cdot 1 = 1$$
\rozwStop
\odpStart
$1$
\odpStop
\testStart
A.$1$ B.$\pi$ C.$0$ D.$\frac{149}{107}$ E.$\frac{107}{149}$
F.$-\frac{149}{107}$ G.$-1$
H.$(\frac{149}{107})^{5}$
I.$(\frac{107}{149})^{5}$
\testStop
\kluczStart
A
\kluczStop



\zadStart{Zadanie z Wikieł Z 1.1 d) moja wersja nr 315}

Obliczyć wartość wyrażenia $(\frac{149}{109})^{5} \cdot (\frac{109}{149})^{5} \cdot \pi^{0}$.
\zadStop
\rozwStart{Patryk Wirkus}{Martyna Czarnobaj}
$$(\frac{149}{109})^{5} \cdot (\frac{109}{149})^{5} \cdot \pi^{0} = (\frac{149}{109} \cdot \frac{109}{149})^{5} \cdot 1 = 1^{5} \cdot 1 = 1$$
\rozwStop
\odpStart
$1$
\odpStop
\testStart
A.$1$ B.$\pi$ C.$0$ D.$\frac{149}{109}$ E.$\frac{109}{149}$
F.$-\frac{149}{109}$ G.$-1$
H.$(\frac{149}{109})^{5}$
I.$(\frac{109}{149})^{5}$
\testStop
\kluczStart
A
\kluczStop



\zadStart{Zadanie z Wikieł Z 1.1 d) moja wersja nr 316}

Obliczyć wartość wyrażenia $(\frac{149}{113})^{5} \cdot (\frac{113}{149})^{5} \cdot \pi^{0}$.
\zadStop
\rozwStart{Patryk Wirkus}{Martyna Czarnobaj}
$$(\frac{149}{113})^{5} \cdot (\frac{113}{149})^{5} \cdot \pi^{0} = (\frac{149}{113} \cdot \frac{113}{149})^{5} \cdot 1 = 1^{5} \cdot 1 = 1$$
\rozwStop
\odpStart
$1$
\odpStop
\testStart
A.$1$ B.$\pi$ C.$0$ D.$\frac{149}{113}$ E.$\frac{113}{149}$
F.$-\frac{149}{113}$ G.$-1$
H.$(\frac{149}{113})^{5}$
I.$(\frac{113}{149})^{5}$
\testStop
\kluczStart
A
\kluczStop



\zadStart{Zadanie z Wikieł Z 1.1 d) moja wersja nr 317}

Obliczyć wartość wyrażenia $(\frac{149}{127})^{5} \cdot (\frac{127}{149})^{5} \cdot \pi^{0}$.
\zadStop
\rozwStart{Patryk Wirkus}{Martyna Czarnobaj}
$$(\frac{149}{127})^{5} \cdot (\frac{127}{149})^{5} \cdot \pi^{0} = (\frac{149}{127} \cdot \frac{127}{149})^{5} \cdot 1 = 1^{5} \cdot 1 = 1$$
\rozwStop
\odpStart
$1$
\odpStop
\testStart
A.$1$ B.$\pi$ C.$0$ D.$\frac{149}{127}$ E.$\frac{127}{149}$
F.$-\frac{149}{127}$ G.$-1$
H.$(\frac{149}{127})^{5}$
I.$(\frac{127}{149})^{5}$
\testStop
\kluczStart
A
\kluczStop



\zadStart{Zadanie z Wikieł Z 1.1 d) moja wersja nr 318}

Obliczyć wartość wyrażenia $(\frac{149}{131})^{5} \cdot (\frac{131}{149})^{5} \cdot \pi^{0}$.
\zadStop
\rozwStart{Patryk Wirkus}{Martyna Czarnobaj}
$$(\frac{149}{131})^{5} \cdot (\frac{131}{149})^{5} \cdot \pi^{0} = (\frac{149}{131} \cdot \frac{131}{149})^{5} \cdot 1 = 1^{5} \cdot 1 = 1$$
\rozwStop
\odpStart
$1$
\odpStop
\testStart
A.$1$ B.$\pi$ C.$0$ D.$\frac{149}{131}$ E.$\frac{131}{149}$
F.$-\frac{149}{131}$ G.$-1$
H.$(\frac{149}{131})^{5}$
I.$(\frac{131}{149})^{5}$
\testStop
\kluczStart
A
\kluczStop



\zadStart{Zadanie z Wikieł Z 1.1 d) moja wersja nr 319}

Obliczyć wartość wyrażenia $(\frac{149}{137})^{5} \cdot (\frac{137}{149})^{5} \cdot \pi^{0}$.
\zadStop
\rozwStart{Patryk Wirkus}{Martyna Czarnobaj}
$$(\frac{149}{137})^{5} \cdot (\frac{137}{149})^{5} \cdot \pi^{0} = (\frac{149}{137} \cdot \frac{137}{149})^{5} \cdot 1 = 1^{5} \cdot 1 = 1$$
\rozwStop
\odpStart
$1$
\odpStop
\testStart
A.$1$ B.$\pi$ C.$0$ D.$\frac{149}{137}$ E.$\frac{137}{149}$
F.$-\frac{149}{137}$ G.$-1$
H.$(\frac{149}{137})^{5}$
I.$(\frac{137}{149})^{5}$
\testStop
\kluczStart
A
\kluczStop



\zadStart{Zadanie z Wikieł Z 1.1 d) moja wersja nr 320}

Obliczyć wartość wyrażenia $(\frac{149}{139})^{5} \cdot (\frac{139}{149})^{5} \cdot \pi^{0}$.
\zadStop
\rozwStart{Patryk Wirkus}{Martyna Czarnobaj}
$$(\frac{149}{139})^{5} \cdot (\frac{139}{149})^{5} \cdot \pi^{0} = (\frac{149}{139} \cdot \frac{139}{149})^{5} \cdot 1 = 1^{5} \cdot 1 = 1$$
\rozwStop
\odpStart
$1$
\odpStop
\testStart
A.$1$ B.$\pi$ C.$0$ D.$\frac{149}{139}$ E.$\frac{139}{149}$
F.$-\frac{149}{139}$ G.$-1$
H.$(\frac{149}{139})^{5}$
I.$(\frac{139}{149})^{5}$
\testStop
\kluczStart
A
\kluczStop



\zadStart{Zadanie z Wikieł Z 1.1 d) moja wersja nr 321}

Obliczyć wartość wyrażenia $(\frac{151}{103})^{5} \cdot (\frac{103}{151})^{5} \cdot \pi^{0}$.
\zadStop
\rozwStart{Patryk Wirkus}{Martyna Czarnobaj}
$$(\frac{151}{103})^{5} \cdot (\frac{103}{151})^{5} \cdot \pi^{0} = (\frac{151}{103} \cdot \frac{103}{151})^{5} \cdot 1 = 1^{5} \cdot 1 = 1$$
\rozwStop
\odpStart
$1$
\odpStop
\testStart
A.$1$ B.$\pi$ C.$0$ D.$\frac{151}{103}$ E.$\frac{103}{151}$
F.$-\frac{151}{103}$ G.$-1$
H.$(\frac{151}{103})^{5}$
I.$(\frac{103}{151})^{5}$
\testStop
\kluczStart
A
\kluczStop



\zadStart{Zadanie z Wikieł Z 1.1 d) moja wersja nr 322}

Obliczyć wartość wyrażenia $(\frac{151}{107})^{5} \cdot (\frac{107}{151})^{5} \cdot \pi^{0}$.
\zadStop
\rozwStart{Patryk Wirkus}{Martyna Czarnobaj}
$$(\frac{151}{107})^{5} \cdot (\frac{107}{151})^{5} \cdot \pi^{0} = (\frac{151}{107} \cdot \frac{107}{151})^{5} \cdot 1 = 1^{5} \cdot 1 = 1$$
\rozwStop
\odpStart
$1$
\odpStop
\testStart
A.$1$ B.$\pi$ C.$0$ D.$\frac{151}{107}$ E.$\frac{107}{151}$
F.$-\frac{151}{107}$ G.$-1$
H.$(\frac{151}{107})^{5}$
I.$(\frac{107}{151})^{5}$
\testStop
\kluczStart
A
\kluczStop



\zadStart{Zadanie z Wikieł Z 1.1 d) moja wersja nr 323}

Obliczyć wartość wyrażenia $(\frac{151}{109})^{5} \cdot (\frac{109}{151})^{5} \cdot \pi^{0}$.
\zadStop
\rozwStart{Patryk Wirkus}{Martyna Czarnobaj}
$$(\frac{151}{109})^{5} \cdot (\frac{109}{151})^{5} \cdot \pi^{0} = (\frac{151}{109} \cdot \frac{109}{151})^{5} \cdot 1 = 1^{5} \cdot 1 = 1$$
\rozwStop
\odpStart
$1$
\odpStop
\testStart
A.$1$ B.$\pi$ C.$0$ D.$\frac{151}{109}$ E.$\frac{109}{151}$
F.$-\frac{151}{109}$ G.$-1$
H.$(\frac{151}{109})^{5}$
I.$(\frac{109}{151})^{5}$
\testStop
\kluczStart
A
\kluczStop



\zadStart{Zadanie z Wikieł Z 1.1 d) moja wersja nr 324}

Obliczyć wartość wyrażenia $(\frac{151}{113})^{5} \cdot (\frac{113}{151})^{5} \cdot \pi^{0}$.
\zadStop
\rozwStart{Patryk Wirkus}{Martyna Czarnobaj}
$$(\frac{151}{113})^{5} \cdot (\frac{113}{151})^{5} \cdot \pi^{0} = (\frac{151}{113} \cdot \frac{113}{151})^{5} \cdot 1 = 1^{5} \cdot 1 = 1$$
\rozwStop
\odpStart
$1$
\odpStop
\testStart
A.$1$ B.$\pi$ C.$0$ D.$\frac{151}{113}$ E.$\frac{113}{151}$
F.$-\frac{151}{113}$ G.$-1$
H.$(\frac{151}{113})^{5}$
I.$(\frac{113}{151})^{5}$
\testStop
\kluczStart
A
\kluczStop



\zadStart{Zadanie z Wikieł Z 1.1 d) moja wersja nr 325}

Obliczyć wartość wyrażenia $(\frac{151}{127})^{5} \cdot (\frac{127}{151})^{5} \cdot \pi^{0}$.
\zadStop
\rozwStart{Patryk Wirkus}{Martyna Czarnobaj}
$$(\frac{151}{127})^{5} \cdot (\frac{127}{151})^{5} \cdot \pi^{0} = (\frac{151}{127} \cdot \frac{127}{151})^{5} \cdot 1 = 1^{5} \cdot 1 = 1$$
\rozwStop
\odpStart
$1$
\odpStop
\testStart
A.$1$ B.$\pi$ C.$0$ D.$\frac{151}{127}$ E.$\frac{127}{151}$
F.$-\frac{151}{127}$ G.$-1$
H.$(\frac{151}{127})^{5}$
I.$(\frac{127}{151})^{5}$
\testStop
\kluczStart
A
\kluczStop



\zadStart{Zadanie z Wikieł Z 1.1 d) moja wersja nr 326}

Obliczyć wartość wyrażenia $(\frac{151}{131})^{5} \cdot (\frac{131}{151})^{5} \cdot \pi^{0}$.
\zadStop
\rozwStart{Patryk Wirkus}{Martyna Czarnobaj}
$$(\frac{151}{131})^{5} \cdot (\frac{131}{151})^{5} \cdot \pi^{0} = (\frac{151}{131} \cdot \frac{131}{151})^{5} \cdot 1 = 1^{5} \cdot 1 = 1$$
\rozwStop
\odpStart
$1$
\odpStop
\testStart
A.$1$ B.$\pi$ C.$0$ D.$\frac{151}{131}$ E.$\frac{131}{151}$
F.$-\frac{151}{131}$ G.$-1$
H.$(\frac{151}{131})^{5}$
I.$(\frac{131}{151})^{5}$
\testStop
\kluczStart
A
\kluczStop



\zadStart{Zadanie z Wikieł Z 1.1 d) moja wersja nr 327}

Obliczyć wartość wyrażenia $(\frac{151}{137})^{5} \cdot (\frac{137}{151})^{5} \cdot \pi^{0}$.
\zadStop
\rozwStart{Patryk Wirkus}{Martyna Czarnobaj}
$$(\frac{151}{137})^{5} \cdot (\frac{137}{151})^{5} \cdot \pi^{0} = (\frac{151}{137} \cdot \frac{137}{151})^{5} \cdot 1 = 1^{5} \cdot 1 = 1$$
\rozwStop
\odpStart
$1$
\odpStop
\testStart
A.$1$ B.$\pi$ C.$0$ D.$\frac{151}{137}$ E.$\frac{137}{151}$
F.$-\frac{151}{137}$ G.$-1$
H.$(\frac{151}{137})^{5}$
I.$(\frac{137}{151})^{5}$
\testStop
\kluczStart
A
\kluczStop



\zadStart{Zadanie z Wikieł Z 1.1 d) moja wersja nr 328}

Obliczyć wartość wyrażenia $(\frac{151}{139})^{5} \cdot (\frac{139}{151})^{5} \cdot \pi^{0}$.
\zadStop
\rozwStart{Patryk Wirkus}{Martyna Czarnobaj}
$$(\frac{151}{139})^{5} \cdot (\frac{139}{151})^{5} \cdot \pi^{0} = (\frac{151}{139} \cdot \frac{139}{151})^{5} \cdot 1 = 1^{5} \cdot 1 = 1$$
\rozwStop
\odpStart
$1$
\odpStop
\testStart
A.$1$ B.$\pi$ C.$0$ D.$\frac{151}{139}$ E.$\frac{139}{151}$
F.$-\frac{151}{139}$ G.$-1$
H.$(\frac{151}{139})^{5}$
I.$(\frac{139}{151})^{5}$
\testStop
\kluczStart
A
\kluczStop



\zadStart{Zadanie z Wikieł Z 1.1 d) moja wersja nr 329}

Obliczyć wartość wyrażenia $(\frac{157}{103})^{5} \cdot (\frac{103}{157})^{5} \cdot \pi^{0}$.
\zadStop
\rozwStart{Patryk Wirkus}{Martyna Czarnobaj}
$$(\frac{157}{103})^{5} \cdot (\frac{103}{157})^{5} \cdot \pi^{0} = (\frac{157}{103} \cdot \frac{103}{157})^{5} \cdot 1 = 1^{5} \cdot 1 = 1$$
\rozwStop
\odpStart
$1$
\odpStop
\testStart
A.$1$ B.$\pi$ C.$0$ D.$\frac{157}{103}$ E.$\frac{103}{157}$
F.$-\frac{157}{103}$ G.$-1$
H.$(\frac{157}{103})^{5}$
I.$(\frac{103}{157})^{5}$
\testStop
\kluczStart
A
\kluczStop



\zadStart{Zadanie z Wikieł Z 1.1 d) moja wersja nr 330}

Obliczyć wartość wyrażenia $(\frac{157}{107})^{5} \cdot (\frac{107}{157})^{5} \cdot \pi^{0}$.
\zadStop
\rozwStart{Patryk Wirkus}{Martyna Czarnobaj}
$$(\frac{157}{107})^{5} \cdot (\frac{107}{157})^{5} \cdot \pi^{0} = (\frac{157}{107} \cdot \frac{107}{157})^{5} \cdot 1 = 1^{5} \cdot 1 = 1$$
\rozwStop
\odpStart
$1$
\odpStop
\testStart
A.$1$ B.$\pi$ C.$0$ D.$\frac{157}{107}$ E.$\frac{107}{157}$
F.$-\frac{157}{107}$ G.$-1$
H.$(\frac{157}{107})^{5}$
I.$(\frac{107}{157})^{5}$
\testStop
\kluczStart
A
\kluczStop



\zadStart{Zadanie z Wikieł Z 1.1 d) moja wersja nr 331}

Obliczyć wartość wyrażenia $(\frac{157}{109})^{5} \cdot (\frac{109}{157})^{5} \cdot \pi^{0}$.
\zadStop
\rozwStart{Patryk Wirkus}{Martyna Czarnobaj}
$$(\frac{157}{109})^{5} \cdot (\frac{109}{157})^{5} \cdot \pi^{0} = (\frac{157}{109} \cdot \frac{109}{157})^{5} \cdot 1 = 1^{5} \cdot 1 = 1$$
\rozwStop
\odpStart
$1$
\odpStop
\testStart
A.$1$ B.$\pi$ C.$0$ D.$\frac{157}{109}$ E.$\frac{109}{157}$
F.$-\frac{157}{109}$ G.$-1$
H.$(\frac{157}{109})^{5}$
I.$(\frac{109}{157})^{5}$
\testStop
\kluczStart
A
\kluczStop



\zadStart{Zadanie z Wikieł Z 1.1 d) moja wersja nr 332}

Obliczyć wartość wyrażenia $(\frac{157}{113})^{5} \cdot (\frac{113}{157})^{5} \cdot \pi^{0}$.
\zadStop
\rozwStart{Patryk Wirkus}{Martyna Czarnobaj}
$$(\frac{157}{113})^{5} \cdot (\frac{113}{157})^{5} \cdot \pi^{0} = (\frac{157}{113} \cdot \frac{113}{157})^{5} \cdot 1 = 1^{5} \cdot 1 = 1$$
\rozwStop
\odpStart
$1$
\odpStop
\testStart
A.$1$ B.$\pi$ C.$0$ D.$\frac{157}{113}$ E.$\frac{113}{157}$
F.$-\frac{157}{113}$ G.$-1$
H.$(\frac{157}{113})^{5}$
I.$(\frac{113}{157})^{5}$
\testStop
\kluczStart
A
\kluczStop



\zadStart{Zadanie z Wikieł Z 1.1 d) moja wersja nr 333}

Obliczyć wartość wyrażenia $(\frac{157}{127})^{5} \cdot (\frac{127}{157})^{5} \cdot \pi^{0}$.
\zadStop
\rozwStart{Patryk Wirkus}{Martyna Czarnobaj}
$$(\frac{157}{127})^{5} \cdot (\frac{127}{157})^{5} \cdot \pi^{0} = (\frac{157}{127} \cdot \frac{127}{157})^{5} \cdot 1 = 1^{5} \cdot 1 = 1$$
\rozwStop
\odpStart
$1$
\odpStop
\testStart
A.$1$ B.$\pi$ C.$0$ D.$\frac{157}{127}$ E.$\frac{127}{157}$
F.$-\frac{157}{127}$ G.$-1$
H.$(\frac{157}{127})^{5}$
I.$(\frac{127}{157})^{5}$
\testStop
\kluczStart
A
\kluczStop



\zadStart{Zadanie z Wikieł Z 1.1 d) moja wersja nr 334}

Obliczyć wartość wyrażenia $(\frac{157}{131})^{5} \cdot (\frac{131}{157})^{5} \cdot \pi^{0}$.
\zadStop
\rozwStart{Patryk Wirkus}{Martyna Czarnobaj}
$$(\frac{157}{131})^{5} \cdot (\frac{131}{157})^{5} \cdot \pi^{0} = (\frac{157}{131} \cdot \frac{131}{157})^{5} \cdot 1 = 1^{5} \cdot 1 = 1$$
\rozwStop
\odpStart
$1$
\odpStop
\testStart
A.$1$ B.$\pi$ C.$0$ D.$\frac{157}{131}$ E.$\frac{131}{157}$
F.$-\frac{157}{131}$ G.$-1$
H.$(\frac{157}{131})^{5}$
I.$(\frac{131}{157})^{5}$
\testStop
\kluczStart
A
\kluczStop



\zadStart{Zadanie z Wikieł Z 1.1 d) moja wersja nr 335}

Obliczyć wartość wyrażenia $(\frac{157}{137})^{5} \cdot (\frac{137}{157})^{5} \cdot \pi^{0}$.
\zadStop
\rozwStart{Patryk Wirkus}{Martyna Czarnobaj}
$$(\frac{157}{137})^{5} \cdot (\frac{137}{157})^{5} \cdot \pi^{0} = (\frac{157}{137} \cdot \frac{137}{157})^{5} \cdot 1 = 1^{5} \cdot 1 = 1$$
\rozwStop
\odpStart
$1$
\odpStop
\testStart
A.$1$ B.$\pi$ C.$0$ D.$\frac{157}{137}$ E.$\frac{137}{157}$
F.$-\frac{157}{137}$ G.$-1$
H.$(\frac{157}{137})^{5}$
I.$(\frac{137}{157})^{5}$
\testStop
\kluczStart
A
\kluczStop



\zadStart{Zadanie z Wikieł Z 1.1 d) moja wersja nr 336}

Obliczyć wartość wyrażenia $(\frac{157}{139})^{5} \cdot (\frac{139}{157})^{5} \cdot \pi^{0}$.
\zadStop
\rozwStart{Patryk Wirkus}{Martyna Czarnobaj}
$$(\frac{157}{139})^{5} \cdot (\frac{139}{157})^{5} \cdot \pi^{0} = (\frac{157}{139} \cdot \frac{139}{157})^{5} \cdot 1 = 1^{5} \cdot 1 = 1$$
\rozwStop
\odpStart
$1$
\odpStop
\testStart
A.$1$ B.$\pi$ C.$0$ D.$\frac{157}{139}$ E.$\frac{139}{157}$
F.$-\frac{157}{139}$ G.$-1$
H.$(\frac{157}{139})^{5}$
I.$(\frac{139}{157})^{5}$
\testStop
\kluczStart
A
\kluczStop



\zadStart{Zadanie z Wikieł Z 1.1 d) moja wersja nr 337}

Obliczyć wartość wyrażenia $(\frac{163}{103})^{5} \cdot (\frac{103}{163})^{5} \cdot \pi^{0}$.
\zadStop
\rozwStart{Patryk Wirkus}{Martyna Czarnobaj}
$$(\frac{163}{103})^{5} \cdot (\frac{103}{163})^{5} \cdot \pi^{0} = (\frac{163}{103} \cdot \frac{103}{163})^{5} \cdot 1 = 1^{5} \cdot 1 = 1$$
\rozwStop
\odpStart
$1$
\odpStop
\testStart
A.$1$ B.$\pi$ C.$0$ D.$\frac{163}{103}$ E.$\frac{103}{163}$
F.$-\frac{163}{103}$ G.$-1$
H.$(\frac{163}{103})^{5}$
I.$(\frac{103}{163})^{5}$
\testStop
\kluczStart
A
\kluczStop



\zadStart{Zadanie z Wikieł Z 1.1 d) moja wersja nr 338}

Obliczyć wartość wyrażenia $(\frac{163}{107})^{5} \cdot (\frac{107}{163})^{5} \cdot \pi^{0}$.
\zadStop
\rozwStart{Patryk Wirkus}{Martyna Czarnobaj}
$$(\frac{163}{107})^{5} \cdot (\frac{107}{163})^{5} \cdot \pi^{0} = (\frac{163}{107} \cdot \frac{107}{163})^{5} \cdot 1 = 1^{5} \cdot 1 = 1$$
\rozwStop
\odpStart
$1$
\odpStop
\testStart
A.$1$ B.$\pi$ C.$0$ D.$\frac{163}{107}$ E.$\frac{107}{163}$
F.$-\frac{163}{107}$ G.$-1$
H.$(\frac{163}{107})^{5}$
I.$(\frac{107}{163})^{5}$
\testStop
\kluczStart
A
\kluczStop



\zadStart{Zadanie z Wikieł Z 1.1 d) moja wersja nr 339}

Obliczyć wartość wyrażenia $(\frac{163}{109})^{5} \cdot (\frac{109}{163})^{5} \cdot \pi^{0}$.
\zadStop
\rozwStart{Patryk Wirkus}{Martyna Czarnobaj}
$$(\frac{163}{109})^{5} \cdot (\frac{109}{163})^{5} \cdot \pi^{0} = (\frac{163}{109} \cdot \frac{109}{163})^{5} \cdot 1 = 1^{5} \cdot 1 = 1$$
\rozwStop
\odpStart
$1$
\odpStop
\testStart
A.$1$ B.$\pi$ C.$0$ D.$\frac{163}{109}$ E.$\frac{109}{163}$
F.$-\frac{163}{109}$ G.$-1$
H.$(\frac{163}{109})^{5}$
I.$(\frac{109}{163})^{5}$
\testStop
\kluczStart
A
\kluczStop



\zadStart{Zadanie z Wikieł Z 1.1 d) moja wersja nr 340}

Obliczyć wartość wyrażenia $(\frac{163}{113})^{5} \cdot (\frac{113}{163})^{5} \cdot \pi^{0}$.
\zadStop
\rozwStart{Patryk Wirkus}{Martyna Czarnobaj}
$$(\frac{163}{113})^{5} \cdot (\frac{113}{163})^{5} \cdot \pi^{0} = (\frac{163}{113} \cdot \frac{113}{163})^{5} \cdot 1 = 1^{5} \cdot 1 = 1$$
\rozwStop
\odpStart
$1$
\odpStop
\testStart
A.$1$ B.$\pi$ C.$0$ D.$\frac{163}{113}$ E.$\frac{113}{163}$
F.$-\frac{163}{113}$ G.$-1$
H.$(\frac{163}{113})^{5}$
I.$(\frac{113}{163})^{5}$
\testStop
\kluczStart
A
\kluczStop



\zadStart{Zadanie z Wikieł Z 1.1 d) moja wersja nr 341}

Obliczyć wartość wyrażenia $(\frac{163}{127})^{5} \cdot (\frac{127}{163})^{5} \cdot \pi^{0}$.
\zadStop
\rozwStart{Patryk Wirkus}{Martyna Czarnobaj}
$$(\frac{163}{127})^{5} \cdot (\frac{127}{163})^{5} \cdot \pi^{0} = (\frac{163}{127} \cdot \frac{127}{163})^{5} \cdot 1 = 1^{5} \cdot 1 = 1$$
\rozwStop
\odpStart
$1$
\odpStop
\testStart
A.$1$ B.$\pi$ C.$0$ D.$\frac{163}{127}$ E.$\frac{127}{163}$
F.$-\frac{163}{127}$ G.$-1$
H.$(\frac{163}{127})^{5}$
I.$(\frac{127}{163})^{5}$
\testStop
\kluczStart
A
\kluczStop



\zadStart{Zadanie z Wikieł Z 1.1 d) moja wersja nr 342}

Obliczyć wartość wyrażenia $(\frac{163}{131})^{5} \cdot (\frac{131}{163})^{5} \cdot \pi^{0}$.
\zadStop
\rozwStart{Patryk Wirkus}{Martyna Czarnobaj}
$$(\frac{163}{131})^{5} \cdot (\frac{131}{163})^{5} \cdot \pi^{0} = (\frac{163}{131} \cdot \frac{131}{163})^{5} \cdot 1 = 1^{5} \cdot 1 = 1$$
\rozwStop
\odpStart
$1$
\odpStop
\testStart
A.$1$ B.$\pi$ C.$0$ D.$\frac{163}{131}$ E.$\frac{131}{163}$
F.$-\frac{163}{131}$ G.$-1$
H.$(\frac{163}{131})^{5}$
I.$(\frac{131}{163})^{5}$
\testStop
\kluczStart
A
\kluczStop



\zadStart{Zadanie z Wikieł Z 1.1 d) moja wersja nr 343}

Obliczyć wartość wyrażenia $(\frac{163}{137})^{5} \cdot (\frac{137}{163})^{5} \cdot \pi^{0}$.
\zadStop
\rozwStart{Patryk Wirkus}{Martyna Czarnobaj}
$$(\frac{163}{137})^{5} \cdot (\frac{137}{163})^{5} \cdot \pi^{0} = (\frac{163}{137} \cdot \frac{137}{163})^{5} \cdot 1 = 1^{5} \cdot 1 = 1$$
\rozwStop
\odpStart
$1$
\odpStop
\testStart
A.$1$ B.$\pi$ C.$0$ D.$\frac{163}{137}$ E.$\frac{137}{163}$
F.$-\frac{163}{137}$ G.$-1$
H.$(\frac{163}{137})^{5}$
I.$(\frac{137}{163})^{5}$
\testStop
\kluczStart
A
\kluczStop



\zadStart{Zadanie z Wikieł Z 1.1 d) moja wersja nr 344}

Obliczyć wartość wyrażenia $(\frac{163}{139})^{5} \cdot (\frac{139}{163})^{5} \cdot \pi^{0}$.
\zadStop
\rozwStart{Patryk Wirkus}{Martyna Czarnobaj}
$$(\frac{163}{139})^{5} \cdot (\frac{139}{163})^{5} \cdot \pi^{0} = (\frac{163}{139} \cdot \frac{139}{163})^{5} \cdot 1 = 1^{5} \cdot 1 = 1$$
\rozwStop
\odpStart
$1$
\odpStop
\testStart
A.$1$ B.$\pi$ C.$0$ D.$\frac{163}{139}$ E.$\frac{139}{163}$
F.$-\frac{163}{139}$ G.$-1$
H.$(\frac{163}{139})^{5}$
I.$(\frac{139}{163})^{5}$
\testStop
\kluczStart
A
\kluczStop



\zadStart{Zadanie z Wikieł Z 1.1 d) moja wersja nr 345}

Obliczyć wartość wyrażenia $(\frac{167}{103})^{5} \cdot (\frac{103}{167})^{5} \cdot \pi^{0}$.
\zadStop
\rozwStart{Patryk Wirkus}{Martyna Czarnobaj}
$$(\frac{167}{103})^{5} \cdot (\frac{103}{167})^{5} \cdot \pi^{0} = (\frac{167}{103} \cdot \frac{103}{167})^{5} \cdot 1 = 1^{5} \cdot 1 = 1$$
\rozwStop
\odpStart
$1$
\odpStop
\testStart
A.$1$ B.$\pi$ C.$0$ D.$\frac{167}{103}$ E.$\frac{103}{167}$
F.$-\frac{167}{103}$ G.$-1$
H.$(\frac{167}{103})^{5}$
I.$(\frac{103}{167})^{5}$
\testStop
\kluczStart
A
\kluczStop



\zadStart{Zadanie z Wikieł Z 1.1 d) moja wersja nr 346}

Obliczyć wartość wyrażenia $(\frac{167}{107})^{5} \cdot (\frac{107}{167})^{5} \cdot \pi^{0}$.
\zadStop
\rozwStart{Patryk Wirkus}{Martyna Czarnobaj}
$$(\frac{167}{107})^{5} \cdot (\frac{107}{167})^{5} \cdot \pi^{0} = (\frac{167}{107} \cdot \frac{107}{167})^{5} \cdot 1 = 1^{5} \cdot 1 = 1$$
\rozwStop
\odpStart
$1$
\odpStop
\testStart
A.$1$ B.$\pi$ C.$0$ D.$\frac{167}{107}$ E.$\frac{107}{167}$
F.$-\frac{167}{107}$ G.$-1$
H.$(\frac{167}{107})^{5}$
I.$(\frac{107}{167})^{5}$
\testStop
\kluczStart
A
\kluczStop



\zadStart{Zadanie z Wikieł Z 1.1 d) moja wersja nr 347}

Obliczyć wartość wyrażenia $(\frac{167}{109})^{5} \cdot (\frac{109}{167})^{5} \cdot \pi^{0}$.
\zadStop
\rozwStart{Patryk Wirkus}{Martyna Czarnobaj}
$$(\frac{167}{109})^{5} \cdot (\frac{109}{167})^{5} \cdot \pi^{0} = (\frac{167}{109} \cdot \frac{109}{167})^{5} \cdot 1 = 1^{5} \cdot 1 = 1$$
\rozwStop
\odpStart
$1$
\odpStop
\testStart
A.$1$ B.$\pi$ C.$0$ D.$\frac{167}{109}$ E.$\frac{109}{167}$
F.$-\frac{167}{109}$ G.$-1$
H.$(\frac{167}{109})^{5}$
I.$(\frac{109}{167})^{5}$
\testStop
\kluczStart
A
\kluczStop



\zadStart{Zadanie z Wikieł Z 1.1 d) moja wersja nr 348}

Obliczyć wartość wyrażenia $(\frac{167}{113})^{5} \cdot (\frac{113}{167})^{5} \cdot \pi^{0}$.
\zadStop
\rozwStart{Patryk Wirkus}{Martyna Czarnobaj}
$$(\frac{167}{113})^{5} \cdot (\frac{113}{167})^{5} \cdot \pi^{0} = (\frac{167}{113} \cdot \frac{113}{167})^{5} \cdot 1 = 1^{5} \cdot 1 = 1$$
\rozwStop
\odpStart
$1$
\odpStop
\testStart
A.$1$ B.$\pi$ C.$0$ D.$\frac{167}{113}$ E.$\frac{113}{167}$
F.$-\frac{167}{113}$ G.$-1$
H.$(\frac{167}{113})^{5}$
I.$(\frac{113}{167})^{5}$
\testStop
\kluczStart
A
\kluczStop



\zadStart{Zadanie z Wikieł Z 1.1 d) moja wersja nr 349}

Obliczyć wartość wyrażenia $(\frac{167}{127})^{5} \cdot (\frac{127}{167})^{5} \cdot \pi^{0}$.
\zadStop
\rozwStart{Patryk Wirkus}{Martyna Czarnobaj}
$$(\frac{167}{127})^{5} \cdot (\frac{127}{167})^{5} \cdot \pi^{0} = (\frac{167}{127} \cdot \frac{127}{167})^{5} \cdot 1 = 1^{5} \cdot 1 = 1$$
\rozwStop
\odpStart
$1$
\odpStop
\testStart
A.$1$ B.$\pi$ C.$0$ D.$\frac{167}{127}$ E.$\frac{127}{167}$
F.$-\frac{167}{127}$ G.$-1$
H.$(\frac{167}{127})^{5}$
I.$(\frac{127}{167})^{5}$
\testStop
\kluczStart
A
\kluczStop



\zadStart{Zadanie z Wikieł Z 1.1 d) moja wersja nr 350}

Obliczyć wartość wyrażenia $(\frac{167}{131})^{5} \cdot (\frac{131}{167})^{5} \cdot \pi^{0}$.
\zadStop
\rozwStart{Patryk Wirkus}{Martyna Czarnobaj}
$$(\frac{167}{131})^{5} \cdot (\frac{131}{167})^{5} \cdot \pi^{0} = (\frac{167}{131} \cdot \frac{131}{167})^{5} \cdot 1 = 1^{5} \cdot 1 = 1$$
\rozwStop
\odpStart
$1$
\odpStop
\testStart
A.$1$ B.$\pi$ C.$0$ D.$\frac{167}{131}$ E.$\frac{131}{167}$
F.$-\frac{167}{131}$ G.$-1$
H.$(\frac{167}{131})^{5}$
I.$(\frac{131}{167})^{5}$
\testStop
\kluczStart
A
\kluczStop



\zadStart{Zadanie z Wikieł Z 1.1 d) moja wersja nr 351}

Obliczyć wartość wyrażenia $(\frac{167}{137})^{5} \cdot (\frac{137}{167})^{5} \cdot \pi^{0}$.
\zadStop
\rozwStart{Patryk Wirkus}{Martyna Czarnobaj}
$$(\frac{167}{137})^{5} \cdot (\frac{137}{167})^{5} \cdot \pi^{0} = (\frac{167}{137} \cdot \frac{137}{167})^{5} \cdot 1 = 1^{5} \cdot 1 = 1$$
\rozwStop
\odpStart
$1$
\odpStop
\testStart
A.$1$ B.$\pi$ C.$0$ D.$\frac{167}{137}$ E.$\frac{137}{167}$
F.$-\frac{167}{137}$ G.$-1$
H.$(\frac{167}{137})^{5}$
I.$(\frac{137}{167})^{5}$
\testStop
\kluczStart
A
\kluczStop



\zadStart{Zadanie z Wikieł Z 1.1 d) moja wersja nr 352}

Obliczyć wartość wyrażenia $(\frac{167}{139})^{5} \cdot (\frac{139}{167})^{5} \cdot \pi^{0}$.
\zadStop
\rozwStart{Patryk Wirkus}{Martyna Czarnobaj}
$$(\frac{167}{139})^{5} \cdot (\frac{139}{167})^{5} \cdot \pi^{0} = (\frac{167}{139} \cdot \frac{139}{167})^{5} \cdot 1 = 1^{5} \cdot 1 = 1$$
\rozwStop
\odpStart
$1$
\odpStop
\testStart
A.$1$ B.$\pi$ C.$0$ D.$\frac{167}{139}$ E.$\frac{139}{167}$
F.$-\frac{167}{139}$ G.$-1$
H.$(\frac{167}{139})^{5}$
I.$(\frac{139}{167})^{5}$
\testStop
\kluczStart
A
\kluczStop



\zadStart{Zadanie z Wikieł Z 1.1 d) moja wersja nr 353}

Obliczyć wartość wyrażenia $(\frac{173}{103})^{5} \cdot (\frac{103}{173})^{5} \cdot \pi^{0}$.
\zadStop
\rozwStart{Patryk Wirkus}{Martyna Czarnobaj}
$$(\frac{173}{103})^{5} \cdot (\frac{103}{173})^{5} \cdot \pi^{0} = (\frac{173}{103} \cdot \frac{103}{173})^{5} \cdot 1 = 1^{5} \cdot 1 = 1$$
\rozwStop
\odpStart
$1$
\odpStop
\testStart
A.$1$ B.$\pi$ C.$0$ D.$\frac{173}{103}$ E.$\frac{103}{173}$
F.$-\frac{173}{103}$ G.$-1$
H.$(\frac{173}{103})^{5}$
I.$(\frac{103}{173})^{5}$
\testStop
\kluczStart
A
\kluczStop



\zadStart{Zadanie z Wikieł Z 1.1 d) moja wersja nr 354}

Obliczyć wartość wyrażenia $(\frac{173}{107})^{5} \cdot (\frac{107}{173})^{5} \cdot \pi^{0}$.
\zadStop
\rozwStart{Patryk Wirkus}{Martyna Czarnobaj}
$$(\frac{173}{107})^{5} \cdot (\frac{107}{173})^{5} \cdot \pi^{0} = (\frac{173}{107} \cdot \frac{107}{173})^{5} \cdot 1 = 1^{5} \cdot 1 = 1$$
\rozwStop
\odpStart
$1$
\odpStop
\testStart
A.$1$ B.$\pi$ C.$0$ D.$\frac{173}{107}$ E.$\frac{107}{173}$
F.$-\frac{173}{107}$ G.$-1$
H.$(\frac{173}{107})^{5}$
I.$(\frac{107}{173})^{5}$
\testStop
\kluczStart
A
\kluczStop



\zadStart{Zadanie z Wikieł Z 1.1 d) moja wersja nr 355}

Obliczyć wartość wyrażenia $(\frac{173}{109})^{5} \cdot (\frac{109}{173})^{5} \cdot \pi^{0}$.
\zadStop
\rozwStart{Patryk Wirkus}{Martyna Czarnobaj}
$$(\frac{173}{109})^{5} \cdot (\frac{109}{173})^{5} \cdot \pi^{0} = (\frac{173}{109} \cdot \frac{109}{173})^{5} \cdot 1 = 1^{5} \cdot 1 = 1$$
\rozwStop
\odpStart
$1$
\odpStop
\testStart
A.$1$ B.$\pi$ C.$0$ D.$\frac{173}{109}$ E.$\frac{109}{173}$
F.$-\frac{173}{109}$ G.$-1$
H.$(\frac{173}{109})^{5}$
I.$(\frac{109}{173})^{5}$
\testStop
\kluczStart
A
\kluczStop



\zadStart{Zadanie z Wikieł Z 1.1 d) moja wersja nr 356}

Obliczyć wartość wyrażenia $(\frac{173}{113})^{5} \cdot (\frac{113}{173})^{5} \cdot \pi^{0}$.
\zadStop
\rozwStart{Patryk Wirkus}{Martyna Czarnobaj}
$$(\frac{173}{113})^{5} \cdot (\frac{113}{173})^{5} \cdot \pi^{0} = (\frac{173}{113} \cdot \frac{113}{173})^{5} \cdot 1 = 1^{5} \cdot 1 = 1$$
\rozwStop
\odpStart
$1$
\odpStop
\testStart
A.$1$ B.$\pi$ C.$0$ D.$\frac{173}{113}$ E.$\frac{113}{173}$
F.$-\frac{173}{113}$ G.$-1$
H.$(\frac{173}{113})^{5}$
I.$(\frac{113}{173})^{5}$
\testStop
\kluczStart
A
\kluczStop



\zadStart{Zadanie z Wikieł Z 1.1 d) moja wersja nr 357}

Obliczyć wartość wyrażenia $(\frac{173}{127})^{5} \cdot (\frac{127}{173})^{5} \cdot \pi^{0}$.
\zadStop
\rozwStart{Patryk Wirkus}{Martyna Czarnobaj}
$$(\frac{173}{127})^{5} \cdot (\frac{127}{173})^{5} \cdot \pi^{0} = (\frac{173}{127} \cdot \frac{127}{173})^{5} \cdot 1 = 1^{5} \cdot 1 = 1$$
\rozwStop
\odpStart
$1$
\odpStop
\testStart
A.$1$ B.$\pi$ C.$0$ D.$\frac{173}{127}$ E.$\frac{127}{173}$
F.$-\frac{173}{127}$ G.$-1$
H.$(\frac{173}{127})^{5}$
I.$(\frac{127}{173})^{5}$
\testStop
\kluczStart
A
\kluczStop



\zadStart{Zadanie z Wikieł Z 1.1 d) moja wersja nr 358}

Obliczyć wartość wyrażenia $(\frac{173}{131})^{5} \cdot (\frac{131}{173})^{5} \cdot \pi^{0}$.
\zadStop
\rozwStart{Patryk Wirkus}{Martyna Czarnobaj}
$$(\frac{173}{131})^{5} \cdot (\frac{131}{173})^{5} \cdot \pi^{0} = (\frac{173}{131} \cdot \frac{131}{173})^{5} \cdot 1 = 1^{5} \cdot 1 = 1$$
\rozwStop
\odpStart
$1$
\odpStop
\testStart
A.$1$ B.$\pi$ C.$0$ D.$\frac{173}{131}$ E.$\frac{131}{173}$
F.$-\frac{173}{131}$ G.$-1$
H.$(\frac{173}{131})^{5}$
I.$(\frac{131}{173})^{5}$
\testStop
\kluczStart
A
\kluczStop



\zadStart{Zadanie z Wikieł Z 1.1 d) moja wersja nr 359}

Obliczyć wartość wyrażenia $(\frac{173}{137})^{5} \cdot (\frac{137}{173})^{5} \cdot \pi^{0}$.
\zadStop
\rozwStart{Patryk Wirkus}{Martyna Czarnobaj}
$$(\frac{173}{137})^{5} \cdot (\frac{137}{173})^{5} \cdot \pi^{0} = (\frac{173}{137} \cdot \frac{137}{173})^{5} \cdot 1 = 1^{5} \cdot 1 = 1$$
\rozwStop
\odpStart
$1$
\odpStop
\testStart
A.$1$ B.$\pi$ C.$0$ D.$\frac{173}{137}$ E.$\frac{137}{173}$
F.$-\frac{173}{137}$ G.$-1$
H.$(\frac{173}{137})^{5}$
I.$(\frac{137}{173})^{5}$
\testStop
\kluczStart
A
\kluczStop



\zadStart{Zadanie z Wikieł Z 1.1 d) moja wersja nr 360}

Obliczyć wartość wyrażenia $(\frac{173}{139})^{5} \cdot (\frac{139}{173})^{5} \cdot \pi^{0}$.
\zadStop
\rozwStart{Patryk Wirkus}{Martyna Czarnobaj}
$$(\frac{173}{139})^{5} \cdot (\frac{139}{173})^{5} \cdot \pi^{0} = (\frac{173}{139} \cdot \frac{139}{173})^{5} \cdot 1 = 1^{5} \cdot 1 = 1$$
\rozwStop
\odpStart
$1$
\odpStop
\testStart
A.$1$ B.$\pi$ C.$0$ D.$\frac{173}{139}$ E.$\frac{139}{173}$
F.$-\frac{173}{139}$ G.$-1$
H.$(\frac{173}{139})^{5}$
I.$(\frac{139}{173})^{5}$
\testStop
\kluczStart
A
\kluczStop



\zadStart{Zadanie z Wikieł Z 1.1 d) moja wersja nr 361}

Obliczyć wartość wyrażenia $(\frac{179}{103})^{5} \cdot (\frac{103}{179})^{5} \cdot \pi^{0}$.
\zadStop
\rozwStart{Patryk Wirkus}{Martyna Czarnobaj}
$$(\frac{179}{103})^{5} \cdot (\frac{103}{179})^{5} \cdot \pi^{0} = (\frac{179}{103} \cdot \frac{103}{179})^{5} \cdot 1 = 1^{5} \cdot 1 = 1$$
\rozwStop
\odpStart
$1$
\odpStop
\testStart
A.$1$ B.$\pi$ C.$0$ D.$\frac{179}{103}$ E.$\frac{103}{179}$
F.$-\frac{179}{103}$ G.$-1$
H.$(\frac{179}{103})^{5}$
I.$(\frac{103}{179})^{5}$
\testStop
\kluczStart
A
\kluczStop



\zadStart{Zadanie z Wikieł Z 1.1 d) moja wersja nr 362}

Obliczyć wartość wyrażenia $(\frac{179}{107})^{5} \cdot (\frac{107}{179})^{5} \cdot \pi^{0}$.
\zadStop
\rozwStart{Patryk Wirkus}{Martyna Czarnobaj}
$$(\frac{179}{107})^{5} \cdot (\frac{107}{179})^{5} \cdot \pi^{0} = (\frac{179}{107} \cdot \frac{107}{179})^{5} \cdot 1 = 1^{5} \cdot 1 = 1$$
\rozwStop
\odpStart
$1$
\odpStop
\testStart
A.$1$ B.$\pi$ C.$0$ D.$\frac{179}{107}$ E.$\frac{107}{179}$
F.$-\frac{179}{107}$ G.$-1$
H.$(\frac{179}{107})^{5}$
I.$(\frac{107}{179})^{5}$
\testStop
\kluczStart
A
\kluczStop



\zadStart{Zadanie z Wikieł Z 1.1 d) moja wersja nr 363}

Obliczyć wartość wyrażenia $(\frac{179}{109})^{5} \cdot (\frac{109}{179})^{5} \cdot \pi^{0}$.
\zadStop
\rozwStart{Patryk Wirkus}{Martyna Czarnobaj}
$$(\frac{179}{109})^{5} \cdot (\frac{109}{179})^{5} \cdot \pi^{0} = (\frac{179}{109} \cdot \frac{109}{179})^{5} \cdot 1 = 1^{5} \cdot 1 = 1$$
\rozwStop
\odpStart
$1$
\odpStop
\testStart
A.$1$ B.$\pi$ C.$0$ D.$\frac{179}{109}$ E.$\frac{109}{179}$
F.$-\frac{179}{109}$ G.$-1$
H.$(\frac{179}{109})^{5}$
I.$(\frac{109}{179})^{5}$
\testStop
\kluczStart
A
\kluczStop



\zadStart{Zadanie z Wikieł Z 1.1 d) moja wersja nr 364}

Obliczyć wartość wyrażenia $(\frac{179}{113})^{5} \cdot (\frac{113}{179})^{5} \cdot \pi^{0}$.
\zadStop
\rozwStart{Patryk Wirkus}{Martyna Czarnobaj}
$$(\frac{179}{113})^{5} \cdot (\frac{113}{179})^{5} \cdot \pi^{0} = (\frac{179}{113} \cdot \frac{113}{179})^{5} \cdot 1 = 1^{5} \cdot 1 = 1$$
\rozwStop
\odpStart
$1$
\odpStop
\testStart
A.$1$ B.$\pi$ C.$0$ D.$\frac{179}{113}$ E.$\frac{113}{179}$
F.$-\frac{179}{113}$ G.$-1$
H.$(\frac{179}{113})^{5}$
I.$(\frac{113}{179})^{5}$
\testStop
\kluczStart
A
\kluczStop



\zadStart{Zadanie z Wikieł Z 1.1 d) moja wersja nr 365}

Obliczyć wartość wyrażenia $(\frac{179}{127})^{5} \cdot (\frac{127}{179})^{5} \cdot \pi^{0}$.
\zadStop
\rozwStart{Patryk Wirkus}{Martyna Czarnobaj}
$$(\frac{179}{127})^{5} \cdot (\frac{127}{179})^{5} \cdot \pi^{0} = (\frac{179}{127} \cdot \frac{127}{179})^{5} \cdot 1 = 1^{5} \cdot 1 = 1$$
\rozwStop
\odpStart
$1$
\odpStop
\testStart
A.$1$ B.$\pi$ C.$0$ D.$\frac{179}{127}$ E.$\frac{127}{179}$
F.$-\frac{179}{127}$ G.$-1$
H.$(\frac{179}{127})^{5}$
I.$(\frac{127}{179})^{5}$
\testStop
\kluczStart
A
\kluczStop



\zadStart{Zadanie z Wikieł Z 1.1 d) moja wersja nr 366}

Obliczyć wartość wyrażenia $(\frac{179}{131})^{5} \cdot (\frac{131}{179})^{5} \cdot \pi^{0}$.
\zadStop
\rozwStart{Patryk Wirkus}{Martyna Czarnobaj}
$$(\frac{179}{131})^{5} \cdot (\frac{131}{179})^{5} \cdot \pi^{0} = (\frac{179}{131} \cdot \frac{131}{179})^{5} \cdot 1 = 1^{5} \cdot 1 = 1$$
\rozwStop
\odpStart
$1$
\odpStop
\testStart
A.$1$ B.$\pi$ C.$0$ D.$\frac{179}{131}$ E.$\frac{131}{179}$
F.$-\frac{179}{131}$ G.$-1$
H.$(\frac{179}{131})^{5}$
I.$(\frac{131}{179})^{5}$
\testStop
\kluczStart
A
\kluczStop



\zadStart{Zadanie z Wikieł Z 1.1 d) moja wersja nr 367}

Obliczyć wartość wyrażenia $(\frac{179}{137})^{5} \cdot (\frac{137}{179})^{5} \cdot \pi^{0}$.
\zadStop
\rozwStart{Patryk Wirkus}{Martyna Czarnobaj}
$$(\frac{179}{137})^{5} \cdot (\frac{137}{179})^{5} \cdot \pi^{0} = (\frac{179}{137} \cdot \frac{137}{179})^{5} \cdot 1 = 1^{5} \cdot 1 = 1$$
\rozwStop
\odpStart
$1$
\odpStop
\testStart
A.$1$ B.$\pi$ C.$0$ D.$\frac{179}{137}$ E.$\frac{137}{179}$
F.$-\frac{179}{137}$ G.$-1$
H.$(\frac{179}{137})^{5}$
I.$(\frac{137}{179})^{5}$
\testStop
\kluczStart
A
\kluczStop



\zadStart{Zadanie z Wikieł Z 1.1 d) moja wersja nr 368}

Obliczyć wartość wyrażenia $(\frac{179}{139})^{5} \cdot (\frac{139}{179})^{5} \cdot \pi^{0}$.
\zadStop
\rozwStart{Patryk Wirkus}{Martyna Czarnobaj}
$$(\frac{179}{139})^{5} \cdot (\frac{139}{179})^{5} \cdot \pi^{0} = (\frac{179}{139} \cdot \frac{139}{179})^{5} \cdot 1 = 1^{5} \cdot 1 = 1$$
\rozwStop
\odpStart
$1$
\odpStop
\testStart
A.$1$ B.$\pi$ C.$0$ D.$\frac{179}{139}$ E.$\frac{139}{179}$
F.$-\frac{179}{139}$ G.$-1$
H.$(\frac{179}{139})^{5}$
I.$(\frac{139}{179})^{5}$
\testStop
\kluczStart
A
\kluczStop



\zadStart{Zadanie z Wikieł Z 1.1 d) moja wersja nr 369}

Obliczyć wartość wyrażenia $(\frac{251}{103})^{5} \cdot (\frac{103}{251})^{5} \cdot \pi^{0}$.
\zadStop
\rozwStart{Patryk Wirkus}{Martyna Czarnobaj}
$$(\frac{251}{103})^{5} \cdot (\frac{103}{251})^{5} \cdot \pi^{0} = (\frac{251}{103} \cdot \frac{103}{251})^{5} \cdot 1 = 1^{5} \cdot 1 = 1$$
\rozwStop
\odpStart
$1$
\odpStop
\testStart
A.$1$ B.$\pi$ C.$0$ D.$\frac{251}{103}$ E.$\frac{103}{251}$
F.$-\frac{251}{103}$ G.$-1$
H.$(\frac{251}{103})^{5}$
I.$(\frac{103}{251})^{5}$
\testStop
\kluczStart
A
\kluczStop



\zadStart{Zadanie z Wikieł Z 1.1 d) moja wersja nr 370}

Obliczyć wartość wyrażenia $(\frac{251}{107})^{5} \cdot (\frac{107}{251})^{5} \cdot \pi^{0}$.
\zadStop
\rozwStart{Patryk Wirkus}{Martyna Czarnobaj}
$$(\frac{251}{107})^{5} \cdot (\frac{107}{251})^{5} \cdot \pi^{0} = (\frac{251}{107} \cdot \frac{107}{251})^{5} \cdot 1 = 1^{5} \cdot 1 = 1$$
\rozwStop
\odpStart
$1$
\odpStop
\testStart
A.$1$ B.$\pi$ C.$0$ D.$\frac{251}{107}$ E.$\frac{107}{251}$
F.$-\frac{251}{107}$ G.$-1$
H.$(\frac{251}{107})^{5}$
I.$(\frac{107}{251})^{5}$
\testStop
\kluczStart
A
\kluczStop



\zadStart{Zadanie z Wikieł Z 1.1 d) moja wersja nr 371}

Obliczyć wartość wyrażenia $(\frac{251}{109})^{5} \cdot (\frac{109}{251})^{5} \cdot \pi^{0}$.
\zadStop
\rozwStart{Patryk Wirkus}{Martyna Czarnobaj}
$$(\frac{251}{109})^{5} \cdot (\frac{109}{251})^{5} \cdot \pi^{0} = (\frac{251}{109} \cdot \frac{109}{251})^{5} \cdot 1 = 1^{5} \cdot 1 = 1$$
\rozwStop
\odpStart
$1$
\odpStop
\testStart
A.$1$ B.$\pi$ C.$0$ D.$\frac{251}{109}$ E.$\frac{109}{251}$
F.$-\frac{251}{109}$ G.$-1$
H.$(\frac{251}{109})^{5}$
I.$(\frac{109}{251})^{5}$
\testStop
\kluczStart
A
\kluczStop



\zadStart{Zadanie z Wikieł Z 1.1 d) moja wersja nr 372}

Obliczyć wartość wyrażenia $(\frac{251}{113})^{5} \cdot (\frac{113}{251})^{5} \cdot \pi^{0}$.
\zadStop
\rozwStart{Patryk Wirkus}{Martyna Czarnobaj}
$$(\frac{251}{113})^{5} \cdot (\frac{113}{251})^{5} \cdot \pi^{0} = (\frac{251}{113} \cdot \frac{113}{251})^{5} \cdot 1 = 1^{5} \cdot 1 = 1$$
\rozwStop
\odpStart
$1$
\odpStop
\testStart
A.$1$ B.$\pi$ C.$0$ D.$\frac{251}{113}$ E.$\frac{113}{251}$
F.$-\frac{251}{113}$ G.$-1$
H.$(\frac{251}{113})^{5}$
I.$(\frac{113}{251})^{5}$
\testStop
\kluczStart
A
\kluczStop



\zadStart{Zadanie z Wikieł Z 1.1 d) moja wersja nr 373}

Obliczyć wartość wyrażenia $(\frac{251}{127})^{5} \cdot (\frac{127}{251})^{5} \cdot \pi^{0}$.
\zadStop
\rozwStart{Patryk Wirkus}{Martyna Czarnobaj}
$$(\frac{251}{127})^{5} \cdot (\frac{127}{251})^{5} \cdot \pi^{0} = (\frac{251}{127} \cdot \frac{127}{251})^{5} \cdot 1 = 1^{5} \cdot 1 = 1$$
\rozwStop
\odpStart
$1$
\odpStop
\testStart
A.$1$ B.$\pi$ C.$0$ D.$\frac{251}{127}$ E.$\frac{127}{251}$
F.$-\frac{251}{127}$ G.$-1$
H.$(\frac{251}{127})^{5}$
I.$(\frac{127}{251})^{5}$
\testStop
\kluczStart
A
\kluczStop



\zadStart{Zadanie z Wikieł Z 1.1 d) moja wersja nr 374}

Obliczyć wartość wyrażenia $(\frac{251}{131})^{5} \cdot (\frac{131}{251})^{5} \cdot \pi^{0}$.
\zadStop
\rozwStart{Patryk Wirkus}{Martyna Czarnobaj}
$$(\frac{251}{131})^{5} \cdot (\frac{131}{251})^{5} \cdot \pi^{0} = (\frac{251}{131} \cdot \frac{131}{251})^{5} \cdot 1 = 1^{5} \cdot 1 = 1$$
\rozwStop
\odpStart
$1$
\odpStop
\testStart
A.$1$ B.$\pi$ C.$0$ D.$\frac{251}{131}$ E.$\frac{131}{251}$
F.$-\frac{251}{131}$ G.$-1$
H.$(\frac{251}{131})^{5}$
I.$(\frac{131}{251})^{5}$
\testStop
\kluczStart
A
\kluczStop



\zadStart{Zadanie z Wikieł Z 1.1 d) moja wersja nr 375}

Obliczyć wartość wyrażenia $(\frac{251}{137})^{5} \cdot (\frac{137}{251})^{5} \cdot \pi^{0}$.
\zadStop
\rozwStart{Patryk Wirkus}{Martyna Czarnobaj}
$$(\frac{251}{137})^{5} \cdot (\frac{137}{251})^{5} \cdot \pi^{0} = (\frac{251}{137} \cdot \frac{137}{251})^{5} \cdot 1 = 1^{5} \cdot 1 = 1$$
\rozwStop
\odpStart
$1$
\odpStop
\testStart
A.$1$ B.$\pi$ C.$0$ D.$\frac{251}{137}$ E.$\frac{137}{251}$
F.$-\frac{251}{137}$ G.$-1$
H.$(\frac{251}{137})^{5}$
I.$(\frac{137}{251})^{5}$
\testStop
\kluczStart
A
\kluczStop



\zadStart{Zadanie z Wikieł Z 1.1 d) moja wersja nr 376}

Obliczyć wartość wyrażenia $(\frac{251}{139})^{5} \cdot (\frac{139}{251})^{5} \cdot \pi^{0}$.
\zadStop
\rozwStart{Patryk Wirkus}{Martyna Czarnobaj}
$$(\frac{251}{139})^{5} \cdot (\frac{139}{251})^{5} \cdot \pi^{0} = (\frac{251}{139} \cdot \frac{139}{251})^{5} \cdot 1 = 1^{5} \cdot 1 = 1$$
\rozwStop
\odpStart
$1$
\odpStop
\testStart
A.$1$ B.$\pi$ C.$0$ D.$\frac{251}{139}$ E.$\frac{139}{251}$
F.$-\frac{251}{139}$ G.$-1$
H.$(\frac{251}{139})^{5}$
I.$(\frac{139}{251})^{5}$
\testStop
\kluczStart
A
\kluczStop



\zadStart{Zadanie z Wikieł Z 1.1 d) moja wersja nr 377}

Obliczyć wartość wyrażenia $(\frac{257}{103})^{5} \cdot (\frac{103}{257})^{5} \cdot \pi^{0}$.
\zadStop
\rozwStart{Patryk Wirkus}{Martyna Czarnobaj}
$$(\frac{257}{103})^{5} \cdot (\frac{103}{257})^{5} \cdot \pi^{0} = (\frac{257}{103} \cdot \frac{103}{257})^{5} \cdot 1 = 1^{5} \cdot 1 = 1$$
\rozwStop
\odpStart
$1$
\odpStop
\testStart
A.$1$ B.$\pi$ C.$0$ D.$\frac{257}{103}$ E.$\frac{103}{257}$
F.$-\frac{257}{103}$ G.$-1$
H.$(\frac{257}{103})^{5}$
I.$(\frac{103}{257})^{5}$
\testStop
\kluczStart
A
\kluczStop



\zadStart{Zadanie z Wikieł Z 1.1 d) moja wersja nr 378}

Obliczyć wartość wyrażenia $(\frac{257}{107})^{5} \cdot (\frac{107}{257})^{5} \cdot \pi^{0}$.
\zadStop
\rozwStart{Patryk Wirkus}{Martyna Czarnobaj}
$$(\frac{257}{107})^{5} \cdot (\frac{107}{257})^{5} \cdot \pi^{0} = (\frac{257}{107} \cdot \frac{107}{257})^{5} \cdot 1 = 1^{5} \cdot 1 = 1$$
\rozwStop
\odpStart
$1$
\odpStop
\testStart
A.$1$ B.$\pi$ C.$0$ D.$\frac{257}{107}$ E.$\frac{107}{257}$
F.$-\frac{257}{107}$ G.$-1$
H.$(\frac{257}{107})^{5}$
I.$(\frac{107}{257})^{5}$
\testStop
\kluczStart
A
\kluczStop



\zadStart{Zadanie z Wikieł Z 1.1 d) moja wersja nr 379}

Obliczyć wartość wyrażenia $(\frac{257}{109})^{5} \cdot (\frac{109}{257})^{5} \cdot \pi^{0}$.
\zadStop
\rozwStart{Patryk Wirkus}{Martyna Czarnobaj}
$$(\frac{257}{109})^{5} \cdot (\frac{109}{257})^{5} \cdot \pi^{0} = (\frac{257}{109} \cdot \frac{109}{257})^{5} \cdot 1 = 1^{5} \cdot 1 = 1$$
\rozwStop
\odpStart
$1$
\odpStop
\testStart
A.$1$ B.$\pi$ C.$0$ D.$\frac{257}{109}$ E.$\frac{109}{257}$
F.$-\frac{257}{109}$ G.$-1$
H.$(\frac{257}{109})^{5}$
I.$(\frac{109}{257})^{5}$
\testStop
\kluczStart
A
\kluczStop



\zadStart{Zadanie z Wikieł Z 1.1 d) moja wersja nr 380}

Obliczyć wartość wyrażenia $(\frac{257}{113})^{5} \cdot (\frac{113}{257})^{5} \cdot \pi^{0}$.
\zadStop
\rozwStart{Patryk Wirkus}{Martyna Czarnobaj}
$$(\frac{257}{113})^{5} \cdot (\frac{113}{257})^{5} \cdot \pi^{0} = (\frac{257}{113} \cdot \frac{113}{257})^{5} \cdot 1 = 1^{5} \cdot 1 = 1$$
\rozwStop
\odpStart
$1$
\odpStop
\testStart
A.$1$ B.$\pi$ C.$0$ D.$\frac{257}{113}$ E.$\frac{113}{257}$
F.$-\frac{257}{113}$ G.$-1$
H.$(\frac{257}{113})^{5}$
I.$(\frac{113}{257})^{5}$
\testStop
\kluczStart
A
\kluczStop



\zadStart{Zadanie z Wikieł Z 1.1 d) moja wersja nr 381}

Obliczyć wartość wyrażenia $(\frac{257}{127})^{5} \cdot (\frac{127}{257})^{5} \cdot \pi^{0}$.
\zadStop
\rozwStart{Patryk Wirkus}{Martyna Czarnobaj}
$$(\frac{257}{127})^{5} \cdot (\frac{127}{257})^{5} \cdot \pi^{0} = (\frac{257}{127} \cdot \frac{127}{257})^{5} \cdot 1 = 1^{5} \cdot 1 = 1$$
\rozwStop
\odpStart
$1$
\odpStop
\testStart
A.$1$ B.$\pi$ C.$0$ D.$\frac{257}{127}$ E.$\frac{127}{257}$
F.$-\frac{257}{127}$ G.$-1$
H.$(\frac{257}{127})^{5}$
I.$(\frac{127}{257})^{5}$
\testStop
\kluczStart
A
\kluczStop



\zadStart{Zadanie z Wikieł Z 1.1 d) moja wersja nr 382}

Obliczyć wartość wyrażenia $(\frac{257}{131})^{5} \cdot (\frac{131}{257})^{5} \cdot \pi^{0}$.
\zadStop
\rozwStart{Patryk Wirkus}{Martyna Czarnobaj}
$$(\frac{257}{131})^{5} \cdot (\frac{131}{257})^{5} \cdot \pi^{0} = (\frac{257}{131} \cdot \frac{131}{257})^{5} \cdot 1 = 1^{5} \cdot 1 = 1$$
\rozwStop
\odpStart
$1$
\odpStop
\testStart
A.$1$ B.$\pi$ C.$0$ D.$\frac{257}{131}$ E.$\frac{131}{257}$
F.$-\frac{257}{131}$ G.$-1$
H.$(\frac{257}{131})^{5}$
I.$(\frac{131}{257})^{5}$
\testStop
\kluczStart
A
\kluczStop



\zadStart{Zadanie z Wikieł Z 1.1 d) moja wersja nr 383}

Obliczyć wartość wyrażenia $(\frac{257}{137})^{5} \cdot (\frac{137}{257})^{5} \cdot \pi^{0}$.
\zadStop
\rozwStart{Patryk Wirkus}{Martyna Czarnobaj}
$$(\frac{257}{137})^{5} \cdot (\frac{137}{257})^{5} \cdot \pi^{0} = (\frac{257}{137} \cdot \frac{137}{257})^{5} \cdot 1 = 1^{5} \cdot 1 = 1$$
\rozwStop
\odpStart
$1$
\odpStop
\testStart
A.$1$ B.$\pi$ C.$0$ D.$\frac{257}{137}$ E.$\frac{137}{257}$
F.$-\frac{257}{137}$ G.$-1$
H.$(\frac{257}{137})^{5}$
I.$(\frac{137}{257})^{5}$
\testStop
\kluczStart
A
\kluczStop



\zadStart{Zadanie z Wikieł Z 1.1 d) moja wersja nr 384}

Obliczyć wartość wyrażenia $(\frac{257}{139})^{5} \cdot (\frac{139}{257})^{5} \cdot \pi^{0}$.
\zadStop
\rozwStart{Patryk Wirkus}{Martyna Czarnobaj}
$$(\frac{257}{139})^{5} \cdot (\frac{139}{257})^{5} \cdot \pi^{0} = (\frac{257}{139} \cdot \frac{139}{257})^{5} \cdot 1 = 1^{5} \cdot 1 = 1$$
\rozwStop
\odpStart
$1$
\odpStop
\testStart
A.$1$ B.$\pi$ C.$0$ D.$\frac{257}{139}$ E.$\frac{139}{257}$
F.$-\frac{257}{139}$ G.$-1$
H.$(\frac{257}{139})^{5}$
I.$(\frac{139}{257})^{5}$
\testStop
\kluczStart
A
\kluczStop



\zadStart{Zadanie z Wikieł Z 1.1 d) moja wersja nr 385}

Obliczyć wartość wyrażenia $(\frac{263}{103})^{5} \cdot (\frac{103}{263})^{5} \cdot \pi^{0}$.
\zadStop
\rozwStart{Patryk Wirkus}{Martyna Czarnobaj}
$$(\frac{263}{103})^{5} \cdot (\frac{103}{263})^{5} \cdot \pi^{0} = (\frac{263}{103} \cdot \frac{103}{263})^{5} \cdot 1 = 1^{5} \cdot 1 = 1$$
\rozwStop
\odpStart
$1$
\odpStop
\testStart
A.$1$ B.$\pi$ C.$0$ D.$\frac{263}{103}$ E.$\frac{103}{263}$
F.$-\frac{263}{103}$ G.$-1$
H.$(\frac{263}{103})^{5}$
I.$(\frac{103}{263})^{5}$
\testStop
\kluczStart
A
\kluczStop



\zadStart{Zadanie z Wikieł Z 1.1 d) moja wersja nr 386}

Obliczyć wartość wyrażenia $(\frac{263}{107})^{5} \cdot (\frac{107}{263})^{5} \cdot \pi^{0}$.
\zadStop
\rozwStart{Patryk Wirkus}{Martyna Czarnobaj}
$$(\frac{263}{107})^{5} \cdot (\frac{107}{263})^{5} \cdot \pi^{0} = (\frac{263}{107} \cdot \frac{107}{263})^{5} \cdot 1 = 1^{5} \cdot 1 = 1$$
\rozwStop
\odpStart
$1$
\odpStop
\testStart
A.$1$ B.$\pi$ C.$0$ D.$\frac{263}{107}$ E.$\frac{107}{263}$
F.$-\frac{263}{107}$ G.$-1$
H.$(\frac{263}{107})^{5}$
I.$(\frac{107}{263})^{5}$
\testStop
\kluczStart
A
\kluczStop



\zadStart{Zadanie z Wikieł Z 1.1 d) moja wersja nr 387}

Obliczyć wartość wyrażenia $(\frac{263}{109})^{5} \cdot (\frac{109}{263})^{5} \cdot \pi^{0}$.
\zadStop
\rozwStart{Patryk Wirkus}{Martyna Czarnobaj}
$$(\frac{263}{109})^{5} \cdot (\frac{109}{263})^{5} \cdot \pi^{0} = (\frac{263}{109} \cdot \frac{109}{263})^{5} \cdot 1 = 1^{5} \cdot 1 = 1$$
\rozwStop
\odpStart
$1$
\odpStop
\testStart
A.$1$ B.$\pi$ C.$0$ D.$\frac{263}{109}$ E.$\frac{109}{263}$
F.$-\frac{263}{109}$ G.$-1$
H.$(\frac{263}{109})^{5}$
I.$(\frac{109}{263})^{5}$
\testStop
\kluczStart
A
\kluczStop



\zadStart{Zadanie z Wikieł Z 1.1 d) moja wersja nr 388}

Obliczyć wartość wyrażenia $(\frac{263}{113})^{5} \cdot (\frac{113}{263})^{5} \cdot \pi^{0}$.
\zadStop
\rozwStart{Patryk Wirkus}{Martyna Czarnobaj}
$$(\frac{263}{113})^{5} \cdot (\frac{113}{263})^{5} \cdot \pi^{0} = (\frac{263}{113} \cdot \frac{113}{263})^{5} \cdot 1 = 1^{5} \cdot 1 = 1$$
\rozwStop
\odpStart
$1$
\odpStop
\testStart
A.$1$ B.$\pi$ C.$0$ D.$\frac{263}{113}$ E.$\frac{113}{263}$
F.$-\frac{263}{113}$ G.$-1$
H.$(\frac{263}{113})^{5}$
I.$(\frac{113}{263})^{5}$
\testStop
\kluczStart
A
\kluczStop



\zadStart{Zadanie z Wikieł Z 1.1 d) moja wersja nr 389}

Obliczyć wartość wyrażenia $(\frac{263}{127})^{5} \cdot (\frac{127}{263})^{5} \cdot \pi^{0}$.
\zadStop
\rozwStart{Patryk Wirkus}{Martyna Czarnobaj}
$$(\frac{263}{127})^{5} \cdot (\frac{127}{263})^{5} \cdot \pi^{0} = (\frac{263}{127} \cdot \frac{127}{263})^{5} \cdot 1 = 1^{5} \cdot 1 = 1$$
\rozwStop
\odpStart
$1$
\odpStop
\testStart
A.$1$ B.$\pi$ C.$0$ D.$\frac{263}{127}$ E.$\frac{127}{263}$
F.$-\frac{263}{127}$ G.$-1$
H.$(\frac{263}{127})^{5}$
I.$(\frac{127}{263})^{5}$
\testStop
\kluczStart
A
\kluczStop



\zadStart{Zadanie z Wikieł Z 1.1 d) moja wersja nr 390}

Obliczyć wartość wyrażenia $(\frac{263}{131})^{5} \cdot (\frac{131}{263})^{5} \cdot \pi^{0}$.
\zadStop
\rozwStart{Patryk Wirkus}{Martyna Czarnobaj}
$$(\frac{263}{131})^{5} \cdot (\frac{131}{263})^{5} \cdot \pi^{0} = (\frac{263}{131} \cdot \frac{131}{263})^{5} \cdot 1 = 1^{5} \cdot 1 = 1$$
\rozwStop
\odpStart
$1$
\odpStop
\testStart
A.$1$ B.$\pi$ C.$0$ D.$\frac{263}{131}$ E.$\frac{131}{263}$
F.$-\frac{263}{131}$ G.$-1$
H.$(\frac{263}{131})^{5}$
I.$(\frac{131}{263})^{5}$
\testStop
\kluczStart
A
\kluczStop



\zadStart{Zadanie z Wikieł Z 1.1 d) moja wersja nr 391}

Obliczyć wartość wyrażenia $(\frac{263}{137})^{5} \cdot (\frac{137}{263})^{5} \cdot \pi^{0}$.
\zadStop
\rozwStart{Patryk Wirkus}{Martyna Czarnobaj}
$$(\frac{263}{137})^{5} \cdot (\frac{137}{263})^{5} \cdot \pi^{0} = (\frac{263}{137} \cdot \frac{137}{263})^{5} \cdot 1 = 1^{5} \cdot 1 = 1$$
\rozwStop
\odpStart
$1$
\odpStop
\testStart
A.$1$ B.$\pi$ C.$0$ D.$\frac{263}{137}$ E.$\frac{137}{263}$
F.$-\frac{263}{137}$ G.$-1$
H.$(\frac{263}{137})^{5}$
I.$(\frac{137}{263})^{5}$
\testStop
\kluczStart
A
\kluczStop



\zadStart{Zadanie z Wikieł Z 1.1 d) moja wersja nr 392}

Obliczyć wartość wyrażenia $(\frac{263}{139})^{5} \cdot (\frac{139}{263})^{5} \cdot \pi^{0}$.
\zadStop
\rozwStart{Patryk Wirkus}{Martyna Czarnobaj}
$$(\frac{263}{139})^{5} \cdot (\frac{139}{263})^{5} \cdot \pi^{0} = (\frac{263}{139} \cdot \frac{139}{263})^{5} \cdot 1 = 1^{5} \cdot 1 = 1$$
\rozwStop
\odpStart
$1$
\odpStop
\testStart
A.$1$ B.$\pi$ C.$0$ D.$\frac{263}{139}$ E.$\frac{139}{263}$
F.$-\frac{263}{139}$ G.$-1$
H.$(\frac{263}{139})^{5}$
I.$(\frac{139}{263})^{5}$
\testStop
\kluczStart
A
\kluczStop



\zadStart{Zadanie z Wikieł Z 1.1 d) moja wersja nr 393}

Obliczyć wartość wyrażenia $(\frac{269}{103})^{5} \cdot (\frac{103}{269})^{5} \cdot \pi^{0}$.
\zadStop
\rozwStart{Patryk Wirkus}{Martyna Czarnobaj}
$$(\frac{269}{103})^{5} \cdot (\frac{103}{269})^{5} \cdot \pi^{0} = (\frac{269}{103} \cdot \frac{103}{269})^{5} \cdot 1 = 1^{5} \cdot 1 = 1$$
\rozwStop
\odpStart
$1$
\odpStop
\testStart
A.$1$ B.$\pi$ C.$0$ D.$\frac{269}{103}$ E.$\frac{103}{269}$
F.$-\frac{269}{103}$ G.$-1$
H.$(\frac{269}{103})^{5}$
I.$(\frac{103}{269})^{5}$
\testStop
\kluczStart
A
\kluczStop



\zadStart{Zadanie z Wikieł Z 1.1 d) moja wersja nr 394}

Obliczyć wartość wyrażenia $(\frac{269}{107})^{5} \cdot (\frac{107}{269})^{5} \cdot \pi^{0}$.
\zadStop
\rozwStart{Patryk Wirkus}{Martyna Czarnobaj}
$$(\frac{269}{107})^{5} \cdot (\frac{107}{269})^{5} \cdot \pi^{0} = (\frac{269}{107} \cdot \frac{107}{269})^{5} \cdot 1 = 1^{5} \cdot 1 = 1$$
\rozwStop
\odpStart
$1$
\odpStop
\testStart
A.$1$ B.$\pi$ C.$0$ D.$\frac{269}{107}$ E.$\frac{107}{269}$
F.$-\frac{269}{107}$ G.$-1$
H.$(\frac{269}{107})^{5}$
I.$(\frac{107}{269})^{5}$
\testStop
\kluczStart
A
\kluczStop



\zadStart{Zadanie z Wikieł Z 1.1 d) moja wersja nr 395}

Obliczyć wartość wyrażenia $(\frac{269}{109})^{5} \cdot (\frac{109}{269})^{5} \cdot \pi^{0}$.
\zadStop
\rozwStart{Patryk Wirkus}{Martyna Czarnobaj}
$$(\frac{269}{109})^{5} \cdot (\frac{109}{269})^{5} \cdot \pi^{0} = (\frac{269}{109} \cdot \frac{109}{269})^{5} \cdot 1 = 1^{5} \cdot 1 = 1$$
\rozwStop
\odpStart
$1$
\odpStop
\testStart
A.$1$ B.$\pi$ C.$0$ D.$\frac{269}{109}$ E.$\frac{109}{269}$
F.$-\frac{269}{109}$ G.$-1$
H.$(\frac{269}{109})^{5}$
I.$(\frac{109}{269})^{5}$
\testStop
\kluczStart
A
\kluczStop



\zadStart{Zadanie z Wikieł Z 1.1 d) moja wersja nr 396}

Obliczyć wartość wyrażenia $(\frac{269}{113})^{5} \cdot (\frac{113}{269})^{5} \cdot \pi^{0}$.
\zadStop
\rozwStart{Patryk Wirkus}{Martyna Czarnobaj}
$$(\frac{269}{113})^{5} \cdot (\frac{113}{269})^{5} \cdot \pi^{0} = (\frac{269}{113} \cdot \frac{113}{269})^{5} \cdot 1 = 1^{5} \cdot 1 = 1$$
\rozwStop
\odpStart
$1$
\odpStop
\testStart
A.$1$ B.$\pi$ C.$0$ D.$\frac{269}{113}$ E.$\frac{113}{269}$
F.$-\frac{269}{113}$ G.$-1$
H.$(\frac{269}{113})^{5}$
I.$(\frac{113}{269})^{5}$
\testStop
\kluczStart
A
\kluczStop



\zadStart{Zadanie z Wikieł Z 1.1 d) moja wersja nr 397}

Obliczyć wartość wyrażenia $(\frac{269}{127})^{5} \cdot (\frac{127}{269})^{5} \cdot \pi^{0}$.
\zadStop
\rozwStart{Patryk Wirkus}{Martyna Czarnobaj}
$$(\frac{269}{127})^{5} \cdot (\frac{127}{269})^{5} \cdot \pi^{0} = (\frac{269}{127} \cdot \frac{127}{269})^{5} \cdot 1 = 1^{5} \cdot 1 = 1$$
\rozwStop
\odpStart
$1$
\odpStop
\testStart
A.$1$ B.$\pi$ C.$0$ D.$\frac{269}{127}$ E.$\frac{127}{269}$
F.$-\frac{269}{127}$ G.$-1$
H.$(\frac{269}{127})^{5}$
I.$(\frac{127}{269})^{5}$
\testStop
\kluczStart
A
\kluczStop



\zadStart{Zadanie z Wikieł Z 1.1 d) moja wersja nr 398}

Obliczyć wartość wyrażenia $(\frac{269}{131})^{5} \cdot (\frac{131}{269})^{5} \cdot \pi^{0}$.
\zadStop
\rozwStart{Patryk Wirkus}{Martyna Czarnobaj}
$$(\frac{269}{131})^{5} \cdot (\frac{131}{269})^{5} \cdot \pi^{0} = (\frac{269}{131} \cdot \frac{131}{269})^{5} \cdot 1 = 1^{5} \cdot 1 = 1$$
\rozwStop
\odpStart
$1$
\odpStop
\testStart
A.$1$ B.$\pi$ C.$0$ D.$\frac{269}{131}$ E.$\frac{131}{269}$
F.$-\frac{269}{131}$ G.$-1$
H.$(\frac{269}{131})^{5}$
I.$(\frac{131}{269})^{5}$
\testStop
\kluczStart
A
\kluczStop



\zadStart{Zadanie z Wikieł Z 1.1 d) moja wersja nr 399}

Obliczyć wartość wyrażenia $(\frac{269}{137})^{5} \cdot (\frac{137}{269})^{5} \cdot \pi^{0}$.
\zadStop
\rozwStart{Patryk Wirkus}{Martyna Czarnobaj}
$$(\frac{269}{137})^{5} \cdot (\frac{137}{269})^{5} \cdot \pi^{0} = (\frac{269}{137} \cdot \frac{137}{269})^{5} \cdot 1 = 1^{5} \cdot 1 = 1$$
\rozwStop
\odpStart
$1$
\odpStop
\testStart
A.$1$ B.$\pi$ C.$0$ D.$\frac{269}{137}$ E.$\frac{137}{269}$
F.$-\frac{269}{137}$ G.$-1$
H.$(\frac{269}{137})^{5}$
I.$(\frac{137}{269})^{5}$
\testStop
\kluczStart
A
\kluczStop



\zadStart{Zadanie z Wikieł Z 1.1 d) moja wersja nr 400}

Obliczyć wartość wyrażenia $(\frac{269}{139})^{5} \cdot (\frac{139}{269})^{5} \cdot \pi^{0}$.
\zadStop
\rozwStart{Patryk Wirkus}{Martyna Czarnobaj}
$$(\frac{269}{139})^{5} \cdot (\frac{139}{269})^{5} \cdot \pi^{0} = (\frac{269}{139} \cdot \frac{139}{269})^{5} \cdot 1 = 1^{5} \cdot 1 = 1$$
\rozwStop
\odpStart
$1$
\odpStop
\testStart
A.$1$ B.$\pi$ C.$0$ D.$\frac{269}{139}$ E.$\frac{139}{269}$
F.$-\frac{269}{139}$ G.$-1$
H.$(\frac{269}{139})^{5}$
I.$(\frac{139}{269})^{5}$
\testStop
\kluczStart
A
\kluczStop



\zadStart{Zadanie z Wikieł Z 1.1 d) moja wersja nr 401}

Obliczyć wartość wyrażenia $(\frac{271}{103})^{5} \cdot (\frac{103}{271})^{5} \cdot \pi^{0}$.
\zadStop
\rozwStart{Patryk Wirkus}{Martyna Czarnobaj}
$$(\frac{271}{103})^{5} \cdot (\frac{103}{271})^{5} \cdot \pi^{0} = (\frac{271}{103} \cdot \frac{103}{271})^{5} \cdot 1 = 1^{5} \cdot 1 = 1$$
\rozwStop
\odpStart
$1$
\odpStop
\testStart
A.$1$ B.$\pi$ C.$0$ D.$\frac{271}{103}$ E.$\frac{103}{271}$
F.$-\frac{271}{103}$ G.$-1$
H.$(\frac{271}{103})^{5}$
I.$(\frac{103}{271})^{5}$
\testStop
\kluczStart
A
\kluczStop



\zadStart{Zadanie z Wikieł Z 1.1 d) moja wersja nr 402}

Obliczyć wartość wyrażenia $(\frac{271}{107})^{5} \cdot (\frac{107}{271})^{5} \cdot \pi^{0}$.
\zadStop
\rozwStart{Patryk Wirkus}{Martyna Czarnobaj}
$$(\frac{271}{107})^{5} \cdot (\frac{107}{271})^{5} \cdot \pi^{0} = (\frac{271}{107} \cdot \frac{107}{271})^{5} \cdot 1 = 1^{5} \cdot 1 = 1$$
\rozwStop
\odpStart
$1$
\odpStop
\testStart
A.$1$ B.$\pi$ C.$0$ D.$\frac{271}{107}$ E.$\frac{107}{271}$
F.$-\frac{271}{107}$ G.$-1$
H.$(\frac{271}{107})^{5}$
I.$(\frac{107}{271})^{5}$
\testStop
\kluczStart
A
\kluczStop



\zadStart{Zadanie z Wikieł Z 1.1 d) moja wersja nr 403}

Obliczyć wartość wyrażenia $(\frac{271}{109})^{5} \cdot (\frac{109}{271})^{5} \cdot \pi^{0}$.
\zadStop
\rozwStart{Patryk Wirkus}{Martyna Czarnobaj}
$$(\frac{271}{109})^{5} \cdot (\frac{109}{271})^{5} \cdot \pi^{0} = (\frac{271}{109} \cdot \frac{109}{271})^{5} \cdot 1 = 1^{5} \cdot 1 = 1$$
\rozwStop
\odpStart
$1$
\odpStop
\testStart
A.$1$ B.$\pi$ C.$0$ D.$\frac{271}{109}$ E.$\frac{109}{271}$
F.$-\frac{271}{109}$ G.$-1$
H.$(\frac{271}{109})^{5}$
I.$(\frac{109}{271})^{5}$
\testStop
\kluczStart
A
\kluczStop



\zadStart{Zadanie z Wikieł Z 1.1 d) moja wersja nr 404}

Obliczyć wartość wyrażenia $(\frac{271}{113})^{5} \cdot (\frac{113}{271})^{5} \cdot \pi^{0}$.
\zadStop
\rozwStart{Patryk Wirkus}{Martyna Czarnobaj}
$$(\frac{271}{113})^{5} \cdot (\frac{113}{271})^{5} \cdot \pi^{0} = (\frac{271}{113} \cdot \frac{113}{271})^{5} \cdot 1 = 1^{5} \cdot 1 = 1$$
\rozwStop
\odpStart
$1$
\odpStop
\testStart
A.$1$ B.$\pi$ C.$0$ D.$\frac{271}{113}$ E.$\frac{113}{271}$
F.$-\frac{271}{113}$ G.$-1$
H.$(\frac{271}{113})^{5}$
I.$(\frac{113}{271})^{5}$
\testStop
\kluczStart
A
\kluczStop



\zadStart{Zadanie z Wikieł Z 1.1 d) moja wersja nr 405}

Obliczyć wartość wyrażenia $(\frac{271}{127})^{5} \cdot (\frac{127}{271})^{5} \cdot \pi^{0}$.
\zadStop
\rozwStart{Patryk Wirkus}{Martyna Czarnobaj}
$$(\frac{271}{127})^{5} \cdot (\frac{127}{271})^{5} \cdot \pi^{0} = (\frac{271}{127} \cdot \frac{127}{271})^{5} \cdot 1 = 1^{5} \cdot 1 = 1$$
\rozwStop
\odpStart
$1$
\odpStop
\testStart
A.$1$ B.$\pi$ C.$0$ D.$\frac{271}{127}$ E.$\frac{127}{271}$
F.$-\frac{271}{127}$ G.$-1$
H.$(\frac{271}{127})^{5}$
I.$(\frac{127}{271})^{5}$
\testStop
\kluczStart
A
\kluczStop



\zadStart{Zadanie z Wikieł Z 1.1 d) moja wersja nr 406}

Obliczyć wartość wyrażenia $(\frac{271}{131})^{5} \cdot (\frac{131}{271})^{5} \cdot \pi^{0}$.
\zadStop
\rozwStart{Patryk Wirkus}{Martyna Czarnobaj}
$$(\frac{271}{131})^{5} \cdot (\frac{131}{271})^{5} \cdot \pi^{0} = (\frac{271}{131} \cdot \frac{131}{271})^{5} \cdot 1 = 1^{5} \cdot 1 = 1$$
\rozwStop
\odpStart
$1$
\odpStop
\testStart
A.$1$ B.$\pi$ C.$0$ D.$\frac{271}{131}$ E.$\frac{131}{271}$
F.$-\frac{271}{131}$ G.$-1$
H.$(\frac{271}{131})^{5}$
I.$(\frac{131}{271})^{5}$
\testStop
\kluczStart
A
\kluczStop



\zadStart{Zadanie z Wikieł Z 1.1 d) moja wersja nr 407}

Obliczyć wartość wyrażenia $(\frac{271}{137})^{5} \cdot (\frac{137}{271})^{5} \cdot \pi^{0}$.
\zadStop
\rozwStart{Patryk Wirkus}{Martyna Czarnobaj}
$$(\frac{271}{137})^{5} \cdot (\frac{137}{271})^{5} \cdot \pi^{0} = (\frac{271}{137} \cdot \frac{137}{271})^{5} \cdot 1 = 1^{5} \cdot 1 = 1$$
\rozwStop
\odpStart
$1$
\odpStop
\testStart
A.$1$ B.$\pi$ C.$0$ D.$\frac{271}{137}$ E.$\frac{137}{271}$
F.$-\frac{271}{137}$ G.$-1$
H.$(\frac{271}{137})^{5}$
I.$(\frac{137}{271})^{5}$
\testStop
\kluczStart
A
\kluczStop



\zadStart{Zadanie z Wikieł Z 1.1 d) moja wersja nr 408}

Obliczyć wartość wyrażenia $(\frac{271}{139})^{5} \cdot (\frac{139}{271})^{5} \cdot \pi^{0}$.
\zadStop
\rozwStart{Patryk Wirkus}{Martyna Czarnobaj}
$$(\frac{271}{139})^{5} \cdot (\frac{139}{271})^{5} \cdot \pi^{0} = (\frac{271}{139} \cdot \frac{139}{271})^{5} \cdot 1 = 1^{5} \cdot 1 = 1$$
\rozwStop
\odpStart
$1$
\odpStop
\testStart
A.$1$ B.$\pi$ C.$0$ D.$\frac{271}{139}$ E.$\frac{139}{271}$
F.$-\frac{271}{139}$ G.$-1$
H.$(\frac{271}{139})^{5}$
I.$(\frac{139}{271})^{5}$
\testStop
\kluczStart
A
\kluczStop



\zadStart{Zadanie z Wikieł Z 1.1 d) moja wersja nr 409}

Obliczyć wartość wyrażenia $(\frac{277}{103})^{5} \cdot (\frac{103}{277})^{5} \cdot \pi^{0}$.
\zadStop
\rozwStart{Patryk Wirkus}{Martyna Czarnobaj}
$$(\frac{277}{103})^{5} \cdot (\frac{103}{277})^{5} \cdot \pi^{0} = (\frac{277}{103} \cdot \frac{103}{277})^{5} \cdot 1 = 1^{5} \cdot 1 = 1$$
\rozwStop
\odpStart
$1$
\odpStop
\testStart
A.$1$ B.$\pi$ C.$0$ D.$\frac{277}{103}$ E.$\frac{103}{277}$
F.$-\frac{277}{103}$ G.$-1$
H.$(\frac{277}{103})^{5}$
I.$(\frac{103}{277})^{5}$
\testStop
\kluczStart
A
\kluczStop



\zadStart{Zadanie z Wikieł Z 1.1 d) moja wersja nr 410}

Obliczyć wartość wyrażenia $(\frac{277}{107})^{5} \cdot (\frac{107}{277})^{5} \cdot \pi^{0}$.
\zadStop
\rozwStart{Patryk Wirkus}{Martyna Czarnobaj}
$$(\frac{277}{107})^{5} \cdot (\frac{107}{277})^{5} \cdot \pi^{0} = (\frac{277}{107} \cdot \frac{107}{277})^{5} \cdot 1 = 1^{5} \cdot 1 = 1$$
\rozwStop
\odpStart
$1$
\odpStop
\testStart
A.$1$ B.$\pi$ C.$0$ D.$\frac{277}{107}$ E.$\frac{107}{277}$
F.$-\frac{277}{107}$ G.$-1$
H.$(\frac{277}{107})^{5}$
I.$(\frac{107}{277})^{5}$
\testStop
\kluczStart
A
\kluczStop



\zadStart{Zadanie z Wikieł Z 1.1 d) moja wersja nr 411}

Obliczyć wartość wyrażenia $(\frac{277}{109})^{5} \cdot (\frac{109}{277})^{5} \cdot \pi^{0}$.
\zadStop
\rozwStart{Patryk Wirkus}{Martyna Czarnobaj}
$$(\frac{277}{109})^{5} \cdot (\frac{109}{277})^{5} \cdot \pi^{0} = (\frac{277}{109} \cdot \frac{109}{277})^{5} \cdot 1 = 1^{5} \cdot 1 = 1$$
\rozwStop
\odpStart
$1$
\odpStop
\testStart
A.$1$ B.$\pi$ C.$0$ D.$\frac{277}{109}$ E.$\frac{109}{277}$
F.$-\frac{277}{109}$ G.$-1$
H.$(\frac{277}{109})^{5}$
I.$(\frac{109}{277})^{5}$
\testStop
\kluczStart
A
\kluczStop



\zadStart{Zadanie z Wikieł Z 1.1 d) moja wersja nr 412}

Obliczyć wartość wyrażenia $(\frac{277}{113})^{5} \cdot (\frac{113}{277})^{5} \cdot \pi^{0}$.
\zadStop
\rozwStart{Patryk Wirkus}{Martyna Czarnobaj}
$$(\frac{277}{113})^{5} \cdot (\frac{113}{277})^{5} \cdot \pi^{0} = (\frac{277}{113} \cdot \frac{113}{277})^{5} \cdot 1 = 1^{5} \cdot 1 = 1$$
\rozwStop
\odpStart
$1$
\odpStop
\testStart
A.$1$ B.$\pi$ C.$0$ D.$\frac{277}{113}$ E.$\frac{113}{277}$
F.$-\frac{277}{113}$ G.$-1$
H.$(\frac{277}{113})^{5}$
I.$(\frac{113}{277})^{5}$
\testStop
\kluczStart
A
\kluczStop



\zadStart{Zadanie z Wikieł Z 1.1 d) moja wersja nr 413}

Obliczyć wartość wyrażenia $(\frac{277}{127})^{5} \cdot (\frac{127}{277})^{5} \cdot \pi^{0}$.
\zadStop
\rozwStart{Patryk Wirkus}{Martyna Czarnobaj}
$$(\frac{277}{127})^{5} \cdot (\frac{127}{277})^{5} \cdot \pi^{0} = (\frac{277}{127} \cdot \frac{127}{277})^{5} \cdot 1 = 1^{5} \cdot 1 = 1$$
\rozwStop
\odpStart
$1$
\odpStop
\testStart
A.$1$ B.$\pi$ C.$0$ D.$\frac{277}{127}$ E.$\frac{127}{277}$
F.$-\frac{277}{127}$ G.$-1$
H.$(\frac{277}{127})^{5}$
I.$(\frac{127}{277})^{5}$
\testStop
\kluczStart
A
\kluczStop



\zadStart{Zadanie z Wikieł Z 1.1 d) moja wersja nr 414}

Obliczyć wartość wyrażenia $(\frac{277}{131})^{5} \cdot (\frac{131}{277})^{5} \cdot \pi^{0}$.
\zadStop
\rozwStart{Patryk Wirkus}{Martyna Czarnobaj}
$$(\frac{277}{131})^{5} \cdot (\frac{131}{277})^{5} \cdot \pi^{0} = (\frac{277}{131} \cdot \frac{131}{277})^{5} \cdot 1 = 1^{5} \cdot 1 = 1$$
\rozwStop
\odpStart
$1$
\odpStop
\testStart
A.$1$ B.$\pi$ C.$0$ D.$\frac{277}{131}$ E.$\frac{131}{277}$
F.$-\frac{277}{131}$ G.$-1$
H.$(\frac{277}{131})^{5}$
I.$(\frac{131}{277})^{5}$
\testStop
\kluczStart
A
\kluczStop



\zadStart{Zadanie z Wikieł Z 1.1 d) moja wersja nr 415}

Obliczyć wartość wyrażenia $(\frac{277}{137})^{5} \cdot (\frac{137}{277})^{5} \cdot \pi^{0}$.
\zadStop
\rozwStart{Patryk Wirkus}{Martyna Czarnobaj}
$$(\frac{277}{137})^{5} \cdot (\frac{137}{277})^{5} \cdot \pi^{0} = (\frac{277}{137} \cdot \frac{137}{277})^{5} \cdot 1 = 1^{5} \cdot 1 = 1$$
\rozwStop
\odpStart
$1$
\odpStop
\testStart
A.$1$ B.$\pi$ C.$0$ D.$\frac{277}{137}$ E.$\frac{137}{277}$
F.$-\frac{277}{137}$ G.$-1$
H.$(\frac{277}{137})^{5}$
I.$(\frac{137}{277})^{5}$
\testStop
\kluczStart
A
\kluczStop



\zadStart{Zadanie z Wikieł Z 1.1 d) moja wersja nr 416}

Obliczyć wartość wyrażenia $(\frac{277}{139})^{5} \cdot (\frac{139}{277})^{5} \cdot \pi^{0}$.
\zadStop
\rozwStart{Patryk Wirkus}{Martyna Czarnobaj}
$$(\frac{277}{139})^{5} \cdot (\frac{139}{277})^{5} \cdot \pi^{0} = (\frac{277}{139} \cdot \frac{139}{277})^{5} \cdot 1 = 1^{5} \cdot 1 = 1$$
\rozwStop
\odpStart
$1$
\odpStop
\testStart
A.$1$ B.$\pi$ C.$0$ D.$\frac{277}{139}$ E.$\frac{139}{277}$
F.$-\frac{277}{139}$ G.$-1$
H.$(\frac{277}{139})^{5}$
I.$(\frac{139}{277})^{5}$
\testStop
\kluczStart
A
\kluczStop



\zadStart{Zadanie z Wikieł Z 1.1 d) moja wersja nr 417}

Obliczyć wartość wyrażenia $(\frac{149}{103})^{6} \cdot (\frac{103}{149})^{6} \cdot \pi^{0}$.
\zadStop
\rozwStart{Patryk Wirkus}{Martyna Czarnobaj}
$$(\frac{149}{103})^{6} \cdot (\frac{103}{149})^{6} \cdot \pi^{0} = (\frac{149}{103} \cdot \frac{103}{149})^{6} \cdot 1 = 1^{6} \cdot 1 = 1$$
\rozwStop
\odpStart
$1$
\odpStop
\testStart
A.$1$ B.$\pi$ C.$0$ D.$\frac{149}{103}$ E.$\frac{103}{149}$
F.$-\frac{149}{103}$ G.$-1$
H.$(\frac{149}{103})^{6}$
I.$(\frac{103}{149})^{6}$
\testStop
\kluczStart
A
\kluczStop



\zadStart{Zadanie z Wikieł Z 1.1 d) moja wersja nr 418}

Obliczyć wartość wyrażenia $(\frac{149}{107})^{6} \cdot (\frac{107}{149})^{6} \cdot \pi^{0}$.
\zadStop
\rozwStart{Patryk Wirkus}{Martyna Czarnobaj}
$$(\frac{149}{107})^{6} \cdot (\frac{107}{149})^{6} \cdot \pi^{0} = (\frac{149}{107} \cdot \frac{107}{149})^{6} \cdot 1 = 1^{6} \cdot 1 = 1$$
\rozwStop
\odpStart
$1$
\odpStop
\testStart
A.$1$ B.$\pi$ C.$0$ D.$\frac{149}{107}$ E.$\frac{107}{149}$
F.$-\frac{149}{107}$ G.$-1$
H.$(\frac{149}{107})^{6}$
I.$(\frac{107}{149})^{6}$
\testStop
\kluczStart
A
\kluczStop



\zadStart{Zadanie z Wikieł Z 1.1 d) moja wersja nr 419}

Obliczyć wartość wyrażenia $(\frac{149}{109})^{6} \cdot (\frac{109}{149})^{6} \cdot \pi^{0}$.
\zadStop
\rozwStart{Patryk Wirkus}{Martyna Czarnobaj}
$$(\frac{149}{109})^{6} \cdot (\frac{109}{149})^{6} \cdot \pi^{0} = (\frac{149}{109} \cdot \frac{109}{149})^{6} \cdot 1 = 1^{6} \cdot 1 = 1$$
\rozwStop
\odpStart
$1$
\odpStop
\testStart
A.$1$ B.$\pi$ C.$0$ D.$\frac{149}{109}$ E.$\frac{109}{149}$
F.$-\frac{149}{109}$ G.$-1$
H.$(\frac{149}{109})^{6}$
I.$(\frac{109}{149})^{6}$
\testStop
\kluczStart
A
\kluczStop



\zadStart{Zadanie z Wikieł Z 1.1 d) moja wersja nr 420}

Obliczyć wartość wyrażenia $(\frac{149}{113})^{6} \cdot (\frac{113}{149})^{6} \cdot \pi^{0}$.
\zadStop
\rozwStart{Patryk Wirkus}{Martyna Czarnobaj}
$$(\frac{149}{113})^{6} \cdot (\frac{113}{149})^{6} \cdot \pi^{0} = (\frac{149}{113} \cdot \frac{113}{149})^{6} \cdot 1 = 1^{6} \cdot 1 = 1$$
\rozwStop
\odpStart
$1$
\odpStop
\testStart
A.$1$ B.$\pi$ C.$0$ D.$\frac{149}{113}$ E.$\frac{113}{149}$
F.$-\frac{149}{113}$ G.$-1$
H.$(\frac{149}{113})^{6}$
I.$(\frac{113}{149})^{6}$
\testStop
\kluczStart
A
\kluczStop



\zadStart{Zadanie z Wikieł Z 1.1 d) moja wersja nr 421}

Obliczyć wartość wyrażenia $(\frac{149}{127})^{6} \cdot (\frac{127}{149})^{6} \cdot \pi^{0}$.
\zadStop
\rozwStart{Patryk Wirkus}{Martyna Czarnobaj}
$$(\frac{149}{127})^{6} \cdot (\frac{127}{149})^{6} \cdot \pi^{0} = (\frac{149}{127} \cdot \frac{127}{149})^{6} \cdot 1 = 1^{6} \cdot 1 = 1$$
\rozwStop
\odpStart
$1$
\odpStop
\testStart
A.$1$ B.$\pi$ C.$0$ D.$\frac{149}{127}$ E.$\frac{127}{149}$
F.$-\frac{149}{127}$ G.$-1$
H.$(\frac{149}{127})^{6}$
I.$(\frac{127}{149})^{6}$
\testStop
\kluczStart
A
\kluczStop



\zadStart{Zadanie z Wikieł Z 1.1 d) moja wersja nr 422}

Obliczyć wartość wyrażenia $(\frac{149}{131})^{6} \cdot (\frac{131}{149})^{6} \cdot \pi^{0}$.
\zadStop
\rozwStart{Patryk Wirkus}{Martyna Czarnobaj}
$$(\frac{149}{131})^{6} \cdot (\frac{131}{149})^{6} \cdot \pi^{0} = (\frac{149}{131} \cdot \frac{131}{149})^{6} \cdot 1 = 1^{6} \cdot 1 = 1$$
\rozwStop
\odpStart
$1$
\odpStop
\testStart
A.$1$ B.$\pi$ C.$0$ D.$\frac{149}{131}$ E.$\frac{131}{149}$
F.$-\frac{149}{131}$ G.$-1$
H.$(\frac{149}{131})^{6}$
I.$(\frac{131}{149})^{6}$
\testStop
\kluczStart
A
\kluczStop



\zadStart{Zadanie z Wikieł Z 1.1 d) moja wersja nr 423}

Obliczyć wartość wyrażenia $(\frac{149}{137})^{6} \cdot (\frac{137}{149})^{6} \cdot \pi^{0}$.
\zadStop
\rozwStart{Patryk Wirkus}{Martyna Czarnobaj}
$$(\frac{149}{137})^{6} \cdot (\frac{137}{149})^{6} \cdot \pi^{0} = (\frac{149}{137} \cdot \frac{137}{149})^{6} \cdot 1 = 1^{6} \cdot 1 = 1$$
\rozwStop
\odpStart
$1$
\odpStop
\testStart
A.$1$ B.$\pi$ C.$0$ D.$\frac{149}{137}$ E.$\frac{137}{149}$
F.$-\frac{149}{137}$ G.$-1$
H.$(\frac{149}{137})^{6}$
I.$(\frac{137}{149})^{6}$
\testStop
\kluczStart
A
\kluczStop



\zadStart{Zadanie z Wikieł Z 1.1 d) moja wersja nr 424}

Obliczyć wartość wyrażenia $(\frac{149}{139})^{6} \cdot (\frac{139}{149})^{6} \cdot \pi^{0}$.
\zadStop
\rozwStart{Patryk Wirkus}{Martyna Czarnobaj}
$$(\frac{149}{139})^{6} \cdot (\frac{139}{149})^{6} \cdot \pi^{0} = (\frac{149}{139} \cdot \frac{139}{149})^{6} \cdot 1 = 1^{6} \cdot 1 = 1$$
\rozwStop
\odpStart
$1$
\odpStop
\testStart
A.$1$ B.$\pi$ C.$0$ D.$\frac{149}{139}$ E.$\frac{139}{149}$
F.$-\frac{149}{139}$ G.$-1$
H.$(\frac{149}{139})^{6}$
I.$(\frac{139}{149})^{6}$
\testStop
\kluczStart
A
\kluczStop



\zadStart{Zadanie z Wikieł Z 1.1 d) moja wersja nr 425}

Obliczyć wartość wyrażenia $(\frac{151}{103})^{6} \cdot (\frac{103}{151})^{6} \cdot \pi^{0}$.
\zadStop
\rozwStart{Patryk Wirkus}{Martyna Czarnobaj}
$$(\frac{151}{103})^{6} \cdot (\frac{103}{151})^{6} \cdot \pi^{0} = (\frac{151}{103} \cdot \frac{103}{151})^{6} \cdot 1 = 1^{6} \cdot 1 = 1$$
\rozwStop
\odpStart
$1$
\odpStop
\testStart
A.$1$ B.$\pi$ C.$0$ D.$\frac{151}{103}$ E.$\frac{103}{151}$
F.$-\frac{151}{103}$ G.$-1$
H.$(\frac{151}{103})^{6}$
I.$(\frac{103}{151})^{6}$
\testStop
\kluczStart
A
\kluczStop



\zadStart{Zadanie z Wikieł Z 1.1 d) moja wersja nr 426}

Obliczyć wartość wyrażenia $(\frac{151}{107})^{6} \cdot (\frac{107}{151})^{6} \cdot \pi^{0}$.
\zadStop
\rozwStart{Patryk Wirkus}{Martyna Czarnobaj}
$$(\frac{151}{107})^{6} \cdot (\frac{107}{151})^{6} \cdot \pi^{0} = (\frac{151}{107} \cdot \frac{107}{151})^{6} \cdot 1 = 1^{6} \cdot 1 = 1$$
\rozwStop
\odpStart
$1$
\odpStop
\testStart
A.$1$ B.$\pi$ C.$0$ D.$\frac{151}{107}$ E.$\frac{107}{151}$
F.$-\frac{151}{107}$ G.$-1$
H.$(\frac{151}{107})^{6}$
I.$(\frac{107}{151})^{6}$
\testStop
\kluczStart
A
\kluczStop



\zadStart{Zadanie z Wikieł Z 1.1 d) moja wersja nr 427}

Obliczyć wartość wyrażenia $(\frac{151}{109})^{6} \cdot (\frac{109}{151})^{6} \cdot \pi^{0}$.
\zadStop
\rozwStart{Patryk Wirkus}{Martyna Czarnobaj}
$$(\frac{151}{109})^{6} \cdot (\frac{109}{151})^{6} \cdot \pi^{0} = (\frac{151}{109} \cdot \frac{109}{151})^{6} \cdot 1 = 1^{6} \cdot 1 = 1$$
\rozwStop
\odpStart
$1$
\odpStop
\testStart
A.$1$ B.$\pi$ C.$0$ D.$\frac{151}{109}$ E.$\frac{109}{151}$
F.$-\frac{151}{109}$ G.$-1$
H.$(\frac{151}{109})^{6}$
I.$(\frac{109}{151})^{6}$
\testStop
\kluczStart
A
\kluczStop



\zadStart{Zadanie z Wikieł Z 1.1 d) moja wersja nr 428}

Obliczyć wartość wyrażenia $(\frac{151}{113})^{6} \cdot (\frac{113}{151})^{6} \cdot \pi^{0}$.
\zadStop
\rozwStart{Patryk Wirkus}{Martyna Czarnobaj}
$$(\frac{151}{113})^{6} \cdot (\frac{113}{151})^{6} \cdot \pi^{0} = (\frac{151}{113} \cdot \frac{113}{151})^{6} \cdot 1 = 1^{6} \cdot 1 = 1$$
\rozwStop
\odpStart
$1$
\odpStop
\testStart
A.$1$ B.$\pi$ C.$0$ D.$\frac{151}{113}$ E.$\frac{113}{151}$
F.$-\frac{151}{113}$ G.$-1$
H.$(\frac{151}{113})^{6}$
I.$(\frac{113}{151})^{6}$
\testStop
\kluczStart
A
\kluczStop



\zadStart{Zadanie z Wikieł Z 1.1 d) moja wersja nr 429}

Obliczyć wartość wyrażenia $(\frac{151}{127})^{6} \cdot (\frac{127}{151})^{6} \cdot \pi^{0}$.
\zadStop
\rozwStart{Patryk Wirkus}{Martyna Czarnobaj}
$$(\frac{151}{127})^{6} \cdot (\frac{127}{151})^{6} \cdot \pi^{0} = (\frac{151}{127} \cdot \frac{127}{151})^{6} \cdot 1 = 1^{6} \cdot 1 = 1$$
\rozwStop
\odpStart
$1$
\odpStop
\testStart
A.$1$ B.$\pi$ C.$0$ D.$\frac{151}{127}$ E.$\frac{127}{151}$
F.$-\frac{151}{127}$ G.$-1$
H.$(\frac{151}{127})^{6}$
I.$(\frac{127}{151})^{6}$
\testStop
\kluczStart
A
\kluczStop



\zadStart{Zadanie z Wikieł Z 1.1 d) moja wersja nr 430}

Obliczyć wartość wyrażenia $(\frac{151}{131})^{6} \cdot (\frac{131}{151})^{6} \cdot \pi^{0}$.
\zadStop
\rozwStart{Patryk Wirkus}{Martyna Czarnobaj}
$$(\frac{151}{131})^{6} \cdot (\frac{131}{151})^{6} \cdot \pi^{0} = (\frac{151}{131} \cdot \frac{131}{151})^{6} \cdot 1 = 1^{6} \cdot 1 = 1$$
\rozwStop
\odpStart
$1$
\odpStop
\testStart
A.$1$ B.$\pi$ C.$0$ D.$\frac{151}{131}$ E.$\frac{131}{151}$
F.$-\frac{151}{131}$ G.$-1$
H.$(\frac{151}{131})^{6}$
I.$(\frac{131}{151})^{6}$
\testStop
\kluczStart
A
\kluczStop



\zadStart{Zadanie z Wikieł Z 1.1 d) moja wersja nr 431}

Obliczyć wartość wyrażenia $(\frac{151}{137})^{6} \cdot (\frac{137}{151})^{6} \cdot \pi^{0}$.
\zadStop
\rozwStart{Patryk Wirkus}{Martyna Czarnobaj}
$$(\frac{151}{137})^{6} \cdot (\frac{137}{151})^{6} \cdot \pi^{0} = (\frac{151}{137} \cdot \frac{137}{151})^{6} \cdot 1 = 1^{6} \cdot 1 = 1$$
\rozwStop
\odpStart
$1$
\odpStop
\testStart
A.$1$ B.$\pi$ C.$0$ D.$\frac{151}{137}$ E.$\frac{137}{151}$
F.$-\frac{151}{137}$ G.$-1$
H.$(\frac{151}{137})^{6}$
I.$(\frac{137}{151})^{6}$
\testStop
\kluczStart
A
\kluczStop



\zadStart{Zadanie z Wikieł Z 1.1 d) moja wersja nr 432}

Obliczyć wartość wyrażenia $(\frac{151}{139})^{6} \cdot (\frac{139}{151})^{6} \cdot \pi^{0}$.
\zadStop
\rozwStart{Patryk Wirkus}{Martyna Czarnobaj}
$$(\frac{151}{139})^{6} \cdot (\frac{139}{151})^{6} \cdot \pi^{0} = (\frac{151}{139} \cdot \frac{139}{151})^{6} \cdot 1 = 1^{6} \cdot 1 = 1$$
\rozwStop
\odpStart
$1$
\odpStop
\testStart
A.$1$ B.$\pi$ C.$0$ D.$\frac{151}{139}$ E.$\frac{139}{151}$
F.$-\frac{151}{139}$ G.$-1$
H.$(\frac{151}{139})^{6}$
I.$(\frac{139}{151})^{6}$
\testStop
\kluczStart
A
\kluczStop



\zadStart{Zadanie z Wikieł Z 1.1 d) moja wersja nr 433}

Obliczyć wartość wyrażenia $(\frac{157}{103})^{6} \cdot (\frac{103}{157})^{6} \cdot \pi^{0}$.
\zadStop
\rozwStart{Patryk Wirkus}{Martyna Czarnobaj}
$$(\frac{157}{103})^{6} \cdot (\frac{103}{157})^{6} \cdot \pi^{0} = (\frac{157}{103} \cdot \frac{103}{157})^{6} \cdot 1 = 1^{6} \cdot 1 = 1$$
\rozwStop
\odpStart
$1$
\odpStop
\testStart
A.$1$ B.$\pi$ C.$0$ D.$\frac{157}{103}$ E.$\frac{103}{157}$
F.$-\frac{157}{103}$ G.$-1$
H.$(\frac{157}{103})^{6}$
I.$(\frac{103}{157})^{6}$
\testStop
\kluczStart
A
\kluczStop



\zadStart{Zadanie z Wikieł Z 1.1 d) moja wersja nr 434}

Obliczyć wartość wyrażenia $(\frac{157}{107})^{6} \cdot (\frac{107}{157})^{6} \cdot \pi^{0}$.
\zadStop
\rozwStart{Patryk Wirkus}{Martyna Czarnobaj}
$$(\frac{157}{107})^{6} \cdot (\frac{107}{157})^{6} \cdot \pi^{0} = (\frac{157}{107} \cdot \frac{107}{157})^{6} \cdot 1 = 1^{6} \cdot 1 = 1$$
\rozwStop
\odpStart
$1$
\odpStop
\testStart
A.$1$ B.$\pi$ C.$0$ D.$\frac{157}{107}$ E.$\frac{107}{157}$
F.$-\frac{157}{107}$ G.$-1$
H.$(\frac{157}{107})^{6}$
I.$(\frac{107}{157})^{6}$
\testStop
\kluczStart
A
\kluczStop



\zadStart{Zadanie z Wikieł Z 1.1 d) moja wersja nr 435}

Obliczyć wartość wyrażenia $(\frac{157}{109})^{6} \cdot (\frac{109}{157})^{6} \cdot \pi^{0}$.
\zadStop
\rozwStart{Patryk Wirkus}{Martyna Czarnobaj}
$$(\frac{157}{109})^{6} \cdot (\frac{109}{157})^{6} \cdot \pi^{0} = (\frac{157}{109} \cdot \frac{109}{157})^{6} \cdot 1 = 1^{6} \cdot 1 = 1$$
\rozwStop
\odpStart
$1$
\odpStop
\testStart
A.$1$ B.$\pi$ C.$0$ D.$\frac{157}{109}$ E.$\frac{109}{157}$
F.$-\frac{157}{109}$ G.$-1$
H.$(\frac{157}{109})^{6}$
I.$(\frac{109}{157})^{6}$
\testStop
\kluczStart
A
\kluczStop



\zadStart{Zadanie z Wikieł Z 1.1 d) moja wersja nr 436}

Obliczyć wartość wyrażenia $(\frac{157}{113})^{6} \cdot (\frac{113}{157})^{6} \cdot \pi^{0}$.
\zadStop
\rozwStart{Patryk Wirkus}{Martyna Czarnobaj}
$$(\frac{157}{113})^{6} \cdot (\frac{113}{157})^{6} \cdot \pi^{0} = (\frac{157}{113} \cdot \frac{113}{157})^{6} \cdot 1 = 1^{6} \cdot 1 = 1$$
\rozwStop
\odpStart
$1$
\odpStop
\testStart
A.$1$ B.$\pi$ C.$0$ D.$\frac{157}{113}$ E.$\frac{113}{157}$
F.$-\frac{157}{113}$ G.$-1$
H.$(\frac{157}{113})^{6}$
I.$(\frac{113}{157})^{6}$
\testStop
\kluczStart
A
\kluczStop



\zadStart{Zadanie z Wikieł Z 1.1 d) moja wersja nr 437}

Obliczyć wartość wyrażenia $(\frac{157}{127})^{6} \cdot (\frac{127}{157})^{6} \cdot \pi^{0}$.
\zadStop
\rozwStart{Patryk Wirkus}{Martyna Czarnobaj}
$$(\frac{157}{127})^{6} \cdot (\frac{127}{157})^{6} \cdot \pi^{0} = (\frac{157}{127} \cdot \frac{127}{157})^{6} \cdot 1 = 1^{6} \cdot 1 = 1$$
\rozwStop
\odpStart
$1$
\odpStop
\testStart
A.$1$ B.$\pi$ C.$0$ D.$\frac{157}{127}$ E.$\frac{127}{157}$
F.$-\frac{157}{127}$ G.$-1$
H.$(\frac{157}{127})^{6}$
I.$(\frac{127}{157})^{6}$
\testStop
\kluczStart
A
\kluczStop



\zadStart{Zadanie z Wikieł Z 1.1 d) moja wersja nr 438}

Obliczyć wartość wyrażenia $(\frac{157}{131})^{6} \cdot (\frac{131}{157})^{6} \cdot \pi^{0}$.
\zadStop
\rozwStart{Patryk Wirkus}{Martyna Czarnobaj}
$$(\frac{157}{131})^{6} \cdot (\frac{131}{157})^{6} \cdot \pi^{0} = (\frac{157}{131} \cdot \frac{131}{157})^{6} \cdot 1 = 1^{6} \cdot 1 = 1$$
\rozwStop
\odpStart
$1$
\odpStop
\testStart
A.$1$ B.$\pi$ C.$0$ D.$\frac{157}{131}$ E.$\frac{131}{157}$
F.$-\frac{157}{131}$ G.$-1$
H.$(\frac{157}{131})^{6}$
I.$(\frac{131}{157})^{6}$
\testStop
\kluczStart
A
\kluczStop



\zadStart{Zadanie z Wikieł Z 1.1 d) moja wersja nr 439}

Obliczyć wartość wyrażenia $(\frac{157}{137})^{6} \cdot (\frac{137}{157})^{6} \cdot \pi^{0}$.
\zadStop
\rozwStart{Patryk Wirkus}{Martyna Czarnobaj}
$$(\frac{157}{137})^{6} \cdot (\frac{137}{157})^{6} \cdot \pi^{0} = (\frac{157}{137} \cdot \frac{137}{157})^{6} \cdot 1 = 1^{6} \cdot 1 = 1$$
\rozwStop
\odpStart
$1$
\odpStop
\testStart
A.$1$ B.$\pi$ C.$0$ D.$\frac{157}{137}$ E.$\frac{137}{157}$
F.$-\frac{157}{137}$ G.$-1$
H.$(\frac{157}{137})^{6}$
I.$(\frac{137}{157})^{6}$
\testStop
\kluczStart
A
\kluczStop



\zadStart{Zadanie z Wikieł Z 1.1 d) moja wersja nr 440}

Obliczyć wartość wyrażenia $(\frac{157}{139})^{6} \cdot (\frac{139}{157})^{6} \cdot \pi^{0}$.
\zadStop
\rozwStart{Patryk Wirkus}{Martyna Czarnobaj}
$$(\frac{157}{139})^{6} \cdot (\frac{139}{157})^{6} \cdot \pi^{0} = (\frac{157}{139} \cdot \frac{139}{157})^{6} \cdot 1 = 1^{6} \cdot 1 = 1$$
\rozwStop
\odpStart
$1$
\odpStop
\testStart
A.$1$ B.$\pi$ C.$0$ D.$\frac{157}{139}$ E.$\frac{139}{157}$
F.$-\frac{157}{139}$ G.$-1$
H.$(\frac{157}{139})^{6}$
I.$(\frac{139}{157})^{6}$
\testStop
\kluczStart
A
\kluczStop



\zadStart{Zadanie z Wikieł Z 1.1 d) moja wersja nr 441}

Obliczyć wartość wyrażenia $(\frac{163}{103})^{6} \cdot (\frac{103}{163})^{6} \cdot \pi^{0}$.
\zadStop
\rozwStart{Patryk Wirkus}{Martyna Czarnobaj}
$$(\frac{163}{103})^{6} \cdot (\frac{103}{163})^{6} \cdot \pi^{0} = (\frac{163}{103} \cdot \frac{103}{163})^{6} \cdot 1 = 1^{6} \cdot 1 = 1$$
\rozwStop
\odpStart
$1$
\odpStop
\testStart
A.$1$ B.$\pi$ C.$0$ D.$\frac{163}{103}$ E.$\frac{103}{163}$
F.$-\frac{163}{103}$ G.$-1$
H.$(\frac{163}{103})^{6}$
I.$(\frac{103}{163})^{6}$
\testStop
\kluczStart
A
\kluczStop



\zadStart{Zadanie z Wikieł Z 1.1 d) moja wersja nr 442}

Obliczyć wartość wyrażenia $(\frac{163}{107})^{6} \cdot (\frac{107}{163})^{6} \cdot \pi^{0}$.
\zadStop
\rozwStart{Patryk Wirkus}{Martyna Czarnobaj}
$$(\frac{163}{107})^{6} \cdot (\frac{107}{163})^{6} \cdot \pi^{0} = (\frac{163}{107} \cdot \frac{107}{163})^{6} \cdot 1 = 1^{6} \cdot 1 = 1$$
\rozwStop
\odpStart
$1$
\odpStop
\testStart
A.$1$ B.$\pi$ C.$0$ D.$\frac{163}{107}$ E.$\frac{107}{163}$
F.$-\frac{163}{107}$ G.$-1$
H.$(\frac{163}{107})^{6}$
I.$(\frac{107}{163})^{6}$
\testStop
\kluczStart
A
\kluczStop



\zadStart{Zadanie z Wikieł Z 1.1 d) moja wersja nr 443}

Obliczyć wartość wyrażenia $(\frac{163}{109})^{6} \cdot (\frac{109}{163})^{6} \cdot \pi^{0}$.
\zadStop
\rozwStart{Patryk Wirkus}{Martyna Czarnobaj}
$$(\frac{163}{109})^{6} \cdot (\frac{109}{163})^{6} \cdot \pi^{0} = (\frac{163}{109} \cdot \frac{109}{163})^{6} \cdot 1 = 1^{6} \cdot 1 = 1$$
\rozwStop
\odpStart
$1$
\odpStop
\testStart
A.$1$ B.$\pi$ C.$0$ D.$\frac{163}{109}$ E.$\frac{109}{163}$
F.$-\frac{163}{109}$ G.$-1$
H.$(\frac{163}{109})^{6}$
I.$(\frac{109}{163})^{6}$
\testStop
\kluczStart
A
\kluczStop



\zadStart{Zadanie z Wikieł Z 1.1 d) moja wersja nr 444}

Obliczyć wartość wyrażenia $(\frac{163}{113})^{6} \cdot (\frac{113}{163})^{6} \cdot \pi^{0}$.
\zadStop
\rozwStart{Patryk Wirkus}{Martyna Czarnobaj}
$$(\frac{163}{113})^{6} \cdot (\frac{113}{163})^{6} \cdot \pi^{0} = (\frac{163}{113} \cdot \frac{113}{163})^{6} \cdot 1 = 1^{6} \cdot 1 = 1$$
\rozwStop
\odpStart
$1$
\odpStop
\testStart
A.$1$ B.$\pi$ C.$0$ D.$\frac{163}{113}$ E.$\frac{113}{163}$
F.$-\frac{163}{113}$ G.$-1$
H.$(\frac{163}{113})^{6}$
I.$(\frac{113}{163})^{6}$
\testStop
\kluczStart
A
\kluczStop



\zadStart{Zadanie z Wikieł Z 1.1 d) moja wersja nr 445}

Obliczyć wartość wyrażenia $(\frac{163}{127})^{6} \cdot (\frac{127}{163})^{6} \cdot \pi^{0}$.
\zadStop
\rozwStart{Patryk Wirkus}{Martyna Czarnobaj}
$$(\frac{163}{127})^{6} \cdot (\frac{127}{163})^{6} \cdot \pi^{0} = (\frac{163}{127} \cdot \frac{127}{163})^{6} \cdot 1 = 1^{6} \cdot 1 = 1$$
\rozwStop
\odpStart
$1$
\odpStop
\testStart
A.$1$ B.$\pi$ C.$0$ D.$\frac{163}{127}$ E.$\frac{127}{163}$
F.$-\frac{163}{127}$ G.$-1$
H.$(\frac{163}{127})^{6}$
I.$(\frac{127}{163})^{6}$
\testStop
\kluczStart
A
\kluczStop



\zadStart{Zadanie z Wikieł Z 1.1 d) moja wersja nr 446}

Obliczyć wartość wyrażenia $(\frac{163}{131})^{6} \cdot (\frac{131}{163})^{6} \cdot \pi^{0}$.
\zadStop
\rozwStart{Patryk Wirkus}{Martyna Czarnobaj}
$$(\frac{163}{131})^{6} \cdot (\frac{131}{163})^{6} \cdot \pi^{0} = (\frac{163}{131} \cdot \frac{131}{163})^{6} \cdot 1 = 1^{6} \cdot 1 = 1$$
\rozwStop
\odpStart
$1$
\odpStop
\testStart
A.$1$ B.$\pi$ C.$0$ D.$\frac{163}{131}$ E.$\frac{131}{163}$
F.$-\frac{163}{131}$ G.$-1$
H.$(\frac{163}{131})^{6}$
I.$(\frac{131}{163})^{6}$
\testStop
\kluczStart
A
\kluczStop



\zadStart{Zadanie z Wikieł Z 1.1 d) moja wersja nr 447}

Obliczyć wartość wyrażenia $(\frac{163}{137})^{6} \cdot (\frac{137}{163})^{6} \cdot \pi^{0}$.
\zadStop
\rozwStart{Patryk Wirkus}{Martyna Czarnobaj}
$$(\frac{163}{137})^{6} \cdot (\frac{137}{163})^{6} \cdot \pi^{0} = (\frac{163}{137} \cdot \frac{137}{163})^{6} \cdot 1 = 1^{6} \cdot 1 = 1$$
\rozwStop
\odpStart
$1$
\odpStop
\testStart
A.$1$ B.$\pi$ C.$0$ D.$\frac{163}{137}$ E.$\frac{137}{163}$
F.$-\frac{163}{137}$ G.$-1$
H.$(\frac{163}{137})^{6}$
I.$(\frac{137}{163})^{6}$
\testStop
\kluczStart
A
\kluczStop



\zadStart{Zadanie z Wikieł Z 1.1 d) moja wersja nr 448}

Obliczyć wartość wyrażenia $(\frac{163}{139})^{6} \cdot (\frac{139}{163})^{6} \cdot \pi^{0}$.
\zadStop
\rozwStart{Patryk Wirkus}{Martyna Czarnobaj}
$$(\frac{163}{139})^{6} \cdot (\frac{139}{163})^{6} \cdot \pi^{0} = (\frac{163}{139} \cdot \frac{139}{163})^{6} \cdot 1 = 1^{6} \cdot 1 = 1$$
\rozwStop
\odpStart
$1$
\odpStop
\testStart
A.$1$ B.$\pi$ C.$0$ D.$\frac{163}{139}$ E.$\frac{139}{163}$
F.$-\frac{163}{139}$ G.$-1$
H.$(\frac{163}{139})^{6}$
I.$(\frac{139}{163})^{6}$
\testStop
\kluczStart
A
\kluczStop



\zadStart{Zadanie z Wikieł Z 1.1 d) moja wersja nr 449}

Obliczyć wartość wyrażenia $(\frac{167}{103})^{6} \cdot (\frac{103}{167})^{6} \cdot \pi^{0}$.
\zadStop
\rozwStart{Patryk Wirkus}{Martyna Czarnobaj}
$$(\frac{167}{103})^{6} \cdot (\frac{103}{167})^{6} \cdot \pi^{0} = (\frac{167}{103} \cdot \frac{103}{167})^{6} \cdot 1 = 1^{6} \cdot 1 = 1$$
\rozwStop
\odpStart
$1$
\odpStop
\testStart
A.$1$ B.$\pi$ C.$0$ D.$\frac{167}{103}$ E.$\frac{103}{167}$
F.$-\frac{167}{103}$ G.$-1$
H.$(\frac{167}{103})^{6}$
I.$(\frac{103}{167})^{6}$
\testStop
\kluczStart
A
\kluczStop



\zadStart{Zadanie z Wikieł Z 1.1 d) moja wersja nr 450}

Obliczyć wartość wyrażenia $(\frac{167}{107})^{6} \cdot (\frac{107}{167})^{6} \cdot \pi^{0}$.
\zadStop
\rozwStart{Patryk Wirkus}{Martyna Czarnobaj}
$$(\frac{167}{107})^{6} \cdot (\frac{107}{167})^{6} \cdot \pi^{0} = (\frac{167}{107} \cdot \frac{107}{167})^{6} \cdot 1 = 1^{6} \cdot 1 = 1$$
\rozwStop
\odpStart
$1$
\odpStop
\testStart
A.$1$ B.$\pi$ C.$0$ D.$\frac{167}{107}$ E.$\frac{107}{167}$
F.$-\frac{167}{107}$ G.$-1$
H.$(\frac{167}{107})^{6}$
I.$(\frac{107}{167})^{6}$
\testStop
\kluczStart
A
\kluczStop



\zadStart{Zadanie z Wikieł Z 1.1 d) moja wersja nr 451}

Obliczyć wartość wyrażenia $(\frac{167}{109})^{6} \cdot (\frac{109}{167})^{6} \cdot \pi^{0}$.
\zadStop
\rozwStart{Patryk Wirkus}{Martyna Czarnobaj}
$$(\frac{167}{109})^{6} \cdot (\frac{109}{167})^{6} \cdot \pi^{0} = (\frac{167}{109} \cdot \frac{109}{167})^{6} \cdot 1 = 1^{6} \cdot 1 = 1$$
\rozwStop
\odpStart
$1$
\odpStop
\testStart
A.$1$ B.$\pi$ C.$0$ D.$\frac{167}{109}$ E.$\frac{109}{167}$
F.$-\frac{167}{109}$ G.$-1$
H.$(\frac{167}{109})^{6}$
I.$(\frac{109}{167})^{6}$
\testStop
\kluczStart
A
\kluczStop



\zadStart{Zadanie z Wikieł Z 1.1 d) moja wersja nr 452}

Obliczyć wartość wyrażenia $(\frac{167}{113})^{6} \cdot (\frac{113}{167})^{6} \cdot \pi^{0}$.
\zadStop
\rozwStart{Patryk Wirkus}{Martyna Czarnobaj}
$$(\frac{167}{113})^{6} \cdot (\frac{113}{167})^{6} \cdot \pi^{0} = (\frac{167}{113} \cdot \frac{113}{167})^{6} \cdot 1 = 1^{6} \cdot 1 = 1$$
\rozwStop
\odpStart
$1$
\odpStop
\testStart
A.$1$ B.$\pi$ C.$0$ D.$\frac{167}{113}$ E.$\frac{113}{167}$
F.$-\frac{167}{113}$ G.$-1$
H.$(\frac{167}{113})^{6}$
I.$(\frac{113}{167})^{6}$
\testStop
\kluczStart
A
\kluczStop



\zadStart{Zadanie z Wikieł Z 1.1 d) moja wersja nr 453}

Obliczyć wartość wyrażenia $(\frac{167}{127})^{6} \cdot (\frac{127}{167})^{6} \cdot \pi^{0}$.
\zadStop
\rozwStart{Patryk Wirkus}{Martyna Czarnobaj}
$$(\frac{167}{127})^{6} \cdot (\frac{127}{167})^{6} \cdot \pi^{0} = (\frac{167}{127} \cdot \frac{127}{167})^{6} \cdot 1 = 1^{6} \cdot 1 = 1$$
\rozwStop
\odpStart
$1$
\odpStop
\testStart
A.$1$ B.$\pi$ C.$0$ D.$\frac{167}{127}$ E.$\frac{127}{167}$
F.$-\frac{167}{127}$ G.$-1$
H.$(\frac{167}{127})^{6}$
I.$(\frac{127}{167})^{6}$
\testStop
\kluczStart
A
\kluczStop



\zadStart{Zadanie z Wikieł Z 1.1 d) moja wersja nr 454}

Obliczyć wartość wyrażenia $(\frac{167}{131})^{6} \cdot (\frac{131}{167})^{6} \cdot \pi^{0}$.
\zadStop
\rozwStart{Patryk Wirkus}{Martyna Czarnobaj}
$$(\frac{167}{131})^{6} \cdot (\frac{131}{167})^{6} \cdot \pi^{0} = (\frac{167}{131} \cdot \frac{131}{167})^{6} \cdot 1 = 1^{6} \cdot 1 = 1$$
\rozwStop
\odpStart
$1$
\odpStop
\testStart
A.$1$ B.$\pi$ C.$0$ D.$\frac{167}{131}$ E.$\frac{131}{167}$
F.$-\frac{167}{131}$ G.$-1$
H.$(\frac{167}{131})^{6}$
I.$(\frac{131}{167})^{6}$
\testStop
\kluczStart
A
\kluczStop



\zadStart{Zadanie z Wikieł Z 1.1 d) moja wersja nr 455}

Obliczyć wartość wyrażenia $(\frac{167}{137})^{6} \cdot (\frac{137}{167})^{6} \cdot \pi^{0}$.
\zadStop
\rozwStart{Patryk Wirkus}{Martyna Czarnobaj}
$$(\frac{167}{137})^{6} \cdot (\frac{137}{167})^{6} \cdot \pi^{0} = (\frac{167}{137} \cdot \frac{137}{167})^{6} \cdot 1 = 1^{6} \cdot 1 = 1$$
\rozwStop
\odpStart
$1$
\odpStop
\testStart
A.$1$ B.$\pi$ C.$0$ D.$\frac{167}{137}$ E.$\frac{137}{167}$
F.$-\frac{167}{137}$ G.$-1$
H.$(\frac{167}{137})^{6}$
I.$(\frac{137}{167})^{6}$
\testStop
\kluczStart
A
\kluczStop



\zadStart{Zadanie z Wikieł Z 1.1 d) moja wersja nr 456}

Obliczyć wartość wyrażenia $(\frac{167}{139})^{6} \cdot (\frac{139}{167})^{6} \cdot \pi^{0}$.
\zadStop
\rozwStart{Patryk Wirkus}{Martyna Czarnobaj}
$$(\frac{167}{139})^{6} \cdot (\frac{139}{167})^{6} \cdot \pi^{0} = (\frac{167}{139} \cdot \frac{139}{167})^{6} \cdot 1 = 1^{6} \cdot 1 = 1$$
\rozwStop
\odpStart
$1$
\odpStop
\testStart
A.$1$ B.$\pi$ C.$0$ D.$\frac{167}{139}$ E.$\frac{139}{167}$
F.$-\frac{167}{139}$ G.$-1$
H.$(\frac{167}{139})^{6}$
I.$(\frac{139}{167})^{6}$
\testStop
\kluczStart
A
\kluczStop



\zadStart{Zadanie z Wikieł Z 1.1 d) moja wersja nr 457}

Obliczyć wartość wyrażenia $(\frac{173}{103})^{6} \cdot (\frac{103}{173})^{6} \cdot \pi^{0}$.
\zadStop
\rozwStart{Patryk Wirkus}{Martyna Czarnobaj}
$$(\frac{173}{103})^{6} \cdot (\frac{103}{173})^{6} \cdot \pi^{0} = (\frac{173}{103} \cdot \frac{103}{173})^{6} \cdot 1 = 1^{6} \cdot 1 = 1$$
\rozwStop
\odpStart
$1$
\odpStop
\testStart
A.$1$ B.$\pi$ C.$0$ D.$\frac{173}{103}$ E.$\frac{103}{173}$
F.$-\frac{173}{103}$ G.$-1$
H.$(\frac{173}{103})^{6}$
I.$(\frac{103}{173})^{6}$
\testStop
\kluczStart
A
\kluczStop



\zadStart{Zadanie z Wikieł Z 1.1 d) moja wersja nr 458}

Obliczyć wartość wyrażenia $(\frac{173}{107})^{6} \cdot (\frac{107}{173})^{6} \cdot \pi^{0}$.
\zadStop
\rozwStart{Patryk Wirkus}{Martyna Czarnobaj}
$$(\frac{173}{107})^{6} \cdot (\frac{107}{173})^{6} \cdot \pi^{0} = (\frac{173}{107} \cdot \frac{107}{173})^{6} \cdot 1 = 1^{6} \cdot 1 = 1$$
\rozwStop
\odpStart
$1$
\odpStop
\testStart
A.$1$ B.$\pi$ C.$0$ D.$\frac{173}{107}$ E.$\frac{107}{173}$
F.$-\frac{173}{107}$ G.$-1$
H.$(\frac{173}{107})^{6}$
I.$(\frac{107}{173})^{6}$
\testStop
\kluczStart
A
\kluczStop



\zadStart{Zadanie z Wikieł Z 1.1 d) moja wersja nr 459}

Obliczyć wartość wyrażenia $(\frac{173}{109})^{6} \cdot (\frac{109}{173})^{6} \cdot \pi^{0}$.
\zadStop
\rozwStart{Patryk Wirkus}{Martyna Czarnobaj}
$$(\frac{173}{109})^{6} \cdot (\frac{109}{173})^{6} \cdot \pi^{0} = (\frac{173}{109} \cdot \frac{109}{173})^{6} \cdot 1 = 1^{6} \cdot 1 = 1$$
\rozwStop
\odpStart
$1$
\odpStop
\testStart
A.$1$ B.$\pi$ C.$0$ D.$\frac{173}{109}$ E.$\frac{109}{173}$
F.$-\frac{173}{109}$ G.$-1$
H.$(\frac{173}{109})^{6}$
I.$(\frac{109}{173})^{6}$
\testStop
\kluczStart
A
\kluczStop



\zadStart{Zadanie z Wikieł Z 1.1 d) moja wersja nr 460}

Obliczyć wartość wyrażenia $(\frac{173}{113})^{6} \cdot (\frac{113}{173})^{6} \cdot \pi^{0}$.
\zadStop
\rozwStart{Patryk Wirkus}{Martyna Czarnobaj}
$$(\frac{173}{113})^{6} \cdot (\frac{113}{173})^{6} \cdot \pi^{0} = (\frac{173}{113} \cdot \frac{113}{173})^{6} \cdot 1 = 1^{6} \cdot 1 = 1$$
\rozwStop
\odpStart
$1$
\odpStop
\testStart
A.$1$ B.$\pi$ C.$0$ D.$\frac{173}{113}$ E.$\frac{113}{173}$
F.$-\frac{173}{113}$ G.$-1$
H.$(\frac{173}{113})^{6}$
I.$(\frac{113}{173})^{6}$
\testStop
\kluczStart
A
\kluczStop



\zadStart{Zadanie z Wikieł Z 1.1 d) moja wersja nr 461}

Obliczyć wartość wyrażenia $(\frac{173}{127})^{6} \cdot (\frac{127}{173})^{6} \cdot \pi^{0}$.
\zadStop
\rozwStart{Patryk Wirkus}{Martyna Czarnobaj}
$$(\frac{173}{127})^{6} \cdot (\frac{127}{173})^{6} \cdot \pi^{0} = (\frac{173}{127} \cdot \frac{127}{173})^{6} \cdot 1 = 1^{6} \cdot 1 = 1$$
\rozwStop
\odpStart
$1$
\odpStop
\testStart
A.$1$ B.$\pi$ C.$0$ D.$\frac{173}{127}$ E.$\frac{127}{173}$
F.$-\frac{173}{127}$ G.$-1$
H.$(\frac{173}{127})^{6}$
I.$(\frac{127}{173})^{6}$
\testStop
\kluczStart
A
\kluczStop



\zadStart{Zadanie z Wikieł Z 1.1 d) moja wersja nr 462}

Obliczyć wartość wyrażenia $(\frac{173}{131})^{6} \cdot (\frac{131}{173})^{6} \cdot \pi^{0}$.
\zadStop
\rozwStart{Patryk Wirkus}{Martyna Czarnobaj}
$$(\frac{173}{131})^{6} \cdot (\frac{131}{173})^{6} \cdot \pi^{0} = (\frac{173}{131} \cdot \frac{131}{173})^{6} \cdot 1 = 1^{6} \cdot 1 = 1$$
\rozwStop
\odpStart
$1$
\odpStop
\testStart
A.$1$ B.$\pi$ C.$0$ D.$\frac{173}{131}$ E.$\frac{131}{173}$
F.$-\frac{173}{131}$ G.$-1$
H.$(\frac{173}{131})^{6}$
I.$(\frac{131}{173})^{6}$
\testStop
\kluczStart
A
\kluczStop



\zadStart{Zadanie z Wikieł Z 1.1 d) moja wersja nr 463}

Obliczyć wartość wyrażenia $(\frac{173}{137})^{6} \cdot (\frac{137}{173})^{6} \cdot \pi^{0}$.
\zadStop
\rozwStart{Patryk Wirkus}{Martyna Czarnobaj}
$$(\frac{173}{137})^{6} \cdot (\frac{137}{173})^{6} \cdot \pi^{0} = (\frac{173}{137} \cdot \frac{137}{173})^{6} \cdot 1 = 1^{6} \cdot 1 = 1$$
\rozwStop
\odpStart
$1$
\odpStop
\testStart
A.$1$ B.$\pi$ C.$0$ D.$\frac{173}{137}$ E.$\frac{137}{173}$
F.$-\frac{173}{137}$ G.$-1$
H.$(\frac{173}{137})^{6}$
I.$(\frac{137}{173})^{6}$
\testStop
\kluczStart
A
\kluczStop



\zadStart{Zadanie z Wikieł Z 1.1 d) moja wersja nr 464}

Obliczyć wartość wyrażenia $(\frac{173}{139})^{6} \cdot (\frac{139}{173})^{6} \cdot \pi^{0}$.
\zadStop
\rozwStart{Patryk Wirkus}{Martyna Czarnobaj}
$$(\frac{173}{139})^{6} \cdot (\frac{139}{173})^{6} \cdot \pi^{0} = (\frac{173}{139} \cdot \frac{139}{173})^{6} \cdot 1 = 1^{6} \cdot 1 = 1$$
\rozwStop
\odpStart
$1$
\odpStop
\testStart
A.$1$ B.$\pi$ C.$0$ D.$\frac{173}{139}$ E.$\frac{139}{173}$
F.$-\frac{173}{139}$ G.$-1$
H.$(\frac{173}{139})^{6}$
I.$(\frac{139}{173})^{6}$
\testStop
\kluczStart
A
\kluczStop



\zadStart{Zadanie z Wikieł Z 1.1 d) moja wersja nr 465}

Obliczyć wartość wyrażenia $(\frac{179}{103})^{6} \cdot (\frac{103}{179})^{6} \cdot \pi^{0}$.
\zadStop
\rozwStart{Patryk Wirkus}{Martyna Czarnobaj}
$$(\frac{179}{103})^{6} \cdot (\frac{103}{179})^{6} \cdot \pi^{0} = (\frac{179}{103} \cdot \frac{103}{179})^{6} \cdot 1 = 1^{6} \cdot 1 = 1$$
\rozwStop
\odpStart
$1$
\odpStop
\testStart
A.$1$ B.$\pi$ C.$0$ D.$\frac{179}{103}$ E.$\frac{103}{179}$
F.$-\frac{179}{103}$ G.$-1$
H.$(\frac{179}{103})^{6}$
I.$(\frac{103}{179})^{6}$
\testStop
\kluczStart
A
\kluczStop



\zadStart{Zadanie z Wikieł Z 1.1 d) moja wersja nr 466}

Obliczyć wartość wyrażenia $(\frac{179}{107})^{6} \cdot (\frac{107}{179})^{6} \cdot \pi^{0}$.
\zadStop
\rozwStart{Patryk Wirkus}{Martyna Czarnobaj}
$$(\frac{179}{107})^{6} \cdot (\frac{107}{179})^{6} \cdot \pi^{0} = (\frac{179}{107} \cdot \frac{107}{179})^{6} \cdot 1 = 1^{6} \cdot 1 = 1$$
\rozwStop
\odpStart
$1$
\odpStop
\testStart
A.$1$ B.$\pi$ C.$0$ D.$\frac{179}{107}$ E.$\frac{107}{179}$
F.$-\frac{179}{107}$ G.$-1$
H.$(\frac{179}{107})^{6}$
I.$(\frac{107}{179})^{6}$
\testStop
\kluczStart
A
\kluczStop



\zadStart{Zadanie z Wikieł Z 1.1 d) moja wersja nr 467}

Obliczyć wartość wyrażenia $(\frac{179}{109})^{6} \cdot (\frac{109}{179})^{6} \cdot \pi^{0}$.
\zadStop
\rozwStart{Patryk Wirkus}{Martyna Czarnobaj}
$$(\frac{179}{109})^{6} \cdot (\frac{109}{179})^{6} \cdot \pi^{0} = (\frac{179}{109} \cdot \frac{109}{179})^{6} \cdot 1 = 1^{6} \cdot 1 = 1$$
\rozwStop
\odpStart
$1$
\odpStop
\testStart
A.$1$ B.$\pi$ C.$0$ D.$\frac{179}{109}$ E.$\frac{109}{179}$
F.$-\frac{179}{109}$ G.$-1$
H.$(\frac{179}{109})^{6}$
I.$(\frac{109}{179})^{6}$
\testStop
\kluczStart
A
\kluczStop



\zadStart{Zadanie z Wikieł Z 1.1 d) moja wersja nr 468}

Obliczyć wartość wyrażenia $(\frac{179}{113})^{6} \cdot (\frac{113}{179})^{6} \cdot \pi^{0}$.
\zadStop
\rozwStart{Patryk Wirkus}{Martyna Czarnobaj}
$$(\frac{179}{113})^{6} \cdot (\frac{113}{179})^{6} \cdot \pi^{0} = (\frac{179}{113} \cdot \frac{113}{179})^{6} \cdot 1 = 1^{6} \cdot 1 = 1$$
\rozwStop
\odpStart
$1$
\odpStop
\testStart
A.$1$ B.$\pi$ C.$0$ D.$\frac{179}{113}$ E.$\frac{113}{179}$
F.$-\frac{179}{113}$ G.$-1$
H.$(\frac{179}{113})^{6}$
I.$(\frac{113}{179})^{6}$
\testStop
\kluczStart
A
\kluczStop



\zadStart{Zadanie z Wikieł Z 1.1 d) moja wersja nr 469}

Obliczyć wartość wyrażenia $(\frac{179}{127})^{6} \cdot (\frac{127}{179})^{6} \cdot \pi^{0}$.
\zadStop
\rozwStart{Patryk Wirkus}{Martyna Czarnobaj}
$$(\frac{179}{127})^{6} \cdot (\frac{127}{179})^{6} \cdot \pi^{0} = (\frac{179}{127} \cdot \frac{127}{179})^{6} \cdot 1 = 1^{6} \cdot 1 = 1$$
\rozwStop
\odpStart
$1$
\odpStop
\testStart
A.$1$ B.$\pi$ C.$0$ D.$\frac{179}{127}$ E.$\frac{127}{179}$
F.$-\frac{179}{127}$ G.$-1$
H.$(\frac{179}{127})^{6}$
I.$(\frac{127}{179})^{6}$
\testStop
\kluczStart
A
\kluczStop



\zadStart{Zadanie z Wikieł Z 1.1 d) moja wersja nr 470}

Obliczyć wartość wyrażenia $(\frac{179}{131})^{6} \cdot (\frac{131}{179})^{6} \cdot \pi^{0}$.
\zadStop
\rozwStart{Patryk Wirkus}{Martyna Czarnobaj}
$$(\frac{179}{131})^{6} \cdot (\frac{131}{179})^{6} \cdot \pi^{0} = (\frac{179}{131} \cdot \frac{131}{179})^{6} \cdot 1 = 1^{6} \cdot 1 = 1$$
\rozwStop
\odpStart
$1$
\odpStop
\testStart
A.$1$ B.$\pi$ C.$0$ D.$\frac{179}{131}$ E.$\frac{131}{179}$
F.$-\frac{179}{131}$ G.$-1$
H.$(\frac{179}{131})^{6}$
I.$(\frac{131}{179})^{6}$
\testStop
\kluczStart
A
\kluczStop



\zadStart{Zadanie z Wikieł Z 1.1 d) moja wersja nr 471}

Obliczyć wartość wyrażenia $(\frac{179}{137})^{6} \cdot (\frac{137}{179})^{6} \cdot \pi^{0}$.
\zadStop
\rozwStart{Patryk Wirkus}{Martyna Czarnobaj}
$$(\frac{179}{137})^{6} \cdot (\frac{137}{179})^{6} \cdot \pi^{0} = (\frac{179}{137} \cdot \frac{137}{179})^{6} \cdot 1 = 1^{6} \cdot 1 = 1$$
\rozwStop
\odpStart
$1$
\odpStop
\testStart
A.$1$ B.$\pi$ C.$0$ D.$\frac{179}{137}$ E.$\frac{137}{179}$
F.$-\frac{179}{137}$ G.$-1$
H.$(\frac{179}{137})^{6}$
I.$(\frac{137}{179})^{6}$
\testStop
\kluczStart
A
\kluczStop



\zadStart{Zadanie z Wikieł Z 1.1 d) moja wersja nr 472}

Obliczyć wartość wyrażenia $(\frac{179}{139})^{6} \cdot (\frac{139}{179})^{6} \cdot \pi^{0}$.
\zadStop
\rozwStart{Patryk Wirkus}{Martyna Czarnobaj}
$$(\frac{179}{139})^{6} \cdot (\frac{139}{179})^{6} \cdot \pi^{0} = (\frac{179}{139} \cdot \frac{139}{179})^{6} \cdot 1 = 1^{6} \cdot 1 = 1$$
\rozwStop
\odpStart
$1$
\odpStop
\testStart
A.$1$ B.$\pi$ C.$0$ D.$\frac{179}{139}$ E.$\frac{139}{179}$
F.$-\frac{179}{139}$ G.$-1$
H.$(\frac{179}{139})^{6}$
I.$(\frac{139}{179})^{6}$
\testStop
\kluczStart
A
\kluczStop



\zadStart{Zadanie z Wikieł Z 1.1 d) moja wersja nr 473}

Obliczyć wartość wyrażenia $(\frac{251}{103})^{6} \cdot (\frac{103}{251})^{6} \cdot \pi^{0}$.
\zadStop
\rozwStart{Patryk Wirkus}{Martyna Czarnobaj}
$$(\frac{251}{103})^{6} \cdot (\frac{103}{251})^{6} \cdot \pi^{0} = (\frac{251}{103} \cdot \frac{103}{251})^{6} \cdot 1 = 1^{6} \cdot 1 = 1$$
\rozwStop
\odpStart
$1$
\odpStop
\testStart
A.$1$ B.$\pi$ C.$0$ D.$\frac{251}{103}$ E.$\frac{103}{251}$
F.$-\frac{251}{103}$ G.$-1$
H.$(\frac{251}{103})^{6}$
I.$(\frac{103}{251})^{6}$
\testStop
\kluczStart
A
\kluczStop



\zadStart{Zadanie z Wikieł Z 1.1 d) moja wersja nr 474}

Obliczyć wartość wyrażenia $(\frac{251}{107})^{6} \cdot (\frac{107}{251})^{6} \cdot \pi^{0}$.
\zadStop
\rozwStart{Patryk Wirkus}{Martyna Czarnobaj}
$$(\frac{251}{107})^{6} \cdot (\frac{107}{251})^{6} \cdot \pi^{0} = (\frac{251}{107} \cdot \frac{107}{251})^{6} \cdot 1 = 1^{6} \cdot 1 = 1$$
\rozwStop
\odpStart
$1$
\odpStop
\testStart
A.$1$ B.$\pi$ C.$0$ D.$\frac{251}{107}$ E.$\frac{107}{251}$
F.$-\frac{251}{107}$ G.$-1$
H.$(\frac{251}{107})^{6}$
I.$(\frac{107}{251})^{6}$
\testStop
\kluczStart
A
\kluczStop



\zadStart{Zadanie z Wikieł Z 1.1 d) moja wersja nr 475}

Obliczyć wartość wyrażenia $(\frac{251}{109})^{6} \cdot (\frac{109}{251})^{6} \cdot \pi^{0}$.
\zadStop
\rozwStart{Patryk Wirkus}{Martyna Czarnobaj}
$$(\frac{251}{109})^{6} \cdot (\frac{109}{251})^{6} \cdot \pi^{0} = (\frac{251}{109} \cdot \frac{109}{251})^{6} \cdot 1 = 1^{6} \cdot 1 = 1$$
\rozwStop
\odpStart
$1$
\odpStop
\testStart
A.$1$ B.$\pi$ C.$0$ D.$\frac{251}{109}$ E.$\frac{109}{251}$
F.$-\frac{251}{109}$ G.$-1$
H.$(\frac{251}{109})^{6}$
I.$(\frac{109}{251})^{6}$
\testStop
\kluczStart
A
\kluczStop



\zadStart{Zadanie z Wikieł Z 1.1 d) moja wersja nr 476}

Obliczyć wartość wyrażenia $(\frac{251}{113})^{6} \cdot (\frac{113}{251})^{6} \cdot \pi^{0}$.
\zadStop
\rozwStart{Patryk Wirkus}{Martyna Czarnobaj}
$$(\frac{251}{113})^{6} \cdot (\frac{113}{251})^{6} \cdot \pi^{0} = (\frac{251}{113} \cdot \frac{113}{251})^{6} \cdot 1 = 1^{6} \cdot 1 = 1$$
\rozwStop
\odpStart
$1$
\odpStop
\testStart
A.$1$ B.$\pi$ C.$0$ D.$\frac{251}{113}$ E.$\frac{113}{251}$
F.$-\frac{251}{113}$ G.$-1$
H.$(\frac{251}{113})^{6}$
I.$(\frac{113}{251})^{6}$
\testStop
\kluczStart
A
\kluczStop



\zadStart{Zadanie z Wikieł Z 1.1 d) moja wersja nr 477}

Obliczyć wartość wyrażenia $(\frac{251}{127})^{6} \cdot (\frac{127}{251})^{6} \cdot \pi^{0}$.
\zadStop
\rozwStart{Patryk Wirkus}{Martyna Czarnobaj}
$$(\frac{251}{127})^{6} \cdot (\frac{127}{251})^{6} \cdot \pi^{0} = (\frac{251}{127} \cdot \frac{127}{251})^{6} \cdot 1 = 1^{6} \cdot 1 = 1$$
\rozwStop
\odpStart
$1$
\odpStop
\testStart
A.$1$ B.$\pi$ C.$0$ D.$\frac{251}{127}$ E.$\frac{127}{251}$
F.$-\frac{251}{127}$ G.$-1$
H.$(\frac{251}{127})^{6}$
I.$(\frac{127}{251})^{6}$
\testStop
\kluczStart
A
\kluczStop



\zadStart{Zadanie z Wikieł Z 1.1 d) moja wersja nr 478}

Obliczyć wartość wyrażenia $(\frac{251}{131})^{6} \cdot (\frac{131}{251})^{6} \cdot \pi^{0}$.
\zadStop
\rozwStart{Patryk Wirkus}{Martyna Czarnobaj}
$$(\frac{251}{131})^{6} \cdot (\frac{131}{251})^{6} \cdot \pi^{0} = (\frac{251}{131} \cdot \frac{131}{251})^{6} \cdot 1 = 1^{6} \cdot 1 = 1$$
\rozwStop
\odpStart
$1$
\odpStop
\testStart
A.$1$ B.$\pi$ C.$0$ D.$\frac{251}{131}$ E.$\frac{131}{251}$
F.$-\frac{251}{131}$ G.$-1$
H.$(\frac{251}{131})^{6}$
I.$(\frac{131}{251})^{6}$
\testStop
\kluczStart
A
\kluczStop



\zadStart{Zadanie z Wikieł Z 1.1 d) moja wersja nr 479}

Obliczyć wartość wyrażenia $(\frac{251}{137})^{6} \cdot (\frac{137}{251})^{6} \cdot \pi^{0}$.
\zadStop
\rozwStart{Patryk Wirkus}{Martyna Czarnobaj}
$$(\frac{251}{137})^{6} \cdot (\frac{137}{251})^{6} \cdot \pi^{0} = (\frac{251}{137} \cdot \frac{137}{251})^{6} \cdot 1 = 1^{6} \cdot 1 = 1$$
\rozwStop
\odpStart
$1$
\odpStop
\testStart
A.$1$ B.$\pi$ C.$0$ D.$\frac{251}{137}$ E.$\frac{137}{251}$
F.$-\frac{251}{137}$ G.$-1$
H.$(\frac{251}{137})^{6}$
I.$(\frac{137}{251})^{6}$
\testStop
\kluczStart
A
\kluczStop



\zadStart{Zadanie z Wikieł Z 1.1 d) moja wersja nr 480}

Obliczyć wartość wyrażenia $(\frac{251}{139})^{6} \cdot (\frac{139}{251})^{6} \cdot \pi^{0}$.
\zadStop
\rozwStart{Patryk Wirkus}{Martyna Czarnobaj}
$$(\frac{251}{139})^{6} \cdot (\frac{139}{251})^{6} \cdot \pi^{0} = (\frac{251}{139} \cdot \frac{139}{251})^{6} \cdot 1 = 1^{6} \cdot 1 = 1$$
\rozwStop
\odpStart
$1$
\odpStop
\testStart
A.$1$ B.$\pi$ C.$0$ D.$\frac{251}{139}$ E.$\frac{139}{251}$
F.$-\frac{251}{139}$ G.$-1$
H.$(\frac{251}{139})^{6}$
I.$(\frac{139}{251})^{6}$
\testStop
\kluczStart
A
\kluczStop



\zadStart{Zadanie z Wikieł Z 1.1 d) moja wersja nr 481}

Obliczyć wartość wyrażenia $(\frac{257}{103})^{6} \cdot (\frac{103}{257})^{6} \cdot \pi^{0}$.
\zadStop
\rozwStart{Patryk Wirkus}{Martyna Czarnobaj}
$$(\frac{257}{103})^{6} \cdot (\frac{103}{257})^{6} \cdot \pi^{0} = (\frac{257}{103} \cdot \frac{103}{257})^{6} \cdot 1 = 1^{6} \cdot 1 = 1$$
\rozwStop
\odpStart
$1$
\odpStop
\testStart
A.$1$ B.$\pi$ C.$0$ D.$\frac{257}{103}$ E.$\frac{103}{257}$
F.$-\frac{257}{103}$ G.$-1$
H.$(\frac{257}{103})^{6}$
I.$(\frac{103}{257})^{6}$
\testStop
\kluczStart
A
\kluczStop



\zadStart{Zadanie z Wikieł Z 1.1 d) moja wersja nr 482}

Obliczyć wartość wyrażenia $(\frac{257}{107})^{6} \cdot (\frac{107}{257})^{6} \cdot \pi^{0}$.
\zadStop
\rozwStart{Patryk Wirkus}{Martyna Czarnobaj}
$$(\frac{257}{107})^{6} \cdot (\frac{107}{257})^{6} \cdot \pi^{0} = (\frac{257}{107} \cdot \frac{107}{257})^{6} \cdot 1 = 1^{6} \cdot 1 = 1$$
\rozwStop
\odpStart
$1$
\odpStop
\testStart
A.$1$ B.$\pi$ C.$0$ D.$\frac{257}{107}$ E.$\frac{107}{257}$
F.$-\frac{257}{107}$ G.$-1$
H.$(\frac{257}{107})^{6}$
I.$(\frac{107}{257})^{6}$
\testStop
\kluczStart
A
\kluczStop



\zadStart{Zadanie z Wikieł Z 1.1 d) moja wersja nr 483}

Obliczyć wartość wyrażenia $(\frac{257}{109})^{6} \cdot (\frac{109}{257})^{6} \cdot \pi^{0}$.
\zadStop
\rozwStart{Patryk Wirkus}{Martyna Czarnobaj}
$$(\frac{257}{109})^{6} \cdot (\frac{109}{257})^{6} \cdot \pi^{0} = (\frac{257}{109} \cdot \frac{109}{257})^{6} \cdot 1 = 1^{6} \cdot 1 = 1$$
\rozwStop
\odpStart
$1$
\odpStop
\testStart
A.$1$ B.$\pi$ C.$0$ D.$\frac{257}{109}$ E.$\frac{109}{257}$
F.$-\frac{257}{109}$ G.$-1$
H.$(\frac{257}{109})^{6}$
I.$(\frac{109}{257})^{6}$
\testStop
\kluczStart
A
\kluczStop



\zadStart{Zadanie z Wikieł Z 1.1 d) moja wersja nr 484}

Obliczyć wartość wyrażenia $(\frac{257}{113})^{6} \cdot (\frac{113}{257})^{6} \cdot \pi^{0}$.
\zadStop
\rozwStart{Patryk Wirkus}{Martyna Czarnobaj}
$$(\frac{257}{113})^{6} \cdot (\frac{113}{257})^{6} \cdot \pi^{0} = (\frac{257}{113} \cdot \frac{113}{257})^{6} \cdot 1 = 1^{6} \cdot 1 = 1$$
\rozwStop
\odpStart
$1$
\odpStop
\testStart
A.$1$ B.$\pi$ C.$0$ D.$\frac{257}{113}$ E.$\frac{113}{257}$
F.$-\frac{257}{113}$ G.$-1$
H.$(\frac{257}{113})^{6}$
I.$(\frac{113}{257})^{6}$
\testStop
\kluczStart
A
\kluczStop



\zadStart{Zadanie z Wikieł Z 1.1 d) moja wersja nr 485}

Obliczyć wartość wyrażenia $(\frac{257}{127})^{6} \cdot (\frac{127}{257})^{6} \cdot \pi^{0}$.
\zadStop
\rozwStart{Patryk Wirkus}{Martyna Czarnobaj}
$$(\frac{257}{127})^{6} \cdot (\frac{127}{257})^{6} \cdot \pi^{0} = (\frac{257}{127} \cdot \frac{127}{257})^{6} \cdot 1 = 1^{6} \cdot 1 = 1$$
\rozwStop
\odpStart
$1$
\odpStop
\testStart
A.$1$ B.$\pi$ C.$0$ D.$\frac{257}{127}$ E.$\frac{127}{257}$
F.$-\frac{257}{127}$ G.$-1$
H.$(\frac{257}{127})^{6}$
I.$(\frac{127}{257})^{6}$
\testStop
\kluczStart
A
\kluczStop



\zadStart{Zadanie z Wikieł Z 1.1 d) moja wersja nr 486}

Obliczyć wartość wyrażenia $(\frac{257}{131})^{6} \cdot (\frac{131}{257})^{6} \cdot \pi^{0}$.
\zadStop
\rozwStart{Patryk Wirkus}{Martyna Czarnobaj}
$$(\frac{257}{131})^{6} \cdot (\frac{131}{257})^{6} \cdot \pi^{0} = (\frac{257}{131} \cdot \frac{131}{257})^{6} \cdot 1 = 1^{6} \cdot 1 = 1$$
\rozwStop
\odpStart
$1$
\odpStop
\testStart
A.$1$ B.$\pi$ C.$0$ D.$\frac{257}{131}$ E.$\frac{131}{257}$
F.$-\frac{257}{131}$ G.$-1$
H.$(\frac{257}{131})^{6}$
I.$(\frac{131}{257})^{6}$
\testStop
\kluczStart
A
\kluczStop



\zadStart{Zadanie z Wikieł Z 1.1 d) moja wersja nr 487}

Obliczyć wartość wyrażenia $(\frac{257}{137})^{6} \cdot (\frac{137}{257})^{6} \cdot \pi^{0}$.
\zadStop
\rozwStart{Patryk Wirkus}{Martyna Czarnobaj}
$$(\frac{257}{137})^{6} \cdot (\frac{137}{257})^{6} \cdot \pi^{0} = (\frac{257}{137} \cdot \frac{137}{257})^{6} \cdot 1 = 1^{6} \cdot 1 = 1$$
\rozwStop
\odpStart
$1$
\odpStop
\testStart
A.$1$ B.$\pi$ C.$0$ D.$\frac{257}{137}$ E.$\frac{137}{257}$
F.$-\frac{257}{137}$ G.$-1$
H.$(\frac{257}{137})^{6}$
I.$(\frac{137}{257})^{6}$
\testStop
\kluczStart
A
\kluczStop



\zadStart{Zadanie z Wikieł Z 1.1 d) moja wersja nr 488}

Obliczyć wartość wyrażenia $(\frac{257}{139})^{6} \cdot (\frac{139}{257})^{6} \cdot \pi^{0}$.
\zadStop
\rozwStart{Patryk Wirkus}{Martyna Czarnobaj}
$$(\frac{257}{139})^{6} \cdot (\frac{139}{257})^{6} \cdot \pi^{0} = (\frac{257}{139} \cdot \frac{139}{257})^{6} \cdot 1 = 1^{6} \cdot 1 = 1$$
\rozwStop
\odpStart
$1$
\odpStop
\testStart
A.$1$ B.$\pi$ C.$0$ D.$\frac{257}{139}$ E.$\frac{139}{257}$
F.$-\frac{257}{139}$ G.$-1$
H.$(\frac{257}{139})^{6}$
I.$(\frac{139}{257})^{6}$
\testStop
\kluczStart
A
\kluczStop



\zadStart{Zadanie z Wikieł Z 1.1 d) moja wersja nr 489}

Obliczyć wartość wyrażenia $(\frac{263}{103})^{6} \cdot (\frac{103}{263})^{6} \cdot \pi^{0}$.
\zadStop
\rozwStart{Patryk Wirkus}{Martyna Czarnobaj}
$$(\frac{263}{103})^{6} \cdot (\frac{103}{263})^{6} \cdot \pi^{0} = (\frac{263}{103} \cdot \frac{103}{263})^{6} \cdot 1 = 1^{6} \cdot 1 = 1$$
\rozwStop
\odpStart
$1$
\odpStop
\testStart
A.$1$ B.$\pi$ C.$0$ D.$\frac{263}{103}$ E.$\frac{103}{263}$
F.$-\frac{263}{103}$ G.$-1$
H.$(\frac{263}{103})^{6}$
I.$(\frac{103}{263})^{6}$
\testStop
\kluczStart
A
\kluczStop



\zadStart{Zadanie z Wikieł Z 1.1 d) moja wersja nr 490}

Obliczyć wartość wyrażenia $(\frac{263}{107})^{6} \cdot (\frac{107}{263})^{6} \cdot \pi^{0}$.
\zadStop
\rozwStart{Patryk Wirkus}{Martyna Czarnobaj}
$$(\frac{263}{107})^{6} \cdot (\frac{107}{263})^{6} \cdot \pi^{0} = (\frac{263}{107} \cdot \frac{107}{263})^{6} \cdot 1 = 1^{6} \cdot 1 = 1$$
\rozwStop
\odpStart
$1$
\odpStop
\testStart
A.$1$ B.$\pi$ C.$0$ D.$\frac{263}{107}$ E.$\frac{107}{263}$
F.$-\frac{263}{107}$ G.$-1$
H.$(\frac{263}{107})^{6}$
I.$(\frac{107}{263})^{6}$
\testStop
\kluczStart
A
\kluczStop



\zadStart{Zadanie z Wikieł Z 1.1 d) moja wersja nr 491}

Obliczyć wartość wyrażenia $(\frac{263}{109})^{6} \cdot (\frac{109}{263})^{6} \cdot \pi^{0}$.
\zadStop
\rozwStart{Patryk Wirkus}{Martyna Czarnobaj}
$$(\frac{263}{109})^{6} \cdot (\frac{109}{263})^{6} \cdot \pi^{0} = (\frac{263}{109} \cdot \frac{109}{263})^{6} \cdot 1 = 1^{6} \cdot 1 = 1$$
\rozwStop
\odpStart
$1$
\odpStop
\testStart
A.$1$ B.$\pi$ C.$0$ D.$\frac{263}{109}$ E.$\frac{109}{263}$
F.$-\frac{263}{109}$ G.$-1$
H.$(\frac{263}{109})^{6}$
I.$(\frac{109}{263})^{6}$
\testStop
\kluczStart
A
\kluczStop



\zadStart{Zadanie z Wikieł Z 1.1 d) moja wersja nr 492}

Obliczyć wartość wyrażenia $(\frac{263}{113})^{6} \cdot (\frac{113}{263})^{6} \cdot \pi^{0}$.
\zadStop
\rozwStart{Patryk Wirkus}{Martyna Czarnobaj}
$$(\frac{263}{113})^{6} \cdot (\frac{113}{263})^{6} \cdot \pi^{0} = (\frac{263}{113} \cdot \frac{113}{263})^{6} \cdot 1 = 1^{6} \cdot 1 = 1$$
\rozwStop
\odpStart
$1$
\odpStop
\testStart
A.$1$ B.$\pi$ C.$0$ D.$\frac{263}{113}$ E.$\frac{113}{263}$
F.$-\frac{263}{113}$ G.$-1$
H.$(\frac{263}{113})^{6}$
I.$(\frac{113}{263})^{6}$
\testStop
\kluczStart
A
\kluczStop



\zadStart{Zadanie z Wikieł Z 1.1 d) moja wersja nr 493}

Obliczyć wartość wyrażenia $(\frac{263}{127})^{6} \cdot (\frac{127}{263})^{6} \cdot \pi^{0}$.
\zadStop
\rozwStart{Patryk Wirkus}{Martyna Czarnobaj}
$$(\frac{263}{127})^{6} \cdot (\frac{127}{263})^{6} \cdot \pi^{0} = (\frac{263}{127} \cdot \frac{127}{263})^{6} \cdot 1 = 1^{6} \cdot 1 = 1$$
\rozwStop
\odpStart
$1$
\odpStop
\testStart
A.$1$ B.$\pi$ C.$0$ D.$\frac{263}{127}$ E.$\frac{127}{263}$
F.$-\frac{263}{127}$ G.$-1$
H.$(\frac{263}{127})^{6}$
I.$(\frac{127}{263})^{6}$
\testStop
\kluczStart
A
\kluczStop



\zadStart{Zadanie z Wikieł Z 1.1 d) moja wersja nr 494}

Obliczyć wartość wyrażenia $(\frac{263}{131})^{6} \cdot (\frac{131}{263})^{6} \cdot \pi^{0}$.
\zadStop
\rozwStart{Patryk Wirkus}{Martyna Czarnobaj}
$$(\frac{263}{131})^{6} \cdot (\frac{131}{263})^{6} \cdot \pi^{0} = (\frac{263}{131} \cdot \frac{131}{263})^{6} \cdot 1 = 1^{6} \cdot 1 = 1$$
\rozwStop
\odpStart
$1$
\odpStop
\testStart
A.$1$ B.$\pi$ C.$0$ D.$\frac{263}{131}$ E.$\frac{131}{263}$
F.$-\frac{263}{131}$ G.$-1$
H.$(\frac{263}{131})^{6}$
I.$(\frac{131}{263})^{6}$
\testStop
\kluczStart
A
\kluczStop



\zadStart{Zadanie z Wikieł Z 1.1 d) moja wersja nr 495}

Obliczyć wartość wyrażenia $(\frac{263}{137})^{6} \cdot (\frac{137}{263})^{6} \cdot \pi^{0}$.
\zadStop
\rozwStart{Patryk Wirkus}{Martyna Czarnobaj}
$$(\frac{263}{137})^{6} \cdot (\frac{137}{263})^{6} \cdot \pi^{0} = (\frac{263}{137} \cdot \frac{137}{263})^{6} \cdot 1 = 1^{6} \cdot 1 = 1$$
\rozwStop
\odpStart
$1$
\odpStop
\testStart
A.$1$ B.$\pi$ C.$0$ D.$\frac{263}{137}$ E.$\frac{137}{263}$
F.$-\frac{263}{137}$ G.$-1$
H.$(\frac{263}{137})^{6}$
I.$(\frac{137}{263})^{6}$
\testStop
\kluczStart
A
\kluczStop



\zadStart{Zadanie z Wikieł Z 1.1 d) moja wersja nr 496}

Obliczyć wartość wyrażenia $(\frac{263}{139})^{6} \cdot (\frac{139}{263})^{6} \cdot \pi^{0}$.
\zadStop
\rozwStart{Patryk Wirkus}{Martyna Czarnobaj}
$$(\frac{263}{139})^{6} \cdot (\frac{139}{263})^{6} \cdot \pi^{0} = (\frac{263}{139} \cdot \frac{139}{263})^{6} \cdot 1 = 1^{6} \cdot 1 = 1$$
\rozwStop
\odpStart
$1$
\odpStop
\testStart
A.$1$ B.$\pi$ C.$0$ D.$\frac{263}{139}$ E.$\frac{139}{263}$
F.$-\frac{263}{139}$ G.$-1$
H.$(\frac{263}{139})^{6}$
I.$(\frac{139}{263})^{6}$
\testStop
\kluczStart
A
\kluczStop



\zadStart{Zadanie z Wikieł Z 1.1 d) moja wersja nr 497}

Obliczyć wartość wyrażenia $(\frac{269}{103})^{6} \cdot (\frac{103}{269})^{6} \cdot \pi^{0}$.
\zadStop
\rozwStart{Patryk Wirkus}{Martyna Czarnobaj}
$$(\frac{269}{103})^{6} \cdot (\frac{103}{269})^{6} \cdot \pi^{0} = (\frac{269}{103} \cdot \frac{103}{269})^{6} \cdot 1 = 1^{6} \cdot 1 = 1$$
\rozwStop
\odpStart
$1$
\odpStop
\testStart
A.$1$ B.$\pi$ C.$0$ D.$\frac{269}{103}$ E.$\frac{103}{269}$
F.$-\frac{269}{103}$ G.$-1$
H.$(\frac{269}{103})^{6}$
I.$(\frac{103}{269})^{6}$
\testStop
\kluczStart
A
\kluczStop



\zadStart{Zadanie z Wikieł Z 1.1 d) moja wersja nr 498}

Obliczyć wartość wyrażenia $(\frac{269}{107})^{6} \cdot (\frac{107}{269})^{6} \cdot \pi^{0}$.
\zadStop
\rozwStart{Patryk Wirkus}{Martyna Czarnobaj}
$$(\frac{269}{107})^{6} \cdot (\frac{107}{269})^{6} \cdot \pi^{0} = (\frac{269}{107} \cdot \frac{107}{269})^{6} \cdot 1 = 1^{6} \cdot 1 = 1$$
\rozwStop
\odpStart
$1$
\odpStop
\testStart
A.$1$ B.$\pi$ C.$0$ D.$\frac{269}{107}$ E.$\frac{107}{269}$
F.$-\frac{269}{107}$ G.$-1$
H.$(\frac{269}{107})^{6}$
I.$(\frac{107}{269})^{6}$
\testStop
\kluczStart
A
\kluczStop



\zadStart{Zadanie z Wikieł Z 1.1 d) moja wersja nr 499}

Obliczyć wartość wyrażenia $(\frac{269}{109})^{6} \cdot (\frac{109}{269})^{6} \cdot \pi^{0}$.
\zadStop
\rozwStart{Patryk Wirkus}{Martyna Czarnobaj}
$$(\frac{269}{109})^{6} \cdot (\frac{109}{269})^{6} \cdot \pi^{0} = (\frac{269}{109} \cdot \frac{109}{269})^{6} \cdot 1 = 1^{6} \cdot 1 = 1$$
\rozwStop
\odpStart
$1$
\odpStop
\testStart
A.$1$ B.$\pi$ C.$0$ D.$\frac{269}{109}$ E.$\frac{109}{269}$
F.$-\frac{269}{109}$ G.$-1$
H.$(\frac{269}{109})^{6}$
I.$(\frac{109}{269})^{6}$
\testStop
\kluczStart
A
\kluczStop



\zadStart{Zadanie z Wikieł Z 1.1 d) moja wersja nr 500}

Obliczyć wartość wyrażenia $(\frac{269}{113})^{6} \cdot (\frac{113}{269})^{6} \cdot \pi^{0}$.
\zadStop
\rozwStart{Patryk Wirkus}{Martyna Czarnobaj}
$$(\frac{269}{113})^{6} \cdot (\frac{113}{269})^{6} \cdot \pi^{0} = (\frac{269}{113} \cdot \frac{113}{269})^{6} \cdot 1 = 1^{6} \cdot 1 = 1$$
\rozwStop
\odpStart
$1$
\odpStop
\testStart
A.$1$ B.$\pi$ C.$0$ D.$\frac{269}{113}$ E.$\frac{113}{269}$
F.$-\frac{269}{113}$ G.$-1$
H.$(\frac{269}{113})^{6}$
I.$(\frac{113}{269})^{6}$
\testStop
\kluczStart
A
\kluczStop



\zadStart{Zadanie z Wikieł Z 1.1 d) moja wersja nr 501}

Obliczyć wartość wyrażenia $(\frac{269}{127})^{6} \cdot (\frac{127}{269})^{6} \cdot \pi^{0}$.
\zadStop
\rozwStart{Patryk Wirkus}{Martyna Czarnobaj}
$$(\frac{269}{127})^{6} \cdot (\frac{127}{269})^{6} \cdot \pi^{0} = (\frac{269}{127} \cdot \frac{127}{269})^{6} \cdot 1 = 1^{6} \cdot 1 = 1$$
\rozwStop
\odpStart
$1$
\odpStop
\testStart
A.$1$ B.$\pi$ C.$0$ D.$\frac{269}{127}$ E.$\frac{127}{269}$
F.$-\frac{269}{127}$ G.$-1$
H.$(\frac{269}{127})^{6}$
I.$(\frac{127}{269})^{6}$
\testStop
\kluczStart
A
\kluczStop



\zadStart{Zadanie z Wikieł Z 1.1 d) moja wersja nr 502}

Obliczyć wartość wyrażenia $(\frac{269}{131})^{6} \cdot (\frac{131}{269})^{6} \cdot \pi^{0}$.
\zadStop
\rozwStart{Patryk Wirkus}{Martyna Czarnobaj}
$$(\frac{269}{131})^{6} \cdot (\frac{131}{269})^{6} \cdot \pi^{0} = (\frac{269}{131} \cdot \frac{131}{269})^{6} \cdot 1 = 1^{6} \cdot 1 = 1$$
\rozwStop
\odpStart
$1$
\odpStop
\testStart
A.$1$ B.$\pi$ C.$0$ D.$\frac{269}{131}$ E.$\frac{131}{269}$
F.$-\frac{269}{131}$ G.$-1$
H.$(\frac{269}{131})^{6}$
I.$(\frac{131}{269})^{6}$
\testStop
\kluczStart
A
\kluczStop



\zadStart{Zadanie z Wikieł Z 1.1 d) moja wersja nr 503}

Obliczyć wartość wyrażenia $(\frac{269}{137})^{6} \cdot (\frac{137}{269})^{6} \cdot \pi^{0}$.
\zadStop
\rozwStart{Patryk Wirkus}{Martyna Czarnobaj}
$$(\frac{269}{137})^{6} \cdot (\frac{137}{269})^{6} \cdot \pi^{0} = (\frac{269}{137} \cdot \frac{137}{269})^{6} \cdot 1 = 1^{6} \cdot 1 = 1$$
\rozwStop
\odpStart
$1$
\odpStop
\testStart
A.$1$ B.$\pi$ C.$0$ D.$\frac{269}{137}$ E.$\frac{137}{269}$
F.$-\frac{269}{137}$ G.$-1$
H.$(\frac{269}{137})^{6}$
I.$(\frac{137}{269})^{6}$
\testStop
\kluczStart
A
\kluczStop



\zadStart{Zadanie z Wikieł Z 1.1 d) moja wersja nr 504}

Obliczyć wartość wyrażenia $(\frac{269}{139})^{6} \cdot (\frac{139}{269})^{6} \cdot \pi^{0}$.
\zadStop
\rozwStart{Patryk Wirkus}{Martyna Czarnobaj}
$$(\frac{269}{139})^{6} \cdot (\frac{139}{269})^{6} \cdot \pi^{0} = (\frac{269}{139} \cdot \frac{139}{269})^{6} \cdot 1 = 1^{6} \cdot 1 = 1$$
\rozwStop
\odpStart
$1$
\odpStop
\testStart
A.$1$ B.$\pi$ C.$0$ D.$\frac{269}{139}$ E.$\frac{139}{269}$
F.$-\frac{269}{139}$ G.$-1$
H.$(\frac{269}{139})^{6}$
I.$(\frac{139}{269})^{6}$
\testStop
\kluczStart
A
\kluczStop



\zadStart{Zadanie z Wikieł Z 1.1 d) moja wersja nr 505}

Obliczyć wartość wyrażenia $(\frac{271}{103})^{6} \cdot (\frac{103}{271})^{6} \cdot \pi^{0}$.
\zadStop
\rozwStart{Patryk Wirkus}{Martyna Czarnobaj}
$$(\frac{271}{103})^{6} \cdot (\frac{103}{271})^{6} \cdot \pi^{0} = (\frac{271}{103} \cdot \frac{103}{271})^{6} \cdot 1 = 1^{6} \cdot 1 = 1$$
\rozwStop
\odpStart
$1$
\odpStop
\testStart
A.$1$ B.$\pi$ C.$0$ D.$\frac{271}{103}$ E.$\frac{103}{271}$
F.$-\frac{271}{103}$ G.$-1$
H.$(\frac{271}{103})^{6}$
I.$(\frac{103}{271})^{6}$
\testStop
\kluczStart
A
\kluczStop



\zadStart{Zadanie z Wikieł Z 1.1 d) moja wersja nr 506}

Obliczyć wartość wyrażenia $(\frac{271}{107})^{6} \cdot (\frac{107}{271})^{6} \cdot \pi^{0}$.
\zadStop
\rozwStart{Patryk Wirkus}{Martyna Czarnobaj}
$$(\frac{271}{107})^{6} \cdot (\frac{107}{271})^{6} \cdot \pi^{0} = (\frac{271}{107} \cdot \frac{107}{271})^{6} \cdot 1 = 1^{6} \cdot 1 = 1$$
\rozwStop
\odpStart
$1$
\odpStop
\testStart
A.$1$ B.$\pi$ C.$0$ D.$\frac{271}{107}$ E.$\frac{107}{271}$
F.$-\frac{271}{107}$ G.$-1$
H.$(\frac{271}{107})^{6}$
I.$(\frac{107}{271})^{6}$
\testStop
\kluczStart
A
\kluczStop



\zadStart{Zadanie z Wikieł Z 1.1 d) moja wersja nr 507}

Obliczyć wartość wyrażenia $(\frac{271}{109})^{6} \cdot (\frac{109}{271})^{6} \cdot \pi^{0}$.
\zadStop
\rozwStart{Patryk Wirkus}{Martyna Czarnobaj}
$$(\frac{271}{109})^{6} \cdot (\frac{109}{271})^{6} \cdot \pi^{0} = (\frac{271}{109} \cdot \frac{109}{271})^{6} \cdot 1 = 1^{6} \cdot 1 = 1$$
\rozwStop
\odpStart
$1$
\odpStop
\testStart
A.$1$ B.$\pi$ C.$0$ D.$\frac{271}{109}$ E.$\frac{109}{271}$
F.$-\frac{271}{109}$ G.$-1$
H.$(\frac{271}{109})^{6}$
I.$(\frac{109}{271})^{6}$
\testStop
\kluczStart
A
\kluczStop



\zadStart{Zadanie z Wikieł Z 1.1 d) moja wersja nr 508}

Obliczyć wartość wyrażenia $(\frac{271}{113})^{6} \cdot (\frac{113}{271})^{6} \cdot \pi^{0}$.
\zadStop
\rozwStart{Patryk Wirkus}{Martyna Czarnobaj}
$$(\frac{271}{113})^{6} \cdot (\frac{113}{271})^{6} \cdot \pi^{0} = (\frac{271}{113} \cdot \frac{113}{271})^{6} \cdot 1 = 1^{6} \cdot 1 = 1$$
\rozwStop
\odpStart
$1$
\odpStop
\testStart
A.$1$ B.$\pi$ C.$0$ D.$\frac{271}{113}$ E.$\frac{113}{271}$
F.$-\frac{271}{113}$ G.$-1$
H.$(\frac{271}{113})^{6}$
I.$(\frac{113}{271})^{6}$
\testStop
\kluczStart
A
\kluczStop



\zadStart{Zadanie z Wikieł Z 1.1 d) moja wersja nr 509}

Obliczyć wartość wyrażenia $(\frac{271}{127})^{6} \cdot (\frac{127}{271})^{6} \cdot \pi^{0}$.
\zadStop
\rozwStart{Patryk Wirkus}{Martyna Czarnobaj}
$$(\frac{271}{127})^{6} \cdot (\frac{127}{271})^{6} \cdot \pi^{0} = (\frac{271}{127} \cdot \frac{127}{271})^{6} \cdot 1 = 1^{6} \cdot 1 = 1$$
\rozwStop
\odpStart
$1$
\odpStop
\testStart
A.$1$ B.$\pi$ C.$0$ D.$\frac{271}{127}$ E.$\frac{127}{271}$
F.$-\frac{271}{127}$ G.$-1$
H.$(\frac{271}{127})^{6}$
I.$(\frac{127}{271})^{6}$
\testStop
\kluczStart
A
\kluczStop



\zadStart{Zadanie z Wikieł Z 1.1 d) moja wersja nr 510}

Obliczyć wartość wyrażenia $(\frac{271}{131})^{6} \cdot (\frac{131}{271})^{6} \cdot \pi^{0}$.
\zadStop
\rozwStart{Patryk Wirkus}{Martyna Czarnobaj}
$$(\frac{271}{131})^{6} \cdot (\frac{131}{271})^{6} \cdot \pi^{0} = (\frac{271}{131} \cdot \frac{131}{271})^{6} \cdot 1 = 1^{6} \cdot 1 = 1$$
\rozwStop
\odpStart
$1$
\odpStop
\testStart
A.$1$ B.$\pi$ C.$0$ D.$\frac{271}{131}$ E.$\frac{131}{271}$
F.$-\frac{271}{131}$ G.$-1$
H.$(\frac{271}{131})^{6}$
I.$(\frac{131}{271})^{6}$
\testStop
\kluczStart
A
\kluczStop



\zadStart{Zadanie z Wikieł Z 1.1 d) moja wersja nr 511}

Obliczyć wartość wyrażenia $(\frac{271}{137})^{6} \cdot (\frac{137}{271})^{6} \cdot \pi^{0}$.
\zadStop
\rozwStart{Patryk Wirkus}{Martyna Czarnobaj}
$$(\frac{271}{137})^{6} \cdot (\frac{137}{271})^{6} \cdot \pi^{0} = (\frac{271}{137} \cdot \frac{137}{271})^{6} \cdot 1 = 1^{6} \cdot 1 = 1$$
\rozwStop
\odpStart
$1$
\odpStop
\testStart
A.$1$ B.$\pi$ C.$0$ D.$\frac{271}{137}$ E.$\frac{137}{271}$
F.$-\frac{271}{137}$ G.$-1$
H.$(\frac{271}{137})^{6}$
I.$(\frac{137}{271})^{6}$
\testStop
\kluczStart
A
\kluczStop



\zadStart{Zadanie z Wikieł Z 1.1 d) moja wersja nr 512}

Obliczyć wartość wyrażenia $(\frac{271}{139})^{6} \cdot (\frac{139}{271})^{6} \cdot \pi^{0}$.
\zadStop
\rozwStart{Patryk Wirkus}{Martyna Czarnobaj}
$$(\frac{271}{139})^{6} \cdot (\frac{139}{271})^{6} \cdot \pi^{0} = (\frac{271}{139} \cdot \frac{139}{271})^{6} \cdot 1 = 1^{6} \cdot 1 = 1$$
\rozwStop
\odpStart
$1$
\odpStop
\testStart
A.$1$ B.$\pi$ C.$0$ D.$\frac{271}{139}$ E.$\frac{139}{271}$
F.$-\frac{271}{139}$ G.$-1$
H.$(\frac{271}{139})^{6}$
I.$(\frac{139}{271})^{6}$
\testStop
\kluczStart
A
\kluczStop



\zadStart{Zadanie z Wikieł Z 1.1 d) moja wersja nr 513}

Obliczyć wartość wyrażenia $(\frac{277}{103})^{6} \cdot (\frac{103}{277})^{6} \cdot \pi^{0}$.
\zadStop
\rozwStart{Patryk Wirkus}{Martyna Czarnobaj}
$$(\frac{277}{103})^{6} \cdot (\frac{103}{277})^{6} \cdot \pi^{0} = (\frac{277}{103} \cdot \frac{103}{277})^{6} \cdot 1 = 1^{6} \cdot 1 = 1$$
\rozwStop
\odpStart
$1$
\odpStop
\testStart
A.$1$ B.$\pi$ C.$0$ D.$\frac{277}{103}$ E.$\frac{103}{277}$
F.$-\frac{277}{103}$ G.$-1$
H.$(\frac{277}{103})^{6}$
I.$(\frac{103}{277})^{6}$
\testStop
\kluczStart
A
\kluczStop



\zadStart{Zadanie z Wikieł Z 1.1 d) moja wersja nr 514}

Obliczyć wartość wyrażenia $(\frac{277}{107})^{6} \cdot (\frac{107}{277})^{6} \cdot \pi^{0}$.
\zadStop
\rozwStart{Patryk Wirkus}{Martyna Czarnobaj}
$$(\frac{277}{107})^{6} \cdot (\frac{107}{277})^{6} \cdot \pi^{0} = (\frac{277}{107} \cdot \frac{107}{277})^{6} \cdot 1 = 1^{6} \cdot 1 = 1$$
\rozwStop
\odpStart
$1$
\odpStop
\testStart
A.$1$ B.$\pi$ C.$0$ D.$\frac{277}{107}$ E.$\frac{107}{277}$
F.$-\frac{277}{107}$ G.$-1$
H.$(\frac{277}{107})^{6}$
I.$(\frac{107}{277})^{6}$
\testStop
\kluczStart
A
\kluczStop



\zadStart{Zadanie z Wikieł Z 1.1 d) moja wersja nr 515}

Obliczyć wartość wyrażenia $(\frac{277}{109})^{6} \cdot (\frac{109}{277})^{6} \cdot \pi^{0}$.
\zadStop
\rozwStart{Patryk Wirkus}{Martyna Czarnobaj}
$$(\frac{277}{109})^{6} \cdot (\frac{109}{277})^{6} \cdot \pi^{0} = (\frac{277}{109} \cdot \frac{109}{277})^{6} \cdot 1 = 1^{6} \cdot 1 = 1$$
\rozwStop
\odpStart
$1$
\odpStop
\testStart
A.$1$ B.$\pi$ C.$0$ D.$\frac{277}{109}$ E.$\frac{109}{277}$
F.$-\frac{277}{109}$ G.$-1$
H.$(\frac{277}{109})^{6}$
I.$(\frac{109}{277})^{6}$
\testStop
\kluczStart
A
\kluczStop



\zadStart{Zadanie z Wikieł Z 1.1 d) moja wersja nr 516}

Obliczyć wartość wyrażenia $(\frac{277}{113})^{6} \cdot (\frac{113}{277})^{6} \cdot \pi^{0}$.
\zadStop
\rozwStart{Patryk Wirkus}{Martyna Czarnobaj}
$$(\frac{277}{113})^{6} \cdot (\frac{113}{277})^{6} \cdot \pi^{0} = (\frac{277}{113} \cdot \frac{113}{277})^{6} \cdot 1 = 1^{6} \cdot 1 = 1$$
\rozwStop
\odpStart
$1$
\odpStop
\testStart
A.$1$ B.$\pi$ C.$0$ D.$\frac{277}{113}$ E.$\frac{113}{277}$
F.$-\frac{277}{113}$ G.$-1$
H.$(\frac{277}{113})^{6}$
I.$(\frac{113}{277})^{6}$
\testStop
\kluczStart
A
\kluczStop



\zadStart{Zadanie z Wikieł Z 1.1 d) moja wersja nr 517}

Obliczyć wartość wyrażenia $(\frac{277}{127})^{6} \cdot (\frac{127}{277})^{6} \cdot \pi^{0}$.
\zadStop
\rozwStart{Patryk Wirkus}{Martyna Czarnobaj}
$$(\frac{277}{127})^{6} \cdot (\frac{127}{277})^{6} \cdot \pi^{0} = (\frac{277}{127} \cdot \frac{127}{277})^{6} \cdot 1 = 1^{6} \cdot 1 = 1$$
\rozwStop
\odpStart
$1$
\odpStop
\testStart
A.$1$ B.$\pi$ C.$0$ D.$\frac{277}{127}$ E.$\frac{127}{277}$
F.$-\frac{277}{127}$ G.$-1$
H.$(\frac{277}{127})^{6}$
I.$(\frac{127}{277})^{6}$
\testStop
\kluczStart
A
\kluczStop



\zadStart{Zadanie z Wikieł Z 1.1 d) moja wersja nr 518}

Obliczyć wartość wyrażenia $(\frac{277}{131})^{6} \cdot (\frac{131}{277})^{6} \cdot \pi^{0}$.
\zadStop
\rozwStart{Patryk Wirkus}{Martyna Czarnobaj}
$$(\frac{277}{131})^{6} \cdot (\frac{131}{277})^{6} \cdot \pi^{0} = (\frac{277}{131} \cdot \frac{131}{277})^{6} \cdot 1 = 1^{6} \cdot 1 = 1$$
\rozwStop
\odpStart
$1$
\odpStop
\testStart
A.$1$ B.$\pi$ C.$0$ D.$\frac{277}{131}$ E.$\frac{131}{277}$
F.$-\frac{277}{131}$ G.$-1$
H.$(\frac{277}{131})^{6}$
I.$(\frac{131}{277})^{6}$
\testStop
\kluczStart
A
\kluczStop



\zadStart{Zadanie z Wikieł Z 1.1 d) moja wersja nr 519}

Obliczyć wartość wyrażenia $(\frac{277}{137})^{6} \cdot (\frac{137}{277})^{6} \cdot \pi^{0}$.
\zadStop
\rozwStart{Patryk Wirkus}{Martyna Czarnobaj}
$$(\frac{277}{137})^{6} \cdot (\frac{137}{277})^{6} \cdot \pi^{0} = (\frac{277}{137} \cdot \frac{137}{277})^{6} \cdot 1 = 1^{6} \cdot 1 = 1$$
\rozwStop
\odpStart
$1$
\odpStop
\testStart
A.$1$ B.$\pi$ C.$0$ D.$\frac{277}{137}$ E.$\frac{137}{277}$
F.$-\frac{277}{137}$ G.$-1$
H.$(\frac{277}{137})^{6}$
I.$(\frac{137}{277})^{6}$
\testStop
\kluczStart
A
\kluczStop



\zadStart{Zadanie z Wikieł Z 1.1 d) moja wersja nr 520}

Obliczyć wartość wyrażenia $(\frac{277}{139})^{6} \cdot (\frac{139}{277})^{6} \cdot \pi^{0}$.
\zadStop
\rozwStart{Patryk Wirkus}{Martyna Czarnobaj}
$$(\frac{277}{139})^{6} \cdot (\frac{139}{277})^{6} \cdot \pi^{0} = (\frac{277}{139} \cdot \frac{139}{277})^{6} \cdot 1 = 1^{6} \cdot 1 = 1$$
\rozwStop
\odpStart
$1$
\odpStop
\testStart
A.$1$ B.$\pi$ C.$0$ D.$\frac{277}{139}$ E.$\frac{139}{277}$
F.$-\frac{277}{139}$ G.$-1$
H.$(\frac{277}{139})^{6}$
I.$(\frac{139}{277})^{6}$
\testStop
\kluczStart
A
\kluczStop



\zadStart{Zadanie z Wikieł Z 1.1 d) moja wersja nr 521}

Obliczyć wartość wyrażenia $(\frac{149}{103})^{7} \cdot (\frac{103}{149})^{7} \cdot \pi^{0}$.
\zadStop
\rozwStart{Patryk Wirkus}{Martyna Czarnobaj}
$$(\frac{149}{103})^{7} \cdot (\frac{103}{149})^{7} \cdot \pi^{0} = (\frac{149}{103} \cdot \frac{103}{149})^{7} \cdot 1 = 1^{7} \cdot 1 = 1$$
\rozwStop
\odpStart
$1$
\odpStop
\testStart
A.$1$ B.$\pi$ C.$0$ D.$\frac{149}{103}$ E.$\frac{103}{149}$
F.$-\frac{149}{103}$ G.$-1$
H.$(\frac{149}{103})^{7}$
I.$(\frac{103}{149})^{7}$
\testStop
\kluczStart
A
\kluczStop



\zadStart{Zadanie z Wikieł Z 1.1 d) moja wersja nr 522}

Obliczyć wartość wyrażenia $(\frac{149}{107})^{7} \cdot (\frac{107}{149})^{7} \cdot \pi^{0}$.
\zadStop
\rozwStart{Patryk Wirkus}{Martyna Czarnobaj}
$$(\frac{149}{107})^{7} \cdot (\frac{107}{149})^{7} \cdot \pi^{0} = (\frac{149}{107} \cdot \frac{107}{149})^{7} \cdot 1 = 1^{7} \cdot 1 = 1$$
\rozwStop
\odpStart
$1$
\odpStop
\testStart
A.$1$ B.$\pi$ C.$0$ D.$\frac{149}{107}$ E.$\frac{107}{149}$
F.$-\frac{149}{107}$ G.$-1$
H.$(\frac{149}{107})^{7}$
I.$(\frac{107}{149})^{7}$
\testStop
\kluczStart
A
\kluczStop



\zadStart{Zadanie z Wikieł Z 1.1 d) moja wersja nr 523}

Obliczyć wartość wyrażenia $(\frac{149}{109})^{7} \cdot (\frac{109}{149})^{7} \cdot \pi^{0}$.
\zadStop
\rozwStart{Patryk Wirkus}{Martyna Czarnobaj}
$$(\frac{149}{109})^{7} \cdot (\frac{109}{149})^{7} \cdot \pi^{0} = (\frac{149}{109} \cdot \frac{109}{149})^{7} \cdot 1 = 1^{7} \cdot 1 = 1$$
\rozwStop
\odpStart
$1$
\odpStop
\testStart
A.$1$ B.$\pi$ C.$0$ D.$\frac{149}{109}$ E.$\frac{109}{149}$
F.$-\frac{149}{109}$ G.$-1$
H.$(\frac{149}{109})^{7}$
I.$(\frac{109}{149})^{7}$
\testStop
\kluczStart
A
\kluczStop



\zadStart{Zadanie z Wikieł Z 1.1 d) moja wersja nr 524}

Obliczyć wartość wyrażenia $(\frac{149}{113})^{7} \cdot (\frac{113}{149})^{7} \cdot \pi^{0}$.
\zadStop
\rozwStart{Patryk Wirkus}{Martyna Czarnobaj}
$$(\frac{149}{113})^{7} \cdot (\frac{113}{149})^{7} \cdot \pi^{0} = (\frac{149}{113} \cdot \frac{113}{149})^{7} \cdot 1 = 1^{7} \cdot 1 = 1$$
\rozwStop
\odpStart
$1$
\odpStop
\testStart
A.$1$ B.$\pi$ C.$0$ D.$\frac{149}{113}$ E.$\frac{113}{149}$
F.$-\frac{149}{113}$ G.$-1$
H.$(\frac{149}{113})^{7}$
I.$(\frac{113}{149})^{7}$
\testStop
\kluczStart
A
\kluczStop



\zadStart{Zadanie z Wikieł Z 1.1 d) moja wersja nr 525}

Obliczyć wartość wyrażenia $(\frac{149}{127})^{7} \cdot (\frac{127}{149})^{7} \cdot \pi^{0}$.
\zadStop
\rozwStart{Patryk Wirkus}{Martyna Czarnobaj}
$$(\frac{149}{127})^{7} \cdot (\frac{127}{149})^{7} \cdot \pi^{0} = (\frac{149}{127} \cdot \frac{127}{149})^{7} \cdot 1 = 1^{7} \cdot 1 = 1$$
\rozwStop
\odpStart
$1$
\odpStop
\testStart
A.$1$ B.$\pi$ C.$0$ D.$\frac{149}{127}$ E.$\frac{127}{149}$
F.$-\frac{149}{127}$ G.$-1$
H.$(\frac{149}{127})^{7}$
I.$(\frac{127}{149})^{7}$
\testStop
\kluczStart
A
\kluczStop



\zadStart{Zadanie z Wikieł Z 1.1 d) moja wersja nr 526}

Obliczyć wartość wyrażenia $(\frac{149}{131})^{7} \cdot (\frac{131}{149})^{7} \cdot \pi^{0}$.
\zadStop
\rozwStart{Patryk Wirkus}{Martyna Czarnobaj}
$$(\frac{149}{131})^{7} \cdot (\frac{131}{149})^{7} \cdot \pi^{0} = (\frac{149}{131} \cdot \frac{131}{149})^{7} \cdot 1 = 1^{7} \cdot 1 = 1$$
\rozwStop
\odpStart
$1$
\odpStop
\testStart
A.$1$ B.$\pi$ C.$0$ D.$\frac{149}{131}$ E.$\frac{131}{149}$
F.$-\frac{149}{131}$ G.$-1$
H.$(\frac{149}{131})^{7}$
I.$(\frac{131}{149})^{7}$
\testStop
\kluczStart
A
\kluczStop



\zadStart{Zadanie z Wikieł Z 1.1 d) moja wersja nr 527}

Obliczyć wartość wyrażenia $(\frac{149}{137})^{7} \cdot (\frac{137}{149})^{7} \cdot \pi^{0}$.
\zadStop
\rozwStart{Patryk Wirkus}{Martyna Czarnobaj}
$$(\frac{149}{137})^{7} \cdot (\frac{137}{149})^{7} \cdot \pi^{0} = (\frac{149}{137} \cdot \frac{137}{149})^{7} \cdot 1 = 1^{7} \cdot 1 = 1$$
\rozwStop
\odpStart
$1$
\odpStop
\testStart
A.$1$ B.$\pi$ C.$0$ D.$\frac{149}{137}$ E.$\frac{137}{149}$
F.$-\frac{149}{137}$ G.$-1$
H.$(\frac{149}{137})^{7}$
I.$(\frac{137}{149})^{7}$
\testStop
\kluczStart
A
\kluczStop



\zadStart{Zadanie z Wikieł Z 1.1 d) moja wersja nr 528}

Obliczyć wartość wyrażenia $(\frac{149}{139})^{7} \cdot (\frac{139}{149})^{7} \cdot \pi^{0}$.
\zadStop
\rozwStart{Patryk Wirkus}{Martyna Czarnobaj}
$$(\frac{149}{139})^{7} \cdot (\frac{139}{149})^{7} \cdot \pi^{0} = (\frac{149}{139} \cdot \frac{139}{149})^{7} \cdot 1 = 1^{7} \cdot 1 = 1$$
\rozwStop
\odpStart
$1$
\odpStop
\testStart
A.$1$ B.$\pi$ C.$0$ D.$\frac{149}{139}$ E.$\frac{139}{149}$
F.$-\frac{149}{139}$ G.$-1$
H.$(\frac{149}{139})^{7}$
I.$(\frac{139}{149})^{7}$
\testStop
\kluczStart
A
\kluczStop



\zadStart{Zadanie z Wikieł Z 1.1 d) moja wersja nr 529}

Obliczyć wartość wyrażenia $(\frac{151}{103})^{7} \cdot (\frac{103}{151})^{7} \cdot \pi^{0}$.
\zadStop
\rozwStart{Patryk Wirkus}{Martyna Czarnobaj}
$$(\frac{151}{103})^{7} \cdot (\frac{103}{151})^{7} \cdot \pi^{0} = (\frac{151}{103} \cdot \frac{103}{151})^{7} \cdot 1 = 1^{7} \cdot 1 = 1$$
\rozwStop
\odpStart
$1$
\odpStop
\testStart
A.$1$ B.$\pi$ C.$0$ D.$\frac{151}{103}$ E.$\frac{103}{151}$
F.$-\frac{151}{103}$ G.$-1$
H.$(\frac{151}{103})^{7}$
I.$(\frac{103}{151})^{7}$
\testStop
\kluczStart
A
\kluczStop



\zadStart{Zadanie z Wikieł Z 1.1 d) moja wersja nr 530}

Obliczyć wartość wyrażenia $(\frac{151}{107})^{7} \cdot (\frac{107}{151})^{7} \cdot \pi^{0}$.
\zadStop
\rozwStart{Patryk Wirkus}{Martyna Czarnobaj}
$$(\frac{151}{107})^{7} \cdot (\frac{107}{151})^{7} \cdot \pi^{0} = (\frac{151}{107} \cdot \frac{107}{151})^{7} \cdot 1 = 1^{7} \cdot 1 = 1$$
\rozwStop
\odpStart
$1$
\odpStop
\testStart
A.$1$ B.$\pi$ C.$0$ D.$\frac{151}{107}$ E.$\frac{107}{151}$
F.$-\frac{151}{107}$ G.$-1$
H.$(\frac{151}{107})^{7}$
I.$(\frac{107}{151})^{7}$
\testStop
\kluczStart
A
\kluczStop



\zadStart{Zadanie z Wikieł Z 1.1 d) moja wersja nr 531}

Obliczyć wartość wyrażenia $(\frac{151}{109})^{7} \cdot (\frac{109}{151})^{7} \cdot \pi^{0}$.
\zadStop
\rozwStart{Patryk Wirkus}{Martyna Czarnobaj}
$$(\frac{151}{109})^{7} \cdot (\frac{109}{151})^{7} \cdot \pi^{0} = (\frac{151}{109} \cdot \frac{109}{151})^{7} \cdot 1 = 1^{7} \cdot 1 = 1$$
\rozwStop
\odpStart
$1$
\odpStop
\testStart
A.$1$ B.$\pi$ C.$0$ D.$\frac{151}{109}$ E.$\frac{109}{151}$
F.$-\frac{151}{109}$ G.$-1$
H.$(\frac{151}{109})^{7}$
I.$(\frac{109}{151})^{7}$
\testStop
\kluczStart
A
\kluczStop



\zadStart{Zadanie z Wikieł Z 1.1 d) moja wersja nr 532}

Obliczyć wartość wyrażenia $(\frac{151}{113})^{7} \cdot (\frac{113}{151})^{7} \cdot \pi^{0}$.
\zadStop
\rozwStart{Patryk Wirkus}{Martyna Czarnobaj}
$$(\frac{151}{113})^{7} \cdot (\frac{113}{151})^{7} \cdot \pi^{0} = (\frac{151}{113} \cdot \frac{113}{151})^{7} \cdot 1 = 1^{7} \cdot 1 = 1$$
\rozwStop
\odpStart
$1$
\odpStop
\testStart
A.$1$ B.$\pi$ C.$0$ D.$\frac{151}{113}$ E.$\frac{113}{151}$
F.$-\frac{151}{113}$ G.$-1$
H.$(\frac{151}{113})^{7}$
I.$(\frac{113}{151})^{7}$
\testStop
\kluczStart
A
\kluczStop



\zadStart{Zadanie z Wikieł Z 1.1 d) moja wersja nr 533}

Obliczyć wartość wyrażenia $(\frac{151}{127})^{7} \cdot (\frac{127}{151})^{7} \cdot \pi^{0}$.
\zadStop
\rozwStart{Patryk Wirkus}{Martyna Czarnobaj}
$$(\frac{151}{127})^{7} \cdot (\frac{127}{151})^{7} \cdot \pi^{0} = (\frac{151}{127} \cdot \frac{127}{151})^{7} \cdot 1 = 1^{7} \cdot 1 = 1$$
\rozwStop
\odpStart
$1$
\odpStop
\testStart
A.$1$ B.$\pi$ C.$0$ D.$\frac{151}{127}$ E.$\frac{127}{151}$
F.$-\frac{151}{127}$ G.$-1$
H.$(\frac{151}{127})^{7}$
I.$(\frac{127}{151})^{7}$
\testStop
\kluczStart
A
\kluczStop



\zadStart{Zadanie z Wikieł Z 1.1 d) moja wersja nr 534}

Obliczyć wartość wyrażenia $(\frac{151}{131})^{7} \cdot (\frac{131}{151})^{7} \cdot \pi^{0}$.
\zadStop
\rozwStart{Patryk Wirkus}{Martyna Czarnobaj}
$$(\frac{151}{131})^{7} \cdot (\frac{131}{151})^{7} \cdot \pi^{0} = (\frac{151}{131} \cdot \frac{131}{151})^{7} \cdot 1 = 1^{7} \cdot 1 = 1$$
\rozwStop
\odpStart
$1$
\odpStop
\testStart
A.$1$ B.$\pi$ C.$0$ D.$\frac{151}{131}$ E.$\frac{131}{151}$
F.$-\frac{151}{131}$ G.$-1$
H.$(\frac{151}{131})^{7}$
I.$(\frac{131}{151})^{7}$
\testStop
\kluczStart
A
\kluczStop



\zadStart{Zadanie z Wikieł Z 1.1 d) moja wersja nr 535}

Obliczyć wartość wyrażenia $(\frac{151}{137})^{7} \cdot (\frac{137}{151})^{7} \cdot \pi^{0}$.
\zadStop
\rozwStart{Patryk Wirkus}{Martyna Czarnobaj}
$$(\frac{151}{137})^{7} \cdot (\frac{137}{151})^{7} \cdot \pi^{0} = (\frac{151}{137} \cdot \frac{137}{151})^{7} \cdot 1 = 1^{7} \cdot 1 = 1$$
\rozwStop
\odpStart
$1$
\odpStop
\testStart
A.$1$ B.$\pi$ C.$0$ D.$\frac{151}{137}$ E.$\frac{137}{151}$
F.$-\frac{151}{137}$ G.$-1$
H.$(\frac{151}{137})^{7}$
I.$(\frac{137}{151})^{7}$
\testStop
\kluczStart
A
\kluczStop



\zadStart{Zadanie z Wikieł Z 1.1 d) moja wersja nr 536}

Obliczyć wartość wyrażenia $(\frac{151}{139})^{7} \cdot (\frac{139}{151})^{7} \cdot \pi^{0}$.
\zadStop
\rozwStart{Patryk Wirkus}{Martyna Czarnobaj}
$$(\frac{151}{139})^{7} \cdot (\frac{139}{151})^{7} \cdot \pi^{0} = (\frac{151}{139} \cdot \frac{139}{151})^{7} \cdot 1 = 1^{7} \cdot 1 = 1$$
\rozwStop
\odpStart
$1$
\odpStop
\testStart
A.$1$ B.$\pi$ C.$0$ D.$\frac{151}{139}$ E.$\frac{139}{151}$
F.$-\frac{151}{139}$ G.$-1$
H.$(\frac{151}{139})^{7}$
I.$(\frac{139}{151})^{7}$
\testStop
\kluczStart
A
\kluczStop



\zadStart{Zadanie z Wikieł Z 1.1 d) moja wersja nr 537}

Obliczyć wartość wyrażenia $(\frac{157}{103})^{7} \cdot (\frac{103}{157})^{7} \cdot \pi^{0}$.
\zadStop
\rozwStart{Patryk Wirkus}{Martyna Czarnobaj}
$$(\frac{157}{103})^{7} \cdot (\frac{103}{157})^{7} \cdot \pi^{0} = (\frac{157}{103} \cdot \frac{103}{157})^{7} \cdot 1 = 1^{7} \cdot 1 = 1$$
\rozwStop
\odpStart
$1$
\odpStop
\testStart
A.$1$ B.$\pi$ C.$0$ D.$\frac{157}{103}$ E.$\frac{103}{157}$
F.$-\frac{157}{103}$ G.$-1$
H.$(\frac{157}{103})^{7}$
I.$(\frac{103}{157})^{7}$
\testStop
\kluczStart
A
\kluczStop



\zadStart{Zadanie z Wikieł Z 1.1 d) moja wersja nr 538}

Obliczyć wartość wyrażenia $(\frac{157}{107})^{7} \cdot (\frac{107}{157})^{7} \cdot \pi^{0}$.
\zadStop
\rozwStart{Patryk Wirkus}{Martyna Czarnobaj}
$$(\frac{157}{107})^{7} \cdot (\frac{107}{157})^{7} \cdot \pi^{0} = (\frac{157}{107} \cdot \frac{107}{157})^{7} \cdot 1 = 1^{7} \cdot 1 = 1$$
\rozwStop
\odpStart
$1$
\odpStop
\testStart
A.$1$ B.$\pi$ C.$0$ D.$\frac{157}{107}$ E.$\frac{107}{157}$
F.$-\frac{157}{107}$ G.$-1$
H.$(\frac{157}{107})^{7}$
I.$(\frac{107}{157})^{7}$
\testStop
\kluczStart
A
\kluczStop



\zadStart{Zadanie z Wikieł Z 1.1 d) moja wersja nr 539}

Obliczyć wartość wyrażenia $(\frac{157}{109})^{7} \cdot (\frac{109}{157})^{7} \cdot \pi^{0}$.
\zadStop
\rozwStart{Patryk Wirkus}{Martyna Czarnobaj}
$$(\frac{157}{109})^{7} \cdot (\frac{109}{157})^{7} \cdot \pi^{0} = (\frac{157}{109} \cdot \frac{109}{157})^{7} \cdot 1 = 1^{7} \cdot 1 = 1$$
\rozwStop
\odpStart
$1$
\odpStop
\testStart
A.$1$ B.$\pi$ C.$0$ D.$\frac{157}{109}$ E.$\frac{109}{157}$
F.$-\frac{157}{109}$ G.$-1$
H.$(\frac{157}{109})^{7}$
I.$(\frac{109}{157})^{7}$
\testStop
\kluczStart
A
\kluczStop



\zadStart{Zadanie z Wikieł Z 1.1 d) moja wersja nr 540}

Obliczyć wartość wyrażenia $(\frac{157}{113})^{7} \cdot (\frac{113}{157})^{7} \cdot \pi^{0}$.
\zadStop
\rozwStart{Patryk Wirkus}{Martyna Czarnobaj}
$$(\frac{157}{113})^{7} \cdot (\frac{113}{157})^{7} \cdot \pi^{0} = (\frac{157}{113} \cdot \frac{113}{157})^{7} \cdot 1 = 1^{7} \cdot 1 = 1$$
\rozwStop
\odpStart
$1$
\odpStop
\testStart
A.$1$ B.$\pi$ C.$0$ D.$\frac{157}{113}$ E.$\frac{113}{157}$
F.$-\frac{157}{113}$ G.$-1$
H.$(\frac{157}{113})^{7}$
I.$(\frac{113}{157})^{7}$
\testStop
\kluczStart
A
\kluczStop



\zadStart{Zadanie z Wikieł Z 1.1 d) moja wersja nr 541}

Obliczyć wartość wyrażenia $(\frac{157}{127})^{7} \cdot (\frac{127}{157})^{7} \cdot \pi^{0}$.
\zadStop
\rozwStart{Patryk Wirkus}{Martyna Czarnobaj}
$$(\frac{157}{127})^{7} \cdot (\frac{127}{157})^{7} \cdot \pi^{0} = (\frac{157}{127} \cdot \frac{127}{157})^{7} \cdot 1 = 1^{7} \cdot 1 = 1$$
\rozwStop
\odpStart
$1$
\odpStop
\testStart
A.$1$ B.$\pi$ C.$0$ D.$\frac{157}{127}$ E.$\frac{127}{157}$
F.$-\frac{157}{127}$ G.$-1$
H.$(\frac{157}{127})^{7}$
I.$(\frac{127}{157})^{7}$
\testStop
\kluczStart
A
\kluczStop



\zadStart{Zadanie z Wikieł Z 1.1 d) moja wersja nr 542}

Obliczyć wartość wyrażenia $(\frac{157}{131})^{7} \cdot (\frac{131}{157})^{7} \cdot \pi^{0}$.
\zadStop
\rozwStart{Patryk Wirkus}{Martyna Czarnobaj}
$$(\frac{157}{131})^{7} \cdot (\frac{131}{157})^{7} \cdot \pi^{0} = (\frac{157}{131} \cdot \frac{131}{157})^{7} \cdot 1 = 1^{7} \cdot 1 = 1$$
\rozwStop
\odpStart
$1$
\odpStop
\testStart
A.$1$ B.$\pi$ C.$0$ D.$\frac{157}{131}$ E.$\frac{131}{157}$
F.$-\frac{157}{131}$ G.$-1$
H.$(\frac{157}{131})^{7}$
I.$(\frac{131}{157})^{7}$
\testStop
\kluczStart
A
\kluczStop



\zadStart{Zadanie z Wikieł Z 1.1 d) moja wersja nr 543}

Obliczyć wartość wyrażenia $(\frac{157}{137})^{7} \cdot (\frac{137}{157})^{7} \cdot \pi^{0}$.
\zadStop
\rozwStart{Patryk Wirkus}{Martyna Czarnobaj}
$$(\frac{157}{137})^{7} \cdot (\frac{137}{157})^{7} \cdot \pi^{0} = (\frac{157}{137} \cdot \frac{137}{157})^{7} \cdot 1 = 1^{7} \cdot 1 = 1$$
\rozwStop
\odpStart
$1$
\odpStop
\testStart
A.$1$ B.$\pi$ C.$0$ D.$\frac{157}{137}$ E.$\frac{137}{157}$
F.$-\frac{157}{137}$ G.$-1$
H.$(\frac{157}{137})^{7}$
I.$(\frac{137}{157})^{7}$
\testStop
\kluczStart
A
\kluczStop



\zadStart{Zadanie z Wikieł Z 1.1 d) moja wersja nr 544}

Obliczyć wartość wyrażenia $(\frac{157}{139})^{7} \cdot (\frac{139}{157})^{7} \cdot \pi^{0}$.
\zadStop
\rozwStart{Patryk Wirkus}{Martyna Czarnobaj}
$$(\frac{157}{139})^{7} \cdot (\frac{139}{157})^{7} \cdot \pi^{0} = (\frac{157}{139} \cdot \frac{139}{157})^{7} \cdot 1 = 1^{7} \cdot 1 = 1$$
\rozwStop
\odpStart
$1$
\odpStop
\testStart
A.$1$ B.$\pi$ C.$0$ D.$\frac{157}{139}$ E.$\frac{139}{157}$
F.$-\frac{157}{139}$ G.$-1$
H.$(\frac{157}{139})^{7}$
I.$(\frac{139}{157})^{7}$
\testStop
\kluczStart
A
\kluczStop



\zadStart{Zadanie z Wikieł Z 1.1 d) moja wersja nr 545}

Obliczyć wartość wyrażenia $(\frac{163}{103})^{7} \cdot (\frac{103}{163})^{7} \cdot \pi^{0}$.
\zadStop
\rozwStart{Patryk Wirkus}{Martyna Czarnobaj}
$$(\frac{163}{103})^{7} \cdot (\frac{103}{163})^{7} \cdot \pi^{0} = (\frac{163}{103} \cdot \frac{103}{163})^{7} \cdot 1 = 1^{7} \cdot 1 = 1$$
\rozwStop
\odpStart
$1$
\odpStop
\testStart
A.$1$ B.$\pi$ C.$0$ D.$\frac{163}{103}$ E.$\frac{103}{163}$
F.$-\frac{163}{103}$ G.$-1$
H.$(\frac{163}{103})^{7}$
I.$(\frac{103}{163})^{7}$
\testStop
\kluczStart
A
\kluczStop



\zadStart{Zadanie z Wikieł Z 1.1 d) moja wersja nr 546}

Obliczyć wartość wyrażenia $(\frac{163}{107})^{7} \cdot (\frac{107}{163})^{7} \cdot \pi^{0}$.
\zadStop
\rozwStart{Patryk Wirkus}{Martyna Czarnobaj}
$$(\frac{163}{107})^{7} \cdot (\frac{107}{163})^{7} \cdot \pi^{0} = (\frac{163}{107} \cdot \frac{107}{163})^{7} \cdot 1 = 1^{7} \cdot 1 = 1$$
\rozwStop
\odpStart
$1$
\odpStop
\testStart
A.$1$ B.$\pi$ C.$0$ D.$\frac{163}{107}$ E.$\frac{107}{163}$
F.$-\frac{163}{107}$ G.$-1$
H.$(\frac{163}{107})^{7}$
I.$(\frac{107}{163})^{7}$
\testStop
\kluczStart
A
\kluczStop



\zadStart{Zadanie z Wikieł Z 1.1 d) moja wersja nr 547}

Obliczyć wartość wyrażenia $(\frac{163}{109})^{7} \cdot (\frac{109}{163})^{7} \cdot \pi^{0}$.
\zadStop
\rozwStart{Patryk Wirkus}{Martyna Czarnobaj}
$$(\frac{163}{109})^{7} \cdot (\frac{109}{163})^{7} \cdot \pi^{0} = (\frac{163}{109} \cdot \frac{109}{163})^{7} \cdot 1 = 1^{7} \cdot 1 = 1$$
\rozwStop
\odpStart
$1$
\odpStop
\testStart
A.$1$ B.$\pi$ C.$0$ D.$\frac{163}{109}$ E.$\frac{109}{163}$
F.$-\frac{163}{109}$ G.$-1$
H.$(\frac{163}{109})^{7}$
I.$(\frac{109}{163})^{7}$
\testStop
\kluczStart
A
\kluczStop



\zadStart{Zadanie z Wikieł Z 1.1 d) moja wersja nr 548}

Obliczyć wartość wyrażenia $(\frac{163}{113})^{7} \cdot (\frac{113}{163})^{7} \cdot \pi^{0}$.
\zadStop
\rozwStart{Patryk Wirkus}{Martyna Czarnobaj}
$$(\frac{163}{113})^{7} \cdot (\frac{113}{163})^{7} \cdot \pi^{0} = (\frac{163}{113} \cdot \frac{113}{163})^{7} \cdot 1 = 1^{7} \cdot 1 = 1$$
\rozwStop
\odpStart
$1$
\odpStop
\testStart
A.$1$ B.$\pi$ C.$0$ D.$\frac{163}{113}$ E.$\frac{113}{163}$
F.$-\frac{163}{113}$ G.$-1$
H.$(\frac{163}{113})^{7}$
I.$(\frac{113}{163})^{7}$
\testStop
\kluczStart
A
\kluczStop



\zadStart{Zadanie z Wikieł Z 1.1 d) moja wersja nr 549}

Obliczyć wartość wyrażenia $(\frac{163}{127})^{7} \cdot (\frac{127}{163})^{7} \cdot \pi^{0}$.
\zadStop
\rozwStart{Patryk Wirkus}{Martyna Czarnobaj}
$$(\frac{163}{127})^{7} \cdot (\frac{127}{163})^{7} \cdot \pi^{0} = (\frac{163}{127} \cdot \frac{127}{163})^{7} \cdot 1 = 1^{7} \cdot 1 = 1$$
\rozwStop
\odpStart
$1$
\odpStop
\testStart
A.$1$ B.$\pi$ C.$0$ D.$\frac{163}{127}$ E.$\frac{127}{163}$
F.$-\frac{163}{127}$ G.$-1$
H.$(\frac{163}{127})^{7}$
I.$(\frac{127}{163})^{7}$
\testStop
\kluczStart
A
\kluczStop



\zadStart{Zadanie z Wikieł Z 1.1 d) moja wersja nr 550}

Obliczyć wartość wyrażenia $(\frac{163}{131})^{7} \cdot (\frac{131}{163})^{7} \cdot \pi^{0}$.
\zadStop
\rozwStart{Patryk Wirkus}{Martyna Czarnobaj}
$$(\frac{163}{131})^{7} \cdot (\frac{131}{163})^{7} \cdot \pi^{0} = (\frac{163}{131} \cdot \frac{131}{163})^{7} \cdot 1 = 1^{7} \cdot 1 = 1$$
\rozwStop
\odpStart
$1$
\odpStop
\testStart
A.$1$ B.$\pi$ C.$0$ D.$\frac{163}{131}$ E.$\frac{131}{163}$
F.$-\frac{163}{131}$ G.$-1$
H.$(\frac{163}{131})^{7}$
I.$(\frac{131}{163})^{7}$
\testStop
\kluczStart
A
\kluczStop



\zadStart{Zadanie z Wikieł Z 1.1 d) moja wersja nr 551}

Obliczyć wartość wyrażenia $(\frac{163}{137})^{7} \cdot (\frac{137}{163})^{7} \cdot \pi^{0}$.
\zadStop
\rozwStart{Patryk Wirkus}{Martyna Czarnobaj}
$$(\frac{163}{137})^{7} \cdot (\frac{137}{163})^{7} \cdot \pi^{0} = (\frac{163}{137} \cdot \frac{137}{163})^{7} \cdot 1 = 1^{7} \cdot 1 = 1$$
\rozwStop
\odpStart
$1$
\odpStop
\testStart
A.$1$ B.$\pi$ C.$0$ D.$\frac{163}{137}$ E.$\frac{137}{163}$
F.$-\frac{163}{137}$ G.$-1$
H.$(\frac{163}{137})^{7}$
I.$(\frac{137}{163})^{7}$
\testStop
\kluczStart
A
\kluczStop



\zadStart{Zadanie z Wikieł Z 1.1 d) moja wersja nr 552}

Obliczyć wartość wyrażenia $(\frac{163}{139})^{7} \cdot (\frac{139}{163})^{7} \cdot \pi^{0}$.
\zadStop
\rozwStart{Patryk Wirkus}{Martyna Czarnobaj}
$$(\frac{163}{139})^{7} \cdot (\frac{139}{163})^{7} \cdot \pi^{0} = (\frac{163}{139} \cdot \frac{139}{163})^{7} \cdot 1 = 1^{7} \cdot 1 = 1$$
\rozwStop
\odpStart
$1$
\odpStop
\testStart
A.$1$ B.$\pi$ C.$0$ D.$\frac{163}{139}$ E.$\frac{139}{163}$
F.$-\frac{163}{139}$ G.$-1$
H.$(\frac{163}{139})^{7}$
I.$(\frac{139}{163})^{7}$
\testStop
\kluczStart
A
\kluczStop



\zadStart{Zadanie z Wikieł Z 1.1 d) moja wersja nr 553}

Obliczyć wartość wyrażenia $(\frac{167}{103})^{7} \cdot (\frac{103}{167})^{7} \cdot \pi^{0}$.
\zadStop
\rozwStart{Patryk Wirkus}{Martyna Czarnobaj}
$$(\frac{167}{103})^{7} \cdot (\frac{103}{167})^{7} \cdot \pi^{0} = (\frac{167}{103} \cdot \frac{103}{167})^{7} \cdot 1 = 1^{7} \cdot 1 = 1$$
\rozwStop
\odpStart
$1$
\odpStop
\testStart
A.$1$ B.$\pi$ C.$0$ D.$\frac{167}{103}$ E.$\frac{103}{167}$
F.$-\frac{167}{103}$ G.$-1$
H.$(\frac{167}{103})^{7}$
I.$(\frac{103}{167})^{7}$
\testStop
\kluczStart
A
\kluczStop



\zadStart{Zadanie z Wikieł Z 1.1 d) moja wersja nr 554}

Obliczyć wartość wyrażenia $(\frac{167}{107})^{7} \cdot (\frac{107}{167})^{7} \cdot \pi^{0}$.
\zadStop
\rozwStart{Patryk Wirkus}{Martyna Czarnobaj}
$$(\frac{167}{107})^{7} \cdot (\frac{107}{167})^{7} \cdot \pi^{0} = (\frac{167}{107} \cdot \frac{107}{167})^{7} \cdot 1 = 1^{7} \cdot 1 = 1$$
\rozwStop
\odpStart
$1$
\odpStop
\testStart
A.$1$ B.$\pi$ C.$0$ D.$\frac{167}{107}$ E.$\frac{107}{167}$
F.$-\frac{167}{107}$ G.$-1$
H.$(\frac{167}{107})^{7}$
I.$(\frac{107}{167})^{7}$
\testStop
\kluczStart
A
\kluczStop



\zadStart{Zadanie z Wikieł Z 1.1 d) moja wersja nr 555}

Obliczyć wartość wyrażenia $(\frac{167}{109})^{7} \cdot (\frac{109}{167})^{7} \cdot \pi^{0}$.
\zadStop
\rozwStart{Patryk Wirkus}{Martyna Czarnobaj}
$$(\frac{167}{109})^{7} \cdot (\frac{109}{167})^{7} \cdot \pi^{0} = (\frac{167}{109} \cdot \frac{109}{167})^{7} \cdot 1 = 1^{7} \cdot 1 = 1$$
\rozwStop
\odpStart
$1$
\odpStop
\testStart
A.$1$ B.$\pi$ C.$0$ D.$\frac{167}{109}$ E.$\frac{109}{167}$
F.$-\frac{167}{109}$ G.$-1$
H.$(\frac{167}{109})^{7}$
I.$(\frac{109}{167})^{7}$
\testStop
\kluczStart
A
\kluczStop



\zadStart{Zadanie z Wikieł Z 1.1 d) moja wersja nr 556}

Obliczyć wartość wyrażenia $(\frac{167}{113})^{7} \cdot (\frac{113}{167})^{7} \cdot \pi^{0}$.
\zadStop
\rozwStart{Patryk Wirkus}{Martyna Czarnobaj}
$$(\frac{167}{113})^{7} \cdot (\frac{113}{167})^{7} \cdot \pi^{0} = (\frac{167}{113} \cdot \frac{113}{167})^{7} \cdot 1 = 1^{7} \cdot 1 = 1$$
\rozwStop
\odpStart
$1$
\odpStop
\testStart
A.$1$ B.$\pi$ C.$0$ D.$\frac{167}{113}$ E.$\frac{113}{167}$
F.$-\frac{167}{113}$ G.$-1$
H.$(\frac{167}{113})^{7}$
I.$(\frac{113}{167})^{7}$
\testStop
\kluczStart
A
\kluczStop



\zadStart{Zadanie z Wikieł Z 1.1 d) moja wersja nr 557}

Obliczyć wartość wyrażenia $(\frac{167}{127})^{7} \cdot (\frac{127}{167})^{7} \cdot \pi^{0}$.
\zadStop
\rozwStart{Patryk Wirkus}{Martyna Czarnobaj}
$$(\frac{167}{127})^{7} \cdot (\frac{127}{167})^{7} \cdot \pi^{0} = (\frac{167}{127} \cdot \frac{127}{167})^{7} \cdot 1 = 1^{7} \cdot 1 = 1$$
\rozwStop
\odpStart
$1$
\odpStop
\testStart
A.$1$ B.$\pi$ C.$0$ D.$\frac{167}{127}$ E.$\frac{127}{167}$
F.$-\frac{167}{127}$ G.$-1$
H.$(\frac{167}{127})^{7}$
I.$(\frac{127}{167})^{7}$
\testStop
\kluczStart
A
\kluczStop



\zadStart{Zadanie z Wikieł Z 1.1 d) moja wersja nr 558}

Obliczyć wartość wyrażenia $(\frac{167}{131})^{7} \cdot (\frac{131}{167})^{7} \cdot \pi^{0}$.
\zadStop
\rozwStart{Patryk Wirkus}{Martyna Czarnobaj}
$$(\frac{167}{131})^{7} \cdot (\frac{131}{167})^{7} \cdot \pi^{0} = (\frac{167}{131} \cdot \frac{131}{167})^{7} \cdot 1 = 1^{7} \cdot 1 = 1$$
\rozwStop
\odpStart
$1$
\odpStop
\testStart
A.$1$ B.$\pi$ C.$0$ D.$\frac{167}{131}$ E.$\frac{131}{167}$
F.$-\frac{167}{131}$ G.$-1$
H.$(\frac{167}{131})^{7}$
I.$(\frac{131}{167})^{7}$
\testStop
\kluczStart
A
\kluczStop



\zadStart{Zadanie z Wikieł Z 1.1 d) moja wersja nr 559}

Obliczyć wartość wyrażenia $(\frac{167}{137})^{7} \cdot (\frac{137}{167})^{7} \cdot \pi^{0}$.
\zadStop
\rozwStart{Patryk Wirkus}{Martyna Czarnobaj}
$$(\frac{167}{137})^{7} \cdot (\frac{137}{167})^{7} \cdot \pi^{0} = (\frac{167}{137} \cdot \frac{137}{167})^{7} \cdot 1 = 1^{7} \cdot 1 = 1$$
\rozwStop
\odpStart
$1$
\odpStop
\testStart
A.$1$ B.$\pi$ C.$0$ D.$\frac{167}{137}$ E.$\frac{137}{167}$
F.$-\frac{167}{137}$ G.$-1$
H.$(\frac{167}{137})^{7}$
I.$(\frac{137}{167})^{7}$
\testStop
\kluczStart
A
\kluczStop



\zadStart{Zadanie z Wikieł Z 1.1 d) moja wersja nr 560}

Obliczyć wartość wyrażenia $(\frac{167}{139})^{7} \cdot (\frac{139}{167})^{7} \cdot \pi^{0}$.
\zadStop
\rozwStart{Patryk Wirkus}{Martyna Czarnobaj}
$$(\frac{167}{139})^{7} \cdot (\frac{139}{167})^{7} \cdot \pi^{0} = (\frac{167}{139} \cdot \frac{139}{167})^{7} \cdot 1 = 1^{7} \cdot 1 = 1$$
\rozwStop
\odpStart
$1$
\odpStop
\testStart
A.$1$ B.$\pi$ C.$0$ D.$\frac{167}{139}$ E.$\frac{139}{167}$
F.$-\frac{167}{139}$ G.$-1$
H.$(\frac{167}{139})^{7}$
I.$(\frac{139}{167})^{7}$
\testStop
\kluczStart
A
\kluczStop



\zadStart{Zadanie z Wikieł Z 1.1 d) moja wersja nr 561}

Obliczyć wartość wyrażenia $(\frac{173}{103})^{7} \cdot (\frac{103}{173})^{7} \cdot \pi^{0}$.
\zadStop
\rozwStart{Patryk Wirkus}{Martyna Czarnobaj}
$$(\frac{173}{103})^{7} \cdot (\frac{103}{173})^{7} \cdot \pi^{0} = (\frac{173}{103} \cdot \frac{103}{173})^{7} \cdot 1 = 1^{7} \cdot 1 = 1$$
\rozwStop
\odpStart
$1$
\odpStop
\testStart
A.$1$ B.$\pi$ C.$0$ D.$\frac{173}{103}$ E.$\frac{103}{173}$
F.$-\frac{173}{103}$ G.$-1$
H.$(\frac{173}{103})^{7}$
I.$(\frac{103}{173})^{7}$
\testStop
\kluczStart
A
\kluczStop



\zadStart{Zadanie z Wikieł Z 1.1 d) moja wersja nr 562}

Obliczyć wartość wyrażenia $(\frac{173}{107})^{7} \cdot (\frac{107}{173})^{7} \cdot \pi^{0}$.
\zadStop
\rozwStart{Patryk Wirkus}{Martyna Czarnobaj}
$$(\frac{173}{107})^{7} \cdot (\frac{107}{173})^{7} \cdot \pi^{0} = (\frac{173}{107} \cdot \frac{107}{173})^{7} \cdot 1 = 1^{7} \cdot 1 = 1$$
\rozwStop
\odpStart
$1$
\odpStop
\testStart
A.$1$ B.$\pi$ C.$0$ D.$\frac{173}{107}$ E.$\frac{107}{173}$
F.$-\frac{173}{107}$ G.$-1$
H.$(\frac{173}{107})^{7}$
I.$(\frac{107}{173})^{7}$
\testStop
\kluczStart
A
\kluczStop



\zadStart{Zadanie z Wikieł Z 1.1 d) moja wersja nr 563}

Obliczyć wartość wyrażenia $(\frac{173}{109})^{7} \cdot (\frac{109}{173})^{7} \cdot \pi^{0}$.
\zadStop
\rozwStart{Patryk Wirkus}{Martyna Czarnobaj}
$$(\frac{173}{109})^{7} \cdot (\frac{109}{173})^{7} \cdot \pi^{0} = (\frac{173}{109} \cdot \frac{109}{173})^{7} \cdot 1 = 1^{7} \cdot 1 = 1$$
\rozwStop
\odpStart
$1$
\odpStop
\testStart
A.$1$ B.$\pi$ C.$0$ D.$\frac{173}{109}$ E.$\frac{109}{173}$
F.$-\frac{173}{109}$ G.$-1$
H.$(\frac{173}{109})^{7}$
I.$(\frac{109}{173})^{7}$
\testStop
\kluczStart
A
\kluczStop



\zadStart{Zadanie z Wikieł Z 1.1 d) moja wersja nr 564}

Obliczyć wartość wyrażenia $(\frac{173}{113})^{7} \cdot (\frac{113}{173})^{7} \cdot \pi^{0}$.
\zadStop
\rozwStart{Patryk Wirkus}{Martyna Czarnobaj}
$$(\frac{173}{113})^{7} \cdot (\frac{113}{173})^{7} \cdot \pi^{0} = (\frac{173}{113} \cdot \frac{113}{173})^{7} \cdot 1 = 1^{7} \cdot 1 = 1$$
\rozwStop
\odpStart
$1$
\odpStop
\testStart
A.$1$ B.$\pi$ C.$0$ D.$\frac{173}{113}$ E.$\frac{113}{173}$
F.$-\frac{173}{113}$ G.$-1$
H.$(\frac{173}{113})^{7}$
I.$(\frac{113}{173})^{7}$
\testStop
\kluczStart
A
\kluczStop



\zadStart{Zadanie z Wikieł Z 1.1 d) moja wersja nr 565}

Obliczyć wartość wyrażenia $(\frac{173}{127})^{7} \cdot (\frac{127}{173})^{7} \cdot \pi^{0}$.
\zadStop
\rozwStart{Patryk Wirkus}{Martyna Czarnobaj}
$$(\frac{173}{127})^{7} \cdot (\frac{127}{173})^{7} \cdot \pi^{0} = (\frac{173}{127} \cdot \frac{127}{173})^{7} \cdot 1 = 1^{7} \cdot 1 = 1$$
\rozwStop
\odpStart
$1$
\odpStop
\testStart
A.$1$ B.$\pi$ C.$0$ D.$\frac{173}{127}$ E.$\frac{127}{173}$
F.$-\frac{173}{127}$ G.$-1$
H.$(\frac{173}{127})^{7}$
I.$(\frac{127}{173})^{7}$
\testStop
\kluczStart
A
\kluczStop



\zadStart{Zadanie z Wikieł Z 1.1 d) moja wersja nr 566}

Obliczyć wartość wyrażenia $(\frac{173}{131})^{7} \cdot (\frac{131}{173})^{7} \cdot \pi^{0}$.
\zadStop
\rozwStart{Patryk Wirkus}{Martyna Czarnobaj}
$$(\frac{173}{131})^{7} \cdot (\frac{131}{173})^{7} \cdot \pi^{0} = (\frac{173}{131} \cdot \frac{131}{173})^{7} \cdot 1 = 1^{7} \cdot 1 = 1$$
\rozwStop
\odpStart
$1$
\odpStop
\testStart
A.$1$ B.$\pi$ C.$0$ D.$\frac{173}{131}$ E.$\frac{131}{173}$
F.$-\frac{173}{131}$ G.$-1$
H.$(\frac{173}{131})^{7}$
I.$(\frac{131}{173})^{7}$
\testStop
\kluczStart
A
\kluczStop



\zadStart{Zadanie z Wikieł Z 1.1 d) moja wersja nr 567}

Obliczyć wartość wyrażenia $(\frac{173}{137})^{7} \cdot (\frac{137}{173})^{7} \cdot \pi^{0}$.
\zadStop
\rozwStart{Patryk Wirkus}{Martyna Czarnobaj}
$$(\frac{173}{137})^{7} \cdot (\frac{137}{173})^{7} \cdot \pi^{0} = (\frac{173}{137} \cdot \frac{137}{173})^{7} \cdot 1 = 1^{7} \cdot 1 = 1$$
\rozwStop
\odpStart
$1$
\odpStop
\testStart
A.$1$ B.$\pi$ C.$0$ D.$\frac{173}{137}$ E.$\frac{137}{173}$
F.$-\frac{173}{137}$ G.$-1$
H.$(\frac{173}{137})^{7}$
I.$(\frac{137}{173})^{7}$
\testStop
\kluczStart
A
\kluczStop



\zadStart{Zadanie z Wikieł Z 1.1 d) moja wersja nr 568}

Obliczyć wartość wyrażenia $(\frac{173}{139})^{7} \cdot (\frac{139}{173})^{7} \cdot \pi^{0}$.
\zadStop
\rozwStart{Patryk Wirkus}{Martyna Czarnobaj}
$$(\frac{173}{139})^{7} \cdot (\frac{139}{173})^{7} \cdot \pi^{0} = (\frac{173}{139} \cdot \frac{139}{173})^{7} \cdot 1 = 1^{7} \cdot 1 = 1$$
\rozwStop
\odpStart
$1$
\odpStop
\testStart
A.$1$ B.$\pi$ C.$0$ D.$\frac{173}{139}$ E.$\frac{139}{173}$
F.$-\frac{173}{139}$ G.$-1$
H.$(\frac{173}{139})^{7}$
I.$(\frac{139}{173})^{7}$
\testStop
\kluczStart
A
\kluczStop



\zadStart{Zadanie z Wikieł Z 1.1 d) moja wersja nr 569}

Obliczyć wartość wyrażenia $(\frac{179}{103})^{7} \cdot (\frac{103}{179})^{7} \cdot \pi^{0}$.
\zadStop
\rozwStart{Patryk Wirkus}{Martyna Czarnobaj}
$$(\frac{179}{103})^{7} \cdot (\frac{103}{179})^{7} \cdot \pi^{0} = (\frac{179}{103} \cdot \frac{103}{179})^{7} \cdot 1 = 1^{7} \cdot 1 = 1$$
\rozwStop
\odpStart
$1$
\odpStop
\testStart
A.$1$ B.$\pi$ C.$0$ D.$\frac{179}{103}$ E.$\frac{103}{179}$
F.$-\frac{179}{103}$ G.$-1$
H.$(\frac{179}{103})^{7}$
I.$(\frac{103}{179})^{7}$
\testStop
\kluczStart
A
\kluczStop



\zadStart{Zadanie z Wikieł Z 1.1 d) moja wersja nr 570}

Obliczyć wartość wyrażenia $(\frac{179}{107})^{7} \cdot (\frac{107}{179})^{7} \cdot \pi^{0}$.
\zadStop
\rozwStart{Patryk Wirkus}{Martyna Czarnobaj}
$$(\frac{179}{107})^{7} \cdot (\frac{107}{179})^{7} \cdot \pi^{0} = (\frac{179}{107} \cdot \frac{107}{179})^{7} \cdot 1 = 1^{7} \cdot 1 = 1$$
\rozwStop
\odpStart
$1$
\odpStop
\testStart
A.$1$ B.$\pi$ C.$0$ D.$\frac{179}{107}$ E.$\frac{107}{179}$
F.$-\frac{179}{107}$ G.$-1$
H.$(\frac{179}{107})^{7}$
I.$(\frac{107}{179})^{7}$
\testStop
\kluczStart
A
\kluczStop



\zadStart{Zadanie z Wikieł Z 1.1 d) moja wersja nr 571}

Obliczyć wartość wyrażenia $(\frac{179}{109})^{7} \cdot (\frac{109}{179})^{7} \cdot \pi^{0}$.
\zadStop
\rozwStart{Patryk Wirkus}{Martyna Czarnobaj}
$$(\frac{179}{109})^{7} \cdot (\frac{109}{179})^{7} \cdot \pi^{0} = (\frac{179}{109} \cdot \frac{109}{179})^{7} \cdot 1 = 1^{7} \cdot 1 = 1$$
\rozwStop
\odpStart
$1$
\odpStop
\testStart
A.$1$ B.$\pi$ C.$0$ D.$\frac{179}{109}$ E.$\frac{109}{179}$
F.$-\frac{179}{109}$ G.$-1$
H.$(\frac{179}{109})^{7}$
I.$(\frac{109}{179})^{7}$
\testStop
\kluczStart
A
\kluczStop



\zadStart{Zadanie z Wikieł Z 1.1 d) moja wersja nr 572}

Obliczyć wartość wyrażenia $(\frac{179}{113})^{7} \cdot (\frac{113}{179})^{7} \cdot \pi^{0}$.
\zadStop
\rozwStart{Patryk Wirkus}{Martyna Czarnobaj}
$$(\frac{179}{113})^{7} \cdot (\frac{113}{179})^{7} \cdot \pi^{0} = (\frac{179}{113} \cdot \frac{113}{179})^{7} \cdot 1 = 1^{7} \cdot 1 = 1$$
\rozwStop
\odpStart
$1$
\odpStop
\testStart
A.$1$ B.$\pi$ C.$0$ D.$\frac{179}{113}$ E.$\frac{113}{179}$
F.$-\frac{179}{113}$ G.$-1$
H.$(\frac{179}{113})^{7}$
I.$(\frac{113}{179})^{7}$
\testStop
\kluczStart
A
\kluczStop



\zadStart{Zadanie z Wikieł Z 1.1 d) moja wersja nr 573}

Obliczyć wartość wyrażenia $(\frac{179}{127})^{7} \cdot (\frac{127}{179})^{7} \cdot \pi^{0}$.
\zadStop
\rozwStart{Patryk Wirkus}{Martyna Czarnobaj}
$$(\frac{179}{127})^{7} \cdot (\frac{127}{179})^{7} \cdot \pi^{0} = (\frac{179}{127} \cdot \frac{127}{179})^{7} \cdot 1 = 1^{7} \cdot 1 = 1$$
\rozwStop
\odpStart
$1$
\odpStop
\testStart
A.$1$ B.$\pi$ C.$0$ D.$\frac{179}{127}$ E.$\frac{127}{179}$
F.$-\frac{179}{127}$ G.$-1$
H.$(\frac{179}{127})^{7}$
I.$(\frac{127}{179})^{7}$
\testStop
\kluczStart
A
\kluczStop



\zadStart{Zadanie z Wikieł Z 1.1 d) moja wersja nr 574}

Obliczyć wartość wyrażenia $(\frac{179}{131})^{7} \cdot (\frac{131}{179})^{7} \cdot \pi^{0}$.
\zadStop
\rozwStart{Patryk Wirkus}{Martyna Czarnobaj}
$$(\frac{179}{131})^{7} \cdot (\frac{131}{179})^{7} \cdot \pi^{0} = (\frac{179}{131} \cdot \frac{131}{179})^{7} \cdot 1 = 1^{7} \cdot 1 = 1$$
\rozwStop
\odpStart
$1$
\odpStop
\testStart
A.$1$ B.$\pi$ C.$0$ D.$\frac{179}{131}$ E.$\frac{131}{179}$
F.$-\frac{179}{131}$ G.$-1$
H.$(\frac{179}{131})^{7}$
I.$(\frac{131}{179})^{7}$
\testStop
\kluczStart
A
\kluczStop



\zadStart{Zadanie z Wikieł Z 1.1 d) moja wersja nr 575}

Obliczyć wartość wyrażenia $(\frac{179}{137})^{7} \cdot (\frac{137}{179})^{7} \cdot \pi^{0}$.
\zadStop
\rozwStart{Patryk Wirkus}{Martyna Czarnobaj}
$$(\frac{179}{137})^{7} \cdot (\frac{137}{179})^{7} \cdot \pi^{0} = (\frac{179}{137} \cdot \frac{137}{179})^{7} \cdot 1 = 1^{7} \cdot 1 = 1$$
\rozwStop
\odpStart
$1$
\odpStop
\testStart
A.$1$ B.$\pi$ C.$0$ D.$\frac{179}{137}$ E.$\frac{137}{179}$
F.$-\frac{179}{137}$ G.$-1$
H.$(\frac{179}{137})^{7}$
I.$(\frac{137}{179})^{7}$
\testStop
\kluczStart
A
\kluczStop



\zadStart{Zadanie z Wikieł Z 1.1 d) moja wersja nr 576}

Obliczyć wartość wyrażenia $(\frac{179}{139})^{7} \cdot (\frac{139}{179})^{7} \cdot \pi^{0}$.
\zadStop
\rozwStart{Patryk Wirkus}{Martyna Czarnobaj}
$$(\frac{179}{139})^{7} \cdot (\frac{139}{179})^{7} \cdot \pi^{0} = (\frac{179}{139} \cdot \frac{139}{179})^{7} \cdot 1 = 1^{7} \cdot 1 = 1$$
\rozwStop
\odpStart
$1$
\odpStop
\testStart
A.$1$ B.$\pi$ C.$0$ D.$\frac{179}{139}$ E.$\frac{139}{179}$
F.$-\frac{179}{139}$ G.$-1$
H.$(\frac{179}{139})^{7}$
I.$(\frac{139}{179})^{7}$
\testStop
\kluczStart
A
\kluczStop



\zadStart{Zadanie z Wikieł Z 1.1 d) moja wersja nr 577}

Obliczyć wartość wyrażenia $(\frac{251}{103})^{7} \cdot (\frac{103}{251})^{7} \cdot \pi^{0}$.
\zadStop
\rozwStart{Patryk Wirkus}{Martyna Czarnobaj}
$$(\frac{251}{103})^{7} \cdot (\frac{103}{251})^{7} \cdot \pi^{0} = (\frac{251}{103} \cdot \frac{103}{251})^{7} \cdot 1 = 1^{7} \cdot 1 = 1$$
\rozwStop
\odpStart
$1$
\odpStop
\testStart
A.$1$ B.$\pi$ C.$0$ D.$\frac{251}{103}$ E.$\frac{103}{251}$
F.$-\frac{251}{103}$ G.$-1$
H.$(\frac{251}{103})^{7}$
I.$(\frac{103}{251})^{7}$
\testStop
\kluczStart
A
\kluczStop



\zadStart{Zadanie z Wikieł Z 1.1 d) moja wersja nr 578}

Obliczyć wartość wyrażenia $(\frac{251}{107})^{7} \cdot (\frac{107}{251})^{7} \cdot \pi^{0}$.
\zadStop
\rozwStart{Patryk Wirkus}{Martyna Czarnobaj}
$$(\frac{251}{107})^{7} \cdot (\frac{107}{251})^{7} \cdot \pi^{0} = (\frac{251}{107} \cdot \frac{107}{251})^{7} \cdot 1 = 1^{7} \cdot 1 = 1$$
\rozwStop
\odpStart
$1$
\odpStop
\testStart
A.$1$ B.$\pi$ C.$0$ D.$\frac{251}{107}$ E.$\frac{107}{251}$
F.$-\frac{251}{107}$ G.$-1$
H.$(\frac{251}{107})^{7}$
I.$(\frac{107}{251})^{7}$
\testStop
\kluczStart
A
\kluczStop



\zadStart{Zadanie z Wikieł Z 1.1 d) moja wersja nr 579}

Obliczyć wartość wyrażenia $(\frac{251}{109})^{7} \cdot (\frac{109}{251})^{7} \cdot \pi^{0}$.
\zadStop
\rozwStart{Patryk Wirkus}{Martyna Czarnobaj}
$$(\frac{251}{109})^{7} \cdot (\frac{109}{251})^{7} \cdot \pi^{0} = (\frac{251}{109} \cdot \frac{109}{251})^{7} \cdot 1 = 1^{7} \cdot 1 = 1$$
\rozwStop
\odpStart
$1$
\odpStop
\testStart
A.$1$ B.$\pi$ C.$0$ D.$\frac{251}{109}$ E.$\frac{109}{251}$
F.$-\frac{251}{109}$ G.$-1$
H.$(\frac{251}{109})^{7}$
I.$(\frac{109}{251})^{7}$
\testStop
\kluczStart
A
\kluczStop



\zadStart{Zadanie z Wikieł Z 1.1 d) moja wersja nr 580}

Obliczyć wartość wyrażenia $(\frac{251}{113})^{7} \cdot (\frac{113}{251})^{7} \cdot \pi^{0}$.
\zadStop
\rozwStart{Patryk Wirkus}{Martyna Czarnobaj}
$$(\frac{251}{113})^{7} \cdot (\frac{113}{251})^{7} \cdot \pi^{0} = (\frac{251}{113} \cdot \frac{113}{251})^{7} \cdot 1 = 1^{7} \cdot 1 = 1$$
\rozwStop
\odpStart
$1$
\odpStop
\testStart
A.$1$ B.$\pi$ C.$0$ D.$\frac{251}{113}$ E.$\frac{113}{251}$
F.$-\frac{251}{113}$ G.$-1$
H.$(\frac{251}{113})^{7}$
I.$(\frac{113}{251})^{7}$
\testStop
\kluczStart
A
\kluczStop



\zadStart{Zadanie z Wikieł Z 1.1 d) moja wersja nr 581}

Obliczyć wartość wyrażenia $(\frac{251}{127})^{7} \cdot (\frac{127}{251})^{7} \cdot \pi^{0}$.
\zadStop
\rozwStart{Patryk Wirkus}{Martyna Czarnobaj}
$$(\frac{251}{127})^{7} \cdot (\frac{127}{251})^{7} \cdot \pi^{0} = (\frac{251}{127} \cdot \frac{127}{251})^{7} \cdot 1 = 1^{7} \cdot 1 = 1$$
\rozwStop
\odpStart
$1$
\odpStop
\testStart
A.$1$ B.$\pi$ C.$0$ D.$\frac{251}{127}$ E.$\frac{127}{251}$
F.$-\frac{251}{127}$ G.$-1$
H.$(\frac{251}{127})^{7}$
I.$(\frac{127}{251})^{7}$
\testStop
\kluczStart
A
\kluczStop



\zadStart{Zadanie z Wikieł Z 1.1 d) moja wersja nr 582}

Obliczyć wartość wyrażenia $(\frac{251}{131})^{7} \cdot (\frac{131}{251})^{7} \cdot \pi^{0}$.
\zadStop
\rozwStart{Patryk Wirkus}{Martyna Czarnobaj}
$$(\frac{251}{131})^{7} \cdot (\frac{131}{251})^{7} \cdot \pi^{0} = (\frac{251}{131} \cdot \frac{131}{251})^{7} \cdot 1 = 1^{7} \cdot 1 = 1$$
\rozwStop
\odpStart
$1$
\odpStop
\testStart
A.$1$ B.$\pi$ C.$0$ D.$\frac{251}{131}$ E.$\frac{131}{251}$
F.$-\frac{251}{131}$ G.$-1$
H.$(\frac{251}{131})^{7}$
I.$(\frac{131}{251})^{7}$
\testStop
\kluczStart
A
\kluczStop



\zadStart{Zadanie z Wikieł Z 1.1 d) moja wersja nr 583}

Obliczyć wartość wyrażenia $(\frac{251}{137})^{7} \cdot (\frac{137}{251})^{7} \cdot \pi^{0}$.
\zadStop
\rozwStart{Patryk Wirkus}{Martyna Czarnobaj}
$$(\frac{251}{137})^{7} \cdot (\frac{137}{251})^{7} \cdot \pi^{0} = (\frac{251}{137} \cdot \frac{137}{251})^{7} \cdot 1 = 1^{7} \cdot 1 = 1$$
\rozwStop
\odpStart
$1$
\odpStop
\testStart
A.$1$ B.$\pi$ C.$0$ D.$\frac{251}{137}$ E.$\frac{137}{251}$
F.$-\frac{251}{137}$ G.$-1$
H.$(\frac{251}{137})^{7}$
I.$(\frac{137}{251})^{7}$
\testStop
\kluczStart
A
\kluczStop



\zadStart{Zadanie z Wikieł Z 1.1 d) moja wersja nr 584}

Obliczyć wartość wyrażenia $(\frac{251}{139})^{7} \cdot (\frac{139}{251})^{7} \cdot \pi^{0}$.
\zadStop
\rozwStart{Patryk Wirkus}{Martyna Czarnobaj}
$$(\frac{251}{139})^{7} \cdot (\frac{139}{251})^{7} \cdot \pi^{0} = (\frac{251}{139} \cdot \frac{139}{251})^{7} \cdot 1 = 1^{7} \cdot 1 = 1$$
\rozwStop
\odpStart
$1$
\odpStop
\testStart
A.$1$ B.$\pi$ C.$0$ D.$\frac{251}{139}$ E.$\frac{139}{251}$
F.$-\frac{251}{139}$ G.$-1$
H.$(\frac{251}{139})^{7}$
I.$(\frac{139}{251})^{7}$
\testStop
\kluczStart
A
\kluczStop



\zadStart{Zadanie z Wikieł Z 1.1 d) moja wersja nr 585}

Obliczyć wartość wyrażenia $(\frac{257}{103})^{7} \cdot (\frac{103}{257})^{7} \cdot \pi^{0}$.
\zadStop
\rozwStart{Patryk Wirkus}{Martyna Czarnobaj}
$$(\frac{257}{103})^{7} \cdot (\frac{103}{257})^{7} \cdot \pi^{0} = (\frac{257}{103} \cdot \frac{103}{257})^{7} \cdot 1 = 1^{7} \cdot 1 = 1$$
\rozwStop
\odpStart
$1$
\odpStop
\testStart
A.$1$ B.$\pi$ C.$0$ D.$\frac{257}{103}$ E.$\frac{103}{257}$
F.$-\frac{257}{103}$ G.$-1$
H.$(\frac{257}{103})^{7}$
I.$(\frac{103}{257})^{7}$
\testStop
\kluczStart
A
\kluczStop



\zadStart{Zadanie z Wikieł Z 1.1 d) moja wersja nr 586}

Obliczyć wartość wyrażenia $(\frac{257}{107})^{7} \cdot (\frac{107}{257})^{7} \cdot \pi^{0}$.
\zadStop
\rozwStart{Patryk Wirkus}{Martyna Czarnobaj}
$$(\frac{257}{107})^{7} \cdot (\frac{107}{257})^{7} \cdot \pi^{0} = (\frac{257}{107} \cdot \frac{107}{257})^{7} \cdot 1 = 1^{7} \cdot 1 = 1$$
\rozwStop
\odpStart
$1$
\odpStop
\testStart
A.$1$ B.$\pi$ C.$0$ D.$\frac{257}{107}$ E.$\frac{107}{257}$
F.$-\frac{257}{107}$ G.$-1$
H.$(\frac{257}{107})^{7}$
I.$(\frac{107}{257})^{7}$
\testStop
\kluczStart
A
\kluczStop



\zadStart{Zadanie z Wikieł Z 1.1 d) moja wersja nr 587}

Obliczyć wartość wyrażenia $(\frac{257}{109})^{7} \cdot (\frac{109}{257})^{7} \cdot \pi^{0}$.
\zadStop
\rozwStart{Patryk Wirkus}{Martyna Czarnobaj}
$$(\frac{257}{109})^{7} \cdot (\frac{109}{257})^{7} \cdot \pi^{0} = (\frac{257}{109} \cdot \frac{109}{257})^{7} \cdot 1 = 1^{7} \cdot 1 = 1$$
\rozwStop
\odpStart
$1$
\odpStop
\testStart
A.$1$ B.$\pi$ C.$0$ D.$\frac{257}{109}$ E.$\frac{109}{257}$
F.$-\frac{257}{109}$ G.$-1$
H.$(\frac{257}{109})^{7}$
I.$(\frac{109}{257})^{7}$
\testStop
\kluczStart
A
\kluczStop



\zadStart{Zadanie z Wikieł Z 1.1 d) moja wersja nr 588}

Obliczyć wartość wyrażenia $(\frac{257}{113})^{7} \cdot (\frac{113}{257})^{7} \cdot \pi^{0}$.
\zadStop
\rozwStart{Patryk Wirkus}{Martyna Czarnobaj}
$$(\frac{257}{113})^{7} \cdot (\frac{113}{257})^{7} \cdot \pi^{0} = (\frac{257}{113} \cdot \frac{113}{257})^{7} \cdot 1 = 1^{7} \cdot 1 = 1$$
\rozwStop
\odpStart
$1$
\odpStop
\testStart
A.$1$ B.$\pi$ C.$0$ D.$\frac{257}{113}$ E.$\frac{113}{257}$
F.$-\frac{257}{113}$ G.$-1$
H.$(\frac{257}{113})^{7}$
I.$(\frac{113}{257})^{7}$
\testStop
\kluczStart
A
\kluczStop



\zadStart{Zadanie z Wikieł Z 1.1 d) moja wersja nr 589}

Obliczyć wartość wyrażenia $(\frac{257}{127})^{7} \cdot (\frac{127}{257})^{7} \cdot \pi^{0}$.
\zadStop
\rozwStart{Patryk Wirkus}{Martyna Czarnobaj}
$$(\frac{257}{127})^{7} \cdot (\frac{127}{257})^{7} \cdot \pi^{0} = (\frac{257}{127} \cdot \frac{127}{257})^{7} \cdot 1 = 1^{7} \cdot 1 = 1$$
\rozwStop
\odpStart
$1$
\odpStop
\testStart
A.$1$ B.$\pi$ C.$0$ D.$\frac{257}{127}$ E.$\frac{127}{257}$
F.$-\frac{257}{127}$ G.$-1$
H.$(\frac{257}{127})^{7}$
I.$(\frac{127}{257})^{7}$
\testStop
\kluczStart
A
\kluczStop



\zadStart{Zadanie z Wikieł Z 1.1 d) moja wersja nr 590}

Obliczyć wartość wyrażenia $(\frac{257}{131})^{7} \cdot (\frac{131}{257})^{7} \cdot \pi^{0}$.
\zadStop
\rozwStart{Patryk Wirkus}{Martyna Czarnobaj}
$$(\frac{257}{131})^{7} \cdot (\frac{131}{257})^{7} \cdot \pi^{0} = (\frac{257}{131} \cdot \frac{131}{257})^{7} \cdot 1 = 1^{7} \cdot 1 = 1$$
\rozwStop
\odpStart
$1$
\odpStop
\testStart
A.$1$ B.$\pi$ C.$0$ D.$\frac{257}{131}$ E.$\frac{131}{257}$
F.$-\frac{257}{131}$ G.$-1$
H.$(\frac{257}{131})^{7}$
I.$(\frac{131}{257})^{7}$
\testStop
\kluczStart
A
\kluczStop



\zadStart{Zadanie z Wikieł Z 1.1 d) moja wersja nr 591}

Obliczyć wartość wyrażenia $(\frac{257}{137})^{7} \cdot (\frac{137}{257})^{7} \cdot \pi^{0}$.
\zadStop
\rozwStart{Patryk Wirkus}{Martyna Czarnobaj}
$$(\frac{257}{137})^{7} \cdot (\frac{137}{257})^{7} \cdot \pi^{0} = (\frac{257}{137} \cdot \frac{137}{257})^{7} \cdot 1 = 1^{7} \cdot 1 = 1$$
\rozwStop
\odpStart
$1$
\odpStop
\testStart
A.$1$ B.$\pi$ C.$0$ D.$\frac{257}{137}$ E.$\frac{137}{257}$
F.$-\frac{257}{137}$ G.$-1$
H.$(\frac{257}{137})^{7}$
I.$(\frac{137}{257})^{7}$
\testStop
\kluczStart
A
\kluczStop



\zadStart{Zadanie z Wikieł Z 1.1 d) moja wersja nr 592}

Obliczyć wartość wyrażenia $(\frac{257}{139})^{7} \cdot (\frac{139}{257})^{7} \cdot \pi^{0}$.
\zadStop
\rozwStart{Patryk Wirkus}{Martyna Czarnobaj}
$$(\frac{257}{139})^{7} \cdot (\frac{139}{257})^{7} \cdot \pi^{0} = (\frac{257}{139} \cdot \frac{139}{257})^{7} \cdot 1 = 1^{7} \cdot 1 = 1$$
\rozwStop
\odpStart
$1$
\odpStop
\testStart
A.$1$ B.$\pi$ C.$0$ D.$\frac{257}{139}$ E.$\frac{139}{257}$
F.$-\frac{257}{139}$ G.$-1$
H.$(\frac{257}{139})^{7}$
I.$(\frac{139}{257})^{7}$
\testStop
\kluczStart
A
\kluczStop



\zadStart{Zadanie z Wikieł Z 1.1 d) moja wersja nr 593}

Obliczyć wartość wyrażenia $(\frac{263}{103})^{7} \cdot (\frac{103}{263})^{7} \cdot \pi^{0}$.
\zadStop
\rozwStart{Patryk Wirkus}{Martyna Czarnobaj}
$$(\frac{263}{103})^{7} \cdot (\frac{103}{263})^{7} \cdot \pi^{0} = (\frac{263}{103} \cdot \frac{103}{263})^{7} \cdot 1 = 1^{7} \cdot 1 = 1$$
\rozwStop
\odpStart
$1$
\odpStop
\testStart
A.$1$ B.$\pi$ C.$0$ D.$\frac{263}{103}$ E.$\frac{103}{263}$
F.$-\frac{263}{103}$ G.$-1$
H.$(\frac{263}{103})^{7}$
I.$(\frac{103}{263})^{7}$
\testStop
\kluczStart
A
\kluczStop



\zadStart{Zadanie z Wikieł Z 1.1 d) moja wersja nr 594}

Obliczyć wartość wyrażenia $(\frac{263}{107})^{7} \cdot (\frac{107}{263})^{7} \cdot \pi^{0}$.
\zadStop
\rozwStart{Patryk Wirkus}{Martyna Czarnobaj}
$$(\frac{263}{107})^{7} \cdot (\frac{107}{263})^{7} \cdot \pi^{0} = (\frac{263}{107} \cdot \frac{107}{263})^{7} \cdot 1 = 1^{7} \cdot 1 = 1$$
\rozwStop
\odpStart
$1$
\odpStop
\testStart
A.$1$ B.$\pi$ C.$0$ D.$\frac{263}{107}$ E.$\frac{107}{263}$
F.$-\frac{263}{107}$ G.$-1$
H.$(\frac{263}{107})^{7}$
I.$(\frac{107}{263})^{7}$
\testStop
\kluczStart
A
\kluczStop



\zadStart{Zadanie z Wikieł Z 1.1 d) moja wersja nr 595}

Obliczyć wartość wyrażenia $(\frac{263}{109})^{7} \cdot (\frac{109}{263})^{7} \cdot \pi^{0}$.
\zadStop
\rozwStart{Patryk Wirkus}{Martyna Czarnobaj}
$$(\frac{263}{109})^{7} \cdot (\frac{109}{263})^{7} \cdot \pi^{0} = (\frac{263}{109} \cdot \frac{109}{263})^{7} \cdot 1 = 1^{7} \cdot 1 = 1$$
\rozwStop
\odpStart
$1$
\odpStop
\testStart
A.$1$ B.$\pi$ C.$0$ D.$\frac{263}{109}$ E.$\frac{109}{263}$
F.$-\frac{263}{109}$ G.$-1$
H.$(\frac{263}{109})^{7}$
I.$(\frac{109}{263})^{7}$
\testStop
\kluczStart
A
\kluczStop



\zadStart{Zadanie z Wikieł Z 1.1 d) moja wersja nr 596}

Obliczyć wartość wyrażenia $(\frac{263}{113})^{7} \cdot (\frac{113}{263})^{7} \cdot \pi^{0}$.
\zadStop
\rozwStart{Patryk Wirkus}{Martyna Czarnobaj}
$$(\frac{263}{113})^{7} \cdot (\frac{113}{263})^{7} \cdot \pi^{0} = (\frac{263}{113} \cdot \frac{113}{263})^{7} \cdot 1 = 1^{7} \cdot 1 = 1$$
\rozwStop
\odpStart
$1$
\odpStop
\testStart
A.$1$ B.$\pi$ C.$0$ D.$\frac{263}{113}$ E.$\frac{113}{263}$
F.$-\frac{263}{113}$ G.$-1$
H.$(\frac{263}{113})^{7}$
I.$(\frac{113}{263})^{7}$
\testStop
\kluczStart
A
\kluczStop



\zadStart{Zadanie z Wikieł Z 1.1 d) moja wersja nr 597}

Obliczyć wartość wyrażenia $(\frac{263}{127})^{7} \cdot (\frac{127}{263})^{7} \cdot \pi^{0}$.
\zadStop
\rozwStart{Patryk Wirkus}{Martyna Czarnobaj}
$$(\frac{263}{127})^{7} \cdot (\frac{127}{263})^{7} \cdot \pi^{0} = (\frac{263}{127} \cdot \frac{127}{263})^{7} \cdot 1 = 1^{7} \cdot 1 = 1$$
\rozwStop
\odpStart
$1$
\odpStop
\testStart
A.$1$ B.$\pi$ C.$0$ D.$\frac{263}{127}$ E.$\frac{127}{263}$
F.$-\frac{263}{127}$ G.$-1$
H.$(\frac{263}{127})^{7}$
I.$(\frac{127}{263})^{7}$
\testStop
\kluczStart
A
\kluczStop



\zadStart{Zadanie z Wikieł Z 1.1 d) moja wersja nr 598}

Obliczyć wartość wyrażenia $(\frac{263}{131})^{7} \cdot (\frac{131}{263})^{7} \cdot \pi^{0}$.
\zadStop
\rozwStart{Patryk Wirkus}{Martyna Czarnobaj}
$$(\frac{263}{131})^{7} \cdot (\frac{131}{263})^{7} \cdot \pi^{0} = (\frac{263}{131} \cdot \frac{131}{263})^{7} \cdot 1 = 1^{7} \cdot 1 = 1$$
\rozwStop
\odpStart
$1$
\odpStop
\testStart
A.$1$ B.$\pi$ C.$0$ D.$\frac{263}{131}$ E.$\frac{131}{263}$
F.$-\frac{263}{131}$ G.$-1$
H.$(\frac{263}{131})^{7}$
I.$(\frac{131}{263})^{7}$
\testStop
\kluczStart
A
\kluczStop



\zadStart{Zadanie z Wikieł Z 1.1 d) moja wersja nr 599}

Obliczyć wartość wyrażenia $(\frac{263}{137})^{7} \cdot (\frac{137}{263})^{7} \cdot \pi^{0}$.
\zadStop
\rozwStart{Patryk Wirkus}{Martyna Czarnobaj}
$$(\frac{263}{137})^{7} \cdot (\frac{137}{263})^{7} \cdot \pi^{0} = (\frac{263}{137} \cdot \frac{137}{263})^{7} \cdot 1 = 1^{7} \cdot 1 = 1$$
\rozwStop
\odpStart
$1$
\odpStop
\testStart
A.$1$ B.$\pi$ C.$0$ D.$\frac{263}{137}$ E.$\frac{137}{263}$
F.$-\frac{263}{137}$ G.$-1$
H.$(\frac{263}{137})^{7}$
I.$(\frac{137}{263})^{7}$
\testStop
\kluczStart
A
\kluczStop



\zadStart{Zadanie z Wikieł Z 1.1 d) moja wersja nr 600}

Obliczyć wartość wyrażenia $(\frac{263}{139})^{7} \cdot (\frac{139}{263})^{7} \cdot \pi^{0}$.
\zadStop
\rozwStart{Patryk Wirkus}{Martyna Czarnobaj}
$$(\frac{263}{139})^{7} \cdot (\frac{139}{263})^{7} \cdot \pi^{0} = (\frac{263}{139} \cdot \frac{139}{263})^{7} \cdot 1 = 1^{7} \cdot 1 = 1$$
\rozwStop
\odpStart
$1$
\odpStop
\testStart
A.$1$ B.$\pi$ C.$0$ D.$\frac{263}{139}$ E.$\frac{139}{263}$
F.$-\frac{263}{139}$ G.$-1$
H.$(\frac{263}{139})^{7}$
I.$(\frac{139}{263})^{7}$
\testStop
\kluczStart
A
\kluczStop



\zadStart{Zadanie z Wikieł Z 1.1 d) moja wersja nr 601}

Obliczyć wartość wyrażenia $(\frac{269}{103})^{7} \cdot (\frac{103}{269})^{7} \cdot \pi^{0}$.
\zadStop
\rozwStart{Patryk Wirkus}{Martyna Czarnobaj}
$$(\frac{269}{103})^{7} \cdot (\frac{103}{269})^{7} \cdot \pi^{0} = (\frac{269}{103} \cdot \frac{103}{269})^{7} \cdot 1 = 1^{7} \cdot 1 = 1$$
\rozwStop
\odpStart
$1$
\odpStop
\testStart
A.$1$ B.$\pi$ C.$0$ D.$\frac{269}{103}$ E.$\frac{103}{269}$
F.$-\frac{269}{103}$ G.$-1$
H.$(\frac{269}{103})^{7}$
I.$(\frac{103}{269})^{7}$
\testStop
\kluczStart
A
\kluczStop



\zadStart{Zadanie z Wikieł Z 1.1 d) moja wersja nr 602}

Obliczyć wartość wyrażenia $(\frac{269}{107})^{7} \cdot (\frac{107}{269})^{7} \cdot \pi^{0}$.
\zadStop
\rozwStart{Patryk Wirkus}{Martyna Czarnobaj}
$$(\frac{269}{107})^{7} \cdot (\frac{107}{269})^{7} \cdot \pi^{0} = (\frac{269}{107} \cdot \frac{107}{269})^{7} \cdot 1 = 1^{7} \cdot 1 = 1$$
\rozwStop
\odpStart
$1$
\odpStop
\testStart
A.$1$ B.$\pi$ C.$0$ D.$\frac{269}{107}$ E.$\frac{107}{269}$
F.$-\frac{269}{107}$ G.$-1$
H.$(\frac{269}{107})^{7}$
I.$(\frac{107}{269})^{7}$
\testStop
\kluczStart
A
\kluczStop



\zadStart{Zadanie z Wikieł Z 1.1 d) moja wersja nr 603}

Obliczyć wartość wyrażenia $(\frac{269}{109})^{7} \cdot (\frac{109}{269})^{7} \cdot \pi^{0}$.
\zadStop
\rozwStart{Patryk Wirkus}{Martyna Czarnobaj}
$$(\frac{269}{109})^{7} \cdot (\frac{109}{269})^{7} \cdot \pi^{0} = (\frac{269}{109} \cdot \frac{109}{269})^{7} \cdot 1 = 1^{7} \cdot 1 = 1$$
\rozwStop
\odpStart
$1$
\odpStop
\testStart
A.$1$ B.$\pi$ C.$0$ D.$\frac{269}{109}$ E.$\frac{109}{269}$
F.$-\frac{269}{109}$ G.$-1$
H.$(\frac{269}{109})^{7}$
I.$(\frac{109}{269})^{7}$
\testStop
\kluczStart
A
\kluczStop



\zadStart{Zadanie z Wikieł Z 1.1 d) moja wersja nr 604}

Obliczyć wartość wyrażenia $(\frac{269}{113})^{7} \cdot (\frac{113}{269})^{7} \cdot \pi^{0}$.
\zadStop
\rozwStart{Patryk Wirkus}{Martyna Czarnobaj}
$$(\frac{269}{113})^{7} \cdot (\frac{113}{269})^{7} \cdot \pi^{0} = (\frac{269}{113} \cdot \frac{113}{269})^{7} \cdot 1 = 1^{7} \cdot 1 = 1$$
\rozwStop
\odpStart
$1$
\odpStop
\testStart
A.$1$ B.$\pi$ C.$0$ D.$\frac{269}{113}$ E.$\frac{113}{269}$
F.$-\frac{269}{113}$ G.$-1$
H.$(\frac{269}{113})^{7}$
I.$(\frac{113}{269})^{7}$
\testStop
\kluczStart
A
\kluczStop



\zadStart{Zadanie z Wikieł Z 1.1 d) moja wersja nr 605}

Obliczyć wartość wyrażenia $(\frac{269}{127})^{7} \cdot (\frac{127}{269})^{7} \cdot \pi^{0}$.
\zadStop
\rozwStart{Patryk Wirkus}{Martyna Czarnobaj}
$$(\frac{269}{127})^{7} \cdot (\frac{127}{269})^{7} \cdot \pi^{0} = (\frac{269}{127} \cdot \frac{127}{269})^{7} \cdot 1 = 1^{7} \cdot 1 = 1$$
\rozwStop
\odpStart
$1$
\odpStop
\testStart
A.$1$ B.$\pi$ C.$0$ D.$\frac{269}{127}$ E.$\frac{127}{269}$
F.$-\frac{269}{127}$ G.$-1$
H.$(\frac{269}{127})^{7}$
I.$(\frac{127}{269})^{7}$
\testStop
\kluczStart
A
\kluczStop



\zadStart{Zadanie z Wikieł Z 1.1 d) moja wersja nr 606}

Obliczyć wartość wyrażenia $(\frac{269}{131})^{7} \cdot (\frac{131}{269})^{7} \cdot \pi^{0}$.
\zadStop
\rozwStart{Patryk Wirkus}{Martyna Czarnobaj}
$$(\frac{269}{131})^{7} \cdot (\frac{131}{269})^{7} \cdot \pi^{0} = (\frac{269}{131} \cdot \frac{131}{269})^{7} \cdot 1 = 1^{7} \cdot 1 = 1$$
\rozwStop
\odpStart
$1$
\odpStop
\testStart
A.$1$ B.$\pi$ C.$0$ D.$\frac{269}{131}$ E.$\frac{131}{269}$
F.$-\frac{269}{131}$ G.$-1$
H.$(\frac{269}{131})^{7}$
I.$(\frac{131}{269})^{7}$
\testStop
\kluczStart
A
\kluczStop



\zadStart{Zadanie z Wikieł Z 1.1 d) moja wersja nr 607}

Obliczyć wartość wyrażenia $(\frac{269}{137})^{7} \cdot (\frac{137}{269})^{7} \cdot \pi^{0}$.
\zadStop
\rozwStart{Patryk Wirkus}{Martyna Czarnobaj}
$$(\frac{269}{137})^{7} \cdot (\frac{137}{269})^{7} \cdot \pi^{0} = (\frac{269}{137} \cdot \frac{137}{269})^{7} \cdot 1 = 1^{7} \cdot 1 = 1$$
\rozwStop
\odpStart
$1$
\odpStop
\testStart
A.$1$ B.$\pi$ C.$0$ D.$\frac{269}{137}$ E.$\frac{137}{269}$
F.$-\frac{269}{137}$ G.$-1$
H.$(\frac{269}{137})^{7}$
I.$(\frac{137}{269})^{7}$
\testStop
\kluczStart
A
\kluczStop



\zadStart{Zadanie z Wikieł Z 1.1 d) moja wersja nr 608}

Obliczyć wartość wyrażenia $(\frac{269}{139})^{7} \cdot (\frac{139}{269})^{7} \cdot \pi^{0}$.
\zadStop
\rozwStart{Patryk Wirkus}{Martyna Czarnobaj}
$$(\frac{269}{139})^{7} \cdot (\frac{139}{269})^{7} \cdot \pi^{0} = (\frac{269}{139} \cdot \frac{139}{269})^{7} \cdot 1 = 1^{7} \cdot 1 = 1$$
\rozwStop
\odpStart
$1$
\odpStop
\testStart
A.$1$ B.$\pi$ C.$0$ D.$\frac{269}{139}$ E.$\frac{139}{269}$
F.$-\frac{269}{139}$ G.$-1$
H.$(\frac{269}{139})^{7}$
I.$(\frac{139}{269})^{7}$
\testStop
\kluczStart
A
\kluczStop



\zadStart{Zadanie z Wikieł Z 1.1 d) moja wersja nr 609}

Obliczyć wartość wyrażenia $(\frac{271}{103})^{7} \cdot (\frac{103}{271})^{7} \cdot \pi^{0}$.
\zadStop
\rozwStart{Patryk Wirkus}{Martyna Czarnobaj}
$$(\frac{271}{103})^{7} \cdot (\frac{103}{271})^{7} \cdot \pi^{0} = (\frac{271}{103} \cdot \frac{103}{271})^{7} \cdot 1 = 1^{7} \cdot 1 = 1$$
\rozwStop
\odpStart
$1$
\odpStop
\testStart
A.$1$ B.$\pi$ C.$0$ D.$\frac{271}{103}$ E.$\frac{103}{271}$
F.$-\frac{271}{103}$ G.$-1$
H.$(\frac{271}{103})^{7}$
I.$(\frac{103}{271})^{7}$
\testStop
\kluczStart
A
\kluczStop



\zadStart{Zadanie z Wikieł Z 1.1 d) moja wersja nr 610}

Obliczyć wartość wyrażenia $(\frac{271}{107})^{7} \cdot (\frac{107}{271})^{7} \cdot \pi^{0}$.
\zadStop
\rozwStart{Patryk Wirkus}{Martyna Czarnobaj}
$$(\frac{271}{107})^{7} \cdot (\frac{107}{271})^{7} \cdot \pi^{0} = (\frac{271}{107} \cdot \frac{107}{271})^{7} \cdot 1 = 1^{7} \cdot 1 = 1$$
\rozwStop
\odpStart
$1$
\odpStop
\testStart
A.$1$ B.$\pi$ C.$0$ D.$\frac{271}{107}$ E.$\frac{107}{271}$
F.$-\frac{271}{107}$ G.$-1$
H.$(\frac{271}{107})^{7}$
I.$(\frac{107}{271})^{7}$
\testStop
\kluczStart
A
\kluczStop



\zadStart{Zadanie z Wikieł Z 1.1 d) moja wersja nr 611}

Obliczyć wartość wyrażenia $(\frac{271}{109})^{7} \cdot (\frac{109}{271})^{7} \cdot \pi^{0}$.
\zadStop
\rozwStart{Patryk Wirkus}{Martyna Czarnobaj}
$$(\frac{271}{109})^{7} \cdot (\frac{109}{271})^{7} \cdot \pi^{0} = (\frac{271}{109} \cdot \frac{109}{271})^{7} \cdot 1 = 1^{7} \cdot 1 = 1$$
\rozwStop
\odpStart
$1$
\odpStop
\testStart
A.$1$ B.$\pi$ C.$0$ D.$\frac{271}{109}$ E.$\frac{109}{271}$
F.$-\frac{271}{109}$ G.$-1$
H.$(\frac{271}{109})^{7}$
I.$(\frac{109}{271})^{7}$
\testStop
\kluczStart
A
\kluczStop



\zadStart{Zadanie z Wikieł Z 1.1 d) moja wersja nr 612}

Obliczyć wartość wyrażenia $(\frac{271}{113})^{7} \cdot (\frac{113}{271})^{7} \cdot \pi^{0}$.
\zadStop
\rozwStart{Patryk Wirkus}{Martyna Czarnobaj}
$$(\frac{271}{113})^{7} \cdot (\frac{113}{271})^{7} \cdot \pi^{0} = (\frac{271}{113} \cdot \frac{113}{271})^{7} \cdot 1 = 1^{7} \cdot 1 = 1$$
\rozwStop
\odpStart
$1$
\odpStop
\testStart
A.$1$ B.$\pi$ C.$0$ D.$\frac{271}{113}$ E.$\frac{113}{271}$
F.$-\frac{271}{113}$ G.$-1$
H.$(\frac{271}{113})^{7}$
I.$(\frac{113}{271})^{7}$
\testStop
\kluczStart
A
\kluczStop



\zadStart{Zadanie z Wikieł Z 1.1 d) moja wersja nr 613}

Obliczyć wartość wyrażenia $(\frac{271}{127})^{7} \cdot (\frac{127}{271})^{7} \cdot \pi^{0}$.
\zadStop
\rozwStart{Patryk Wirkus}{Martyna Czarnobaj}
$$(\frac{271}{127})^{7} \cdot (\frac{127}{271})^{7} \cdot \pi^{0} = (\frac{271}{127} \cdot \frac{127}{271})^{7} \cdot 1 = 1^{7} \cdot 1 = 1$$
\rozwStop
\odpStart
$1$
\odpStop
\testStart
A.$1$ B.$\pi$ C.$0$ D.$\frac{271}{127}$ E.$\frac{127}{271}$
F.$-\frac{271}{127}$ G.$-1$
H.$(\frac{271}{127})^{7}$
I.$(\frac{127}{271})^{7}$
\testStop
\kluczStart
A
\kluczStop



\zadStart{Zadanie z Wikieł Z 1.1 d) moja wersja nr 614}

Obliczyć wartość wyrażenia $(\frac{271}{131})^{7} \cdot (\frac{131}{271})^{7} \cdot \pi^{0}$.
\zadStop
\rozwStart{Patryk Wirkus}{Martyna Czarnobaj}
$$(\frac{271}{131})^{7} \cdot (\frac{131}{271})^{7} \cdot \pi^{0} = (\frac{271}{131} \cdot \frac{131}{271})^{7} \cdot 1 = 1^{7} \cdot 1 = 1$$
\rozwStop
\odpStart
$1$
\odpStop
\testStart
A.$1$ B.$\pi$ C.$0$ D.$\frac{271}{131}$ E.$\frac{131}{271}$
F.$-\frac{271}{131}$ G.$-1$
H.$(\frac{271}{131})^{7}$
I.$(\frac{131}{271})^{7}$
\testStop
\kluczStart
A
\kluczStop



\zadStart{Zadanie z Wikieł Z 1.1 d) moja wersja nr 615}

Obliczyć wartość wyrażenia $(\frac{271}{137})^{7} \cdot (\frac{137}{271})^{7} \cdot \pi^{0}$.
\zadStop
\rozwStart{Patryk Wirkus}{Martyna Czarnobaj}
$$(\frac{271}{137})^{7} \cdot (\frac{137}{271})^{7} \cdot \pi^{0} = (\frac{271}{137} \cdot \frac{137}{271})^{7} \cdot 1 = 1^{7} \cdot 1 = 1$$
\rozwStop
\odpStart
$1$
\odpStop
\testStart
A.$1$ B.$\pi$ C.$0$ D.$\frac{271}{137}$ E.$\frac{137}{271}$
F.$-\frac{271}{137}$ G.$-1$
H.$(\frac{271}{137})^{7}$
I.$(\frac{137}{271})^{7}$
\testStop
\kluczStart
A
\kluczStop



\zadStart{Zadanie z Wikieł Z 1.1 d) moja wersja nr 616}

Obliczyć wartość wyrażenia $(\frac{271}{139})^{7} \cdot (\frac{139}{271})^{7} \cdot \pi^{0}$.
\zadStop
\rozwStart{Patryk Wirkus}{Martyna Czarnobaj}
$$(\frac{271}{139})^{7} \cdot (\frac{139}{271})^{7} \cdot \pi^{0} = (\frac{271}{139} \cdot \frac{139}{271})^{7} \cdot 1 = 1^{7} \cdot 1 = 1$$
\rozwStop
\odpStart
$1$
\odpStop
\testStart
A.$1$ B.$\pi$ C.$0$ D.$\frac{271}{139}$ E.$\frac{139}{271}$
F.$-\frac{271}{139}$ G.$-1$
H.$(\frac{271}{139})^{7}$
I.$(\frac{139}{271})^{7}$
\testStop
\kluczStart
A
\kluczStop



\zadStart{Zadanie z Wikieł Z 1.1 d) moja wersja nr 617}

Obliczyć wartość wyrażenia $(\frac{277}{103})^{7} \cdot (\frac{103}{277})^{7} \cdot \pi^{0}$.
\zadStop
\rozwStart{Patryk Wirkus}{Martyna Czarnobaj}
$$(\frac{277}{103})^{7} \cdot (\frac{103}{277})^{7} \cdot \pi^{0} = (\frac{277}{103} \cdot \frac{103}{277})^{7} \cdot 1 = 1^{7} \cdot 1 = 1$$
\rozwStop
\odpStart
$1$
\odpStop
\testStart
A.$1$ B.$\pi$ C.$0$ D.$\frac{277}{103}$ E.$\frac{103}{277}$
F.$-\frac{277}{103}$ G.$-1$
H.$(\frac{277}{103})^{7}$
I.$(\frac{103}{277})^{7}$
\testStop
\kluczStart
A
\kluczStop



\zadStart{Zadanie z Wikieł Z 1.1 d) moja wersja nr 618}

Obliczyć wartość wyrażenia $(\frac{277}{107})^{7} \cdot (\frac{107}{277})^{7} \cdot \pi^{0}$.
\zadStop
\rozwStart{Patryk Wirkus}{Martyna Czarnobaj}
$$(\frac{277}{107})^{7} \cdot (\frac{107}{277})^{7} \cdot \pi^{0} = (\frac{277}{107} \cdot \frac{107}{277})^{7} \cdot 1 = 1^{7} \cdot 1 = 1$$
\rozwStop
\odpStart
$1$
\odpStop
\testStart
A.$1$ B.$\pi$ C.$0$ D.$\frac{277}{107}$ E.$\frac{107}{277}$
F.$-\frac{277}{107}$ G.$-1$
H.$(\frac{277}{107})^{7}$
I.$(\frac{107}{277})^{7}$
\testStop
\kluczStart
A
\kluczStop



\zadStart{Zadanie z Wikieł Z 1.1 d) moja wersja nr 619}

Obliczyć wartość wyrażenia $(\frac{277}{109})^{7} \cdot (\frac{109}{277})^{7} \cdot \pi^{0}$.
\zadStop
\rozwStart{Patryk Wirkus}{Martyna Czarnobaj}
$$(\frac{277}{109})^{7} \cdot (\frac{109}{277})^{7} \cdot \pi^{0} = (\frac{277}{109} \cdot \frac{109}{277})^{7} \cdot 1 = 1^{7} \cdot 1 = 1$$
\rozwStop
\odpStart
$1$
\odpStop
\testStart
A.$1$ B.$\pi$ C.$0$ D.$\frac{277}{109}$ E.$\frac{109}{277}$
F.$-\frac{277}{109}$ G.$-1$
H.$(\frac{277}{109})^{7}$
I.$(\frac{109}{277})^{7}$
\testStop
\kluczStart
A
\kluczStop



\zadStart{Zadanie z Wikieł Z 1.1 d) moja wersja nr 620}

Obliczyć wartość wyrażenia $(\frac{277}{113})^{7} \cdot (\frac{113}{277})^{7} \cdot \pi^{0}$.
\zadStop
\rozwStart{Patryk Wirkus}{Martyna Czarnobaj}
$$(\frac{277}{113})^{7} \cdot (\frac{113}{277})^{7} \cdot \pi^{0} = (\frac{277}{113} \cdot \frac{113}{277})^{7} \cdot 1 = 1^{7} \cdot 1 = 1$$
\rozwStop
\odpStart
$1$
\odpStop
\testStart
A.$1$ B.$\pi$ C.$0$ D.$\frac{277}{113}$ E.$\frac{113}{277}$
F.$-\frac{277}{113}$ G.$-1$
H.$(\frac{277}{113})^{7}$
I.$(\frac{113}{277})^{7}$
\testStop
\kluczStart
A
\kluczStop



\zadStart{Zadanie z Wikieł Z 1.1 d) moja wersja nr 621}

Obliczyć wartość wyrażenia $(\frac{277}{127})^{7} \cdot (\frac{127}{277})^{7} \cdot \pi^{0}$.
\zadStop
\rozwStart{Patryk Wirkus}{Martyna Czarnobaj}
$$(\frac{277}{127})^{7} \cdot (\frac{127}{277})^{7} \cdot \pi^{0} = (\frac{277}{127} \cdot \frac{127}{277})^{7} \cdot 1 = 1^{7} \cdot 1 = 1$$
\rozwStop
\odpStart
$1$
\odpStop
\testStart
A.$1$ B.$\pi$ C.$0$ D.$\frac{277}{127}$ E.$\frac{127}{277}$
F.$-\frac{277}{127}$ G.$-1$
H.$(\frac{277}{127})^{7}$
I.$(\frac{127}{277})^{7}$
\testStop
\kluczStart
A
\kluczStop



\zadStart{Zadanie z Wikieł Z 1.1 d) moja wersja nr 622}

Obliczyć wartość wyrażenia $(\frac{277}{131})^{7} \cdot (\frac{131}{277})^{7} \cdot \pi^{0}$.
\zadStop
\rozwStart{Patryk Wirkus}{Martyna Czarnobaj}
$$(\frac{277}{131})^{7} \cdot (\frac{131}{277})^{7} \cdot \pi^{0} = (\frac{277}{131} \cdot \frac{131}{277})^{7} \cdot 1 = 1^{7} \cdot 1 = 1$$
\rozwStop
\odpStart
$1$
\odpStop
\testStart
A.$1$ B.$\pi$ C.$0$ D.$\frac{277}{131}$ E.$\frac{131}{277}$
F.$-\frac{277}{131}$ G.$-1$
H.$(\frac{277}{131})^{7}$
I.$(\frac{131}{277})^{7}$
\testStop
\kluczStart
A
\kluczStop



\zadStart{Zadanie z Wikieł Z 1.1 d) moja wersja nr 623}

Obliczyć wartość wyrażenia $(\frac{277}{137})^{7} \cdot (\frac{137}{277})^{7} \cdot \pi^{0}$.
\zadStop
\rozwStart{Patryk Wirkus}{Martyna Czarnobaj}
$$(\frac{277}{137})^{7} \cdot (\frac{137}{277})^{7} \cdot \pi^{0} = (\frac{277}{137} \cdot \frac{137}{277})^{7} \cdot 1 = 1^{7} \cdot 1 = 1$$
\rozwStop
\odpStart
$1$
\odpStop
\testStart
A.$1$ B.$\pi$ C.$0$ D.$\frac{277}{137}$ E.$\frac{137}{277}$
F.$-\frac{277}{137}$ G.$-1$
H.$(\frac{277}{137})^{7}$
I.$(\frac{137}{277})^{7}$
\testStop
\kluczStart
A
\kluczStop



\zadStart{Zadanie z Wikieł Z 1.1 d) moja wersja nr 624}

Obliczyć wartość wyrażenia $(\frac{277}{139})^{7} \cdot (\frac{139}{277})^{7} \cdot \pi^{0}$.
\zadStop
\rozwStart{Patryk Wirkus}{Martyna Czarnobaj}
$$(\frac{277}{139})^{7} \cdot (\frac{139}{277})^{7} \cdot \pi^{0} = (\frac{277}{139} \cdot \frac{139}{277})^{7} \cdot 1 = 1^{7} \cdot 1 = 1$$
\rozwStop
\odpStart
$1$
\odpStop
\testStart
A.$1$ B.$\pi$ C.$0$ D.$\frac{277}{139}$ E.$\frac{139}{277}$
F.$-\frac{277}{139}$ G.$-1$
H.$(\frac{277}{139})^{7}$
I.$(\frac{139}{277})^{7}$
\testStop
\kluczStart
A
\kluczStop



\zadStart{Zadanie z Wikieł Z 1.1 d) moja wersja nr 625}

Obliczyć wartość wyrażenia $(\frac{149}{103})^{8} \cdot (\frac{103}{149})^{8} \cdot \pi^{0}$.
\zadStop
\rozwStart{Patryk Wirkus}{Martyna Czarnobaj}
$$(\frac{149}{103})^{8} \cdot (\frac{103}{149})^{8} \cdot \pi^{0} = (\frac{149}{103} \cdot \frac{103}{149})^{8} \cdot 1 = 1^{8} \cdot 1 = 1$$
\rozwStop
\odpStart
$1$
\odpStop
\testStart
A.$1$ B.$\pi$ C.$0$ D.$\frac{149}{103}$ E.$\frac{103}{149}$
F.$-\frac{149}{103}$ G.$-1$
H.$(\frac{149}{103})^{8}$
I.$(\frac{103}{149})^{8}$
\testStop
\kluczStart
A
\kluczStop



\zadStart{Zadanie z Wikieł Z 1.1 d) moja wersja nr 626}

Obliczyć wartość wyrażenia $(\frac{149}{107})^{8} \cdot (\frac{107}{149})^{8} \cdot \pi^{0}$.
\zadStop
\rozwStart{Patryk Wirkus}{Martyna Czarnobaj}
$$(\frac{149}{107})^{8} \cdot (\frac{107}{149})^{8} \cdot \pi^{0} = (\frac{149}{107} \cdot \frac{107}{149})^{8} \cdot 1 = 1^{8} \cdot 1 = 1$$
\rozwStop
\odpStart
$1$
\odpStop
\testStart
A.$1$ B.$\pi$ C.$0$ D.$\frac{149}{107}$ E.$\frac{107}{149}$
F.$-\frac{149}{107}$ G.$-1$
H.$(\frac{149}{107})^{8}$
I.$(\frac{107}{149})^{8}$
\testStop
\kluczStart
A
\kluczStop



\zadStart{Zadanie z Wikieł Z 1.1 d) moja wersja nr 627}

Obliczyć wartość wyrażenia $(\frac{149}{109})^{8} \cdot (\frac{109}{149})^{8} \cdot \pi^{0}$.
\zadStop
\rozwStart{Patryk Wirkus}{Martyna Czarnobaj}
$$(\frac{149}{109})^{8} \cdot (\frac{109}{149})^{8} \cdot \pi^{0} = (\frac{149}{109} \cdot \frac{109}{149})^{8} \cdot 1 = 1^{8} \cdot 1 = 1$$
\rozwStop
\odpStart
$1$
\odpStop
\testStart
A.$1$ B.$\pi$ C.$0$ D.$\frac{149}{109}$ E.$\frac{109}{149}$
F.$-\frac{149}{109}$ G.$-1$
H.$(\frac{149}{109})^{8}$
I.$(\frac{109}{149})^{8}$
\testStop
\kluczStart
A
\kluczStop



\zadStart{Zadanie z Wikieł Z 1.1 d) moja wersja nr 628}

Obliczyć wartość wyrażenia $(\frac{149}{113})^{8} \cdot (\frac{113}{149})^{8} \cdot \pi^{0}$.
\zadStop
\rozwStart{Patryk Wirkus}{Martyna Czarnobaj}
$$(\frac{149}{113})^{8} \cdot (\frac{113}{149})^{8} \cdot \pi^{0} = (\frac{149}{113} \cdot \frac{113}{149})^{8} \cdot 1 = 1^{8} \cdot 1 = 1$$
\rozwStop
\odpStart
$1$
\odpStop
\testStart
A.$1$ B.$\pi$ C.$0$ D.$\frac{149}{113}$ E.$\frac{113}{149}$
F.$-\frac{149}{113}$ G.$-1$
H.$(\frac{149}{113})^{8}$
I.$(\frac{113}{149})^{8}$
\testStop
\kluczStart
A
\kluczStop



\zadStart{Zadanie z Wikieł Z 1.1 d) moja wersja nr 629}

Obliczyć wartość wyrażenia $(\frac{149}{127})^{8} \cdot (\frac{127}{149})^{8} \cdot \pi^{0}$.
\zadStop
\rozwStart{Patryk Wirkus}{Martyna Czarnobaj}
$$(\frac{149}{127})^{8} \cdot (\frac{127}{149})^{8} \cdot \pi^{0} = (\frac{149}{127} \cdot \frac{127}{149})^{8} \cdot 1 = 1^{8} \cdot 1 = 1$$
\rozwStop
\odpStart
$1$
\odpStop
\testStart
A.$1$ B.$\pi$ C.$0$ D.$\frac{149}{127}$ E.$\frac{127}{149}$
F.$-\frac{149}{127}$ G.$-1$
H.$(\frac{149}{127})^{8}$
I.$(\frac{127}{149})^{8}$
\testStop
\kluczStart
A
\kluczStop



\zadStart{Zadanie z Wikieł Z 1.1 d) moja wersja nr 630}

Obliczyć wartość wyrażenia $(\frac{149}{131})^{8} \cdot (\frac{131}{149})^{8} \cdot \pi^{0}$.
\zadStop
\rozwStart{Patryk Wirkus}{Martyna Czarnobaj}
$$(\frac{149}{131})^{8} \cdot (\frac{131}{149})^{8} \cdot \pi^{0} = (\frac{149}{131} \cdot \frac{131}{149})^{8} \cdot 1 = 1^{8} \cdot 1 = 1$$
\rozwStop
\odpStart
$1$
\odpStop
\testStart
A.$1$ B.$\pi$ C.$0$ D.$\frac{149}{131}$ E.$\frac{131}{149}$
F.$-\frac{149}{131}$ G.$-1$
H.$(\frac{149}{131})^{8}$
I.$(\frac{131}{149})^{8}$
\testStop
\kluczStart
A
\kluczStop



\zadStart{Zadanie z Wikieł Z 1.1 d) moja wersja nr 631}

Obliczyć wartość wyrażenia $(\frac{149}{137})^{8} \cdot (\frac{137}{149})^{8} \cdot \pi^{0}$.
\zadStop
\rozwStart{Patryk Wirkus}{Martyna Czarnobaj}
$$(\frac{149}{137})^{8} \cdot (\frac{137}{149})^{8} \cdot \pi^{0} = (\frac{149}{137} \cdot \frac{137}{149})^{8} \cdot 1 = 1^{8} \cdot 1 = 1$$
\rozwStop
\odpStart
$1$
\odpStop
\testStart
A.$1$ B.$\pi$ C.$0$ D.$\frac{149}{137}$ E.$\frac{137}{149}$
F.$-\frac{149}{137}$ G.$-1$
H.$(\frac{149}{137})^{8}$
I.$(\frac{137}{149})^{8}$
\testStop
\kluczStart
A
\kluczStop



\zadStart{Zadanie z Wikieł Z 1.1 d) moja wersja nr 632}

Obliczyć wartość wyrażenia $(\frac{149}{139})^{8} \cdot (\frac{139}{149})^{8} \cdot \pi^{0}$.
\zadStop
\rozwStart{Patryk Wirkus}{Martyna Czarnobaj}
$$(\frac{149}{139})^{8} \cdot (\frac{139}{149})^{8} \cdot \pi^{0} = (\frac{149}{139} \cdot \frac{139}{149})^{8} \cdot 1 = 1^{8} \cdot 1 = 1$$
\rozwStop
\odpStart
$1$
\odpStop
\testStart
A.$1$ B.$\pi$ C.$0$ D.$\frac{149}{139}$ E.$\frac{139}{149}$
F.$-\frac{149}{139}$ G.$-1$
H.$(\frac{149}{139})^{8}$
I.$(\frac{139}{149})^{8}$
\testStop
\kluczStart
A
\kluczStop



\zadStart{Zadanie z Wikieł Z 1.1 d) moja wersja nr 633}

Obliczyć wartość wyrażenia $(\frac{151}{103})^{8} \cdot (\frac{103}{151})^{8} \cdot \pi^{0}$.
\zadStop
\rozwStart{Patryk Wirkus}{Martyna Czarnobaj}
$$(\frac{151}{103})^{8} \cdot (\frac{103}{151})^{8} \cdot \pi^{0} = (\frac{151}{103} \cdot \frac{103}{151})^{8} \cdot 1 = 1^{8} \cdot 1 = 1$$
\rozwStop
\odpStart
$1$
\odpStop
\testStart
A.$1$ B.$\pi$ C.$0$ D.$\frac{151}{103}$ E.$\frac{103}{151}$
F.$-\frac{151}{103}$ G.$-1$
H.$(\frac{151}{103})^{8}$
I.$(\frac{103}{151})^{8}$
\testStop
\kluczStart
A
\kluczStop



\zadStart{Zadanie z Wikieł Z 1.1 d) moja wersja nr 634}

Obliczyć wartość wyrażenia $(\frac{151}{107})^{8} \cdot (\frac{107}{151})^{8} \cdot \pi^{0}$.
\zadStop
\rozwStart{Patryk Wirkus}{Martyna Czarnobaj}
$$(\frac{151}{107})^{8} \cdot (\frac{107}{151})^{8} \cdot \pi^{0} = (\frac{151}{107} \cdot \frac{107}{151})^{8} \cdot 1 = 1^{8} \cdot 1 = 1$$
\rozwStop
\odpStart
$1$
\odpStop
\testStart
A.$1$ B.$\pi$ C.$0$ D.$\frac{151}{107}$ E.$\frac{107}{151}$
F.$-\frac{151}{107}$ G.$-1$
H.$(\frac{151}{107})^{8}$
I.$(\frac{107}{151})^{8}$
\testStop
\kluczStart
A
\kluczStop



\zadStart{Zadanie z Wikieł Z 1.1 d) moja wersja nr 635}

Obliczyć wartość wyrażenia $(\frac{151}{109})^{8} \cdot (\frac{109}{151})^{8} \cdot \pi^{0}$.
\zadStop
\rozwStart{Patryk Wirkus}{Martyna Czarnobaj}
$$(\frac{151}{109})^{8} \cdot (\frac{109}{151})^{8} \cdot \pi^{0} = (\frac{151}{109} \cdot \frac{109}{151})^{8} \cdot 1 = 1^{8} \cdot 1 = 1$$
\rozwStop
\odpStart
$1$
\odpStop
\testStart
A.$1$ B.$\pi$ C.$0$ D.$\frac{151}{109}$ E.$\frac{109}{151}$
F.$-\frac{151}{109}$ G.$-1$
H.$(\frac{151}{109})^{8}$
I.$(\frac{109}{151})^{8}$
\testStop
\kluczStart
A
\kluczStop



\zadStart{Zadanie z Wikieł Z 1.1 d) moja wersja nr 636}

Obliczyć wartość wyrażenia $(\frac{151}{113})^{8} \cdot (\frac{113}{151})^{8} \cdot \pi^{0}$.
\zadStop
\rozwStart{Patryk Wirkus}{Martyna Czarnobaj}
$$(\frac{151}{113})^{8} \cdot (\frac{113}{151})^{8} \cdot \pi^{0} = (\frac{151}{113} \cdot \frac{113}{151})^{8} \cdot 1 = 1^{8} \cdot 1 = 1$$
\rozwStop
\odpStart
$1$
\odpStop
\testStart
A.$1$ B.$\pi$ C.$0$ D.$\frac{151}{113}$ E.$\frac{113}{151}$
F.$-\frac{151}{113}$ G.$-1$
H.$(\frac{151}{113})^{8}$
I.$(\frac{113}{151})^{8}$
\testStop
\kluczStart
A
\kluczStop



\zadStart{Zadanie z Wikieł Z 1.1 d) moja wersja nr 637}

Obliczyć wartość wyrażenia $(\frac{151}{127})^{8} \cdot (\frac{127}{151})^{8} \cdot \pi^{0}$.
\zadStop
\rozwStart{Patryk Wirkus}{Martyna Czarnobaj}
$$(\frac{151}{127})^{8} \cdot (\frac{127}{151})^{8} \cdot \pi^{0} = (\frac{151}{127} \cdot \frac{127}{151})^{8} \cdot 1 = 1^{8} \cdot 1 = 1$$
\rozwStop
\odpStart
$1$
\odpStop
\testStart
A.$1$ B.$\pi$ C.$0$ D.$\frac{151}{127}$ E.$\frac{127}{151}$
F.$-\frac{151}{127}$ G.$-1$
H.$(\frac{151}{127})^{8}$
I.$(\frac{127}{151})^{8}$
\testStop
\kluczStart
A
\kluczStop



\zadStart{Zadanie z Wikieł Z 1.1 d) moja wersja nr 638}

Obliczyć wartość wyrażenia $(\frac{151}{131})^{8} \cdot (\frac{131}{151})^{8} \cdot \pi^{0}$.
\zadStop
\rozwStart{Patryk Wirkus}{Martyna Czarnobaj}
$$(\frac{151}{131})^{8} \cdot (\frac{131}{151})^{8} \cdot \pi^{0} = (\frac{151}{131} \cdot \frac{131}{151})^{8} \cdot 1 = 1^{8} \cdot 1 = 1$$
\rozwStop
\odpStart
$1$
\odpStop
\testStart
A.$1$ B.$\pi$ C.$0$ D.$\frac{151}{131}$ E.$\frac{131}{151}$
F.$-\frac{151}{131}$ G.$-1$
H.$(\frac{151}{131})^{8}$
I.$(\frac{131}{151})^{8}$
\testStop
\kluczStart
A
\kluczStop



\zadStart{Zadanie z Wikieł Z 1.1 d) moja wersja nr 639}

Obliczyć wartość wyrażenia $(\frac{151}{137})^{8} \cdot (\frac{137}{151})^{8} \cdot \pi^{0}$.
\zadStop
\rozwStart{Patryk Wirkus}{Martyna Czarnobaj}
$$(\frac{151}{137})^{8} \cdot (\frac{137}{151})^{8} \cdot \pi^{0} = (\frac{151}{137} \cdot \frac{137}{151})^{8} \cdot 1 = 1^{8} \cdot 1 = 1$$
\rozwStop
\odpStart
$1$
\odpStop
\testStart
A.$1$ B.$\pi$ C.$0$ D.$\frac{151}{137}$ E.$\frac{137}{151}$
F.$-\frac{151}{137}$ G.$-1$
H.$(\frac{151}{137})^{8}$
I.$(\frac{137}{151})^{8}$
\testStop
\kluczStart
A
\kluczStop



\zadStart{Zadanie z Wikieł Z 1.1 d) moja wersja nr 640}

Obliczyć wartość wyrażenia $(\frac{151}{139})^{8} \cdot (\frac{139}{151})^{8} \cdot \pi^{0}$.
\zadStop
\rozwStart{Patryk Wirkus}{Martyna Czarnobaj}
$$(\frac{151}{139})^{8} \cdot (\frac{139}{151})^{8} \cdot \pi^{0} = (\frac{151}{139} \cdot \frac{139}{151})^{8} \cdot 1 = 1^{8} \cdot 1 = 1$$
\rozwStop
\odpStart
$1$
\odpStop
\testStart
A.$1$ B.$\pi$ C.$0$ D.$\frac{151}{139}$ E.$\frac{139}{151}$
F.$-\frac{151}{139}$ G.$-1$
H.$(\frac{151}{139})^{8}$
I.$(\frac{139}{151})^{8}$
\testStop
\kluczStart
A
\kluczStop



\zadStart{Zadanie z Wikieł Z 1.1 d) moja wersja nr 641}

Obliczyć wartość wyrażenia $(\frac{157}{103})^{8} \cdot (\frac{103}{157})^{8} \cdot \pi^{0}$.
\zadStop
\rozwStart{Patryk Wirkus}{Martyna Czarnobaj}
$$(\frac{157}{103})^{8} \cdot (\frac{103}{157})^{8} \cdot \pi^{0} = (\frac{157}{103} \cdot \frac{103}{157})^{8} \cdot 1 = 1^{8} \cdot 1 = 1$$
\rozwStop
\odpStart
$1$
\odpStop
\testStart
A.$1$ B.$\pi$ C.$0$ D.$\frac{157}{103}$ E.$\frac{103}{157}$
F.$-\frac{157}{103}$ G.$-1$
H.$(\frac{157}{103})^{8}$
I.$(\frac{103}{157})^{8}$
\testStop
\kluczStart
A
\kluczStop



\zadStart{Zadanie z Wikieł Z 1.1 d) moja wersja nr 642}

Obliczyć wartość wyrażenia $(\frac{157}{107})^{8} \cdot (\frac{107}{157})^{8} \cdot \pi^{0}$.
\zadStop
\rozwStart{Patryk Wirkus}{Martyna Czarnobaj}
$$(\frac{157}{107})^{8} \cdot (\frac{107}{157})^{8} \cdot \pi^{0} = (\frac{157}{107} \cdot \frac{107}{157})^{8} \cdot 1 = 1^{8} \cdot 1 = 1$$
\rozwStop
\odpStart
$1$
\odpStop
\testStart
A.$1$ B.$\pi$ C.$0$ D.$\frac{157}{107}$ E.$\frac{107}{157}$
F.$-\frac{157}{107}$ G.$-1$
H.$(\frac{157}{107})^{8}$
I.$(\frac{107}{157})^{8}$
\testStop
\kluczStart
A
\kluczStop



\zadStart{Zadanie z Wikieł Z 1.1 d) moja wersja nr 643}

Obliczyć wartość wyrażenia $(\frac{157}{109})^{8} \cdot (\frac{109}{157})^{8} \cdot \pi^{0}$.
\zadStop
\rozwStart{Patryk Wirkus}{Martyna Czarnobaj}
$$(\frac{157}{109})^{8} \cdot (\frac{109}{157})^{8} \cdot \pi^{0} = (\frac{157}{109} \cdot \frac{109}{157})^{8} \cdot 1 = 1^{8} \cdot 1 = 1$$
\rozwStop
\odpStart
$1$
\odpStop
\testStart
A.$1$ B.$\pi$ C.$0$ D.$\frac{157}{109}$ E.$\frac{109}{157}$
F.$-\frac{157}{109}$ G.$-1$
H.$(\frac{157}{109})^{8}$
I.$(\frac{109}{157})^{8}$
\testStop
\kluczStart
A
\kluczStop



\zadStart{Zadanie z Wikieł Z 1.1 d) moja wersja nr 644}

Obliczyć wartość wyrażenia $(\frac{157}{113})^{8} \cdot (\frac{113}{157})^{8} \cdot \pi^{0}$.
\zadStop
\rozwStart{Patryk Wirkus}{Martyna Czarnobaj}
$$(\frac{157}{113})^{8} \cdot (\frac{113}{157})^{8} \cdot \pi^{0} = (\frac{157}{113} \cdot \frac{113}{157})^{8} \cdot 1 = 1^{8} \cdot 1 = 1$$
\rozwStop
\odpStart
$1$
\odpStop
\testStart
A.$1$ B.$\pi$ C.$0$ D.$\frac{157}{113}$ E.$\frac{113}{157}$
F.$-\frac{157}{113}$ G.$-1$
H.$(\frac{157}{113})^{8}$
I.$(\frac{113}{157})^{8}$
\testStop
\kluczStart
A
\kluczStop



\zadStart{Zadanie z Wikieł Z 1.1 d) moja wersja nr 645}

Obliczyć wartość wyrażenia $(\frac{157}{127})^{8} \cdot (\frac{127}{157})^{8} \cdot \pi^{0}$.
\zadStop
\rozwStart{Patryk Wirkus}{Martyna Czarnobaj}
$$(\frac{157}{127})^{8} \cdot (\frac{127}{157})^{8} \cdot \pi^{0} = (\frac{157}{127} \cdot \frac{127}{157})^{8} \cdot 1 = 1^{8} \cdot 1 = 1$$
\rozwStop
\odpStart
$1$
\odpStop
\testStart
A.$1$ B.$\pi$ C.$0$ D.$\frac{157}{127}$ E.$\frac{127}{157}$
F.$-\frac{157}{127}$ G.$-1$
H.$(\frac{157}{127})^{8}$
I.$(\frac{127}{157})^{8}$
\testStop
\kluczStart
A
\kluczStop



\zadStart{Zadanie z Wikieł Z 1.1 d) moja wersja nr 646}

Obliczyć wartość wyrażenia $(\frac{157}{131})^{8} \cdot (\frac{131}{157})^{8} \cdot \pi^{0}$.
\zadStop
\rozwStart{Patryk Wirkus}{Martyna Czarnobaj}
$$(\frac{157}{131})^{8} \cdot (\frac{131}{157})^{8} \cdot \pi^{0} = (\frac{157}{131} \cdot \frac{131}{157})^{8} \cdot 1 = 1^{8} \cdot 1 = 1$$
\rozwStop
\odpStart
$1$
\odpStop
\testStart
A.$1$ B.$\pi$ C.$0$ D.$\frac{157}{131}$ E.$\frac{131}{157}$
F.$-\frac{157}{131}$ G.$-1$
H.$(\frac{157}{131})^{8}$
I.$(\frac{131}{157})^{8}$
\testStop
\kluczStart
A
\kluczStop



\zadStart{Zadanie z Wikieł Z 1.1 d) moja wersja nr 647}

Obliczyć wartość wyrażenia $(\frac{157}{137})^{8} \cdot (\frac{137}{157})^{8} \cdot \pi^{0}$.
\zadStop
\rozwStart{Patryk Wirkus}{Martyna Czarnobaj}
$$(\frac{157}{137})^{8} \cdot (\frac{137}{157})^{8} \cdot \pi^{0} = (\frac{157}{137} \cdot \frac{137}{157})^{8} \cdot 1 = 1^{8} \cdot 1 = 1$$
\rozwStop
\odpStart
$1$
\odpStop
\testStart
A.$1$ B.$\pi$ C.$0$ D.$\frac{157}{137}$ E.$\frac{137}{157}$
F.$-\frac{157}{137}$ G.$-1$
H.$(\frac{157}{137})^{8}$
I.$(\frac{137}{157})^{8}$
\testStop
\kluczStart
A
\kluczStop



\zadStart{Zadanie z Wikieł Z 1.1 d) moja wersja nr 648}

Obliczyć wartość wyrażenia $(\frac{157}{139})^{8} \cdot (\frac{139}{157})^{8} \cdot \pi^{0}$.
\zadStop
\rozwStart{Patryk Wirkus}{Martyna Czarnobaj}
$$(\frac{157}{139})^{8} \cdot (\frac{139}{157})^{8} \cdot \pi^{0} = (\frac{157}{139} \cdot \frac{139}{157})^{8} \cdot 1 = 1^{8} \cdot 1 = 1$$
\rozwStop
\odpStart
$1$
\odpStop
\testStart
A.$1$ B.$\pi$ C.$0$ D.$\frac{157}{139}$ E.$\frac{139}{157}$
F.$-\frac{157}{139}$ G.$-1$
H.$(\frac{157}{139})^{8}$
I.$(\frac{139}{157})^{8}$
\testStop
\kluczStart
A
\kluczStop



\zadStart{Zadanie z Wikieł Z 1.1 d) moja wersja nr 649}

Obliczyć wartość wyrażenia $(\frac{163}{103})^{8} \cdot (\frac{103}{163})^{8} \cdot \pi^{0}$.
\zadStop
\rozwStart{Patryk Wirkus}{Martyna Czarnobaj}
$$(\frac{163}{103})^{8} \cdot (\frac{103}{163})^{8} \cdot \pi^{0} = (\frac{163}{103} \cdot \frac{103}{163})^{8} \cdot 1 = 1^{8} \cdot 1 = 1$$
\rozwStop
\odpStart
$1$
\odpStop
\testStart
A.$1$ B.$\pi$ C.$0$ D.$\frac{163}{103}$ E.$\frac{103}{163}$
F.$-\frac{163}{103}$ G.$-1$
H.$(\frac{163}{103})^{8}$
I.$(\frac{103}{163})^{8}$
\testStop
\kluczStart
A
\kluczStop



\zadStart{Zadanie z Wikieł Z 1.1 d) moja wersja nr 650}

Obliczyć wartość wyrażenia $(\frac{163}{107})^{8} \cdot (\frac{107}{163})^{8} \cdot \pi^{0}$.
\zadStop
\rozwStart{Patryk Wirkus}{Martyna Czarnobaj}
$$(\frac{163}{107})^{8} \cdot (\frac{107}{163})^{8} \cdot \pi^{0} = (\frac{163}{107} \cdot \frac{107}{163})^{8} \cdot 1 = 1^{8} \cdot 1 = 1$$
\rozwStop
\odpStart
$1$
\odpStop
\testStart
A.$1$ B.$\pi$ C.$0$ D.$\frac{163}{107}$ E.$\frac{107}{163}$
F.$-\frac{163}{107}$ G.$-1$
H.$(\frac{163}{107})^{8}$
I.$(\frac{107}{163})^{8}$
\testStop
\kluczStart
A
\kluczStop



\zadStart{Zadanie z Wikieł Z 1.1 d) moja wersja nr 651}

Obliczyć wartość wyrażenia $(\frac{163}{109})^{8} \cdot (\frac{109}{163})^{8} \cdot \pi^{0}$.
\zadStop
\rozwStart{Patryk Wirkus}{Martyna Czarnobaj}
$$(\frac{163}{109})^{8} \cdot (\frac{109}{163})^{8} \cdot \pi^{0} = (\frac{163}{109} \cdot \frac{109}{163})^{8} \cdot 1 = 1^{8} \cdot 1 = 1$$
\rozwStop
\odpStart
$1$
\odpStop
\testStart
A.$1$ B.$\pi$ C.$0$ D.$\frac{163}{109}$ E.$\frac{109}{163}$
F.$-\frac{163}{109}$ G.$-1$
H.$(\frac{163}{109})^{8}$
I.$(\frac{109}{163})^{8}$
\testStop
\kluczStart
A
\kluczStop



\zadStart{Zadanie z Wikieł Z 1.1 d) moja wersja nr 652}

Obliczyć wartość wyrażenia $(\frac{163}{113})^{8} \cdot (\frac{113}{163})^{8} \cdot \pi^{0}$.
\zadStop
\rozwStart{Patryk Wirkus}{Martyna Czarnobaj}
$$(\frac{163}{113})^{8} \cdot (\frac{113}{163})^{8} \cdot \pi^{0} = (\frac{163}{113} \cdot \frac{113}{163})^{8} \cdot 1 = 1^{8} \cdot 1 = 1$$
\rozwStop
\odpStart
$1$
\odpStop
\testStart
A.$1$ B.$\pi$ C.$0$ D.$\frac{163}{113}$ E.$\frac{113}{163}$
F.$-\frac{163}{113}$ G.$-1$
H.$(\frac{163}{113})^{8}$
I.$(\frac{113}{163})^{8}$
\testStop
\kluczStart
A
\kluczStop



\zadStart{Zadanie z Wikieł Z 1.1 d) moja wersja nr 653}

Obliczyć wartość wyrażenia $(\frac{163}{127})^{8} \cdot (\frac{127}{163})^{8} \cdot \pi^{0}$.
\zadStop
\rozwStart{Patryk Wirkus}{Martyna Czarnobaj}
$$(\frac{163}{127})^{8} \cdot (\frac{127}{163})^{8} \cdot \pi^{0} = (\frac{163}{127} \cdot \frac{127}{163})^{8} \cdot 1 = 1^{8} \cdot 1 = 1$$
\rozwStop
\odpStart
$1$
\odpStop
\testStart
A.$1$ B.$\pi$ C.$0$ D.$\frac{163}{127}$ E.$\frac{127}{163}$
F.$-\frac{163}{127}$ G.$-1$
H.$(\frac{163}{127})^{8}$
I.$(\frac{127}{163})^{8}$
\testStop
\kluczStart
A
\kluczStop



\zadStart{Zadanie z Wikieł Z 1.1 d) moja wersja nr 654}

Obliczyć wartość wyrażenia $(\frac{163}{131})^{8} \cdot (\frac{131}{163})^{8} \cdot \pi^{0}$.
\zadStop
\rozwStart{Patryk Wirkus}{Martyna Czarnobaj}
$$(\frac{163}{131})^{8} \cdot (\frac{131}{163})^{8} \cdot \pi^{0} = (\frac{163}{131} \cdot \frac{131}{163})^{8} \cdot 1 = 1^{8} \cdot 1 = 1$$
\rozwStop
\odpStart
$1$
\odpStop
\testStart
A.$1$ B.$\pi$ C.$0$ D.$\frac{163}{131}$ E.$\frac{131}{163}$
F.$-\frac{163}{131}$ G.$-1$
H.$(\frac{163}{131})^{8}$
I.$(\frac{131}{163})^{8}$
\testStop
\kluczStart
A
\kluczStop



\zadStart{Zadanie z Wikieł Z 1.1 d) moja wersja nr 655}

Obliczyć wartość wyrażenia $(\frac{163}{137})^{8} \cdot (\frac{137}{163})^{8} \cdot \pi^{0}$.
\zadStop
\rozwStart{Patryk Wirkus}{Martyna Czarnobaj}
$$(\frac{163}{137})^{8} \cdot (\frac{137}{163})^{8} \cdot \pi^{0} = (\frac{163}{137} \cdot \frac{137}{163})^{8} \cdot 1 = 1^{8} \cdot 1 = 1$$
\rozwStop
\odpStart
$1$
\odpStop
\testStart
A.$1$ B.$\pi$ C.$0$ D.$\frac{163}{137}$ E.$\frac{137}{163}$
F.$-\frac{163}{137}$ G.$-1$
H.$(\frac{163}{137})^{8}$
I.$(\frac{137}{163})^{8}$
\testStop
\kluczStart
A
\kluczStop



\zadStart{Zadanie z Wikieł Z 1.1 d) moja wersja nr 656}

Obliczyć wartość wyrażenia $(\frac{163}{139})^{8} \cdot (\frac{139}{163})^{8} \cdot \pi^{0}$.
\zadStop
\rozwStart{Patryk Wirkus}{Martyna Czarnobaj}
$$(\frac{163}{139})^{8} \cdot (\frac{139}{163})^{8} \cdot \pi^{0} = (\frac{163}{139} \cdot \frac{139}{163})^{8} \cdot 1 = 1^{8} \cdot 1 = 1$$
\rozwStop
\odpStart
$1$
\odpStop
\testStart
A.$1$ B.$\pi$ C.$0$ D.$\frac{163}{139}$ E.$\frac{139}{163}$
F.$-\frac{163}{139}$ G.$-1$
H.$(\frac{163}{139})^{8}$
I.$(\frac{139}{163})^{8}$
\testStop
\kluczStart
A
\kluczStop



\zadStart{Zadanie z Wikieł Z 1.1 d) moja wersja nr 657}

Obliczyć wartość wyrażenia $(\frac{167}{103})^{8} \cdot (\frac{103}{167})^{8} \cdot \pi^{0}$.
\zadStop
\rozwStart{Patryk Wirkus}{Martyna Czarnobaj}
$$(\frac{167}{103})^{8} \cdot (\frac{103}{167})^{8} \cdot \pi^{0} = (\frac{167}{103} \cdot \frac{103}{167})^{8} \cdot 1 = 1^{8} \cdot 1 = 1$$
\rozwStop
\odpStart
$1$
\odpStop
\testStart
A.$1$ B.$\pi$ C.$0$ D.$\frac{167}{103}$ E.$\frac{103}{167}$
F.$-\frac{167}{103}$ G.$-1$
H.$(\frac{167}{103})^{8}$
I.$(\frac{103}{167})^{8}$
\testStop
\kluczStart
A
\kluczStop



\zadStart{Zadanie z Wikieł Z 1.1 d) moja wersja nr 658}

Obliczyć wartość wyrażenia $(\frac{167}{107})^{8} \cdot (\frac{107}{167})^{8} \cdot \pi^{0}$.
\zadStop
\rozwStart{Patryk Wirkus}{Martyna Czarnobaj}
$$(\frac{167}{107})^{8} \cdot (\frac{107}{167})^{8} \cdot \pi^{0} = (\frac{167}{107} \cdot \frac{107}{167})^{8} \cdot 1 = 1^{8} \cdot 1 = 1$$
\rozwStop
\odpStart
$1$
\odpStop
\testStart
A.$1$ B.$\pi$ C.$0$ D.$\frac{167}{107}$ E.$\frac{107}{167}$
F.$-\frac{167}{107}$ G.$-1$
H.$(\frac{167}{107})^{8}$
I.$(\frac{107}{167})^{8}$
\testStop
\kluczStart
A
\kluczStop



\zadStart{Zadanie z Wikieł Z 1.1 d) moja wersja nr 659}

Obliczyć wartość wyrażenia $(\frac{167}{109})^{8} \cdot (\frac{109}{167})^{8} \cdot \pi^{0}$.
\zadStop
\rozwStart{Patryk Wirkus}{Martyna Czarnobaj}
$$(\frac{167}{109})^{8} \cdot (\frac{109}{167})^{8} \cdot \pi^{0} = (\frac{167}{109} \cdot \frac{109}{167})^{8} \cdot 1 = 1^{8} \cdot 1 = 1$$
\rozwStop
\odpStart
$1$
\odpStop
\testStart
A.$1$ B.$\pi$ C.$0$ D.$\frac{167}{109}$ E.$\frac{109}{167}$
F.$-\frac{167}{109}$ G.$-1$
H.$(\frac{167}{109})^{8}$
I.$(\frac{109}{167})^{8}$
\testStop
\kluczStart
A
\kluczStop



\zadStart{Zadanie z Wikieł Z 1.1 d) moja wersja nr 660}

Obliczyć wartość wyrażenia $(\frac{167}{113})^{8} \cdot (\frac{113}{167})^{8} \cdot \pi^{0}$.
\zadStop
\rozwStart{Patryk Wirkus}{Martyna Czarnobaj}
$$(\frac{167}{113})^{8} \cdot (\frac{113}{167})^{8} \cdot \pi^{0} = (\frac{167}{113} \cdot \frac{113}{167})^{8} \cdot 1 = 1^{8} \cdot 1 = 1$$
\rozwStop
\odpStart
$1$
\odpStop
\testStart
A.$1$ B.$\pi$ C.$0$ D.$\frac{167}{113}$ E.$\frac{113}{167}$
F.$-\frac{167}{113}$ G.$-1$
H.$(\frac{167}{113})^{8}$
I.$(\frac{113}{167})^{8}$
\testStop
\kluczStart
A
\kluczStop



\zadStart{Zadanie z Wikieł Z 1.1 d) moja wersja nr 661}

Obliczyć wartość wyrażenia $(\frac{167}{127})^{8} \cdot (\frac{127}{167})^{8} \cdot \pi^{0}$.
\zadStop
\rozwStart{Patryk Wirkus}{Martyna Czarnobaj}
$$(\frac{167}{127})^{8} \cdot (\frac{127}{167})^{8} \cdot \pi^{0} = (\frac{167}{127} \cdot \frac{127}{167})^{8} \cdot 1 = 1^{8} \cdot 1 = 1$$
\rozwStop
\odpStart
$1$
\odpStop
\testStart
A.$1$ B.$\pi$ C.$0$ D.$\frac{167}{127}$ E.$\frac{127}{167}$
F.$-\frac{167}{127}$ G.$-1$
H.$(\frac{167}{127})^{8}$
I.$(\frac{127}{167})^{8}$
\testStop
\kluczStart
A
\kluczStop



\zadStart{Zadanie z Wikieł Z 1.1 d) moja wersja nr 662}

Obliczyć wartość wyrażenia $(\frac{167}{131})^{8} \cdot (\frac{131}{167})^{8} \cdot \pi^{0}$.
\zadStop
\rozwStart{Patryk Wirkus}{Martyna Czarnobaj}
$$(\frac{167}{131})^{8} \cdot (\frac{131}{167})^{8} \cdot \pi^{0} = (\frac{167}{131} \cdot \frac{131}{167})^{8} \cdot 1 = 1^{8} \cdot 1 = 1$$
\rozwStop
\odpStart
$1$
\odpStop
\testStart
A.$1$ B.$\pi$ C.$0$ D.$\frac{167}{131}$ E.$\frac{131}{167}$
F.$-\frac{167}{131}$ G.$-1$
H.$(\frac{167}{131})^{8}$
I.$(\frac{131}{167})^{8}$
\testStop
\kluczStart
A
\kluczStop



\zadStart{Zadanie z Wikieł Z 1.1 d) moja wersja nr 663}

Obliczyć wartość wyrażenia $(\frac{167}{137})^{8} \cdot (\frac{137}{167})^{8} \cdot \pi^{0}$.
\zadStop
\rozwStart{Patryk Wirkus}{Martyna Czarnobaj}
$$(\frac{167}{137})^{8} \cdot (\frac{137}{167})^{8} \cdot \pi^{0} = (\frac{167}{137} \cdot \frac{137}{167})^{8} \cdot 1 = 1^{8} \cdot 1 = 1$$
\rozwStop
\odpStart
$1$
\odpStop
\testStart
A.$1$ B.$\pi$ C.$0$ D.$\frac{167}{137}$ E.$\frac{137}{167}$
F.$-\frac{167}{137}$ G.$-1$
H.$(\frac{167}{137})^{8}$
I.$(\frac{137}{167})^{8}$
\testStop
\kluczStart
A
\kluczStop



\zadStart{Zadanie z Wikieł Z 1.1 d) moja wersja nr 664}

Obliczyć wartość wyrażenia $(\frac{167}{139})^{8} \cdot (\frac{139}{167})^{8} \cdot \pi^{0}$.
\zadStop
\rozwStart{Patryk Wirkus}{Martyna Czarnobaj}
$$(\frac{167}{139})^{8} \cdot (\frac{139}{167})^{8} \cdot \pi^{0} = (\frac{167}{139} \cdot \frac{139}{167})^{8} \cdot 1 = 1^{8} \cdot 1 = 1$$
\rozwStop
\odpStart
$1$
\odpStop
\testStart
A.$1$ B.$\pi$ C.$0$ D.$\frac{167}{139}$ E.$\frac{139}{167}$
F.$-\frac{167}{139}$ G.$-1$
H.$(\frac{167}{139})^{8}$
I.$(\frac{139}{167})^{8}$
\testStop
\kluczStart
A
\kluczStop



\zadStart{Zadanie z Wikieł Z 1.1 d) moja wersja nr 665}

Obliczyć wartość wyrażenia $(\frac{173}{103})^{8} \cdot (\frac{103}{173})^{8} \cdot \pi^{0}$.
\zadStop
\rozwStart{Patryk Wirkus}{Martyna Czarnobaj}
$$(\frac{173}{103})^{8} \cdot (\frac{103}{173})^{8} \cdot \pi^{0} = (\frac{173}{103} \cdot \frac{103}{173})^{8} \cdot 1 = 1^{8} \cdot 1 = 1$$
\rozwStop
\odpStart
$1$
\odpStop
\testStart
A.$1$ B.$\pi$ C.$0$ D.$\frac{173}{103}$ E.$\frac{103}{173}$
F.$-\frac{173}{103}$ G.$-1$
H.$(\frac{173}{103})^{8}$
I.$(\frac{103}{173})^{8}$
\testStop
\kluczStart
A
\kluczStop



\zadStart{Zadanie z Wikieł Z 1.1 d) moja wersja nr 666}

Obliczyć wartość wyrażenia $(\frac{173}{107})^{8} \cdot (\frac{107}{173})^{8} \cdot \pi^{0}$.
\zadStop
\rozwStart{Patryk Wirkus}{Martyna Czarnobaj}
$$(\frac{173}{107})^{8} \cdot (\frac{107}{173})^{8} \cdot \pi^{0} = (\frac{173}{107} \cdot \frac{107}{173})^{8} \cdot 1 = 1^{8} \cdot 1 = 1$$
\rozwStop
\odpStart
$1$
\odpStop
\testStart
A.$1$ B.$\pi$ C.$0$ D.$\frac{173}{107}$ E.$\frac{107}{173}$
F.$-\frac{173}{107}$ G.$-1$
H.$(\frac{173}{107})^{8}$
I.$(\frac{107}{173})^{8}$
\testStop
\kluczStart
A
\kluczStop



\zadStart{Zadanie z Wikieł Z 1.1 d) moja wersja nr 667}

Obliczyć wartość wyrażenia $(\frac{173}{109})^{8} \cdot (\frac{109}{173})^{8} \cdot \pi^{0}$.
\zadStop
\rozwStart{Patryk Wirkus}{Martyna Czarnobaj}
$$(\frac{173}{109})^{8} \cdot (\frac{109}{173})^{8} \cdot \pi^{0} = (\frac{173}{109} \cdot \frac{109}{173})^{8} \cdot 1 = 1^{8} \cdot 1 = 1$$
\rozwStop
\odpStart
$1$
\odpStop
\testStart
A.$1$ B.$\pi$ C.$0$ D.$\frac{173}{109}$ E.$\frac{109}{173}$
F.$-\frac{173}{109}$ G.$-1$
H.$(\frac{173}{109})^{8}$
I.$(\frac{109}{173})^{8}$
\testStop
\kluczStart
A
\kluczStop



\zadStart{Zadanie z Wikieł Z 1.1 d) moja wersja nr 668}

Obliczyć wartość wyrażenia $(\frac{173}{113})^{8} \cdot (\frac{113}{173})^{8} \cdot \pi^{0}$.
\zadStop
\rozwStart{Patryk Wirkus}{Martyna Czarnobaj}
$$(\frac{173}{113})^{8} \cdot (\frac{113}{173})^{8} \cdot \pi^{0} = (\frac{173}{113} \cdot \frac{113}{173})^{8} \cdot 1 = 1^{8} \cdot 1 = 1$$
\rozwStop
\odpStart
$1$
\odpStop
\testStart
A.$1$ B.$\pi$ C.$0$ D.$\frac{173}{113}$ E.$\frac{113}{173}$
F.$-\frac{173}{113}$ G.$-1$
H.$(\frac{173}{113})^{8}$
I.$(\frac{113}{173})^{8}$
\testStop
\kluczStart
A
\kluczStop



\zadStart{Zadanie z Wikieł Z 1.1 d) moja wersja nr 669}

Obliczyć wartość wyrażenia $(\frac{173}{127})^{8} \cdot (\frac{127}{173})^{8} \cdot \pi^{0}$.
\zadStop
\rozwStart{Patryk Wirkus}{Martyna Czarnobaj}
$$(\frac{173}{127})^{8} \cdot (\frac{127}{173})^{8} \cdot \pi^{0} = (\frac{173}{127} \cdot \frac{127}{173})^{8} \cdot 1 = 1^{8} \cdot 1 = 1$$
\rozwStop
\odpStart
$1$
\odpStop
\testStart
A.$1$ B.$\pi$ C.$0$ D.$\frac{173}{127}$ E.$\frac{127}{173}$
F.$-\frac{173}{127}$ G.$-1$
H.$(\frac{173}{127})^{8}$
I.$(\frac{127}{173})^{8}$
\testStop
\kluczStart
A
\kluczStop



\zadStart{Zadanie z Wikieł Z 1.1 d) moja wersja nr 670}

Obliczyć wartość wyrażenia $(\frac{173}{131})^{8} \cdot (\frac{131}{173})^{8} \cdot \pi^{0}$.
\zadStop
\rozwStart{Patryk Wirkus}{Martyna Czarnobaj}
$$(\frac{173}{131})^{8} \cdot (\frac{131}{173})^{8} \cdot \pi^{0} = (\frac{173}{131} \cdot \frac{131}{173})^{8} \cdot 1 = 1^{8} \cdot 1 = 1$$
\rozwStop
\odpStart
$1$
\odpStop
\testStart
A.$1$ B.$\pi$ C.$0$ D.$\frac{173}{131}$ E.$\frac{131}{173}$
F.$-\frac{173}{131}$ G.$-1$
H.$(\frac{173}{131})^{8}$
I.$(\frac{131}{173})^{8}$
\testStop
\kluczStart
A
\kluczStop



\zadStart{Zadanie z Wikieł Z 1.1 d) moja wersja nr 671}

Obliczyć wartość wyrażenia $(\frac{173}{137})^{8} \cdot (\frac{137}{173})^{8} \cdot \pi^{0}$.
\zadStop
\rozwStart{Patryk Wirkus}{Martyna Czarnobaj}
$$(\frac{173}{137})^{8} \cdot (\frac{137}{173})^{8} \cdot \pi^{0} = (\frac{173}{137} \cdot \frac{137}{173})^{8} \cdot 1 = 1^{8} \cdot 1 = 1$$
\rozwStop
\odpStart
$1$
\odpStop
\testStart
A.$1$ B.$\pi$ C.$0$ D.$\frac{173}{137}$ E.$\frac{137}{173}$
F.$-\frac{173}{137}$ G.$-1$
H.$(\frac{173}{137})^{8}$
I.$(\frac{137}{173})^{8}$
\testStop
\kluczStart
A
\kluczStop



\zadStart{Zadanie z Wikieł Z 1.1 d) moja wersja nr 672}

Obliczyć wartość wyrażenia $(\frac{173}{139})^{8} \cdot (\frac{139}{173})^{8} \cdot \pi^{0}$.
\zadStop
\rozwStart{Patryk Wirkus}{Martyna Czarnobaj}
$$(\frac{173}{139})^{8} \cdot (\frac{139}{173})^{8} \cdot \pi^{0} = (\frac{173}{139} \cdot \frac{139}{173})^{8} \cdot 1 = 1^{8} \cdot 1 = 1$$
\rozwStop
\odpStart
$1$
\odpStop
\testStart
A.$1$ B.$\pi$ C.$0$ D.$\frac{173}{139}$ E.$\frac{139}{173}$
F.$-\frac{173}{139}$ G.$-1$
H.$(\frac{173}{139})^{8}$
I.$(\frac{139}{173})^{8}$
\testStop
\kluczStart
A
\kluczStop



\zadStart{Zadanie z Wikieł Z 1.1 d) moja wersja nr 673}

Obliczyć wartość wyrażenia $(\frac{179}{103})^{8} \cdot (\frac{103}{179})^{8} \cdot \pi^{0}$.
\zadStop
\rozwStart{Patryk Wirkus}{Martyna Czarnobaj}
$$(\frac{179}{103})^{8} \cdot (\frac{103}{179})^{8} \cdot \pi^{0} = (\frac{179}{103} \cdot \frac{103}{179})^{8} \cdot 1 = 1^{8} \cdot 1 = 1$$
\rozwStop
\odpStart
$1$
\odpStop
\testStart
A.$1$ B.$\pi$ C.$0$ D.$\frac{179}{103}$ E.$\frac{103}{179}$
F.$-\frac{179}{103}$ G.$-1$
H.$(\frac{179}{103})^{8}$
I.$(\frac{103}{179})^{8}$
\testStop
\kluczStart
A
\kluczStop



\zadStart{Zadanie z Wikieł Z 1.1 d) moja wersja nr 674}

Obliczyć wartość wyrażenia $(\frac{179}{107})^{8} \cdot (\frac{107}{179})^{8} \cdot \pi^{0}$.
\zadStop
\rozwStart{Patryk Wirkus}{Martyna Czarnobaj}
$$(\frac{179}{107})^{8} \cdot (\frac{107}{179})^{8} \cdot \pi^{0} = (\frac{179}{107} \cdot \frac{107}{179})^{8} \cdot 1 = 1^{8} \cdot 1 = 1$$
\rozwStop
\odpStart
$1$
\odpStop
\testStart
A.$1$ B.$\pi$ C.$0$ D.$\frac{179}{107}$ E.$\frac{107}{179}$
F.$-\frac{179}{107}$ G.$-1$
H.$(\frac{179}{107})^{8}$
I.$(\frac{107}{179})^{8}$
\testStop
\kluczStart
A
\kluczStop



\zadStart{Zadanie z Wikieł Z 1.1 d) moja wersja nr 675}

Obliczyć wartość wyrażenia $(\frac{179}{109})^{8} \cdot (\frac{109}{179})^{8} \cdot \pi^{0}$.
\zadStop
\rozwStart{Patryk Wirkus}{Martyna Czarnobaj}
$$(\frac{179}{109})^{8} \cdot (\frac{109}{179})^{8} \cdot \pi^{0} = (\frac{179}{109} \cdot \frac{109}{179})^{8} \cdot 1 = 1^{8} \cdot 1 = 1$$
\rozwStop
\odpStart
$1$
\odpStop
\testStart
A.$1$ B.$\pi$ C.$0$ D.$\frac{179}{109}$ E.$\frac{109}{179}$
F.$-\frac{179}{109}$ G.$-1$
H.$(\frac{179}{109})^{8}$
I.$(\frac{109}{179})^{8}$
\testStop
\kluczStart
A
\kluczStop



\zadStart{Zadanie z Wikieł Z 1.1 d) moja wersja nr 676}

Obliczyć wartość wyrażenia $(\frac{179}{113})^{8} \cdot (\frac{113}{179})^{8} \cdot \pi^{0}$.
\zadStop
\rozwStart{Patryk Wirkus}{Martyna Czarnobaj}
$$(\frac{179}{113})^{8} \cdot (\frac{113}{179})^{8} \cdot \pi^{0} = (\frac{179}{113} \cdot \frac{113}{179})^{8} \cdot 1 = 1^{8} \cdot 1 = 1$$
\rozwStop
\odpStart
$1$
\odpStop
\testStart
A.$1$ B.$\pi$ C.$0$ D.$\frac{179}{113}$ E.$\frac{113}{179}$
F.$-\frac{179}{113}$ G.$-1$
H.$(\frac{179}{113})^{8}$
I.$(\frac{113}{179})^{8}$
\testStop
\kluczStart
A
\kluczStop



\zadStart{Zadanie z Wikieł Z 1.1 d) moja wersja nr 677}

Obliczyć wartość wyrażenia $(\frac{179}{127})^{8} \cdot (\frac{127}{179})^{8} \cdot \pi^{0}$.
\zadStop
\rozwStart{Patryk Wirkus}{Martyna Czarnobaj}
$$(\frac{179}{127})^{8} \cdot (\frac{127}{179})^{8} \cdot \pi^{0} = (\frac{179}{127} \cdot \frac{127}{179})^{8} \cdot 1 = 1^{8} \cdot 1 = 1$$
\rozwStop
\odpStart
$1$
\odpStop
\testStart
A.$1$ B.$\pi$ C.$0$ D.$\frac{179}{127}$ E.$\frac{127}{179}$
F.$-\frac{179}{127}$ G.$-1$
H.$(\frac{179}{127})^{8}$
I.$(\frac{127}{179})^{8}$
\testStop
\kluczStart
A
\kluczStop



\zadStart{Zadanie z Wikieł Z 1.1 d) moja wersja nr 678}

Obliczyć wartość wyrażenia $(\frac{179}{131})^{8} \cdot (\frac{131}{179})^{8} \cdot \pi^{0}$.
\zadStop
\rozwStart{Patryk Wirkus}{Martyna Czarnobaj}
$$(\frac{179}{131})^{8} \cdot (\frac{131}{179})^{8} \cdot \pi^{0} = (\frac{179}{131} \cdot \frac{131}{179})^{8} \cdot 1 = 1^{8} \cdot 1 = 1$$
\rozwStop
\odpStart
$1$
\odpStop
\testStart
A.$1$ B.$\pi$ C.$0$ D.$\frac{179}{131}$ E.$\frac{131}{179}$
F.$-\frac{179}{131}$ G.$-1$
H.$(\frac{179}{131})^{8}$
I.$(\frac{131}{179})^{8}$
\testStop
\kluczStart
A
\kluczStop



\zadStart{Zadanie z Wikieł Z 1.1 d) moja wersja nr 679}

Obliczyć wartość wyrażenia $(\frac{179}{137})^{8} \cdot (\frac{137}{179})^{8} \cdot \pi^{0}$.
\zadStop
\rozwStart{Patryk Wirkus}{Martyna Czarnobaj}
$$(\frac{179}{137})^{8} \cdot (\frac{137}{179})^{8} \cdot \pi^{0} = (\frac{179}{137} \cdot \frac{137}{179})^{8} \cdot 1 = 1^{8} \cdot 1 = 1$$
\rozwStop
\odpStart
$1$
\odpStop
\testStart
A.$1$ B.$\pi$ C.$0$ D.$\frac{179}{137}$ E.$\frac{137}{179}$
F.$-\frac{179}{137}$ G.$-1$
H.$(\frac{179}{137})^{8}$
I.$(\frac{137}{179})^{8}$
\testStop
\kluczStart
A
\kluczStop



\zadStart{Zadanie z Wikieł Z 1.1 d) moja wersja nr 680}

Obliczyć wartość wyrażenia $(\frac{179}{139})^{8} \cdot (\frac{139}{179})^{8} \cdot \pi^{0}$.
\zadStop
\rozwStart{Patryk Wirkus}{Martyna Czarnobaj}
$$(\frac{179}{139})^{8} \cdot (\frac{139}{179})^{8} \cdot \pi^{0} = (\frac{179}{139} \cdot \frac{139}{179})^{8} \cdot 1 = 1^{8} \cdot 1 = 1$$
\rozwStop
\odpStart
$1$
\odpStop
\testStart
A.$1$ B.$\pi$ C.$0$ D.$\frac{179}{139}$ E.$\frac{139}{179}$
F.$-\frac{179}{139}$ G.$-1$
H.$(\frac{179}{139})^{8}$
I.$(\frac{139}{179})^{8}$
\testStop
\kluczStart
A
\kluczStop



\zadStart{Zadanie z Wikieł Z 1.1 d) moja wersja nr 681}

Obliczyć wartość wyrażenia $(\frac{251}{103})^{8} \cdot (\frac{103}{251})^{8} \cdot \pi^{0}$.
\zadStop
\rozwStart{Patryk Wirkus}{Martyna Czarnobaj}
$$(\frac{251}{103})^{8} \cdot (\frac{103}{251})^{8} \cdot \pi^{0} = (\frac{251}{103} \cdot \frac{103}{251})^{8} \cdot 1 = 1^{8} \cdot 1 = 1$$
\rozwStop
\odpStart
$1$
\odpStop
\testStart
A.$1$ B.$\pi$ C.$0$ D.$\frac{251}{103}$ E.$\frac{103}{251}$
F.$-\frac{251}{103}$ G.$-1$
H.$(\frac{251}{103})^{8}$
I.$(\frac{103}{251})^{8}$
\testStop
\kluczStart
A
\kluczStop



\zadStart{Zadanie z Wikieł Z 1.1 d) moja wersja nr 682}

Obliczyć wartość wyrażenia $(\frac{251}{107})^{8} \cdot (\frac{107}{251})^{8} \cdot \pi^{0}$.
\zadStop
\rozwStart{Patryk Wirkus}{Martyna Czarnobaj}
$$(\frac{251}{107})^{8} \cdot (\frac{107}{251})^{8} \cdot \pi^{0} = (\frac{251}{107} \cdot \frac{107}{251})^{8} \cdot 1 = 1^{8} \cdot 1 = 1$$
\rozwStop
\odpStart
$1$
\odpStop
\testStart
A.$1$ B.$\pi$ C.$0$ D.$\frac{251}{107}$ E.$\frac{107}{251}$
F.$-\frac{251}{107}$ G.$-1$
H.$(\frac{251}{107})^{8}$
I.$(\frac{107}{251})^{8}$
\testStop
\kluczStart
A
\kluczStop



\zadStart{Zadanie z Wikieł Z 1.1 d) moja wersja nr 683}

Obliczyć wartość wyrażenia $(\frac{251}{109})^{8} \cdot (\frac{109}{251})^{8} \cdot \pi^{0}$.
\zadStop
\rozwStart{Patryk Wirkus}{Martyna Czarnobaj}
$$(\frac{251}{109})^{8} \cdot (\frac{109}{251})^{8} \cdot \pi^{0} = (\frac{251}{109} \cdot \frac{109}{251})^{8} \cdot 1 = 1^{8} \cdot 1 = 1$$
\rozwStop
\odpStart
$1$
\odpStop
\testStart
A.$1$ B.$\pi$ C.$0$ D.$\frac{251}{109}$ E.$\frac{109}{251}$
F.$-\frac{251}{109}$ G.$-1$
H.$(\frac{251}{109})^{8}$
I.$(\frac{109}{251})^{8}$
\testStop
\kluczStart
A
\kluczStop



\zadStart{Zadanie z Wikieł Z 1.1 d) moja wersja nr 684}

Obliczyć wartość wyrażenia $(\frac{251}{113})^{8} \cdot (\frac{113}{251})^{8} \cdot \pi^{0}$.
\zadStop
\rozwStart{Patryk Wirkus}{Martyna Czarnobaj}
$$(\frac{251}{113})^{8} \cdot (\frac{113}{251})^{8} \cdot \pi^{0} = (\frac{251}{113} \cdot \frac{113}{251})^{8} \cdot 1 = 1^{8} \cdot 1 = 1$$
\rozwStop
\odpStart
$1$
\odpStop
\testStart
A.$1$ B.$\pi$ C.$0$ D.$\frac{251}{113}$ E.$\frac{113}{251}$
F.$-\frac{251}{113}$ G.$-1$
H.$(\frac{251}{113})^{8}$
I.$(\frac{113}{251})^{8}$
\testStop
\kluczStart
A
\kluczStop



\zadStart{Zadanie z Wikieł Z 1.1 d) moja wersja nr 685}

Obliczyć wartość wyrażenia $(\frac{251}{127})^{8} \cdot (\frac{127}{251})^{8} \cdot \pi^{0}$.
\zadStop
\rozwStart{Patryk Wirkus}{Martyna Czarnobaj}
$$(\frac{251}{127})^{8} \cdot (\frac{127}{251})^{8} \cdot \pi^{0} = (\frac{251}{127} \cdot \frac{127}{251})^{8} \cdot 1 = 1^{8} \cdot 1 = 1$$
\rozwStop
\odpStart
$1$
\odpStop
\testStart
A.$1$ B.$\pi$ C.$0$ D.$\frac{251}{127}$ E.$\frac{127}{251}$
F.$-\frac{251}{127}$ G.$-1$
H.$(\frac{251}{127})^{8}$
I.$(\frac{127}{251})^{8}$
\testStop
\kluczStart
A
\kluczStop



\zadStart{Zadanie z Wikieł Z 1.1 d) moja wersja nr 686}

Obliczyć wartość wyrażenia $(\frac{251}{131})^{8} \cdot (\frac{131}{251})^{8} \cdot \pi^{0}$.
\zadStop
\rozwStart{Patryk Wirkus}{Martyna Czarnobaj}
$$(\frac{251}{131})^{8} \cdot (\frac{131}{251})^{8} \cdot \pi^{0} = (\frac{251}{131} \cdot \frac{131}{251})^{8} \cdot 1 = 1^{8} \cdot 1 = 1$$
\rozwStop
\odpStart
$1$
\odpStop
\testStart
A.$1$ B.$\pi$ C.$0$ D.$\frac{251}{131}$ E.$\frac{131}{251}$
F.$-\frac{251}{131}$ G.$-1$
H.$(\frac{251}{131})^{8}$
I.$(\frac{131}{251})^{8}$
\testStop
\kluczStart
A
\kluczStop



\zadStart{Zadanie z Wikieł Z 1.1 d) moja wersja nr 687}

Obliczyć wartość wyrażenia $(\frac{251}{137})^{8} \cdot (\frac{137}{251})^{8} \cdot \pi^{0}$.
\zadStop
\rozwStart{Patryk Wirkus}{Martyna Czarnobaj}
$$(\frac{251}{137})^{8} \cdot (\frac{137}{251})^{8} \cdot \pi^{0} = (\frac{251}{137} \cdot \frac{137}{251})^{8} \cdot 1 = 1^{8} \cdot 1 = 1$$
\rozwStop
\odpStart
$1$
\odpStop
\testStart
A.$1$ B.$\pi$ C.$0$ D.$\frac{251}{137}$ E.$\frac{137}{251}$
F.$-\frac{251}{137}$ G.$-1$
H.$(\frac{251}{137})^{8}$
I.$(\frac{137}{251})^{8}$
\testStop
\kluczStart
A
\kluczStop



\zadStart{Zadanie z Wikieł Z 1.1 d) moja wersja nr 688}

Obliczyć wartość wyrażenia $(\frac{251}{139})^{8} \cdot (\frac{139}{251})^{8} \cdot \pi^{0}$.
\zadStop
\rozwStart{Patryk Wirkus}{Martyna Czarnobaj}
$$(\frac{251}{139})^{8} \cdot (\frac{139}{251})^{8} \cdot \pi^{0} = (\frac{251}{139} \cdot \frac{139}{251})^{8} \cdot 1 = 1^{8} \cdot 1 = 1$$
\rozwStop
\odpStart
$1$
\odpStop
\testStart
A.$1$ B.$\pi$ C.$0$ D.$\frac{251}{139}$ E.$\frac{139}{251}$
F.$-\frac{251}{139}$ G.$-1$
H.$(\frac{251}{139})^{8}$
I.$(\frac{139}{251})^{8}$
\testStop
\kluczStart
A
\kluczStop



\zadStart{Zadanie z Wikieł Z 1.1 d) moja wersja nr 689}

Obliczyć wartość wyrażenia $(\frac{257}{103})^{8} \cdot (\frac{103}{257})^{8} \cdot \pi^{0}$.
\zadStop
\rozwStart{Patryk Wirkus}{Martyna Czarnobaj}
$$(\frac{257}{103})^{8} \cdot (\frac{103}{257})^{8} \cdot \pi^{0} = (\frac{257}{103} \cdot \frac{103}{257})^{8} \cdot 1 = 1^{8} \cdot 1 = 1$$
\rozwStop
\odpStart
$1$
\odpStop
\testStart
A.$1$ B.$\pi$ C.$0$ D.$\frac{257}{103}$ E.$\frac{103}{257}$
F.$-\frac{257}{103}$ G.$-1$
H.$(\frac{257}{103})^{8}$
I.$(\frac{103}{257})^{8}$
\testStop
\kluczStart
A
\kluczStop



\zadStart{Zadanie z Wikieł Z 1.1 d) moja wersja nr 690}

Obliczyć wartość wyrażenia $(\frac{257}{107})^{8} \cdot (\frac{107}{257})^{8} \cdot \pi^{0}$.
\zadStop
\rozwStart{Patryk Wirkus}{Martyna Czarnobaj}
$$(\frac{257}{107})^{8} \cdot (\frac{107}{257})^{8} \cdot \pi^{0} = (\frac{257}{107} \cdot \frac{107}{257})^{8} \cdot 1 = 1^{8} \cdot 1 = 1$$
\rozwStop
\odpStart
$1$
\odpStop
\testStart
A.$1$ B.$\pi$ C.$0$ D.$\frac{257}{107}$ E.$\frac{107}{257}$
F.$-\frac{257}{107}$ G.$-1$
H.$(\frac{257}{107})^{8}$
I.$(\frac{107}{257})^{8}$
\testStop
\kluczStart
A
\kluczStop



\zadStart{Zadanie z Wikieł Z 1.1 d) moja wersja nr 691}

Obliczyć wartość wyrażenia $(\frac{257}{109})^{8} \cdot (\frac{109}{257})^{8} \cdot \pi^{0}$.
\zadStop
\rozwStart{Patryk Wirkus}{Martyna Czarnobaj}
$$(\frac{257}{109})^{8} \cdot (\frac{109}{257})^{8} \cdot \pi^{0} = (\frac{257}{109} \cdot \frac{109}{257})^{8} \cdot 1 = 1^{8} \cdot 1 = 1$$
\rozwStop
\odpStart
$1$
\odpStop
\testStart
A.$1$ B.$\pi$ C.$0$ D.$\frac{257}{109}$ E.$\frac{109}{257}$
F.$-\frac{257}{109}$ G.$-1$
H.$(\frac{257}{109})^{8}$
I.$(\frac{109}{257})^{8}$
\testStop
\kluczStart
A
\kluczStop



\zadStart{Zadanie z Wikieł Z 1.1 d) moja wersja nr 692}

Obliczyć wartość wyrażenia $(\frac{257}{113})^{8} \cdot (\frac{113}{257})^{8} \cdot \pi^{0}$.
\zadStop
\rozwStart{Patryk Wirkus}{Martyna Czarnobaj}
$$(\frac{257}{113})^{8} \cdot (\frac{113}{257})^{8} \cdot \pi^{0} = (\frac{257}{113} \cdot \frac{113}{257})^{8} \cdot 1 = 1^{8} \cdot 1 = 1$$
\rozwStop
\odpStart
$1$
\odpStop
\testStart
A.$1$ B.$\pi$ C.$0$ D.$\frac{257}{113}$ E.$\frac{113}{257}$
F.$-\frac{257}{113}$ G.$-1$
H.$(\frac{257}{113})^{8}$
I.$(\frac{113}{257})^{8}$
\testStop
\kluczStart
A
\kluczStop



\zadStart{Zadanie z Wikieł Z 1.1 d) moja wersja nr 693}

Obliczyć wartość wyrażenia $(\frac{257}{127})^{8} \cdot (\frac{127}{257})^{8} \cdot \pi^{0}$.
\zadStop
\rozwStart{Patryk Wirkus}{Martyna Czarnobaj}
$$(\frac{257}{127})^{8} \cdot (\frac{127}{257})^{8} \cdot \pi^{0} = (\frac{257}{127} \cdot \frac{127}{257})^{8} \cdot 1 = 1^{8} \cdot 1 = 1$$
\rozwStop
\odpStart
$1$
\odpStop
\testStart
A.$1$ B.$\pi$ C.$0$ D.$\frac{257}{127}$ E.$\frac{127}{257}$
F.$-\frac{257}{127}$ G.$-1$
H.$(\frac{257}{127})^{8}$
I.$(\frac{127}{257})^{8}$
\testStop
\kluczStart
A
\kluczStop



\zadStart{Zadanie z Wikieł Z 1.1 d) moja wersja nr 694}

Obliczyć wartość wyrażenia $(\frac{257}{131})^{8} \cdot (\frac{131}{257})^{8} \cdot \pi^{0}$.
\zadStop
\rozwStart{Patryk Wirkus}{Martyna Czarnobaj}
$$(\frac{257}{131})^{8} \cdot (\frac{131}{257})^{8} \cdot \pi^{0} = (\frac{257}{131} \cdot \frac{131}{257})^{8} \cdot 1 = 1^{8} \cdot 1 = 1$$
\rozwStop
\odpStart
$1$
\odpStop
\testStart
A.$1$ B.$\pi$ C.$0$ D.$\frac{257}{131}$ E.$\frac{131}{257}$
F.$-\frac{257}{131}$ G.$-1$
H.$(\frac{257}{131})^{8}$
I.$(\frac{131}{257})^{8}$
\testStop
\kluczStart
A
\kluczStop



\zadStart{Zadanie z Wikieł Z 1.1 d) moja wersja nr 695}

Obliczyć wartość wyrażenia $(\frac{257}{137})^{8} \cdot (\frac{137}{257})^{8} \cdot \pi^{0}$.
\zadStop
\rozwStart{Patryk Wirkus}{Martyna Czarnobaj}
$$(\frac{257}{137})^{8} \cdot (\frac{137}{257})^{8} \cdot \pi^{0} = (\frac{257}{137} \cdot \frac{137}{257})^{8} \cdot 1 = 1^{8} \cdot 1 = 1$$
\rozwStop
\odpStart
$1$
\odpStop
\testStart
A.$1$ B.$\pi$ C.$0$ D.$\frac{257}{137}$ E.$\frac{137}{257}$
F.$-\frac{257}{137}$ G.$-1$
H.$(\frac{257}{137})^{8}$
I.$(\frac{137}{257})^{8}$
\testStop
\kluczStart
A
\kluczStop



\zadStart{Zadanie z Wikieł Z 1.1 d) moja wersja nr 696}

Obliczyć wartość wyrażenia $(\frac{257}{139})^{8} \cdot (\frac{139}{257})^{8} \cdot \pi^{0}$.
\zadStop
\rozwStart{Patryk Wirkus}{Martyna Czarnobaj}
$$(\frac{257}{139})^{8} \cdot (\frac{139}{257})^{8} \cdot \pi^{0} = (\frac{257}{139} \cdot \frac{139}{257})^{8} \cdot 1 = 1^{8} \cdot 1 = 1$$
\rozwStop
\odpStart
$1$
\odpStop
\testStart
A.$1$ B.$\pi$ C.$0$ D.$\frac{257}{139}$ E.$\frac{139}{257}$
F.$-\frac{257}{139}$ G.$-1$
H.$(\frac{257}{139})^{8}$
I.$(\frac{139}{257})^{8}$
\testStop
\kluczStart
A
\kluczStop



\zadStart{Zadanie z Wikieł Z 1.1 d) moja wersja nr 697}

Obliczyć wartość wyrażenia $(\frac{263}{103})^{8} \cdot (\frac{103}{263})^{8} \cdot \pi^{0}$.
\zadStop
\rozwStart{Patryk Wirkus}{Martyna Czarnobaj}
$$(\frac{263}{103})^{8} \cdot (\frac{103}{263})^{8} \cdot \pi^{0} = (\frac{263}{103} \cdot \frac{103}{263})^{8} \cdot 1 = 1^{8} \cdot 1 = 1$$
\rozwStop
\odpStart
$1$
\odpStop
\testStart
A.$1$ B.$\pi$ C.$0$ D.$\frac{263}{103}$ E.$\frac{103}{263}$
F.$-\frac{263}{103}$ G.$-1$
H.$(\frac{263}{103})^{8}$
I.$(\frac{103}{263})^{8}$
\testStop
\kluczStart
A
\kluczStop



\zadStart{Zadanie z Wikieł Z 1.1 d) moja wersja nr 698}

Obliczyć wartość wyrażenia $(\frac{263}{107})^{8} \cdot (\frac{107}{263})^{8} \cdot \pi^{0}$.
\zadStop
\rozwStart{Patryk Wirkus}{Martyna Czarnobaj}
$$(\frac{263}{107})^{8} \cdot (\frac{107}{263})^{8} \cdot \pi^{0} = (\frac{263}{107} \cdot \frac{107}{263})^{8} \cdot 1 = 1^{8} \cdot 1 = 1$$
\rozwStop
\odpStart
$1$
\odpStop
\testStart
A.$1$ B.$\pi$ C.$0$ D.$\frac{263}{107}$ E.$\frac{107}{263}$
F.$-\frac{263}{107}$ G.$-1$
H.$(\frac{263}{107})^{8}$
I.$(\frac{107}{263})^{8}$
\testStop
\kluczStart
A
\kluczStop



\zadStart{Zadanie z Wikieł Z 1.1 d) moja wersja nr 699}

Obliczyć wartość wyrażenia $(\frac{263}{109})^{8} \cdot (\frac{109}{263})^{8} \cdot \pi^{0}$.
\zadStop
\rozwStart{Patryk Wirkus}{Martyna Czarnobaj}
$$(\frac{263}{109})^{8} \cdot (\frac{109}{263})^{8} \cdot \pi^{0} = (\frac{263}{109} \cdot \frac{109}{263})^{8} \cdot 1 = 1^{8} \cdot 1 = 1$$
\rozwStop
\odpStart
$1$
\odpStop
\testStart
A.$1$ B.$\pi$ C.$0$ D.$\frac{263}{109}$ E.$\frac{109}{263}$
F.$-\frac{263}{109}$ G.$-1$
H.$(\frac{263}{109})^{8}$
I.$(\frac{109}{263})^{8}$
\testStop
\kluczStart
A
\kluczStop



\zadStart{Zadanie z Wikieł Z 1.1 d) moja wersja nr 700}

Obliczyć wartość wyrażenia $(\frac{263}{113})^{8} \cdot (\frac{113}{263})^{8} \cdot \pi^{0}$.
\zadStop
\rozwStart{Patryk Wirkus}{Martyna Czarnobaj}
$$(\frac{263}{113})^{8} \cdot (\frac{113}{263})^{8} \cdot \pi^{0} = (\frac{263}{113} \cdot \frac{113}{263})^{8} \cdot 1 = 1^{8} \cdot 1 = 1$$
\rozwStop
\odpStart
$1$
\odpStop
\testStart
A.$1$ B.$\pi$ C.$0$ D.$\frac{263}{113}$ E.$\frac{113}{263}$
F.$-\frac{263}{113}$ G.$-1$
H.$(\frac{263}{113})^{8}$
I.$(\frac{113}{263})^{8}$
\testStop
\kluczStart
A
\kluczStop



\zadStart{Zadanie z Wikieł Z 1.1 d) moja wersja nr 701}

Obliczyć wartość wyrażenia $(\frac{263}{127})^{8} \cdot (\frac{127}{263})^{8} \cdot \pi^{0}$.
\zadStop
\rozwStart{Patryk Wirkus}{Martyna Czarnobaj}
$$(\frac{263}{127})^{8} \cdot (\frac{127}{263})^{8} \cdot \pi^{0} = (\frac{263}{127} \cdot \frac{127}{263})^{8} \cdot 1 = 1^{8} \cdot 1 = 1$$
\rozwStop
\odpStart
$1$
\odpStop
\testStart
A.$1$ B.$\pi$ C.$0$ D.$\frac{263}{127}$ E.$\frac{127}{263}$
F.$-\frac{263}{127}$ G.$-1$
H.$(\frac{263}{127})^{8}$
I.$(\frac{127}{263})^{8}$
\testStop
\kluczStart
A
\kluczStop



\zadStart{Zadanie z Wikieł Z 1.1 d) moja wersja nr 702}

Obliczyć wartość wyrażenia $(\frac{263}{131})^{8} \cdot (\frac{131}{263})^{8} \cdot \pi^{0}$.
\zadStop
\rozwStart{Patryk Wirkus}{Martyna Czarnobaj}
$$(\frac{263}{131})^{8} \cdot (\frac{131}{263})^{8} \cdot \pi^{0} = (\frac{263}{131} \cdot \frac{131}{263})^{8} \cdot 1 = 1^{8} \cdot 1 = 1$$
\rozwStop
\odpStart
$1$
\odpStop
\testStart
A.$1$ B.$\pi$ C.$0$ D.$\frac{263}{131}$ E.$\frac{131}{263}$
F.$-\frac{263}{131}$ G.$-1$
H.$(\frac{263}{131})^{8}$
I.$(\frac{131}{263})^{8}$
\testStop
\kluczStart
A
\kluczStop



\zadStart{Zadanie z Wikieł Z 1.1 d) moja wersja nr 703}

Obliczyć wartość wyrażenia $(\frac{263}{137})^{8} \cdot (\frac{137}{263})^{8} \cdot \pi^{0}$.
\zadStop
\rozwStart{Patryk Wirkus}{Martyna Czarnobaj}
$$(\frac{263}{137})^{8} \cdot (\frac{137}{263})^{8} \cdot \pi^{0} = (\frac{263}{137} \cdot \frac{137}{263})^{8} \cdot 1 = 1^{8} \cdot 1 = 1$$
\rozwStop
\odpStart
$1$
\odpStop
\testStart
A.$1$ B.$\pi$ C.$0$ D.$\frac{263}{137}$ E.$\frac{137}{263}$
F.$-\frac{263}{137}$ G.$-1$
H.$(\frac{263}{137})^{8}$
I.$(\frac{137}{263})^{8}$
\testStop
\kluczStart
A
\kluczStop



\zadStart{Zadanie z Wikieł Z 1.1 d) moja wersja nr 704}

Obliczyć wartość wyrażenia $(\frac{263}{139})^{8} \cdot (\frac{139}{263})^{8} \cdot \pi^{0}$.
\zadStop
\rozwStart{Patryk Wirkus}{Martyna Czarnobaj}
$$(\frac{263}{139})^{8} \cdot (\frac{139}{263})^{8} \cdot \pi^{0} = (\frac{263}{139} \cdot \frac{139}{263})^{8} \cdot 1 = 1^{8} \cdot 1 = 1$$
\rozwStop
\odpStart
$1$
\odpStop
\testStart
A.$1$ B.$\pi$ C.$0$ D.$\frac{263}{139}$ E.$\frac{139}{263}$
F.$-\frac{263}{139}$ G.$-1$
H.$(\frac{263}{139})^{8}$
I.$(\frac{139}{263})^{8}$
\testStop
\kluczStart
A
\kluczStop



\zadStart{Zadanie z Wikieł Z 1.1 d) moja wersja nr 705}

Obliczyć wartość wyrażenia $(\frac{269}{103})^{8} \cdot (\frac{103}{269})^{8} \cdot \pi^{0}$.
\zadStop
\rozwStart{Patryk Wirkus}{Martyna Czarnobaj}
$$(\frac{269}{103})^{8} \cdot (\frac{103}{269})^{8} \cdot \pi^{0} = (\frac{269}{103} \cdot \frac{103}{269})^{8} \cdot 1 = 1^{8} \cdot 1 = 1$$
\rozwStop
\odpStart
$1$
\odpStop
\testStart
A.$1$ B.$\pi$ C.$0$ D.$\frac{269}{103}$ E.$\frac{103}{269}$
F.$-\frac{269}{103}$ G.$-1$
H.$(\frac{269}{103})^{8}$
I.$(\frac{103}{269})^{8}$
\testStop
\kluczStart
A
\kluczStop



\zadStart{Zadanie z Wikieł Z 1.1 d) moja wersja nr 706}

Obliczyć wartość wyrażenia $(\frac{269}{107})^{8} \cdot (\frac{107}{269})^{8} \cdot \pi^{0}$.
\zadStop
\rozwStart{Patryk Wirkus}{Martyna Czarnobaj}
$$(\frac{269}{107})^{8} \cdot (\frac{107}{269})^{8} \cdot \pi^{0} = (\frac{269}{107} \cdot \frac{107}{269})^{8} \cdot 1 = 1^{8} \cdot 1 = 1$$
\rozwStop
\odpStart
$1$
\odpStop
\testStart
A.$1$ B.$\pi$ C.$0$ D.$\frac{269}{107}$ E.$\frac{107}{269}$
F.$-\frac{269}{107}$ G.$-1$
H.$(\frac{269}{107})^{8}$
I.$(\frac{107}{269})^{8}$
\testStop
\kluczStart
A
\kluczStop



\zadStart{Zadanie z Wikieł Z 1.1 d) moja wersja nr 707}

Obliczyć wartość wyrażenia $(\frac{269}{109})^{8} \cdot (\frac{109}{269})^{8} \cdot \pi^{0}$.
\zadStop
\rozwStart{Patryk Wirkus}{Martyna Czarnobaj}
$$(\frac{269}{109})^{8} \cdot (\frac{109}{269})^{8} \cdot \pi^{0} = (\frac{269}{109} \cdot \frac{109}{269})^{8} \cdot 1 = 1^{8} \cdot 1 = 1$$
\rozwStop
\odpStart
$1$
\odpStop
\testStart
A.$1$ B.$\pi$ C.$0$ D.$\frac{269}{109}$ E.$\frac{109}{269}$
F.$-\frac{269}{109}$ G.$-1$
H.$(\frac{269}{109})^{8}$
I.$(\frac{109}{269})^{8}$
\testStop
\kluczStart
A
\kluczStop



\zadStart{Zadanie z Wikieł Z 1.1 d) moja wersja nr 708}

Obliczyć wartość wyrażenia $(\frac{269}{113})^{8} \cdot (\frac{113}{269})^{8} \cdot \pi^{0}$.
\zadStop
\rozwStart{Patryk Wirkus}{Martyna Czarnobaj}
$$(\frac{269}{113})^{8} \cdot (\frac{113}{269})^{8} \cdot \pi^{0} = (\frac{269}{113} \cdot \frac{113}{269})^{8} \cdot 1 = 1^{8} \cdot 1 = 1$$
\rozwStop
\odpStart
$1$
\odpStop
\testStart
A.$1$ B.$\pi$ C.$0$ D.$\frac{269}{113}$ E.$\frac{113}{269}$
F.$-\frac{269}{113}$ G.$-1$
H.$(\frac{269}{113})^{8}$
I.$(\frac{113}{269})^{8}$
\testStop
\kluczStart
A
\kluczStop



\zadStart{Zadanie z Wikieł Z 1.1 d) moja wersja nr 709}

Obliczyć wartość wyrażenia $(\frac{269}{127})^{8} \cdot (\frac{127}{269})^{8} \cdot \pi^{0}$.
\zadStop
\rozwStart{Patryk Wirkus}{Martyna Czarnobaj}
$$(\frac{269}{127})^{8} \cdot (\frac{127}{269})^{8} \cdot \pi^{0} = (\frac{269}{127} \cdot \frac{127}{269})^{8} \cdot 1 = 1^{8} \cdot 1 = 1$$
\rozwStop
\odpStart
$1$
\odpStop
\testStart
A.$1$ B.$\pi$ C.$0$ D.$\frac{269}{127}$ E.$\frac{127}{269}$
F.$-\frac{269}{127}$ G.$-1$
H.$(\frac{269}{127})^{8}$
I.$(\frac{127}{269})^{8}$
\testStop
\kluczStart
A
\kluczStop



\zadStart{Zadanie z Wikieł Z 1.1 d) moja wersja nr 710}

Obliczyć wartość wyrażenia $(\frac{269}{131})^{8} \cdot (\frac{131}{269})^{8} \cdot \pi^{0}$.
\zadStop
\rozwStart{Patryk Wirkus}{Martyna Czarnobaj}
$$(\frac{269}{131})^{8} \cdot (\frac{131}{269})^{8} \cdot \pi^{0} = (\frac{269}{131} \cdot \frac{131}{269})^{8} \cdot 1 = 1^{8} \cdot 1 = 1$$
\rozwStop
\odpStart
$1$
\odpStop
\testStart
A.$1$ B.$\pi$ C.$0$ D.$\frac{269}{131}$ E.$\frac{131}{269}$
F.$-\frac{269}{131}$ G.$-1$
H.$(\frac{269}{131})^{8}$
I.$(\frac{131}{269})^{8}$
\testStop
\kluczStart
A
\kluczStop



\zadStart{Zadanie z Wikieł Z 1.1 d) moja wersja nr 711}

Obliczyć wartość wyrażenia $(\frac{269}{137})^{8} \cdot (\frac{137}{269})^{8} \cdot \pi^{0}$.
\zadStop
\rozwStart{Patryk Wirkus}{Martyna Czarnobaj}
$$(\frac{269}{137})^{8} \cdot (\frac{137}{269})^{8} \cdot \pi^{0} = (\frac{269}{137} \cdot \frac{137}{269})^{8} \cdot 1 = 1^{8} \cdot 1 = 1$$
\rozwStop
\odpStart
$1$
\odpStop
\testStart
A.$1$ B.$\pi$ C.$0$ D.$\frac{269}{137}$ E.$\frac{137}{269}$
F.$-\frac{269}{137}$ G.$-1$
H.$(\frac{269}{137})^{8}$
I.$(\frac{137}{269})^{8}$
\testStop
\kluczStart
A
\kluczStop



\zadStart{Zadanie z Wikieł Z 1.1 d) moja wersja nr 712}

Obliczyć wartość wyrażenia $(\frac{269}{139})^{8} \cdot (\frac{139}{269})^{8} \cdot \pi^{0}$.
\zadStop
\rozwStart{Patryk Wirkus}{Martyna Czarnobaj}
$$(\frac{269}{139})^{8} \cdot (\frac{139}{269})^{8} \cdot \pi^{0} = (\frac{269}{139} \cdot \frac{139}{269})^{8} \cdot 1 = 1^{8} \cdot 1 = 1$$
\rozwStop
\odpStart
$1$
\odpStop
\testStart
A.$1$ B.$\pi$ C.$0$ D.$\frac{269}{139}$ E.$\frac{139}{269}$
F.$-\frac{269}{139}$ G.$-1$
H.$(\frac{269}{139})^{8}$
I.$(\frac{139}{269})^{8}$
\testStop
\kluczStart
A
\kluczStop



\zadStart{Zadanie z Wikieł Z 1.1 d) moja wersja nr 713}

Obliczyć wartość wyrażenia $(\frac{271}{103})^{8} \cdot (\frac{103}{271})^{8} \cdot \pi^{0}$.
\zadStop
\rozwStart{Patryk Wirkus}{Martyna Czarnobaj}
$$(\frac{271}{103})^{8} \cdot (\frac{103}{271})^{8} \cdot \pi^{0} = (\frac{271}{103} \cdot \frac{103}{271})^{8} \cdot 1 = 1^{8} \cdot 1 = 1$$
\rozwStop
\odpStart
$1$
\odpStop
\testStart
A.$1$ B.$\pi$ C.$0$ D.$\frac{271}{103}$ E.$\frac{103}{271}$
F.$-\frac{271}{103}$ G.$-1$
H.$(\frac{271}{103})^{8}$
I.$(\frac{103}{271})^{8}$
\testStop
\kluczStart
A
\kluczStop



\zadStart{Zadanie z Wikieł Z 1.1 d) moja wersja nr 714}

Obliczyć wartość wyrażenia $(\frac{271}{107})^{8} \cdot (\frac{107}{271})^{8} \cdot \pi^{0}$.
\zadStop
\rozwStart{Patryk Wirkus}{Martyna Czarnobaj}
$$(\frac{271}{107})^{8} \cdot (\frac{107}{271})^{8} \cdot \pi^{0} = (\frac{271}{107} \cdot \frac{107}{271})^{8} \cdot 1 = 1^{8} \cdot 1 = 1$$
\rozwStop
\odpStart
$1$
\odpStop
\testStart
A.$1$ B.$\pi$ C.$0$ D.$\frac{271}{107}$ E.$\frac{107}{271}$
F.$-\frac{271}{107}$ G.$-1$
H.$(\frac{271}{107})^{8}$
I.$(\frac{107}{271})^{8}$
\testStop
\kluczStart
A
\kluczStop



\zadStart{Zadanie z Wikieł Z 1.1 d) moja wersja nr 715}

Obliczyć wartość wyrażenia $(\frac{271}{109})^{8} \cdot (\frac{109}{271})^{8} \cdot \pi^{0}$.
\zadStop
\rozwStart{Patryk Wirkus}{Martyna Czarnobaj}
$$(\frac{271}{109})^{8} \cdot (\frac{109}{271})^{8} \cdot \pi^{0} = (\frac{271}{109} \cdot \frac{109}{271})^{8} \cdot 1 = 1^{8} \cdot 1 = 1$$
\rozwStop
\odpStart
$1$
\odpStop
\testStart
A.$1$ B.$\pi$ C.$0$ D.$\frac{271}{109}$ E.$\frac{109}{271}$
F.$-\frac{271}{109}$ G.$-1$
H.$(\frac{271}{109})^{8}$
I.$(\frac{109}{271})^{8}$
\testStop
\kluczStart
A
\kluczStop



\zadStart{Zadanie z Wikieł Z 1.1 d) moja wersja nr 716}

Obliczyć wartość wyrażenia $(\frac{271}{113})^{8} \cdot (\frac{113}{271})^{8} \cdot \pi^{0}$.
\zadStop
\rozwStart{Patryk Wirkus}{Martyna Czarnobaj}
$$(\frac{271}{113})^{8} \cdot (\frac{113}{271})^{8} \cdot \pi^{0} = (\frac{271}{113} \cdot \frac{113}{271})^{8} \cdot 1 = 1^{8} \cdot 1 = 1$$
\rozwStop
\odpStart
$1$
\odpStop
\testStart
A.$1$ B.$\pi$ C.$0$ D.$\frac{271}{113}$ E.$\frac{113}{271}$
F.$-\frac{271}{113}$ G.$-1$
H.$(\frac{271}{113})^{8}$
I.$(\frac{113}{271})^{8}$
\testStop
\kluczStart
A
\kluczStop



\zadStart{Zadanie z Wikieł Z 1.1 d) moja wersja nr 717}

Obliczyć wartość wyrażenia $(\frac{271}{127})^{8} \cdot (\frac{127}{271})^{8} \cdot \pi^{0}$.
\zadStop
\rozwStart{Patryk Wirkus}{Martyna Czarnobaj}
$$(\frac{271}{127})^{8} \cdot (\frac{127}{271})^{8} \cdot \pi^{0} = (\frac{271}{127} \cdot \frac{127}{271})^{8} \cdot 1 = 1^{8} \cdot 1 = 1$$
\rozwStop
\odpStart
$1$
\odpStop
\testStart
A.$1$ B.$\pi$ C.$0$ D.$\frac{271}{127}$ E.$\frac{127}{271}$
F.$-\frac{271}{127}$ G.$-1$
H.$(\frac{271}{127})^{8}$
I.$(\frac{127}{271})^{8}$
\testStop
\kluczStart
A
\kluczStop



\zadStart{Zadanie z Wikieł Z 1.1 d) moja wersja nr 718}

Obliczyć wartość wyrażenia $(\frac{271}{131})^{8} \cdot (\frac{131}{271})^{8} \cdot \pi^{0}$.
\zadStop
\rozwStart{Patryk Wirkus}{Martyna Czarnobaj}
$$(\frac{271}{131})^{8} \cdot (\frac{131}{271})^{8} \cdot \pi^{0} = (\frac{271}{131} \cdot \frac{131}{271})^{8} \cdot 1 = 1^{8} \cdot 1 = 1$$
\rozwStop
\odpStart
$1$
\odpStop
\testStart
A.$1$ B.$\pi$ C.$0$ D.$\frac{271}{131}$ E.$\frac{131}{271}$
F.$-\frac{271}{131}$ G.$-1$
H.$(\frac{271}{131})^{8}$
I.$(\frac{131}{271})^{8}$
\testStop
\kluczStart
A
\kluczStop



\zadStart{Zadanie z Wikieł Z 1.1 d) moja wersja nr 719}

Obliczyć wartość wyrażenia $(\frac{271}{137})^{8} \cdot (\frac{137}{271})^{8} \cdot \pi^{0}$.
\zadStop
\rozwStart{Patryk Wirkus}{Martyna Czarnobaj}
$$(\frac{271}{137})^{8} \cdot (\frac{137}{271})^{8} \cdot \pi^{0} = (\frac{271}{137} \cdot \frac{137}{271})^{8} \cdot 1 = 1^{8} \cdot 1 = 1$$
\rozwStop
\odpStart
$1$
\odpStop
\testStart
A.$1$ B.$\pi$ C.$0$ D.$\frac{271}{137}$ E.$\frac{137}{271}$
F.$-\frac{271}{137}$ G.$-1$
H.$(\frac{271}{137})^{8}$
I.$(\frac{137}{271})^{8}$
\testStop
\kluczStart
A
\kluczStop



\zadStart{Zadanie z Wikieł Z 1.1 d) moja wersja nr 720}

Obliczyć wartość wyrażenia $(\frac{271}{139})^{8} \cdot (\frac{139}{271})^{8} \cdot \pi^{0}$.
\zadStop
\rozwStart{Patryk Wirkus}{Martyna Czarnobaj}
$$(\frac{271}{139})^{8} \cdot (\frac{139}{271})^{8} \cdot \pi^{0} = (\frac{271}{139} \cdot \frac{139}{271})^{8} \cdot 1 = 1^{8} \cdot 1 = 1$$
\rozwStop
\odpStart
$1$
\odpStop
\testStart
A.$1$ B.$\pi$ C.$0$ D.$\frac{271}{139}$ E.$\frac{139}{271}$
F.$-\frac{271}{139}$ G.$-1$
H.$(\frac{271}{139})^{8}$
I.$(\frac{139}{271})^{8}$
\testStop
\kluczStart
A
\kluczStop



\zadStart{Zadanie z Wikieł Z 1.1 d) moja wersja nr 721}

Obliczyć wartość wyrażenia $(\frac{277}{103})^{8} \cdot (\frac{103}{277})^{8} \cdot \pi^{0}$.
\zadStop
\rozwStart{Patryk Wirkus}{Martyna Czarnobaj}
$$(\frac{277}{103})^{8} \cdot (\frac{103}{277})^{8} \cdot \pi^{0} = (\frac{277}{103} \cdot \frac{103}{277})^{8} \cdot 1 = 1^{8} \cdot 1 = 1$$
\rozwStop
\odpStart
$1$
\odpStop
\testStart
A.$1$ B.$\pi$ C.$0$ D.$\frac{277}{103}$ E.$\frac{103}{277}$
F.$-\frac{277}{103}$ G.$-1$
H.$(\frac{277}{103})^{8}$
I.$(\frac{103}{277})^{8}$
\testStop
\kluczStart
A
\kluczStop



\zadStart{Zadanie z Wikieł Z 1.1 d) moja wersja nr 722}

Obliczyć wartość wyrażenia $(\frac{277}{107})^{8} \cdot (\frac{107}{277})^{8} \cdot \pi^{0}$.
\zadStop
\rozwStart{Patryk Wirkus}{Martyna Czarnobaj}
$$(\frac{277}{107})^{8} \cdot (\frac{107}{277})^{8} \cdot \pi^{0} = (\frac{277}{107} \cdot \frac{107}{277})^{8} \cdot 1 = 1^{8} \cdot 1 = 1$$
\rozwStop
\odpStart
$1$
\odpStop
\testStart
A.$1$ B.$\pi$ C.$0$ D.$\frac{277}{107}$ E.$\frac{107}{277}$
F.$-\frac{277}{107}$ G.$-1$
H.$(\frac{277}{107})^{8}$
I.$(\frac{107}{277})^{8}$
\testStop
\kluczStart
A
\kluczStop



\zadStart{Zadanie z Wikieł Z 1.1 d) moja wersja nr 723}

Obliczyć wartość wyrażenia $(\frac{277}{109})^{8} \cdot (\frac{109}{277})^{8} \cdot \pi^{0}$.
\zadStop
\rozwStart{Patryk Wirkus}{Martyna Czarnobaj}
$$(\frac{277}{109})^{8} \cdot (\frac{109}{277})^{8} \cdot \pi^{0} = (\frac{277}{109} \cdot \frac{109}{277})^{8} \cdot 1 = 1^{8} \cdot 1 = 1$$
\rozwStop
\odpStart
$1$
\odpStop
\testStart
A.$1$ B.$\pi$ C.$0$ D.$\frac{277}{109}$ E.$\frac{109}{277}$
F.$-\frac{277}{109}$ G.$-1$
H.$(\frac{277}{109})^{8}$
I.$(\frac{109}{277})^{8}$
\testStop
\kluczStart
A
\kluczStop



\zadStart{Zadanie z Wikieł Z 1.1 d) moja wersja nr 724}

Obliczyć wartość wyrażenia $(\frac{277}{113})^{8} \cdot (\frac{113}{277})^{8} \cdot \pi^{0}$.
\zadStop
\rozwStart{Patryk Wirkus}{Martyna Czarnobaj}
$$(\frac{277}{113})^{8} \cdot (\frac{113}{277})^{8} \cdot \pi^{0} = (\frac{277}{113} \cdot \frac{113}{277})^{8} \cdot 1 = 1^{8} \cdot 1 = 1$$
\rozwStop
\odpStart
$1$
\odpStop
\testStart
A.$1$ B.$\pi$ C.$0$ D.$\frac{277}{113}$ E.$\frac{113}{277}$
F.$-\frac{277}{113}$ G.$-1$
H.$(\frac{277}{113})^{8}$
I.$(\frac{113}{277})^{8}$
\testStop
\kluczStart
A
\kluczStop



\zadStart{Zadanie z Wikieł Z 1.1 d) moja wersja nr 725}

Obliczyć wartość wyrażenia $(\frac{277}{127})^{8} \cdot (\frac{127}{277})^{8} \cdot \pi^{0}$.
\zadStop
\rozwStart{Patryk Wirkus}{Martyna Czarnobaj}
$$(\frac{277}{127})^{8} \cdot (\frac{127}{277})^{8} \cdot \pi^{0} = (\frac{277}{127} \cdot \frac{127}{277})^{8} \cdot 1 = 1^{8} \cdot 1 = 1$$
\rozwStop
\odpStart
$1$
\odpStop
\testStart
A.$1$ B.$\pi$ C.$0$ D.$\frac{277}{127}$ E.$\frac{127}{277}$
F.$-\frac{277}{127}$ G.$-1$
H.$(\frac{277}{127})^{8}$
I.$(\frac{127}{277})^{8}$
\testStop
\kluczStart
A
\kluczStop



\zadStart{Zadanie z Wikieł Z 1.1 d) moja wersja nr 726}

Obliczyć wartość wyrażenia $(\frac{277}{131})^{8} \cdot (\frac{131}{277})^{8} \cdot \pi^{0}$.
\zadStop
\rozwStart{Patryk Wirkus}{Martyna Czarnobaj}
$$(\frac{277}{131})^{8} \cdot (\frac{131}{277})^{8} \cdot \pi^{0} = (\frac{277}{131} \cdot \frac{131}{277})^{8} \cdot 1 = 1^{8} \cdot 1 = 1$$
\rozwStop
\odpStart
$1$
\odpStop
\testStart
A.$1$ B.$\pi$ C.$0$ D.$\frac{277}{131}$ E.$\frac{131}{277}$
F.$-\frac{277}{131}$ G.$-1$
H.$(\frac{277}{131})^{8}$
I.$(\frac{131}{277})^{8}$
\testStop
\kluczStart
A
\kluczStop



\zadStart{Zadanie z Wikieł Z 1.1 d) moja wersja nr 727}

Obliczyć wartość wyrażenia $(\frac{277}{137})^{8} \cdot (\frac{137}{277})^{8} \cdot \pi^{0}$.
\zadStop
\rozwStart{Patryk Wirkus}{Martyna Czarnobaj}
$$(\frac{277}{137})^{8} \cdot (\frac{137}{277})^{8} \cdot \pi^{0} = (\frac{277}{137} \cdot \frac{137}{277})^{8} \cdot 1 = 1^{8} \cdot 1 = 1$$
\rozwStop
\odpStart
$1$
\odpStop
\testStart
A.$1$ B.$\pi$ C.$0$ D.$\frac{277}{137}$ E.$\frac{137}{277}$
F.$-\frac{277}{137}$ G.$-1$
H.$(\frac{277}{137})^{8}$
I.$(\frac{137}{277})^{8}$
\testStop
\kluczStart
A
\kluczStop



\zadStart{Zadanie z Wikieł Z 1.1 d) moja wersja nr 728}

Obliczyć wartość wyrażenia $(\frac{277}{139})^{8} \cdot (\frac{139}{277})^{8} \cdot \pi^{0}$.
\zadStop
\rozwStart{Patryk Wirkus}{Martyna Czarnobaj}
$$(\frac{277}{139})^{8} \cdot (\frac{139}{277})^{8} \cdot \pi^{0} = (\frac{277}{139} \cdot \frac{139}{277})^{8} \cdot 1 = 1^{8} \cdot 1 = 1$$
\rozwStop
\odpStart
$1$
\odpStop
\testStart
A.$1$ B.$\pi$ C.$0$ D.$\frac{277}{139}$ E.$\frac{139}{277}$
F.$-\frac{277}{139}$ G.$-1$
H.$(\frac{277}{139})^{8}$
I.$(\frac{139}{277})^{8}$
\testStop
\kluczStart
A
\kluczStop



\zadStart{Zadanie z Wikieł Z 1.1 d) moja wersja nr 729}

Obliczyć wartość wyrażenia $(\frac{149}{103})^{9} \cdot (\frac{103}{149})^{9} \cdot \pi^{0}$.
\zadStop
\rozwStart{Patryk Wirkus}{Martyna Czarnobaj}
$$(\frac{149}{103})^{9} \cdot (\frac{103}{149})^{9} \cdot \pi^{0} = (\frac{149}{103} \cdot \frac{103}{149})^{9} \cdot 1 = 1^{9} \cdot 1 = 1$$
\rozwStop
\odpStart
$1$
\odpStop
\testStart
A.$1$ B.$\pi$ C.$0$ D.$\frac{149}{103}$ E.$\frac{103}{149}$
F.$-\frac{149}{103}$ G.$-1$
H.$(\frac{149}{103})^{9}$
I.$(\frac{103}{149})^{9}$
\testStop
\kluczStart
A
\kluczStop



\zadStart{Zadanie z Wikieł Z 1.1 d) moja wersja nr 730}

Obliczyć wartość wyrażenia $(\frac{149}{107})^{9} \cdot (\frac{107}{149})^{9} \cdot \pi^{0}$.
\zadStop
\rozwStart{Patryk Wirkus}{Martyna Czarnobaj}
$$(\frac{149}{107})^{9} \cdot (\frac{107}{149})^{9} \cdot \pi^{0} = (\frac{149}{107} \cdot \frac{107}{149})^{9} \cdot 1 = 1^{9} \cdot 1 = 1$$
\rozwStop
\odpStart
$1$
\odpStop
\testStart
A.$1$ B.$\pi$ C.$0$ D.$\frac{149}{107}$ E.$\frac{107}{149}$
F.$-\frac{149}{107}$ G.$-1$
H.$(\frac{149}{107})^{9}$
I.$(\frac{107}{149})^{9}$
\testStop
\kluczStart
A
\kluczStop



\zadStart{Zadanie z Wikieł Z 1.1 d) moja wersja nr 731}

Obliczyć wartość wyrażenia $(\frac{149}{109})^{9} \cdot (\frac{109}{149})^{9} \cdot \pi^{0}$.
\zadStop
\rozwStart{Patryk Wirkus}{Martyna Czarnobaj}
$$(\frac{149}{109})^{9} \cdot (\frac{109}{149})^{9} \cdot \pi^{0} = (\frac{149}{109} \cdot \frac{109}{149})^{9} \cdot 1 = 1^{9} \cdot 1 = 1$$
\rozwStop
\odpStart
$1$
\odpStop
\testStart
A.$1$ B.$\pi$ C.$0$ D.$\frac{149}{109}$ E.$\frac{109}{149}$
F.$-\frac{149}{109}$ G.$-1$
H.$(\frac{149}{109})^{9}$
I.$(\frac{109}{149})^{9}$
\testStop
\kluczStart
A
\kluczStop



\zadStart{Zadanie z Wikieł Z 1.1 d) moja wersja nr 732}

Obliczyć wartość wyrażenia $(\frac{149}{113})^{9} \cdot (\frac{113}{149})^{9} \cdot \pi^{0}$.
\zadStop
\rozwStart{Patryk Wirkus}{Martyna Czarnobaj}
$$(\frac{149}{113})^{9} \cdot (\frac{113}{149})^{9} \cdot \pi^{0} = (\frac{149}{113} \cdot \frac{113}{149})^{9} \cdot 1 = 1^{9} \cdot 1 = 1$$
\rozwStop
\odpStart
$1$
\odpStop
\testStart
A.$1$ B.$\pi$ C.$0$ D.$\frac{149}{113}$ E.$\frac{113}{149}$
F.$-\frac{149}{113}$ G.$-1$
H.$(\frac{149}{113})^{9}$
I.$(\frac{113}{149})^{9}$
\testStop
\kluczStart
A
\kluczStop



\zadStart{Zadanie z Wikieł Z 1.1 d) moja wersja nr 733}

Obliczyć wartość wyrażenia $(\frac{149}{127})^{9} \cdot (\frac{127}{149})^{9} \cdot \pi^{0}$.
\zadStop
\rozwStart{Patryk Wirkus}{Martyna Czarnobaj}
$$(\frac{149}{127})^{9} \cdot (\frac{127}{149})^{9} \cdot \pi^{0} = (\frac{149}{127} \cdot \frac{127}{149})^{9} \cdot 1 = 1^{9} \cdot 1 = 1$$
\rozwStop
\odpStart
$1$
\odpStop
\testStart
A.$1$ B.$\pi$ C.$0$ D.$\frac{149}{127}$ E.$\frac{127}{149}$
F.$-\frac{149}{127}$ G.$-1$
H.$(\frac{149}{127})^{9}$
I.$(\frac{127}{149})^{9}$
\testStop
\kluczStart
A
\kluczStop



\zadStart{Zadanie z Wikieł Z 1.1 d) moja wersja nr 734}

Obliczyć wartość wyrażenia $(\frac{149}{131})^{9} \cdot (\frac{131}{149})^{9} \cdot \pi^{0}$.
\zadStop
\rozwStart{Patryk Wirkus}{Martyna Czarnobaj}
$$(\frac{149}{131})^{9} \cdot (\frac{131}{149})^{9} \cdot \pi^{0} = (\frac{149}{131} \cdot \frac{131}{149})^{9} \cdot 1 = 1^{9} \cdot 1 = 1$$
\rozwStop
\odpStart
$1$
\odpStop
\testStart
A.$1$ B.$\pi$ C.$0$ D.$\frac{149}{131}$ E.$\frac{131}{149}$
F.$-\frac{149}{131}$ G.$-1$
H.$(\frac{149}{131})^{9}$
I.$(\frac{131}{149})^{9}$
\testStop
\kluczStart
A
\kluczStop



\zadStart{Zadanie z Wikieł Z 1.1 d) moja wersja nr 735}

Obliczyć wartość wyrażenia $(\frac{149}{137})^{9} \cdot (\frac{137}{149})^{9} \cdot \pi^{0}$.
\zadStop
\rozwStart{Patryk Wirkus}{Martyna Czarnobaj}
$$(\frac{149}{137})^{9} \cdot (\frac{137}{149})^{9} \cdot \pi^{0} = (\frac{149}{137} \cdot \frac{137}{149})^{9} \cdot 1 = 1^{9} \cdot 1 = 1$$
\rozwStop
\odpStart
$1$
\odpStop
\testStart
A.$1$ B.$\pi$ C.$0$ D.$\frac{149}{137}$ E.$\frac{137}{149}$
F.$-\frac{149}{137}$ G.$-1$
H.$(\frac{149}{137})^{9}$
I.$(\frac{137}{149})^{9}$
\testStop
\kluczStart
A
\kluczStop



\zadStart{Zadanie z Wikieł Z 1.1 d) moja wersja nr 736}

Obliczyć wartość wyrażenia $(\frac{149}{139})^{9} \cdot (\frac{139}{149})^{9} \cdot \pi^{0}$.
\zadStop
\rozwStart{Patryk Wirkus}{Martyna Czarnobaj}
$$(\frac{149}{139})^{9} \cdot (\frac{139}{149})^{9} \cdot \pi^{0} = (\frac{149}{139} \cdot \frac{139}{149})^{9} \cdot 1 = 1^{9} \cdot 1 = 1$$
\rozwStop
\odpStart
$1$
\odpStop
\testStart
A.$1$ B.$\pi$ C.$0$ D.$\frac{149}{139}$ E.$\frac{139}{149}$
F.$-\frac{149}{139}$ G.$-1$
H.$(\frac{149}{139})^{9}$
I.$(\frac{139}{149})^{9}$
\testStop
\kluczStart
A
\kluczStop



\zadStart{Zadanie z Wikieł Z 1.1 d) moja wersja nr 737}

Obliczyć wartość wyrażenia $(\frac{151}{103})^{9} \cdot (\frac{103}{151})^{9} \cdot \pi^{0}$.
\zadStop
\rozwStart{Patryk Wirkus}{Martyna Czarnobaj}
$$(\frac{151}{103})^{9} \cdot (\frac{103}{151})^{9} \cdot \pi^{0} = (\frac{151}{103} \cdot \frac{103}{151})^{9} \cdot 1 = 1^{9} \cdot 1 = 1$$
\rozwStop
\odpStart
$1$
\odpStop
\testStart
A.$1$ B.$\pi$ C.$0$ D.$\frac{151}{103}$ E.$\frac{103}{151}$
F.$-\frac{151}{103}$ G.$-1$
H.$(\frac{151}{103})^{9}$
I.$(\frac{103}{151})^{9}$
\testStop
\kluczStart
A
\kluczStop



\zadStart{Zadanie z Wikieł Z 1.1 d) moja wersja nr 738}

Obliczyć wartość wyrażenia $(\frac{151}{107})^{9} \cdot (\frac{107}{151})^{9} \cdot \pi^{0}$.
\zadStop
\rozwStart{Patryk Wirkus}{Martyna Czarnobaj}
$$(\frac{151}{107})^{9} \cdot (\frac{107}{151})^{9} \cdot \pi^{0} = (\frac{151}{107} \cdot \frac{107}{151})^{9} \cdot 1 = 1^{9} \cdot 1 = 1$$
\rozwStop
\odpStart
$1$
\odpStop
\testStart
A.$1$ B.$\pi$ C.$0$ D.$\frac{151}{107}$ E.$\frac{107}{151}$
F.$-\frac{151}{107}$ G.$-1$
H.$(\frac{151}{107})^{9}$
I.$(\frac{107}{151})^{9}$
\testStop
\kluczStart
A
\kluczStop



\zadStart{Zadanie z Wikieł Z 1.1 d) moja wersja nr 739}

Obliczyć wartość wyrażenia $(\frac{151}{109})^{9} \cdot (\frac{109}{151})^{9} \cdot \pi^{0}$.
\zadStop
\rozwStart{Patryk Wirkus}{Martyna Czarnobaj}
$$(\frac{151}{109})^{9} \cdot (\frac{109}{151})^{9} \cdot \pi^{0} = (\frac{151}{109} \cdot \frac{109}{151})^{9} \cdot 1 = 1^{9} \cdot 1 = 1$$
\rozwStop
\odpStart
$1$
\odpStop
\testStart
A.$1$ B.$\pi$ C.$0$ D.$\frac{151}{109}$ E.$\frac{109}{151}$
F.$-\frac{151}{109}$ G.$-1$
H.$(\frac{151}{109})^{9}$
I.$(\frac{109}{151})^{9}$
\testStop
\kluczStart
A
\kluczStop



\zadStart{Zadanie z Wikieł Z 1.1 d) moja wersja nr 740}

Obliczyć wartość wyrażenia $(\frac{151}{113})^{9} \cdot (\frac{113}{151})^{9} \cdot \pi^{0}$.
\zadStop
\rozwStart{Patryk Wirkus}{Martyna Czarnobaj}
$$(\frac{151}{113})^{9} \cdot (\frac{113}{151})^{9} \cdot \pi^{0} = (\frac{151}{113} \cdot \frac{113}{151})^{9} \cdot 1 = 1^{9} \cdot 1 = 1$$
\rozwStop
\odpStart
$1$
\odpStop
\testStart
A.$1$ B.$\pi$ C.$0$ D.$\frac{151}{113}$ E.$\frac{113}{151}$
F.$-\frac{151}{113}$ G.$-1$
H.$(\frac{151}{113})^{9}$
I.$(\frac{113}{151})^{9}$
\testStop
\kluczStart
A
\kluczStop



\zadStart{Zadanie z Wikieł Z 1.1 d) moja wersja nr 741}

Obliczyć wartość wyrażenia $(\frac{151}{127})^{9} \cdot (\frac{127}{151})^{9} \cdot \pi^{0}$.
\zadStop
\rozwStart{Patryk Wirkus}{Martyna Czarnobaj}
$$(\frac{151}{127})^{9} \cdot (\frac{127}{151})^{9} \cdot \pi^{0} = (\frac{151}{127} \cdot \frac{127}{151})^{9} \cdot 1 = 1^{9} \cdot 1 = 1$$
\rozwStop
\odpStart
$1$
\odpStop
\testStart
A.$1$ B.$\pi$ C.$0$ D.$\frac{151}{127}$ E.$\frac{127}{151}$
F.$-\frac{151}{127}$ G.$-1$
H.$(\frac{151}{127})^{9}$
I.$(\frac{127}{151})^{9}$
\testStop
\kluczStart
A
\kluczStop



\zadStart{Zadanie z Wikieł Z 1.1 d) moja wersja nr 742}

Obliczyć wartość wyrażenia $(\frac{151}{131})^{9} \cdot (\frac{131}{151})^{9} \cdot \pi^{0}$.
\zadStop
\rozwStart{Patryk Wirkus}{Martyna Czarnobaj}
$$(\frac{151}{131})^{9} \cdot (\frac{131}{151})^{9} \cdot \pi^{0} = (\frac{151}{131} \cdot \frac{131}{151})^{9} \cdot 1 = 1^{9} \cdot 1 = 1$$
\rozwStop
\odpStart
$1$
\odpStop
\testStart
A.$1$ B.$\pi$ C.$0$ D.$\frac{151}{131}$ E.$\frac{131}{151}$
F.$-\frac{151}{131}$ G.$-1$
H.$(\frac{151}{131})^{9}$
I.$(\frac{131}{151})^{9}$
\testStop
\kluczStart
A
\kluczStop



\zadStart{Zadanie z Wikieł Z 1.1 d) moja wersja nr 743}

Obliczyć wartość wyrażenia $(\frac{151}{137})^{9} \cdot (\frac{137}{151})^{9} \cdot \pi^{0}$.
\zadStop
\rozwStart{Patryk Wirkus}{Martyna Czarnobaj}
$$(\frac{151}{137})^{9} \cdot (\frac{137}{151})^{9} \cdot \pi^{0} = (\frac{151}{137} \cdot \frac{137}{151})^{9} \cdot 1 = 1^{9} \cdot 1 = 1$$
\rozwStop
\odpStart
$1$
\odpStop
\testStart
A.$1$ B.$\pi$ C.$0$ D.$\frac{151}{137}$ E.$\frac{137}{151}$
F.$-\frac{151}{137}$ G.$-1$
H.$(\frac{151}{137})^{9}$
I.$(\frac{137}{151})^{9}$
\testStop
\kluczStart
A
\kluczStop



\zadStart{Zadanie z Wikieł Z 1.1 d) moja wersja nr 744}

Obliczyć wartość wyrażenia $(\frac{151}{139})^{9} \cdot (\frac{139}{151})^{9} \cdot \pi^{0}$.
\zadStop
\rozwStart{Patryk Wirkus}{Martyna Czarnobaj}
$$(\frac{151}{139})^{9} \cdot (\frac{139}{151})^{9} \cdot \pi^{0} = (\frac{151}{139} \cdot \frac{139}{151})^{9} \cdot 1 = 1^{9} \cdot 1 = 1$$
\rozwStop
\odpStart
$1$
\odpStop
\testStart
A.$1$ B.$\pi$ C.$0$ D.$\frac{151}{139}$ E.$\frac{139}{151}$
F.$-\frac{151}{139}$ G.$-1$
H.$(\frac{151}{139})^{9}$
I.$(\frac{139}{151})^{9}$
\testStop
\kluczStart
A
\kluczStop



\zadStart{Zadanie z Wikieł Z 1.1 d) moja wersja nr 745}

Obliczyć wartość wyrażenia $(\frac{157}{103})^{9} \cdot (\frac{103}{157})^{9} \cdot \pi^{0}$.
\zadStop
\rozwStart{Patryk Wirkus}{Martyna Czarnobaj}
$$(\frac{157}{103})^{9} \cdot (\frac{103}{157})^{9} \cdot \pi^{0} = (\frac{157}{103} \cdot \frac{103}{157})^{9} \cdot 1 = 1^{9} \cdot 1 = 1$$
\rozwStop
\odpStart
$1$
\odpStop
\testStart
A.$1$ B.$\pi$ C.$0$ D.$\frac{157}{103}$ E.$\frac{103}{157}$
F.$-\frac{157}{103}$ G.$-1$
H.$(\frac{157}{103})^{9}$
I.$(\frac{103}{157})^{9}$
\testStop
\kluczStart
A
\kluczStop



\zadStart{Zadanie z Wikieł Z 1.1 d) moja wersja nr 746}

Obliczyć wartość wyrażenia $(\frac{157}{107})^{9} \cdot (\frac{107}{157})^{9} \cdot \pi^{0}$.
\zadStop
\rozwStart{Patryk Wirkus}{Martyna Czarnobaj}
$$(\frac{157}{107})^{9} \cdot (\frac{107}{157})^{9} \cdot \pi^{0} = (\frac{157}{107} \cdot \frac{107}{157})^{9} \cdot 1 = 1^{9} \cdot 1 = 1$$
\rozwStop
\odpStart
$1$
\odpStop
\testStart
A.$1$ B.$\pi$ C.$0$ D.$\frac{157}{107}$ E.$\frac{107}{157}$
F.$-\frac{157}{107}$ G.$-1$
H.$(\frac{157}{107})^{9}$
I.$(\frac{107}{157})^{9}$
\testStop
\kluczStart
A
\kluczStop



\zadStart{Zadanie z Wikieł Z 1.1 d) moja wersja nr 747}

Obliczyć wartość wyrażenia $(\frac{157}{109})^{9} \cdot (\frac{109}{157})^{9} \cdot \pi^{0}$.
\zadStop
\rozwStart{Patryk Wirkus}{Martyna Czarnobaj}
$$(\frac{157}{109})^{9} \cdot (\frac{109}{157})^{9} \cdot \pi^{0} = (\frac{157}{109} \cdot \frac{109}{157})^{9} \cdot 1 = 1^{9} \cdot 1 = 1$$
\rozwStop
\odpStart
$1$
\odpStop
\testStart
A.$1$ B.$\pi$ C.$0$ D.$\frac{157}{109}$ E.$\frac{109}{157}$
F.$-\frac{157}{109}$ G.$-1$
H.$(\frac{157}{109})^{9}$
I.$(\frac{109}{157})^{9}$
\testStop
\kluczStart
A
\kluczStop



\zadStart{Zadanie z Wikieł Z 1.1 d) moja wersja nr 748}

Obliczyć wartość wyrażenia $(\frac{157}{113})^{9} \cdot (\frac{113}{157})^{9} \cdot \pi^{0}$.
\zadStop
\rozwStart{Patryk Wirkus}{Martyna Czarnobaj}
$$(\frac{157}{113})^{9} \cdot (\frac{113}{157})^{9} \cdot \pi^{0} = (\frac{157}{113} \cdot \frac{113}{157})^{9} \cdot 1 = 1^{9} \cdot 1 = 1$$
\rozwStop
\odpStart
$1$
\odpStop
\testStart
A.$1$ B.$\pi$ C.$0$ D.$\frac{157}{113}$ E.$\frac{113}{157}$
F.$-\frac{157}{113}$ G.$-1$
H.$(\frac{157}{113})^{9}$
I.$(\frac{113}{157})^{9}$
\testStop
\kluczStart
A
\kluczStop



\zadStart{Zadanie z Wikieł Z 1.1 d) moja wersja nr 749}

Obliczyć wartość wyrażenia $(\frac{157}{127})^{9} \cdot (\frac{127}{157})^{9} \cdot \pi^{0}$.
\zadStop
\rozwStart{Patryk Wirkus}{Martyna Czarnobaj}
$$(\frac{157}{127})^{9} \cdot (\frac{127}{157})^{9} \cdot \pi^{0} = (\frac{157}{127} \cdot \frac{127}{157})^{9} \cdot 1 = 1^{9} \cdot 1 = 1$$
\rozwStop
\odpStart
$1$
\odpStop
\testStart
A.$1$ B.$\pi$ C.$0$ D.$\frac{157}{127}$ E.$\frac{127}{157}$
F.$-\frac{157}{127}$ G.$-1$
H.$(\frac{157}{127})^{9}$
I.$(\frac{127}{157})^{9}$
\testStop
\kluczStart
A
\kluczStop



\zadStart{Zadanie z Wikieł Z 1.1 d) moja wersja nr 750}

Obliczyć wartość wyrażenia $(\frac{157}{131})^{9} \cdot (\frac{131}{157})^{9} \cdot \pi^{0}$.
\zadStop
\rozwStart{Patryk Wirkus}{Martyna Czarnobaj}
$$(\frac{157}{131})^{9} \cdot (\frac{131}{157})^{9} \cdot \pi^{0} = (\frac{157}{131} \cdot \frac{131}{157})^{9} \cdot 1 = 1^{9} \cdot 1 = 1$$
\rozwStop
\odpStart
$1$
\odpStop
\testStart
A.$1$ B.$\pi$ C.$0$ D.$\frac{157}{131}$ E.$\frac{131}{157}$
F.$-\frac{157}{131}$ G.$-1$
H.$(\frac{157}{131})^{9}$
I.$(\frac{131}{157})^{9}$
\testStop
\kluczStart
A
\kluczStop



\zadStart{Zadanie z Wikieł Z 1.1 d) moja wersja nr 751}

Obliczyć wartość wyrażenia $(\frac{157}{137})^{9} \cdot (\frac{137}{157})^{9} \cdot \pi^{0}$.
\zadStop
\rozwStart{Patryk Wirkus}{Martyna Czarnobaj}
$$(\frac{157}{137})^{9} \cdot (\frac{137}{157})^{9} \cdot \pi^{0} = (\frac{157}{137} \cdot \frac{137}{157})^{9} \cdot 1 = 1^{9} \cdot 1 = 1$$
\rozwStop
\odpStart
$1$
\odpStop
\testStart
A.$1$ B.$\pi$ C.$0$ D.$\frac{157}{137}$ E.$\frac{137}{157}$
F.$-\frac{157}{137}$ G.$-1$
H.$(\frac{157}{137})^{9}$
I.$(\frac{137}{157})^{9}$
\testStop
\kluczStart
A
\kluczStop



\zadStart{Zadanie z Wikieł Z 1.1 d) moja wersja nr 752}

Obliczyć wartość wyrażenia $(\frac{157}{139})^{9} \cdot (\frac{139}{157})^{9} \cdot \pi^{0}$.
\zadStop
\rozwStart{Patryk Wirkus}{Martyna Czarnobaj}
$$(\frac{157}{139})^{9} \cdot (\frac{139}{157})^{9} \cdot \pi^{0} = (\frac{157}{139} \cdot \frac{139}{157})^{9} \cdot 1 = 1^{9} \cdot 1 = 1$$
\rozwStop
\odpStart
$1$
\odpStop
\testStart
A.$1$ B.$\pi$ C.$0$ D.$\frac{157}{139}$ E.$\frac{139}{157}$
F.$-\frac{157}{139}$ G.$-1$
H.$(\frac{157}{139})^{9}$
I.$(\frac{139}{157})^{9}$
\testStop
\kluczStart
A
\kluczStop



\zadStart{Zadanie z Wikieł Z 1.1 d) moja wersja nr 753}

Obliczyć wartość wyrażenia $(\frac{163}{103})^{9} \cdot (\frac{103}{163})^{9} \cdot \pi^{0}$.
\zadStop
\rozwStart{Patryk Wirkus}{Martyna Czarnobaj}
$$(\frac{163}{103})^{9} \cdot (\frac{103}{163})^{9} \cdot \pi^{0} = (\frac{163}{103} \cdot \frac{103}{163})^{9} \cdot 1 = 1^{9} \cdot 1 = 1$$
\rozwStop
\odpStart
$1$
\odpStop
\testStart
A.$1$ B.$\pi$ C.$0$ D.$\frac{163}{103}$ E.$\frac{103}{163}$
F.$-\frac{163}{103}$ G.$-1$
H.$(\frac{163}{103})^{9}$
I.$(\frac{103}{163})^{9}$
\testStop
\kluczStart
A
\kluczStop



\zadStart{Zadanie z Wikieł Z 1.1 d) moja wersja nr 754}

Obliczyć wartość wyrażenia $(\frac{163}{107})^{9} \cdot (\frac{107}{163})^{9} \cdot \pi^{0}$.
\zadStop
\rozwStart{Patryk Wirkus}{Martyna Czarnobaj}
$$(\frac{163}{107})^{9} \cdot (\frac{107}{163})^{9} \cdot \pi^{0} = (\frac{163}{107} \cdot \frac{107}{163})^{9} \cdot 1 = 1^{9} \cdot 1 = 1$$
\rozwStop
\odpStart
$1$
\odpStop
\testStart
A.$1$ B.$\pi$ C.$0$ D.$\frac{163}{107}$ E.$\frac{107}{163}$
F.$-\frac{163}{107}$ G.$-1$
H.$(\frac{163}{107})^{9}$
I.$(\frac{107}{163})^{9}$
\testStop
\kluczStart
A
\kluczStop



\zadStart{Zadanie z Wikieł Z 1.1 d) moja wersja nr 755}

Obliczyć wartość wyrażenia $(\frac{163}{109})^{9} \cdot (\frac{109}{163})^{9} \cdot \pi^{0}$.
\zadStop
\rozwStart{Patryk Wirkus}{Martyna Czarnobaj}
$$(\frac{163}{109})^{9} \cdot (\frac{109}{163})^{9} \cdot \pi^{0} = (\frac{163}{109} \cdot \frac{109}{163})^{9} \cdot 1 = 1^{9} \cdot 1 = 1$$
\rozwStop
\odpStart
$1$
\odpStop
\testStart
A.$1$ B.$\pi$ C.$0$ D.$\frac{163}{109}$ E.$\frac{109}{163}$
F.$-\frac{163}{109}$ G.$-1$
H.$(\frac{163}{109})^{9}$
I.$(\frac{109}{163})^{9}$
\testStop
\kluczStart
A
\kluczStop



\zadStart{Zadanie z Wikieł Z 1.1 d) moja wersja nr 756}

Obliczyć wartość wyrażenia $(\frac{163}{113})^{9} \cdot (\frac{113}{163})^{9} \cdot \pi^{0}$.
\zadStop
\rozwStart{Patryk Wirkus}{Martyna Czarnobaj}
$$(\frac{163}{113})^{9} \cdot (\frac{113}{163})^{9} \cdot \pi^{0} = (\frac{163}{113} \cdot \frac{113}{163})^{9} \cdot 1 = 1^{9} \cdot 1 = 1$$
\rozwStop
\odpStart
$1$
\odpStop
\testStart
A.$1$ B.$\pi$ C.$0$ D.$\frac{163}{113}$ E.$\frac{113}{163}$
F.$-\frac{163}{113}$ G.$-1$
H.$(\frac{163}{113})^{9}$
I.$(\frac{113}{163})^{9}$
\testStop
\kluczStart
A
\kluczStop



\zadStart{Zadanie z Wikieł Z 1.1 d) moja wersja nr 757}

Obliczyć wartość wyrażenia $(\frac{163}{127})^{9} \cdot (\frac{127}{163})^{9} \cdot \pi^{0}$.
\zadStop
\rozwStart{Patryk Wirkus}{Martyna Czarnobaj}
$$(\frac{163}{127})^{9} \cdot (\frac{127}{163})^{9} \cdot \pi^{0} = (\frac{163}{127} \cdot \frac{127}{163})^{9} \cdot 1 = 1^{9} \cdot 1 = 1$$
\rozwStop
\odpStart
$1$
\odpStop
\testStart
A.$1$ B.$\pi$ C.$0$ D.$\frac{163}{127}$ E.$\frac{127}{163}$
F.$-\frac{163}{127}$ G.$-1$
H.$(\frac{163}{127})^{9}$
I.$(\frac{127}{163})^{9}$
\testStop
\kluczStart
A
\kluczStop



\zadStart{Zadanie z Wikieł Z 1.1 d) moja wersja nr 758}

Obliczyć wartość wyrażenia $(\frac{163}{131})^{9} \cdot (\frac{131}{163})^{9} \cdot \pi^{0}$.
\zadStop
\rozwStart{Patryk Wirkus}{Martyna Czarnobaj}
$$(\frac{163}{131})^{9} \cdot (\frac{131}{163})^{9} \cdot \pi^{0} = (\frac{163}{131} \cdot \frac{131}{163})^{9} \cdot 1 = 1^{9} \cdot 1 = 1$$
\rozwStop
\odpStart
$1$
\odpStop
\testStart
A.$1$ B.$\pi$ C.$0$ D.$\frac{163}{131}$ E.$\frac{131}{163}$
F.$-\frac{163}{131}$ G.$-1$
H.$(\frac{163}{131})^{9}$
I.$(\frac{131}{163})^{9}$
\testStop
\kluczStart
A
\kluczStop



\zadStart{Zadanie z Wikieł Z 1.1 d) moja wersja nr 759}

Obliczyć wartość wyrażenia $(\frac{163}{137})^{9} \cdot (\frac{137}{163})^{9} \cdot \pi^{0}$.
\zadStop
\rozwStart{Patryk Wirkus}{Martyna Czarnobaj}
$$(\frac{163}{137})^{9} \cdot (\frac{137}{163})^{9} \cdot \pi^{0} = (\frac{163}{137} \cdot \frac{137}{163})^{9} \cdot 1 = 1^{9} \cdot 1 = 1$$
\rozwStop
\odpStart
$1$
\odpStop
\testStart
A.$1$ B.$\pi$ C.$0$ D.$\frac{163}{137}$ E.$\frac{137}{163}$
F.$-\frac{163}{137}$ G.$-1$
H.$(\frac{163}{137})^{9}$
I.$(\frac{137}{163})^{9}$
\testStop
\kluczStart
A
\kluczStop



\zadStart{Zadanie z Wikieł Z 1.1 d) moja wersja nr 760}

Obliczyć wartość wyrażenia $(\frac{163}{139})^{9} \cdot (\frac{139}{163})^{9} \cdot \pi^{0}$.
\zadStop
\rozwStart{Patryk Wirkus}{Martyna Czarnobaj}
$$(\frac{163}{139})^{9} \cdot (\frac{139}{163})^{9} \cdot \pi^{0} = (\frac{163}{139} \cdot \frac{139}{163})^{9} \cdot 1 = 1^{9} \cdot 1 = 1$$
\rozwStop
\odpStart
$1$
\odpStop
\testStart
A.$1$ B.$\pi$ C.$0$ D.$\frac{163}{139}$ E.$\frac{139}{163}$
F.$-\frac{163}{139}$ G.$-1$
H.$(\frac{163}{139})^{9}$
I.$(\frac{139}{163})^{9}$
\testStop
\kluczStart
A
\kluczStop



\zadStart{Zadanie z Wikieł Z 1.1 d) moja wersja nr 761}

Obliczyć wartość wyrażenia $(\frac{167}{103})^{9} \cdot (\frac{103}{167})^{9} \cdot \pi^{0}$.
\zadStop
\rozwStart{Patryk Wirkus}{Martyna Czarnobaj}
$$(\frac{167}{103})^{9} \cdot (\frac{103}{167})^{9} \cdot \pi^{0} = (\frac{167}{103} \cdot \frac{103}{167})^{9} \cdot 1 = 1^{9} \cdot 1 = 1$$
\rozwStop
\odpStart
$1$
\odpStop
\testStart
A.$1$ B.$\pi$ C.$0$ D.$\frac{167}{103}$ E.$\frac{103}{167}$
F.$-\frac{167}{103}$ G.$-1$
H.$(\frac{167}{103})^{9}$
I.$(\frac{103}{167})^{9}$
\testStop
\kluczStart
A
\kluczStop



\zadStart{Zadanie z Wikieł Z 1.1 d) moja wersja nr 762}

Obliczyć wartość wyrażenia $(\frac{167}{107})^{9} \cdot (\frac{107}{167})^{9} \cdot \pi^{0}$.
\zadStop
\rozwStart{Patryk Wirkus}{Martyna Czarnobaj}
$$(\frac{167}{107})^{9} \cdot (\frac{107}{167})^{9} \cdot \pi^{0} = (\frac{167}{107} \cdot \frac{107}{167})^{9} \cdot 1 = 1^{9} \cdot 1 = 1$$
\rozwStop
\odpStart
$1$
\odpStop
\testStart
A.$1$ B.$\pi$ C.$0$ D.$\frac{167}{107}$ E.$\frac{107}{167}$
F.$-\frac{167}{107}$ G.$-1$
H.$(\frac{167}{107})^{9}$
I.$(\frac{107}{167})^{9}$
\testStop
\kluczStart
A
\kluczStop



\zadStart{Zadanie z Wikieł Z 1.1 d) moja wersja nr 763}

Obliczyć wartość wyrażenia $(\frac{167}{109})^{9} \cdot (\frac{109}{167})^{9} \cdot \pi^{0}$.
\zadStop
\rozwStart{Patryk Wirkus}{Martyna Czarnobaj}
$$(\frac{167}{109})^{9} \cdot (\frac{109}{167})^{9} \cdot \pi^{0} = (\frac{167}{109} \cdot \frac{109}{167})^{9} \cdot 1 = 1^{9} \cdot 1 = 1$$
\rozwStop
\odpStart
$1$
\odpStop
\testStart
A.$1$ B.$\pi$ C.$0$ D.$\frac{167}{109}$ E.$\frac{109}{167}$
F.$-\frac{167}{109}$ G.$-1$
H.$(\frac{167}{109})^{9}$
I.$(\frac{109}{167})^{9}$
\testStop
\kluczStart
A
\kluczStop



\zadStart{Zadanie z Wikieł Z 1.1 d) moja wersja nr 764}

Obliczyć wartość wyrażenia $(\frac{167}{113})^{9} \cdot (\frac{113}{167})^{9} \cdot \pi^{0}$.
\zadStop
\rozwStart{Patryk Wirkus}{Martyna Czarnobaj}
$$(\frac{167}{113})^{9} \cdot (\frac{113}{167})^{9} \cdot \pi^{0} = (\frac{167}{113} \cdot \frac{113}{167})^{9} \cdot 1 = 1^{9} \cdot 1 = 1$$
\rozwStop
\odpStart
$1$
\odpStop
\testStart
A.$1$ B.$\pi$ C.$0$ D.$\frac{167}{113}$ E.$\frac{113}{167}$
F.$-\frac{167}{113}$ G.$-1$
H.$(\frac{167}{113})^{9}$
I.$(\frac{113}{167})^{9}$
\testStop
\kluczStart
A
\kluczStop



\zadStart{Zadanie z Wikieł Z 1.1 d) moja wersja nr 765}

Obliczyć wartość wyrażenia $(\frac{167}{127})^{9} \cdot (\frac{127}{167})^{9} \cdot \pi^{0}$.
\zadStop
\rozwStart{Patryk Wirkus}{Martyna Czarnobaj}
$$(\frac{167}{127})^{9} \cdot (\frac{127}{167})^{9} \cdot \pi^{0} = (\frac{167}{127} \cdot \frac{127}{167})^{9} \cdot 1 = 1^{9} \cdot 1 = 1$$
\rozwStop
\odpStart
$1$
\odpStop
\testStart
A.$1$ B.$\pi$ C.$0$ D.$\frac{167}{127}$ E.$\frac{127}{167}$
F.$-\frac{167}{127}$ G.$-1$
H.$(\frac{167}{127})^{9}$
I.$(\frac{127}{167})^{9}$
\testStop
\kluczStart
A
\kluczStop



\zadStart{Zadanie z Wikieł Z 1.1 d) moja wersja nr 766}

Obliczyć wartość wyrażenia $(\frac{167}{131})^{9} \cdot (\frac{131}{167})^{9} \cdot \pi^{0}$.
\zadStop
\rozwStart{Patryk Wirkus}{Martyna Czarnobaj}
$$(\frac{167}{131})^{9} \cdot (\frac{131}{167})^{9} \cdot \pi^{0} = (\frac{167}{131} \cdot \frac{131}{167})^{9} \cdot 1 = 1^{9} \cdot 1 = 1$$
\rozwStop
\odpStart
$1$
\odpStop
\testStart
A.$1$ B.$\pi$ C.$0$ D.$\frac{167}{131}$ E.$\frac{131}{167}$
F.$-\frac{167}{131}$ G.$-1$
H.$(\frac{167}{131})^{9}$
I.$(\frac{131}{167})^{9}$
\testStop
\kluczStart
A
\kluczStop



\zadStart{Zadanie z Wikieł Z 1.1 d) moja wersja nr 767}

Obliczyć wartość wyrażenia $(\frac{167}{137})^{9} \cdot (\frac{137}{167})^{9} \cdot \pi^{0}$.
\zadStop
\rozwStart{Patryk Wirkus}{Martyna Czarnobaj}
$$(\frac{167}{137})^{9} \cdot (\frac{137}{167})^{9} \cdot \pi^{0} = (\frac{167}{137} \cdot \frac{137}{167})^{9} \cdot 1 = 1^{9} \cdot 1 = 1$$
\rozwStop
\odpStart
$1$
\odpStop
\testStart
A.$1$ B.$\pi$ C.$0$ D.$\frac{167}{137}$ E.$\frac{137}{167}$
F.$-\frac{167}{137}$ G.$-1$
H.$(\frac{167}{137})^{9}$
I.$(\frac{137}{167})^{9}$
\testStop
\kluczStart
A
\kluczStop



\zadStart{Zadanie z Wikieł Z 1.1 d) moja wersja nr 768}

Obliczyć wartość wyrażenia $(\frac{167}{139})^{9} \cdot (\frac{139}{167})^{9} \cdot \pi^{0}$.
\zadStop
\rozwStart{Patryk Wirkus}{Martyna Czarnobaj}
$$(\frac{167}{139})^{9} \cdot (\frac{139}{167})^{9} \cdot \pi^{0} = (\frac{167}{139} \cdot \frac{139}{167})^{9} \cdot 1 = 1^{9} \cdot 1 = 1$$
\rozwStop
\odpStart
$1$
\odpStop
\testStart
A.$1$ B.$\pi$ C.$0$ D.$\frac{167}{139}$ E.$\frac{139}{167}$
F.$-\frac{167}{139}$ G.$-1$
H.$(\frac{167}{139})^{9}$
I.$(\frac{139}{167})^{9}$
\testStop
\kluczStart
A
\kluczStop



\zadStart{Zadanie z Wikieł Z 1.1 d) moja wersja nr 769}

Obliczyć wartość wyrażenia $(\frac{173}{103})^{9} \cdot (\frac{103}{173})^{9} \cdot \pi^{0}$.
\zadStop
\rozwStart{Patryk Wirkus}{Martyna Czarnobaj}
$$(\frac{173}{103})^{9} \cdot (\frac{103}{173})^{9} \cdot \pi^{0} = (\frac{173}{103} \cdot \frac{103}{173})^{9} \cdot 1 = 1^{9} \cdot 1 = 1$$
\rozwStop
\odpStart
$1$
\odpStop
\testStart
A.$1$ B.$\pi$ C.$0$ D.$\frac{173}{103}$ E.$\frac{103}{173}$
F.$-\frac{173}{103}$ G.$-1$
H.$(\frac{173}{103})^{9}$
I.$(\frac{103}{173})^{9}$
\testStop
\kluczStart
A
\kluczStop



\zadStart{Zadanie z Wikieł Z 1.1 d) moja wersja nr 770}

Obliczyć wartość wyrażenia $(\frac{173}{107})^{9} \cdot (\frac{107}{173})^{9} \cdot \pi^{0}$.
\zadStop
\rozwStart{Patryk Wirkus}{Martyna Czarnobaj}
$$(\frac{173}{107})^{9} \cdot (\frac{107}{173})^{9} \cdot \pi^{0} = (\frac{173}{107} \cdot \frac{107}{173})^{9} \cdot 1 = 1^{9} \cdot 1 = 1$$
\rozwStop
\odpStart
$1$
\odpStop
\testStart
A.$1$ B.$\pi$ C.$0$ D.$\frac{173}{107}$ E.$\frac{107}{173}$
F.$-\frac{173}{107}$ G.$-1$
H.$(\frac{173}{107})^{9}$
I.$(\frac{107}{173})^{9}$
\testStop
\kluczStart
A
\kluczStop



\zadStart{Zadanie z Wikieł Z 1.1 d) moja wersja nr 771}

Obliczyć wartość wyrażenia $(\frac{173}{109})^{9} \cdot (\frac{109}{173})^{9} \cdot \pi^{0}$.
\zadStop
\rozwStart{Patryk Wirkus}{Martyna Czarnobaj}
$$(\frac{173}{109})^{9} \cdot (\frac{109}{173})^{9} \cdot \pi^{0} = (\frac{173}{109} \cdot \frac{109}{173})^{9} \cdot 1 = 1^{9} \cdot 1 = 1$$
\rozwStop
\odpStart
$1$
\odpStop
\testStart
A.$1$ B.$\pi$ C.$0$ D.$\frac{173}{109}$ E.$\frac{109}{173}$
F.$-\frac{173}{109}$ G.$-1$
H.$(\frac{173}{109})^{9}$
I.$(\frac{109}{173})^{9}$
\testStop
\kluczStart
A
\kluczStop



\zadStart{Zadanie z Wikieł Z 1.1 d) moja wersja nr 772}

Obliczyć wartość wyrażenia $(\frac{173}{113})^{9} \cdot (\frac{113}{173})^{9} \cdot \pi^{0}$.
\zadStop
\rozwStart{Patryk Wirkus}{Martyna Czarnobaj}
$$(\frac{173}{113})^{9} \cdot (\frac{113}{173})^{9} \cdot \pi^{0} = (\frac{173}{113} \cdot \frac{113}{173})^{9} \cdot 1 = 1^{9} \cdot 1 = 1$$
\rozwStop
\odpStart
$1$
\odpStop
\testStart
A.$1$ B.$\pi$ C.$0$ D.$\frac{173}{113}$ E.$\frac{113}{173}$
F.$-\frac{173}{113}$ G.$-1$
H.$(\frac{173}{113})^{9}$
I.$(\frac{113}{173})^{9}$
\testStop
\kluczStart
A
\kluczStop



\zadStart{Zadanie z Wikieł Z 1.1 d) moja wersja nr 773}

Obliczyć wartość wyrażenia $(\frac{173}{127})^{9} \cdot (\frac{127}{173})^{9} \cdot \pi^{0}$.
\zadStop
\rozwStart{Patryk Wirkus}{Martyna Czarnobaj}
$$(\frac{173}{127})^{9} \cdot (\frac{127}{173})^{9} \cdot \pi^{0} = (\frac{173}{127} \cdot \frac{127}{173})^{9} \cdot 1 = 1^{9} \cdot 1 = 1$$
\rozwStop
\odpStart
$1$
\odpStop
\testStart
A.$1$ B.$\pi$ C.$0$ D.$\frac{173}{127}$ E.$\frac{127}{173}$
F.$-\frac{173}{127}$ G.$-1$
H.$(\frac{173}{127})^{9}$
I.$(\frac{127}{173})^{9}$
\testStop
\kluczStart
A
\kluczStop



\zadStart{Zadanie z Wikieł Z 1.1 d) moja wersja nr 774}

Obliczyć wartość wyrażenia $(\frac{173}{131})^{9} \cdot (\frac{131}{173})^{9} \cdot \pi^{0}$.
\zadStop
\rozwStart{Patryk Wirkus}{Martyna Czarnobaj}
$$(\frac{173}{131})^{9} \cdot (\frac{131}{173})^{9} \cdot \pi^{0} = (\frac{173}{131} \cdot \frac{131}{173})^{9} \cdot 1 = 1^{9} \cdot 1 = 1$$
\rozwStop
\odpStart
$1$
\odpStop
\testStart
A.$1$ B.$\pi$ C.$0$ D.$\frac{173}{131}$ E.$\frac{131}{173}$
F.$-\frac{173}{131}$ G.$-1$
H.$(\frac{173}{131})^{9}$
I.$(\frac{131}{173})^{9}$
\testStop
\kluczStart
A
\kluczStop



\zadStart{Zadanie z Wikieł Z 1.1 d) moja wersja nr 775}

Obliczyć wartość wyrażenia $(\frac{173}{137})^{9} \cdot (\frac{137}{173})^{9} \cdot \pi^{0}$.
\zadStop
\rozwStart{Patryk Wirkus}{Martyna Czarnobaj}
$$(\frac{173}{137})^{9} \cdot (\frac{137}{173})^{9} \cdot \pi^{0} = (\frac{173}{137} \cdot \frac{137}{173})^{9} \cdot 1 = 1^{9} \cdot 1 = 1$$
\rozwStop
\odpStart
$1$
\odpStop
\testStart
A.$1$ B.$\pi$ C.$0$ D.$\frac{173}{137}$ E.$\frac{137}{173}$
F.$-\frac{173}{137}$ G.$-1$
H.$(\frac{173}{137})^{9}$
I.$(\frac{137}{173})^{9}$
\testStop
\kluczStart
A
\kluczStop



\zadStart{Zadanie z Wikieł Z 1.1 d) moja wersja nr 776}

Obliczyć wartość wyrażenia $(\frac{173}{139})^{9} \cdot (\frac{139}{173})^{9} \cdot \pi^{0}$.
\zadStop
\rozwStart{Patryk Wirkus}{Martyna Czarnobaj}
$$(\frac{173}{139})^{9} \cdot (\frac{139}{173})^{9} \cdot \pi^{0} = (\frac{173}{139} \cdot \frac{139}{173})^{9} \cdot 1 = 1^{9} \cdot 1 = 1$$
\rozwStop
\odpStart
$1$
\odpStop
\testStart
A.$1$ B.$\pi$ C.$0$ D.$\frac{173}{139}$ E.$\frac{139}{173}$
F.$-\frac{173}{139}$ G.$-1$
H.$(\frac{173}{139})^{9}$
I.$(\frac{139}{173})^{9}$
\testStop
\kluczStart
A
\kluczStop



\zadStart{Zadanie z Wikieł Z 1.1 d) moja wersja nr 777}

Obliczyć wartość wyrażenia $(\frac{179}{103})^{9} \cdot (\frac{103}{179})^{9} \cdot \pi^{0}$.
\zadStop
\rozwStart{Patryk Wirkus}{Martyna Czarnobaj}
$$(\frac{179}{103})^{9} \cdot (\frac{103}{179})^{9} \cdot \pi^{0} = (\frac{179}{103} \cdot \frac{103}{179})^{9} \cdot 1 = 1^{9} \cdot 1 = 1$$
\rozwStop
\odpStart
$1$
\odpStop
\testStart
A.$1$ B.$\pi$ C.$0$ D.$\frac{179}{103}$ E.$\frac{103}{179}$
F.$-\frac{179}{103}$ G.$-1$
H.$(\frac{179}{103})^{9}$
I.$(\frac{103}{179})^{9}$
\testStop
\kluczStart
A
\kluczStop



\zadStart{Zadanie z Wikieł Z 1.1 d) moja wersja nr 778}

Obliczyć wartość wyrażenia $(\frac{179}{107})^{9} \cdot (\frac{107}{179})^{9} \cdot \pi^{0}$.
\zadStop
\rozwStart{Patryk Wirkus}{Martyna Czarnobaj}
$$(\frac{179}{107})^{9} \cdot (\frac{107}{179})^{9} \cdot \pi^{0} = (\frac{179}{107} \cdot \frac{107}{179})^{9} \cdot 1 = 1^{9} \cdot 1 = 1$$
\rozwStop
\odpStart
$1$
\odpStop
\testStart
A.$1$ B.$\pi$ C.$0$ D.$\frac{179}{107}$ E.$\frac{107}{179}$
F.$-\frac{179}{107}$ G.$-1$
H.$(\frac{179}{107})^{9}$
I.$(\frac{107}{179})^{9}$
\testStop
\kluczStart
A
\kluczStop



\zadStart{Zadanie z Wikieł Z 1.1 d) moja wersja nr 779}

Obliczyć wartość wyrażenia $(\frac{179}{109})^{9} \cdot (\frac{109}{179})^{9} \cdot \pi^{0}$.
\zadStop
\rozwStart{Patryk Wirkus}{Martyna Czarnobaj}
$$(\frac{179}{109})^{9} \cdot (\frac{109}{179})^{9} \cdot \pi^{0} = (\frac{179}{109} \cdot \frac{109}{179})^{9} \cdot 1 = 1^{9} \cdot 1 = 1$$
\rozwStop
\odpStart
$1$
\odpStop
\testStart
A.$1$ B.$\pi$ C.$0$ D.$\frac{179}{109}$ E.$\frac{109}{179}$
F.$-\frac{179}{109}$ G.$-1$
H.$(\frac{179}{109})^{9}$
I.$(\frac{109}{179})^{9}$
\testStop
\kluczStart
A
\kluczStop



\zadStart{Zadanie z Wikieł Z 1.1 d) moja wersja nr 780}

Obliczyć wartość wyrażenia $(\frac{179}{113})^{9} \cdot (\frac{113}{179})^{9} \cdot \pi^{0}$.
\zadStop
\rozwStart{Patryk Wirkus}{Martyna Czarnobaj}
$$(\frac{179}{113})^{9} \cdot (\frac{113}{179})^{9} \cdot \pi^{0} = (\frac{179}{113} \cdot \frac{113}{179})^{9} \cdot 1 = 1^{9} \cdot 1 = 1$$
\rozwStop
\odpStart
$1$
\odpStop
\testStart
A.$1$ B.$\pi$ C.$0$ D.$\frac{179}{113}$ E.$\frac{113}{179}$
F.$-\frac{179}{113}$ G.$-1$
H.$(\frac{179}{113})^{9}$
I.$(\frac{113}{179})^{9}$
\testStop
\kluczStart
A
\kluczStop



\zadStart{Zadanie z Wikieł Z 1.1 d) moja wersja nr 781}

Obliczyć wartość wyrażenia $(\frac{179}{127})^{9} \cdot (\frac{127}{179})^{9} \cdot \pi^{0}$.
\zadStop
\rozwStart{Patryk Wirkus}{Martyna Czarnobaj}
$$(\frac{179}{127})^{9} \cdot (\frac{127}{179})^{9} \cdot \pi^{0} = (\frac{179}{127} \cdot \frac{127}{179})^{9} \cdot 1 = 1^{9} \cdot 1 = 1$$
\rozwStop
\odpStart
$1$
\odpStop
\testStart
A.$1$ B.$\pi$ C.$0$ D.$\frac{179}{127}$ E.$\frac{127}{179}$
F.$-\frac{179}{127}$ G.$-1$
H.$(\frac{179}{127})^{9}$
I.$(\frac{127}{179})^{9}$
\testStop
\kluczStart
A
\kluczStop



\zadStart{Zadanie z Wikieł Z 1.1 d) moja wersja nr 782}

Obliczyć wartość wyrażenia $(\frac{179}{131})^{9} \cdot (\frac{131}{179})^{9} \cdot \pi^{0}$.
\zadStop
\rozwStart{Patryk Wirkus}{Martyna Czarnobaj}
$$(\frac{179}{131})^{9} \cdot (\frac{131}{179})^{9} \cdot \pi^{0} = (\frac{179}{131} \cdot \frac{131}{179})^{9} \cdot 1 = 1^{9} \cdot 1 = 1$$
\rozwStop
\odpStart
$1$
\odpStop
\testStart
A.$1$ B.$\pi$ C.$0$ D.$\frac{179}{131}$ E.$\frac{131}{179}$
F.$-\frac{179}{131}$ G.$-1$
H.$(\frac{179}{131})^{9}$
I.$(\frac{131}{179})^{9}$
\testStop
\kluczStart
A
\kluczStop



\zadStart{Zadanie z Wikieł Z 1.1 d) moja wersja nr 783}

Obliczyć wartość wyrażenia $(\frac{179}{137})^{9} \cdot (\frac{137}{179})^{9} \cdot \pi^{0}$.
\zadStop
\rozwStart{Patryk Wirkus}{Martyna Czarnobaj}
$$(\frac{179}{137})^{9} \cdot (\frac{137}{179})^{9} \cdot \pi^{0} = (\frac{179}{137} \cdot \frac{137}{179})^{9} \cdot 1 = 1^{9} \cdot 1 = 1$$
\rozwStop
\odpStart
$1$
\odpStop
\testStart
A.$1$ B.$\pi$ C.$0$ D.$\frac{179}{137}$ E.$\frac{137}{179}$
F.$-\frac{179}{137}$ G.$-1$
H.$(\frac{179}{137})^{9}$
I.$(\frac{137}{179})^{9}$
\testStop
\kluczStart
A
\kluczStop



\zadStart{Zadanie z Wikieł Z 1.1 d) moja wersja nr 784}

Obliczyć wartość wyrażenia $(\frac{179}{139})^{9} \cdot (\frac{139}{179})^{9} \cdot \pi^{0}$.
\zadStop
\rozwStart{Patryk Wirkus}{Martyna Czarnobaj}
$$(\frac{179}{139})^{9} \cdot (\frac{139}{179})^{9} \cdot \pi^{0} = (\frac{179}{139} \cdot \frac{139}{179})^{9} \cdot 1 = 1^{9} \cdot 1 = 1$$
\rozwStop
\odpStart
$1$
\odpStop
\testStart
A.$1$ B.$\pi$ C.$0$ D.$\frac{179}{139}$ E.$\frac{139}{179}$
F.$-\frac{179}{139}$ G.$-1$
H.$(\frac{179}{139})^{9}$
I.$(\frac{139}{179})^{9}$
\testStop
\kluczStart
A
\kluczStop



\zadStart{Zadanie z Wikieł Z 1.1 d) moja wersja nr 785}

Obliczyć wartość wyrażenia $(\frac{251}{103})^{9} \cdot (\frac{103}{251})^{9} \cdot \pi^{0}$.
\zadStop
\rozwStart{Patryk Wirkus}{Martyna Czarnobaj}
$$(\frac{251}{103})^{9} \cdot (\frac{103}{251})^{9} \cdot \pi^{0} = (\frac{251}{103} \cdot \frac{103}{251})^{9} \cdot 1 = 1^{9} \cdot 1 = 1$$
\rozwStop
\odpStart
$1$
\odpStop
\testStart
A.$1$ B.$\pi$ C.$0$ D.$\frac{251}{103}$ E.$\frac{103}{251}$
F.$-\frac{251}{103}$ G.$-1$
H.$(\frac{251}{103})^{9}$
I.$(\frac{103}{251})^{9}$
\testStop
\kluczStart
A
\kluczStop



\zadStart{Zadanie z Wikieł Z 1.1 d) moja wersja nr 786}

Obliczyć wartość wyrażenia $(\frac{251}{107})^{9} \cdot (\frac{107}{251})^{9} \cdot \pi^{0}$.
\zadStop
\rozwStart{Patryk Wirkus}{Martyna Czarnobaj}
$$(\frac{251}{107})^{9} \cdot (\frac{107}{251})^{9} \cdot \pi^{0} = (\frac{251}{107} \cdot \frac{107}{251})^{9} \cdot 1 = 1^{9} \cdot 1 = 1$$
\rozwStop
\odpStart
$1$
\odpStop
\testStart
A.$1$ B.$\pi$ C.$0$ D.$\frac{251}{107}$ E.$\frac{107}{251}$
F.$-\frac{251}{107}$ G.$-1$
H.$(\frac{251}{107})^{9}$
I.$(\frac{107}{251})^{9}$
\testStop
\kluczStart
A
\kluczStop



\zadStart{Zadanie z Wikieł Z 1.1 d) moja wersja nr 787}

Obliczyć wartość wyrażenia $(\frac{251}{109})^{9} \cdot (\frac{109}{251})^{9} \cdot \pi^{0}$.
\zadStop
\rozwStart{Patryk Wirkus}{Martyna Czarnobaj}
$$(\frac{251}{109})^{9} \cdot (\frac{109}{251})^{9} \cdot \pi^{0} = (\frac{251}{109} \cdot \frac{109}{251})^{9} \cdot 1 = 1^{9} \cdot 1 = 1$$
\rozwStop
\odpStart
$1$
\odpStop
\testStart
A.$1$ B.$\pi$ C.$0$ D.$\frac{251}{109}$ E.$\frac{109}{251}$
F.$-\frac{251}{109}$ G.$-1$
H.$(\frac{251}{109})^{9}$
I.$(\frac{109}{251})^{9}$
\testStop
\kluczStart
A
\kluczStop



\zadStart{Zadanie z Wikieł Z 1.1 d) moja wersja nr 788}

Obliczyć wartość wyrażenia $(\frac{251}{113})^{9} \cdot (\frac{113}{251})^{9} \cdot \pi^{0}$.
\zadStop
\rozwStart{Patryk Wirkus}{Martyna Czarnobaj}
$$(\frac{251}{113})^{9} \cdot (\frac{113}{251})^{9} \cdot \pi^{0} = (\frac{251}{113} \cdot \frac{113}{251})^{9} \cdot 1 = 1^{9} \cdot 1 = 1$$
\rozwStop
\odpStart
$1$
\odpStop
\testStart
A.$1$ B.$\pi$ C.$0$ D.$\frac{251}{113}$ E.$\frac{113}{251}$
F.$-\frac{251}{113}$ G.$-1$
H.$(\frac{251}{113})^{9}$
I.$(\frac{113}{251})^{9}$
\testStop
\kluczStart
A
\kluczStop



\zadStart{Zadanie z Wikieł Z 1.1 d) moja wersja nr 789}

Obliczyć wartość wyrażenia $(\frac{251}{127})^{9} \cdot (\frac{127}{251})^{9} \cdot \pi^{0}$.
\zadStop
\rozwStart{Patryk Wirkus}{Martyna Czarnobaj}
$$(\frac{251}{127})^{9} \cdot (\frac{127}{251})^{9} \cdot \pi^{0} = (\frac{251}{127} \cdot \frac{127}{251})^{9} \cdot 1 = 1^{9} \cdot 1 = 1$$
\rozwStop
\odpStart
$1$
\odpStop
\testStart
A.$1$ B.$\pi$ C.$0$ D.$\frac{251}{127}$ E.$\frac{127}{251}$
F.$-\frac{251}{127}$ G.$-1$
H.$(\frac{251}{127})^{9}$
I.$(\frac{127}{251})^{9}$
\testStop
\kluczStart
A
\kluczStop



\zadStart{Zadanie z Wikieł Z 1.1 d) moja wersja nr 790}

Obliczyć wartość wyrażenia $(\frac{251}{131})^{9} \cdot (\frac{131}{251})^{9} \cdot \pi^{0}$.
\zadStop
\rozwStart{Patryk Wirkus}{Martyna Czarnobaj}
$$(\frac{251}{131})^{9} \cdot (\frac{131}{251})^{9} \cdot \pi^{0} = (\frac{251}{131} \cdot \frac{131}{251})^{9} \cdot 1 = 1^{9} \cdot 1 = 1$$
\rozwStop
\odpStart
$1$
\odpStop
\testStart
A.$1$ B.$\pi$ C.$0$ D.$\frac{251}{131}$ E.$\frac{131}{251}$
F.$-\frac{251}{131}$ G.$-1$
H.$(\frac{251}{131})^{9}$
I.$(\frac{131}{251})^{9}$
\testStop
\kluczStart
A
\kluczStop



\zadStart{Zadanie z Wikieł Z 1.1 d) moja wersja nr 791}

Obliczyć wartość wyrażenia $(\frac{251}{137})^{9} \cdot (\frac{137}{251})^{9} \cdot \pi^{0}$.
\zadStop
\rozwStart{Patryk Wirkus}{Martyna Czarnobaj}
$$(\frac{251}{137})^{9} \cdot (\frac{137}{251})^{9} \cdot \pi^{0} = (\frac{251}{137} \cdot \frac{137}{251})^{9} \cdot 1 = 1^{9} \cdot 1 = 1$$
\rozwStop
\odpStart
$1$
\odpStop
\testStart
A.$1$ B.$\pi$ C.$0$ D.$\frac{251}{137}$ E.$\frac{137}{251}$
F.$-\frac{251}{137}$ G.$-1$
H.$(\frac{251}{137})^{9}$
I.$(\frac{137}{251})^{9}$
\testStop
\kluczStart
A
\kluczStop



\zadStart{Zadanie z Wikieł Z 1.1 d) moja wersja nr 792}

Obliczyć wartość wyrażenia $(\frac{251}{139})^{9} \cdot (\frac{139}{251})^{9} \cdot \pi^{0}$.
\zadStop
\rozwStart{Patryk Wirkus}{Martyna Czarnobaj}
$$(\frac{251}{139})^{9} \cdot (\frac{139}{251})^{9} \cdot \pi^{0} = (\frac{251}{139} \cdot \frac{139}{251})^{9} \cdot 1 = 1^{9} \cdot 1 = 1$$
\rozwStop
\odpStart
$1$
\odpStop
\testStart
A.$1$ B.$\pi$ C.$0$ D.$\frac{251}{139}$ E.$\frac{139}{251}$
F.$-\frac{251}{139}$ G.$-1$
H.$(\frac{251}{139})^{9}$
I.$(\frac{139}{251})^{9}$
\testStop
\kluczStart
A
\kluczStop



\zadStart{Zadanie z Wikieł Z 1.1 d) moja wersja nr 793}

Obliczyć wartość wyrażenia $(\frac{257}{103})^{9} \cdot (\frac{103}{257})^{9} \cdot \pi^{0}$.
\zadStop
\rozwStart{Patryk Wirkus}{Martyna Czarnobaj}
$$(\frac{257}{103})^{9} \cdot (\frac{103}{257})^{9} \cdot \pi^{0} = (\frac{257}{103} \cdot \frac{103}{257})^{9} \cdot 1 = 1^{9} \cdot 1 = 1$$
\rozwStop
\odpStart
$1$
\odpStop
\testStart
A.$1$ B.$\pi$ C.$0$ D.$\frac{257}{103}$ E.$\frac{103}{257}$
F.$-\frac{257}{103}$ G.$-1$
H.$(\frac{257}{103})^{9}$
I.$(\frac{103}{257})^{9}$
\testStop
\kluczStart
A
\kluczStop



\zadStart{Zadanie z Wikieł Z 1.1 d) moja wersja nr 794}

Obliczyć wartość wyrażenia $(\frac{257}{107})^{9} \cdot (\frac{107}{257})^{9} \cdot \pi^{0}$.
\zadStop
\rozwStart{Patryk Wirkus}{Martyna Czarnobaj}
$$(\frac{257}{107})^{9} \cdot (\frac{107}{257})^{9} \cdot \pi^{0} = (\frac{257}{107} \cdot \frac{107}{257})^{9} \cdot 1 = 1^{9} \cdot 1 = 1$$
\rozwStop
\odpStart
$1$
\odpStop
\testStart
A.$1$ B.$\pi$ C.$0$ D.$\frac{257}{107}$ E.$\frac{107}{257}$
F.$-\frac{257}{107}$ G.$-1$
H.$(\frac{257}{107})^{9}$
I.$(\frac{107}{257})^{9}$
\testStop
\kluczStart
A
\kluczStop



\zadStart{Zadanie z Wikieł Z 1.1 d) moja wersja nr 795}

Obliczyć wartość wyrażenia $(\frac{257}{109})^{9} \cdot (\frac{109}{257})^{9} \cdot \pi^{0}$.
\zadStop
\rozwStart{Patryk Wirkus}{Martyna Czarnobaj}
$$(\frac{257}{109})^{9} \cdot (\frac{109}{257})^{9} \cdot \pi^{0} = (\frac{257}{109} \cdot \frac{109}{257})^{9} \cdot 1 = 1^{9} \cdot 1 = 1$$
\rozwStop
\odpStart
$1$
\odpStop
\testStart
A.$1$ B.$\pi$ C.$0$ D.$\frac{257}{109}$ E.$\frac{109}{257}$
F.$-\frac{257}{109}$ G.$-1$
H.$(\frac{257}{109})^{9}$
I.$(\frac{109}{257})^{9}$
\testStop
\kluczStart
A
\kluczStop



\zadStart{Zadanie z Wikieł Z 1.1 d) moja wersja nr 796}

Obliczyć wartość wyrażenia $(\frac{257}{113})^{9} \cdot (\frac{113}{257})^{9} \cdot \pi^{0}$.
\zadStop
\rozwStart{Patryk Wirkus}{Martyna Czarnobaj}
$$(\frac{257}{113})^{9} \cdot (\frac{113}{257})^{9} \cdot \pi^{0} = (\frac{257}{113} \cdot \frac{113}{257})^{9} \cdot 1 = 1^{9} \cdot 1 = 1$$
\rozwStop
\odpStart
$1$
\odpStop
\testStart
A.$1$ B.$\pi$ C.$0$ D.$\frac{257}{113}$ E.$\frac{113}{257}$
F.$-\frac{257}{113}$ G.$-1$
H.$(\frac{257}{113})^{9}$
I.$(\frac{113}{257})^{9}$
\testStop
\kluczStart
A
\kluczStop



\zadStart{Zadanie z Wikieł Z 1.1 d) moja wersja nr 797}

Obliczyć wartość wyrażenia $(\frac{257}{127})^{9} \cdot (\frac{127}{257})^{9} \cdot \pi^{0}$.
\zadStop
\rozwStart{Patryk Wirkus}{Martyna Czarnobaj}
$$(\frac{257}{127})^{9} \cdot (\frac{127}{257})^{9} \cdot \pi^{0} = (\frac{257}{127} \cdot \frac{127}{257})^{9} \cdot 1 = 1^{9} \cdot 1 = 1$$
\rozwStop
\odpStart
$1$
\odpStop
\testStart
A.$1$ B.$\pi$ C.$0$ D.$\frac{257}{127}$ E.$\frac{127}{257}$
F.$-\frac{257}{127}$ G.$-1$
H.$(\frac{257}{127})^{9}$
I.$(\frac{127}{257})^{9}$
\testStop
\kluczStart
A
\kluczStop



\zadStart{Zadanie z Wikieł Z 1.1 d) moja wersja nr 798}

Obliczyć wartość wyrażenia $(\frac{257}{131})^{9} \cdot (\frac{131}{257})^{9} \cdot \pi^{0}$.
\zadStop
\rozwStart{Patryk Wirkus}{Martyna Czarnobaj}
$$(\frac{257}{131})^{9} \cdot (\frac{131}{257})^{9} \cdot \pi^{0} = (\frac{257}{131} \cdot \frac{131}{257})^{9} \cdot 1 = 1^{9} \cdot 1 = 1$$
\rozwStop
\odpStart
$1$
\odpStop
\testStart
A.$1$ B.$\pi$ C.$0$ D.$\frac{257}{131}$ E.$\frac{131}{257}$
F.$-\frac{257}{131}$ G.$-1$
H.$(\frac{257}{131})^{9}$
I.$(\frac{131}{257})^{9}$
\testStop
\kluczStart
A
\kluczStop



\zadStart{Zadanie z Wikieł Z 1.1 d) moja wersja nr 799}

Obliczyć wartość wyrażenia $(\frac{257}{137})^{9} \cdot (\frac{137}{257})^{9} \cdot \pi^{0}$.
\zadStop
\rozwStart{Patryk Wirkus}{Martyna Czarnobaj}
$$(\frac{257}{137})^{9} \cdot (\frac{137}{257})^{9} \cdot \pi^{0} = (\frac{257}{137} \cdot \frac{137}{257})^{9} \cdot 1 = 1^{9} \cdot 1 = 1$$
\rozwStop
\odpStart
$1$
\odpStop
\testStart
A.$1$ B.$\pi$ C.$0$ D.$\frac{257}{137}$ E.$\frac{137}{257}$
F.$-\frac{257}{137}$ G.$-1$
H.$(\frac{257}{137})^{9}$
I.$(\frac{137}{257})^{9}$
\testStop
\kluczStart
A
\kluczStop



\zadStart{Zadanie z Wikieł Z 1.1 d) moja wersja nr 800}

Obliczyć wartość wyrażenia $(\frac{257}{139})^{9} \cdot (\frac{139}{257})^{9} \cdot \pi^{0}$.
\zadStop
\rozwStart{Patryk Wirkus}{Martyna Czarnobaj}
$$(\frac{257}{139})^{9} \cdot (\frac{139}{257})^{9} \cdot \pi^{0} = (\frac{257}{139} \cdot \frac{139}{257})^{9} \cdot 1 = 1^{9} \cdot 1 = 1$$
\rozwStop
\odpStart
$1$
\odpStop
\testStart
A.$1$ B.$\pi$ C.$0$ D.$\frac{257}{139}$ E.$\frac{139}{257}$
F.$-\frac{257}{139}$ G.$-1$
H.$(\frac{257}{139})^{9}$
I.$(\frac{139}{257})^{9}$
\testStop
\kluczStart
A
\kluczStop



\zadStart{Zadanie z Wikieł Z 1.1 d) moja wersja nr 801}

Obliczyć wartość wyrażenia $(\frac{263}{103})^{9} \cdot (\frac{103}{263})^{9} \cdot \pi^{0}$.
\zadStop
\rozwStart{Patryk Wirkus}{Martyna Czarnobaj}
$$(\frac{263}{103})^{9} \cdot (\frac{103}{263})^{9} \cdot \pi^{0} = (\frac{263}{103} \cdot \frac{103}{263})^{9} \cdot 1 = 1^{9} \cdot 1 = 1$$
\rozwStop
\odpStart
$1$
\odpStop
\testStart
A.$1$ B.$\pi$ C.$0$ D.$\frac{263}{103}$ E.$\frac{103}{263}$
F.$-\frac{263}{103}$ G.$-1$
H.$(\frac{263}{103})^{9}$
I.$(\frac{103}{263})^{9}$
\testStop
\kluczStart
A
\kluczStop



\zadStart{Zadanie z Wikieł Z 1.1 d) moja wersja nr 802}

Obliczyć wartość wyrażenia $(\frac{263}{107})^{9} \cdot (\frac{107}{263})^{9} \cdot \pi^{0}$.
\zadStop
\rozwStart{Patryk Wirkus}{Martyna Czarnobaj}
$$(\frac{263}{107})^{9} \cdot (\frac{107}{263})^{9} \cdot \pi^{0} = (\frac{263}{107} \cdot \frac{107}{263})^{9} \cdot 1 = 1^{9} \cdot 1 = 1$$
\rozwStop
\odpStart
$1$
\odpStop
\testStart
A.$1$ B.$\pi$ C.$0$ D.$\frac{263}{107}$ E.$\frac{107}{263}$
F.$-\frac{263}{107}$ G.$-1$
H.$(\frac{263}{107})^{9}$
I.$(\frac{107}{263})^{9}$
\testStop
\kluczStart
A
\kluczStop



\zadStart{Zadanie z Wikieł Z 1.1 d) moja wersja nr 803}

Obliczyć wartość wyrażenia $(\frac{263}{109})^{9} \cdot (\frac{109}{263})^{9} \cdot \pi^{0}$.
\zadStop
\rozwStart{Patryk Wirkus}{Martyna Czarnobaj}
$$(\frac{263}{109})^{9} \cdot (\frac{109}{263})^{9} \cdot \pi^{0} = (\frac{263}{109} \cdot \frac{109}{263})^{9} \cdot 1 = 1^{9} \cdot 1 = 1$$
\rozwStop
\odpStart
$1$
\odpStop
\testStart
A.$1$ B.$\pi$ C.$0$ D.$\frac{263}{109}$ E.$\frac{109}{263}$
F.$-\frac{263}{109}$ G.$-1$
H.$(\frac{263}{109})^{9}$
I.$(\frac{109}{263})^{9}$
\testStop
\kluczStart
A
\kluczStop



\zadStart{Zadanie z Wikieł Z 1.1 d) moja wersja nr 804}

Obliczyć wartość wyrażenia $(\frac{263}{113})^{9} \cdot (\frac{113}{263})^{9} \cdot \pi^{0}$.
\zadStop
\rozwStart{Patryk Wirkus}{Martyna Czarnobaj}
$$(\frac{263}{113})^{9} \cdot (\frac{113}{263})^{9} \cdot \pi^{0} = (\frac{263}{113} \cdot \frac{113}{263})^{9} \cdot 1 = 1^{9} \cdot 1 = 1$$
\rozwStop
\odpStart
$1$
\odpStop
\testStart
A.$1$ B.$\pi$ C.$0$ D.$\frac{263}{113}$ E.$\frac{113}{263}$
F.$-\frac{263}{113}$ G.$-1$
H.$(\frac{263}{113})^{9}$
I.$(\frac{113}{263})^{9}$
\testStop
\kluczStart
A
\kluczStop



\zadStart{Zadanie z Wikieł Z 1.1 d) moja wersja nr 805}

Obliczyć wartość wyrażenia $(\frac{263}{127})^{9} \cdot (\frac{127}{263})^{9} \cdot \pi^{0}$.
\zadStop
\rozwStart{Patryk Wirkus}{Martyna Czarnobaj}
$$(\frac{263}{127})^{9} \cdot (\frac{127}{263})^{9} \cdot \pi^{0} = (\frac{263}{127} \cdot \frac{127}{263})^{9} \cdot 1 = 1^{9} \cdot 1 = 1$$
\rozwStop
\odpStart
$1$
\odpStop
\testStart
A.$1$ B.$\pi$ C.$0$ D.$\frac{263}{127}$ E.$\frac{127}{263}$
F.$-\frac{263}{127}$ G.$-1$
H.$(\frac{263}{127})^{9}$
I.$(\frac{127}{263})^{9}$
\testStop
\kluczStart
A
\kluczStop



\zadStart{Zadanie z Wikieł Z 1.1 d) moja wersja nr 806}

Obliczyć wartość wyrażenia $(\frac{263}{131})^{9} \cdot (\frac{131}{263})^{9} \cdot \pi^{0}$.
\zadStop
\rozwStart{Patryk Wirkus}{Martyna Czarnobaj}
$$(\frac{263}{131})^{9} \cdot (\frac{131}{263})^{9} \cdot \pi^{0} = (\frac{263}{131} \cdot \frac{131}{263})^{9} \cdot 1 = 1^{9} \cdot 1 = 1$$
\rozwStop
\odpStart
$1$
\odpStop
\testStart
A.$1$ B.$\pi$ C.$0$ D.$\frac{263}{131}$ E.$\frac{131}{263}$
F.$-\frac{263}{131}$ G.$-1$
H.$(\frac{263}{131})^{9}$
I.$(\frac{131}{263})^{9}$
\testStop
\kluczStart
A
\kluczStop



\zadStart{Zadanie z Wikieł Z 1.1 d) moja wersja nr 807}

Obliczyć wartość wyrażenia $(\frac{263}{137})^{9} \cdot (\frac{137}{263})^{9} \cdot \pi^{0}$.
\zadStop
\rozwStart{Patryk Wirkus}{Martyna Czarnobaj}
$$(\frac{263}{137})^{9} \cdot (\frac{137}{263})^{9} \cdot \pi^{0} = (\frac{263}{137} \cdot \frac{137}{263})^{9} \cdot 1 = 1^{9} \cdot 1 = 1$$
\rozwStop
\odpStart
$1$
\odpStop
\testStart
A.$1$ B.$\pi$ C.$0$ D.$\frac{263}{137}$ E.$\frac{137}{263}$
F.$-\frac{263}{137}$ G.$-1$
H.$(\frac{263}{137})^{9}$
I.$(\frac{137}{263})^{9}$
\testStop
\kluczStart
A
\kluczStop



\zadStart{Zadanie z Wikieł Z 1.1 d) moja wersja nr 808}

Obliczyć wartość wyrażenia $(\frac{263}{139})^{9} \cdot (\frac{139}{263})^{9} \cdot \pi^{0}$.
\zadStop
\rozwStart{Patryk Wirkus}{Martyna Czarnobaj}
$$(\frac{263}{139})^{9} \cdot (\frac{139}{263})^{9} \cdot \pi^{0} = (\frac{263}{139} \cdot \frac{139}{263})^{9} \cdot 1 = 1^{9} \cdot 1 = 1$$
\rozwStop
\odpStart
$1$
\odpStop
\testStart
A.$1$ B.$\pi$ C.$0$ D.$\frac{263}{139}$ E.$\frac{139}{263}$
F.$-\frac{263}{139}$ G.$-1$
H.$(\frac{263}{139})^{9}$
I.$(\frac{139}{263})^{9}$
\testStop
\kluczStart
A
\kluczStop



\zadStart{Zadanie z Wikieł Z 1.1 d) moja wersja nr 809}

Obliczyć wartość wyrażenia $(\frac{269}{103})^{9} \cdot (\frac{103}{269})^{9} \cdot \pi^{0}$.
\zadStop
\rozwStart{Patryk Wirkus}{Martyna Czarnobaj}
$$(\frac{269}{103})^{9} \cdot (\frac{103}{269})^{9} \cdot \pi^{0} = (\frac{269}{103} \cdot \frac{103}{269})^{9} \cdot 1 = 1^{9} \cdot 1 = 1$$
\rozwStop
\odpStart
$1$
\odpStop
\testStart
A.$1$ B.$\pi$ C.$0$ D.$\frac{269}{103}$ E.$\frac{103}{269}$
F.$-\frac{269}{103}$ G.$-1$
H.$(\frac{269}{103})^{9}$
I.$(\frac{103}{269})^{9}$
\testStop
\kluczStart
A
\kluczStop



\zadStart{Zadanie z Wikieł Z 1.1 d) moja wersja nr 810}

Obliczyć wartość wyrażenia $(\frac{269}{107})^{9} \cdot (\frac{107}{269})^{9} \cdot \pi^{0}$.
\zadStop
\rozwStart{Patryk Wirkus}{Martyna Czarnobaj}
$$(\frac{269}{107})^{9} \cdot (\frac{107}{269})^{9} \cdot \pi^{0} = (\frac{269}{107} \cdot \frac{107}{269})^{9} \cdot 1 = 1^{9} \cdot 1 = 1$$
\rozwStop
\odpStart
$1$
\odpStop
\testStart
A.$1$ B.$\pi$ C.$0$ D.$\frac{269}{107}$ E.$\frac{107}{269}$
F.$-\frac{269}{107}$ G.$-1$
H.$(\frac{269}{107})^{9}$
I.$(\frac{107}{269})^{9}$
\testStop
\kluczStart
A
\kluczStop



\zadStart{Zadanie z Wikieł Z 1.1 d) moja wersja nr 811}

Obliczyć wartość wyrażenia $(\frac{269}{109})^{9} \cdot (\frac{109}{269})^{9} \cdot \pi^{0}$.
\zadStop
\rozwStart{Patryk Wirkus}{Martyna Czarnobaj}
$$(\frac{269}{109})^{9} \cdot (\frac{109}{269})^{9} \cdot \pi^{0} = (\frac{269}{109} \cdot \frac{109}{269})^{9} \cdot 1 = 1^{9} \cdot 1 = 1$$
\rozwStop
\odpStart
$1$
\odpStop
\testStart
A.$1$ B.$\pi$ C.$0$ D.$\frac{269}{109}$ E.$\frac{109}{269}$
F.$-\frac{269}{109}$ G.$-1$
H.$(\frac{269}{109})^{9}$
I.$(\frac{109}{269})^{9}$
\testStop
\kluczStart
A
\kluczStop



\zadStart{Zadanie z Wikieł Z 1.1 d) moja wersja nr 812}

Obliczyć wartość wyrażenia $(\frac{269}{113})^{9} \cdot (\frac{113}{269})^{9} \cdot \pi^{0}$.
\zadStop
\rozwStart{Patryk Wirkus}{Martyna Czarnobaj}
$$(\frac{269}{113})^{9} \cdot (\frac{113}{269})^{9} \cdot \pi^{0} = (\frac{269}{113} \cdot \frac{113}{269})^{9} \cdot 1 = 1^{9} \cdot 1 = 1$$
\rozwStop
\odpStart
$1$
\odpStop
\testStart
A.$1$ B.$\pi$ C.$0$ D.$\frac{269}{113}$ E.$\frac{113}{269}$
F.$-\frac{269}{113}$ G.$-1$
H.$(\frac{269}{113})^{9}$
I.$(\frac{113}{269})^{9}$
\testStop
\kluczStart
A
\kluczStop



\zadStart{Zadanie z Wikieł Z 1.1 d) moja wersja nr 813}

Obliczyć wartość wyrażenia $(\frac{269}{127})^{9} \cdot (\frac{127}{269})^{9} \cdot \pi^{0}$.
\zadStop
\rozwStart{Patryk Wirkus}{Martyna Czarnobaj}
$$(\frac{269}{127})^{9} \cdot (\frac{127}{269})^{9} \cdot \pi^{0} = (\frac{269}{127} \cdot \frac{127}{269})^{9} \cdot 1 = 1^{9} \cdot 1 = 1$$
\rozwStop
\odpStart
$1$
\odpStop
\testStart
A.$1$ B.$\pi$ C.$0$ D.$\frac{269}{127}$ E.$\frac{127}{269}$
F.$-\frac{269}{127}$ G.$-1$
H.$(\frac{269}{127})^{9}$
I.$(\frac{127}{269})^{9}$
\testStop
\kluczStart
A
\kluczStop



\zadStart{Zadanie z Wikieł Z 1.1 d) moja wersja nr 814}

Obliczyć wartość wyrażenia $(\frac{269}{131})^{9} \cdot (\frac{131}{269})^{9} \cdot \pi^{0}$.
\zadStop
\rozwStart{Patryk Wirkus}{Martyna Czarnobaj}
$$(\frac{269}{131})^{9} \cdot (\frac{131}{269})^{9} \cdot \pi^{0} = (\frac{269}{131} \cdot \frac{131}{269})^{9} \cdot 1 = 1^{9} \cdot 1 = 1$$
\rozwStop
\odpStart
$1$
\odpStop
\testStart
A.$1$ B.$\pi$ C.$0$ D.$\frac{269}{131}$ E.$\frac{131}{269}$
F.$-\frac{269}{131}$ G.$-1$
H.$(\frac{269}{131})^{9}$
I.$(\frac{131}{269})^{9}$
\testStop
\kluczStart
A
\kluczStop



\zadStart{Zadanie z Wikieł Z 1.1 d) moja wersja nr 815}

Obliczyć wartość wyrażenia $(\frac{269}{137})^{9} \cdot (\frac{137}{269})^{9} \cdot \pi^{0}$.
\zadStop
\rozwStart{Patryk Wirkus}{Martyna Czarnobaj}
$$(\frac{269}{137})^{9} \cdot (\frac{137}{269})^{9} \cdot \pi^{0} = (\frac{269}{137} \cdot \frac{137}{269})^{9} \cdot 1 = 1^{9} \cdot 1 = 1$$
\rozwStop
\odpStart
$1$
\odpStop
\testStart
A.$1$ B.$\pi$ C.$0$ D.$\frac{269}{137}$ E.$\frac{137}{269}$
F.$-\frac{269}{137}$ G.$-1$
H.$(\frac{269}{137})^{9}$
I.$(\frac{137}{269})^{9}$
\testStop
\kluczStart
A
\kluczStop



\zadStart{Zadanie z Wikieł Z 1.1 d) moja wersja nr 816}

Obliczyć wartość wyrażenia $(\frac{269}{139})^{9} \cdot (\frac{139}{269})^{9} \cdot \pi^{0}$.
\zadStop
\rozwStart{Patryk Wirkus}{Martyna Czarnobaj}
$$(\frac{269}{139})^{9} \cdot (\frac{139}{269})^{9} \cdot \pi^{0} = (\frac{269}{139} \cdot \frac{139}{269})^{9} \cdot 1 = 1^{9} \cdot 1 = 1$$
\rozwStop
\odpStart
$1$
\odpStop
\testStart
A.$1$ B.$\pi$ C.$0$ D.$\frac{269}{139}$ E.$\frac{139}{269}$
F.$-\frac{269}{139}$ G.$-1$
H.$(\frac{269}{139})^{9}$
I.$(\frac{139}{269})^{9}$
\testStop
\kluczStart
A
\kluczStop



\zadStart{Zadanie z Wikieł Z 1.1 d) moja wersja nr 817}

Obliczyć wartość wyrażenia $(\frac{271}{103})^{9} \cdot (\frac{103}{271})^{9} \cdot \pi^{0}$.
\zadStop
\rozwStart{Patryk Wirkus}{Martyna Czarnobaj}
$$(\frac{271}{103})^{9} \cdot (\frac{103}{271})^{9} \cdot \pi^{0} = (\frac{271}{103} \cdot \frac{103}{271})^{9} \cdot 1 = 1^{9} \cdot 1 = 1$$
\rozwStop
\odpStart
$1$
\odpStop
\testStart
A.$1$ B.$\pi$ C.$0$ D.$\frac{271}{103}$ E.$\frac{103}{271}$
F.$-\frac{271}{103}$ G.$-1$
H.$(\frac{271}{103})^{9}$
I.$(\frac{103}{271})^{9}$
\testStop
\kluczStart
A
\kluczStop



\zadStart{Zadanie z Wikieł Z 1.1 d) moja wersja nr 818}

Obliczyć wartość wyrażenia $(\frac{271}{107})^{9} \cdot (\frac{107}{271})^{9} \cdot \pi^{0}$.
\zadStop
\rozwStart{Patryk Wirkus}{Martyna Czarnobaj}
$$(\frac{271}{107})^{9} \cdot (\frac{107}{271})^{9} \cdot \pi^{0} = (\frac{271}{107} \cdot \frac{107}{271})^{9} \cdot 1 = 1^{9} \cdot 1 = 1$$
\rozwStop
\odpStart
$1$
\odpStop
\testStart
A.$1$ B.$\pi$ C.$0$ D.$\frac{271}{107}$ E.$\frac{107}{271}$
F.$-\frac{271}{107}$ G.$-1$
H.$(\frac{271}{107})^{9}$
I.$(\frac{107}{271})^{9}$
\testStop
\kluczStart
A
\kluczStop



\zadStart{Zadanie z Wikieł Z 1.1 d) moja wersja nr 819}

Obliczyć wartość wyrażenia $(\frac{271}{109})^{9} \cdot (\frac{109}{271})^{9} \cdot \pi^{0}$.
\zadStop
\rozwStart{Patryk Wirkus}{Martyna Czarnobaj}
$$(\frac{271}{109})^{9} \cdot (\frac{109}{271})^{9} \cdot \pi^{0} = (\frac{271}{109} \cdot \frac{109}{271})^{9} \cdot 1 = 1^{9} \cdot 1 = 1$$
\rozwStop
\odpStart
$1$
\odpStop
\testStart
A.$1$ B.$\pi$ C.$0$ D.$\frac{271}{109}$ E.$\frac{109}{271}$
F.$-\frac{271}{109}$ G.$-1$
H.$(\frac{271}{109})^{9}$
I.$(\frac{109}{271})^{9}$
\testStop
\kluczStart
A
\kluczStop



\zadStart{Zadanie z Wikieł Z 1.1 d) moja wersja nr 820}

Obliczyć wartość wyrażenia $(\frac{271}{113})^{9} \cdot (\frac{113}{271})^{9} \cdot \pi^{0}$.
\zadStop
\rozwStart{Patryk Wirkus}{Martyna Czarnobaj}
$$(\frac{271}{113})^{9} \cdot (\frac{113}{271})^{9} \cdot \pi^{0} = (\frac{271}{113} \cdot \frac{113}{271})^{9} \cdot 1 = 1^{9} \cdot 1 = 1$$
\rozwStop
\odpStart
$1$
\odpStop
\testStart
A.$1$ B.$\pi$ C.$0$ D.$\frac{271}{113}$ E.$\frac{113}{271}$
F.$-\frac{271}{113}$ G.$-1$
H.$(\frac{271}{113})^{9}$
I.$(\frac{113}{271})^{9}$
\testStop
\kluczStart
A
\kluczStop



\zadStart{Zadanie z Wikieł Z 1.1 d) moja wersja nr 821}

Obliczyć wartość wyrażenia $(\frac{271}{127})^{9} \cdot (\frac{127}{271})^{9} \cdot \pi^{0}$.
\zadStop
\rozwStart{Patryk Wirkus}{Martyna Czarnobaj}
$$(\frac{271}{127})^{9} \cdot (\frac{127}{271})^{9} \cdot \pi^{0} = (\frac{271}{127} \cdot \frac{127}{271})^{9} \cdot 1 = 1^{9} \cdot 1 = 1$$
\rozwStop
\odpStart
$1$
\odpStop
\testStart
A.$1$ B.$\pi$ C.$0$ D.$\frac{271}{127}$ E.$\frac{127}{271}$
F.$-\frac{271}{127}$ G.$-1$
H.$(\frac{271}{127})^{9}$
I.$(\frac{127}{271})^{9}$
\testStop
\kluczStart
A
\kluczStop



\zadStart{Zadanie z Wikieł Z 1.1 d) moja wersja nr 822}

Obliczyć wartość wyrażenia $(\frac{271}{131})^{9} \cdot (\frac{131}{271})^{9} \cdot \pi^{0}$.
\zadStop
\rozwStart{Patryk Wirkus}{Martyna Czarnobaj}
$$(\frac{271}{131})^{9} \cdot (\frac{131}{271})^{9} \cdot \pi^{0} = (\frac{271}{131} \cdot \frac{131}{271})^{9} \cdot 1 = 1^{9} \cdot 1 = 1$$
\rozwStop
\odpStart
$1$
\odpStop
\testStart
A.$1$ B.$\pi$ C.$0$ D.$\frac{271}{131}$ E.$\frac{131}{271}$
F.$-\frac{271}{131}$ G.$-1$
H.$(\frac{271}{131})^{9}$
I.$(\frac{131}{271})^{9}$
\testStop
\kluczStart
A
\kluczStop



\zadStart{Zadanie z Wikieł Z 1.1 d) moja wersja nr 823}

Obliczyć wartość wyrażenia $(\frac{271}{137})^{9} \cdot (\frac{137}{271})^{9} \cdot \pi^{0}$.
\zadStop
\rozwStart{Patryk Wirkus}{Martyna Czarnobaj}
$$(\frac{271}{137})^{9} \cdot (\frac{137}{271})^{9} \cdot \pi^{0} = (\frac{271}{137} \cdot \frac{137}{271})^{9} \cdot 1 = 1^{9} \cdot 1 = 1$$
\rozwStop
\odpStart
$1$
\odpStop
\testStart
A.$1$ B.$\pi$ C.$0$ D.$\frac{271}{137}$ E.$\frac{137}{271}$
F.$-\frac{271}{137}$ G.$-1$
H.$(\frac{271}{137})^{9}$
I.$(\frac{137}{271})^{9}$
\testStop
\kluczStart
A
\kluczStop



\zadStart{Zadanie z Wikieł Z 1.1 d) moja wersja nr 824}

Obliczyć wartość wyrażenia $(\frac{271}{139})^{9} \cdot (\frac{139}{271})^{9} \cdot \pi^{0}$.
\zadStop
\rozwStart{Patryk Wirkus}{Martyna Czarnobaj}
$$(\frac{271}{139})^{9} \cdot (\frac{139}{271})^{9} \cdot \pi^{0} = (\frac{271}{139} \cdot \frac{139}{271})^{9} \cdot 1 = 1^{9} \cdot 1 = 1$$
\rozwStop
\odpStart
$1$
\odpStop
\testStart
A.$1$ B.$\pi$ C.$0$ D.$\frac{271}{139}$ E.$\frac{139}{271}$
F.$-\frac{271}{139}$ G.$-1$
H.$(\frac{271}{139})^{9}$
I.$(\frac{139}{271})^{9}$
\testStop
\kluczStart
A
\kluczStop



\zadStart{Zadanie z Wikieł Z 1.1 d) moja wersja nr 825}

Obliczyć wartość wyrażenia $(\frac{277}{103})^{9} \cdot (\frac{103}{277})^{9} \cdot \pi^{0}$.
\zadStop
\rozwStart{Patryk Wirkus}{Martyna Czarnobaj}
$$(\frac{277}{103})^{9} \cdot (\frac{103}{277})^{9} \cdot \pi^{0} = (\frac{277}{103} \cdot \frac{103}{277})^{9} \cdot 1 = 1^{9} \cdot 1 = 1$$
\rozwStop
\odpStart
$1$
\odpStop
\testStart
A.$1$ B.$\pi$ C.$0$ D.$\frac{277}{103}$ E.$\frac{103}{277}$
F.$-\frac{277}{103}$ G.$-1$
H.$(\frac{277}{103})^{9}$
I.$(\frac{103}{277})^{9}$
\testStop
\kluczStart
A
\kluczStop



\zadStart{Zadanie z Wikieł Z 1.1 d) moja wersja nr 826}

Obliczyć wartość wyrażenia $(\frac{277}{107})^{9} \cdot (\frac{107}{277})^{9} \cdot \pi^{0}$.
\zadStop
\rozwStart{Patryk Wirkus}{Martyna Czarnobaj}
$$(\frac{277}{107})^{9} \cdot (\frac{107}{277})^{9} \cdot \pi^{0} = (\frac{277}{107} \cdot \frac{107}{277})^{9} \cdot 1 = 1^{9} \cdot 1 = 1$$
\rozwStop
\odpStart
$1$
\odpStop
\testStart
A.$1$ B.$\pi$ C.$0$ D.$\frac{277}{107}$ E.$\frac{107}{277}$
F.$-\frac{277}{107}$ G.$-1$
H.$(\frac{277}{107})^{9}$
I.$(\frac{107}{277})^{9}$
\testStop
\kluczStart
A
\kluczStop



\zadStart{Zadanie z Wikieł Z 1.1 d) moja wersja nr 827}

Obliczyć wartość wyrażenia $(\frac{277}{109})^{9} \cdot (\frac{109}{277})^{9} \cdot \pi^{0}$.
\zadStop
\rozwStart{Patryk Wirkus}{Martyna Czarnobaj}
$$(\frac{277}{109})^{9} \cdot (\frac{109}{277})^{9} \cdot \pi^{0} = (\frac{277}{109} \cdot \frac{109}{277})^{9} \cdot 1 = 1^{9} \cdot 1 = 1$$
\rozwStop
\odpStart
$1$
\odpStop
\testStart
A.$1$ B.$\pi$ C.$0$ D.$\frac{277}{109}$ E.$\frac{109}{277}$
F.$-\frac{277}{109}$ G.$-1$
H.$(\frac{277}{109})^{9}$
I.$(\frac{109}{277})^{9}$
\testStop
\kluczStart
A
\kluczStop



\zadStart{Zadanie z Wikieł Z 1.1 d) moja wersja nr 828}

Obliczyć wartość wyrażenia $(\frac{277}{113})^{9} \cdot (\frac{113}{277})^{9} \cdot \pi^{0}$.
\zadStop
\rozwStart{Patryk Wirkus}{Martyna Czarnobaj}
$$(\frac{277}{113})^{9} \cdot (\frac{113}{277})^{9} \cdot \pi^{0} = (\frac{277}{113} \cdot \frac{113}{277})^{9} \cdot 1 = 1^{9} \cdot 1 = 1$$
\rozwStop
\odpStart
$1$
\odpStop
\testStart
A.$1$ B.$\pi$ C.$0$ D.$\frac{277}{113}$ E.$\frac{113}{277}$
F.$-\frac{277}{113}$ G.$-1$
H.$(\frac{277}{113})^{9}$
I.$(\frac{113}{277})^{9}$
\testStop
\kluczStart
A
\kluczStop



\zadStart{Zadanie z Wikieł Z 1.1 d) moja wersja nr 829}

Obliczyć wartość wyrażenia $(\frac{277}{127})^{9} \cdot (\frac{127}{277})^{9} \cdot \pi^{0}$.
\zadStop
\rozwStart{Patryk Wirkus}{Martyna Czarnobaj}
$$(\frac{277}{127})^{9} \cdot (\frac{127}{277})^{9} \cdot \pi^{0} = (\frac{277}{127} \cdot \frac{127}{277})^{9} \cdot 1 = 1^{9} \cdot 1 = 1$$
\rozwStop
\odpStart
$1$
\odpStop
\testStart
A.$1$ B.$\pi$ C.$0$ D.$\frac{277}{127}$ E.$\frac{127}{277}$
F.$-\frac{277}{127}$ G.$-1$
H.$(\frac{277}{127})^{9}$
I.$(\frac{127}{277})^{9}$
\testStop
\kluczStart
A
\kluczStop



\zadStart{Zadanie z Wikieł Z 1.1 d) moja wersja nr 830}

Obliczyć wartość wyrażenia $(\frac{277}{131})^{9} \cdot (\frac{131}{277})^{9} \cdot \pi^{0}$.
\zadStop
\rozwStart{Patryk Wirkus}{Martyna Czarnobaj}
$$(\frac{277}{131})^{9} \cdot (\frac{131}{277})^{9} \cdot \pi^{0} = (\frac{277}{131} \cdot \frac{131}{277})^{9} \cdot 1 = 1^{9} \cdot 1 = 1$$
\rozwStop
\odpStart
$1$
\odpStop
\testStart
A.$1$ B.$\pi$ C.$0$ D.$\frac{277}{131}$ E.$\frac{131}{277}$
F.$-\frac{277}{131}$ G.$-1$
H.$(\frac{277}{131})^{9}$
I.$(\frac{131}{277})^{9}$
\testStop
\kluczStart
A
\kluczStop



\zadStart{Zadanie z Wikieł Z 1.1 d) moja wersja nr 831}

Obliczyć wartość wyrażenia $(\frac{277}{137})^{9} \cdot (\frac{137}{277})^{9} \cdot \pi^{0}$.
\zadStop
\rozwStart{Patryk Wirkus}{Martyna Czarnobaj}
$$(\frac{277}{137})^{9} \cdot (\frac{137}{277})^{9} \cdot \pi^{0} = (\frac{277}{137} \cdot \frac{137}{277})^{9} \cdot 1 = 1^{9} \cdot 1 = 1$$
\rozwStop
\odpStart
$1$
\odpStop
\testStart
A.$1$ B.$\pi$ C.$0$ D.$\frac{277}{137}$ E.$\frac{137}{277}$
F.$-\frac{277}{137}$ G.$-1$
H.$(\frac{277}{137})^{9}$
I.$(\frac{137}{277})^{9}$
\testStop
\kluczStart
A
\kluczStop



\zadStart{Zadanie z Wikieł Z 1.1 d) moja wersja nr 832}

Obliczyć wartość wyrażenia $(\frac{277}{139})^{9} \cdot (\frac{139}{277})^{9} \cdot \pi^{0}$.
\zadStop
\rozwStart{Patryk Wirkus}{Martyna Czarnobaj}
$$(\frac{277}{139})^{9} \cdot (\frac{139}{277})^{9} \cdot \pi^{0} = (\frac{277}{139} \cdot \frac{139}{277})^{9} \cdot 1 = 1^{9} \cdot 1 = 1$$
\rozwStop
\odpStart
$1$
\odpStop
\testStart
A.$1$ B.$\pi$ C.$0$ D.$\frac{277}{139}$ E.$\frac{139}{277}$
F.$-\frac{277}{139}$ G.$-1$
H.$(\frac{277}{139})^{9}$
I.$(\frac{139}{277})^{9}$
\testStop
\kluczStart
A
\kluczStop



\zadStart{Zadanie z Wikieł Z 1.1 d) moja wersja nr 833}

Obliczyć wartość wyrażenia $(\frac{149}{103})^{10} \cdot (\frac{103}{149})^{10} \cdot \pi^{0}$.
\zadStop
\rozwStart{Patryk Wirkus}{Martyna Czarnobaj}
$$(\frac{149}{103})^{10} \cdot (\frac{103}{149})^{10} \cdot \pi^{0} = (\frac{149}{103} \cdot \frac{103}{149})^{10} \cdot 1 = 1^{10} \cdot 1 = 1$$
\rozwStop
\odpStart
$1$
\odpStop
\testStart
A.$1$ B.$\pi$ C.$0$ D.$\frac{149}{103}$ E.$\frac{103}{149}$
F.$-\frac{149}{103}$ G.$-1$
H.$(\frac{149}{103})^{10}$
I.$(\frac{103}{149})^{10}$
\testStop
\kluczStart
A
\kluczStop



\zadStart{Zadanie z Wikieł Z 1.1 d) moja wersja nr 834}

Obliczyć wartość wyrażenia $(\frac{149}{107})^{10} \cdot (\frac{107}{149})^{10} \cdot \pi^{0}$.
\zadStop
\rozwStart{Patryk Wirkus}{Martyna Czarnobaj}
$$(\frac{149}{107})^{10} \cdot (\frac{107}{149})^{10} \cdot \pi^{0} = (\frac{149}{107} \cdot \frac{107}{149})^{10} \cdot 1 = 1^{10} \cdot 1 = 1$$
\rozwStop
\odpStart
$1$
\odpStop
\testStart
A.$1$ B.$\pi$ C.$0$ D.$\frac{149}{107}$ E.$\frac{107}{149}$
F.$-\frac{149}{107}$ G.$-1$
H.$(\frac{149}{107})^{10}$
I.$(\frac{107}{149})^{10}$
\testStop
\kluczStart
A
\kluczStop



\zadStart{Zadanie z Wikieł Z 1.1 d) moja wersja nr 835}

Obliczyć wartość wyrażenia $(\frac{149}{109})^{10} \cdot (\frac{109}{149})^{10} \cdot \pi^{0}$.
\zadStop
\rozwStart{Patryk Wirkus}{Martyna Czarnobaj}
$$(\frac{149}{109})^{10} \cdot (\frac{109}{149})^{10} \cdot \pi^{0} = (\frac{149}{109} \cdot \frac{109}{149})^{10} \cdot 1 = 1^{10} \cdot 1 = 1$$
\rozwStop
\odpStart
$1$
\odpStop
\testStart
A.$1$ B.$\pi$ C.$0$ D.$\frac{149}{109}$ E.$\frac{109}{149}$
F.$-\frac{149}{109}$ G.$-1$
H.$(\frac{149}{109})^{10}$
I.$(\frac{109}{149})^{10}$
\testStop
\kluczStart
A
\kluczStop



\zadStart{Zadanie z Wikieł Z 1.1 d) moja wersja nr 836}

Obliczyć wartość wyrażenia $(\frac{149}{113})^{10} \cdot (\frac{113}{149})^{10} \cdot \pi^{0}$.
\zadStop
\rozwStart{Patryk Wirkus}{Martyna Czarnobaj}
$$(\frac{149}{113})^{10} \cdot (\frac{113}{149})^{10} \cdot \pi^{0} = (\frac{149}{113} \cdot \frac{113}{149})^{10} \cdot 1 = 1^{10} \cdot 1 = 1$$
\rozwStop
\odpStart
$1$
\odpStop
\testStart
A.$1$ B.$\pi$ C.$0$ D.$\frac{149}{113}$ E.$\frac{113}{149}$
F.$-\frac{149}{113}$ G.$-1$
H.$(\frac{149}{113})^{10}$
I.$(\frac{113}{149})^{10}$
\testStop
\kluczStart
A
\kluczStop



\zadStart{Zadanie z Wikieł Z 1.1 d) moja wersja nr 837}

Obliczyć wartość wyrażenia $(\frac{149}{127})^{10} \cdot (\frac{127}{149})^{10} \cdot \pi^{0}$.
\zadStop
\rozwStart{Patryk Wirkus}{Martyna Czarnobaj}
$$(\frac{149}{127})^{10} \cdot (\frac{127}{149})^{10} \cdot \pi^{0} = (\frac{149}{127} \cdot \frac{127}{149})^{10} \cdot 1 = 1^{10} \cdot 1 = 1$$
\rozwStop
\odpStart
$1$
\odpStop
\testStart
A.$1$ B.$\pi$ C.$0$ D.$\frac{149}{127}$ E.$\frac{127}{149}$
F.$-\frac{149}{127}$ G.$-1$
H.$(\frac{149}{127})^{10}$
I.$(\frac{127}{149})^{10}$
\testStop
\kluczStart
A
\kluczStop



\zadStart{Zadanie z Wikieł Z 1.1 d) moja wersja nr 838}

Obliczyć wartość wyrażenia $(\frac{149}{131})^{10} \cdot (\frac{131}{149})^{10} \cdot \pi^{0}$.
\zadStop
\rozwStart{Patryk Wirkus}{Martyna Czarnobaj}
$$(\frac{149}{131})^{10} \cdot (\frac{131}{149})^{10} \cdot \pi^{0} = (\frac{149}{131} \cdot \frac{131}{149})^{10} \cdot 1 = 1^{10} \cdot 1 = 1$$
\rozwStop
\odpStart
$1$
\odpStop
\testStart
A.$1$ B.$\pi$ C.$0$ D.$\frac{149}{131}$ E.$\frac{131}{149}$
F.$-\frac{149}{131}$ G.$-1$
H.$(\frac{149}{131})^{10}$
I.$(\frac{131}{149})^{10}$
\testStop
\kluczStart
A
\kluczStop



\zadStart{Zadanie z Wikieł Z 1.1 d) moja wersja nr 839}

Obliczyć wartość wyrażenia $(\frac{149}{137})^{10} \cdot (\frac{137}{149})^{10} \cdot \pi^{0}$.
\zadStop
\rozwStart{Patryk Wirkus}{Martyna Czarnobaj}
$$(\frac{149}{137})^{10} \cdot (\frac{137}{149})^{10} \cdot \pi^{0} = (\frac{149}{137} \cdot \frac{137}{149})^{10} \cdot 1 = 1^{10} \cdot 1 = 1$$
\rozwStop
\odpStart
$1$
\odpStop
\testStart
A.$1$ B.$\pi$ C.$0$ D.$\frac{149}{137}$ E.$\frac{137}{149}$
F.$-\frac{149}{137}$ G.$-1$
H.$(\frac{149}{137})^{10}$
I.$(\frac{137}{149})^{10}$
\testStop
\kluczStart
A
\kluczStop



\zadStart{Zadanie z Wikieł Z 1.1 d) moja wersja nr 840}

Obliczyć wartość wyrażenia $(\frac{149}{139})^{10} \cdot (\frac{139}{149})^{10} \cdot \pi^{0}$.
\zadStop
\rozwStart{Patryk Wirkus}{Martyna Czarnobaj}
$$(\frac{149}{139})^{10} \cdot (\frac{139}{149})^{10} \cdot \pi^{0} = (\frac{149}{139} \cdot \frac{139}{149})^{10} \cdot 1 = 1^{10} \cdot 1 = 1$$
\rozwStop
\odpStart
$1$
\odpStop
\testStart
A.$1$ B.$\pi$ C.$0$ D.$\frac{149}{139}$ E.$\frac{139}{149}$
F.$-\frac{149}{139}$ G.$-1$
H.$(\frac{149}{139})^{10}$
I.$(\frac{139}{149})^{10}$
\testStop
\kluczStart
A
\kluczStop



\zadStart{Zadanie z Wikieł Z 1.1 d) moja wersja nr 841}

Obliczyć wartość wyrażenia $(\frac{151}{103})^{10} \cdot (\frac{103}{151})^{10} \cdot \pi^{0}$.
\zadStop
\rozwStart{Patryk Wirkus}{Martyna Czarnobaj}
$$(\frac{151}{103})^{10} \cdot (\frac{103}{151})^{10} \cdot \pi^{0} = (\frac{151}{103} \cdot \frac{103}{151})^{10} \cdot 1 = 1^{10} \cdot 1 = 1$$
\rozwStop
\odpStart
$1$
\odpStop
\testStart
A.$1$ B.$\pi$ C.$0$ D.$\frac{151}{103}$ E.$\frac{103}{151}$
F.$-\frac{151}{103}$ G.$-1$
H.$(\frac{151}{103})^{10}$
I.$(\frac{103}{151})^{10}$
\testStop
\kluczStart
A
\kluczStop



\zadStart{Zadanie z Wikieł Z 1.1 d) moja wersja nr 842}

Obliczyć wartość wyrażenia $(\frac{151}{107})^{10} \cdot (\frac{107}{151})^{10} \cdot \pi^{0}$.
\zadStop
\rozwStart{Patryk Wirkus}{Martyna Czarnobaj}
$$(\frac{151}{107})^{10} \cdot (\frac{107}{151})^{10} \cdot \pi^{0} = (\frac{151}{107} \cdot \frac{107}{151})^{10} \cdot 1 = 1^{10} \cdot 1 = 1$$
\rozwStop
\odpStart
$1$
\odpStop
\testStart
A.$1$ B.$\pi$ C.$0$ D.$\frac{151}{107}$ E.$\frac{107}{151}$
F.$-\frac{151}{107}$ G.$-1$
H.$(\frac{151}{107})^{10}$
I.$(\frac{107}{151})^{10}$
\testStop
\kluczStart
A
\kluczStop



\zadStart{Zadanie z Wikieł Z 1.1 d) moja wersja nr 843}

Obliczyć wartość wyrażenia $(\frac{151}{109})^{10} \cdot (\frac{109}{151})^{10} \cdot \pi^{0}$.
\zadStop
\rozwStart{Patryk Wirkus}{Martyna Czarnobaj}
$$(\frac{151}{109})^{10} \cdot (\frac{109}{151})^{10} \cdot \pi^{0} = (\frac{151}{109} \cdot \frac{109}{151})^{10} \cdot 1 = 1^{10} \cdot 1 = 1$$
\rozwStop
\odpStart
$1$
\odpStop
\testStart
A.$1$ B.$\pi$ C.$0$ D.$\frac{151}{109}$ E.$\frac{109}{151}$
F.$-\frac{151}{109}$ G.$-1$
H.$(\frac{151}{109})^{10}$
I.$(\frac{109}{151})^{10}$
\testStop
\kluczStart
A
\kluczStop



\zadStart{Zadanie z Wikieł Z 1.1 d) moja wersja nr 844}

Obliczyć wartość wyrażenia $(\frac{151}{113})^{10} \cdot (\frac{113}{151})^{10} \cdot \pi^{0}$.
\zadStop
\rozwStart{Patryk Wirkus}{Martyna Czarnobaj}
$$(\frac{151}{113})^{10} \cdot (\frac{113}{151})^{10} \cdot \pi^{0} = (\frac{151}{113} \cdot \frac{113}{151})^{10} \cdot 1 = 1^{10} \cdot 1 = 1$$
\rozwStop
\odpStart
$1$
\odpStop
\testStart
A.$1$ B.$\pi$ C.$0$ D.$\frac{151}{113}$ E.$\frac{113}{151}$
F.$-\frac{151}{113}$ G.$-1$
H.$(\frac{151}{113})^{10}$
I.$(\frac{113}{151})^{10}$
\testStop
\kluczStart
A
\kluczStop



\zadStart{Zadanie z Wikieł Z 1.1 d) moja wersja nr 845}

Obliczyć wartość wyrażenia $(\frac{151}{127})^{10} \cdot (\frac{127}{151})^{10} \cdot \pi^{0}$.
\zadStop
\rozwStart{Patryk Wirkus}{Martyna Czarnobaj}
$$(\frac{151}{127})^{10} \cdot (\frac{127}{151})^{10} \cdot \pi^{0} = (\frac{151}{127} \cdot \frac{127}{151})^{10} \cdot 1 = 1^{10} \cdot 1 = 1$$
\rozwStop
\odpStart
$1$
\odpStop
\testStart
A.$1$ B.$\pi$ C.$0$ D.$\frac{151}{127}$ E.$\frac{127}{151}$
F.$-\frac{151}{127}$ G.$-1$
H.$(\frac{151}{127})^{10}$
I.$(\frac{127}{151})^{10}$
\testStop
\kluczStart
A
\kluczStop



\zadStart{Zadanie z Wikieł Z 1.1 d) moja wersja nr 846}

Obliczyć wartość wyrażenia $(\frac{151}{131})^{10} \cdot (\frac{131}{151})^{10} \cdot \pi^{0}$.
\zadStop
\rozwStart{Patryk Wirkus}{Martyna Czarnobaj}
$$(\frac{151}{131})^{10} \cdot (\frac{131}{151})^{10} \cdot \pi^{0} = (\frac{151}{131} \cdot \frac{131}{151})^{10} \cdot 1 = 1^{10} \cdot 1 = 1$$
\rozwStop
\odpStart
$1$
\odpStop
\testStart
A.$1$ B.$\pi$ C.$0$ D.$\frac{151}{131}$ E.$\frac{131}{151}$
F.$-\frac{151}{131}$ G.$-1$
H.$(\frac{151}{131})^{10}$
I.$(\frac{131}{151})^{10}$
\testStop
\kluczStart
A
\kluczStop



\zadStart{Zadanie z Wikieł Z 1.1 d) moja wersja nr 847}

Obliczyć wartość wyrażenia $(\frac{151}{137})^{10} \cdot (\frac{137}{151})^{10} \cdot \pi^{0}$.
\zadStop
\rozwStart{Patryk Wirkus}{Martyna Czarnobaj}
$$(\frac{151}{137})^{10} \cdot (\frac{137}{151})^{10} \cdot \pi^{0} = (\frac{151}{137} \cdot \frac{137}{151})^{10} \cdot 1 = 1^{10} \cdot 1 = 1$$
\rozwStop
\odpStart
$1$
\odpStop
\testStart
A.$1$ B.$\pi$ C.$0$ D.$\frac{151}{137}$ E.$\frac{137}{151}$
F.$-\frac{151}{137}$ G.$-1$
H.$(\frac{151}{137})^{10}$
I.$(\frac{137}{151})^{10}$
\testStop
\kluczStart
A
\kluczStop



\zadStart{Zadanie z Wikieł Z 1.1 d) moja wersja nr 848}

Obliczyć wartość wyrażenia $(\frac{151}{139})^{10} \cdot (\frac{139}{151})^{10} \cdot \pi^{0}$.
\zadStop
\rozwStart{Patryk Wirkus}{Martyna Czarnobaj}
$$(\frac{151}{139})^{10} \cdot (\frac{139}{151})^{10} \cdot \pi^{0} = (\frac{151}{139} \cdot \frac{139}{151})^{10} \cdot 1 = 1^{10} \cdot 1 = 1$$
\rozwStop
\odpStart
$1$
\odpStop
\testStart
A.$1$ B.$\pi$ C.$0$ D.$\frac{151}{139}$ E.$\frac{139}{151}$
F.$-\frac{151}{139}$ G.$-1$
H.$(\frac{151}{139})^{10}$
I.$(\frac{139}{151})^{10}$
\testStop
\kluczStart
A
\kluczStop



\zadStart{Zadanie z Wikieł Z 1.1 d) moja wersja nr 849}

Obliczyć wartość wyrażenia $(\frac{157}{103})^{10} \cdot (\frac{103}{157})^{10} \cdot \pi^{0}$.
\zadStop
\rozwStart{Patryk Wirkus}{Martyna Czarnobaj}
$$(\frac{157}{103})^{10} \cdot (\frac{103}{157})^{10} \cdot \pi^{0} = (\frac{157}{103} \cdot \frac{103}{157})^{10} \cdot 1 = 1^{10} \cdot 1 = 1$$
\rozwStop
\odpStart
$1$
\odpStop
\testStart
A.$1$ B.$\pi$ C.$0$ D.$\frac{157}{103}$ E.$\frac{103}{157}$
F.$-\frac{157}{103}$ G.$-1$
H.$(\frac{157}{103})^{10}$
I.$(\frac{103}{157})^{10}$
\testStop
\kluczStart
A
\kluczStop



\zadStart{Zadanie z Wikieł Z 1.1 d) moja wersja nr 850}

Obliczyć wartość wyrażenia $(\frac{157}{107})^{10} \cdot (\frac{107}{157})^{10} \cdot \pi^{0}$.
\zadStop
\rozwStart{Patryk Wirkus}{Martyna Czarnobaj}
$$(\frac{157}{107})^{10} \cdot (\frac{107}{157})^{10} \cdot \pi^{0} = (\frac{157}{107} \cdot \frac{107}{157})^{10} \cdot 1 = 1^{10} \cdot 1 = 1$$
\rozwStop
\odpStart
$1$
\odpStop
\testStart
A.$1$ B.$\pi$ C.$0$ D.$\frac{157}{107}$ E.$\frac{107}{157}$
F.$-\frac{157}{107}$ G.$-1$
H.$(\frac{157}{107})^{10}$
I.$(\frac{107}{157})^{10}$
\testStop
\kluczStart
A
\kluczStop



\zadStart{Zadanie z Wikieł Z 1.1 d) moja wersja nr 851}

Obliczyć wartość wyrażenia $(\frac{157}{109})^{10} \cdot (\frac{109}{157})^{10} \cdot \pi^{0}$.
\zadStop
\rozwStart{Patryk Wirkus}{Martyna Czarnobaj}
$$(\frac{157}{109})^{10} \cdot (\frac{109}{157})^{10} \cdot \pi^{0} = (\frac{157}{109} \cdot \frac{109}{157})^{10} \cdot 1 = 1^{10} \cdot 1 = 1$$
\rozwStop
\odpStart
$1$
\odpStop
\testStart
A.$1$ B.$\pi$ C.$0$ D.$\frac{157}{109}$ E.$\frac{109}{157}$
F.$-\frac{157}{109}$ G.$-1$
H.$(\frac{157}{109})^{10}$
I.$(\frac{109}{157})^{10}$
\testStop
\kluczStart
A
\kluczStop



\zadStart{Zadanie z Wikieł Z 1.1 d) moja wersja nr 852}

Obliczyć wartość wyrażenia $(\frac{157}{113})^{10} \cdot (\frac{113}{157})^{10} \cdot \pi^{0}$.
\zadStop
\rozwStart{Patryk Wirkus}{Martyna Czarnobaj}
$$(\frac{157}{113})^{10} \cdot (\frac{113}{157})^{10} \cdot \pi^{0} = (\frac{157}{113} \cdot \frac{113}{157})^{10} \cdot 1 = 1^{10} \cdot 1 = 1$$
\rozwStop
\odpStart
$1$
\odpStop
\testStart
A.$1$ B.$\pi$ C.$0$ D.$\frac{157}{113}$ E.$\frac{113}{157}$
F.$-\frac{157}{113}$ G.$-1$
H.$(\frac{157}{113})^{10}$
I.$(\frac{113}{157})^{10}$
\testStop
\kluczStart
A
\kluczStop



\zadStart{Zadanie z Wikieł Z 1.1 d) moja wersja nr 853}

Obliczyć wartość wyrażenia $(\frac{157}{127})^{10} \cdot (\frac{127}{157})^{10} \cdot \pi^{0}$.
\zadStop
\rozwStart{Patryk Wirkus}{Martyna Czarnobaj}
$$(\frac{157}{127})^{10} \cdot (\frac{127}{157})^{10} \cdot \pi^{0} = (\frac{157}{127} \cdot \frac{127}{157})^{10} \cdot 1 = 1^{10} \cdot 1 = 1$$
\rozwStop
\odpStart
$1$
\odpStop
\testStart
A.$1$ B.$\pi$ C.$0$ D.$\frac{157}{127}$ E.$\frac{127}{157}$
F.$-\frac{157}{127}$ G.$-1$
H.$(\frac{157}{127})^{10}$
I.$(\frac{127}{157})^{10}$
\testStop
\kluczStart
A
\kluczStop



\zadStart{Zadanie z Wikieł Z 1.1 d) moja wersja nr 854}

Obliczyć wartość wyrażenia $(\frac{157}{131})^{10} \cdot (\frac{131}{157})^{10} \cdot \pi^{0}$.
\zadStop
\rozwStart{Patryk Wirkus}{Martyna Czarnobaj}
$$(\frac{157}{131})^{10} \cdot (\frac{131}{157})^{10} \cdot \pi^{0} = (\frac{157}{131} \cdot \frac{131}{157})^{10} \cdot 1 = 1^{10} \cdot 1 = 1$$
\rozwStop
\odpStart
$1$
\odpStop
\testStart
A.$1$ B.$\pi$ C.$0$ D.$\frac{157}{131}$ E.$\frac{131}{157}$
F.$-\frac{157}{131}$ G.$-1$
H.$(\frac{157}{131})^{10}$
I.$(\frac{131}{157})^{10}$
\testStop
\kluczStart
A
\kluczStop



\zadStart{Zadanie z Wikieł Z 1.1 d) moja wersja nr 855}

Obliczyć wartość wyrażenia $(\frac{157}{137})^{10} \cdot (\frac{137}{157})^{10} \cdot \pi^{0}$.
\zadStop
\rozwStart{Patryk Wirkus}{Martyna Czarnobaj}
$$(\frac{157}{137})^{10} \cdot (\frac{137}{157})^{10} \cdot \pi^{0} = (\frac{157}{137} \cdot \frac{137}{157})^{10} \cdot 1 = 1^{10} \cdot 1 = 1$$
\rozwStop
\odpStart
$1$
\odpStop
\testStart
A.$1$ B.$\pi$ C.$0$ D.$\frac{157}{137}$ E.$\frac{137}{157}$
F.$-\frac{157}{137}$ G.$-1$
H.$(\frac{157}{137})^{10}$
I.$(\frac{137}{157})^{10}$
\testStop
\kluczStart
A
\kluczStop



\zadStart{Zadanie z Wikieł Z 1.1 d) moja wersja nr 856}

Obliczyć wartość wyrażenia $(\frac{157}{139})^{10} \cdot (\frac{139}{157})^{10} \cdot \pi^{0}$.
\zadStop
\rozwStart{Patryk Wirkus}{Martyna Czarnobaj}
$$(\frac{157}{139})^{10} \cdot (\frac{139}{157})^{10} \cdot \pi^{0} = (\frac{157}{139} \cdot \frac{139}{157})^{10} \cdot 1 = 1^{10} \cdot 1 = 1$$
\rozwStop
\odpStart
$1$
\odpStop
\testStart
A.$1$ B.$\pi$ C.$0$ D.$\frac{157}{139}$ E.$\frac{139}{157}$
F.$-\frac{157}{139}$ G.$-1$
H.$(\frac{157}{139})^{10}$
I.$(\frac{139}{157})^{10}$
\testStop
\kluczStart
A
\kluczStop



\zadStart{Zadanie z Wikieł Z 1.1 d) moja wersja nr 857}

Obliczyć wartość wyrażenia $(\frac{163}{103})^{10} \cdot (\frac{103}{163})^{10} \cdot \pi^{0}$.
\zadStop
\rozwStart{Patryk Wirkus}{Martyna Czarnobaj}
$$(\frac{163}{103})^{10} \cdot (\frac{103}{163})^{10} \cdot \pi^{0} = (\frac{163}{103} \cdot \frac{103}{163})^{10} \cdot 1 = 1^{10} \cdot 1 = 1$$
\rozwStop
\odpStart
$1$
\odpStop
\testStart
A.$1$ B.$\pi$ C.$0$ D.$\frac{163}{103}$ E.$\frac{103}{163}$
F.$-\frac{163}{103}$ G.$-1$
H.$(\frac{163}{103})^{10}$
I.$(\frac{103}{163})^{10}$
\testStop
\kluczStart
A
\kluczStop



\zadStart{Zadanie z Wikieł Z 1.1 d) moja wersja nr 858}

Obliczyć wartość wyrażenia $(\frac{163}{107})^{10} \cdot (\frac{107}{163})^{10} \cdot \pi^{0}$.
\zadStop
\rozwStart{Patryk Wirkus}{Martyna Czarnobaj}
$$(\frac{163}{107})^{10} \cdot (\frac{107}{163})^{10} \cdot \pi^{0} = (\frac{163}{107} \cdot \frac{107}{163})^{10} \cdot 1 = 1^{10} \cdot 1 = 1$$
\rozwStop
\odpStart
$1$
\odpStop
\testStart
A.$1$ B.$\pi$ C.$0$ D.$\frac{163}{107}$ E.$\frac{107}{163}$
F.$-\frac{163}{107}$ G.$-1$
H.$(\frac{163}{107})^{10}$
I.$(\frac{107}{163})^{10}$
\testStop
\kluczStart
A
\kluczStop



\zadStart{Zadanie z Wikieł Z 1.1 d) moja wersja nr 859}

Obliczyć wartość wyrażenia $(\frac{163}{109})^{10} \cdot (\frac{109}{163})^{10} \cdot \pi^{0}$.
\zadStop
\rozwStart{Patryk Wirkus}{Martyna Czarnobaj}
$$(\frac{163}{109})^{10} \cdot (\frac{109}{163})^{10} \cdot \pi^{0} = (\frac{163}{109} \cdot \frac{109}{163})^{10} \cdot 1 = 1^{10} \cdot 1 = 1$$
\rozwStop
\odpStart
$1$
\odpStop
\testStart
A.$1$ B.$\pi$ C.$0$ D.$\frac{163}{109}$ E.$\frac{109}{163}$
F.$-\frac{163}{109}$ G.$-1$
H.$(\frac{163}{109})^{10}$
I.$(\frac{109}{163})^{10}$
\testStop
\kluczStart
A
\kluczStop



\zadStart{Zadanie z Wikieł Z 1.1 d) moja wersja nr 860}

Obliczyć wartość wyrażenia $(\frac{163}{113})^{10} \cdot (\frac{113}{163})^{10} \cdot \pi^{0}$.
\zadStop
\rozwStart{Patryk Wirkus}{Martyna Czarnobaj}
$$(\frac{163}{113})^{10} \cdot (\frac{113}{163})^{10} \cdot \pi^{0} = (\frac{163}{113} \cdot \frac{113}{163})^{10} \cdot 1 = 1^{10} \cdot 1 = 1$$
\rozwStop
\odpStart
$1$
\odpStop
\testStart
A.$1$ B.$\pi$ C.$0$ D.$\frac{163}{113}$ E.$\frac{113}{163}$
F.$-\frac{163}{113}$ G.$-1$
H.$(\frac{163}{113})^{10}$
I.$(\frac{113}{163})^{10}$
\testStop
\kluczStart
A
\kluczStop



\zadStart{Zadanie z Wikieł Z 1.1 d) moja wersja nr 861}

Obliczyć wartość wyrażenia $(\frac{163}{127})^{10} \cdot (\frac{127}{163})^{10} \cdot \pi^{0}$.
\zadStop
\rozwStart{Patryk Wirkus}{Martyna Czarnobaj}
$$(\frac{163}{127})^{10} \cdot (\frac{127}{163})^{10} \cdot \pi^{0} = (\frac{163}{127} \cdot \frac{127}{163})^{10} \cdot 1 = 1^{10} \cdot 1 = 1$$
\rozwStop
\odpStart
$1$
\odpStop
\testStart
A.$1$ B.$\pi$ C.$0$ D.$\frac{163}{127}$ E.$\frac{127}{163}$
F.$-\frac{163}{127}$ G.$-1$
H.$(\frac{163}{127})^{10}$
I.$(\frac{127}{163})^{10}$
\testStop
\kluczStart
A
\kluczStop



\zadStart{Zadanie z Wikieł Z 1.1 d) moja wersja nr 862}

Obliczyć wartość wyrażenia $(\frac{163}{131})^{10} \cdot (\frac{131}{163})^{10} \cdot \pi^{0}$.
\zadStop
\rozwStart{Patryk Wirkus}{Martyna Czarnobaj}
$$(\frac{163}{131})^{10} \cdot (\frac{131}{163})^{10} \cdot \pi^{0} = (\frac{163}{131} \cdot \frac{131}{163})^{10} \cdot 1 = 1^{10} \cdot 1 = 1$$
\rozwStop
\odpStart
$1$
\odpStop
\testStart
A.$1$ B.$\pi$ C.$0$ D.$\frac{163}{131}$ E.$\frac{131}{163}$
F.$-\frac{163}{131}$ G.$-1$
H.$(\frac{163}{131})^{10}$
I.$(\frac{131}{163})^{10}$
\testStop
\kluczStart
A
\kluczStop



\zadStart{Zadanie z Wikieł Z 1.1 d) moja wersja nr 863}

Obliczyć wartość wyrażenia $(\frac{163}{137})^{10} \cdot (\frac{137}{163})^{10} \cdot \pi^{0}$.
\zadStop
\rozwStart{Patryk Wirkus}{Martyna Czarnobaj}
$$(\frac{163}{137})^{10} \cdot (\frac{137}{163})^{10} \cdot \pi^{0} = (\frac{163}{137} \cdot \frac{137}{163})^{10} \cdot 1 = 1^{10} \cdot 1 = 1$$
\rozwStop
\odpStart
$1$
\odpStop
\testStart
A.$1$ B.$\pi$ C.$0$ D.$\frac{163}{137}$ E.$\frac{137}{163}$
F.$-\frac{163}{137}$ G.$-1$
H.$(\frac{163}{137})^{10}$
I.$(\frac{137}{163})^{10}$
\testStop
\kluczStart
A
\kluczStop



\zadStart{Zadanie z Wikieł Z 1.1 d) moja wersja nr 864}

Obliczyć wartość wyrażenia $(\frac{163}{139})^{10} \cdot (\frac{139}{163})^{10} \cdot \pi^{0}$.
\zadStop
\rozwStart{Patryk Wirkus}{Martyna Czarnobaj}
$$(\frac{163}{139})^{10} \cdot (\frac{139}{163})^{10} \cdot \pi^{0} = (\frac{163}{139} \cdot \frac{139}{163})^{10} \cdot 1 = 1^{10} \cdot 1 = 1$$
\rozwStop
\odpStart
$1$
\odpStop
\testStart
A.$1$ B.$\pi$ C.$0$ D.$\frac{163}{139}$ E.$\frac{139}{163}$
F.$-\frac{163}{139}$ G.$-1$
H.$(\frac{163}{139})^{10}$
I.$(\frac{139}{163})^{10}$
\testStop
\kluczStart
A
\kluczStop



\zadStart{Zadanie z Wikieł Z 1.1 d) moja wersja nr 865}

Obliczyć wartość wyrażenia $(\frac{167}{103})^{10} \cdot (\frac{103}{167})^{10} \cdot \pi^{0}$.
\zadStop
\rozwStart{Patryk Wirkus}{Martyna Czarnobaj}
$$(\frac{167}{103})^{10} \cdot (\frac{103}{167})^{10} \cdot \pi^{0} = (\frac{167}{103} \cdot \frac{103}{167})^{10} \cdot 1 = 1^{10} \cdot 1 = 1$$
\rozwStop
\odpStart
$1$
\odpStop
\testStart
A.$1$ B.$\pi$ C.$0$ D.$\frac{167}{103}$ E.$\frac{103}{167}$
F.$-\frac{167}{103}$ G.$-1$
H.$(\frac{167}{103})^{10}$
I.$(\frac{103}{167})^{10}$
\testStop
\kluczStart
A
\kluczStop



\zadStart{Zadanie z Wikieł Z 1.1 d) moja wersja nr 866}

Obliczyć wartość wyrażenia $(\frac{167}{107})^{10} \cdot (\frac{107}{167})^{10} \cdot \pi^{0}$.
\zadStop
\rozwStart{Patryk Wirkus}{Martyna Czarnobaj}
$$(\frac{167}{107})^{10} \cdot (\frac{107}{167})^{10} \cdot \pi^{0} = (\frac{167}{107} \cdot \frac{107}{167})^{10} \cdot 1 = 1^{10} \cdot 1 = 1$$
\rozwStop
\odpStart
$1$
\odpStop
\testStart
A.$1$ B.$\pi$ C.$0$ D.$\frac{167}{107}$ E.$\frac{107}{167}$
F.$-\frac{167}{107}$ G.$-1$
H.$(\frac{167}{107})^{10}$
I.$(\frac{107}{167})^{10}$
\testStop
\kluczStart
A
\kluczStop



\zadStart{Zadanie z Wikieł Z 1.1 d) moja wersja nr 867}

Obliczyć wartość wyrażenia $(\frac{167}{109})^{10} \cdot (\frac{109}{167})^{10} \cdot \pi^{0}$.
\zadStop
\rozwStart{Patryk Wirkus}{Martyna Czarnobaj}
$$(\frac{167}{109})^{10} \cdot (\frac{109}{167})^{10} \cdot \pi^{0} = (\frac{167}{109} \cdot \frac{109}{167})^{10} \cdot 1 = 1^{10} \cdot 1 = 1$$
\rozwStop
\odpStart
$1$
\odpStop
\testStart
A.$1$ B.$\pi$ C.$0$ D.$\frac{167}{109}$ E.$\frac{109}{167}$
F.$-\frac{167}{109}$ G.$-1$
H.$(\frac{167}{109})^{10}$
I.$(\frac{109}{167})^{10}$
\testStop
\kluczStart
A
\kluczStop



\zadStart{Zadanie z Wikieł Z 1.1 d) moja wersja nr 868}

Obliczyć wartość wyrażenia $(\frac{167}{113})^{10} \cdot (\frac{113}{167})^{10} \cdot \pi^{0}$.
\zadStop
\rozwStart{Patryk Wirkus}{Martyna Czarnobaj}
$$(\frac{167}{113})^{10} \cdot (\frac{113}{167})^{10} \cdot \pi^{0} = (\frac{167}{113} \cdot \frac{113}{167})^{10} \cdot 1 = 1^{10} \cdot 1 = 1$$
\rozwStop
\odpStart
$1$
\odpStop
\testStart
A.$1$ B.$\pi$ C.$0$ D.$\frac{167}{113}$ E.$\frac{113}{167}$
F.$-\frac{167}{113}$ G.$-1$
H.$(\frac{167}{113})^{10}$
I.$(\frac{113}{167})^{10}$
\testStop
\kluczStart
A
\kluczStop



\zadStart{Zadanie z Wikieł Z 1.1 d) moja wersja nr 869}

Obliczyć wartość wyrażenia $(\frac{167}{127})^{10} \cdot (\frac{127}{167})^{10} \cdot \pi^{0}$.
\zadStop
\rozwStart{Patryk Wirkus}{Martyna Czarnobaj}
$$(\frac{167}{127})^{10} \cdot (\frac{127}{167})^{10} \cdot \pi^{0} = (\frac{167}{127} \cdot \frac{127}{167})^{10} \cdot 1 = 1^{10} \cdot 1 = 1$$
\rozwStop
\odpStart
$1$
\odpStop
\testStart
A.$1$ B.$\pi$ C.$0$ D.$\frac{167}{127}$ E.$\frac{127}{167}$
F.$-\frac{167}{127}$ G.$-1$
H.$(\frac{167}{127})^{10}$
I.$(\frac{127}{167})^{10}$
\testStop
\kluczStart
A
\kluczStop



\zadStart{Zadanie z Wikieł Z 1.1 d) moja wersja nr 870}

Obliczyć wartość wyrażenia $(\frac{167}{131})^{10} \cdot (\frac{131}{167})^{10} \cdot \pi^{0}$.
\zadStop
\rozwStart{Patryk Wirkus}{Martyna Czarnobaj}
$$(\frac{167}{131})^{10} \cdot (\frac{131}{167})^{10} \cdot \pi^{0} = (\frac{167}{131} \cdot \frac{131}{167})^{10} \cdot 1 = 1^{10} \cdot 1 = 1$$
\rozwStop
\odpStart
$1$
\odpStop
\testStart
A.$1$ B.$\pi$ C.$0$ D.$\frac{167}{131}$ E.$\frac{131}{167}$
F.$-\frac{167}{131}$ G.$-1$
H.$(\frac{167}{131})^{10}$
I.$(\frac{131}{167})^{10}$
\testStop
\kluczStart
A
\kluczStop



\zadStart{Zadanie z Wikieł Z 1.1 d) moja wersja nr 871}

Obliczyć wartość wyrażenia $(\frac{167}{137})^{10} \cdot (\frac{137}{167})^{10} \cdot \pi^{0}$.
\zadStop
\rozwStart{Patryk Wirkus}{Martyna Czarnobaj}
$$(\frac{167}{137})^{10} \cdot (\frac{137}{167})^{10} \cdot \pi^{0} = (\frac{167}{137} \cdot \frac{137}{167})^{10} \cdot 1 = 1^{10} \cdot 1 = 1$$
\rozwStop
\odpStart
$1$
\odpStop
\testStart
A.$1$ B.$\pi$ C.$0$ D.$\frac{167}{137}$ E.$\frac{137}{167}$
F.$-\frac{167}{137}$ G.$-1$
H.$(\frac{167}{137})^{10}$
I.$(\frac{137}{167})^{10}$
\testStop
\kluczStart
A
\kluczStop



\zadStart{Zadanie z Wikieł Z 1.1 d) moja wersja nr 872}

Obliczyć wartość wyrażenia $(\frac{167}{139})^{10} \cdot (\frac{139}{167})^{10} \cdot \pi^{0}$.
\zadStop
\rozwStart{Patryk Wirkus}{Martyna Czarnobaj}
$$(\frac{167}{139})^{10} \cdot (\frac{139}{167})^{10} \cdot \pi^{0} = (\frac{167}{139} \cdot \frac{139}{167})^{10} \cdot 1 = 1^{10} \cdot 1 = 1$$
\rozwStop
\odpStart
$1$
\odpStop
\testStart
A.$1$ B.$\pi$ C.$0$ D.$\frac{167}{139}$ E.$\frac{139}{167}$
F.$-\frac{167}{139}$ G.$-1$
H.$(\frac{167}{139})^{10}$
I.$(\frac{139}{167})^{10}$
\testStop
\kluczStart
A
\kluczStop



\zadStart{Zadanie z Wikieł Z 1.1 d) moja wersja nr 873}

Obliczyć wartość wyrażenia $(\frac{173}{103})^{10} \cdot (\frac{103}{173})^{10} \cdot \pi^{0}$.
\zadStop
\rozwStart{Patryk Wirkus}{Martyna Czarnobaj}
$$(\frac{173}{103})^{10} \cdot (\frac{103}{173})^{10} \cdot \pi^{0} = (\frac{173}{103} \cdot \frac{103}{173})^{10} \cdot 1 = 1^{10} \cdot 1 = 1$$
\rozwStop
\odpStart
$1$
\odpStop
\testStart
A.$1$ B.$\pi$ C.$0$ D.$\frac{173}{103}$ E.$\frac{103}{173}$
F.$-\frac{173}{103}$ G.$-1$
H.$(\frac{173}{103})^{10}$
I.$(\frac{103}{173})^{10}$
\testStop
\kluczStart
A
\kluczStop



\zadStart{Zadanie z Wikieł Z 1.1 d) moja wersja nr 874}

Obliczyć wartość wyrażenia $(\frac{173}{107})^{10} \cdot (\frac{107}{173})^{10} \cdot \pi^{0}$.
\zadStop
\rozwStart{Patryk Wirkus}{Martyna Czarnobaj}
$$(\frac{173}{107})^{10} \cdot (\frac{107}{173})^{10} \cdot \pi^{0} = (\frac{173}{107} \cdot \frac{107}{173})^{10} \cdot 1 = 1^{10} \cdot 1 = 1$$
\rozwStop
\odpStart
$1$
\odpStop
\testStart
A.$1$ B.$\pi$ C.$0$ D.$\frac{173}{107}$ E.$\frac{107}{173}$
F.$-\frac{173}{107}$ G.$-1$
H.$(\frac{173}{107})^{10}$
I.$(\frac{107}{173})^{10}$
\testStop
\kluczStart
A
\kluczStop



\zadStart{Zadanie z Wikieł Z 1.1 d) moja wersja nr 875}

Obliczyć wartość wyrażenia $(\frac{173}{109})^{10} \cdot (\frac{109}{173})^{10} \cdot \pi^{0}$.
\zadStop
\rozwStart{Patryk Wirkus}{Martyna Czarnobaj}
$$(\frac{173}{109})^{10} \cdot (\frac{109}{173})^{10} \cdot \pi^{0} = (\frac{173}{109} \cdot \frac{109}{173})^{10} \cdot 1 = 1^{10} \cdot 1 = 1$$
\rozwStop
\odpStart
$1$
\odpStop
\testStart
A.$1$ B.$\pi$ C.$0$ D.$\frac{173}{109}$ E.$\frac{109}{173}$
F.$-\frac{173}{109}$ G.$-1$
H.$(\frac{173}{109})^{10}$
I.$(\frac{109}{173})^{10}$
\testStop
\kluczStart
A
\kluczStop



\zadStart{Zadanie z Wikieł Z 1.1 d) moja wersja nr 876}

Obliczyć wartość wyrażenia $(\frac{173}{113})^{10} \cdot (\frac{113}{173})^{10} \cdot \pi^{0}$.
\zadStop
\rozwStart{Patryk Wirkus}{Martyna Czarnobaj}
$$(\frac{173}{113})^{10} \cdot (\frac{113}{173})^{10} \cdot \pi^{0} = (\frac{173}{113} \cdot \frac{113}{173})^{10} \cdot 1 = 1^{10} \cdot 1 = 1$$
\rozwStop
\odpStart
$1$
\odpStop
\testStart
A.$1$ B.$\pi$ C.$0$ D.$\frac{173}{113}$ E.$\frac{113}{173}$
F.$-\frac{173}{113}$ G.$-1$
H.$(\frac{173}{113})^{10}$
I.$(\frac{113}{173})^{10}$
\testStop
\kluczStart
A
\kluczStop



\zadStart{Zadanie z Wikieł Z 1.1 d) moja wersja nr 877}

Obliczyć wartość wyrażenia $(\frac{173}{127})^{10} \cdot (\frac{127}{173})^{10} \cdot \pi^{0}$.
\zadStop
\rozwStart{Patryk Wirkus}{Martyna Czarnobaj}
$$(\frac{173}{127})^{10} \cdot (\frac{127}{173})^{10} \cdot \pi^{0} = (\frac{173}{127} \cdot \frac{127}{173})^{10} \cdot 1 = 1^{10} \cdot 1 = 1$$
\rozwStop
\odpStart
$1$
\odpStop
\testStart
A.$1$ B.$\pi$ C.$0$ D.$\frac{173}{127}$ E.$\frac{127}{173}$
F.$-\frac{173}{127}$ G.$-1$
H.$(\frac{173}{127})^{10}$
I.$(\frac{127}{173})^{10}$
\testStop
\kluczStart
A
\kluczStop



\zadStart{Zadanie z Wikieł Z 1.1 d) moja wersja nr 878}

Obliczyć wartość wyrażenia $(\frac{173}{131})^{10} \cdot (\frac{131}{173})^{10} \cdot \pi^{0}$.
\zadStop
\rozwStart{Patryk Wirkus}{Martyna Czarnobaj}
$$(\frac{173}{131})^{10} \cdot (\frac{131}{173})^{10} \cdot \pi^{0} = (\frac{173}{131} \cdot \frac{131}{173})^{10} \cdot 1 = 1^{10} \cdot 1 = 1$$
\rozwStop
\odpStart
$1$
\odpStop
\testStart
A.$1$ B.$\pi$ C.$0$ D.$\frac{173}{131}$ E.$\frac{131}{173}$
F.$-\frac{173}{131}$ G.$-1$
H.$(\frac{173}{131})^{10}$
I.$(\frac{131}{173})^{10}$
\testStop
\kluczStart
A
\kluczStop



\zadStart{Zadanie z Wikieł Z 1.1 d) moja wersja nr 879}

Obliczyć wartość wyrażenia $(\frac{173}{137})^{10} \cdot (\frac{137}{173})^{10} \cdot \pi^{0}$.
\zadStop
\rozwStart{Patryk Wirkus}{Martyna Czarnobaj}
$$(\frac{173}{137})^{10} \cdot (\frac{137}{173})^{10} \cdot \pi^{0} = (\frac{173}{137} \cdot \frac{137}{173})^{10} \cdot 1 = 1^{10} \cdot 1 = 1$$
\rozwStop
\odpStart
$1$
\odpStop
\testStart
A.$1$ B.$\pi$ C.$0$ D.$\frac{173}{137}$ E.$\frac{137}{173}$
F.$-\frac{173}{137}$ G.$-1$
H.$(\frac{173}{137})^{10}$
I.$(\frac{137}{173})^{10}$
\testStop
\kluczStart
A
\kluczStop



\zadStart{Zadanie z Wikieł Z 1.1 d) moja wersja nr 880}

Obliczyć wartość wyrażenia $(\frac{173}{139})^{10} \cdot (\frac{139}{173})^{10} \cdot \pi^{0}$.
\zadStop
\rozwStart{Patryk Wirkus}{Martyna Czarnobaj}
$$(\frac{173}{139})^{10} \cdot (\frac{139}{173})^{10} \cdot \pi^{0} = (\frac{173}{139} \cdot \frac{139}{173})^{10} \cdot 1 = 1^{10} \cdot 1 = 1$$
\rozwStop
\odpStart
$1$
\odpStop
\testStart
A.$1$ B.$\pi$ C.$0$ D.$\frac{173}{139}$ E.$\frac{139}{173}$
F.$-\frac{173}{139}$ G.$-1$
H.$(\frac{173}{139})^{10}$
I.$(\frac{139}{173})^{10}$
\testStop
\kluczStart
A
\kluczStop



\zadStart{Zadanie z Wikieł Z 1.1 d) moja wersja nr 881}

Obliczyć wartość wyrażenia $(\frac{179}{103})^{10} \cdot (\frac{103}{179})^{10} \cdot \pi^{0}$.
\zadStop
\rozwStart{Patryk Wirkus}{Martyna Czarnobaj}
$$(\frac{179}{103})^{10} \cdot (\frac{103}{179})^{10} \cdot \pi^{0} = (\frac{179}{103} \cdot \frac{103}{179})^{10} \cdot 1 = 1^{10} \cdot 1 = 1$$
\rozwStop
\odpStart
$1$
\odpStop
\testStart
A.$1$ B.$\pi$ C.$0$ D.$\frac{179}{103}$ E.$\frac{103}{179}$
F.$-\frac{179}{103}$ G.$-1$
H.$(\frac{179}{103})^{10}$
I.$(\frac{103}{179})^{10}$
\testStop
\kluczStart
A
\kluczStop



\zadStart{Zadanie z Wikieł Z 1.1 d) moja wersja nr 882}

Obliczyć wartość wyrażenia $(\frac{179}{107})^{10} \cdot (\frac{107}{179})^{10} \cdot \pi^{0}$.
\zadStop
\rozwStart{Patryk Wirkus}{Martyna Czarnobaj}
$$(\frac{179}{107})^{10} \cdot (\frac{107}{179})^{10} \cdot \pi^{0} = (\frac{179}{107} \cdot \frac{107}{179})^{10} \cdot 1 = 1^{10} \cdot 1 = 1$$
\rozwStop
\odpStart
$1$
\odpStop
\testStart
A.$1$ B.$\pi$ C.$0$ D.$\frac{179}{107}$ E.$\frac{107}{179}$
F.$-\frac{179}{107}$ G.$-1$
H.$(\frac{179}{107})^{10}$
I.$(\frac{107}{179})^{10}$
\testStop
\kluczStart
A
\kluczStop



\zadStart{Zadanie z Wikieł Z 1.1 d) moja wersja nr 883}

Obliczyć wartość wyrażenia $(\frac{179}{109})^{10} \cdot (\frac{109}{179})^{10} \cdot \pi^{0}$.
\zadStop
\rozwStart{Patryk Wirkus}{Martyna Czarnobaj}
$$(\frac{179}{109})^{10} \cdot (\frac{109}{179})^{10} \cdot \pi^{0} = (\frac{179}{109} \cdot \frac{109}{179})^{10} \cdot 1 = 1^{10} \cdot 1 = 1$$
\rozwStop
\odpStart
$1$
\odpStop
\testStart
A.$1$ B.$\pi$ C.$0$ D.$\frac{179}{109}$ E.$\frac{109}{179}$
F.$-\frac{179}{109}$ G.$-1$
H.$(\frac{179}{109})^{10}$
I.$(\frac{109}{179})^{10}$
\testStop
\kluczStart
A
\kluczStop



\zadStart{Zadanie z Wikieł Z 1.1 d) moja wersja nr 884}

Obliczyć wartość wyrażenia $(\frac{179}{113})^{10} \cdot (\frac{113}{179})^{10} \cdot \pi^{0}$.
\zadStop
\rozwStart{Patryk Wirkus}{Martyna Czarnobaj}
$$(\frac{179}{113})^{10} \cdot (\frac{113}{179})^{10} \cdot \pi^{0} = (\frac{179}{113} \cdot \frac{113}{179})^{10} \cdot 1 = 1^{10} \cdot 1 = 1$$
\rozwStop
\odpStart
$1$
\odpStop
\testStart
A.$1$ B.$\pi$ C.$0$ D.$\frac{179}{113}$ E.$\frac{113}{179}$
F.$-\frac{179}{113}$ G.$-1$
H.$(\frac{179}{113})^{10}$
I.$(\frac{113}{179})^{10}$
\testStop
\kluczStart
A
\kluczStop



\zadStart{Zadanie z Wikieł Z 1.1 d) moja wersja nr 885}

Obliczyć wartość wyrażenia $(\frac{179}{127})^{10} \cdot (\frac{127}{179})^{10} \cdot \pi^{0}$.
\zadStop
\rozwStart{Patryk Wirkus}{Martyna Czarnobaj}
$$(\frac{179}{127})^{10} \cdot (\frac{127}{179})^{10} \cdot \pi^{0} = (\frac{179}{127} \cdot \frac{127}{179})^{10} \cdot 1 = 1^{10} \cdot 1 = 1$$
\rozwStop
\odpStart
$1$
\odpStop
\testStart
A.$1$ B.$\pi$ C.$0$ D.$\frac{179}{127}$ E.$\frac{127}{179}$
F.$-\frac{179}{127}$ G.$-1$
H.$(\frac{179}{127})^{10}$
I.$(\frac{127}{179})^{10}$
\testStop
\kluczStart
A
\kluczStop



\zadStart{Zadanie z Wikieł Z 1.1 d) moja wersja nr 886}

Obliczyć wartość wyrażenia $(\frac{179}{131})^{10} \cdot (\frac{131}{179})^{10} \cdot \pi^{0}$.
\zadStop
\rozwStart{Patryk Wirkus}{Martyna Czarnobaj}
$$(\frac{179}{131})^{10} \cdot (\frac{131}{179})^{10} \cdot \pi^{0} = (\frac{179}{131} \cdot \frac{131}{179})^{10} \cdot 1 = 1^{10} \cdot 1 = 1$$
\rozwStop
\odpStart
$1$
\odpStop
\testStart
A.$1$ B.$\pi$ C.$0$ D.$\frac{179}{131}$ E.$\frac{131}{179}$
F.$-\frac{179}{131}$ G.$-1$
H.$(\frac{179}{131})^{10}$
I.$(\frac{131}{179})^{10}$
\testStop
\kluczStart
A
\kluczStop



\zadStart{Zadanie z Wikieł Z 1.1 d) moja wersja nr 887}

Obliczyć wartość wyrażenia $(\frac{179}{137})^{10} \cdot (\frac{137}{179})^{10} \cdot \pi^{0}$.
\zadStop
\rozwStart{Patryk Wirkus}{Martyna Czarnobaj}
$$(\frac{179}{137})^{10} \cdot (\frac{137}{179})^{10} \cdot \pi^{0} = (\frac{179}{137} \cdot \frac{137}{179})^{10} \cdot 1 = 1^{10} \cdot 1 = 1$$
\rozwStop
\odpStart
$1$
\odpStop
\testStart
A.$1$ B.$\pi$ C.$0$ D.$\frac{179}{137}$ E.$\frac{137}{179}$
F.$-\frac{179}{137}$ G.$-1$
H.$(\frac{179}{137})^{10}$
I.$(\frac{137}{179})^{10}$
\testStop
\kluczStart
A
\kluczStop



\zadStart{Zadanie z Wikieł Z 1.1 d) moja wersja nr 888}

Obliczyć wartość wyrażenia $(\frac{179}{139})^{10} \cdot (\frac{139}{179})^{10} \cdot \pi^{0}$.
\zadStop
\rozwStart{Patryk Wirkus}{Martyna Czarnobaj}
$$(\frac{179}{139})^{10} \cdot (\frac{139}{179})^{10} \cdot \pi^{0} = (\frac{179}{139} \cdot \frac{139}{179})^{10} \cdot 1 = 1^{10} \cdot 1 = 1$$
\rozwStop
\odpStart
$1$
\odpStop
\testStart
A.$1$ B.$\pi$ C.$0$ D.$\frac{179}{139}$ E.$\frac{139}{179}$
F.$-\frac{179}{139}$ G.$-1$
H.$(\frac{179}{139})^{10}$
I.$(\frac{139}{179})^{10}$
\testStop
\kluczStart
A
\kluczStop



\zadStart{Zadanie z Wikieł Z 1.1 d) moja wersja nr 889}

Obliczyć wartość wyrażenia $(\frac{251}{103})^{10} \cdot (\frac{103}{251})^{10} \cdot \pi^{0}$.
\zadStop
\rozwStart{Patryk Wirkus}{Martyna Czarnobaj}
$$(\frac{251}{103})^{10} \cdot (\frac{103}{251})^{10} \cdot \pi^{0} = (\frac{251}{103} \cdot \frac{103}{251})^{10} \cdot 1 = 1^{10} \cdot 1 = 1$$
\rozwStop
\odpStart
$1$
\odpStop
\testStart
A.$1$ B.$\pi$ C.$0$ D.$\frac{251}{103}$ E.$\frac{103}{251}$
F.$-\frac{251}{103}$ G.$-1$
H.$(\frac{251}{103})^{10}$
I.$(\frac{103}{251})^{10}$
\testStop
\kluczStart
A
\kluczStop



\zadStart{Zadanie z Wikieł Z 1.1 d) moja wersja nr 890}

Obliczyć wartość wyrażenia $(\frac{251}{107})^{10} \cdot (\frac{107}{251})^{10} \cdot \pi^{0}$.
\zadStop
\rozwStart{Patryk Wirkus}{Martyna Czarnobaj}
$$(\frac{251}{107})^{10} \cdot (\frac{107}{251})^{10} \cdot \pi^{0} = (\frac{251}{107} \cdot \frac{107}{251})^{10} \cdot 1 = 1^{10} \cdot 1 = 1$$
\rozwStop
\odpStart
$1$
\odpStop
\testStart
A.$1$ B.$\pi$ C.$0$ D.$\frac{251}{107}$ E.$\frac{107}{251}$
F.$-\frac{251}{107}$ G.$-1$
H.$(\frac{251}{107})^{10}$
I.$(\frac{107}{251})^{10}$
\testStop
\kluczStart
A
\kluczStop



\zadStart{Zadanie z Wikieł Z 1.1 d) moja wersja nr 891}

Obliczyć wartość wyrażenia $(\frac{251}{109})^{10} \cdot (\frac{109}{251})^{10} \cdot \pi^{0}$.
\zadStop
\rozwStart{Patryk Wirkus}{Martyna Czarnobaj}
$$(\frac{251}{109})^{10} \cdot (\frac{109}{251})^{10} \cdot \pi^{0} = (\frac{251}{109} \cdot \frac{109}{251})^{10} \cdot 1 = 1^{10} \cdot 1 = 1$$
\rozwStop
\odpStart
$1$
\odpStop
\testStart
A.$1$ B.$\pi$ C.$0$ D.$\frac{251}{109}$ E.$\frac{109}{251}$
F.$-\frac{251}{109}$ G.$-1$
H.$(\frac{251}{109})^{10}$
I.$(\frac{109}{251})^{10}$
\testStop
\kluczStart
A
\kluczStop



\zadStart{Zadanie z Wikieł Z 1.1 d) moja wersja nr 892}

Obliczyć wartość wyrażenia $(\frac{251}{113})^{10} \cdot (\frac{113}{251})^{10} \cdot \pi^{0}$.
\zadStop
\rozwStart{Patryk Wirkus}{Martyna Czarnobaj}
$$(\frac{251}{113})^{10} \cdot (\frac{113}{251})^{10} \cdot \pi^{0} = (\frac{251}{113} \cdot \frac{113}{251})^{10} \cdot 1 = 1^{10} \cdot 1 = 1$$
\rozwStop
\odpStart
$1$
\odpStop
\testStart
A.$1$ B.$\pi$ C.$0$ D.$\frac{251}{113}$ E.$\frac{113}{251}$
F.$-\frac{251}{113}$ G.$-1$
H.$(\frac{251}{113})^{10}$
I.$(\frac{113}{251})^{10}$
\testStop
\kluczStart
A
\kluczStop



\zadStart{Zadanie z Wikieł Z 1.1 d) moja wersja nr 893}

Obliczyć wartość wyrażenia $(\frac{251}{127})^{10} \cdot (\frac{127}{251})^{10} \cdot \pi^{0}$.
\zadStop
\rozwStart{Patryk Wirkus}{Martyna Czarnobaj}
$$(\frac{251}{127})^{10} \cdot (\frac{127}{251})^{10} \cdot \pi^{0} = (\frac{251}{127} \cdot \frac{127}{251})^{10} \cdot 1 = 1^{10} \cdot 1 = 1$$
\rozwStop
\odpStart
$1$
\odpStop
\testStart
A.$1$ B.$\pi$ C.$0$ D.$\frac{251}{127}$ E.$\frac{127}{251}$
F.$-\frac{251}{127}$ G.$-1$
H.$(\frac{251}{127})^{10}$
I.$(\frac{127}{251})^{10}$
\testStop
\kluczStart
A
\kluczStop



\zadStart{Zadanie z Wikieł Z 1.1 d) moja wersja nr 894}

Obliczyć wartość wyrażenia $(\frac{251}{131})^{10} \cdot (\frac{131}{251})^{10} \cdot \pi^{0}$.
\zadStop
\rozwStart{Patryk Wirkus}{Martyna Czarnobaj}
$$(\frac{251}{131})^{10} \cdot (\frac{131}{251})^{10} \cdot \pi^{0} = (\frac{251}{131} \cdot \frac{131}{251})^{10} \cdot 1 = 1^{10} \cdot 1 = 1$$
\rozwStop
\odpStart
$1$
\odpStop
\testStart
A.$1$ B.$\pi$ C.$0$ D.$\frac{251}{131}$ E.$\frac{131}{251}$
F.$-\frac{251}{131}$ G.$-1$
H.$(\frac{251}{131})^{10}$
I.$(\frac{131}{251})^{10}$
\testStop
\kluczStart
A
\kluczStop



\zadStart{Zadanie z Wikieł Z 1.1 d) moja wersja nr 895}

Obliczyć wartość wyrażenia $(\frac{251}{137})^{10} \cdot (\frac{137}{251})^{10} \cdot \pi^{0}$.
\zadStop
\rozwStart{Patryk Wirkus}{Martyna Czarnobaj}
$$(\frac{251}{137})^{10} \cdot (\frac{137}{251})^{10} \cdot \pi^{0} = (\frac{251}{137} \cdot \frac{137}{251})^{10} \cdot 1 = 1^{10} \cdot 1 = 1$$
\rozwStop
\odpStart
$1$
\odpStop
\testStart
A.$1$ B.$\pi$ C.$0$ D.$\frac{251}{137}$ E.$\frac{137}{251}$
F.$-\frac{251}{137}$ G.$-1$
H.$(\frac{251}{137})^{10}$
I.$(\frac{137}{251})^{10}$
\testStop
\kluczStart
A
\kluczStop



\zadStart{Zadanie z Wikieł Z 1.1 d) moja wersja nr 896}

Obliczyć wartość wyrażenia $(\frac{251}{139})^{10} \cdot (\frac{139}{251})^{10} \cdot \pi^{0}$.
\zadStop
\rozwStart{Patryk Wirkus}{Martyna Czarnobaj}
$$(\frac{251}{139})^{10} \cdot (\frac{139}{251})^{10} \cdot \pi^{0} = (\frac{251}{139} \cdot \frac{139}{251})^{10} \cdot 1 = 1^{10} \cdot 1 = 1$$
\rozwStop
\odpStart
$1$
\odpStop
\testStart
A.$1$ B.$\pi$ C.$0$ D.$\frac{251}{139}$ E.$\frac{139}{251}$
F.$-\frac{251}{139}$ G.$-1$
H.$(\frac{251}{139})^{10}$
I.$(\frac{139}{251})^{10}$
\testStop
\kluczStart
A
\kluczStop



\zadStart{Zadanie z Wikieł Z 1.1 d) moja wersja nr 897}

Obliczyć wartość wyrażenia $(\frac{257}{103})^{10} \cdot (\frac{103}{257})^{10} \cdot \pi^{0}$.
\zadStop
\rozwStart{Patryk Wirkus}{Martyna Czarnobaj}
$$(\frac{257}{103})^{10} \cdot (\frac{103}{257})^{10} \cdot \pi^{0} = (\frac{257}{103} \cdot \frac{103}{257})^{10} \cdot 1 = 1^{10} \cdot 1 = 1$$
\rozwStop
\odpStart
$1$
\odpStop
\testStart
A.$1$ B.$\pi$ C.$0$ D.$\frac{257}{103}$ E.$\frac{103}{257}$
F.$-\frac{257}{103}$ G.$-1$
H.$(\frac{257}{103})^{10}$
I.$(\frac{103}{257})^{10}$
\testStop
\kluczStart
A
\kluczStop



\zadStart{Zadanie z Wikieł Z 1.1 d) moja wersja nr 898}

Obliczyć wartość wyrażenia $(\frac{257}{107})^{10} \cdot (\frac{107}{257})^{10} \cdot \pi^{0}$.
\zadStop
\rozwStart{Patryk Wirkus}{Martyna Czarnobaj}
$$(\frac{257}{107})^{10} \cdot (\frac{107}{257})^{10} \cdot \pi^{0} = (\frac{257}{107} \cdot \frac{107}{257})^{10} \cdot 1 = 1^{10} \cdot 1 = 1$$
\rozwStop
\odpStart
$1$
\odpStop
\testStart
A.$1$ B.$\pi$ C.$0$ D.$\frac{257}{107}$ E.$\frac{107}{257}$
F.$-\frac{257}{107}$ G.$-1$
H.$(\frac{257}{107})^{10}$
I.$(\frac{107}{257})^{10}$
\testStop
\kluczStart
A
\kluczStop



\zadStart{Zadanie z Wikieł Z 1.1 d) moja wersja nr 899}

Obliczyć wartość wyrażenia $(\frac{257}{109})^{10} \cdot (\frac{109}{257})^{10} \cdot \pi^{0}$.
\zadStop
\rozwStart{Patryk Wirkus}{Martyna Czarnobaj}
$$(\frac{257}{109})^{10} \cdot (\frac{109}{257})^{10} \cdot \pi^{0} = (\frac{257}{109} \cdot \frac{109}{257})^{10} \cdot 1 = 1^{10} \cdot 1 = 1$$
\rozwStop
\odpStart
$1$
\odpStop
\testStart
A.$1$ B.$\pi$ C.$0$ D.$\frac{257}{109}$ E.$\frac{109}{257}$
F.$-\frac{257}{109}$ G.$-1$
H.$(\frac{257}{109})^{10}$
I.$(\frac{109}{257})^{10}$
\testStop
\kluczStart
A
\kluczStop



\zadStart{Zadanie z Wikieł Z 1.1 d) moja wersja nr 900}

Obliczyć wartość wyrażenia $(\frac{257}{113})^{10} \cdot (\frac{113}{257})^{10} \cdot \pi^{0}$.
\zadStop
\rozwStart{Patryk Wirkus}{Martyna Czarnobaj}
$$(\frac{257}{113})^{10} \cdot (\frac{113}{257})^{10} \cdot \pi^{0} = (\frac{257}{113} \cdot \frac{113}{257})^{10} \cdot 1 = 1^{10} \cdot 1 = 1$$
\rozwStop
\odpStart
$1$
\odpStop
\testStart
A.$1$ B.$\pi$ C.$0$ D.$\frac{257}{113}$ E.$\frac{113}{257}$
F.$-\frac{257}{113}$ G.$-1$
H.$(\frac{257}{113})^{10}$
I.$(\frac{113}{257})^{10}$
\testStop
\kluczStart
A
\kluczStop



\zadStart{Zadanie z Wikieł Z 1.1 d) moja wersja nr 901}

Obliczyć wartość wyrażenia $(\frac{257}{127})^{10} \cdot (\frac{127}{257})^{10} \cdot \pi^{0}$.
\zadStop
\rozwStart{Patryk Wirkus}{Martyna Czarnobaj}
$$(\frac{257}{127})^{10} \cdot (\frac{127}{257})^{10} \cdot \pi^{0} = (\frac{257}{127} \cdot \frac{127}{257})^{10} \cdot 1 = 1^{10} \cdot 1 = 1$$
\rozwStop
\odpStart
$1$
\odpStop
\testStart
A.$1$ B.$\pi$ C.$0$ D.$\frac{257}{127}$ E.$\frac{127}{257}$
F.$-\frac{257}{127}$ G.$-1$
H.$(\frac{257}{127})^{10}$
I.$(\frac{127}{257})^{10}$
\testStop
\kluczStart
A
\kluczStop



\zadStart{Zadanie z Wikieł Z 1.1 d) moja wersja nr 902}

Obliczyć wartość wyrażenia $(\frac{257}{131})^{10} \cdot (\frac{131}{257})^{10} \cdot \pi^{0}$.
\zadStop
\rozwStart{Patryk Wirkus}{Martyna Czarnobaj}
$$(\frac{257}{131})^{10} \cdot (\frac{131}{257})^{10} \cdot \pi^{0} = (\frac{257}{131} \cdot \frac{131}{257})^{10} \cdot 1 = 1^{10} \cdot 1 = 1$$
\rozwStop
\odpStart
$1$
\odpStop
\testStart
A.$1$ B.$\pi$ C.$0$ D.$\frac{257}{131}$ E.$\frac{131}{257}$
F.$-\frac{257}{131}$ G.$-1$
H.$(\frac{257}{131})^{10}$
I.$(\frac{131}{257})^{10}$
\testStop
\kluczStart
A
\kluczStop



\zadStart{Zadanie z Wikieł Z 1.1 d) moja wersja nr 903}

Obliczyć wartość wyrażenia $(\frac{257}{137})^{10} \cdot (\frac{137}{257})^{10} \cdot \pi^{0}$.
\zadStop
\rozwStart{Patryk Wirkus}{Martyna Czarnobaj}
$$(\frac{257}{137})^{10} \cdot (\frac{137}{257})^{10} \cdot \pi^{0} = (\frac{257}{137} \cdot \frac{137}{257})^{10} \cdot 1 = 1^{10} \cdot 1 = 1$$
\rozwStop
\odpStart
$1$
\odpStop
\testStart
A.$1$ B.$\pi$ C.$0$ D.$\frac{257}{137}$ E.$\frac{137}{257}$
F.$-\frac{257}{137}$ G.$-1$
H.$(\frac{257}{137})^{10}$
I.$(\frac{137}{257})^{10}$
\testStop
\kluczStart
A
\kluczStop



\zadStart{Zadanie z Wikieł Z 1.1 d) moja wersja nr 904}

Obliczyć wartość wyrażenia $(\frac{257}{139})^{10} \cdot (\frac{139}{257})^{10} \cdot \pi^{0}$.
\zadStop
\rozwStart{Patryk Wirkus}{Martyna Czarnobaj}
$$(\frac{257}{139})^{10} \cdot (\frac{139}{257})^{10} \cdot \pi^{0} = (\frac{257}{139} \cdot \frac{139}{257})^{10} \cdot 1 = 1^{10} \cdot 1 = 1$$
\rozwStop
\odpStart
$1$
\odpStop
\testStart
A.$1$ B.$\pi$ C.$0$ D.$\frac{257}{139}$ E.$\frac{139}{257}$
F.$-\frac{257}{139}$ G.$-1$
H.$(\frac{257}{139})^{10}$
I.$(\frac{139}{257})^{10}$
\testStop
\kluczStart
A
\kluczStop



\zadStart{Zadanie z Wikieł Z 1.1 d) moja wersja nr 905}

Obliczyć wartość wyrażenia $(\frac{263}{103})^{10} \cdot (\frac{103}{263})^{10} \cdot \pi^{0}$.
\zadStop
\rozwStart{Patryk Wirkus}{Martyna Czarnobaj}
$$(\frac{263}{103})^{10} \cdot (\frac{103}{263})^{10} \cdot \pi^{0} = (\frac{263}{103} \cdot \frac{103}{263})^{10} \cdot 1 = 1^{10} \cdot 1 = 1$$
\rozwStop
\odpStart
$1$
\odpStop
\testStart
A.$1$ B.$\pi$ C.$0$ D.$\frac{263}{103}$ E.$\frac{103}{263}$
F.$-\frac{263}{103}$ G.$-1$
H.$(\frac{263}{103})^{10}$
I.$(\frac{103}{263})^{10}$
\testStop
\kluczStart
A
\kluczStop



\zadStart{Zadanie z Wikieł Z 1.1 d) moja wersja nr 906}

Obliczyć wartość wyrażenia $(\frac{263}{107})^{10} \cdot (\frac{107}{263})^{10} \cdot \pi^{0}$.
\zadStop
\rozwStart{Patryk Wirkus}{Martyna Czarnobaj}
$$(\frac{263}{107})^{10} \cdot (\frac{107}{263})^{10} \cdot \pi^{0} = (\frac{263}{107} \cdot \frac{107}{263})^{10} \cdot 1 = 1^{10} \cdot 1 = 1$$
\rozwStop
\odpStart
$1$
\odpStop
\testStart
A.$1$ B.$\pi$ C.$0$ D.$\frac{263}{107}$ E.$\frac{107}{263}$
F.$-\frac{263}{107}$ G.$-1$
H.$(\frac{263}{107})^{10}$
I.$(\frac{107}{263})^{10}$
\testStop
\kluczStart
A
\kluczStop



\zadStart{Zadanie z Wikieł Z 1.1 d) moja wersja nr 907}

Obliczyć wartość wyrażenia $(\frac{263}{109})^{10} \cdot (\frac{109}{263})^{10} \cdot \pi^{0}$.
\zadStop
\rozwStart{Patryk Wirkus}{Martyna Czarnobaj}
$$(\frac{263}{109})^{10} \cdot (\frac{109}{263})^{10} \cdot \pi^{0} = (\frac{263}{109} \cdot \frac{109}{263})^{10} \cdot 1 = 1^{10} \cdot 1 = 1$$
\rozwStop
\odpStart
$1$
\odpStop
\testStart
A.$1$ B.$\pi$ C.$0$ D.$\frac{263}{109}$ E.$\frac{109}{263}$
F.$-\frac{263}{109}$ G.$-1$
H.$(\frac{263}{109})^{10}$
I.$(\frac{109}{263})^{10}$
\testStop
\kluczStart
A
\kluczStop



\zadStart{Zadanie z Wikieł Z 1.1 d) moja wersja nr 908}

Obliczyć wartość wyrażenia $(\frac{263}{113})^{10} \cdot (\frac{113}{263})^{10} \cdot \pi^{0}$.
\zadStop
\rozwStart{Patryk Wirkus}{Martyna Czarnobaj}
$$(\frac{263}{113})^{10} \cdot (\frac{113}{263})^{10} \cdot \pi^{0} = (\frac{263}{113} \cdot \frac{113}{263})^{10} \cdot 1 = 1^{10} \cdot 1 = 1$$
\rozwStop
\odpStart
$1$
\odpStop
\testStart
A.$1$ B.$\pi$ C.$0$ D.$\frac{263}{113}$ E.$\frac{113}{263}$
F.$-\frac{263}{113}$ G.$-1$
H.$(\frac{263}{113})^{10}$
I.$(\frac{113}{263})^{10}$
\testStop
\kluczStart
A
\kluczStop



\zadStart{Zadanie z Wikieł Z 1.1 d) moja wersja nr 909}

Obliczyć wartość wyrażenia $(\frac{263}{127})^{10} \cdot (\frac{127}{263})^{10} \cdot \pi^{0}$.
\zadStop
\rozwStart{Patryk Wirkus}{Martyna Czarnobaj}
$$(\frac{263}{127})^{10} \cdot (\frac{127}{263})^{10} \cdot \pi^{0} = (\frac{263}{127} \cdot \frac{127}{263})^{10} \cdot 1 = 1^{10} \cdot 1 = 1$$
\rozwStop
\odpStart
$1$
\odpStop
\testStart
A.$1$ B.$\pi$ C.$0$ D.$\frac{263}{127}$ E.$\frac{127}{263}$
F.$-\frac{263}{127}$ G.$-1$
H.$(\frac{263}{127})^{10}$
I.$(\frac{127}{263})^{10}$
\testStop
\kluczStart
A
\kluczStop



\zadStart{Zadanie z Wikieł Z 1.1 d) moja wersja nr 910}

Obliczyć wartość wyrażenia $(\frac{263}{131})^{10} \cdot (\frac{131}{263})^{10} \cdot \pi^{0}$.
\zadStop
\rozwStart{Patryk Wirkus}{Martyna Czarnobaj}
$$(\frac{263}{131})^{10} \cdot (\frac{131}{263})^{10} \cdot \pi^{0} = (\frac{263}{131} \cdot \frac{131}{263})^{10} \cdot 1 = 1^{10} \cdot 1 = 1$$
\rozwStop
\odpStart
$1$
\odpStop
\testStart
A.$1$ B.$\pi$ C.$0$ D.$\frac{263}{131}$ E.$\frac{131}{263}$
F.$-\frac{263}{131}$ G.$-1$
H.$(\frac{263}{131})^{10}$
I.$(\frac{131}{263})^{10}$
\testStop
\kluczStart
A
\kluczStop



\zadStart{Zadanie z Wikieł Z 1.1 d) moja wersja nr 911}

Obliczyć wartość wyrażenia $(\frac{263}{137})^{10} \cdot (\frac{137}{263})^{10} \cdot \pi^{0}$.
\zadStop
\rozwStart{Patryk Wirkus}{Martyna Czarnobaj}
$$(\frac{263}{137})^{10} \cdot (\frac{137}{263})^{10} \cdot \pi^{0} = (\frac{263}{137} \cdot \frac{137}{263})^{10} \cdot 1 = 1^{10} \cdot 1 = 1$$
\rozwStop
\odpStart
$1$
\odpStop
\testStart
A.$1$ B.$\pi$ C.$0$ D.$\frac{263}{137}$ E.$\frac{137}{263}$
F.$-\frac{263}{137}$ G.$-1$
H.$(\frac{263}{137})^{10}$
I.$(\frac{137}{263})^{10}$
\testStop
\kluczStart
A
\kluczStop



\zadStart{Zadanie z Wikieł Z 1.1 d) moja wersja nr 912}

Obliczyć wartość wyrażenia $(\frac{263}{139})^{10} \cdot (\frac{139}{263})^{10} \cdot \pi^{0}$.
\zadStop
\rozwStart{Patryk Wirkus}{Martyna Czarnobaj}
$$(\frac{263}{139})^{10} \cdot (\frac{139}{263})^{10} \cdot \pi^{0} = (\frac{263}{139} \cdot \frac{139}{263})^{10} \cdot 1 = 1^{10} \cdot 1 = 1$$
\rozwStop
\odpStart
$1$
\odpStop
\testStart
A.$1$ B.$\pi$ C.$0$ D.$\frac{263}{139}$ E.$\frac{139}{263}$
F.$-\frac{263}{139}$ G.$-1$
H.$(\frac{263}{139})^{10}$
I.$(\frac{139}{263})^{10}$
\testStop
\kluczStart
A
\kluczStop



\zadStart{Zadanie z Wikieł Z 1.1 d) moja wersja nr 913}

Obliczyć wartość wyrażenia $(\frac{269}{103})^{10} \cdot (\frac{103}{269})^{10} \cdot \pi^{0}$.
\zadStop
\rozwStart{Patryk Wirkus}{Martyna Czarnobaj}
$$(\frac{269}{103})^{10} \cdot (\frac{103}{269})^{10} \cdot \pi^{0} = (\frac{269}{103} \cdot \frac{103}{269})^{10} \cdot 1 = 1^{10} \cdot 1 = 1$$
\rozwStop
\odpStart
$1$
\odpStop
\testStart
A.$1$ B.$\pi$ C.$0$ D.$\frac{269}{103}$ E.$\frac{103}{269}$
F.$-\frac{269}{103}$ G.$-1$
H.$(\frac{269}{103})^{10}$
I.$(\frac{103}{269})^{10}$
\testStop
\kluczStart
A
\kluczStop



\zadStart{Zadanie z Wikieł Z 1.1 d) moja wersja nr 914}

Obliczyć wartość wyrażenia $(\frac{269}{107})^{10} \cdot (\frac{107}{269})^{10} \cdot \pi^{0}$.
\zadStop
\rozwStart{Patryk Wirkus}{Martyna Czarnobaj}
$$(\frac{269}{107})^{10} \cdot (\frac{107}{269})^{10} \cdot \pi^{0} = (\frac{269}{107} \cdot \frac{107}{269})^{10} \cdot 1 = 1^{10} \cdot 1 = 1$$
\rozwStop
\odpStart
$1$
\odpStop
\testStart
A.$1$ B.$\pi$ C.$0$ D.$\frac{269}{107}$ E.$\frac{107}{269}$
F.$-\frac{269}{107}$ G.$-1$
H.$(\frac{269}{107})^{10}$
I.$(\frac{107}{269})^{10}$
\testStop
\kluczStart
A
\kluczStop



\zadStart{Zadanie z Wikieł Z 1.1 d) moja wersja nr 915}

Obliczyć wartość wyrażenia $(\frac{269}{109})^{10} \cdot (\frac{109}{269})^{10} \cdot \pi^{0}$.
\zadStop
\rozwStart{Patryk Wirkus}{Martyna Czarnobaj}
$$(\frac{269}{109})^{10} \cdot (\frac{109}{269})^{10} \cdot \pi^{0} = (\frac{269}{109} \cdot \frac{109}{269})^{10} \cdot 1 = 1^{10} \cdot 1 = 1$$
\rozwStop
\odpStart
$1$
\odpStop
\testStart
A.$1$ B.$\pi$ C.$0$ D.$\frac{269}{109}$ E.$\frac{109}{269}$
F.$-\frac{269}{109}$ G.$-1$
H.$(\frac{269}{109})^{10}$
I.$(\frac{109}{269})^{10}$
\testStop
\kluczStart
A
\kluczStop



\zadStart{Zadanie z Wikieł Z 1.1 d) moja wersja nr 916}

Obliczyć wartość wyrażenia $(\frac{269}{113})^{10} \cdot (\frac{113}{269})^{10} \cdot \pi^{0}$.
\zadStop
\rozwStart{Patryk Wirkus}{Martyna Czarnobaj}
$$(\frac{269}{113})^{10} \cdot (\frac{113}{269})^{10} \cdot \pi^{0} = (\frac{269}{113} \cdot \frac{113}{269})^{10} \cdot 1 = 1^{10} \cdot 1 = 1$$
\rozwStop
\odpStart
$1$
\odpStop
\testStart
A.$1$ B.$\pi$ C.$0$ D.$\frac{269}{113}$ E.$\frac{113}{269}$
F.$-\frac{269}{113}$ G.$-1$
H.$(\frac{269}{113})^{10}$
I.$(\frac{113}{269})^{10}$
\testStop
\kluczStart
A
\kluczStop



\zadStart{Zadanie z Wikieł Z 1.1 d) moja wersja nr 917}

Obliczyć wartość wyrażenia $(\frac{269}{127})^{10} \cdot (\frac{127}{269})^{10} \cdot \pi^{0}$.
\zadStop
\rozwStart{Patryk Wirkus}{Martyna Czarnobaj}
$$(\frac{269}{127})^{10} \cdot (\frac{127}{269})^{10} \cdot \pi^{0} = (\frac{269}{127} \cdot \frac{127}{269})^{10} \cdot 1 = 1^{10} \cdot 1 = 1$$
\rozwStop
\odpStart
$1$
\odpStop
\testStart
A.$1$ B.$\pi$ C.$0$ D.$\frac{269}{127}$ E.$\frac{127}{269}$
F.$-\frac{269}{127}$ G.$-1$
H.$(\frac{269}{127})^{10}$
I.$(\frac{127}{269})^{10}$
\testStop
\kluczStart
A
\kluczStop



\zadStart{Zadanie z Wikieł Z 1.1 d) moja wersja nr 918}

Obliczyć wartość wyrażenia $(\frac{269}{131})^{10} \cdot (\frac{131}{269})^{10} \cdot \pi^{0}$.
\zadStop
\rozwStart{Patryk Wirkus}{Martyna Czarnobaj}
$$(\frac{269}{131})^{10} \cdot (\frac{131}{269})^{10} \cdot \pi^{0} = (\frac{269}{131} \cdot \frac{131}{269})^{10} \cdot 1 = 1^{10} \cdot 1 = 1$$
\rozwStop
\odpStart
$1$
\odpStop
\testStart
A.$1$ B.$\pi$ C.$0$ D.$\frac{269}{131}$ E.$\frac{131}{269}$
F.$-\frac{269}{131}$ G.$-1$
H.$(\frac{269}{131})^{10}$
I.$(\frac{131}{269})^{10}$
\testStop
\kluczStart
A
\kluczStop



\zadStart{Zadanie z Wikieł Z 1.1 d) moja wersja nr 919}

Obliczyć wartość wyrażenia $(\frac{269}{137})^{10} \cdot (\frac{137}{269})^{10} \cdot \pi^{0}$.
\zadStop
\rozwStart{Patryk Wirkus}{Martyna Czarnobaj}
$$(\frac{269}{137})^{10} \cdot (\frac{137}{269})^{10} \cdot \pi^{0} = (\frac{269}{137} \cdot \frac{137}{269})^{10} \cdot 1 = 1^{10} \cdot 1 = 1$$
\rozwStop
\odpStart
$1$
\odpStop
\testStart
A.$1$ B.$\pi$ C.$0$ D.$\frac{269}{137}$ E.$\frac{137}{269}$
F.$-\frac{269}{137}$ G.$-1$
H.$(\frac{269}{137})^{10}$
I.$(\frac{137}{269})^{10}$
\testStop
\kluczStart
A
\kluczStop



\zadStart{Zadanie z Wikieł Z 1.1 d) moja wersja nr 920}

Obliczyć wartość wyrażenia $(\frac{269}{139})^{10} \cdot (\frac{139}{269})^{10} \cdot \pi^{0}$.
\zadStop
\rozwStart{Patryk Wirkus}{Martyna Czarnobaj}
$$(\frac{269}{139})^{10} \cdot (\frac{139}{269})^{10} \cdot \pi^{0} = (\frac{269}{139} \cdot \frac{139}{269})^{10} \cdot 1 = 1^{10} \cdot 1 = 1$$
\rozwStop
\odpStart
$1$
\odpStop
\testStart
A.$1$ B.$\pi$ C.$0$ D.$\frac{269}{139}$ E.$\frac{139}{269}$
F.$-\frac{269}{139}$ G.$-1$
H.$(\frac{269}{139})^{10}$
I.$(\frac{139}{269})^{10}$
\testStop
\kluczStart
A
\kluczStop



\zadStart{Zadanie z Wikieł Z 1.1 d) moja wersja nr 921}

Obliczyć wartość wyrażenia $(\frac{271}{103})^{10} \cdot (\frac{103}{271})^{10} \cdot \pi^{0}$.
\zadStop
\rozwStart{Patryk Wirkus}{Martyna Czarnobaj}
$$(\frac{271}{103})^{10} \cdot (\frac{103}{271})^{10} \cdot \pi^{0} = (\frac{271}{103} \cdot \frac{103}{271})^{10} \cdot 1 = 1^{10} \cdot 1 = 1$$
\rozwStop
\odpStart
$1$
\odpStop
\testStart
A.$1$ B.$\pi$ C.$0$ D.$\frac{271}{103}$ E.$\frac{103}{271}$
F.$-\frac{271}{103}$ G.$-1$
H.$(\frac{271}{103})^{10}$
I.$(\frac{103}{271})^{10}$
\testStop
\kluczStart
A
\kluczStop



\zadStart{Zadanie z Wikieł Z 1.1 d) moja wersja nr 922}

Obliczyć wartość wyrażenia $(\frac{271}{107})^{10} \cdot (\frac{107}{271})^{10} \cdot \pi^{0}$.
\zadStop
\rozwStart{Patryk Wirkus}{Martyna Czarnobaj}
$$(\frac{271}{107})^{10} \cdot (\frac{107}{271})^{10} \cdot \pi^{0} = (\frac{271}{107} \cdot \frac{107}{271})^{10} \cdot 1 = 1^{10} \cdot 1 = 1$$
\rozwStop
\odpStart
$1$
\odpStop
\testStart
A.$1$ B.$\pi$ C.$0$ D.$\frac{271}{107}$ E.$\frac{107}{271}$
F.$-\frac{271}{107}$ G.$-1$
H.$(\frac{271}{107})^{10}$
I.$(\frac{107}{271})^{10}$
\testStop
\kluczStart
A
\kluczStop



\zadStart{Zadanie z Wikieł Z 1.1 d) moja wersja nr 923}

Obliczyć wartość wyrażenia $(\frac{271}{109})^{10} \cdot (\frac{109}{271})^{10} \cdot \pi^{0}$.
\zadStop
\rozwStart{Patryk Wirkus}{Martyna Czarnobaj}
$$(\frac{271}{109})^{10} \cdot (\frac{109}{271})^{10} \cdot \pi^{0} = (\frac{271}{109} \cdot \frac{109}{271})^{10} \cdot 1 = 1^{10} \cdot 1 = 1$$
\rozwStop
\odpStart
$1$
\odpStop
\testStart
A.$1$ B.$\pi$ C.$0$ D.$\frac{271}{109}$ E.$\frac{109}{271}$
F.$-\frac{271}{109}$ G.$-1$
H.$(\frac{271}{109})^{10}$
I.$(\frac{109}{271})^{10}$
\testStop
\kluczStart
A
\kluczStop



\zadStart{Zadanie z Wikieł Z 1.1 d) moja wersja nr 924}

Obliczyć wartość wyrażenia $(\frac{271}{113})^{10} \cdot (\frac{113}{271})^{10} \cdot \pi^{0}$.
\zadStop
\rozwStart{Patryk Wirkus}{Martyna Czarnobaj}
$$(\frac{271}{113})^{10} \cdot (\frac{113}{271})^{10} \cdot \pi^{0} = (\frac{271}{113} \cdot \frac{113}{271})^{10} \cdot 1 = 1^{10} \cdot 1 = 1$$
\rozwStop
\odpStart
$1$
\odpStop
\testStart
A.$1$ B.$\pi$ C.$0$ D.$\frac{271}{113}$ E.$\frac{113}{271}$
F.$-\frac{271}{113}$ G.$-1$
H.$(\frac{271}{113})^{10}$
I.$(\frac{113}{271})^{10}$
\testStop
\kluczStart
A
\kluczStop



\zadStart{Zadanie z Wikieł Z 1.1 d) moja wersja nr 925}

Obliczyć wartość wyrażenia $(\frac{271}{127})^{10} \cdot (\frac{127}{271})^{10} \cdot \pi^{0}$.
\zadStop
\rozwStart{Patryk Wirkus}{Martyna Czarnobaj}
$$(\frac{271}{127})^{10} \cdot (\frac{127}{271})^{10} \cdot \pi^{0} = (\frac{271}{127} \cdot \frac{127}{271})^{10} \cdot 1 = 1^{10} \cdot 1 = 1$$
\rozwStop
\odpStart
$1$
\odpStop
\testStart
A.$1$ B.$\pi$ C.$0$ D.$\frac{271}{127}$ E.$\frac{127}{271}$
F.$-\frac{271}{127}$ G.$-1$
H.$(\frac{271}{127})^{10}$
I.$(\frac{127}{271})^{10}$
\testStop
\kluczStart
A
\kluczStop



\zadStart{Zadanie z Wikieł Z 1.1 d) moja wersja nr 926}

Obliczyć wartość wyrażenia $(\frac{271}{131})^{10} \cdot (\frac{131}{271})^{10} \cdot \pi^{0}$.
\zadStop
\rozwStart{Patryk Wirkus}{Martyna Czarnobaj}
$$(\frac{271}{131})^{10} \cdot (\frac{131}{271})^{10} \cdot \pi^{0} = (\frac{271}{131} \cdot \frac{131}{271})^{10} \cdot 1 = 1^{10} \cdot 1 = 1$$
\rozwStop
\odpStart
$1$
\odpStop
\testStart
A.$1$ B.$\pi$ C.$0$ D.$\frac{271}{131}$ E.$\frac{131}{271}$
F.$-\frac{271}{131}$ G.$-1$
H.$(\frac{271}{131})^{10}$
I.$(\frac{131}{271})^{10}$
\testStop
\kluczStart
A
\kluczStop



\zadStart{Zadanie z Wikieł Z 1.1 d) moja wersja nr 927}

Obliczyć wartość wyrażenia $(\frac{271}{137})^{10} \cdot (\frac{137}{271})^{10} \cdot \pi^{0}$.
\zadStop
\rozwStart{Patryk Wirkus}{Martyna Czarnobaj}
$$(\frac{271}{137})^{10} \cdot (\frac{137}{271})^{10} \cdot \pi^{0} = (\frac{271}{137} \cdot \frac{137}{271})^{10} \cdot 1 = 1^{10} \cdot 1 = 1$$
\rozwStop
\odpStart
$1$
\odpStop
\testStart
A.$1$ B.$\pi$ C.$0$ D.$\frac{271}{137}$ E.$\frac{137}{271}$
F.$-\frac{271}{137}$ G.$-1$
H.$(\frac{271}{137})^{10}$
I.$(\frac{137}{271})^{10}$
\testStop
\kluczStart
A
\kluczStop



\zadStart{Zadanie z Wikieł Z 1.1 d) moja wersja nr 928}

Obliczyć wartość wyrażenia $(\frac{271}{139})^{10} \cdot (\frac{139}{271})^{10} \cdot \pi^{0}$.
\zadStop
\rozwStart{Patryk Wirkus}{Martyna Czarnobaj}
$$(\frac{271}{139})^{10} \cdot (\frac{139}{271})^{10} \cdot \pi^{0} = (\frac{271}{139} \cdot \frac{139}{271})^{10} \cdot 1 = 1^{10} \cdot 1 = 1$$
\rozwStop
\odpStart
$1$
\odpStop
\testStart
A.$1$ B.$\pi$ C.$0$ D.$\frac{271}{139}$ E.$\frac{139}{271}$
F.$-\frac{271}{139}$ G.$-1$
H.$(\frac{271}{139})^{10}$
I.$(\frac{139}{271})^{10}$
\testStop
\kluczStart
A
\kluczStop



\zadStart{Zadanie z Wikieł Z 1.1 d) moja wersja nr 929}

Obliczyć wartość wyrażenia $(\frac{277}{103})^{10} \cdot (\frac{103}{277})^{10} \cdot \pi^{0}$.
\zadStop
\rozwStart{Patryk Wirkus}{Martyna Czarnobaj}
$$(\frac{277}{103})^{10} \cdot (\frac{103}{277})^{10} \cdot \pi^{0} = (\frac{277}{103} \cdot \frac{103}{277})^{10} \cdot 1 = 1^{10} \cdot 1 = 1$$
\rozwStop
\odpStart
$1$
\odpStop
\testStart
A.$1$ B.$\pi$ C.$0$ D.$\frac{277}{103}$ E.$\frac{103}{277}$
F.$-\frac{277}{103}$ G.$-1$
H.$(\frac{277}{103})^{10}$
I.$(\frac{103}{277})^{10}$
\testStop
\kluczStart
A
\kluczStop



\zadStart{Zadanie z Wikieł Z 1.1 d) moja wersja nr 930}

Obliczyć wartość wyrażenia $(\frac{277}{107})^{10} \cdot (\frac{107}{277})^{10} \cdot \pi^{0}$.
\zadStop
\rozwStart{Patryk Wirkus}{Martyna Czarnobaj}
$$(\frac{277}{107})^{10} \cdot (\frac{107}{277})^{10} \cdot \pi^{0} = (\frac{277}{107} \cdot \frac{107}{277})^{10} \cdot 1 = 1^{10} \cdot 1 = 1$$
\rozwStop
\odpStart
$1$
\odpStop
\testStart
A.$1$ B.$\pi$ C.$0$ D.$\frac{277}{107}$ E.$\frac{107}{277}$
F.$-\frac{277}{107}$ G.$-1$
H.$(\frac{277}{107})^{10}$
I.$(\frac{107}{277})^{10}$
\testStop
\kluczStart
A
\kluczStop



\zadStart{Zadanie z Wikieł Z 1.1 d) moja wersja nr 931}

Obliczyć wartość wyrażenia $(\frac{277}{109})^{10} \cdot (\frac{109}{277})^{10} \cdot \pi^{0}$.
\zadStop
\rozwStart{Patryk Wirkus}{Martyna Czarnobaj}
$$(\frac{277}{109})^{10} \cdot (\frac{109}{277})^{10} \cdot \pi^{0} = (\frac{277}{109} \cdot \frac{109}{277})^{10} \cdot 1 = 1^{10} \cdot 1 = 1$$
\rozwStop
\odpStart
$1$
\odpStop
\testStart
A.$1$ B.$\pi$ C.$0$ D.$\frac{277}{109}$ E.$\frac{109}{277}$
F.$-\frac{277}{109}$ G.$-1$
H.$(\frac{277}{109})^{10}$
I.$(\frac{109}{277})^{10}$
\testStop
\kluczStart
A
\kluczStop



\zadStart{Zadanie z Wikieł Z 1.1 d) moja wersja nr 932}

Obliczyć wartość wyrażenia $(\frac{277}{113})^{10} \cdot (\frac{113}{277})^{10} \cdot \pi^{0}$.
\zadStop
\rozwStart{Patryk Wirkus}{Martyna Czarnobaj}
$$(\frac{277}{113})^{10} \cdot (\frac{113}{277})^{10} \cdot \pi^{0} = (\frac{277}{113} \cdot \frac{113}{277})^{10} \cdot 1 = 1^{10} \cdot 1 = 1$$
\rozwStop
\odpStart
$1$
\odpStop
\testStart
A.$1$ B.$\pi$ C.$0$ D.$\frac{277}{113}$ E.$\frac{113}{277}$
F.$-\frac{277}{113}$ G.$-1$
H.$(\frac{277}{113})^{10}$
I.$(\frac{113}{277})^{10}$
\testStop
\kluczStart
A
\kluczStop



\zadStart{Zadanie z Wikieł Z 1.1 d) moja wersja nr 933}

Obliczyć wartość wyrażenia $(\frac{277}{127})^{10} \cdot (\frac{127}{277})^{10} \cdot \pi^{0}$.
\zadStop
\rozwStart{Patryk Wirkus}{Martyna Czarnobaj}
$$(\frac{277}{127})^{10} \cdot (\frac{127}{277})^{10} \cdot \pi^{0} = (\frac{277}{127} \cdot \frac{127}{277})^{10} \cdot 1 = 1^{10} \cdot 1 = 1$$
\rozwStop
\odpStart
$1$
\odpStop
\testStart
A.$1$ B.$\pi$ C.$0$ D.$\frac{277}{127}$ E.$\frac{127}{277}$
F.$-\frac{277}{127}$ G.$-1$
H.$(\frac{277}{127})^{10}$
I.$(\frac{127}{277})^{10}$
\testStop
\kluczStart
A
\kluczStop



\zadStart{Zadanie z Wikieł Z 1.1 d) moja wersja nr 934}

Obliczyć wartość wyrażenia $(\frac{277}{131})^{10} \cdot (\frac{131}{277})^{10} \cdot \pi^{0}$.
\zadStop
\rozwStart{Patryk Wirkus}{Martyna Czarnobaj}
$$(\frac{277}{131})^{10} \cdot (\frac{131}{277})^{10} \cdot \pi^{0} = (\frac{277}{131} \cdot \frac{131}{277})^{10} \cdot 1 = 1^{10} \cdot 1 = 1$$
\rozwStop
\odpStart
$1$
\odpStop
\testStart
A.$1$ B.$\pi$ C.$0$ D.$\frac{277}{131}$ E.$\frac{131}{277}$
F.$-\frac{277}{131}$ G.$-1$
H.$(\frac{277}{131})^{10}$
I.$(\frac{131}{277})^{10}$
\testStop
\kluczStart
A
\kluczStop



\zadStart{Zadanie z Wikieł Z 1.1 d) moja wersja nr 935}

Obliczyć wartość wyrażenia $(\frac{277}{137})^{10} \cdot (\frac{137}{277})^{10} \cdot \pi^{0}$.
\zadStop
\rozwStart{Patryk Wirkus}{Martyna Czarnobaj}
$$(\frac{277}{137})^{10} \cdot (\frac{137}{277})^{10} \cdot \pi^{0} = (\frac{277}{137} \cdot \frac{137}{277})^{10} \cdot 1 = 1^{10} \cdot 1 = 1$$
\rozwStop
\odpStart
$1$
\odpStop
\testStart
A.$1$ B.$\pi$ C.$0$ D.$\frac{277}{137}$ E.$\frac{137}{277}$
F.$-\frac{277}{137}$ G.$-1$
H.$(\frac{277}{137})^{10}$
I.$(\frac{137}{277})^{10}$
\testStop
\kluczStart
A
\kluczStop



\zadStart{Zadanie z Wikieł Z 1.1 d) moja wersja nr 936}

Obliczyć wartość wyrażenia $(\frac{277}{139})^{10} \cdot (\frac{139}{277})^{10} \cdot \pi^{0}$.
\zadStop
\rozwStart{Patryk Wirkus}{Martyna Czarnobaj}
$$(\frac{277}{139})^{10} \cdot (\frac{139}{277})^{10} \cdot \pi^{0} = (\frac{277}{139} \cdot \frac{139}{277})^{10} \cdot 1 = 1^{10} \cdot 1 = 1$$
\rozwStop
\odpStart
$1$
\odpStop
\testStart
A.$1$ B.$\pi$ C.$0$ D.$\frac{277}{139}$ E.$\frac{139}{277}$
F.$-\frac{277}{139}$ G.$-1$
H.$(\frac{277}{139})^{10}$
I.$(\frac{139}{277})^{10}$
\testStop
\kluczStart
A
\kluczStop





\end{document}
