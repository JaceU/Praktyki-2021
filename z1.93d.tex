\documentclass[12pt, a4paper]{article}
\usepackage[utf8]{inputenc}
\usepackage{polski}

\usepackage{amsthm}  %pakiet do tworzenia twierdzeń itp.
\usepackage{amsmath} %pakiet do niektórych symboli matematycznych
\usepackage{amssymb} %pakiet do symboli mat., np. \nsubseteq
\usepackage{amsfonts}
\usepackage{graphicx} %obsługa plików graficznych z rozszerzeniem png, jpg
\theoremstyle{definition} %styl dla definicji
\newtheorem{zad}{} 
\title{Multizestaw zadań}
\author{Robert Fidytek}
%\date{\today}
\date{}
\newcounter{liczniksekcji}
\newcommand{\kategoria}[1]{\section{#1}} %olreślamy nazwę kateforii zadań
\newcommand{\zadStart}[1]{\begin{zad}#1\newline} %oznaczenie początku zadania
\newcommand{\zadStop}{\end{zad}}   %oznaczenie końca zadania
%Makra opcjonarne (nie muszą występować):
\newcommand{\rozwStart}[2]{\noindent \textbf{Rozwiązanie (autor #1 , recenzent #2): }\newline} %oznaczenie początku rozwiązania, opcjonarnie można wprowadzić informację o autorze rozwiązania zadania i recenzencie poprawności wykonania rozwiązania zadania
\newcommand{\rozwStop}{\newline}                                            %oznaczenie końca rozwiązania
\newcommand{\odpStart}{\noindent \textbf{Odpowiedź:}\newline}    %oznaczenie początku odpowiedzi końcowej (wypisanie wyniku)
\newcommand{\odpStop}{\newline}                                             %oznaczenie końca odpowiedzi końcowej (wypisanie wyniku)
\newcommand{\testStart}{\noindent \textbf{Test:}\newline} %ewentualne możliwe opcje odpowiedzi testowej: A. ? B. ? C. ? D. ? itd.
\newcommand{\testStop}{\newline} %koniec wprowadzania odpowiedzi testowych
\newcommand{\kluczStart}{\noindent \textbf{Test poprawna odpowiedź:}\newline} %klucz, poprawna odpowiedź pytania testowego (jedna literka): A lub B lub C lub D itd.
\newcommand{\kluczStop}{\newline} %koniec poprawnej odpowiedzi pytania testowego 
\newcommand{\wstawGrafike}[2]{\begin{figure}[h] \includegraphics[scale=#2] {#1} \end{figure}} %gdyby była potrzeba wstawienia obrazka, parametry: nazwa pliku, skala (jak nie wiesz co wpisać, to wpisz 1)

\begin{document}
\maketitle


\kategoria{Wikieł/Z1.93d}
\zadStart{Zadanie z Wikieł Z 1.93 d) moja wersja nr [nrWersji]}
%[p]:[2,3,4,5,6,7,8]
%[r]:[2,3,4,5,6,7,8]
%[a]:[2,3,4,5,6,7,8]
%[b]:[1,2,3,4,5,6,7,8,9,10,11,12,13,14,15,16,17,18,19,20]
%[ba]=[b]/[a]
%[b2]=int([ba])
%[pr]=(pow([p],[r]))
%[apr]=(([pr])/([a]))
%[pra]=int([apr])
%[d]=(([b2]**2)-(4*[pra]))
%[pr2]=(pow([d],(1/2)))
%[pr1]=[pr2].real
%[ppp]=int([pr1])
%[zz1]=([b2]-[ppp])/2
%[z1]=int([zz1])
%[zz2]=([b2]+[ppp])/2
%[z2]=int([zz2])
%[pr]<40 and [apr].is_integer()==True and [ba].is_integer()==True and [d]>0 and [pr2].is_integer()==True and [zz1].is_integer()==True and [zz2].is_integer()==True and [z2]>0 and [z1]>0 and [z2]<[b2] and [z1]<[b2]
Rozwiązać równanie $\log_{[p]}{x} + \log_{[p]}{([b]-[a]x)} = [r]$
\zadStop
\rozwStart{Małgorzata Ugowska}{}
Szukamy dziedziny:
$$x>0 \quad \land \quad [b]-[a]x >0 \quad \Longrightarrow \quad D = (0, [b2])$$
Następnie szukamy rozwiązania równania.
$$\log_{[p]}{x} + \log_{[p]}{([b]-[a]x)} = [r] \quad \Longleftrightarrow \quad \log_{[p]}{([b]x-[a]x^2)} = [r]$$
$$ \Longleftrightarrow \quad [p]^{[r]} = [b]x-[a]x^2 \quad \Longleftrightarrow \quad [b]x-[a]x^2 = [pr]$$
$$ \Longleftrightarrow \quad [a]x^2 - [b]x +[pr]= 0 \quad \Longleftrightarrow \quad x^2 - [b2]x +[pra]= 0 $$
Szukamy miejsc zerowych funkcji $x^2 - [b2]x +[pra]= 0 $:
$$ \bigtriangleup = [b2]^2-4 \cdot 1 \cdot [pra] = [d] \quad  \Longrightarrow \quad \sqrt{\bigtriangleup}=[ppp]$$
$$ x_1=\frac{[b2]-\sqrt{\bigtriangleup}}{2} = [z1] \in D, \qquad x_2=\frac{[b2]+\sqrt{\bigtriangleup}}{2} = [z2] \in D$$
\rozwStop
\odpStart
$x \in \{[z1], [z2]\}$
\odpStop
\testStart
A. $x \in \{[z1], [z2]\}$
B. $x \in \{-\frac{1}{2}, 0\}$
C. $x \in \{0, \frac{1}{2}\}$
D. $x \in \{\frac{4}{3}, \frac{-1}{3}\}$
E. $[pra]$
\testStop
\kluczStart
A
\kluczStop



\end{document}