\documentclass[12pt, a4paper]{article}
\usepackage[utf8]{inputenc}
\usepackage{polski}
\usepackage{amsthm}  %pakiet do tworzenia twierdzeń itp.
\usepackage{amsmath} %pakiet do niektórych symboli matematycznych
\usepackage{amssymb} %pakiet do symboli mat., np. \nsubseteq
\usepackage{amsfonts}
\usepackage{graphicx} %obsługa plików graficznych z rozszerzeniem png, jpg
\theoremstyle{definition} %styl dla definicji
\newtheorem{zad}{} 
\title{Multizestaw zadań}
\author{Radosław Grzyb}
%\date{\today}
\date{}
\newcounter{liczniksekcji}
\newcommand{\kategoria}[1]{\section{#1}} %olreślamy nazwę kateforii zadań
\newcommand{\zadStart}[1]{\begin{zad}#1\newline} %oznaczenie początku zadania
\newcommand{\zadStop}{\end{zad}}   %oznaczenie końca zadania
%Makra opcjonarne (nie muszą występować):
\newcommand{\rozwStart}[2]{\noindent \textbf{Rozwiązanie (autor #1 , recenzent #2): }\newline} %oznaczenie początku rozwiązania, opcjonarnie można wprowadzić informację o autorze rozwiązania zadania i recenzencie poprawności wykonania rozwiązania zadania
\newcommand{\rozwStop}{\newline}                                            %oznaczenie końca rozwiązania
\newcommand{\odpStart}{\noindent \textbf{Odpowiedź:}\newline}    %oznaczenie początku odpowiedzi końcowej (wypisanie wyniku)
\newcommand{\odpStop}{\newline}                                             %oznaczenie końca odpowiedzi końcowej (wypisanie wyniku)
\newcommand{\testStart}{\noindent \textbf{Test:}\newline} %ewentualne możliwe opcje odpowiedzi testowej: A. ? B. ? C. ? D. ? itd.
\newcommand{\testStop}{\newline} %koniec wprowadzania odpowiedzi testowych
\newcommand{\kluczStart}{\noindent \textbf{Test poprawna odpowiedź:}\newline} %klucz, poprawna odpowiedź pytania testowego (jedna literka): A lub B lub C lub D itd.
\newcommand{\kluczStop}{\newline} %koniec poprawnej odpowiedzi pytania testowego 
\newcommand{\wstawGrafike}[2]{\begin{figure}[h] \includegraphics[scale=#2] {#1} \end{figure}} %gdyby była potrzeba wstawienia obrazka, parametry: nazwa pliku, skala (jak nie wiesz co wpisać, to wpisz 1)
\begin{document}
\maketitle
\kategoria{Wikieł/Z1.84f}
\zadStart{Zadanie z Wikieł Z 1.84f moja wersja nr [nrWersji]}
%[p1]:[1,2,4,5,8,10,16,20]
%[p2]:[1,2,4,5,7,8,10,11]
%[c1]=3-[p2]
%[c2]=-4+1/[p1]
%[c3]=-[c2]
%[wynik]=[c3]/[c1]
%[zlywynik1]=1-[wynik]
%[zlywynik2]=-[wynik]
%[zlywynik3]=2*[wynik]-5
%([wynik]).is_integer() is False
Rozwiązać równanie:
$$2\cdot4^{\sqrt{[p2]x}}=\sqrt[[p1]]{2}\cdot8^{x-1}$$.
\zadStop
\rozwStart{Radosław Grzyb}{}
$$2\cdot2^{[p2]x}=2^{\frac{1}{[p1]}}\cdot2^{3x-3}$$
Dzielimy obie strony równania przez $2\cdot2^{[p2]x}$ otrzymując:
$$1=2^{\frac{1}{[p1]}}\cdot2^{[c1]x-4}$$
$$1=2^{[c1]x[c2]}$$
Logarytmując obie strony równania otrzymujemy:
$$[c1]x[c2]=0$$
$$[c1]x=[c3]$$
$$x=[wynik]$$
\rozwStop
\odpStart
$[wynik]$
\odpStop
\testStart
A.$[zlywynik1]$
B.$[wynik]$
C.$[zlywynik2]$
D.$10\cdot3$
E.$[zlywynik3]$
\testStop
\kluczStart
B
\kluczStop
\end{document}