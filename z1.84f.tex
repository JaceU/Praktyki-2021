\documentclass[12pt, a4paper]{article}
\usepackage[utf8]{inputenc}
\usepackage{polski}
\usepackage{amsthm}  %pakiet do tworzenia twierdzeń itp.
\usepackage{amsmath} %pakiet do niektórych symboli matematycznych
\usepackage{amssymb} %pakiet do symboli mat., np. \nsubseteq
\usepackage{amsfonts}
\usepackage{graphicx} %obsługa plików graficznych z rozszerzeniem png, jpg
\theoremstyle{definition} %styl dla definicji
\newtheorem{zad}{} 
\title{Multizestaw zadań}
\author{Radosław Grzyb}
%\date{\today}
\date{}
\newcounter{liczniksekcji}
\newcommand{\kategoria}[1]{\section{#1}} %olreślamy nazwę kateforii zadań
\newcommand{\zadStart}[1]{\begin{zad}#1\newline} %oznaczenie początku zadania
\newcommand{\zadStop}{\end{zad}}   %oznaczenie końca zadania
%Makra opcjonarne (nie muszą występować):
\newcommand{\rozwStart}[2]{\noindent \textbf{Rozwiązanie (autor #1 , recenzent #2): }\newline} %oznaczenie początku rozwiązania, opcjonarnie można wprowadzić informację o autorze rozwiązania zadania i recenzencie poprawności wykonania rozwiązania zadania
\newcommand{\rozwStop}{\newline}                                            %oznaczenie końca rozwiązania
\newcommand{\odpStart}{\noindent \textbf{Odpowiedź:}\newline}    %oznaczenie początku odpowiedzi końcowej (wypisanie wyniku)
\newcommand{\odpStop}{\newline}                                             %oznaczenie końca odpowiedzi końcowej (wypisanie wyniku)
\newcommand{\testStart}{\noindent \textbf{Test:}\newline} %ewentualne możliwe opcje odpowiedzi testowej: A. ? B. ? C. ? D. ? itd.
\newcommand{\testStop}{\newline} %koniec wprowadzania odpowiedzi testowych
\newcommand{\kluczStart}{\noindent \textbf{Test poprawna odpowiedź:}\newline} %klucz, poprawna odpowiedź pytania testowego (jedna literka): A lub B lub C lub D itd.
\newcommand{\kluczStop}{\newline} %koniec poprawnej odpowiedzi pytania testowego 
\newcommand{\wstawGrafike}[2]{\begin{figure}[h] \includegraphics[scale=#2] {#1} \end{figure}} %gdyby była potrzeba wstawienia obrazka, parametry: nazwa pliku, skala (jak nie wiesz co wpisać, to wpisz 1)
\begin{document}
\maketitle
\kategoria{Wikieł/Z1.84f}
\zadStart{Zadanie z Wikieł Z 1.84f moja wersja nr [nrWersji]}
%[p1]:[2,3,4,5,6,7,8,9,10]
%[p2]:[2,3,4,5,6,7,8,9,10]
%[s1]=2**[p1]
%[s2]=2**[p2]
%[g1]=4*[p1]
%[g2]=4*[p2]
%[g3]=3+[g2]
%[Delta]=[g1]**2+4*[g2]*[g3]
%[tDelta]=math.sqrt([Delta])
%[sDelta]=int([tDelta])
%[h1]=[g1]+[sDelta]
%[h2]=2*[g2]
%[gcd]=math.gcd([h1],[h2])
%[h11]=int([h1]/[gcd])
%[h22]=int([h2]/[gcd])
%[a1]=[g1]-[sDelta]
%[gcd2]=math.gcd([a1],[h2])
%[a11]=-int([a1]/[gcd2])
%[ak11]=int([h2]/[gcd2])
%([tDelta]).is_integer() is True
Rozwiązać równanie:
$$2\cdot[s1]^{\sqrt{x}}=\sqrt[4]{2}\cdot[s2]^{x-1}$$.
\zadStop
\rozwStart{Radosław Grzyb}{}
$$2\cdot2^{[p1]\sqrt{x}}=2^{\frac{1}{4}}\cdot2^{[p2]x-[p2]}$$
$$2^{[p1]\sqrt{x}+1}=2^{[p2]x-[p2]+\frac{1}{4}}$$
Logarytmując obie strony równania otrzymujemy::
$$[p1]\sqrt{x}+1=[p2]x-[p2]+\frac{1}{4}/\cdot4$$
$$[g1]\sqrt{x}+4=[g2]x-[g2]+1$$
Przenosimy wszystko na lewą stronę równania i upraszczamy:
$$-[g2]x+[g1]\sqrt{x}+4-1+[g2]=0$$
$$-[g2]x+[g1]\sqrt{x}+[g3]=0$$
Podstawiając $t=\sqrt{x}$ otrzymamy równanie kwadratowe. Zakładamy także, że $t\geq0$.
$$-[g2]t^{2}+[g1]t+[g3]=0$$
Policzmy więc deltę:
$$\Delta_{t}=[g1]^2-4(-[g2])[g3]=[Delta]\implies\sqrt{[Delta]}=[sDelta]$$
Znajdźmy miejsca zerowe:
$$t_{1}=\frac{-[g1]-[sDelta]}{2\cdot(-[g2])}=\frac{[g1]+[sDelta]}{2\cdot[g2]}=\frac{[h1]}{[h2]}=\frac{[h11]}{[h22]}$$
$$t_{2}=\frac{-[g1]+[sDelta]}{2\cdot(-[g2])}=\frac{[g1]-[sDelta]}{2\cdot[g2]}=\frac{[a1]}{[h2]}=-\frac{[a11]}{[ak11]}$$
Z racji naszego założenia, odrzucami wynik $t_{2}<0$ i rozpatrujemy jedynie $t_{1}$:
$$t_{1}=\sqrt{x}=\frac{[h11]}{[h22]}$$
Podnosimy obie strony do kwadratu otrzymując wynik:
$$x=(\frac{[h11]}{[h22]})^{2}$$
\rozwStop
\odpStart
$$(\frac{[h11]}{[h22]})^{2}$$
\odpStop
\testStart
A.$$(\frac{[h22]}{[h11]})^{2}$$
B.$$(\frac{[h11]}{[h22]})^{2}$$
C.$$(\frac{[h11]}{[h22]})^{3}$$
D.$$\frac{[h11]}{[h22]}$$
\testStop
\kluczStart
B
\kluczStop
\end{document}