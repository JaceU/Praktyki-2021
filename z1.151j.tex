\documentclass[12pt, a4paper]{article}
\usepackage[utf8]{inputenc}
\usepackage{polski}

\usepackage{amsthm}  %pakiet do tworzenia twierdzeń itp.
\usepackage{amsmath} %pakiet do niektórych symboli matematycznych
\usepackage{amssymb} %pakiet do symboli mat., np. \nsubseteq
\usepackage{amsfonts}
\usepackage{graphicx} %obsługa plików graficznych z rozszerzeniem png, jpg
\theoremstyle{definition} %styl dla definicji
\newtheorem{zad}{} 
\title{Multizestaw zadań}
\author{Laura Mieczkowska}
%\date{\today}
\date{}
\newcounter{liczniksekcji}
\newcommand{\kategoria}[1]{\section{#1}} %olreślamy nazwę kateforii zadań
\newcommand{\zadStart}[1]{\begin{zad}#1\newline} %oznaczenie początku zadania
\newcommand{\zadStop}{\end{zad}}   %oznaczenie końca zadania
%Makra opcjonarne (nie muszą występować):
\newcommand{\rozwStart}[2]{\noindent \textbf{Rozwiązanie (autor #1 , recenzent #2): }\newline} %oznaczenie początku rozwiązania, opcjonarnie można wprowadzić informację o autorze rozwiązania zadania i recenzencie poprawności wykonania rozwiązania zadania
\newcommand{\rozwStop}{\newline}                                            %oznaczenie końca rozwiązania
\newcommand{\odpStart}{\noindent \textbf{Odpowiedź:}\newline}    %oznaczenie początku odpowiedzi końcowej (wypisanie wyniku)
\newcommand{\odpStop}{\newline}                                             %oznaczenie końca odpowiedzi końcowej (wypisanie wyniku)
\newcommand{\testStart}{\noindent \textbf{Test:}\newline} %ewentualne możliwe opcje odpowiedzi testowej: A. ? B. ? C. ? D. ? itd.
\newcommand{\testStop}{\newline} %koniec wprowadzania odpowiedzi testowych
\newcommand{\kluczStart}{\noindent \textbf{Test poprawna odpowiedź:}\newline} %klucz, poprawna odpowiedź pytania testowego (jedna literka): A lub B lub C lub D itd.
\newcommand{\kluczStop}{\newline} %koniec poprawnej odpowiedzi pytania testowego 
\newcommand{\wstawGrafike}[2]{\begin{figure}[h] \includegraphics[scale=#2] {#1} \end{figure}} %gdyby była potrzeba wstawienia obrazka, parametry: nazwa pliku, skala (jak nie wiesz co wpisać, to wpisz 1)

\begin{document}
\maketitle


\kategoria{Wikieł/Z1.151j}
\zadStart{Zadanie z Wikieł Z 1.151 j) moja wersja nr [nrWersji]}
%[a]:[1,2,3,4,5,6,7,8,9,10]
%[b]:[2,3,4,5,6,7,8,9,10]
%[c]=([a]**2)*3
%[d]=3*[a]
%[e]=[a]**3
%[f]=abs([e]-[b])
%[ckw]=[c]**2
%[4df]=4*[d]*[f]
%[delta]=[ckw]+[4df]
%[pierw1]=math.sqrt([delta])
%[pierw]=int([pierw1])
%[m]=2*[d]
%[l1]=[c]+[pierw]
%[l2]=-[c]+[pierw]
%[ulamek1]=[l1]/[m]
%[ulamek]=int([ulamek1])
%[ulamek3]=[l2]/[m]
%[ulamek2]=int([ulamek3])
%[w1]=[a]-[ulamek]
%[w2]=[a]+[ulamek2]
%[pierw1].is_integer()==True and [ulamek1].is_integer()==True and [ulamek3].is_integer()==True
Rozwiązać układ równań $\left\{ \begin{array}{l}
x-y=[a]\\
x^3-y^3=[b]
\end{array} \right.$
\zadStop
\rozwStart{Laura Mieczkowska}{}
$$\left\{ \begin{array}{l}
x-y=[a]\\
x^3-y^3=[b] 
\end{array} \right.
\Rightarrow
\left\{ \begin{array}{l}
x=[a]+y\\
([a]+y)^3-y^3=[b]
\end{array} \right.$$
$$([a]+y)^3-y^3=[b]$$
$$[a]^3+3\cdot[a]^2\cdot y+3\cdot[a]\cdot y^2+y^3-y^3=[b]$$
$$[e]+[c]y+[d]y^2=[b]$$
$$[d]y^2+[c]y-[f]=0$$
$$\Delta=[c]^2-4\cdot[d]\cdot(-[f])=[delta]\Rightarrow \sqrt{\Delta}=[pierw]$$
$$y_1=\frac{-[c]-[pierw]}{[m]} \vee y_2=\frac{-[c]+[pierw]}{[m]}$$
$$y_1=-\frac{[l1]}{[m]} \vee y_2=\frac{[l2]}{[m]}$$
$$y_1=-[ulamek] \vee y_2=[ulamek2]$$
$$x_1=[a]-[ulamek] \vee x_2=[a]+[ulamek2]$$
$$x_1=[w1] \vee x_2=[w2]$$
\odpStart
$x_1=[w1], y_1=-[ulamek] \vee x_2=[w2], y_2=[ulamek2]$
\odpStop
\testStart
A. $x_1=0, y_1=-[ulamek] \vee x_2=1, y_2=[ulamek2]$ \\
B. $x_1=[w1], y_1=[ulamek] \vee x_2=[w2], y_2=-[ulamek2]$ \\
C. $x_1=[w2], y_1=-[ulamek] \vee x_2=[w1], y_2=[ulamek2]$ \\
D. $x_1=[w1], y_1=-[ulamek] \vee x_2=[w2], y_2=[ulamek2]$ 
\testStop
\kluczStart
D
\kluczStop



\end{document}