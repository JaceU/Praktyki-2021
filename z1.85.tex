\documentclass[12pt, a4paper]{article}
\usepackage[utf8]{inputenc}
\usepackage{polski}
\usepackage{amsthm}  %pakiet do tworzenia twierdzeń itp.
\usepackage{amsmath} %pakiet do niektórych symboli matematycznych
\usepackage{amssymb} %pakiet do symboli mat., np. \nsubseteq
\usepackage{amsfonts}
\usepackage{graphicx} %obsługa plików graficznych z rozszerzeniem png, jpg
\theoremstyle{definition} %styl dla definicji
\newtheorem{zad}{} 
\title{Multizestaw zadań}
\author{Radosław Grzyb}
%\date{\today}
\date{}
\newcounter{liczniksekcji}
\newcommand{\kategoria}[1]{\section{#1}} %olreślamy nazwę kateforii zadań
\newcommand{\zadStart}[1]{\begin{zad}#1\newline} %oznaczenie początku zadania
\newcommand{\zadStop}{\end{zad}}   %oznaczenie końca zadania
%Makra opcjonarne (nie muszą występować):
\newcommand{\rozwStart}[2]{\noindent \textbf{Rozwiązanie (autor #1 , recenzent #2): }\newline} %oznaczenie początku rozwiązania, opcjonarnie można wprowadzić informację o autorze rozwiązania zadania i recenzencie poprawności wykonania rozwiązania zadania
\newcommand{\rozwStop}{\newline}                                            %oznaczenie końca rozwiązania
\newcommand{\odpStart}{\noindent \textbf{Odpowiedź:}\newline}    %oznaczenie początku odpowiedzi końcowej (wypisanie wyniku)
\newcommand{\odpStop}{\newline}                                             %oznaczenie końca odpowiedzi końcowej (wypisanie wyniku)
\newcommand{\testStart}{\noindent \textbf{Test:}\newline} %ewentualne możliwe opcje odpowiedzi testowej: A. ? B. ? C. ? D. ? itd.
\newcommand{\testStop}{\newline} %koniec wprowadzania odpowiedzi testowych
\newcommand{\kluczStart}{\noindent \textbf{Test poprawna odpowiedź:}\newline} %klucz, poprawna odpowiedź pytania testowego (jedna literka): A lub B lub C lub D itd.
\newcommand{\kluczStop}{\newline} %koniec poprawnej odpowiedzi pytania testowego 
\newcommand{\wstawGrafike}[2]{\begin{figure}[h] \includegraphics[scale=#2] {#1} \end{figure}} %gdyby była potrzeba wstawienia obrazka, parametry: nazwa pliku, skala (jak nie wiesz co wpisać, to wpisz 1)
\begin{document}
\maketitle
\kategoria{Wikieł/Z1.85}
\zadStart{Zadanie z Wikieł Z 1.85 moja wersja nr [nrWersji]}
%[p2]:[3,5,7,9,11]
%[o2]=2**[p2]
%[c1]=int((-3-[p2])/2)
%[c2]=int((1+[p2])/2)
%[b1]=-4*[c2]
%[b2]=-4*[c1]
%[b11]=-[b1]
%[Deltam]=int(math.pow([b1],2)-4*1*[b2])
%[licz]=int([Deltam]/16)
%[b12]=int([b11]/2)
%[Deltam]>0
Wyznaczyć wartości parametru m, dla których równanie ma rowiązanie 
$$(0.5)^{x^{2}-mx+0.5m-1.5}=(\sqrt{[o2]})^{m-1}$$:
\zadStop
\rozwStart{Radosław Grzyb}{}
$$(\frac{1}{2})^{x^{2}-mx+\frac{1}{2}m-\frac{3}{2}}=(\sqrt{2^{[p2]}})^{m-1}$$
$$(\frac{1}{2})^{x^{2}-mx+\frac{1}{2}m-\frac{3}{2}}=2^{\frac{[p2]}{2}(m-1)}$$
$$(\frac{1}{2})^{x^{2}-mx+\frac{1}{2}m-\frac{3}{2}}=2^{\frac{[p2]}{2}m-\frac{[p2]}{2}}$$
$$(\frac{1}{2})^{x^{2}-mx+\frac{1}{2}m-\frac{3}{2}}=(\frac{1}{2})^{\frac{[p2]}{2}-\frac{[p2]}{2}m}$$
Logarytmujemy obie strony równania:
$$x^{2}-mx+\frac{1}{2}m-\frac{3}{2}=\frac{[p2]}{2}-\frac{[p2]}{2}m$$
$$x^{2}-mx+\frac{1}{2}m-\frac{3}{2}-\frac{[p2]}{2}+\frac{[p2]}{2}m=0$$
$$x^{2}-mx+[c2]m[c1]=0$$
Czas policzyć deltę:
$$\Delta_{x}=(-m)^{2}-4\cdot1\cdot([c2]m[c1])=m^{2}[b1]m+[b2]$$
Aby rozwiązać zadanie, musimy znaleźć wartości m, dla których powyższy dwumian jest $\geq0$. W tym celu ponownie policzymy sobie deltę:
$$\Delta_{m}=([b1])^{2}-4\cdot1\cdot[b2]=[Deltam]\geq0\implies\sqrt{\Delta_{m}}=\sqrt{[Deltam]}$$
Czas znaleźć miejsca zerowe naszej delty:
$$m_{1}=\frac{[b11]-\sqrt{[Deltam]}}{2}=\frac{[b11]-\sqrt{2^4}\sqrt{[licz]}}{2}=[b12]-2\sqrt{[licz]}$$
$$m_{2}=\frac{[b11]+\sqrt{[Deltam]}}{2}=\frac{[b11]+\sqrt{2^4}\sqrt{[licz]}}{2}=[b12]+2\sqrt{[licz]}$$
Nasz współczynnik $a=$ jest dodatni, zatem równanie ma rozwiązanie dla\\ $m\in(-\infty;[b12]-2\sqrt{[licz]}]\cup[[b12]+2\sqrt{[licz]};\infty)$
\rozwStop
\odpStart
$$m\in(-\infty;[b12]-2\sqrt{[licz]}]\cup[[b12]+2\sqrt{[licz]};\infty)$$
\odpStop
\testStart
A.$$m\in(-\infty;[b2]]\cup[[b1];\infty)$$
B.$$m\in(-\infty;[b12]-16\sqrt{[licz]}]\cup[[b12]+16\sqrt{[licz]};\infty)$$
C.$$m\in(-\infty;[b12]-2\sqrt{[licz]}]\cup[[b12]+2\sqrt{[licz]};\infty)$$
D.$$m\in(-\infty;[p2])$$
\testStop
\kluczStart
C
\kluczStop
\end{document}