\documentclass[12pt, a4paper]{article}
\usepackage[utf8]{inputenc}
\usepackage{polski}

\usepackage{amsthm}  %pakiet do tworzenia twierdzeń itp.
\usepackage{amsmath} %pakiet do niektórych symboli matematycznych
\usepackage{amssymb} %pakiet do symboli mat., np. \nsubseteq
\usepackage{amsfonts}
\usepackage{graphicx} %obsługa plików graficznych z rozszerzeniem png, jpg
\theoremstyle{definition} %styl dla definicji
\newtheorem{zad}{} 
\title{Multizestaw zadań}
\author{Robert Fidytek}
%\date{\today}
\date{}
\newcounter{liczniksekcji}
\newcommand{\kategoria}[1]{\section{#1}} %olreślamy nazwę kateforii zadań
\newcommand{\zadStart}[1]{\begin{zad}#1\newline} %oznaczenie początku zadania
\newcommand{\zadStop}{\end{zad}}   %oznaczenie końca zadania
%Makra opcjonarne (nie muszą występować):
\newcommand{\rozwStart}[2]{\noindent \textbf{Rozwiązanie (autor #1 , recenzent #2): }\newline} %oznaczenie początku rozwiązania, opcjonarnie można wprowadzić informację o autorze rozwiązania zadania i recenzencie poprawności wykonania rozwiązania zadania
\newcommand{\rozwStop}{\newline}                                            %oznaczenie końca rozwiązania
\newcommand{\odpStart}{\noindent \textbf{Odpowiedź:}\newline}    %oznaczenie początku odpowiedzi końcowej (wypisanie wyniku)
\newcommand{\odpStop}{\newline}                                             %oznaczenie końca odpowiedzi końcowej (wypisanie wyniku)
\newcommand{\testStart}{\noindent \textbf{Test:}\newline} %ewentualne możliwe opcje odpowiedzi testowej: A. ? B. ? C. ? D. ? itd.
\newcommand{\testStop}{\newline} %koniec wprowadzania odpowiedzi testowych
\newcommand{\kluczStart}{\noindent \textbf{Test poprawna odpowiedź:}\newline} %klucz, poprawna odpowiedź pytania testowego (jedna literka): A lub B lub C lub D itd.
\newcommand{\kluczStop}{\newline} %koniec poprawnej odpowiedzi pytania testowego 
\newcommand{\wstawGrafike}[2]{\begin{figure}[h] \includegraphics[scale=#2] {#1} \end{figure}} %gdyby była potrzeba wstawienia obrazka, parametry: nazwa pliku, skala (jak nie wiesz co wpisać, to wpisz 1)

\begin{document}
\maketitle


\kategoria{Wikieł/Z1.76b}
\zadStart{Zadanie z Wikieł Z 1.76 b) moja wersja nr [nrWersji]}
%[a]:[2,3,4,5,6]
%[b]:[1,2,3,4,5]
%[c]:[3,4,5,6,7]
%[d]:[2,3,4,5,6]
%[a]<[c]
Wyznaczyć dziedzinę naturalną funkcji określonej podanym wzorem.
$$f(x)=(x-[a])^{\frac{3}{2}}(x-[b])^{\frac{2}{3}}-([c]-x)^{\frac{5}{4}}([d]-x)^{\frac{4}{5}}$$
\zadStop
\rozwStart{Adrianna Stobiecka}{}
Funkcje potęgowe $f_1(t)=t^{\frac{3}{2}}=\sqrt{t^3}$ oraz $f_3(u)=u^{\frac{5}{4}}=\sqrt[4]{u^5}$ są określone na zbiorze $\mathbb{R}_+\cup\{0\}$. Natomiast funkcje potęgowe $f_2(w)=w^{\frac{2}{3}}=\sqrt[3]{w^2}$ oraz $f_4(z)=z^{\frac{4}{5}}=\sqrt[5]{z^4}$ są określone na zbiorze $\mathbb{R}$. Zatem dziedziną funkcji $f$ jest zbiór
$$D_{f}=\{x\in\mathbb{R}:x-[a]\geq0 \land [c]-x\geq0\}=\{x\in\mathbb{R}:x\geq[a] \land x\leq[c]\}=[[a],[c]].$$
\rozwStop
\odpStart
$D_f=[[a],[c]]$
\odpStop
\testStart
A.$D_f=(-[c],-[a])$
B.$D_f=\mathbb{R}$
C.$D_f=[[a],[c]]$
D.$D_f=[[c],\infty)$
E.$D_f=([a],[c])$
F.$D_f=\mathbb{R}\setminus\{0\}$
G.$D_f=[[a],\infty)]$
H.$D_f=[-[c],-[a]]$
I.$D_f=(-\infty,[c]]$
\testStop
\kluczStart
C
\kluczStop



\end{document}
