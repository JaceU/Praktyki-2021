\documentclass[12pt, a4paper]{article}
\usepackage[utf8]{inputenc}
\usepackage{polski}

\usepackage{amsthm}  %pakiet do tworzenia twierdzeń itp.
\usepackage{amsmath} %pakiet do niektórych symboli matematycznych
\usepackage{amssymb} %pakiet do symboli mat., np. \nsubseteq
\usepackage{amsfonts}
\usepackage{graphicx} %obsługa plików graficznych z rozszerzeniem png, jpg
\theoremstyle{definition} %styl dla definicji
\newtheorem{zad}{} 
\title{Multizestaw zadań}
\author{Robert Fidytek}
%\date{\today}
\date{}
\newcounter{liczniksekcji}
\newcommand{\kategoria}[1]{\section{#1}} %olreślamy nazwę kateforii zadań
\newcommand{\zadStart}[1]{\begin{zad}#1\newline} %oznaczenie początku zadania
\newcommand{\zadStop}{\end{zad}}   %oznaczenie końca zadania
%Makra opcjonarne (nie muszą występować):
\newcommand{\rozwStart}[2]{\noindent \textbf{Rozwiązanie (autor #1 , recenzent #2): }\newline} %oznaczenie początku rozwiązania, opcjonarnie można wprowadzić informację o autorze rozwiązania zadania i recenzencie poprawności wykonania rozwiązania zadania
\newcommand{\rozwStop}{\newline}                                            %oznaczenie końca rozwiązania
\newcommand{\odpStart}{\noindent \textbf{Odpowiedź:}\newline}    %oznaczenie początku odpowiedzi końcowej (wypisanie wyniku)
\newcommand{\odpStop}{\newline}                                             %oznaczenie końca odpowiedzi końcowej (wypisanie wyniku)
\newcommand{\testStart}{\noindent \textbf{Test:}\newline} %ewentualne możliwe opcje odpowiedzi testowej: A. ? B. ? C. ? D. ? itd.
\newcommand{\testStop}{\newline} %koniec wprowadzania odpowiedzi testowych
\newcommand{\kluczStart}{\noindent \textbf{Test poprawna odpowiedź:}\newline} %klucz, poprawna odpowiedź pytania testowego (jedna literka): A lub B lub C lub D itd.
\newcommand{\kluczStop}{\newline} %koniec poprawnej odpowiedzi pytania testowego 
\newcommand{\wstawGrafike}[2]{\begin{figure}[h] \includegraphics[scale=#2] {#1} \end{figure}} %gdyby była potrzeba wstawienia obrazka, parametry: nazwa pliku, skala (jak nie wiesz co wpisać, to wpisz 1)

\begin{document}
\maketitle


\kategoria{Wikieł/Z1.136a}
\zadStart{Zadanie z Wikieł Z 1.136 a) moja wersja nr [nrWersji]}
%[b]:[2,3,4,5,6]
%[a]:[2,3,5,6,7,8,9,10]
%[c]=random.randint(2,10)
%[a3]=[a]**3
%[x3]=[c]**4
%[3x2y]=3*[b]*([c]**3)
%[3xy2]=3*[c]*[c]*[b]*[b]
%[y3]=([b]**3)*[c]
%[y3b]=[y3]+[b]
Dla danych funkcji $f$ i $g$ znaleźć złożenia funkcji $f \circ f$, $g \circ f$, $f \circ g$ oraz $g \circ g$.\\
a)$f(x)=[a]^{x}$, $g(x)=[c]x^{3}+[b]$
\zadStop
\rozwStart{Wojciech Przybylski}{}
$1. f \circ f$
$$f(f(x))=[a]^{[a]^{x}}=e^{ln([a]^{[a]^{x}})}=e^{[a]^{x}ln([a])}$$
$2. f \circ g$
$$f(g(x))=[a]^{[c]x^{3}+[b]}=e^{ln([a]^{[c]x^{3}+[b]})}=e^{ln([a]^{[c]x^{3}}\cdot[a]^{[b]})}=e^{([c]x^{3}ln([a])+[b]ln([a]))}$$
$3. g \circ f$
$$g(f(x))=[c]([a]^{x})^{3}+[b]=[c]\cdot[a3]^{x}+[b]$$
$4. g \circ g$
$$g(g(x))=[c]([c]x^{3}+[b])^{3}+[b]=[x3]x^{3}+[3x2y]x^{2}+[3xy2]x+[y3]+[b]=$$
$$=[x3]x^{3}+[3x2y]x^{2}+[3xy2]x+[y3b]$$
\rozwStop
\odpStart
$1. f \circ f = e^{[a]^{x}ln([a])}$\\
$2. f \circ g = e^{([c]x^{3}ln([a])+[b]ln([a]))}$\\
$3. g \circ f = [c]\cdot[a3]^{x}+[b]$\\
$4. g \circ g =[x3]x^{3}+[3x2y]x^{2}+[3xy2]x+[y3b] $
\odpStop
\testStart
A. $1. f \circ f = e^{[a]^{x}ln([a])}$\\
$2. f \circ g = e^{([c]x^{3}ln([a])+[b]ln([a]))}$\\
$3. g \circ f = [c]\cdot[a3]^{x}+[b]$\\
$4. g \circ g =[x3]x^{3}+[3x2y]x^{2}+[3xy2]x+[y3b] $\\
\\
B. $1. f \circ f = [a]^{x}ln([a])$\\
$2. f \circ g = e^{([c]x^{3}ln([a])+[b]ln([a]))}$\\
$3. g \circ f = [c]\cdot[a3]^{x}+[b]$\\
$4. g \circ g =[x3]x^{3}+[3x2y]x^{2}+[3xy2]x+[y3b] $\\
\\
C. $1. f \circ f = e^{[a]^{x}ln([a])}$\\
$2. f \circ g = e^{([c]x^{3}ln([a])+[b]ln([a]))}$\\
$3. g \circ f = [c]\cdot[a3]^{x}$\\
$4. g \circ g =[x3]x^{3}+[3x2y]x^{2}+[3xy2]x+[y3b] $\\
\\
D. $1. f \circ f = e^{[a]^{x}ln([a])}$\\
$2. f \circ g = e^{([c]x^{3}ln([a])+[b]ln([a]))}$\\
$3. g \circ f = [c]\cdot[a3]^{x}+[b]$\\
$4. g \circ g =[x3]x^{4}+[3x2y]x^{2}+[3xy2]x+[y3b] $\\
\\
E. $1. f \circ f = e^{[a]^{x}log([a])}$\\
$2. f \circ g = e^{([c]x^{3}log([a])+[b]log([a]))}$\\
$3. g \circ f = [c]\cdot[a3]^{x}+[b]$\\
$4. g \circ g =[x3]x^{3}+[3x2y]x^{2}+[3xy2]x+[y3b] $\\
\\
F. $1. f \circ f = e^{[a]^{x}log([a])}$\\
$2. f \circ g = e^{([c]x^{3}log([a])+[b]log([a]))}$\\
$3. g \circ f = [c]\cdot[a3]^{x}+[b]$\\
$4. g \circ g =[x3]x^{3}+[3x2y]x^{2}+[3xy2]x+[y3] $\\
\testStop
\kluczStart
A
\kluczStop



\end{document}