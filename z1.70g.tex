\documentclass[12pt, a4paper]{article}
\usepackage[utf8]{inputenc}
\usepackage{polski}

\usepackage{amsthm}  %pakiet do tworzenia twierdzeń itp.
\usepackage{amsmath} %pakiet do niektórych symboli matematycznych
\usepackage{amssymb} %pakiet do symboli mat., np. \nsubseteq
\usepackage{amsfonts}
\usepackage{graphicx} %obsługa plików graficznych z rozszerzeniem png, jpg
\theoremstyle{definition} %styl dla definicji
\newtheorem{zad}{} 
\title{Multizestaw zadań}
\author{Robert Fidytek}
%\date{\today}
\date{}
\newcounter{liczniksekcji}
\newcommand{\kategoria}[1]{\section{#1}} %olreślamy nazwę kateforii zadań
\newcommand{\zadStart}[1]{\begin{zad}#1\newline} %oznaczenie początku zadania
\newcommand{\zadStop}{\end{zad}}   %oznaczenie końca zadania
%Makra opcjonarne (nie muszą występować):
\newcommand{\rozwStart}[2]{\noindent \textbf{Rozwiązanie (autor #1 , recenzent #2): }\newline} %oznaczenie początku rozwiązania, opcjonarnie można wprowadzić informację o autorze rozwiązania zadania i recenzencie poprawności wykonania rozwiązania zadania
\newcommand{\rozwStop}{\newline}                                            %oznaczenie końca rozwiązania
\newcommand{\odpStart}{\noindent \textbf{Odpowiedź:}\newline}    %oznaczenie początku odpowiedzi końcowej (wypisanie wyniku)
\newcommand{\odpStop}{\newline}                                             %oznaczenie końca odpowiedzi końcowej (wypisanie wyniku)
\newcommand{\testStart}{\noindent \textbf{Test:}\newline} %ewentualne możliwe opcje odpowiedzi testowej: A. ? B. ? C. ? D. ? itd.
\newcommand{\testStop}{\newline} %koniec wprowadzania odpowiedzi testowych
\newcommand{\kluczStart}{\noindent \textbf{Test poprawna odpowiedź:}\newline} %klucz, poprawna odpowiedź pytania testowego (jedna literka): A lub B lub C lub D itd.
\newcommand{\kluczStop}{\newline} %koniec poprawnej odpowiedzi pytania testowego 
\newcommand{\wstawGrafike}[2]{\begin{figure}[h] \includegraphics[scale=#2] {#1} \end{figure}} %gdyby była potrzeba wstawienia obrazka, parametry: nazwa pliku, skala (jak nie wiesz co wpisać, to wpisz 1)

\begin{document}
\maketitle


\kategoria{Wikieł/Z1.70g}
\zadStart{Zadanie z Wikieł Z 1.70 g) moja wersja nr [nrWersji]}
%[a]:[2,3,4,5,6,7,8,9,10,11,12,13]
%[b]:[2,3,4,5,6,7,8,9,10,11,12,13]
%[c]:[2,3,4,5,6,7,8,9,10,11,12,13]
%[d]:[2,3,4,5,6,7,8,9,10,11,12,13]
%[a]=random.randint(2,25)
%[b]=random.randint(2,25)
%[c]=random.randint(2,25)
%[d]=random.randint(2,25)
%[dbc]=[d]-[b]-[c]
%[bd]=[b]*[d]
%[bc]=[b]*[c]
%[reszta]=[a]-[bd]+[bc]
%[cpb]=[c]+[b]
%[delta1]=pow([dbc],2)-4*[reszta]
%[delta2]=pow([cpb],2)-4*[bc]
%[sdelta1]=pow([delta1],1/2)
%[sdelta11]=int([sdelta1].real)
%[sdelta2]=pow([delta2],1/2)
%[sdelta22]=int([sdelta2].real)
%[x0]=(-[dbc])/2
%[x2]=(-[dbc]+[sdelta11])/2
%[x3]=([cpb]-[sdelta22])/2
%[x4]=([cpb]+[sdelta22])/2
%[x01]=int([x0])
%[x31]=int([x3])
%[x41]=int([x4])
%[b]!=[c] and [dbc]>0 and [reszta]!=0 and (pow([dbc],2)-4*[reszta])==0 and [x0]<[x3] 
Rozwiązać nierówność $\frac{[a]}{(x-[b])(x-[c])}+\frac{[d]}{(x-[c])}+1\leq0$
\zadStop
\rozwStart{Jakub Ulrych}{Pascal Nawrocki}
założenie: $$x-[b]\neq0\land x-[c]\neq0$$
$$x\neq[b]\land x\neq[c]$$
dziedzina:$$x\in \mathbb{R}-\{[b],[c]\}$$
rozwiązanie:$$\frac{[a]}{(x-[b])(x-[c])}+\frac{[d]}{(x-[c])}+1\leq0$$
$$\frac{[a]+[d](x-[b])+(x-[b])(x-[c])}{(x-[b])(x-[c])}\leq0$$
$$\frac{x^{2}+[dbc]x+[reszta]}{x^{2}-[cpb]x+[bc]}\leq0$$
Podzielimy rozwiązanie na 2 przypadki:
$$\textbf{1)}x^{2}+[dbc]x+[reszta]\leq0 \land x^{2}-[cpb]x+[bc]\geq0$$ $$\vee$$ $$\textbf{2)}x^{2}+[dbc]x+[reszta]\geq0 \land x^{2}-[cpb]x+[bc]\leq0$$
Liczymy delty (dla obu przypadków wynoszą tyle samo ponieważ są to te same wartości po lewej stronie, różni się tylko strona w którą zwrócona jest nierówność)
$$\Delta_{1}=[dbc]^{2}-4\cdot[reszta]=[delta1], \Delta_{2}=[cpb]^{2}-4\cdot [bc]=[delta2]$$
$$\sqrt{\Delta_{1}}=0,\sqrt{\Delta_{2}}=[sdelta22]$$
$$x_{0}=[x01], \text{   }x_{3}=[x31],x_{4}=[x41]$$
Pierwszy przypadek:
$$\textbf{1)}(x-([x01]))^{2}\leq0 \land (x-([x31]))(x-([x41]))\geq0$$
$$x=[x01]\land x\in(-\infty,[x31]]\cup [[x41],\infty)$$
Drugi przypadek:$$\textbf{2)}(x-([x01]))^{2}\geq0 \land (x-[x31])(x-[x41])\leq0$$
$$x\in\mathbb{R}\land x\in[[x31],[x41]]$$
Suma obu przypadków i uwzględnienie dziedziny:
$$(x=[x01] \vee x\in[[x31],[x41]])\land\text{dziedzina}$$
$$x\in[[x31],[x41]]\cup \{[x01]\}-\{[b],[c]\}$$
\rozwStop
\odpStart
$$x\in([x31],[x41])\cup \{[x01]\}$$
\odpStop
\testStart
A.$$x\in([x31],[x41])\cup \{[x01]\}$$
B.$$x\in(-[x31],[x41])\cup \{[x01]\}$$
C.$$x\in(-\infty,[x31])\cup\{28\}$$
D.$$x\in(-\infty,0)\cup\{28\}$$
\testStop
\kluczStart
A
\kluczStop
\end{document}