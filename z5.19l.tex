\documentclass[12pt, a4paper]{article}
\usepackage[utf8]{inputenc}
\usepackage{polski}

\usepackage{amsthm}  %pakiet do tworzenia twierdzeń itp.
\usepackage{amsmath} %pakiet do niektórych symboli matematycznych
\usepackage{amssymb} %pakiet do symboli mat., np. \nsubseteq
\usepackage{amsfonts}
\usepackage{graphicx} %obsługa plików graficznych z rozszerzeniem png, jpg
\theoremstyle{definition} %styl dla definicji
\newtheorem{zad}{} 
\title{Multizestaw zadań}
\author{Robert Fidytek}
%\date{\today}
\date{}
\newcounter{liczniksekcji}
\newcommand{\kategoria}[1]{\section{#1}} %olreślamy nazwę kateforii zadań
\newcommand{\zadStart}[1]{\begin{zad}#1\newline} %oznaczenie początku zadania
\newcommand{\zadStop}{\end{zad}}   %oznaczenie końca zadania
%Makra opcjonarne (nie muszą występować):
\newcommand{\rozwStart}[2]{\noindent \textbf{Rozwiązanie (autor #1 , recenzent #2): }\newline} %oznaczenie początku rozwiązania, opcjonarnie można wprowadzić informację o autorze rozwiązania zadania i recenzencie poprawności wykonania rozwiązania zadania
\newcommand{\rozwStop}{\newline}                                            %oznaczenie końca rozwiązania
\newcommand{\odpStart}{\noindent \textbf{Odpowiedź:}\newline}    %oznaczenie początku odpowiedzi końcowej (wypisanie wyniku)
\newcommand{\odpStop}{\newline}                                             %oznaczenie końca odpowiedzi końcowej (wypisanie wyniku)
\newcommand{\testStart}{\noindent \textbf{Test:}\newline} %ewentualne możliwe opcje odpowiedzi testowej: A. ? B. ? C. ? D. ? itd.
\newcommand{\testStop}{\newline} %koniec wprowadzania odpowiedzi testowych
\newcommand{\kluczStart}{\noindent \textbf{Test poprawna odpowiedź:}\newline} %klucz, poprawna odpowiedź pytania testowego (jedna literka): A lub B lub C lub D itd.
\newcommand{\kluczStop}{\newline} %koniec poprawnej odpowiedzi pytania testowego 
\newcommand{\wstawGrafike}[2]{\begin{figure}[h] \includegraphics[scale=#2] {#1} \end{figure}} %gdyby była potrzeba wstawienia obrazka, parametry: nazwa pliku, skala (jak nie wiesz co wpisać, to wpisz 1)

\begin{document}
\maketitle


\kategoria{Wikieł/Z5.19 l}
\zadStart{Zadanie z Wikieł Z 5.19 l) moja wersja nr [nrWersji]}
%[a]:[2,4,6, 8]
%[b]:[2,3,4,5,6,7,8,9]
%[a]!=0
Oblicz granicę $\lim_{x \rightarrow 0} \left( \frac{[b]}{\pi} \tg^{-1}(x) \right)^{x^{[a]}}$.
\zadStop
\rozwStart{Joanna Świerzbin}{}
$$\lim_{x \rightarrow 0} \left( \frac{[b]}{\pi} \tg^{-1}(x) \right)^{x^{[a]}} = 
\lim_{x \rightarrow 0} e^{ \ln\left(\left( \frac{[b]}{\pi} \tg^{-1}(x) \right)^{x^{[a]}}\right)}
= \lim_{x \rightarrow 0} e^{ x^{[a]}\ln\left( \frac{[b]}{\pi} \tg^{-1}(x) \right)} =$$
$$ = e^{ \lim_{x \rightarrow 0} x^{[a]}\ln\left( \frac{[b]}{\pi} \tg^{-1}(x) \right)} $$
Obliczmy ${ \lim_{x \rightarrow 0} x^{[a]}\ln\left( \frac{[b]}{\pi} \tg^{-1}(x) \right)}$
$${ \lim_{x \rightarrow 0} x^{[a]}\ln\left( \frac{[b]}{\pi} \tg^{-1}(x) \right)}
=  \lim_{x \rightarrow 0} \frac{\ln\left( \frac{[b]}{\pi} \tg^{-1}(x) \right)}{\frac{1}{x^{[a]}}}$$
Otrzymujemy $ \left[ \frac{-\infty}{\infty} \right] $ więc możemy skorzystać z twierdzenia de l'Hospitala.
$$\lim_{x \rightarrow 0} \frac{\left(\ln\left(\frac{[b]}{\pi} \tg^{-1}(x) \right)\right)'}{\left(\frac{1}{x^{[a]}}\right)'} 
=\lim_{x \rightarrow 0} \frac{ \frac{1}{\frac{[b]}{\pi} \tg^{-1}(x)} \left(\frac{[b]}{\pi} \tg^{-1}(x)\right)'}{\frac{-[a]}{x^{[a]+1}}} 
=\lim_{x \rightarrow 0} \frac{ \frac{\frac{[b]}{\pi} \cdot \frac{1}{x^2+1}}{\frac{[b]}{\pi} \tg^{-1}(x)} }{\frac{-[a]}{x^{[a]+1}}} =$$
$$=\lim_{x \rightarrow 0} \frac{\frac{1}{(x^2+1)\tg^{-1}(x)}}{\frac{-[a]}{x^{[a]+1}}} =
\lim_{x \rightarrow 0} \frac{x^{[a]+1}}{-[a](x^2+1)\tg^{-1}(x)} =-\frac{1}{[a]} \lim_{x \rightarrow 0} \frac{x^{[a]+1}}{(x^2+1)\tg^{-1}(x)}$$
Otrzymujemy $ \left[ \frac{0}{0} \right] $ więc możemy skorzystać z twierdzenia de l'Hospitala.
$$-\frac{1}{[a]} \lim_{x \rightarrow 0} \frac{\left(x^{[a]+1}\right)'}{\left((x^2+1)\tg^{-1}(x)\right)'} = -\frac{1}{[a]} \lim_{x \rightarrow 0} \frac{([a]+1)x^{[a]}}{2x \tg^{-1}(x)+(x^2+1)\frac{1}{x^2+1}}=$$
$$= -\frac{[a]+1}{[a]} \lim_{x \rightarrow 0} \frac{x^{[a]}}{2x \tg^{-1}(x)+1} = 0 $$
Podstawmy do początkowego przykładu.
$$ e^{ \lim_{x \rightarrow 0} x^{[a]}\ln\left( \frac{[b]}{\pi} \tg^{-1}(x) \right)}=e^0=1$$
\rozwStop
\odpStart
$ \lim_{x \rightarrow 0} \left( \frac{[b]}{\pi} \tg^{-1}(x) \right)^{x^{[a]}} =1 $
\odpStop
\testStart
A. $ \lim_{x \rightarrow 0} \left( \frac{[b]}{\pi} \tg^{-1}(x) \right)^{x^{[a]}} =1 $\\
B. $ \lim_{x \rightarrow 0} \left( \frac{[b]}{\pi} \tg^{-1}(x) \right)^{x^{[a]}} =\infty $\\
C. $ \lim_{x \rightarrow 0} \left( \frac{[b]}{\pi} \tg^{-1}(x) \right)^{x^{[a]}} =0 $\\
D. $ \lim_{x \rightarrow 0} \left( \frac{[b]}{\pi} \tg^{-1}(x) \right)^{x^{[a]}} =e $\\
E. $ \lim_{x \rightarrow 0} \left( \frac{[b]}{\pi} \tg^{-1}(x) \right)^{x^{[a]}} =[a] $\\
F. $ \lim_{x \rightarrow 0} \left( \frac{[b]}{\pi} \tg^{-1}(x) \right)^{x^{[a]}} =-1 $
\testStop
\kluczStart
A
\kluczStop



\end{document}