\documentclass[12pt, a4paper]{article}
\usepackage[utf8]{inputenc}
\usepackage{polski}

\usepackage{amsthm}  %pakiet do tworzenia twierdzeń itp.
\usepackage{amsmath} %pakiet do niektórych symboli matematycznych
\usepackage{amssymb} %pakiet do symboli mat., np. \nsubseteq
\usepackage{amsfonts}
\usepackage{graphicx} %obsługa plików graficznych z rozszerzeniem png, jpg
\theoremstyle{definition} %styl dla definicji
\newtheorem{zad}{} 
\title{Multizestaw zadań}
\author{Robert Fidytek}
%\date{\today}
\date{}
\newcounter{liczniksekcji}
\newcommand{\kategoria}[1]{\section{#1}} %olreślamy nazwę kateforii zadań
\newcommand{\zadStart}[1]{\begin{zad}#1\newline} %oznaczenie początku zadania
\newcommand{\zadStop}{\end{zad}}   %oznaczenie końca zadania
%Makra opcjonarne (nie muszą występować):
\newcommand{\rozwStart}[2]{\noindent \textbf{Rozwiązanie (autor #1 , recenzent #2): }\newline} %oznaczenie początku rozwiązania, opcjonarnie można wprowadzić informację o autorze rozwiązania zadania i recenzencie poprawności wykonania rozwiązania zadania
\newcommand{\rozwStop}{\newline}                                            %oznaczenie końca rozwiązania
\newcommand{\odpStart}{\noindent \textbf{Odpowiedź:}\newline}    %oznaczenie początku odpowiedzi końcowej (wypisanie wyniku)
\newcommand{\odpStop}{\newline}                                             %oznaczenie końca odpowiedzi końcowej (wypisanie wyniku)
\newcommand{\testStart}{\noindent \textbf{Test:}\newline} %ewentualne możliwe opcje odpowiedzi testowej: A. ? B. ? C. ? D. ? itd.
\newcommand{\testStop}{\newline} %koniec wprowadzania odpowiedzi testowych
\newcommand{\kluczStart}{\noindent \textbf{Test poprawna odpowiedź:}\newline} %klucz, poprawna odpowiedź pytania testowego (jedna literka): A lub B lub C lub D itd.
\newcommand{\kluczStop}{\newline} %koniec poprawnej odpowiedzi pytania testowego 
\newcommand{\wstawGrafike}[2]{\begin{figure}[h] \includegraphics[scale=#2] {#1} \end{figure}} %gdyby była potrzeba wstawienia obrazka, parametry: nazwa pliku, skala (jak nie wiesz co wpisać, to wpisz 1)

\begin{document}
\maketitle


\kategoria{Wikieł/Z5.59a}
\zadStart{Zadanie z Wikieł Z 5.59 a) moja wersja nr [nrWersji]}
%[a]:[2,3,4,5,6,7,8,9]
%[b]:[2,3,4,5,6,7,8,9]
%[a3]=[a]*3
%[b2]=[b]*2
%[a6]=[a3]*2
%[c]=round([a6]/[b2],2)
%[x1]=round([c]+1,2)
%[x2]=round([c]-1,2)
%[dodat]=round([a6]*[x1]-[b2],2)
%[ujem]=round([a6]*[x2]-[b2],2)
%[ujem]<0 and [dodat]>0
Zbadać wypukłość funkcji:
a) $ f(x)=[a]x^{3}-[b]x^{2}$
\zadStop
\rozwStart{Wojciech Przybylski}{}
$$ f(x)=[a]x^{3}-[b]x^{2}\hspace{5mm} \mathcal{D}_{f}=\mathbb{R}$$
$$ f'(x)=[a3]x^{2}-[b2]x, \hspace {5mm} f''(x)=[a6]x-[b2]$$
$$\mbox{Sprawdzamy warunek konieczny: }f''(x_{0})=0 \Rightarrow [a6]x_{0}-[b2]=0\Rightarrow x_{0}=[c]$$
$$\mbox{Sprawdzamy warunek dostateczny: }x_{1}=x_{0}+1, \hspace{3mm} x_{2}=x_{0}-1$$
$$f''(x_{1})=[a6]\cdot[x1]-[b2]=[dodat]>0,\hspace{3mm}f''(x_{2})=[a6]\cdot[x2]-[b2]=[ujem]<0$$
$$\mbox{Funkcja } f(x)=[a]x^{3}-[b]x^{2} \mbox{ jest wypukła na } ([c],\infty) \mbox{ oraz wklęsła na }(-\infty,[c])$$
\rozwStop
\odpStart
$f(x)$ jest wypukła na $([c],\infty)$, wklęsła na $(-\infty,[c])$.
\odpStop
\testStart
A. $f(x)$ jest wypukła na $([c],\infty)$, wklęsła na $(-\infty,[c])$.\\
B. $f(x)$ jest wypukła na $\mathbb{R}$.\\
C. $f(x)$ jest wypukła na $(-\infty,[c])$, wklęsła na $([c],\infty)$.\\
D. $f(x)$ jest wypukła na $([x2],\infty)$, wklęsła na $(-\infty,[x2])$.\\
E. $f(x)$ jest wypukła na $([x1],\infty)$, wklęsła na $(-\infty,[x1])$.\\
F. $f(x)$ nie jest wypukła.
\testStop
\kluczStart
A
\kluczStop



\end{document}