\documentclass[12pt, a4paper]{article}
\usepackage[utf8]{inputenc}
\usepackage{polski}

\usepackage{amsthm}  %pakiet do tworzenia twierdzeń itp.
\usepackage{amsmath} %pakiet do niektórych symboli matematycznych
\usepackage{amssymb} %pakiet do symboli mat., np. \nsubseteq
\usepackage{amsfonts}
\usepackage{graphicx} %obsługa plików graficznych z rozszerzeniem png, jpg
\theoremstyle{definition} %styl dla definicji
\newtheorem{zad}{} 
\title{Multizestaw zadań}
\author{Robert Fidytek}
%\date{\today}
\date{}
\newcounter{liczniksekcji}
\newcommand{\kategoria}[1]{\section{#1}} %olreślamy nazwę kateforii zadań
\newcommand{\zadStart}[1]{\begin{zad}#1\newline} %oznaczenie początku zadania
\newcommand{\zadStop}{\end{zad}}   %oznaczenie końca zadania
%Makra opcjonarne (nie muszą występować):
\newcommand{\rozwStart}[2]{\noindent \textbf{Rozwiązanie (autor #1 , recenzent #2): }\newline} %oznaczenie początku rozwiązania, opcjonarnie można wprowadzić informację o autorze rozwiązania zadania i recenzencie poprawności wykonania rozwiązania zadania
\newcommand{\rozwStop}{\newline}                                            %oznaczenie końca rozwiązania
\newcommand{\odpStart}{\noindent \textbf{Odpowiedź:}\newline}    %oznaczenie początku odpowiedzi końcowej (wypisanie wyniku)
\newcommand{\odpStop}{\newline}                                             %oznaczenie końca odpowiedzi końcowej (wypisanie wyniku)
\newcommand{\testStart}{\noindent \textbf{Test:}\newline} %ewentualne możliwe opcje odpowiedzi testowej: A. ? B. ? C. ? D. ? itd.
\newcommand{\testStop}{\newline} %koniec wprowadzania odpowiedzi testowych
\newcommand{\kluczStart}{\noindent \textbf{Test poprawna odpowiedź:}\newline} %klucz, poprawna odpowiedź pytania testowego (jedna literka): A lub B lub C lub D itd.
\newcommand{\kluczStop}{\newline} %koniec poprawnej odpowiedzi pytania testowego 
\newcommand{\wstawGrafike}[2]{\begin{figure}[h] \includegraphics[scale=#2] {#1} \end{figure}} %gdyby była potrzeba wstawienia obrazka, parametry: nazwa pliku, skala (jak nie wiesz co wpisać, to wpisz 1)

\begin{document}
\maketitle


\kategoria{Wikieł/Z5.6d}
\zadStart{Zadanie z Wikieł Z 5.6d) moja wersja nr [nrWersji]}
%[x]:[3,5,7,9,11,13,15,17,19,20,21,22]
%[y]:[2,3,4,5,6,7,8,9,10,11,12,15,17]
%[y]:[2,3,4,5,6,7,8,9,10,11]
%[a]=random.randint(2,30)
%[b]=random.randint(2,30)
%[c]=random.randint(2,30)
%[m]=[c]*[b]
%[n]=2*[a]
%[p]=3*[c]
%math.gcd([n],[p])==1 and math.gcd([b],[p])==1
Obliczyć pochodną funkcji $f$ oraz określić dziedzinę funkcji $f$ i funkcji pochodnej $f'$.\\
$f(x)=\frac{[a]x+[b]}{[c]\sqrt[3]{x}}$
\zadStop
\rozwStart{Katarzyna Filipowicz}{}
Dziedzina $D_f: ([c]\sqrt[3]{x}) \neq 0 \Rightarrow  x \in R\backslash \{0\}$
$$
f'(x)=\frac{[a]\cdot [c]\sqrt[3]{x}-\left([c] \cdot\frac{1}{3}x^{\frac{-2}{3}}\right)\left([a]x+[b]\right)}{[c] \cdot [c] \cdot x^{\frac{2}{3}}}=
\frac{[a]x^{\frac{1}{3}}-\left( \frac{[a]}{3}x^{\frac{-2}{3}}\cdot x + \frac{[b]}{3}x^{\frac{-2}{3}}\right)}{[c]}\cdot x^{\frac{-2}{3}}=
$$ $$
=\frac{[a]x^{\frac{-1}{3}}- \frac{[a]}{3}x^{\frac{-1}{3}} - \frac{[b]}{3}x^{\frac{-4}{3}}}{[c]}=
\frac{[n]x^{\frac{-1}{3}}- [b]x^{\frac{-4}{3}}}{3 \cdot [c]}=
$$ $$
=\frac{[n]x-[b]}{[p]x\sqrt[3]{x}}
$$
Dziedzina $D_{f'}: ([p]x\sqrt[3]{x}) \neq 0  \Rightarrow   x \in R\backslash \{0\}$
\rozwStop
\odpStart
$f'(x)=\frac{[n]x-[b]}{[p]x\sqrt[3]{x}}, D_f:x \in R\backslash \{0\}, D_{f'}:x \in R\backslash \{0\}$
\odpStop
\testStart
A.$f'(x)=\frac{[n]x-[b]}{[p]x\sqrt[3]{x}}, D_f:x \in R\backslash \{0\}, D_{f'}:x \in R\backslash \{0\}$\\
B.$f'(x)=\frac{[n]x-[b]}{[p]x\sqrt[3]{x}}, D_f:x \in x \in (0,\infty], D_{f'}:x  \in (0,\infty]$\\
C.$f'(x)=\frac{[n]x-[m]}{[c]x\sqrt[3]{x}}, D_f:x \in R\backslash \{0\}, D_{f'}:x \in R\backslash \{0\}$\\
D.$f'(x)=\frac{[a]x-[b]}{[p]x\sqrt[3]{x}}, D_f:x \in R\backslash \{0\}, D_{f'}:x \in R\backslash \{0\}$\\
E.$f'(x)=\frac{[n]x-[m]}{[n]x\sqrt[3]{x}}, D_f:x \in R\backslash \{0\}, D_{f'}:x \in R\backslash \{0\}$\\
F.$f'(x)=\frac{[n]x-[b]}{[p]x\sqrt[3]{x}}, D_f:x \in x \in (0,\infty], D_{f'}:x \in R\backslash \{0\}$\\
G.$f'(x)=[n]x-[b], D_f:x \in R\backslash \{0\}, D_{f'}:x \in R\backslash \{0\}$\\
H.$f'(x)=\frac{[a]}{\frac{1}{3}x^{\frac{-2}{3}}}, D_f:x \in R\backslash \{0\}, D_{f'}:x \in R\backslash \{0\}$\\
\testStop
\kluczStart
A
\kluczStop



\end{document}