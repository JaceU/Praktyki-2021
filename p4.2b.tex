\documentclass[12pt, a4paper]{article}
\usepackage[utf8]{inputenc}
\usepackage{polski}
\usepackage{amsthm}  %pakiet do tworzenia twierdzeń itp.
\usepackage{amsmath} %pakiet do niektórych symboli matematycznych
\usepackage{amssymb} %pakiet do symboli mat., np. \nsubseteq
\usepackage{amsfonts}
\usepackage{graphicx} %obsługa plików graficznych z rozszerzeniem png, jpg
\theoremstyle{definition} %styl dla definicji
\newtheorem{zad}{} 
\title{Multizestaw zadań}
\author{Patryk Wirkus}
%\date{\today}
\date{}
\newcommand{\kategoria}[1]{\section{#1}}
\newcommand{\zadStart}[1]{\begin{zad}#1\newline}
\newcommand{\zadStop}{\end{zad}}
\newcommand{\rozwStart}[2]{\noindent \textbf{Rozwiązanie (autor #1 , recenzent #2): }\newline}
\newcommand{\rozwStop}{\newline}                                           
\newcommand{\odpStart}{\noindent \textbf{Odpowiedź:}\newline}
\newcommand{\odpStop}{\newline}
\newcommand{\testStart}{\noindent \textbf{Test:}\newline}
\newcommand{\testStop}{\newline}
\newcommand{\kluczStart}{\noindent \textbf{Test poprawna odpowiedź:}\newline}
\newcommand{\kluczStop}{\newline}
\newcommand{\wstawGrafike}[2]{\begin{figure}[h] \includegraphics[scale=#2] {#1} \end{figure}}

\begin{document}
\maketitle

\kategoria{Wikieł/P4.2b}


\zadStart{Przykład z Wikieł P 4.2b moja wersja nr 1}
Obliczyć granicę $\lim\limits_{x\to\ 4}\frac{x^{2}-4^{2}}{(x-4)(x-7)}$.
\zadStop
\rozwStart{Patryk Wirkus}{Martyna Czarnobaj}
$$\frac{x^{2}-4^{2}}{(x-4)(x-7)}=\frac{x+4}{x-7}$$

$$\lim\limits_{x\to\ 4}\frac{x^{2}-4^{2}}{(x-4)(x-7)}=[\frac{0}{0}]=\lim\limits_{x\to\ 4}\frac{x+4}{x-7}=2 \cdot \frac{4}{4-7} = \frac{8}{-3}$$
\rozwStop
\odpStart
$\frac{8}{-3}$
\odpStop
\testStart
A.$\frac{8}{-3}$
B.$\infty$
C.$-\infty$
D.$0$
E.$\frac{8}{3}$
F.$\frac{4}{7}$
G.$-\frac{8}{3}$
H.$1$
I.$4$
\testStop
\kluczStart
A
\kluczStop



\zadStart{Przykład z Wikieł P 4.2b moja wersja nr 2}
Obliczyć granicę $\lim\limits_{x\to\ 4}\frac{x^{2}-4^{2}}{(x-4)(x-9)}$.
\zadStop
\rozwStart{Patryk Wirkus}{Martyna Czarnobaj}
$$\frac{x^{2}-4^{2}}{(x-4)(x-9)}=\frac{x+4}{x-9}$$

$$\lim\limits_{x\to\ 4}\frac{x^{2}-4^{2}}{(x-4)(x-9)}=[\frac{0}{0}]=\lim\limits_{x\to\ 4}\frac{x+4}{x-9}=2 \cdot \frac{4}{4-9} = \frac{8}{-5}$$
\rozwStop
\odpStart
$\frac{8}{-5}$
\odpStop
\testStart
A.$\frac{8}{-5}$
B.$\infty$
C.$-\infty$
D.$0$
E.$\frac{8}{5}$
F.$\frac{4}{9}$
G.$-\frac{8}{5}$
H.$1$
I.$4$
\testStop
\kluczStart
A
\kluczStop



\zadStart{Przykład z Wikieł P 4.2b moja wersja nr 3}
Obliczyć granicę $\lim\limits_{x\to\ 4}\frac{x^{2}-4^{2}}{(x-4)(x-11)}$.
\zadStop
\rozwStart{Patryk Wirkus}{Martyna Czarnobaj}
$$\frac{x^{2}-4^{2}}{(x-4)(x-11)}=\frac{x+4}{x-11}$$

$$\lim\limits_{x\to\ 4}\frac{x^{2}-4^{2}}{(x-4)(x-11)}=[\frac{0}{0}]=\lim\limits_{x\to\ 4}\frac{x+4}{x-11}=2 \cdot \frac{4}{4-11} = \frac{8}{-7}$$
\rozwStop
\odpStart
$\frac{8}{-7}$
\odpStop
\testStart
A.$\frac{8}{-7}$
B.$\infty$
C.$-\infty$
D.$0$
E.$\frac{8}{7}$
F.$\frac{4}{11}$
G.$-\frac{8}{7}$
H.$1$
I.$4$
\testStop
\kluczStart
A
\kluczStop



\zadStart{Przykład z Wikieł P 4.2b moja wersja nr 4}
Obliczyć granicę $\lim\limits_{x\to\ 4}\frac{x^{2}-4^{2}}{(x-4)(x-13)}$.
\zadStop
\rozwStart{Patryk Wirkus}{Martyna Czarnobaj}
$$\frac{x^{2}-4^{2}}{(x-4)(x-13)}=\frac{x+4}{x-13}$$

$$\lim\limits_{x\to\ 4}\frac{x^{2}-4^{2}}{(x-4)(x-13)}=[\frac{0}{0}]=\lim\limits_{x\to\ 4}\frac{x+4}{x-13}=2 \cdot \frac{4}{4-13} = \frac{8}{-9}$$
\rozwStop
\odpStart
$\frac{8}{-9}$
\odpStop
\testStart
A.$\frac{8}{-9}$
B.$\infty$
C.$-\infty$
D.$0$
E.$\frac{8}{9}$
F.$\frac{4}{13}$
G.$-\frac{8}{9}$
H.$1$
I.$4$
\testStop
\kluczStart
A
\kluczStop



\zadStart{Przykład z Wikieł P 4.2b moja wersja nr 5}
Obliczyć granicę $\lim\limits_{x\to\ 4}\frac{x^{2}-4^{2}}{(x-4)(x-15)}$.
\zadStop
\rozwStart{Patryk Wirkus}{Martyna Czarnobaj}
$$\frac{x^{2}-4^{2}}{(x-4)(x-15)}=\frac{x+4}{x-15}$$

$$\lim\limits_{x\to\ 4}\frac{x^{2}-4^{2}}{(x-4)(x-15)}=[\frac{0}{0}]=\lim\limits_{x\to\ 4}\frac{x+4}{x-15}=2 \cdot \frac{4}{4-15} = \frac{8}{-11}$$
\rozwStop
\odpStart
$\frac{8}{-11}$
\odpStop
\testStart
A.$\frac{8}{-11}$
B.$\infty$
C.$-\infty$
D.$0$
E.$\frac{8}{11}$
F.$\frac{4}{15}$
G.$-\frac{8}{11}$
H.$1$
I.$4$
\testStop
\kluczStart
A
\kluczStop



\zadStart{Przykład z Wikieł P 4.2b moja wersja nr 6}
Obliczyć granicę $\lim\limits_{x\to\ 4}\frac{x^{2}-4^{2}}{(x-4)(x-17)}$.
\zadStop
\rozwStart{Patryk Wirkus}{Martyna Czarnobaj}
$$\frac{x^{2}-4^{2}}{(x-4)(x-17)}=\frac{x+4}{x-17}$$

$$\lim\limits_{x\to\ 4}\frac{x^{2}-4^{2}}{(x-4)(x-17)}=[\frac{0}{0}]=\lim\limits_{x\to\ 4}\frac{x+4}{x-17}=2 \cdot \frac{4}{4-17} = \frac{8}{-13}$$
\rozwStop
\odpStart
$\frac{8}{-13}$
\odpStop
\testStart
A.$\frac{8}{-13}$
B.$\infty$
C.$-\infty$
D.$0$
E.$\frac{8}{13}$
F.$\frac{4}{17}$
G.$-\frac{8}{13}$
H.$1$
I.$4$
\testStop
\kluczStart
A
\kluczStop



\zadStart{Przykład z Wikieł P 4.2b moja wersja nr 7}
Obliczyć granicę $\lim\limits_{x\to\ 4}\frac{x^{2}-4^{2}}{(x-4)(x-19)}$.
\zadStop
\rozwStart{Patryk Wirkus}{Martyna Czarnobaj}
$$\frac{x^{2}-4^{2}}{(x-4)(x-19)}=\frac{x+4}{x-19}$$

$$\lim\limits_{x\to\ 4}\frac{x^{2}-4^{2}}{(x-4)(x-19)}=[\frac{0}{0}]=\lim\limits_{x\to\ 4}\frac{x+4}{x-19}=2 \cdot \frac{4}{4-19} = \frac{8}{-15}$$
\rozwStop
\odpStart
$\frac{8}{-15}$
\odpStop
\testStart
A.$\frac{8}{-15}$
B.$\infty$
C.$-\infty$
D.$0$
E.$\frac{8}{15}$
F.$\frac{4}{19}$
G.$-\frac{8}{15}$
H.$1$
I.$4$
\testStop
\kluczStart
A
\kluczStop



\zadStart{Przykład z Wikieł P 4.2b moja wersja nr 8}
Obliczyć granicę $\lim\limits_{x\to\ 4}\frac{x^{2}-4^{2}}{(x-4)(x-21)}$.
\zadStop
\rozwStart{Patryk Wirkus}{Martyna Czarnobaj}
$$\frac{x^{2}-4^{2}}{(x-4)(x-21)}=\frac{x+4}{x-21}$$

$$\lim\limits_{x\to\ 4}\frac{x^{2}-4^{2}}{(x-4)(x-21)}=[\frac{0}{0}]=\lim\limits_{x\to\ 4}\frac{x+4}{x-21}=2 \cdot \frac{4}{4-21} = \frac{8}{-17}$$
\rozwStop
\odpStart
$\frac{8}{-17}$
\odpStop
\testStart
A.$\frac{8}{-17}$
B.$\infty$
C.$-\infty$
D.$0$
E.$\frac{8}{17}$
F.$\frac{4}{21}$
G.$-\frac{8}{17}$
H.$1$
I.$4$
\testStop
\kluczStart
A
\kluczStop



\zadStart{Przykład z Wikieł P 4.2b moja wersja nr 9}
Obliczyć granicę $\lim\limits_{x\to\ 4}\frac{x^{2}-4^{2}}{(x-4)(x-23)}$.
\zadStop
\rozwStart{Patryk Wirkus}{Martyna Czarnobaj}
$$\frac{x^{2}-4^{2}}{(x-4)(x-23)}=\frac{x+4}{x-23}$$

$$\lim\limits_{x\to\ 4}\frac{x^{2}-4^{2}}{(x-4)(x-23)}=[\frac{0}{0}]=\lim\limits_{x\to\ 4}\frac{x+4}{x-23}=2 \cdot \frac{4}{4-23} = \frac{8}{-19}$$
\rozwStop
\odpStart
$\frac{8}{-19}$
\odpStop
\testStart
A.$\frac{8}{-19}$
B.$\infty$
C.$-\infty$
D.$0$
E.$\frac{8}{19}$
F.$\frac{4}{23}$
G.$-\frac{8}{19}$
H.$1$
I.$4$
\testStop
\kluczStart
A
\kluczStop



\zadStart{Przykład z Wikieł P 4.2b moja wersja nr 10}
Obliczyć granicę $\lim\limits_{x\to\ 4}\frac{x^{2}-4^{2}}{(x-4)(x-25)}$.
\zadStop
\rozwStart{Patryk Wirkus}{Martyna Czarnobaj}
$$\frac{x^{2}-4^{2}}{(x-4)(x-25)}=\frac{x+4}{x-25}$$

$$\lim\limits_{x\to\ 4}\frac{x^{2}-4^{2}}{(x-4)(x-25)}=[\frac{0}{0}]=\lim\limits_{x\to\ 4}\frac{x+4}{x-25}=2 \cdot \frac{4}{4-25} = \frac{8}{-21}$$
\rozwStop
\odpStart
$\frac{8}{-21}$
\odpStop
\testStart
A.$\frac{8}{-21}$
B.$\infty$
C.$-\infty$
D.$0$
E.$\frac{8}{21}$
F.$\frac{4}{25}$
G.$-\frac{8}{21}$
H.$1$
I.$4$
\testStop
\kluczStart
A
\kluczStop



\zadStart{Przykład z Wikieł P 4.2b moja wersja nr 11}
Obliczyć granicę $\lim\limits_{x\to\ 4}\frac{x^{2}-4^{2}}{(x-4)(x-27)}$.
\zadStop
\rozwStart{Patryk Wirkus}{Martyna Czarnobaj}
$$\frac{x^{2}-4^{2}}{(x-4)(x-27)}=\frac{x+4}{x-27}$$

$$\lim\limits_{x\to\ 4}\frac{x^{2}-4^{2}}{(x-4)(x-27)}=[\frac{0}{0}]=\lim\limits_{x\to\ 4}\frac{x+4}{x-27}=2 \cdot \frac{4}{4-27} = \frac{8}{-23}$$
\rozwStop
\odpStart
$\frac{8}{-23}$
\odpStop
\testStart
A.$\frac{8}{-23}$
B.$\infty$
C.$-\infty$
D.$0$
E.$\frac{8}{23}$
F.$\frac{4}{27}$
G.$-\frac{8}{23}$
H.$1$
I.$4$
\testStop
\kluczStart
A
\kluczStop



\zadStart{Przykład z Wikieł P 4.2b moja wersja nr 12}
Obliczyć granicę $\lim\limits_{x\to\ 4}\frac{x^{2}-4^{2}}{(x-4)(x-29)}$.
\zadStop
\rozwStart{Patryk Wirkus}{Martyna Czarnobaj}
$$\frac{x^{2}-4^{2}}{(x-4)(x-29)}=\frac{x+4}{x-29}$$

$$\lim\limits_{x\to\ 4}\frac{x^{2}-4^{2}}{(x-4)(x-29)}=[\frac{0}{0}]=\lim\limits_{x\to\ 4}\frac{x+4}{x-29}=2 \cdot \frac{4}{4-29} = \frac{8}{-25}$$
\rozwStop
\odpStart
$\frac{8}{-25}$
\odpStop
\testStart
A.$\frac{8}{-25}$
B.$\infty$
C.$-\infty$
D.$0$
E.$\frac{8}{25}$
F.$\frac{4}{29}$
G.$-\frac{8}{25}$
H.$1$
I.$4$
\testStop
\kluczStart
A
\kluczStop



\zadStart{Przykład z Wikieł P 4.2b moja wersja nr 13}
Obliczyć granicę $\lim\limits_{x\to\ 4}\frac{x^{2}-4^{2}}{(x-4)(x-31)}$.
\zadStop
\rozwStart{Patryk Wirkus}{Martyna Czarnobaj}
$$\frac{x^{2}-4^{2}}{(x-4)(x-31)}=\frac{x+4}{x-31}$$

$$\lim\limits_{x\to\ 4}\frac{x^{2}-4^{2}}{(x-4)(x-31)}=[\frac{0}{0}]=\lim\limits_{x\to\ 4}\frac{x+4}{x-31}=2 \cdot \frac{4}{4-31} = \frac{8}{-27}$$
\rozwStop
\odpStart
$\frac{8}{-27}$
\odpStop
\testStart
A.$\frac{8}{-27}$
B.$\infty$
C.$-\infty$
D.$0$
E.$\frac{8}{27}$
F.$\frac{4}{31}$
G.$-\frac{8}{27}$
H.$1$
I.$4$
\testStop
\kluczStart
A
\kluczStop



\zadStart{Przykład z Wikieł P 4.2b moja wersja nr 14}
Obliczyć granicę $\lim\limits_{x\to\ 4}\frac{x^{2}-4^{2}}{(x-4)(x-33)}$.
\zadStop
\rozwStart{Patryk Wirkus}{Martyna Czarnobaj}
$$\frac{x^{2}-4^{2}}{(x-4)(x-33)}=\frac{x+4}{x-33}$$

$$\lim\limits_{x\to\ 4}\frac{x^{2}-4^{2}}{(x-4)(x-33)}=[\frac{0}{0}]=\lim\limits_{x\to\ 4}\frac{x+4}{x-33}=2 \cdot \frac{4}{4-33} = \frac{8}{-29}$$
\rozwStop
\odpStart
$\frac{8}{-29}$
\odpStop
\testStart
A.$\frac{8}{-29}$
B.$\infty$
C.$-\infty$
D.$0$
E.$\frac{8}{29}$
F.$\frac{4}{33}$
G.$-\frac{8}{29}$
H.$1$
I.$4$
\testStop
\kluczStart
A
\kluczStop



\zadStart{Przykład z Wikieł P 4.2b moja wersja nr 15}
Obliczyć granicę $\lim\limits_{x\to\ 4}\frac{x^{2}-4^{2}}{(x-4)(x-35)}$.
\zadStop
\rozwStart{Patryk Wirkus}{Martyna Czarnobaj}
$$\frac{x^{2}-4^{2}}{(x-4)(x-35)}=\frac{x+4}{x-35}$$

$$\lim\limits_{x\to\ 4}\frac{x^{2}-4^{2}}{(x-4)(x-35)}=[\frac{0}{0}]=\lim\limits_{x\to\ 4}\frac{x+4}{x-35}=2 \cdot \frac{4}{4-35} = \frac{8}{-31}$$
\rozwStop
\odpStart
$\frac{8}{-31}$
\odpStop
\testStart
A.$\frac{8}{-31}$
B.$\infty$
C.$-\infty$
D.$0$
E.$\frac{8}{31}$
F.$\frac{4}{35}$
G.$-\frac{8}{31}$
H.$1$
I.$4$
\testStop
\kluczStart
A
\kluczStop



\zadStart{Przykład z Wikieł P 4.2b moja wersja nr 16}
Obliczyć granicę $\lim\limits_{x\to\ 4}\frac{x^{2}-4^{2}}{(x-4)(x-37)}$.
\zadStop
\rozwStart{Patryk Wirkus}{Martyna Czarnobaj}
$$\frac{x^{2}-4^{2}}{(x-4)(x-37)}=\frac{x+4}{x-37}$$

$$\lim\limits_{x\to\ 4}\frac{x^{2}-4^{2}}{(x-4)(x-37)}=[\frac{0}{0}]=\lim\limits_{x\to\ 4}\frac{x+4}{x-37}=2 \cdot \frac{4}{4-37} = \frac{8}{-33}$$
\rozwStop
\odpStart
$\frac{8}{-33}$
\odpStop
\testStart
A.$\frac{8}{-33}$
B.$\infty$
C.$-\infty$
D.$0$
E.$\frac{8}{33}$
F.$\frac{4}{37}$
G.$-\frac{8}{33}$
H.$1$
I.$4$
\testStop
\kluczStart
A
\kluczStop



\zadStart{Przykład z Wikieł P 4.2b moja wersja nr 17}
Obliczyć granicę $\lim\limits_{x\to\ 4}\frac{x^{2}-4^{2}}{(x-4)(x-39)}$.
\zadStop
\rozwStart{Patryk Wirkus}{Martyna Czarnobaj}
$$\frac{x^{2}-4^{2}}{(x-4)(x-39)}=\frac{x+4}{x-39}$$

$$\lim\limits_{x\to\ 4}\frac{x^{2}-4^{2}}{(x-4)(x-39)}=[\frac{0}{0}]=\lim\limits_{x\to\ 4}\frac{x+4}{x-39}=2 \cdot \frac{4}{4-39} = \frac{8}{-35}$$
\rozwStop
\odpStart
$\frac{8}{-35}$
\odpStop
\testStart
A.$\frac{8}{-35}$
B.$\infty$
C.$-\infty$
D.$0$
E.$\frac{8}{35}$
F.$\frac{4}{39}$
G.$-\frac{8}{35}$
H.$1$
I.$4$
\testStop
\kluczStart
A
\kluczStop



\zadStart{Przykład z Wikieł P 4.2b moja wersja nr 18}
Obliczyć granicę $\lim\limits_{x\to\ 5}\frac{x^{2}-5^{2}}{(x-5)(x-2)}$.
\zadStop
\rozwStart{Patryk Wirkus}{Martyna Czarnobaj}
$$\frac{x^{2}-5^{2}}{(x-5)(x-2)}=\frac{x+5}{x-2}$$

$$\lim\limits_{x\to\ 5}\frac{x^{2}-5^{2}}{(x-5)(x-2)}=[\frac{0}{0}]=\lim\limits_{x\to\ 5}\frac{x+5}{x-2}=2 \cdot \frac{5}{5-2} = \frac{10}{3}$$
\rozwStop
\odpStart
$\frac{10}{3}$
\odpStop
\testStart
A.$\frac{10}{3}$
B.$\infty$
C.$-\infty$
D.$0$
E.$\frac{10}{-3}$
F.$\frac{5}{2}$
G.$-\frac{10}{-3}$
H.$1$
I.$5$
\testStop
\kluczStart
A
\kluczStop



\zadStart{Przykład z Wikieł P 4.2b moja wersja nr 19}
Obliczyć granicę $\lim\limits_{x\to\ 5}\frac{x^{2}-5^{2}}{(x-5)(x-3)}$.
\zadStop
\rozwStart{Patryk Wirkus}{Martyna Czarnobaj}
$$\frac{x^{2}-5^{2}}{(x-5)(x-3)}=\frac{x+5}{x-3}$$

$$\lim\limits_{x\to\ 5}\frac{x^{2}-5^{2}}{(x-5)(x-3)}=[\frac{0}{0}]=\lim\limits_{x\to\ 5}\frac{x+5}{x-3}=2 \cdot \frac{5}{5-3} = \frac{10}{2}$$
\rozwStop
\odpStart
$\frac{10}{2}$
\odpStop
\testStart
A.$\frac{10}{2}$
B.$\infty$
C.$-\infty$
D.$0$
E.$\frac{10}{-2}$
F.$\frac{5}{3}$
G.$-\frac{10}{-2}$
H.$1$
I.$5$
\testStop
\kluczStart
A
\kluczStop



\zadStart{Przykład z Wikieł P 4.2b moja wersja nr 20}
Obliczyć granicę $\lim\limits_{x\to\ 5}\frac{x^{2}-5^{2}}{(x-5)(x-7)}$.
\zadStop
\rozwStart{Patryk Wirkus}{Martyna Czarnobaj}
$$\frac{x^{2}-5^{2}}{(x-5)(x-7)}=\frac{x+5}{x-7}$$

$$\lim\limits_{x\to\ 5}\frac{x^{2}-5^{2}}{(x-5)(x-7)}=[\frac{0}{0}]=\lim\limits_{x\to\ 5}\frac{x+5}{x-7}=2 \cdot \frac{5}{5-7} = \frac{10}{-2}$$
\rozwStop
\odpStart
$\frac{10}{-2}$
\odpStop
\testStart
A.$\frac{10}{-2}$
B.$\infty$
C.$-\infty$
D.$0$
E.$\frac{10}{2}$
F.$\frac{5}{7}$
G.$-\frac{10}{2}$
H.$1$
I.$5$
\testStop
\kluczStart
A
\kluczStop



\zadStart{Przykład z Wikieł P 4.2b moja wersja nr 21}
Obliczyć granicę $\lim\limits_{x\to\ 5}\frac{x^{2}-5^{2}}{(x-5)(x-8)}$.
\zadStop
\rozwStart{Patryk Wirkus}{Martyna Czarnobaj}
$$\frac{x^{2}-5^{2}}{(x-5)(x-8)}=\frac{x+5}{x-8}$$

$$\lim\limits_{x\to\ 5}\frac{x^{2}-5^{2}}{(x-5)(x-8)}=[\frac{0}{0}]=\lim\limits_{x\to\ 5}\frac{x+5}{x-8}=2 \cdot \frac{5}{5-8} = \frac{10}{-3}$$
\rozwStop
\odpStart
$\frac{10}{-3}$
\odpStop
\testStart
A.$\frac{10}{-3}$
B.$\infty$
C.$-\infty$
D.$0$
E.$\frac{10}{3}$
F.$\frac{5}{8}$
G.$-\frac{10}{3}$
H.$1$
I.$5$
\testStop
\kluczStart
A
\kluczStop



\zadStart{Przykład z Wikieł P 4.2b moja wersja nr 22}
Obliczyć granicę $\lim\limits_{x\to\ 5}\frac{x^{2}-5^{2}}{(x-5)(x-9)}$.
\zadStop
\rozwStart{Patryk Wirkus}{Martyna Czarnobaj}
$$\frac{x^{2}-5^{2}}{(x-5)(x-9)}=\frac{x+5}{x-9}$$

$$\lim\limits_{x\to\ 5}\frac{x^{2}-5^{2}}{(x-5)(x-9)}=[\frac{0}{0}]=\lim\limits_{x\to\ 5}\frac{x+5}{x-9}=2 \cdot \frac{5}{5-9} = \frac{10}{-4}$$
\rozwStop
\odpStart
$\frac{10}{-4}$
\odpStop
\testStart
A.$\frac{10}{-4}$
B.$\infty$
C.$-\infty$
D.$0$
E.$\frac{10}{4}$
F.$\frac{5}{9}$
G.$-\frac{10}{4}$
H.$1$
I.$5$
\testStop
\kluczStart
A
\kluczStop



\zadStart{Przykład z Wikieł P 4.2b moja wersja nr 23}
Obliczyć granicę $\lim\limits_{x\to\ 5}\frac{x^{2}-5^{2}}{(x-5)(x-11)}$.
\zadStop
\rozwStart{Patryk Wirkus}{Martyna Czarnobaj}
$$\frac{x^{2}-5^{2}}{(x-5)(x-11)}=\frac{x+5}{x-11}$$

$$\lim\limits_{x\to\ 5}\frac{x^{2}-5^{2}}{(x-5)(x-11)}=[\frac{0}{0}]=\lim\limits_{x\to\ 5}\frac{x+5}{x-11}=2 \cdot \frac{5}{5-11} = \frac{10}{-6}$$
\rozwStop
\odpStart
$\frac{10}{-6}$
\odpStop
\testStart
A.$\frac{10}{-6}$
B.$\infty$
C.$-\infty$
D.$0$
E.$\frac{10}{6}$
F.$\frac{5}{11}$
G.$-\frac{10}{6}$
H.$1$
I.$5$
\testStop
\kluczStart
A
\kluczStop



\zadStart{Przykład z Wikieł P 4.2b moja wersja nr 24}
Obliczyć granicę $\lim\limits_{x\to\ 5}\frac{x^{2}-5^{2}}{(x-5)(x-12)}$.
\zadStop
\rozwStart{Patryk Wirkus}{Martyna Czarnobaj}
$$\frac{x^{2}-5^{2}}{(x-5)(x-12)}=\frac{x+5}{x-12}$$

$$\lim\limits_{x\to\ 5}\frac{x^{2}-5^{2}}{(x-5)(x-12)}=[\frac{0}{0}]=\lim\limits_{x\to\ 5}\frac{x+5}{x-12}=2 \cdot \frac{5}{5-12} = \frac{10}{-7}$$
\rozwStop
\odpStart
$\frac{10}{-7}$
\odpStop
\testStart
A.$\frac{10}{-7}$
B.$\infty$
C.$-\infty$
D.$0$
E.$\frac{10}{7}$
F.$\frac{5}{12}$
G.$-\frac{10}{7}$
H.$1$
I.$5$
\testStop
\kluczStart
A
\kluczStop



\zadStart{Przykład z Wikieł P 4.2b moja wersja nr 25}
Obliczyć granicę $\lim\limits_{x\to\ 5}\frac{x^{2}-5^{2}}{(x-5)(x-13)}$.
\zadStop
\rozwStart{Patryk Wirkus}{Martyna Czarnobaj}
$$\frac{x^{2}-5^{2}}{(x-5)(x-13)}=\frac{x+5}{x-13}$$

$$\lim\limits_{x\to\ 5}\frac{x^{2}-5^{2}}{(x-5)(x-13)}=[\frac{0}{0}]=\lim\limits_{x\to\ 5}\frac{x+5}{x-13}=2 \cdot \frac{5}{5-13} = \frac{10}{-8}$$
\rozwStop
\odpStart
$\frac{10}{-8}$
\odpStop
\testStart
A.$\frac{10}{-8}$
B.$\infty$
C.$-\infty$
D.$0$
E.$\frac{10}{8}$
F.$\frac{5}{13}$
G.$-\frac{10}{8}$
H.$1$
I.$5$
\testStop
\kluczStart
A
\kluczStop



\zadStart{Przykład z Wikieł P 4.2b moja wersja nr 26}
Obliczyć granicę $\lim\limits_{x\to\ 5}\frac{x^{2}-5^{2}}{(x-5)(x-14)}$.
\zadStop
\rozwStart{Patryk Wirkus}{Martyna Czarnobaj}
$$\frac{x^{2}-5^{2}}{(x-5)(x-14)}=\frac{x+5}{x-14}$$

$$\lim\limits_{x\to\ 5}\frac{x^{2}-5^{2}}{(x-5)(x-14)}=[\frac{0}{0}]=\lim\limits_{x\to\ 5}\frac{x+5}{x-14}=2 \cdot \frac{5}{5-14} = \frac{10}{-9}$$
\rozwStop
\odpStart
$\frac{10}{-9}$
\odpStop
\testStart
A.$\frac{10}{-9}$
B.$\infty$
C.$-\infty$
D.$0$
E.$\frac{10}{9}$
F.$\frac{5}{14}$
G.$-\frac{10}{9}$
H.$1$
I.$5$
\testStop
\kluczStart
A
\kluczStop



\zadStart{Przykład z Wikieł P 4.2b moja wersja nr 27}
Obliczyć granicę $\lim\limits_{x\to\ 5}\frac{x^{2}-5^{2}}{(x-5)(x-16)}$.
\zadStop
\rozwStart{Patryk Wirkus}{Martyna Czarnobaj}
$$\frac{x^{2}-5^{2}}{(x-5)(x-16)}=\frac{x+5}{x-16}$$

$$\lim\limits_{x\to\ 5}\frac{x^{2}-5^{2}}{(x-5)(x-16)}=[\frac{0}{0}]=\lim\limits_{x\to\ 5}\frac{x+5}{x-16}=2 \cdot \frac{5}{5-16} = \frac{10}{-11}$$
\rozwStop
\odpStart
$\frac{10}{-11}$
\odpStop
\testStart
A.$\frac{10}{-11}$
B.$\infty$
C.$-\infty$
D.$0$
E.$\frac{10}{11}$
F.$\frac{5}{16}$
G.$-\frac{10}{11}$
H.$1$
I.$5$
\testStop
\kluczStart
A
\kluczStop



\zadStart{Przykład z Wikieł P 4.2b moja wersja nr 28}
Obliczyć granicę $\lim\limits_{x\to\ 5}\frac{x^{2}-5^{2}}{(x-5)(x-17)}$.
\zadStop
\rozwStart{Patryk Wirkus}{Martyna Czarnobaj}
$$\frac{x^{2}-5^{2}}{(x-5)(x-17)}=\frac{x+5}{x-17}$$

$$\lim\limits_{x\to\ 5}\frac{x^{2}-5^{2}}{(x-5)(x-17)}=[\frac{0}{0}]=\lim\limits_{x\to\ 5}\frac{x+5}{x-17}=2 \cdot \frac{5}{5-17} = \frac{10}{-12}$$
\rozwStop
\odpStart
$\frac{10}{-12}$
\odpStop
\testStart
A.$\frac{10}{-12}$
B.$\infty$
C.$-\infty$
D.$0$
E.$\frac{10}{12}$
F.$\frac{5}{17}$
G.$-\frac{10}{12}$
H.$1$
I.$5$
\testStop
\kluczStart
A
\kluczStop



\zadStart{Przykład z Wikieł P 4.2b moja wersja nr 29}
Obliczyć granicę $\lim\limits_{x\to\ 5}\frac{x^{2}-5^{2}}{(x-5)(x-18)}$.
\zadStop
\rozwStart{Patryk Wirkus}{Martyna Czarnobaj}
$$\frac{x^{2}-5^{2}}{(x-5)(x-18)}=\frac{x+5}{x-18}$$

$$\lim\limits_{x\to\ 5}\frac{x^{2}-5^{2}}{(x-5)(x-18)}=[\frac{0}{0}]=\lim\limits_{x\to\ 5}\frac{x+5}{x-18}=2 \cdot \frac{5}{5-18} = \frac{10}{-13}$$
\rozwStop
\odpStart
$\frac{10}{-13}$
\odpStop
\testStart
A.$\frac{10}{-13}$
B.$\infty$
C.$-\infty$
D.$0$
E.$\frac{10}{13}$
F.$\frac{5}{18}$
G.$-\frac{10}{13}$
H.$1$
I.$5$
\testStop
\kluczStart
A
\kluczStop



\zadStart{Przykład z Wikieł P 4.2b moja wersja nr 30}
Obliczyć granicę $\lim\limits_{x\to\ 5}\frac{x^{2}-5^{2}}{(x-5)(x-19)}$.
\zadStop
\rozwStart{Patryk Wirkus}{Martyna Czarnobaj}
$$\frac{x^{2}-5^{2}}{(x-5)(x-19)}=\frac{x+5}{x-19}$$

$$\lim\limits_{x\to\ 5}\frac{x^{2}-5^{2}}{(x-5)(x-19)}=[\frac{0}{0}]=\lim\limits_{x\to\ 5}\frac{x+5}{x-19}=2 \cdot \frac{5}{5-19} = \frac{10}{-14}$$
\rozwStop
\odpStart
$\frac{10}{-14}$
\odpStop
\testStart
A.$\frac{10}{-14}$
B.$\infty$
C.$-\infty$
D.$0$
E.$\frac{10}{14}$
F.$\frac{5}{19}$
G.$-\frac{10}{14}$
H.$1$
I.$5$
\testStop
\kluczStart
A
\kluczStop



\zadStart{Przykład z Wikieł P 4.2b moja wersja nr 31}
Obliczyć granicę $\lim\limits_{x\to\ 5}\frac{x^{2}-5^{2}}{(x-5)(x-21)}$.
\zadStop
\rozwStart{Patryk Wirkus}{Martyna Czarnobaj}
$$\frac{x^{2}-5^{2}}{(x-5)(x-21)}=\frac{x+5}{x-21}$$

$$\lim\limits_{x\to\ 5}\frac{x^{2}-5^{2}}{(x-5)(x-21)}=[\frac{0}{0}]=\lim\limits_{x\to\ 5}\frac{x+5}{x-21}=2 \cdot \frac{5}{5-21} = \frac{10}{-16}$$
\rozwStop
\odpStart
$\frac{10}{-16}$
\odpStop
\testStart
A.$\frac{10}{-16}$
B.$\infty$
C.$-\infty$
D.$0$
E.$\frac{10}{16}$
F.$\frac{5}{21}$
G.$-\frac{10}{16}$
H.$1$
I.$5$
\testStop
\kluczStart
A
\kluczStop



\zadStart{Przykład z Wikieł P 4.2b moja wersja nr 32}
Obliczyć granicę $\lim\limits_{x\to\ 5}\frac{x^{2}-5^{2}}{(x-5)(x-22)}$.
\zadStop
\rozwStart{Patryk Wirkus}{Martyna Czarnobaj}
$$\frac{x^{2}-5^{2}}{(x-5)(x-22)}=\frac{x+5}{x-22}$$

$$\lim\limits_{x\to\ 5}\frac{x^{2}-5^{2}}{(x-5)(x-22)}=[\frac{0}{0}]=\lim\limits_{x\to\ 5}\frac{x+5}{x-22}=2 \cdot \frac{5}{5-22} = \frac{10}{-17}$$
\rozwStop
\odpStart
$\frac{10}{-17}$
\odpStop
\testStart
A.$\frac{10}{-17}$
B.$\infty$
C.$-\infty$
D.$0$
E.$\frac{10}{17}$
F.$\frac{5}{22}$
G.$-\frac{10}{17}$
H.$1$
I.$5$
\testStop
\kluczStart
A
\kluczStop



\zadStart{Przykład z Wikieł P 4.2b moja wersja nr 33}
Obliczyć granicę $\lim\limits_{x\to\ 5}\frac{x^{2}-5^{2}}{(x-5)(x-23)}$.
\zadStop
\rozwStart{Patryk Wirkus}{Martyna Czarnobaj}
$$\frac{x^{2}-5^{2}}{(x-5)(x-23)}=\frac{x+5}{x-23}$$

$$\lim\limits_{x\to\ 5}\frac{x^{2}-5^{2}}{(x-5)(x-23)}=[\frac{0}{0}]=\lim\limits_{x\to\ 5}\frac{x+5}{x-23}=2 \cdot \frac{5}{5-23} = \frac{10}{-18}$$
\rozwStop
\odpStart
$\frac{10}{-18}$
\odpStop
\testStart
A.$\frac{10}{-18}$
B.$\infty$
C.$-\infty$
D.$0$
E.$\frac{10}{18}$
F.$\frac{5}{23}$
G.$-\frac{10}{18}$
H.$1$
I.$5$
\testStop
\kluczStart
A
\kluczStop



\zadStart{Przykład z Wikieł P 4.2b moja wersja nr 34}
Obliczyć granicę $\lim\limits_{x\to\ 5}\frac{x^{2}-5^{2}}{(x-5)(x-24)}$.
\zadStop
\rozwStart{Patryk Wirkus}{Martyna Czarnobaj}
$$\frac{x^{2}-5^{2}}{(x-5)(x-24)}=\frac{x+5}{x-24}$$

$$\lim\limits_{x\to\ 5}\frac{x^{2}-5^{2}}{(x-5)(x-24)}=[\frac{0}{0}]=\lim\limits_{x\to\ 5}\frac{x+5}{x-24}=2 \cdot \frac{5}{5-24} = \frac{10}{-19}$$
\rozwStop
\odpStart
$\frac{10}{-19}$
\odpStop
\testStart
A.$\frac{10}{-19}$
B.$\infty$
C.$-\infty$
D.$0$
E.$\frac{10}{19}$
F.$\frac{5}{24}$
G.$-\frac{10}{19}$
H.$1$
I.$5$
\testStop
\kluczStart
A
\kluczStop



\zadStart{Przykład z Wikieł P 4.2b moja wersja nr 35}
Obliczyć granicę $\lim\limits_{x\to\ 5}\frac{x^{2}-5^{2}}{(x-5)(x-26)}$.
\zadStop
\rozwStart{Patryk Wirkus}{Martyna Czarnobaj}
$$\frac{x^{2}-5^{2}}{(x-5)(x-26)}=\frac{x+5}{x-26}$$

$$\lim\limits_{x\to\ 5}\frac{x^{2}-5^{2}}{(x-5)(x-26)}=[\frac{0}{0}]=\lim\limits_{x\to\ 5}\frac{x+5}{x-26}=2 \cdot \frac{5}{5-26} = \frac{10}{-21}$$
\rozwStop
\odpStart
$\frac{10}{-21}$
\odpStop
\testStart
A.$\frac{10}{-21}$
B.$\infty$
C.$-\infty$
D.$0$
E.$\frac{10}{21}$
F.$\frac{5}{26}$
G.$-\frac{10}{21}$
H.$1$
I.$5$
\testStop
\kluczStart
A
\kluczStop



\zadStart{Przykład z Wikieł P 4.2b moja wersja nr 36}
Obliczyć granicę $\lim\limits_{x\to\ 5}\frac{x^{2}-5^{2}}{(x-5)(x-27)}$.
\zadStop
\rozwStart{Patryk Wirkus}{Martyna Czarnobaj}
$$\frac{x^{2}-5^{2}}{(x-5)(x-27)}=\frac{x+5}{x-27}$$

$$\lim\limits_{x\to\ 5}\frac{x^{2}-5^{2}}{(x-5)(x-27)}=[\frac{0}{0}]=\lim\limits_{x\to\ 5}\frac{x+5}{x-27}=2 \cdot \frac{5}{5-27} = \frac{10}{-22}$$
\rozwStop
\odpStart
$\frac{10}{-22}$
\odpStop
\testStart
A.$\frac{10}{-22}$
B.$\infty$
C.$-\infty$
D.$0$
E.$\frac{10}{22}$
F.$\frac{5}{27}$
G.$-\frac{10}{22}$
H.$1$
I.$5$
\testStop
\kluczStart
A
\kluczStop



\zadStart{Przykład z Wikieł P 4.2b moja wersja nr 37}
Obliczyć granicę $\lim\limits_{x\to\ 5}\frac{x^{2}-5^{2}}{(x-5)(x-28)}$.
\zadStop
\rozwStart{Patryk Wirkus}{Martyna Czarnobaj}
$$\frac{x^{2}-5^{2}}{(x-5)(x-28)}=\frac{x+5}{x-28}$$

$$\lim\limits_{x\to\ 5}\frac{x^{2}-5^{2}}{(x-5)(x-28)}=[\frac{0}{0}]=\lim\limits_{x\to\ 5}\frac{x+5}{x-28}=2 \cdot \frac{5}{5-28} = \frac{10}{-23}$$
\rozwStop
\odpStart
$\frac{10}{-23}$
\odpStop
\testStart
A.$\frac{10}{-23}$
B.$\infty$
C.$-\infty$
D.$0$
E.$\frac{10}{23}$
F.$\frac{5}{28}$
G.$-\frac{10}{23}$
H.$1$
I.$5$
\testStop
\kluczStart
A
\kluczStop



\zadStart{Przykład z Wikieł P 4.2b moja wersja nr 38}
Obliczyć granicę $\lim\limits_{x\to\ 5}\frac{x^{2}-5^{2}}{(x-5)(x-29)}$.
\zadStop
\rozwStart{Patryk Wirkus}{Martyna Czarnobaj}
$$\frac{x^{2}-5^{2}}{(x-5)(x-29)}=\frac{x+5}{x-29}$$

$$\lim\limits_{x\to\ 5}\frac{x^{2}-5^{2}}{(x-5)(x-29)}=[\frac{0}{0}]=\lim\limits_{x\to\ 5}\frac{x+5}{x-29}=2 \cdot \frac{5}{5-29} = \frac{10}{-24}$$
\rozwStop
\odpStart
$\frac{10}{-24}$
\odpStop
\testStart
A.$\frac{10}{-24}$
B.$\infty$
C.$-\infty$
D.$0$
E.$\frac{10}{24}$
F.$\frac{5}{29}$
G.$-\frac{10}{24}$
H.$1$
I.$5$
\testStop
\kluczStart
A
\kluczStop



\zadStart{Przykład z Wikieł P 4.2b moja wersja nr 39}
Obliczyć granicę $\lim\limits_{x\to\ 5}\frac{x^{2}-5^{2}}{(x-5)(x-31)}$.
\zadStop
\rozwStart{Patryk Wirkus}{Martyna Czarnobaj}
$$\frac{x^{2}-5^{2}}{(x-5)(x-31)}=\frac{x+5}{x-31}$$

$$\lim\limits_{x\to\ 5}\frac{x^{2}-5^{2}}{(x-5)(x-31)}=[\frac{0}{0}]=\lim\limits_{x\to\ 5}\frac{x+5}{x-31}=2 \cdot \frac{5}{5-31} = \frac{10}{-26}$$
\rozwStop
\odpStart
$\frac{10}{-26}$
\odpStop
\testStart
A.$\frac{10}{-26}$
B.$\infty$
C.$-\infty$
D.$0$
E.$\frac{10}{26}$
F.$\frac{5}{31}$
G.$-\frac{10}{26}$
H.$1$
I.$5$
\testStop
\kluczStart
A
\kluczStop



\zadStart{Przykład z Wikieł P 4.2b moja wersja nr 40}
Obliczyć granicę $\lim\limits_{x\to\ 5}\frac{x^{2}-5^{2}}{(x-5)(x-32)}$.
\zadStop
\rozwStart{Patryk Wirkus}{Martyna Czarnobaj}
$$\frac{x^{2}-5^{2}}{(x-5)(x-32)}=\frac{x+5}{x-32}$$

$$\lim\limits_{x\to\ 5}\frac{x^{2}-5^{2}}{(x-5)(x-32)}=[\frac{0}{0}]=\lim\limits_{x\to\ 5}\frac{x+5}{x-32}=2 \cdot \frac{5}{5-32} = \frac{10}{-27}$$
\rozwStop
\odpStart
$\frac{10}{-27}$
\odpStop
\testStart
A.$\frac{10}{-27}$
B.$\infty$
C.$-\infty$
D.$0$
E.$\frac{10}{27}$
F.$\frac{5}{32}$
G.$-\frac{10}{27}$
H.$1$
I.$5$
\testStop
\kluczStart
A
\kluczStop



\zadStart{Przykład z Wikieł P 4.2b moja wersja nr 41}
Obliczyć granicę $\lim\limits_{x\to\ 5}\frac{x^{2}-5^{2}}{(x-5)(x-33)}$.
\zadStop
\rozwStart{Patryk Wirkus}{Martyna Czarnobaj}
$$\frac{x^{2}-5^{2}}{(x-5)(x-33)}=\frac{x+5}{x-33}$$

$$\lim\limits_{x\to\ 5}\frac{x^{2}-5^{2}}{(x-5)(x-33)}=[\frac{0}{0}]=\lim\limits_{x\to\ 5}\frac{x+5}{x-33}=2 \cdot \frac{5}{5-33} = \frac{10}{-28}$$
\rozwStop
\odpStart
$\frac{10}{-28}$
\odpStop
\testStart
A.$\frac{10}{-28}$
B.$\infty$
C.$-\infty$
D.$0$
E.$\frac{10}{28}$
F.$\frac{5}{33}$
G.$-\frac{10}{28}$
H.$1$
I.$5$
\testStop
\kluczStart
A
\kluczStop



\zadStart{Przykład z Wikieł P 4.2b moja wersja nr 42}
Obliczyć granicę $\lim\limits_{x\to\ 5}\frac{x^{2}-5^{2}}{(x-5)(x-34)}$.
\zadStop
\rozwStart{Patryk Wirkus}{Martyna Czarnobaj}
$$\frac{x^{2}-5^{2}}{(x-5)(x-34)}=\frac{x+5}{x-34}$$

$$\lim\limits_{x\to\ 5}\frac{x^{2}-5^{2}}{(x-5)(x-34)}=[\frac{0}{0}]=\lim\limits_{x\to\ 5}\frac{x+5}{x-34}=2 \cdot \frac{5}{5-34} = \frac{10}{-29}$$
\rozwStop
\odpStart
$\frac{10}{-29}$
\odpStop
\testStart
A.$\frac{10}{-29}$
B.$\infty$
C.$-\infty$
D.$0$
E.$\frac{10}{29}$
F.$\frac{5}{34}$
G.$-\frac{10}{29}$
H.$1$
I.$5$
\testStop
\kluczStart
A
\kluczStop



\zadStart{Przykład z Wikieł P 4.2b moja wersja nr 43}
Obliczyć granicę $\lim\limits_{x\to\ 5}\frac{x^{2}-5^{2}}{(x-5)(x-36)}$.
\zadStop
\rozwStart{Patryk Wirkus}{Martyna Czarnobaj}
$$\frac{x^{2}-5^{2}}{(x-5)(x-36)}=\frac{x+5}{x-36}$$

$$\lim\limits_{x\to\ 5}\frac{x^{2}-5^{2}}{(x-5)(x-36)}=[\frac{0}{0}]=\lim\limits_{x\to\ 5}\frac{x+5}{x-36}=2 \cdot \frac{5}{5-36} = \frac{10}{-31}$$
\rozwStop
\odpStart
$\frac{10}{-31}$
\odpStop
\testStart
A.$\frac{10}{-31}$
B.$\infty$
C.$-\infty$
D.$0$
E.$\frac{10}{31}$
F.$\frac{5}{36}$
G.$-\frac{10}{31}$
H.$1$
I.$5$
\testStop
\kluczStart
A
\kluczStop



\zadStart{Przykład z Wikieł P 4.2b moja wersja nr 44}
Obliczyć granicę $\lim\limits_{x\to\ 5}\frac{x^{2}-5^{2}}{(x-5)(x-37)}$.
\zadStop
\rozwStart{Patryk Wirkus}{Martyna Czarnobaj}
$$\frac{x^{2}-5^{2}}{(x-5)(x-37)}=\frac{x+5}{x-37}$$

$$\lim\limits_{x\to\ 5}\frac{x^{2}-5^{2}}{(x-5)(x-37)}=[\frac{0}{0}]=\lim\limits_{x\to\ 5}\frac{x+5}{x-37}=2 \cdot \frac{5}{5-37} = \frac{10}{-32}$$
\rozwStop
\odpStart
$\frac{10}{-32}$
\odpStop
\testStart
A.$\frac{10}{-32}$
B.$\infty$
C.$-\infty$
D.$0$
E.$\frac{10}{32}$
F.$\frac{5}{37}$
G.$-\frac{10}{32}$
H.$1$
I.$5$
\testStop
\kluczStart
A
\kluczStop



\zadStart{Przykład z Wikieł P 4.2b moja wersja nr 45}
Obliczyć granicę $\lim\limits_{x\to\ 5}\frac{x^{2}-5^{2}}{(x-5)(x-38)}$.
\zadStop
\rozwStart{Patryk Wirkus}{Martyna Czarnobaj}
$$\frac{x^{2}-5^{2}}{(x-5)(x-38)}=\frac{x+5}{x-38}$$

$$\lim\limits_{x\to\ 5}\frac{x^{2}-5^{2}}{(x-5)(x-38)}=[\frac{0}{0}]=\lim\limits_{x\to\ 5}\frac{x+5}{x-38}=2 \cdot \frac{5}{5-38} = \frac{10}{-33}$$
\rozwStop
\odpStart
$\frac{10}{-33}$
\odpStop
\testStart
A.$\frac{10}{-33}$
B.$\infty$
C.$-\infty$
D.$0$
E.$\frac{10}{33}$
F.$\frac{5}{38}$
G.$-\frac{10}{33}$
H.$1$
I.$5$
\testStop
\kluczStart
A
\kluczStop



\zadStart{Przykład z Wikieł P 4.2b moja wersja nr 46}
Obliczyć granicę $\lim\limits_{x\to\ 5}\frac{x^{2}-5^{2}}{(x-5)(x-39)}$.
\zadStop
\rozwStart{Patryk Wirkus}{Martyna Czarnobaj}
$$\frac{x^{2}-5^{2}}{(x-5)(x-39)}=\frac{x+5}{x-39}$$

$$\lim\limits_{x\to\ 5}\frac{x^{2}-5^{2}}{(x-5)(x-39)}=[\frac{0}{0}]=\lim\limits_{x\to\ 5}\frac{x+5}{x-39}=2 \cdot \frac{5}{5-39} = \frac{10}{-34}$$
\rozwStop
\odpStart
$\frac{10}{-34}$
\odpStop
\testStart
A.$\frac{10}{-34}$
B.$\infty$
C.$-\infty$
D.$0$
E.$\frac{10}{34}$
F.$\frac{5}{39}$
G.$-\frac{10}{34}$
H.$1$
I.$5$
\testStop
\kluczStart
A
\kluczStop



\zadStart{Przykład z Wikieł P 4.2b moja wersja nr 47}
Obliczyć granicę $\lim\limits_{x\to\ 6}\frac{x^{2}-6^{2}}{(x-6)(x-11)}$.
\zadStop
\rozwStart{Patryk Wirkus}{Martyna Czarnobaj}
$$\frac{x^{2}-6^{2}}{(x-6)(x-11)}=\frac{x+6}{x-11}$$

$$\lim\limits_{x\to\ 6}\frac{x^{2}-6^{2}}{(x-6)(x-11)}=[\frac{0}{0}]=\lim\limits_{x\to\ 6}\frac{x+6}{x-11}=2 \cdot \frac{6}{6-11} = \frac{12}{-5}$$
\rozwStop
\odpStart
$\frac{12}{-5}$
\odpStop
\testStart
A.$\frac{12}{-5}$
B.$\infty$
C.$-\infty$
D.$0$
E.$\frac{12}{5}$
F.$\frac{6}{11}$
G.$-\frac{12}{5}$
H.$1$
I.$6$
\testStop
\kluczStart
A
\kluczStop



\zadStart{Przykład z Wikieł P 4.2b moja wersja nr 48}
Obliczyć granicę $\lim\limits_{x\to\ 6}\frac{x^{2}-6^{2}}{(x-6)(x-13)}$.
\zadStop
\rozwStart{Patryk Wirkus}{Martyna Czarnobaj}
$$\frac{x^{2}-6^{2}}{(x-6)(x-13)}=\frac{x+6}{x-13}$$

$$\lim\limits_{x\to\ 6}\frac{x^{2}-6^{2}}{(x-6)(x-13)}=[\frac{0}{0}]=\lim\limits_{x\to\ 6}\frac{x+6}{x-13}=2 \cdot \frac{6}{6-13} = \frac{12}{-7}$$
\rozwStop
\odpStart
$\frac{12}{-7}$
\odpStop
\testStart
A.$\frac{12}{-7}$
B.$\infty$
C.$-\infty$
D.$0$
E.$\frac{12}{7}$
F.$\frac{6}{13}$
G.$-\frac{12}{7}$
H.$1$
I.$6$
\testStop
\kluczStart
A
\kluczStop



\zadStart{Przykład z Wikieł P 4.2b moja wersja nr 49}
Obliczyć granicę $\lim\limits_{x\to\ 6}\frac{x^{2}-6^{2}}{(x-6)(x-17)}$.
\zadStop
\rozwStart{Patryk Wirkus}{Martyna Czarnobaj}
$$\frac{x^{2}-6^{2}}{(x-6)(x-17)}=\frac{x+6}{x-17}$$

$$\lim\limits_{x\to\ 6}\frac{x^{2}-6^{2}}{(x-6)(x-17)}=[\frac{0}{0}]=\lim\limits_{x\to\ 6}\frac{x+6}{x-17}=2 \cdot \frac{6}{6-17} = \frac{12}{-11}$$
\rozwStop
\odpStart
$\frac{12}{-11}$
\odpStop
\testStart
A.$\frac{12}{-11}$
B.$\infty$
C.$-\infty$
D.$0$
E.$\frac{12}{11}$
F.$\frac{6}{17}$
G.$-\frac{12}{11}$
H.$1$
I.$6$
\testStop
\kluczStart
A
\kluczStop



\zadStart{Przykład z Wikieł P 4.2b moja wersja nr 50}
Obliczyć granicę $\lim\limits_{x\to\ 6}\frac{x^{2}-6^{2}}{(x-6)(x-19)}$.
\zadStop
\rozwStart{Patryk Wirkus}{Martyna Czarnobaj}
$$\frac{x^{2}-6^{2}}{(x-6)(x-19)}=\frac{x+6}{x-19}$$

$$\lim\limits_{x\to\ 6}\frac{x^{2}-6^{2}}{(x-6)(x-19)}=[\frac{0}{0}]=\lim\limits_{x\to\ 6}\frac{x+6}{x-19}=2 \cdot \frac{6}{6-19} = \frac{12}{-13}$$
\rozwStop
\odpStart
$\frac{12}{-13}$
\odpStop
\testStart
A.$\frac{12}{-13}$
B.$\infty$
C.$-\infty$
D.$0$
E.$\frac{12}{13}$
F.$\frac{6}{19}$
G.$-\frac{12}{13}$
H.$1$
I.$6$
\testStop
\kluczStart
A
\kluczStop



\zadStart{Przykład z Wikieł P 4.2b moja wersja nr 51}
Obliczyć granicę $\lim\limits_{x\to\ 6}\frac{x^{2}-6^{2}}{(x-6)(x-23)}$.
\zadStop
\rozwStart{Patryk Wirkus}{Martyna Czarnobaj}
$$\frac{x^{2}-6^{2}}{(x-6)(x-23)}=\frac{x+6}{x-23}$$

$$\lim\limits_{x\to\ 6}\frac{x^{2}-6^{2}}{(x-6)(x-23)}=[\frac{0}{0}]=\lim\limits_{x\to\ 6}\frac{x+6}{x-23}=2 \cdot \frac{6}{6-23} = \frac{12}{-17}$$
\rozwStop
\odpStart
$\frac{12}{-17}$
\odpStop
\testStart
A.$\frac{12}{-17}$
B.$\infty$
C.$-\infty$
D.$0$
E.$\frac{12}{17}$
F.$\frac{6}{23}$
G.$-\frac{12}{17}$
H.$1$
I.$6$
\testStop
\kluczStart
A
\kluczStop



\zadStart{Przykład z Wikieł P 4.2b moja wersja nr 52}
Obliczyć granicę $\lim\limits_{x\to\ 6}\frac{x^{2}-6^{2}}{(x-6)(x-25)}$.
\zadStop
\rozwStart{Patryk Wirkus}{Martyna Czarnobaj}
$$\frac{x^{2}-6^{2}}{(x-6)(x-25)}=\frac{x+6}{x-25}$$

$$\lim\limits_{x\to\ 6}\frac{x^{2}-6^{2}}{(x-6)(x-25)}=[\frac{0}{0}]=\lim\limits_{x\to\ 6}\frac{x+6}{x-25}=2 \cdot \frac{6}{6-25} = \frac{12}{-19}$$
\rozwStop
\odpStart
$\frac{12}{-19}$
\odpStop
\testStart
A.$\frac{12}{-19}$
B.$\infty$
C.$-\infty$
D.$0$
E.$\frac{12}{19}$
F.$\frac{6}{25}$
G.$-\frac{12}{19}$
H.$1$
I.$6$
\testStop
\kluczStart
A
\kluczStop



\zadStart{Przykład z Wikieł P 4.2b moja wersja nr 53}
Obliczyć granicę $\lim\limits_{x\to\ 6}\frac{x^{2}-6^{2}}{(x-6)(x-29)}$.
\zadStop
\rozwStart{Patryk Wirkus}{Martyna Czarnobaj}
$$\frac{x^{2}-6^{2}}{(x-6)(x-29)}=\frac{x+6}{x-29}$$

$$\lim\limits_{x\to\ 6}\frac{x^{2}-6^{2}}{(x-6)(x-29)}=[\frac{0}{0}]=\lim\limits_{x\to\ 6}\frac{x+6}{x-29}=2 \cdot \frac{6}{6-29} = \frac{12}{-23}$$
\rozwStop
\odpStart
$\frac{12}{-23}$
\odpStop
\testStart
A.$\frac{12}{-23}$
B.$\infty$
C.$-\infty$
D.$0$
E.$\frac{12}{23}$
F.$\frac{6}{29}$
G.$-\frac{12}{23}$
H.$1$
I.$6$
\testStop
\kluczStart
A
\kluczStop



\zadStart{Przykład z Wikieł P 4.2b moja wersja nr 54}
Obliczyć granicę $\lim\limits_{x\to\ 6}\frac{x^{2}-6^{2}}{(x-6)(x-31)}$.
\zadStop
\rozwStart{Patryk Wirkus}{Martyna Czarnobaj}
$$\frac{x^{2}-6^{2}}{(x-6)(x-31)}=\frac{x+6}{x-31}$$

$$\lim\limits_{x\to\ 6}\frac{x^{2}-6^{2}}{(x-6)(x-31)}=[\frac{0}{0}]=\lim\limits_{x\to\ 6}\frac{x+6}{x-31}=2 \cdot \frac{6}{6-31} = \frac{12}{-25}$$
\rozwStop
\odpStart
$\frac{12}{-25}$
\odpStop
\testStart
A.$\frac{12}{-25}$
B.$\infty$
C.$-\infty$
D.$0$
E.$\frac{12}{25}$
F.$\frac{6}{31}$
G.$-\frac{12}{25}$
H.$1$
I.$6$
\testStop
\kluczStart
A
\kluczStop



\zadStart{Przykład z Wikieł P 4.2b moja wersja nr 55}
Obliczyć granicę $\lim\limits_{x\to\ 6}\frac{x^{2}-6^{2}}{(x-6)(x-35)}$.
\zadStop
\rozwStart{Patryk Wirkus}{Martyna Czarnobaj}
$$\frac{x^{2}-6^{2}}{(x-6)(x-35)}=\frac{x+6}{x-35}$$

$$\lim\limits_{x\to\ 6}\frac{x^{2}-6^{2}}{(x-6)(x-35)}=[\frac{0}{0}]=\lim\limits_{x\to\ 6}\frac{x+6}{x-35}=2 \cdot \frac{6}{6-35} = \frac{12}{-29}$$
\rozwStop
\odpStart
$\frac{12}{-29}$
\odpStop
\testStart
A.$\frac{12}{-29}$
B.$\infty$
C.$-\infty$
D.$0$
E.$\frac{12}{29}$
F.$\frac{6}{35}$
G.$-\frac{12}{29}$
H.$1$
I.$6$
\testStop
\kluczStart
A
\kluczStop



\zadStart{Przykład z Wikieł P 4.2b moja wersja nr 56}
Obliczyć granicę $\lim\limits_{x\to\ 6}\frac{x^{2}-6^{2}}{(x-6)(x-37)}$.
\zadStop
\rozwStart{Patryk Wirkus}{Martyna Czarnobaj}
$$\frac{x^{2}-6^{2}}{(x-6)(x-37)}=\frac{x+6}{x-37}$$

$$\lim\limits_{x\to\ 6}\frac{x^{2}-6^{2}}{(x-6)(x-37)}=[\frac{0}{0}]=\lim\limits_{x\to\ 6}\frac{x+6}{x-37}=2 \cdot \frac{6}{6-37} = \frac{12}{-31}$$
\rozwStop
\odpStart
$\frac{12}{-31}$
\odpStop
\testStart
A.$\frac{12}{-31}$
B.$\infty$
C.$-\infty$
D.$0$
E.$\frac{12}{31}$
F.$\frac{6}{37}$
G.$-\frac{12}{31}$
H.$1$
I.$6$
\testStop
\kluczStart
A
\kluczStop



\zadStart{Przykład z Wikieł P 4.2b moja wersja nr 57}
Obliczyć granicę $\lim\limits_{x\to\ 7}\frac{x^{2}-7^{2}}{(x-7)(x-2)}$.
\zadStop
\rozwStart{Patryk Wirkus}{Martyna Czarnobaj}
$$\frac{x^{2}-7^{2}}{(x-7)(x-2)}=\frac{x+7}{x-2}$$

$$\lim\limits_{x\to\ 7}\frac{x^{2}-7^{2}}{(x-7)(x-2)}=[\frac{0}{0}]=\lim\limits_{x\to\ 7}\frac{x+7}{x-2}=2 \cdot \frac{7}{7-2} = \frac{14}{5}$$
\rozwStop
\odpStart
$\frac{14}{5}$
\odpStop
\testStart
A.$\frac{14}{5}$
B.$\infty$
C.$-\infty$
D.$0$
E.$\frac{14}{-5}$
F.$\frac{7}{2}$
G.$-\frac{14}{-5}$
H.$1$
I.$7$
\testStop
\kluczStart
A
\kluczStop



\zadStart{Przykład z Wikieł P 4.2b moja wersja nr 58}
Obliczyć granicę $\lim\limits_{x\to\ 7}\frac{x^{2}-7^{2}}{(x-7)(x-3)}$.
\zadStop
\rozwStart{Patryk Wirkus}{Martyna Czarnobaj}
$$\frac{x^{2}-7^{2}}{(x-7)(x-3)}=\frac{x+7}{x-3}$$

$$\lim\limits_{x\to\ 7}\frac{x^{2}-7^{2}}{(x-7)(x-3)}=[\frac{0}{0}]=\lim\limits_{x\to\ 7}\frac{x+7}{x-3}=2 \cdot \frac{7}{7-3} = \frac{14}{4}$$
\rozwStop
\odpStart
$\frac{14}{4}$
\odpStop
\testStart
A.$\frac{14}{4}$
B.$\infty$
C.$-\infty$
D.$0$
E.$\frac{14}{-4}$
F.$\frac{7}{3}$
G.$-\frac{14}{-4}$
H.$1$
I.$7$
\testStop
\kluczStart
A
\kluczStop



\zadStart{Przykład z Wikieł P 4.2b moja wersja nr 59}
Obliczyć granicę $\lim\limits_{x\to\ 7}\frac{x^{2}-7^{2}}{(x-7)(x-4)}$.
\zadStop
\rozwStart{Patryk Wirkus}{Martyna Czarnobaj}
$$\frac{x^{2}-7^{2}}{(x-7)(x-4)}=\frac{x+7}{x-4}$$

$$\lim\limits_{x\to\ 7}\frac{x^{2}-7^{2}}{(x-7)(x-4)}=[\frac{0}{0}]=\lim\limits_{x\to\ 7}\frac{x+7}{x-4}=2 \cdot \frac{7}{7-4} = \frac{14}{3}$$
\rozwStop
\odpStart
$\frac{14}{3}$
\odpStop
\testStart
A.$\frac{14}{3}$
B.$\infty$
C.$-\infty$
D.$0$
E.$\frac{14}{-3}$
F.$\frac{7}{4}$
G.$-\frac{14}{-3}$
H.$1$
I.$7$
\testStop
\kluczStart
A
\kluczStop



\zadStart{Przykład z Wikieł P 4.2b moja wersja nr 60}
Obliczyć granicę $\lim\limits_{x\to\ 7}\frac{x^{2}-7^{2}}{(x-7)(x-5)}$.
\zadStop
\rozwStart{Patryk Wirkus}{Martyna Czarnobaj}
$$\frac{x^{2}-7^{2}}{(x-7)(x-5)}=\frac{x+7}{x-5}$$

$$\lim\limits_{x\to\ 7}\frac{x^{2}-7^{2}}{(x-7)(x-5)}=[\frac{0}{0}]=\lim\limits_{x\to\ 7}\frac{x+7}{x-5}=2 \cdot \frac{7}{7-5} = \frac{14}{2}$$
\rozwStop
\odpStart
$\frac{14}{2}$
\odpStop
\testStart
A.$\frac{14}{2}$
B.$\infty$
C.$-\infty$
D.$0$
E.$\frac{14}{-2}$
F.$\frac{7}{5}$
G.$-\frac{14}{-2}$
H.$1$
I.$7$
\testStop
\kluczStart
A
\kluczStop



\zadStart{Przykład z Wikieł P 4.2b moja wersja nr 61}
Obliczyć granicę $\lim\limits_{x\to\ 7}\frac{x^{2}-7^{2}}{(x-7)(x-9)}$.
\zadStop
\rozwStart{Patryk Wirkus}{Martyna Czarnobaj}
$$\frac{x^{2}-7^{2}}{(x-7)(x-9)}=\frac{x+7}{x-9}$$

$$\lim\limits_{x\to\ 7}\frac{x^{2}-7^{2}}{(x-7)(x-9)}=[\frac{0}{0}]=\lim\limits_{x\to\ 7}\frac{x+7}{x-9}=2 \cdot \frac{7}{7-9} = \frac{14}{-2}$$
\rozwStop
\odpStart
$\frac{14}{-2}$
\odpStop
\testStart
A.$\frac{14}{-2}$
B.$\infty$
C.$-\infty$
D.$0$
E.$\frac{14}{2}$
F.$\frac{7}{9}$
G.$-\frac{14}{2}$
H.$1$
I.$7$
\testStop
\kluczStart
A
\kluczStop



\zadStart{Przykład z Wikieł P 4.2b moja wersja nr 62}
Obliczyć granicę $\lim\limits_{x\to\ 7}\frac{x^{2}-7^{2}}{(x-7)(x-10)}$.
\zadStop
\rozwStart{Patryk Wirkus}{Martyna Czarnobaj}
$$\frac{x^{2}-7^{2}}{(x-7)(x-10)}=\frac{x+7}{x-10}$$

$$\lim\limits_{x\to\ 7}\frac{x^{2}-7^{2}}{(x-7)(x-10)}=[\frac{0}{0}]=\lim\limits_{x\to\ 7}\frac{x+7}{x-10}=2 \cdot \frac{7}{7-10} = \frac{14}{-3}$$
\rozwStop
\odpStart
$\frac{14}{-3}$
\odpStop
\testStart
A.$\frac{14}{-3}$
B.$\infty$
C.$-\infty$
D.$0$
E.$\frac{14}{3}$
F.$\frac{7}{10}$
G.$-\frac{14}{3}$
H.$1$
I.$7$
\testStop
\kluczStart
A
\kluczStop



\zadStart{Przykład z Wikieł P 4.2b moja wersja nr 63}
Obliczyć granicę $\lim\limits_{x\to\ 7}\frac{x^{2}-7^{2}}{(x-7)(x-11)}$.
\zadStop
\rozwStart{Patryk Wirkus}{Martyna Czarnobaj}
$$\frac{x^{2}-7^{2}}{(x-7)(x-11)}=\frac{x+7}{x-11}$$

$$\lim\limits_{x\to\ 7}\frac{x^{2}-7^{2}}{(x-7)(x-11)}=[\frac{0}{0}]=\lim\limits_{x\to\ 7}\frac{x+7}{x-11}=2 \cdot \frac{7}{7-11} = \frac{14}{-4}$$
\rozwStop
\odpStart
$\frac{14}{-4}$
\odpStop
\testStart
A.$\frac{14}{-4}$
B.$\infty$
C.$-\infty$
D.$0$
E.$\frac{14}{4}$
F.$\frac{7}{11}$
G.$-\frac{14}{4}$
H.$1$
I.$7$
\testStop
\kluczStart
A
\kluczStop



\zadStart{Przykład z Wikieł P 4.2b moja wersja nr 64}
Obliczyć granicę $\lim\limits_{x\to\ 7}\frac{x^{2}-7^{2}}{(x-7)(x-12)}$.
\zadStop
\rozwStart{Patryk Wirkus}{Martyna Czarnobaj}
$$\frac{x^{2}-7^{2}}{(x-7)(x-12)}=\frac{x+7}{x-12}$$

$$\lim\limits_{x\to\ 7}\frac{x^{2}-7^{2}}{(x-7)(x-12)}=[\frac{0}{0}]=\lim\limits_{x\to\ 7}\frac{x+7}{x-12}=2 \cdot \frac{7}{7-12} = \frac{14}{-5}$$
\rozwStop
\odpStart
$\frac{14}{-5}$
\odpStop
\testStart
A.$\frac{14}{-5}$
B.$\infty$
C.$-\infty$
D.$0$
E.$\frac{14}{5}$
F.$\frac{7}{12}$
G.$-\frac{14}{5}$
H.$1$
I.$7$
\testStop
\kluczStart
A
\kluczStop



\zadStart{Przykład z Wikieł P 4.2b moja wersja nr 65}
Obliczyć granicę $\lim\limits_{x\to\ 7}\frac{x^{2}-7^{2}}{(x-7)(x-13)}$.
\zadStop
\rozwStart{Patryk Wirkus}{Martyna Czarnobaj}
$$\frac{x^{2}-7^{2}}{(x-7)(x-13)}=\frac{x+7}{x-13}$$

$$\lim\limits_{x\to\ 7}\frac{x^{2}-7^{2}}{(x-7)(x-13)}=[\frac{0}{0}]=\lim\limits_{x\to\ 7}\frac{x+7}{x-13}=2 \cdot \frac{7}{7-13} = \frac{14}{-6}$$
\rozwStop
\odpStart
$\frac{14}{-6}$
\odpStop
\testStart
A.$\frac{14}{-6}$
B.$\infty$
C.$-\infty$
D.$0$
E.$\frac{14}{6}$
F.$\frac{7}{13}$
G.$-\frac{14}{6}$
H.$1$
I.$7$
\testStop
\kluczStart
A
\kluczStop



\zadStart{Przykład z Wikieł P 4.2b moja wersja nr 66}
Obliczyć granicę $\lim\limits_{x\to\ 7}\frac{x^{2}-7^{2}}{(x-7)(x-15)}$.
\zadStop
\rozwStart{Patryk Wirkus}{Martyna Czarnobaj}
$$\frac{x^{2}-7^{2}}{(x-7)(x-15)}=\frac{x+7}{x-15}$$

$$\lim\limits_{x\to\ 7}\frac{x^{2}-7^{2}}{(x-7)(x-15)}=[\frac{0}{0}]=\lim\limits_{x\to\ 7}\frac{x+7}{x-15}=2 \cdot \frac{7}{7-15} = \frac{14}{-8}$$
\rozwStop
\odpStart
$\frac{14}{-8}$
\odpStop
\testStart
A.$\frac{14}{-8}$
B.$\infty$
C.$-\infty$
D.$0$
E.$\frac{14}{8}$
F.$\frac{7}{15}$
G.$-\frac{14}{8}$
H.$1$
I.$7$
\testStop
\kluczStart
A
\kluczStop



\zadStart{Przykład z Wikieł P 4.2b moja wersja nr 67}
Obliczyć granicę $\lim\limits_{x\to\ 7}\frac{x^{2}-7^{2}}{(x-7)(x-16)}$.
\zadStop
\rozwStart{Patryk Wirkus}{Martyna Czarnobaj}
$$\frac{x^{2}-7^{2}}{(x-7)(x-16)}=\frac{x+7}{x-16}$$

$$\lim\limits_{x\to\ 7}\frac{x^{2}-7^{2}}{(x-7)(x-16)}=[\frac{0}{0}]=\lim\limits_{x\to\ 7}\frac{x+7}{x-16}=2 \cdot \frac{7}{7-16} = \frac{14}{-9}$$
\rozwStop
\odpStart
$\frac{14}{-9}$
\odpStop
\testStart
A.$\frac{14}{-9}$
B.$\infty$
C.$-\infty$
D.$0$
E.$\frac{14}{9}$
F.$\frac{7}{16}$
G.$-\frac{14}{9}$
H.$1$
I.$7$
\testStop
\kluczStart
A
\kluczStop



\zadStart{Przykład z Wikieł P 4.2b moja wersja nr 68}
Obliczyć granicę $\lim\limits_{x\to\ 7}\frac{x^{2}-7^{2}}{(x-7)(x-17)}$.
\zadStop
\rozwStart{Patryk Wirkus}{Martyna Czarnobaj}
$$\frac{x^{2}-7^{2}}{(x-7)(x-17)}=\frac{x+7}{x-17}$$

$$\lim\limits_{x\to\ 7}\frac{x^{2}-7^{2}}{(x-7)(x-17)}=[\frac{0}{0}]=\lim\limits_{x\to\ 7}\frac{x+7}{x-17}=2 \cdot \frac{7}{7-17} = \frac{14}{-10}$$
\rozwStop
\odpStart
$\frac{14}{-10}$
\odpStop
\testStart
A.$\frac{14}{-10}$
B.$\infty$
C.$-\infty$
D.$0$
E.$\frac{14}{10}$
F.$\frac{7}{17}$
G.$-\frac{14}{10}$
H.$1$
I.$7$
\testStop
\kluczStart
A
\kluczStop



\zadStart{Przykład z Wikieł P 4.2b moja wersja nr 69}
Obliczyć granicę $\lim\limits_{x\to\ 7}\frac{x^{2}-7^{2}}{(x-7)(x-18)}$.
\zadStop
\rozwStart{Patryk Wirkus}{Martyna Czarnobaj}
$$\frac{x^{2}-7^{2}}{(x-7)(x-18)}=\frac{x+7}{x-18}$$

$$\lim\limits_{x\to\ 7}\frac{x^{2}-7^{2}}{(x-7)(x-18)}=[\frac{0}{0}]=\lim\limits_{x\to\ 7}\frac{x+7}{x-18}=2 \cdot \frac{7}{7-18} = \frac{14}{-11}$$
\rozwStop
\odpStart
$\frac{14}{-11}$
\odpStop
\testStart
A.$\frac{14}{-11}$
B.$\infty$
C.$-\infty$
D.$0$
E.$\frac{14}{11}$
F.$\frac{7}{18}$
G.$-\frac{14}{11}$
H.$1$
I.$7$
\testStop
\kluczStart
A
\kluczStop



\zadStart{Przykład z Wikieł P 4.2b moja wersja nr 70}
Obliczyć granicę $\lim\limits_{x\to\ 7}\frac{x^{2}-7^{2}}{(x-7)(x-19)}$.
\zadStop
\rozwStart{Patryk Wirkus}{Martyna Czarnobaj}
$$\frac{x^{2}-7^{2}}{(x-7)(x-19)}=\frac{x+7}{x-19}$$

$$\lim\limits_{x\to\ 7}\frac{x^{2}-7^{2}}{(x-7)(x-19)}=[\frac{0}{0}]=\lim\limits_{x\to\ 7}\frac{x+7}{x-19}=2 \cdot \frac{7}{7-19} = \frac{14}{-12}$$
\rozwStop
\odpStart
$\frac{14}{-12}$
\odpStop
\testStart
A.$\frac{14}{-12}$
B.$\infty$
C.$-\infty$
D.$0$
E.$\frac{14}{12}$
F.$\frac{7}{19}$
G.$-\frac{14}{12}$
H.$1$
I.$7$
\testStop
\kluczStart
A
\kluczStop



\zadStart{Przykład z Wikieł P 4.2b moja wersja nr 71}
Obliczyć granicę $\lim\limits_{x\to\ 7}\frac{x^{2}-7^{2}}{(x-7)(x-20)}$.
\zadStop
\rozwStart{Patryk Wirkus}{Martyna Czarnobaj}
$$\frac{x^{2}-7^{2}}{(x-7)(x-20)}=\frac{x+7}{x-20}$$

$$\lim\limits_{x\to\ 7}\frac{x^{2}-7^{2}}{(x-7)(x-20)}=[\frac{0}{0}]=\lim\limits_{x\to\ 7}\frac{x+7}{x-20}=2 \cdot \frac{7}{7-20} = \frac{14}{-13}$$
\rozwStop
\odpStart
$\frac{14}{-13}$
\odpStop
\testStart
A.$\frac{14}{-13}$
B.$\infty$
C.$-\infty$
D.$0$
E.$\frac{14}{13}$
F.$\frac{7}{20}$
G.$-\frac{14}{13}$
H.$1$
I.$7$
\testStop
\kluczStart
A
\kluczStop



\zadStart{Przykład z Wikieł P 4.2b moja wersja nr 72}
Obliczyć granicę $\lim\limits_{x\to\ 7}\frac{x^{2}-7^{2}}{(x-7)(x-22)}$.
\zadStop
\rozwStart{Patryk Wirkus}{Martyna Czarnobaj}
$$\frac{x^{2}-7^{2}}{(x-7)(x-22)}=\frac{x+7}{x-22}$$

$$\lim\limits_{x\to\ 7}\frac{x^{2}-7^{2}}{(x-7)(x-22)}=[\frac{0}{0}]=\lim\limits_{x\to\ 7}\frac{x+7}{x-22}=2 \cdot \frac{7}{7-22} = \frac{14}{-15}$$
\rozwStop
\odpStart
$\frac{14}{-15}$
\odpStop
\testStart
A.$\frac{14}{-15}$
B.$\infty$
C.$-\infty$
D.$0$
E.$\frac{14}{15}$
F.$\frac{7}{22}$
G.$-\frac{14}{15}$
H.$1$
I.$7$
\testStop
\kluczStart
A
\kluczStop



\zadStart{Przykład z Wikieł P 4.2b moja wersja nr 73}
Obliczyć granicę $\lim\limits_{x\to\ 7}\frac{x^{2}-7^{2}}{(x-7)(x-23)}$.
\zadStop
\rozwStart{Patryk Wirkus}{Martyna Czarnobaj}
$$\frac{x^{2}-7^{2}}{(x-7)(x-23)}=\frac{x+7}{x-23}$$

$$\lim\limits_{x\to\ 7}\frac{x^{2}-7^{2}}{(x-7)(x-23)}=[\frac{0}{0}]=\lim\limits_{x\to\ 7}\frac{x+7}{x-23}=2 \cdot \frac{7}{7-23} = \frac{14}{-16}$$
\rozwStop
\odpStart
$\frac{14}{-16}$
\odpStop
\testStart
A.$\frac{14}{-16}$
B.$\infty$
C.$-\infty$
D.$0$
E.$\frac{14}{16}$
F.$\frac{7}{23}$
G.$-\frac{14}{16}$
H.$1$
I.$7$
\testStop
\kluczStart
A
\kluczStop



\zadStart{Przykład z Wikieł P 4.2b moja wersja nr 74}
Obliczyć granicę $\lim\limits_{x\to\ 7}\frac{x^{2}-7^{2}}{(x-7)(x-24)}$.
\zadStop
\rozwStart{Patryk Wirkus}{Martyna Czarnobaj}
$$\frac{x^{2}-7^{2}}{(x-7)(x-24)}=\frac{x+7}{x-24}$$

$$\lim\limits_{x\to\ 7}\frac{x^{2}-7^{2}}{(x-7)(x-24)}=[\frac{0}{0}]=\lim\limits_{x\to\ 7}\frac{x+7}{x-24}=2 \cdot \frac{7}{7-24} = \frac{14}{-17}$$
\rozwStop
\odpStart
$\frac{14}{-17}$
\odpStop
\testStart
A.$\frac{14}{-17}$
B.$\infty$
C.$-\infty$
D.$0$
E.$\frac{14}{17}$
F.$\frac{7}{24}$
G.$-\frac{14}{17}$
H.$1$
I.$7$
\testStop
\kluczStart
A
\kluczStop



\zadStart{Przykład z Wikieł P 4.2b moja wersja nr 75}
Obliczyć granicę $\lim\limits_{x\to\ 7}\frac{x^{2}-7^{2}}{(x-7)(x-25)}$.
\zadStop
\rozwStart{Patryk Wirkus}{Martyna Czarnobaj}
$$\frac{x^{2}-7^{2}}{(x-7)(x-25)}=\frac{x+7}{x-25}$$

$$\lim\limits_{x\to\ 7}\frac{x^{2}-7^{2}}{(x-7)(x-25)}=[\frac{0}{0}]=\lim\limits_{x\to\ 7}\frac{x+7}{x-25}=2 \cdot \frac{7}{7-25} = \frac{14}{-18}$$
\rozwStop
\odpStart
$\frac{14}{-18}$
\odpStop
\testStart
A.$\frac{14}{-18}$
B.$\infty$
C.$-\infty$
D.$0$
E.$\frac{14}{18}$
F.$\frac{7}{25}$
G.$-\frac{14}{18}$
H.$1$
I.$7$
\testStop
\kluczStart
A
\kluczStop



\zadStart{Przykład z Wikieł P 4.2b moja wersja nr 76}
Obliczyć granicę $\lim\limits_{x\to\ 7}\frac{x^{2}-7^{2}}{(x-7)(x-26)}$.
\zadStop
\rozwStart{Patryk Wirkus}{Martyna Czarnobaj}
$$\frac{x^{2}-7^{2}}{(x-7)(x-26)}=\frac{x+7}{x-26}$$

$$\lim\limits_{x\to\ 7}\frac{x^{2}-7^{2}}{(x-7)(x-26)}=[\frac{0}{0}]=\lim\limits_{x\to\ 7}\frac{x+7}{x-26}=2 \cdot \frac{7}{7-26} = \frac{14}{-19}$$
\rozwStop
\odpStart
$\frac{14}{-19}$
\odpStop
\testStart
A.$\frac{14}{-19}$
B.$\infty$
C.$-\infty$
D.$0$
E.$\frac{14}{19}$
F.$\frac{7}{26}$
G.$-\frac{14}{19}$
H.$1$
I.$7$
\testStop
\kluczStart
A
\kluczStop



\zadStart{Przykład z Wikieł P 4.2b moja wersja nr 77}
Obliczyć granicę $\lim\limits_{x\to\ 7}\frac{x^{2}-7^{2}}{(x-7)(x-27)}$.
\zadStop
\rozwStart{Patryk Wirkus}{Martyna Czarnobaj}
$$\frac{x^{2}-7^{2}}{(x-7)(x-27)}=\frac{x+7}{x-27}$$

$$\lim\limits_{x\to\ 7}\frac{x^{2}-7^{2}}{(x-7)(x-27)}=[\frac{0}{0}]=\lim\limits_{x\to\ 7}\frac{x+7}{x-27}=2 \cdot \frac{7}{7-27} = \frac{14}{-20}$$
\rozwStop
\odpStart
$\frac{14}{-20}$
\odpStop
\testStart
A.$\frac{14}{-20}$
B.$\infty$
C.$-\infty$
D.$0$
E.$\frac{14}{20}$
F.$\frac{7}{27}$
G.$-\frac{14}{20}$
H.$1$
I.$7$
\testStop
\kluczStart
A
\kluczStop



\zadStart{Przykład z Wikieł P 4.2b moja wersja nr 78}
Obliczyć granicę $\lim\limits_{x\to\ 7}\frac{x^{2}-7^{2}}{(x-7)(x-29)}$.
\zadStop
\rozwStart{Patryk Wirkus}{Martyna Czarnobaj}
$$\frac{x^{2}-7^{2}}{(x-7)(x-29)}=\frac{x+7}{x-29}$$

$$\lim\limits_{x\to\ 7}\frac{x^{2}-7^{2}}{(x-7)(x-29)}=[\frac{0}{0}]=\lim\limits_{x\to\ 7}\frac{x+7}{x-29}=2 \cdot \frac{7}{7-29} = \frac{14}{-22}$$
\rozwStop
\odpStart
$\frac{14}{-22}$
\odpStop
\testStart
A.$\frac{14}{-22}$
B.$\infty$
C.$-\infty$
D.$0$
E.$\frac{14}{22}$
F.$\frac{7}{29}$
G.$-\frac{14}{22}$
H.$1$
I.$7$
\testStop
\kluczStart
A
\kluczStop



\zadStart{Przykład z Wikieł P 4.2b moja wersja nr 79}
Obliczyć granicę $\lim\limits_{x\to\ 7}\frac{x^{2}-7^{2}}{(x-7)(x-30)}$.
\zadStop
\rozwStart{Patryk Wirkus}{Martyna Czarnobaj}
$$\frac{x^{2}-7^{2}}{(x-7)(x-30)}=\frac{x+7}{x-30}$$

$$\lim\limits_{x\to\ 7}\frac{x^{2}-7^{2}}{(x-7)(x-30)}=[\frac{0}{0}]=\lim\limits_{x\to\ 7}\frac{x+7}{x-30}=2 \cdot \frac{7}{7-30} = \frac{14}{-23}$$
\rozwStop
\odpStart
$\frac{14}{-23}$
\odpStop
\testStart
A.$\frac{14}{-23}$
B.$\infty$
C.$-\infty$
D.$0$
E.$\frac{14}{23}$
F.$\frac{7}{30}$
G.$-\frac{14}{23}$
H.$1$
I.$7$
\testStop
\kluczStart
A
\kluczStop



\zadStart{Przykład z Wikieł P 4.2b moja wersja nr 80}
Obliczyć granicę $\lim\limits_{x\to\ 7}\frac{x^{2}-7^{2}}{(x-7)(x-31)}$.
\zadStop
\rozwStart{Patryk Wirkus}{Martyna Czarnobaj}
$$\frac{x^{2}-7^{2}}{(x-7)(x-31)}=\frac{x+7}{x-31}$$

$$\lim\limits_{x\to\ 7}\frac{x^{2}-7^{2}}{(x-7)(x-31)}=[\frac{0}{0}]=\lim\limits_{x\to\ 7}\frac{x+7}{x-31}=2 \cdot \frac{7}{7-31} = \frac{14}{-24}$$
\rozwStop
\odpStart
$\frac{14}{-24}$
\odpStop
\testStart
A.$\frac{14}{-24}$
B.$\infty$
C.$-\infty$
D.$0$
E.$\frac{14}{24}$
F.$\frac{7}{31}$
G.$-\frac{14}{24}$
H.$1$
I.$7$
\testStop
\kluczStart
A
\kluczStop



\zadStart{Przykład z Wikieł P 4.2b moja wersja nr 81}
Obliczyć granicę $\lim\limits_{x\to\ 7}\frac{x^{2}-7^{2}}{(x-7)(x-32)}$.
\zadStop
\rozwStart{Patryk Wirkus}{Martyna Czarnobaj}
$$\frac{x^{2}-7^{2}}{(x-7)(x-32)}=\frac{x+7}{x-32}$$

$$\lim\limits_{x\to\ 7}\frac{x^{2}-7^{2}}{(x-7)(x-32)}=[\frac{0}{0}]=\lim\limits_{x\to\ 7}\frac{x+7}{x-32}=2 \cdot \frac{7}{7-32} = \frac{14}{-25}$$
\rozwStop
\odpStart
$\frac{14}{-25}$
\odpStop
\testStart
A.$\frac{14}{-25}$
B.$\infty$
C.$-\infty$
D.$0$
E.$\frac{14}{25}$
F.$\frac{7}{32}$
G.$-\frac{14}{25}$
H.$1$
I.$7$
\testStop
\kluczStart
A
\kluczStop



\zadStart{Przykład z Wikieł P 4.2b moja wersja nr 82}
Obliczyć granicę $\lim\limits_{x\to\ 7}\frac{x^{2}-7^{2}}{(x-7)(x-33)}$.
\zadStop
\rozwStart{Patryk Wirkus}{Martyna Czarnobaj}
$$\frac{x^{2}-7^{2}}{(x-7)(x-33)}=\frac{x+7}{x-33}$$

$$\lim\limits_{x\to\ 7}\frac{x^{2}-7^{2}}{(x-7)(x-33)}=[\frac{0}{0}]=\lim\limits_{x\to\ 7}\frac{x+7}{x-33}=2 \cdot \frac{7}{7-33} = \frac{14}{-26}$$
\rozwStop
\odpStart
$\frac{14}{-26}$
\odpStop
\testStart
A.$\frac{14}{-26}$
B.$\infty$
C.$-\infty$
D.$0$
E.$\frac{14}{26}$
F.$\frac{7}{33}$
G.$-\frac{14}{26}$
H.$1$
I.$7$
\testStop
\kluczStart
A
\kluczStop



\zadStart{Przykład z Wikieł P 4.2b moja wersja nr 83}
Obliczyć granicę $\lim\limits_{x\to\ 7}\frac{x^{2}-7^{2}}{(x-7)(x-34)}$.
\zadStop
\rozwStart{Patryk Wirkus}{Martyna Czarnobaj}
$$\frac{x^{2}-7^{2}}{(x-7)(x-34)}=\frac{x+7}{x-34}$$

$$\lim\limits_{x\to\ 7}\frac{x^{2}-7^{2}}{(x-7)(x-34)}=[\frac{0}{0}]=\lim\limits_{x\to\ 7}\frac{x+7}{x-34}=2 \cdot \frac{7}{7-34} = \frac{14}{-27}$$
\rozwStop
\odpStart
$\frac{14}{-27}$
\odpStop
\testStart
A.$\frac{14}{-27}$
B.$\infty$
C.$-\infty$
D.$0$
E.$\frac{14}{27}$
F.$\frac{7}{34}$
G.$-\frac{14}{27}$
H.$1$
I.$7$
\testStop
\kluczStart
A
\kluczStop



\zadStart{Przykład z Wikieł P 4.2b moja wersja nr 84}
Obliczyć granicę $\lim\limits_{x\to\ 7}\frac{x^{2}-7^{2}}{(x-7)(x-36)}$.
\zadStop
\rozwStart{Patryk Wirkus}{Martyna Czarnobaj}
$$\frac{x^{2}-7^{2}}{(x-7)(x-36)}=\frac{x+7}{x-36}$$

$$\lim\limits_{x\to\ 7}\frac{x^{2}-7^{2}}{(x-7)(x-36)}=[\frac{0}{0}]=\lim\limits_{x\to\ 7}\frac{x+7}{x-36}=2 \cdot \frac{7}{7-36} = \frac{14}{-29}$$
\rozwStop
\odpStart
$\frac{14}{-29}$
\odpStop
\testStart
A.$\frac{14}{-29}$
B.$\infty$
C.$-\infty$
D.$0$
E.$\frac{14}{29}$
F.$\frac{7}{36}$
G.$-\frac{14}{29}$
H.$1$
I.$7$
\testStop
\kluczStart
A
\kluczStop



\zadStart{Przykład z Wikieł P 4.2b moja wersja nr 85}
Obliczyć granicę $\lim\limits_{x\to\ 7}\frac{x^{2}-7^{2}}{(x-7)(x-37)}$.
\zadStop
\rozwStart{Patryk Wirkus}{Martyna Czarnobaj}
$$\frac{x^{2}-7^{2}}{(x-7)(x-37)}=\frac{x+7}{x-37}$$

$$\lim\limits_{x\to\ 7}\frac{x^{2}-7^{2}}{(x-7)(x-37)}=[\frac{0}{0}]=\lim\limits_{x\to\ 7}\frac{x+7}{x-37}=2 \cdot \frac{7}{7-37} = \frac{14}{-30}$$
\rozwStop
\odpStart
$\frac{14}{-30}$
\odpStop
\testStart
A.$\frac{14}{-30}$
B.$\infty$
C.$-\infty$
D.$0$
E.$\frac{14}{30}$
F.$\frac{7}{37}$
G.$-\frac{14}{30}$
H.$1$
I.$7$
\testStop
\kluczStart
A
\kluczStop



\zadStart{Przykład z Wikieł P 4.2b moja wersja nr 86}
Obliczyć granicę $\lim\limits_{x\to\ 7}\frac{x^{2}-7^{2}}{(x-7)(x-38)}$.
\zadStop
\rozwStart{Patryk Wirkus}{Martyna Czarnobaj}
$$\frac{x^{2}-7^{2}}{(x-7)(x-38)}=\frac{x+7}{x-38}$$

$$\lim\limits_{x\to\ 7}\frac{x^{2}-7^{2}}{(x-7)(x-38)}=[\frac{0}{0}]=\lim\limits_{x\to\ 7}\frac{x+7}{x-38}=2 \cdot \frac{7}{7-38} = \frac{14}{-31}$$
\rozwStop
\odpStart
$\frac{14}{-31}$
\odpStop
\testStart
A.$\frac{14}{-31}$
B.$\infty$
C.$-\infty$
D.$0$
E.$\frac{14}{31}$
F.$\frac{7}{38}$
G.$-\frac{14}{31}$
H.$1$
I.$7$
\testStop
\kluczStart
A
\kluczStop



\zadStart{Przykład z Wikieł P 4.2b moja wersja nr 87}
Obliczyć granicę $\lim\limits_{x\to\ 7}\frac{x^{2}-7^{2}}{(x-7)(x-39)}$.
\zadStop
\rozwStart{Patryk Wirkus}{Martyna Czarnobaj}
$$\frac{x^{2}-7^{2}}{(x-7)(x-39)}=\frac{x+7}{x-39}$$

$$\lim\limits_{x\to\ 7}\frac{x^{2}-7^{2}}{(x-7)(x-39)}=[\frac{0}{0}]=\lim\limits_{x\to\ 7}\frac{x+7}{x-39}=2 \cdot \frac{7}{7-39} = \frac{14}{-32}$$
\rozwStop
\odpStart
$\frac{14}{-32}$
\odpStop
\testStart
A.$\frac{14}{-32}$
B.$\infty$
C.$-\infty$
D.$0$
E.$\frac{14}{32}$
F.$\frac{7}{39}$
G.$-\frac{14}{32}$
H.$1$
I.$7$
\testStop
\kluczStart
A
\kluczStop



\zadStart{Przykład z Wikieł P 4.2b moja wersja nr 88}
Obliczyć granicę $\lim\limits_{x\to\ 7}\frac{x^{2}-7^{2}}{(x-7)(x-40)}$.
\zadStop
\rozwStart{Patryk Wirkus}{Martyna Czarnobaj}
$$\frac{x^{2}-7^{2}}{(x-7)(x-40)}=\frac{x+7}{x-40}$$

$$\lim\limits_{x\to\ 7}\frac{x^{2}-7^{2}}{(x-7)(x-40)}=[\frac{0}{0}]=\lim\limits_{x\to\ 7}\frac{x+7}{x-40}=2 \cdot \frac{7}{7-40} = \frac{14}{-33}$$
\rozwStop
\odpStart
$\frac{14}{-33}$
\odpStop
\testStart
A.$\frac{14}{-33}$
B.$\infty$
C.$-\infty$
D.$0$
E.$\frac{14}{33}$
F.$\frac{7}{40}$
G.$-\frac{14}{33}$
H.$1$
I.$7$
\testStop
\kluczStart
A
\kluczStop



\zadStart{Przykład z Wikieł P 4.2b moja wersja nr 89}
Obliczyć granicę $\lim\limits_{x\to\ 8}\frac{x^{2}-8^{2}}{(x-8)(x-3)}$.
\zadStop
\rozwStart{Patryk Wirkus}{Martyna Czarnobaj}
$$\frac{x^{2}-8^{2}}{(x-8)(x-3)}=\frac{x+8}{x-3}$$

$$\lim\limits_{x\to\ 8}\frac{x^{2}-8^{2}}{(x-8)(x-3)}=[\frac{0}{0}]=\lim\limits_{x\to\ 8}\frac{x+8}{x-3}=2 \cdot \frac{8}{8-3} = \frac{16}{5}$$
\rozwStop
\odpStart
$\frac{16}{5}$
\odpStop
\testStart
A.$\frac{16}{5}$
B.$\infty$
C.$-\infty$
D.$0$
E.$\frac{16}{-5}$
F.$\frac{8}{3}$
G.$-\frac{16}{-5}$
H.$1$
I.$8$
\testStop
\kluczStart
A
\kluczStop



\zadStart{Przykład z Wikieł P 4.2b moja wersja nr 90}
Obliczyć granicę $\lim\limits_{x\to\ 8}\frac{x^{2}-8^{2}}{(x-8)(x-5)}$.
\zadStop
\rozwStart{Patryk Wirkus}{Martyna Czarnobaj}
$$\frac{x^{2}-8^{2}}{(x-8)(x-5)}=\frac{x+8}{x-5}$$

$$\lim\limits_{x\to\ 8}\frac{x^{2}-8^{2}}{(x-8)(x-5)}=[\frac{0}{0}]=\lim\limits_{x\to\ 8}\frac{x+8}{x-5}=2 \cdot \frac{8}{8-5} = \frac{16}{3}$$
\rozwStop
\odpStart
$\frac{16}{3}$
\odpStop
\testStart
A.$\frac{16}{3}$
B.$\infty$
C.$-\infty$
D.$0$
E.$\frac{16}{-3}$
F.$\frac{8}{5}$
G.$-\frac{16}{-3}$
H.$1$
I.$8$
\testStop
\kluczStart
A
\kluczStop



\zadStart{Przykład z Wikieł P 4.2b moja wersja nr 91}
Obliczyć granicę $\lim\limits_{x\to\ 8}\frac{x^{2}-8^{2}}{(x-8)(x-11)}$.
\zadStop
\rozwStart{Patryk Wirkus}{Martyna Czarnobaj}
$$\frac{x^{2}-8^{2}}{(x-8)(x-11)}=\frac{x+8}{x-11}$$

$$\lim\limits_{x\to\ 8}\frac{x^{2}-8^{2}}{(x-8)(x-11)}=[\frac{0}{0}]=\lim\limits_{x\to\ 8}\frac{x+8}{x-11}=2 \cdot \frac{8}{8-11} = \frac{16}{-3}$$
\rozwStop
\odpStart
$\frac{16}{-3}$
\odpStop
\testStart
A.$\frac{16}{-3}$
B.$\infty$
C.$-\infty$
D.$0$
E.$\frac{16}{3}$
F.$\frac{8}{11}$
G.$-\frac{16}{3}$
H.$1$
I.$8$
\testStop
\kluczStart
A
\kluczStop



\zadStart{Przykład z Wikieł P 4.2b moja wersja nr 92}
Obliczyć granicę $\lim\limits_{x\to\ 8}\frac{x^{2}-8^{2}}{(x-8)(x-13)}$.
\zadStop
\rozwStart{Patryk Wirkus}{Martyna Czarnobaj}
$$\frac{x^{2}-8^{2}}{(x-8)(x-13)}=\frac{x+8}{x-13}$$

$$\lim\limits_{x\to\ 8}\frac{x^{2}-8^{2}}{(x-8)(x-13)}=[\frac{0}{0}]=\lim\limits_{x\to\ 8}\frac{x+8}{x-13}=2 \cdot \frac{8}{8-13} = \frac{16}{-5}$$
\rozwStop
\odpStart
$\frac{16}{-5}$
\odpStop
\testStart
A.$\frac{16}{-5}$
B.$\infty$
C.$-\infty$
D.$0$
E.$\frac{16}{5}$
F.$\frac{8}{13}$
G.$-\frac{16}{5}$
H.$1$
I.$8$
\testStop
\kluczStart
A
\kluczStop



\zadStart{Przykład z Wikieł P 4.2b moja wersja nr 93}
Obliczyć granicę $\lim\limits_{x\to\ 8}\frac{x^{2}-8^{2}}{(x-8)(x-15)}$.
\zadStop
\rozwStart{Patryk Wirkus}{Martyna Czarnobaj}
$$\frac{x^{2}-8^{2}}{(x-8)(x-15)}=\frac{x+8}{x-15}$$

$$\lim\limits_{x\to\ 8}\frac{x^{2}-8^{2}}{(x-8)(x-15)}=[\frac{0}{0}]=\lim\limits_{x\to\ 8}\frac{x+8}{x-15}=2 \cdot \frac{8}{8-15} = \frac{16}{-7}$$
\rozwStop
\odpStart
$\frac{16}{-7}$
\odpStop
\testStart
A.$\frac{16}{-7}$
B.$\infty$
C.$-\infty$
D.$0$
E.$\frac{16}{7}$
F.$\frac{8}{15}$
G.$-\frac{16}{7}$
H.$1$
I.$8$
\testStop
\kluczStart
A
\kluczStop



\zadStart{Przykład z Wikieł P 4.2b moja wersja nr 94}
Obliczyć granicę $\lim\limits_{x\to\ 8}\frac{x^{2}-8^{2}}{(x-8)(x-17)}$.
\zadStop
\rozwStart{Patryk Wirkus}{Martyna Czarnobaj}
$$\frac{x^{2}-8^{2}}{(x-8)(x-17)}=\frac{x+8}{x-17}$$

$$\lim\limits_{x\to\ 8}\frac{x^{2}-8^{2}}{(x-8)(x-17)}=[\frac{0}{0}]=\lim\limits_{x\to\ 8}\frac{x+8}{x-17}=2 \cdot \frac{8}{8-17} = \frac{16}{-9}$$
\rozwStop
\odpStart
$\frac{16}{-9}$
\odpStop
\testStart
A.$\frac{16}{-9}$
B.$\infty$
C.$-\infty$
D.$0$
E.$\frac{16}{9}$
F.$\frac{8}{17}$
G.$-\frac{16}{9}$
H.$1$
I.$8$
\testStop
\kluczStart
A
\kluczStop



\zadStart{Przykład z Wikieł P 4.2b moja wersja nr 95}
Obliczyć granicę $\lim\limits_{x\to\ 8}\frac{x^{2}-8^{2}}{(x-8)(x-19)}$.
\zadStop
\rozwStart{Patryk Wirkus}{Martyna Czarnobaj}
$$\frac{x^{2}-8^{2}}{(x-8)(x-19)}=\frac{x+8}{x-19}$$

$$\lim\limits_{x\to\ 8}\frac{x^{2}-8^{2}}{(x-8)(x-19)}=[\frac{0}{0}]=\lim\limits_{x\to\ 8}\frac{x+8}{x-19}=2 \cdot \frac{8}{8-19} = \frac{16}{-11}$$
\rozwStop
\odpStart
$\frac{16}{-11}$
\odpStop
\testStart
A.$\frac{16}{-11}$
B.$\infty$
C.$-\infty$
D.$0$
E.$\frac{16}{11}$
F.$\frac{8}{19}$
G.$-\frac{16}{11}$
H.$1$
I.$8$
\testStop
\kluczStart
A
\kluczStop



\zadStart{Przykład z Wikieł P 4.2b moja wersja nr 96}
Obliczyć granicę $\lim\limits_{x\to\ 8}\frac{x^{2}-8^{2}}{(x-8)(x-21)}$.
\zadStop
\rozwStart{Patryk Wirkus}{Martyna Czarnobaj}
$$\frac{x^{2}-8^{2}}{(x-8)(x-21)}=\frac{x+8}{x-21}$$

$$\lim\limits_{x\to\ 8}\frac{x^{2}-8^{2}}{(x-8)(x-21)}=[\frac{0}{0}]=\lim\limits_{x\to\ 8}\frac{x+8}{x-21}=2 \cdot \frac{8}{8-21} = \frac{16}{-13}$$
\rozwStop
\odpStart
$\frac{16}{-13}$
\odpStop
\testStart
A.$\frac{16}{-13}$
B.$\infty$
C.$-\infty$
D.$0$
E.$\frac{16}{13}$
F.$\frac{8}{21}$
G.$-\frac{16}{13}$
H.$1$
I.$8$
\testStop
\kluczStart
A
\kluczStop



\zadStart{Przykład z Wikieł P 4.2b moja wersja nr 97}
Obliczyć granicę $\lim\limits_{x\to\ 8}\frac{x^{2}-8^{2}}{(x-8)(x-23)}$.
\zadStop
\rozwStart{Patryk Wirkus}{Martyna Czarnobaj}
$$\frac{x^{2}-8^{2}}{(x-8)(x-23)}=\frac{x+8}{x-23}$$

$$\lim\limits_{x\to\ 8}\frac{x^{2}-8^{2}}{(x-8)(x-23)}=[\frac{0}{0}]=\lim\limits_{x\to\ 8}\frac{x+8}{x-23}=2 \cdot \frac{8}{8-23} = \frac{16}{-15}$$
\rozwStop
\odpStart
$\frac{16}{-15}$
\odpStop
\testStart
A.$\frac{16}{-15}$
B.$\infty$
C.$-\infty$
D.$0$
E.$\frac{16}{15}$
F.$\frac{8}{23}$
G.$-\frac{16}{15}$
H.$1$
I.$8$
\testStop
\kluczStart
A
\kluczStop



\zadStart{Przykład z Wikieł P 4.2b moja wersja nr 98}
Obliczyć granicę $\lim\limits_{x\to\ 8}\frac{x^{2}-8^{2}}{(x-8)(x-25)}$.
\zadStop
\rozwStart{Patryk Wirkus}{Martyna Czarnobaj}
$$\frac{x^{2}-8^{2}}{(x-8)(x-25)}=\frac{x+8}{x-25}$$

$$\lim\limits_{x\to\ 8}\frac{x^{2}-8^{2}}{(x-8)(x-25)}=[\frac{0}{0}]=\lim\limits_{x\to\ 8}\frac{x+8}{x-25}=2 \cdot \frac{8}{8-25} = \frac{16}{-17}$$
\rozwStop
\odpStart
$\frac{16}{-17}$
\odpStop
\testStart
A.$\frac{16}{-17}$
B.$\infty$
C.$-\infty$
D.$0$
E.$\frac{16}{17}$
F.$\frac{8}{25}$
G.$-\frac{16}{17}$
H.$1$
I.$8$
\testStop
\kluczStart
A
\kluczStop



\zadStart{Przykład z Wikieł P 4.2b moja wersja nr 99}
Obliczyć granicę $\lim\limits_{x\to\ 8}\frac{x^{2}-8^{2}}{(x-8)(x-27)}$.
\zadStop
\rozwStart{Patryk Wirkus}{Martyna Czarnobaj}
$$\frac{x^{2}-8^{2}}{(x-8)(x-27)}=\frac{x+8}{x-27}$$

$$\lim\limits_{x\to\ 8}\frac{x^{2}-8^{2}}{(x-8)(x-27)}=[\frac{0}{0}]=\lim\limits_{x\to\ 8}\frac{x+8}{x-27}=2 \cdot \frac{8}{8-27} = \frac{16}{-19}$$
\rozwStop
\odpStart
$\frac{16}{-19}$
\odpStop
\testStart
A.$\frac{16}{-19}$
B.$\infty$
C.$-\infty$
D.$0$
E.$\frac{16}{19}$
F.$\frac{8}{27}$
G.$-\frac{16}{19}$
H.$1$
I.$8$
\testStop
\kluczStart
A
\kluczStop



\zadStart{Przykład z Wikieł P 4.2b moja wersja nr 100}
Obliczyć granicę $\lim\limits_{x\to\ 8}\frac{x^{2}-8^{2}}{(x-8)(x-29)}$.
\zadStop
\rozwStart{Patryk Wirkus}{Martyna Czarnobaj}
$$\frac{x^{2}-8^{2}}{(x-8)(x-29)}=\frac{x+8}{x-29}$$

$$\lim\limits_{x\to\ 8}\frac{x^{2}-8^{2}}{(x-8)(x-29)}=[\frac{0}{0}]=\lim\limits_{x\to\ 8}\frac{x+8}{x-29}=2 \cdot \frac{8}{8-29} = \frac{16}{-21}$$
\rozwStop
\odpStart
$\frac{16}{-21}$
\odpStop
\testStart
A.$\frac{16}{-21}$
B.$\infty$
C.$-\infty$
D.$0$
E.$\frac{16}{21}$
F.$\frac{8}{29}$
G.$-\frac{16}{21}$
H.$1$
I.$8$
\testStop
\kluczStart
A
\kluczStop



\zadStart{Przykład z Wikieł P 4.2b moja wersja nr 101}
Obliczyć granicę $\lim\limits_{x\to\ 8}\frac{x^{2}-8^{2}}{(x-8)(x-31)}$.
\zadStop
\rozwStart{Patryk Wirkus}{Martyna Czarnobaj}
$$\frac{x^{2}-8^{2}}{(x-8)(x-31)}=\frac{x+8}{x-31}$$

$$\lim\limits_{x\to\ 8}\frac{x^{2}-8^{2}}{(x-8)(x-31)}=[\frac{0}{0}]=\lim\limits_{x\to\ 8}\frac{x+8}{x-31}=2 \cdot \frac{8}{8-31} = \frac{16}{-23}$$
\rozwStop
\odpStart
$\frac{16}{-23}$
\odpStop
\testStart
A.$\frac{16}{-23}$
B.$\infty$
C.$-\infty$
D.$0$
E.$\frac{16}{23}$
F.$\frac{8}{31}$
G.$-\frac{16}{23}$
H.$1$
I.$8$
\testStop
\kluczStart
A
\kluczStop



\zadStart{Przykład z Wikieł P 4.2b moja wersja nr 102}
Obliczyć granicę $\lim\limits_{x\to\ 8}\frac{x^{2}-8^{2}}{(x-8)(x-33)}$.
\zadStop
\rozwStart{Patryk Wirkus}{Martyna Czarnobaj}
$$\frac{x^{2}-8^{2}}{(x-8)(x-33)}=\frac{x+8}{x-33}$$

$$\lim\limits_{x\to\ 8}\frac{x^{2}-8^{2}}{(x-8)(x-33)}=[\frac{0}{0}]=\lim\limits_{x\to\ 8}\frac{x+8}{x-33}=2 \cdot \frac{8}{8-33} = \frac{16}{-25}$$
\rozwStop
\odpStart
$\frac{16}{-25}$
\odpStop
\testStart
A.$\frac{16}{-25}$
B.$\infty$
C.$-\infty$
D.$0$
E.$\frac{16}{25}$
F.$\frac{8}{33}$
G.$-\frac{16}{25}$
H.$1$
I.$8$
\testStop
\kluczStart
A
\kluczStop



\zadStart{Przykład z Wikieł P 4.2b moja wersja nr 103}
Obliczyć granicę $\lim\limits_{x\to\ 8}\frac{x^{2}-8^{2}}{(x-8)(x-35)}$.
\zadStop
\rozwStart{Patryk Wirkus}{Martyna Czarnobaj}
$$\frac{x^{2}-8^{2}}{(x-8)(x-35)}=\frac{x+8}{x-35}$$

$$\lim\limits_{x\to\ 8}\frac{x^{2}-8^{2}}{(x-8)(x-35)}=[\frac{0}{0}]=\lim\limits_{x\to\ 8}\frac{x+8}{x-35}=2 \cdot \frac{8}{8-35} = \frac{16}{-27}$$
\rozwStop
\odpStart
$\frac{16}{-27}$
\odpStop
\testStart
A.$\frac{16}{-27}$
B.$\infty$
C.$-\infty$
D.$0$
E.$\frac{16}{27}$
F.$\frac{8}{35}$
G.$-\frac{16}{27}$
H.$1$
I.$8$
\testStop
\kluczStart
A
\kluczStop



\zadStart{Przykład z Wikieł P 4.2b moja wersja nr 104}
Obliczyć granicę $\lim\limits_{x\to\ 8}\frac{x^{2}-8^{2}}{(x-8)(x-37)}$.
\zadStop
\rozwStart{Patryk Wirkus}{Martyna Czarnobaj}
$$\frac{x^{2}-8^{2}}{(x-8)(x-37)}=\frac{x+8}{x-37}$$

$$\lim\limits_{x\to\ 8}\frac{x^{2}-8^{2}}{(x-8)(x-37)}=[\frac{0}{0}]=\lim\limits_{x\to\ 8}\frac{x+8}{x-37}=2 \cdot \frac{8}{8-37} = \frac{16}{-29}$$
\rozwStop
\odpStart
$\frac{16}{-29}$
\odpStop
\testStart
A.$\frac{16}{-29}$
B.$\infty$
C.$-\infty$
D.$0$
E.$\frac{16}{29}$
F.$\frac{8}{37}$
G.$-\frac{16}{29}$
H.$1$
I.$8$
\testStop
\kluczStart
A
\kluczStop



\zadStart{Przykład z Wikieł P 4.2b moja wersja nr 105}
Obliczyć granicę $\lim\limits_{x\to\ 8}\frac{x^{2}-8^{2}}{(x-8)(x-39)}$.
\zadStop
\rozwStart{Patryk Wirkus}{Martyna Czarnobaj}
$$\frac{x^{2}-8^{2}}{(x-8)(x-39)}=\frac{x+8}{x-39}$$

$$\lim\limits_{x\to\ 8}\frac{x^{2}-8^{2}}{(x-8)(x-39)}=[\frac{0}{0}]=\lim\limits_{x\to\ 8}\frac{x+8}{x-39}=2 \cdot \frac{8}{8-39} = \frac{16}{-31}$$
\rozwStop
\odpStart
$\frac{16}{-31}$
\odpStop
\testStart
A.$\frac{16}{-31}$
B.$\infty$
C.$-\infty$
D.$0$
E.$\frac{16}{31}$
F.$\frac{8}{39}$
G.$-\frac{16}{31}$
H.$1$
I.$8$
\testStop
\kluczStart
A
\kluczStop



\zadStart{Przykład z Wikieł P 4.2b moja wersja nr 106}
Obliczyć granicę $\lim\limits_{x\to\ 9}\frac{x^{2}-9^{2}}{(x-9)(x-2)}$.
\zadStop
\rozwStart{Patryk Wirkus}{Martyna Czarnobaj}
$$\frac{x^{2}-9^{2}}{(x-9)(x-2)}=\frac{x+9}{x-2}$$

$$\lim\limits_{x\to\ 9}\frac{x^{2}-9^{2}}{(x-9)(x-2)}=[\frac{0}{0}]=\lim\limits_{x\to\ 9}\frac{x+9}{x-2}=2 \cdot \frac{9}{9-2} = \frac{18}{7}$$
\rozwStop
\odpStart
$\frac{18}{7}$
\odpStop
\testStart
A.$\frac{18}{7}$
B.$\infty$
C.$-\infty$
D.$0$
E.$\frac{18}{-7}$
F.$\frac{9}{2}$
G.$-\frac{18}{-7}$
H.$1$
I.$9$
\testStop
\kluczStart
A
\kluczStop



\zadStart{Przykład z Wikieł P 4.2b moja wersja nr 107}
Obliczyć granicę $\lim\limits_{x\to\ 9}\frac{x^{2}-9^{2}}{(x-9)(x-4)}$.
\zadStop
\rozwStart{Patryk Wirkus}{Martyna Czarnobaj}
$$\frac{x^{2}-9^{2}}{(x-9)(x-4)}=\frac{x+9}{x-4}$$

$$\lim\limits_{x\to\ 9}\frac{x^{2}-9^{2}}{(x-9)(x-4)}=[\frac{0}{0}]=\lim\limits_{x\to\ 9}\frac{x+9}{x-4}=2 \cdot \frac{9}{9-4} = \frac{18}{5}$$
\rozwStop
\odpStart
$\frac{18}{5}$
\odpStop
\testStart
A.$\frac{18}{5}$
B.$\infty$
C.$-\infty$
D.$0$
E.$\frac{18}{-5}$
F.$\frac{9}{4}$
G.$-\frac{18}{-5}$
H.$1$
I.$9$
\testStop
\kluczStart
A
\kluczStop



\zadStart{Przykład z Wikieł P 4.2b moja wersja nr 108}
Obliczyć granicę $\lim\limits_{x\to\ 9}\frac{x^{2}-9^{2}}{(x-9)(x-5)}$.
\zadStop
\rozwStart{Patryk Wirkus}{Martyna Czarnobaj}
$$\frac{x^{2}-9^{2}}{(x-9)(x-5)}=\frac{x+9}{x-5}$$

$$\lim\limits_{x\to\ 9}\frac{x^{2}-9^{2}}{(x-9)(x-5)}=[\frac{0}{0}]=\lim\limits_{x\to\ 9}\frac{x+9}{x-5}=2 \cdot \frac{9}{9-5} = \frac{18}{4}$$
\rozwStop
\odpStart
$\frac{18}{4}$
\odpStop
\testStart
A.$\frac{18}{4}$
B.$\infty$
C.$-\infty$
D.$0$
E.$\frac{18}{-4}$
F.$\frac{9}{5}$
G.$-\frac{18}{-4}$
H.$1$
I.$9$
\testStop
\kluczStart
A
\kluczStop



\zadStart{Przykład z Wikieł P 4.2b moja wersja nr 109}
Obliczyć granicę $\lim\limits_{x\to\ 9}\frac{x^{2}-9^{2}}{(x-9)(x-7)}$.
\zadStop
\rozwStart{Patryk Wirkus}{Martyna Czarnobaj}
$$\frac{x^{2}-9^{2}}{(x-9)(x-7)}=\frac{x+9}{x-7}$$

$$\lim\limits_{x\to\ 9}\frac{x^{2}-9^{2}}{(x-9)(x-7)}=[\frac{0}{0}]=\lim\limits_{x\to\ 9}\frac{x+9}{x-7}=2 \cdot \frac{9}{9-7} = \frac{18}{2}$$
\rozwStop
\odpStart
$\frac{18}{2}$
\odpStop
\testStart
A.$\frac{18}{2}$
B.$\infty$
C.$-\infty$
D.$0$
E.$\frac{18}{-2}$
F.$\frac{9}{7}$
G.$-\frac{18}{-2}$
H.$1$
I.$9$
\testStop
\kluczStart
A
\kluczStop



\zadStart{Przykład z Wikieł P 4.2b moja wersja nr 110}
Obliczyć granicę $\lim\limits_{x\to\ 9}\frac{x^{2}-9^{2}}{(x-9)(x-11)}$.
\zadStop
\rozwStart{Patryk Wirkus}{Martyna Czarnobaj}
$$\frac{x^{2}-9^{2}}{(x-9)(x-11)}=\frac{x+9}{x-11}$$

$$\lim\limits_{x\to\ 9}\frac{x^{2}-9^{2}}{(x-9)(x-11)}=[\frac{0}{0}]=\lim\limits_{x\to\ 9}\frac{x+9}{x-11}=2 \cdot \frac{9}{9-11} = \frac{18}{-2}$$
\rozwStop
\odpStart
$\frac{18}{-2}$
\odpStop
\testStart
A.$\frac{18}{-2}$
B.$\infty$
C.$-\infty$
D.$0$
E.$\frac{18}{2}$
F.$\frac{9}{11}$
G.$-\frac{18}{2}$
H.$1$
I.$9$
\testStop
\kluczStart
A
\kluczStop



\zadStart{Przykład z Wikieł P 4.2b moja wersja nr 111}
Obliczyć granicę $\lim\limits_{x\to\ 9}\frac{x^{2}-9^{2}}{(x-9)(x-13)}$.
\zadStop
\rozwStart{Patryk Wirkus}{Martyna Czarnobaj}
$$\frac{x^{2}-9^{2}}{(x-9)(x-13)}=\frac{x+9}{x-13}$$

$$\lim\limits_{x\to\ 9}\frac{x^{2}-9^{2}}{(x-9)(x-13)}=[\frac{0}{0}]=\lim\limits_{x\to\ 9}\frac{x+9}{x-13}=2 \cdot \frac{9}{9-13} = \frac{18}{-4}$$
\rozwStop
\odpStart
$\frac{18}{-4}$
\odpStop
\testStart
A.$\frac{18}{-4}$
B.$\infty$
C.$-\infty$
D.$0$
E.$\frac{18}{4}$
F.$\frac{9}{13}$
G.$-\frac{18}{4}$
H.$1$
I.$9$
\testStop
\kluczStart
A
\kluczStop



\zadStart{Przykład z Wikieł P 4.2b moja wersja nr 112}
Obliczyć granicę $\lim\limits_{x\to\ 9}\frac{x^{2}-9^{2}}{(x-9)(x-14)}$.
\zadStop
\rozwStart{Patryk Wirkus}{Martyna Czarnobaj}
$$\frac{x^{2}-9^{2}}{(x-9)(x-14)}=\frac{x+9}{x-14}$$

$$\lim\limits_{x\to\ 9}\frac{x^{2}-9^{2}}{(x-9)(x-14)}=[\frac{0}{0}]=\lim\limits_{x\to\ 9}\frac{x+9}{x-14}=2 \cdot \frac{9}{9-14} = \frac{18}{-5}$$
\rozwStop
\odpStart
$\frac{18}{-5}$
\odpStop
\testStart
A.$\frac{18}{-5}$
B.$\infty$
C.$-\infty$
D.$0$
E.$\frac{18}{5}$
F.$\frac{9}{14}$
G.$-\frac{18}{5}$
H.$1$
I.$9$
\testStop
\kluczStart
A
\kluczStop



\zadStart{Przykład z Wikieł P 4.2b moja wersja nr 113}
Obliczyć granicę $\lim\limits_{x\to\ 9}\frac{x^{2}-9^{2}}{(x-9)(x-16)}$.
\zadStop
\rozwStart{Patryk Wirkus}{Martyna Czarnobaj}
$$\frac{x^{2}-9^{2}}{(x-9)(x-16)}=\frac{x+9}{x-16}$$

$$\lim\limits_{x\to\ 9}\frac{x^{2}-9^{2}}{(x-9)(x-16)}=[\frac{0}{0}]=\lim\limits_{x\to\ 9}\frac{x+9}{x-16}=2 \cdot \frac{9}{9-16} = \frac{18}{-7}$$
\rozwStop
\odpStart
$\frac{18}{-7}$
\odpStop
\testStart
A.$\frac{18}{-7}$
B.$\infty$
C.$-\infty$
D.$0$
E.$\frac{18}{7}$
F.$\frac{9}{16}$
G.$-\frac{18}{7}$
H.$1$
I.$9$
\testStop
\kluczStart
A
\kluczStop



\zadStart{Przykład z Wikieł P 4.2b moja wersja nr 114}
Obliczyć granicę $\lim\limits_{x\to\ 9}\frac{x^{2}-9^{2}}{(x-9)(x-17)}$.
\zadStop
\rozwStart{Patryk Wirkus}{Martyna Czarnobaj}
$$\frac{x^{2}-9^{2}}{(x-9)(x-17)}=\frac{x+9}{x-17}$$

$$\lim\limits_{x\to\ 9}\frac{x^{2}-9^{2}}{(x-9)(x-17)}=[\frac{0}{0}]=\lim\limits_{x\to\ 9}\frac{x+9}{x-17}=2 \cdot \frac{9}{9-17} = \frac{18}{-8}$$
\rozwStop
\odpStart
$\frac{18}{-8}$
\odpStop
\testStart
A.$\frac{18}{-8}$
B.$\infty$
C.$-\infty$
D.$0$
E.$\frac{18}{8}$
F.$\frac{9}{17}$
G.$-\frac{18}{8}$
H.$1$
I.$9$
\testStop
\kluczStart
A
\kluczStop



\zadStart{Przykład z Wikieł P 4.2b moja wersja nr 115}
Obliczyć granicę $\lim\limits_{x\to\ 9}\frac{x^{2}-9^{2}}{(x-9)(x-19)}$.
\zadStop
\rozwStart{Patryk Wirkus}{Martyna Czarnobaj}
$$\frac{x^{2}-9^{2}}{(x-9)(x-19)}=\frac{x+9}{x-19}$$

$$\lim\limits_{x\to\ 9}\frac{x^{2}-9^{2}}{(x-9)(x-19)}=[\frac{0}{0}]=\lim\limits_{x\to\ 9}\frac{x+9}{x-19}=2 \cdot \frac{9}{9-19} = \frac{18}{-10}$$
\rozwStop
\odpStart
$\frac{18}{-10}$
\odpStop
\testStart
A.$\frac{18}{-10}$
B.$\infty$
C.$-\infty$
D.$0$
E.$\frac{18}{10}$
F.$\frac{9}{19}$
G.$-\frac{18}{10}$
H.$1$
I.$9$
\testStop
\kluczStart
A
\kluczStop



\zadStart{Przykład z Wikieł P 4.2b moja wersja nr 116}
Obliczyć granicę $\lim\limits_{x\to\ 9}\frac{x^{2}-9^{2}}{(x-9)(x-20)}$.
\zadStop
\rozwStart{Patryk Wirkus}{Martyna Czarnobaj}
$$\frac{x^{2}-9^{2}}{(x-9)(x-20)}=\frac{x+9}{x-20}$$

$$\lim\limits_{x\to\ 9}\frac{x^{2}-9^{2}}{(x-9)(x-20)}=[\frac{0}{0}]=\lim\limits_{x\to\ 9}\frac{x+9}{x-20}=2 \cdot \frac{9}{9-20} = \frac{18}{-11}$$
\rozwStop
\odpStart
$\frac{18}{-11}$
\odpStop
\testStart
A.$\frac{18}{-11}$
B.$\infty$
C.$-\infty$
D.$0$
E.$\frac{18}{11}$
F.$\frac{9}{20}$
G.$-\frac{18}{11}$
H.$1$
I.$9$
\testStop
\kluczStart
A
\kluczStop



\zadStart{Przykład z Wikieł P 4.2b moja wersja nr 117}
Obliczyć granicę $\lim\limits_{x\to\ 9}\frac{x^{2}-9^{2}}{(x-9)(x-22)}$.
\zadStop
\rozwStart{Patryk Wirkus}{Martyna Czarnobaj}
$$\frac{x^{2}-9^{2}}{(x-9)(x-22)}=\frac{x+9}{x-22}$$

$$\lim\limits_{x\to\ 9}\frac{x^{2}-9^{2}}{(x-9)(x-22)}=[\frac{0}{0}]=\lim\limits_{x\to\ 9}\frac{x+9}{x-22}=2 \cdot \frac{9}{9-22} = \frac{18}{-13}$$
\rozwStop
\odpStart
$\frac{18}{-13}$
\odpStop
\testStart
A.$\frac{18}{-13}$
B.$\infty$
C.$-\infty$
D.$0$
E.$\frac{18}{13}$
F.$\frac{9}{22}$
G.$-\frac{18}{13}$
H.$1$
I.$9$
\testStop
\kluczStart
A
\kluczStop



\zadStart{Przykład z Wikieł P 4.2b moja wersja nr 118}
Obliczyć granicę $\lim\limits_{x\to\ 9}\frac{x^{2}-9^{2}}{(x-9)(x-23)}$.
\zadStop
\rozwStart{Patryk Wirkus}{Martyna Czarnobaj}
$$\frac{x^{2}-9^{2}}{(x-9)(x-23)}=\frac{x+9}{x-23}$$

$$\lim\limits_{x\to\ 9}\frac{x^{2}-9^{2}}{(x-9)(x-23)}=[\frac{0}{0}]=\lim\limits_{x\to\ 9}\frac{x+9}{x-23}=2 \cdot \frac{9}{9-23} = \frac{18}{-14}$$
\rozwStop
\odpStart
$\frac{18}{-14}$
\odpStop
\testStart
A.$\frac{18}{-14}$
B.$\infty$
C.$-\infty$
D.$0$
E.$\frac{18}{14}$
F.$\frac{9}{23}$
G.$-\frac{18}{14}$
H.$1$
I.$9$
\testStop
\kluczStart
A
\kluczStop



\zadStart{Przykład z Wikieł P 4.2b moja wersja nr 119}
Obliczyć granicę $\lim\limits_{x\to\ 9}\frac{x^{2}-9^{2}}{(x-9)(x-25)}$.
\zadStop
\rozwStart{Patryk Wirkus}{Martyna Czarnobaj}
$$\frac{x^{2}-9^{2}}{(x-9)(x-25)}=\frac{x+9}{x-25}$$

$$\lim\limits_{x\to\ 9}\frac{x^{2}-9^{2}}{(x-9)(x-25)}=[\frac{0}{0}]=\lim\limits_{x\to\ 9}\frac{x+9}{x-25}=2 \cdot \frac{9}{9-25} = \frac{18}{-16}$$
\rozwStop
\odpStart
$\frac{18}{-16}$
\odpStop
\testStart
A.$\frac{18}{-16}$
B.$\infty$
C.$-\infty$
D.$0$
E.$\frac{18}{16}$
F.$\frac{9}{25}$
G.$-\frac{18}{16}$
H.$1$
I.$9$
\testStop
\kluczStart
A
\kluczStop



\zadStart{Przykład z Wikieł P 4.2b moja wersja nr 120}
Obliczyć granicę $\lim\limits_{x\to\ 9}\frac{x^{2}-9^{2}}{(x-9)(x-26)}$.
\zadStop
\rozwStart{Patryk Wirkus}{Martyna Czarnobaj}
$$\frac{x^{2}-9^{2}}{(x-9)(x-26)}=\frac{x+9}{x-26}$$

$$\lim\limits_{x\to\ 9}\frac{x^{2}-9^{2}}{(x-9)(x-26)}=[\frac{0}{0}]=\lim\limits_{x\to\ 9}\frac{x+9}{x-26}=2 \cdot \frac{9}{9-26} = \frac{18}{-17}$$
\rozwStop
\odpStart
$\frac{18}{-17}$
\odpStop
\testStart
A.$\frac{18}{-17}$
B.$\infty$
C.$-\infty$
D.$0$
E.$\frac{18}{17}$
F.$\frac{9}{26}$
G.$-\frac{18}{17}$
H.$1$
I.$9$
\testStop
\kluczStart
A
\kluczStop



\zadStart{Przykład z Wikieł P 4.2b moja wersja nr 121}
Obliczyć granicę $\lim\limits_{x\to\ 9}\frac{x^{2}-9^{2}}{(x-9)(x-28)}$.
\zadStop
\rozwStart{Patryk Wirkus}{Martyna Czarnobaj}
$$\frac{x^{2}-9^{2}}{(x-9)(x-28)}=\frac{x+9}{x-28}$$

$$\lim\limits_{x\to\ 9}\frac{x^{2}-9^{2}}{(x-9)(x-28)}=[\frac{0}{0}]=\lim\limits_{x\to\ 9}\frac{x+9}{x-28}=2 \cdot \frac{9}{9-28} = \frac{18}{-19}$$
\rozwStop
\odpStart
$\frac{18}{-19}$
\odpStop
\testStart
A.$\frac{18}{-19}$
B.$\infty$
C.$-\infty$
D.$0$
E.$\frac{18}{19}$
F.$\frac{9}{28}$
G.$-\frac{18}{19}$
H.$1$
I.$9$
\testStop
\kluczStart
A
\kluczStop



\zadStart{Przykład z Wikieł P 4.2b moja wersja nr 122}
Obliczyć granicę $\lim\limits_{x\to\ 9}\frac{x^{2}-9^{2}}{(x-9)(x-29)}$.
\zadStop
\rozwStart{Patryk Wirkus}{Martyna Czarnobaj}
$$\frac{x^{2}-9^{2}}{(x-9)(x-29)}=\frac{x+9}{x-29}$$

$$\lim\limits_{x\to\ 9}\frac{x^{2}-9^{2}}{(x-9)(x-29)}=[\frac{0}{0}]=\lim\limits_{x\to\ 9}\frac{x+9}{x-29}=2 \cdot \frac{9}{9-29} = \frac{18}{-20}$$
\rozwStop
\odpStart
$\frac{18}{-20}$
\odpStop
\testStart
A.$\frac{18}{-20}$
B.$\infty$
C.$-\infty$
D.$0$
E.$\frac{18}{20}$
F.$\frac{9}{29}$
G.$-\frac{18}{20}$
H.$1$
I.$9$
\testStop
\kluczStart
A
\kluczStop



\zadStart{Przykład z Wikieł P 4.2b moja wersja nr 123}
Obliczyć granicę $\lim\limits_{x\to\ 9}\frac{x^{2}-9^{2}}{(x-9)(x-31)}$.
\zadStop
\rozwStart{Patryk Wirkus}{Martyna Czarnobaj}
$$\frac{x^{2}-9^{2}}{(x-9)(x-31)}=\frac{x+9}{x-31}$$

$$\lim\limits_{x\to\ 9}\frac{x^{2}-9^{2}}{(x-9)(x-31)}=[\frac{0}{0}]=\lim\limits_{x\to\ 9}\frac{x+9}{x-31}=2 \cdot \frac{9}{9-31} = \frac{18}{-22}$$
\rozwStop
\odpStart
$\frac{18}{-22}$
\odpStop
\testStart
A.$\frac{18}{-22}$
B.$\infty$
C.$-\infty$
D.$0$
E.$\frac{18}{22}$
F.$\frac{9}{31}$
G.$-\frac{18}{22}$
H.$1$
I.$9$
\testStop
\kluczStart
A
\kluczStop



\zadStart{Przykład z Wikieł P 4.2b moja wersja nr 124}
Obliczyć granicę $\lim\limits_{x\to\ 9}\frac{x^{2}-9^{2}}{(x-9)(x-32)}$.
\zadStop
\rozwStart{Patryk Wirkus}{Martyna Czarnobaj}
$$\frac{x^{2}-9^{2}}{(x-9)(x-32)}=\frac{x+9}{x-32}$$

$$\lim\limits_{x\to\ 9}\frac{x^{2}-9^{2}}{(x-9)(x-32)}=[\frac{0}{0}]=\lim\limits_{x\to\ 9}\frac{x+9}{x-32}=2 \cdot \frac{9}{9-32} = \frac{18}{-23}$$
\rozwStop
\odpStart
$\frac{18}{-23}$
\odpStop
\testStart
A.$\frac{18}{-23}$
B.$\infty$
C.$-\infty$
D.$0$
E.$\frac{18}{23}$
F.$\frac{9}{32}$
G.$-\frac{18}{23}$
H.$1$
I.$9$
\testStop
\kluczStart
A
\kluczStop



\zadStart{Przykład z Wikieł P 4.2b moja wersja nr 125}
Obliczyć granicę $\lim\limits_{x\to\ 9}\frac{x^{2}-9^{2}}{(x-9)(x-34)}$.
\zadStop
\rozwStart{Patryk Wirkus}{Martyna Czarnobaj}
$$\frac{x^{2}-9^{2}}{(x-9)(x-34)}=\frac{x+9}{x-34}$$

$$\lim\limits_{x\to\ 9}\frac{x^{2}-9^{2}}{(x-9)(x-34)}=[\frac{0}{0}]=\lim\limits_{x\to\ 9}\frac{x+9}{x-34}=2 \cdot \frac{9}{9-34} = \frac{18}{-25}$$
\rozwStop
\odpStart
$\frac{18}{-25}$
\odpStop
\testStart
A.$\frac{18}{-25}$
B.$\infty$
C.$-\infty$
D.$0$
E.$\frac{18}{25}$
F.$\frac{9}{34}$
G.$-\frac{18}{25}$
H.$1$
I.$9$
\testStop
\kluczStart
A
\kluczStop



\zadStart{Przykład z Wikieł P 4.2b moja wersja nr 126}
Obliczyć granicę $\lim\limits_{x\to\ 9}\frac{x^{2}-9^{2}}{(x-9)(x-35)}$.
\zadStop
\rozwStart{Patryk Wirkus}{Martyna Czarnobaj}
$$\frac{x^{2}-9^{2}}{(x-9)(x-35)}=\frac{x+9}{x-35}$$

$$\lim\limits_{x\to\ 9}\frac{x^{2}-9^{2}}{(x-9)(x-35)}=[\frac{0}{0}]=\lim\limits_{x\to\ 9}\frac{x+9}{x-35}=2 \cdot \frac{9}{9-35} = \frac{18}{-26}$$
\rozwStop
\odpStart
$\frac{18}{-26}$
\odpStop
\testStart
A.$\frac{18}{-26}$
B.$\infty$
C.$-\infty$
D.$0$
E.$\frac{18}{26}$
F.$\frac{9}{35}$
G.$-\frac{18}{26}$
H.$1$
I.$9$
\testStop
\kluczStart
A
\kluczStop



\zadStart{Przykład z Wikieł P 4.2b moja wersja nr 127}
Obliczyć granicę $\lim\limits_{x\to\ 9}\frac{x^{2}-9^{2}}{(x-9)(x-37)}$.
\zadStop
\rozwStart{Patryk Wirkus}{Martyna Czarnobaj}
$$\frac{x^{2}-9^{2}}{(x-9)(x-37)}=\frac{x+9}{x-37}$$

$$\lim\limits_{x\to\ 9}\frac{x^{2}-9^{2}}{(x-9)(x-37)}=[\frac{0}{0}]=\lim\limits_{x\to\ 9}\frac{x+9}{x-37}=2 \cdot \frac{9}{9-37} = \frac{18}{-28}$$
\rozwStop
\odpStart
$\frac{18}{-28}$
\odpStop
\testStart
A.$\frac{18}{-28}$
B.$\infty$
C.$-\infty$
D.$0$
E.$\frac{18}{28}$
F.$\frac{9}{37}$
G.$-\frac{18}{28}$
H.$1$
I.$9$
\testStop
\kluczStart
A
\kluczStop



\zadStart{Przykład z Wikieł P 4.2b moja wersja nr 128}
Obliczyć granicę $\lim\limits_{x\to\ 9}\frac{x^{2}-9^{2}}{(x-9)(x-38)}$.
\zadStop
\rozwStart{Patryk Wirkus}{Martyna Czarnobaj}
$$\frac{x^{2}-9^{2}}{(x-9)(x-38)}=\frac{x+9}{x-38}$$

$$\lim\limits_{x\to\ 9}\frac{x^{2}-9^{2}}{(x-9)(x-38)}=[\frac{0}{0}]=\lim\limits_{x\to\ 9}\frac{x+9}{x-38}=2 \cdot \frac{9}{9-38} = \frac{18}{-29}$$
\rozwStop
\odpStart
$\frac{18}{-29}$
\odpStop
\testStart
A.$\frac{18}{-29}$
B.$\infty$
C.$-\infty$
D.$0$
E.$\frac{18}{29}$
F.$\frac{9}{38}$
G.$-\frac{18}{29}$
H.$1$
I.$9$
\testStop
\kluczStart
A
\kluczStop



\zadStart{Przykład z Wikieł P 4.2b moja wersja nr 129}
Obliczyć granicę $\lim\limits_{x\to\ 9}\frac{x^{2}-9^{2}}{(x-9)(x-40)}$.
\zadStop
\rozwStart{Patryk Wirkus}{Martyna Czarnobaj}
$$\frac{x^{2}-9^{2}}{(x-9)(x-40)}=\frac{x+9}{x-40}$$

$$\lim\limits_{x\to\ 9}\frac{x^{2}-9^{2}}{(x-9)(x-40)}=[\frac{0}{0}]=\lim\limits_{x\to\ 9}\frac{x+9}{x-40}=2 \cdot \frac{9}{9-40} = \frac{18}{-31}$$
\rozwStop
\odpStart
$\frac{18}{-31}$
\odpStop
\testStart
A.$\frac{18}{-31}$
B.$\infty$
C.$-\infty$
D.$0$
E.$\frac{18}{31}$
F.$\frac{9}{40}$
G.$-\frac{18}{31}$
H.$1$
I.$9$
\testStop
\kluczStart
A
\kluczStop



\zadStart{Przykład z Wikieł P 4.2b moja wersja nr 130}
Obliczyć granicę $\lim\limits_{x\to\ 10}\frac{x^{2}-10^{2}}{(x-10)(x-3)}$.
\zadStop
\rozwStart{Patryk Wirkus}{Martyna Czarnobaj}
$$\frac{x^{2}-10^{2}}{(x-10)(x-3)}=\frac{x+10}{x-3}$$

$$\lim\limits_{x\to\ 10}\frac{x^{2}-10^{2}}{(x-10)(x-3)}=[\frac{0}{0}]=\lim\limits_{x\to\ 10}\frac{x+10}{x-3}=2 \cdot \frac{10}{10-3} = \frac{20}{7}$$
\rozwStop
\odpStart
$\frac{20}{7}$
\odpStop
\testStart
A.$\frac{20}{7}$
B.$\infty$
C.$-\infty$
D.$0$
E.$\frac{20}{-7}$
F.$\frac{10}{3}$
G.$-\frac{20}{-7}$
H.$1$
I.$10$
\testStop
\kluczStart
A
\kluczStop



\zadStart{Przykład z Wikieł P 4.2b moja wersja nr 131}
Obliczyć granicę $\lim\limits_{x\to\ 10}\frac{x^{2}-10^{2}}{(x-10)(x-7)}$.
\zadStop
\rozwStart{Patryk Wirkus}{Martyna Czarnobaj}
$$\frac{x^{2}-10^{2}}{(x-10)(x-7)}=\frac{x+10}{x-7}$$

$$\lim\limits_{x\to\ 10}\frac{x^{2}-10^{2}}{(x-10)(x-7)}=[\frac{0}{0}]=\lim\limits_{x\to\ 10}\frac{x+10}{x-7}=2 \cdot \frac{10}{10-7} = \frac{20}{3}$$
\rozwStop
\odpStart
$\frac{20}{3}$
\odpStop
\testStart
A.$\frac{20}{3}$
B.$\infty$
C.$-\infty$
D.$0$
E.$\frac{20}{-3}$
F.$\frac{10}{7}$
G.$-\frac{20}{-3}$
H.$1$
I.$10$
\testStop
\kluczStart
A
\kluczStop



\zadStart{Przykład z Wikieł P 4.2b moja wersja nr 132}
Obliczyć granicę $\lim\limits_{x\to\ 10}\frac{x^{2}-10^{2}}{(x-10)(x-13)}$.
\zadStop
\rozwStart{Patryk Wirkus}{Martyna Czarnobaj}
$$\frac{x^{2}-10^{2}}{(x-10)(x-13)}=\frac{x+10}{x-13}$$

$$\lim\limits_{x\to\ 10}\frac{x^{2}-10^{2}}{(x-10)(x-13)}=[\frac{0}{0}]=\lim\limits_{x\to\ 10}\frac{x+10}{x-13}=2 \cdot \frac{10}{10-13} = \frac{20}{-3}$$
\rozwStop
\odpStart
$\frac{20}{-3}$
\odpStop
\testStart
A.$\frac{20}{-3}$
B.$\infty$
C.$-\infty$
D.$0$
E.$\frac{20}{3}$
F.$\frac{10}{13}$
G.$-\frac{20}{3}$
H.$1$
I.$10$
\testStop
\kluczStart
A
\kluczStop



\zadStart{Przykład z Wikieł P 4.2b moja wersja nr 133}
Obliczyć granicę $\lim\limits_{x\to\ 10}\frac{x^{2}-10^{2}}{(x-10)(x-17)}$.
\zadStop
\rozwStart{Patryk Wirkus}{Martyna Czarnobaj}
$$\frac{x^{2}-10^{2}}{(x-10)(x-17)}=\frac{x+10}{x-17}$$

$$\lim\limits_{x\to\ 10}\frac{x^{2}-10^{2}}{(x-10)(x-17)}=[\frac{0}{0}]=\lim\limits_{x\to\ 10}\frac{x+10}{x-17}=2 \cdot \frac{10}{10-17} = \frac{20}{-7}$$
\rozwStop
\odpStart
$\frac{20}{-7}$
\odpStop
\testStart
A.$\frac{20}{-7}$
B.$\infty$
C.$-\infty$
D.$0$
E.$\frac{20}{7}$
F.$\frac{10}{17}$
G.$-\frac{20}{7}$
H.$1$
I.$10$
\testStop
\kluczStart
A
\kluczStop



\zadStart{Przykład z Wikieł P 4.2b moja wersja nr 134}
Obliczyć granicę $\lim\limits_{x\to\ 10}\frac{x^{2}-10^{2}}{(x-10)(x-19)}$.
\zadStop
\rozwStart{Patryk Wirkus}{Martyna Czarnobaj}
$$\frac{x^{2}-10^{2}}{(x-10)(x-19)}=\frac{x+10}{x-19}$$

$$\lim\limits_{x\to\ 10}\frac{x^{2}-10^{2}}{(x-10)(x-19)}=[\frac{0}{0}]=\lim\limits_{x\to\ 10}\frac{x+10}{x-19}=2 \cdot \frac{10}{10-19} = \frac{20}{-9}$$
\rozwStop
\odpStart
$\frac{20}{-9}$
\odpStop
\testStart
A.$\frac{20}{-9}$
B.$\infty$
C.$-\infty$
D.$0$
E.$\frac{20}{9}$
F.$\frac{10}{19}$
G.$-\frac{20}{9}$
H.$1$
I.$10$
\testStop
\kluczStart
A
\kluczStop



\zadStart{Przykład z Wikieł P 4.2b moja wersja nr 135}
Obliczyć granicę $\lim\limits_{x\to\ 10}\frac{x^{2}-10^{2}}{(x-10)(x-21)}$.
\zadStop
\rozwStart{Patryk Wirkus}{Martyna Czarnobaj}
$$\frac{x^{2}-10^{2}}{(x-10)(x-21)}=\frac{x+10}{x-21}$$

$$\lim\limits_{x\to\ 10}\frac{x^{2}-10^{2}}{(x-10)(x-21)}=[\frac{0}{0}]=\lim\limits_{x\to\ 10}\frac{x+10}{x-21}=2 \cdot \frac{10}{10-21} = \frac{20}{-11}$$
\rozwStop
\odpStart
$\frac{20}{-11}$
\odpStop
\testStart
A.$\frac{20}{-11}$
B.$\infty$
C.$-\infty$
D.$0$
E.$\frac{20}{11}$
F.$\frac{10}{21}$
G.$-\frac{20}{11}$
H.$1$
I.$10$
\testStop
\kluczStart
A
\kluczStop



\zadStart{Przykład z Wikieł P 4.2b moja wersja nr 136}
Obliczyć granicę $\lim\limits_{x\to\ 10}\frac{x^{2}-10^{2}}{(x-10)(x-23)}$.
\zadStop
\rozwStart{Patryk Wirkus}{Martyna Czarnobaj}
$$\frac{x^{2}-10^{2}}{(x-10)(x-23)}=\frac{x+10}{x-23}$$

$$\lim\limits_{x\to\ 10}\frac{x^{2}-10^{2}}{(x-10)(x-23)}=[\frac{0}{0}]=\lim\limits_{x\to\ 10}\frac{x+10}{x-23}=2 \cdot \frac{10}{10-23} = \frac{20}{-13}$$
\rozwStop
\odpStart
$\frac{20}{-13}$
\odpStop
\testStart
A.$\frac{20}{-13}$
B.$\infty$
C.$-\infty$
D.$0$
E.$\frac{20}{13}$
F.$\frac{10}{23}$
G.$-\frac{20}{13}$
H.$1$
I.$10$
\testStop
\kluczStart
A
\kluczStop



\zadStart{Przykład z Wikieł P 4.2b moja wersja nr 137}
Obliczyć granicę $\lim\limits_{x\to\ 10}\frac{x^{2}-10^{2}}{(x-10)(x-27)}$.
\zadStop
\rozwStart{Patryk Wirkus}{Martyna Czarnobaj}
$$\frac{x^{2}-10^{2}}{(x-10)(x-27)}=\frac{x+10}{x-27}$$

$$\lim\limits_{x\to\ 10}\frac{x^{2}-10^{2}}{(x-10)(x-27)}=[\frac{0}{0}]=\lim\limits_{x\to\ 10}\frac{x+10}{x-27}=2 \cdot \frac{10}{10-27} = \frac{20}{-17}$$
\rozwStop
\odpStart
$\frac{20}{-17}$
\odpStop
\testStart
A.$\frac{20}{-17}$
B.$\infty$
C.$-\infty$
D.$0$
E.$\frac{20}{17}$
F.$\frac{10}{27}$
G.$-\frac{20}{17}$
H.$1$
I.$10$
\testStop
\kluczStart
A
\kluczStop



\zadStart{Przykład z Wikieł P 4.2b moja wersja nr 138}
Obliczyć granicę $\lim\limits_{x\to\ 10}\frac{x^{2}-10^{2}}{(x-10)(x-29)}$.
\zadStop
\rozwStart{Patryk Wirkus}{Martyna Czarnobaj}
$$\frac{x^{2}-10^{2}}{(x-10)(x-29)}=\frac{x+10}{x-29}$$

$$\lim\limits_{x\to\ 10}\frac{x^{2}-10^{2}}{(x-10)(x-29)}=[\frac{0}{0}]=\lim\limits_{x\to\ 10}\frac{x+10}{x-29}=2 \cdot \frac{10}{10-29} = \frac{20}{-19}$$
\rozwStop
\odpStart
$\frac{20}{-19}$
\odpStop
\testStart
A.$\frac{20}{-19}$
B.$\infty$
C.$-\infty$
D.$0$
E.$\frac{20}{19}$
F.$\frac{10}{29}$
G.$-\frac{20}{19}$
H.$1$
I.$10$
\testStop
\kluczStart
A
\kluczStop



\zadStart{Przykład z Wikieł P 4.2b moja wersja nr 139}
Obliczyć granicę $\lim\limits_{x\to\ 10}\frac{x^{2}-10^{2}}{(x-10)(x-31)}$.
\zadStop
\rozwStart{Patryk Wirkus}{Martyna Czarnobaj}
$$\frac{x^{2}-10^{2}}{(x-10)(x-31)}=\frac{x+10}{x-31}$$

$$\lim\limits_{x\to\ 10}\frac{x^{2}-10^{2}}{(x-10)(x-31)}=[\frac{0}{0}]=\lim\limits_{x\to\ 10}\frac{x+10}{x-31}=2 \cdot \frac{10}{10-31} = \frac{20}{-21}$$
\rozwStop
\odpStart
$\frac{20}{-21}$
\odpStop
\testStart
A.$\frac{20}{-21}$
B.$\infty$
C.$-\infty$
D.$0$
E.$\frac{20}{21}$
F.$\frac{10}{31}$
G.$-\frac{20}{21}$
H.$1$
I.$10$
\testStop
\kluczStart
A
\kluczStop



\zadStart{Przykład z Wikieł P 4.2b moja wersja nr 140}
Obliczyć granicę $\lim\limits_{x\to\ 10}\frac{x^{2}-10^{2}}{(x-10)(x-33)}$.
\zadStop
\rozwStart{Patryk Wirkus}{Martyna Czarnobaj}
$$\frac{x^{2}-10^{2}}{(x-10)(x-33)}=\frac{x+10}{x-33}$$

$$\lim\limits_{x\to\ 10}\frac{x^{2}-10^{2}}{(x-10)(x-33)}=[\frac{0}{0}]=\lim\limits_{x\to\ 10}\frac{x+10}{x-33}=2 \cdot \frac{10}{10-33} = \frac{20}{-23}$$
\rozwStop
\odpStart
$\frac{20}{-23}$
\odpStop
\testStart
A.$\frac{20}{-23}$
B.$\infty$
C.$-\infty$
D.$0$
E.$\frac{20}{23}$
F.$\frac{10}{33}$
G.$-\frac{20}{23}$
H.$1$
I.$10$
\testStop
\kluczStart
A
\kluczStop



\zadStart{Przykład z Wikieł P 4.2b moja wersja nr 141}
Obliczyć granicę $\lim\limits_{x\to\ 10}\frac{x^{2}-10^{2}}{(x-10)(x-37)}$.
\zadStop
\rozwStart{Patryk Wirkus}{Martyna Czarnobaj}
$$\frac{x^{2}-10^{2}}{(x-10)(x-37)}=\frac{x+10}{x-37}$$

$$\lim\limits_{x\to\ 10}\frac{x^{2}-10^{2}}{(x-10)(x-37)}=[\frac{0}{0}]=\lim\limits_{x\to\ 10}\frac{x+10}{x-37}=2 \cdot \frac{10}{10-37} = \frac{20}{-27}$$
\rozwStop
\odpStart
$\frac{20}{-27}$
\odpStop
\testStart
A.$\frac{20}{-27}$
B.$\infty$
C.$-\infty$
D.$0$
E.$\frac{20}{27}$
F.$\frac{10}{37}$
G.$-\frac{20}{27}$
H.$1$
I.$10$
\testStop
\kluczStart
A
\kluczStop



\zadStart{Przykład z Wikieł P 4.2b moja wersja nr 142}
Obliczyć granicę $\lim\limits_{x\to\ 10}\frac{x^{2}-10^{2}}{(x-10)(x-39)}$.
\zadStop
\rozwStart{Patryk Wirkus}{Martyna Czarnobaj}
$$\frac{x^{2}-10^{2}}{(x-10)(x-39)}=\frac{x+10}{x-39}$$

$$\lim\limits_{x\to\ 10}\frac{x^{2}-10^{2}}{(x-10)(x-39)}=[\frac{0}{0}]=\lim\limits_{x\to\ 10}\frac{x+10}{x-39}=2 \cdot \frac{10}{10-39} = \frac{20}{-29}$$
\rozwStop
\odpStart
$\frac{20}{-29}$
\odpStop
\testStart
A.$\frac{20}{-29}$
B.$\infty$
C.$-\infty$
D.$0$
E.$\frac{20}{29}$
F.$\frac{10}{39}$
G.$-\frac{20}{29}$
H.$1$
I.$10$
\testStop
\kluczStart
A
\kluczStop



\zadStart{Przykład z Wikieł P 4.2b moja wersja nr 143}
Obliczyć granicę $\lim\limits_{x\to\ 11}\frac{x^{2}-11^{2}}{(x-11)(x-2)}$.
\zadStop
\rozwStart{Patryk Wirkus}{Martyna Czarnobaj}
$$\frac{x^{2}-11^{2}}{(x-11)(x-2)}=\frac{x+11}{x-2}$$

$$\lim\limits_{x\to\ 11}\frac{x^{2}-11^{2}}{(x-11)(x-2)}=[\frac{0}{0}]=\lim\limits_{x\to\ 11}\frac{x+11}{x-2}=2 \cdot \frac{11}{11-2} = \frac{22}{9}$$
\rozwStop
\odpStart
$\frac{22}{9}$
\odpStop
\testStart
A.$\frac{22}{9}$
B.$\infty$
C.$-\infty$
D.$0$
E.$\frac{22}{-9}$
F.$\frac{11}{2}$
G.$-\frac{22}{-9}$
H.$1$
I.$11$
\testStop
\kluczStart
A
\kluczStop



\zadStart{Przykład z Wikieł P 4.2b moja wersja nr 144}
Obliczyć granicę $\lim\limits_{x\to\ 11}\frac{x^{2}-11^{2}}{(x-11)(x-3)}$.
\zadStop
\rozwStart{Patryk Wirkus}{Martyna Czarnobaj}
$$\frac{x^{2}-11^{2}}{(x-11)(x-3)}=\frac{x+11}{x-3}$$

$$\lim\limits_{x\to\ 11}\frac{x^{2}-11^{2}}{(x-11)(x-3)}=[\frac{0}{0}]=\lim\limits_{x\to\ 11}\frac{x+11}{x-3}=2 \cdot \frac{11}{11-3} = \frac{22}{8}$$
\rozwStop
\odpStart
$\frac{22}{8}$
\odpStop
\testStart
A.$\frac{22}{8}$
B.$\infty$
C.$-\infty$
D.$0$
E.$\frac{22}{-8}$
F.$\frac{11}{3}$
G.$-\frac{22}{-8}$
H.$1$
I.$11$
\testStop
\kluczStart
A
\kluczStop



\zadStart{Przykład z Wikieł P 4.2b moja wersja nr 145}
Obliczyć granicę $\lim\limits_{x\to\ 11}\frac{x^{2}-11^{2}}{(x-11)(x-4)}$.
\zadStop
\rozwStart{Patryk Wirkus}{Martyna Czarnobaj}
$$\frac{x^{2}-11^{2}}{(x-11)(x-4)}=\frac{x+11}{x-4}$$

$$\lim\limits_{x\to\ 11}\frac{x^{2}-11^{2}}{(x-11)(x-4)}=[\frac{0}{0}]=\lim\limits_{x\to\ 11}\frac{x+11}{x-4}=2 \cdot \frac{11}{11-4} = \frac{22}{7}$$
\rozwStop
\odpStart
$\frac{22}{7}$
\odpStop
\testStart
A.$\frac{22}{7}$
B.$\infty$
C.$-\infty$
D.$0$
E.$\frac{22}{-7}$
F.$\frac{11}{4}$
G.$-\frac{22}{-7}$
H.$1$
I.$11$
\testStop
\kluczStart
A
\kluczStop



\zadStart{Przykład z Wikieł P 4.2b moja wersja nr 146}
Obliczyć granicę $\lim\limits_{x\to\ 11}\frac{x^{2}-11^{2}}{(x-11)(x-5)}$.
\zadStop
\rozwStart{Patryk Wirkus}{Martyna Czarnobaj}
$$\frac{x^{2}-11^{2}}{(x-11)(x-5)}=\frac{x+11}{x-5}$$

$$\lim\limits_{x\to\ 11}\frac{x^{2}-11^{2}}{(x-11)(x-5)}=[\frac{0}{0}]=\lim\limits_{x\to\ 11}\frac{x+11}{x-5}=2 \cdot \frac{11}{11-5} = \frac{22}{6}$$
\rozwStop
\odpStart
$\frac{22}{6}$
\odpStop
\testStart
A.$\frac{22}{6}$
B.$\infty$
C.$-\infty$
D.$0$
E.$\frac{22}{-6}$
F.$\frac{11}{5}$
G.$-\frac{22}{-6}$
H.$1$
I.$11$
\testStop
\kluczStart
A
\kluczStop



\zadStart{Przykład z Wikieł P 4.2b moja wersja nr 147}
Obliczyć granicę $\lim\limits_{x\to\ 11}\frac{x^{2}-11^{2}}{(x-11)(x-6)}$.
\zadStop
\rozwStart{Patryk Wirkus}{Martyna Czarnobaj}
$$\frac{x^{2}-11^{2}}{(x-11)(x-6)}=\frac{x+11}{x-6}$$

$$\lim\limits_{x\to\ 11}\frac{x^{2}-11^{2}}{(x-11)(x-6)}=[\frac{0}{0}]=\lim\limits_{x\to\ 11}\frac{x+11}{x-6}=2 \cdot \frac{11}{11-6} = \frac{22}{5}$$
\rozwStop
\odpStart
$\frac{22}{5}$
\odpStop
\testStart
A.$\frac{22}{5}$
B.$\infty$
C.$-\infty$
D.$0$
E.$\frac{22}{-5}$
F.$\frac{11}{6}$
G.$-\frac{22}{-5}$
H.$1$
I.$11$
\testStop
\kluczStart
A
\kluczStop



\zadStart{Przykład z Wikieł P 4.2b moja wersja nr 148}
Obliczyć granicę $\lim\limits_{x\to\ 11}\frac{x^{2}-11^{2}}{(x-11)(x-7)}$.
\zadStop
\rozwStart{Patryk Wirkus}{Martyna Czarnobaj}
$$\frac{x^{2}-11^{2}}{(x-11)(x-7)}=\frac{x+11}{x-7}$$

$$\lim\limits_{x\to\ 11}\frac{x^{2}-11^{2}}{(x-11)(x-7)}=[\frac{0}{0}]=\lim\limits_{x\to\ 11}\frac{x+11}{x-7}=2 \cdot \frac{11}{11-7} = \frac{22}{4}$$
\rozwStop
\odpStart
$\frac{22}{4}$
\odpStop
\testStart
A.$\frac{22}{4}$
B.$\infty$
C.$-\infty$
D.$0$
E.$\frac{22}{-4}$
F.$\frac{11}{7}$
G.$-\frac{22}{-4}$
H.$1$
I.$11$
\testStop
\kluczStart
A
\kluczStop



\zadStart{Przykład z Wikieł P 4.2b moja wersja nr 149}
Obliczyć granicę $\lim\limits_{x\to\ 11}\frac{x^{2}-11^{2}}{(x-11)(x-8)}$.
\zadStop
\rozwStart{Patryk Wirkus}{Martyna Czarnobaj}
$$\frac{x^{2}-11^{2}}{(x-11)(x-8)}=\frac{x+11}{x-8}$$

$$\lim\limits_{x\to\ 11}\frac{x^{2}-11^{2}}{(x-11)(x-8)}=[\frac{0}{0}]=\lim\limits_{x\to\ 11}\frac{x+11}{x-8}=2 \cdot \frac{11}{11-8} = \frac{22}{3}$$
\rozwStop
\odpStart
$\frac{22}{3}$
\odpStop
\testStart
A.$\frac{22}{3}$
B.$\infty$
C.$-\infty$
D.$0$
E.$\frac{22}{-3}$
F.$\frac{11}{8}$
G.$-\frac{22}{-3}$
H.$1$
I.$11$
\testStop
\kluczStart
A
\kluczStop



\zadStart{Przykład z Wikieł P 4.2b moja wersja nr 150}
Obliczyć granicę $\lim\limits_{x\to\ 11}\frac{x^{2}-11^{2}}{(x-11)(x-9)}$.
\zadStop
\rozwStart{Patryk Wirkus}{Martyna Czarnobaj}
$$\frac{x^{2}-11^{2}}{(x-11)(x-9)}=\frac{x+11}{x-9}$$

$$\lim\limits_{x\to\ 11}\frac{x^{2}-11^{2}}{(x-11)(x-9)}=[\frac{0}{0}]=\lim\limits_{x\to\ 11}\frac{x+11}{x-9}=2 \cdot \frac{11}{11-9} = \frac{22}{2}$$
\rozwStop
\odpStart
$\frac{22}{2}$
\odpStop
\testStart
A.$\frac{22}{2}$
B.$\infty$
C.$-\infty$
D.$0$
E.$\frac{22}{-2}$
F.$\frac{11}{9}$
G.$-\frac{22}{-2}$
H.$1$
I.$11$
\testStop
\kluczStart
A
\kluczStop



\zadStart{Przykład z Wikieł P 4.2b moja wersja nr 151}
Obliczyć granicę $\lim\limits_{x\to\ 11}\frac{x^{2}-11^{2}}{(x-11)(x-13)}$.
\zadStop
\rozwStart{Patryk Wirkus}{Martyna Czarnobaj}
$$\frac{x^{2}-11^{2}}{(x-11)(x-13)}=\frac{x+11}{x-13}$$

$$\lim\limits_{x\to\ 11}\frac{x^{2}-11^{2}}{(x-11)(x-13)}=[\frac{0}{0}]=\lim\limits_{x\to\ 11}\frac{x+11}{x-13}=2 \cdot \frac{11}{11-13} = \frac{22}{-2}$$
\rozwStop
\odpStart
$\frac{22}{-2}$
\odpStop
\testStart
A.$\frac{22}{-2}$
B.$\infty$
C.$-\infty$
D.$0$
E.$\frac{22}{2}$
F.$\frac{11}{13}$
G.$-\frac{22}{2}$
H.$1$
I.$11$
\testStop
\kluczStart
A
\kluczStop



\zadStart{Przykład z Wikieł P 4.2b moja wersja nr 152}
Obliczyć granicę $\lim\limits_{x\to\ 11}\frac{x^{2}-11^{2}}{(x-11)(x-14)}$.
\zadStop
\rozwStart{Patryk Wirkus}{Martyna Czarnobaj}
$$\frac{x^{2}-11^{2}}{(x-11)(x-14)}=\frac{x+11}{x-14}$$

$$\lim\limits_{x\to\ 11}\frac{x^{2}-11^{2}}{(x-11)(x-14)}=[\frac{0}{0}]=\lim\limits_{x\to\ 11}\frac{x+11}{x-14}=2 \cdot \frac{11}{11-14} = \frac{22}{-3}$$
\rozwStop
\odpStart
$\frac{22}{-3}$
\odpStop
\testStart
A.$\frac{22}{-3}$
B.$\infty$
C.$-\infty$
D.$0$
E.$\frac{22}{3}$
F.$\frac{11}{14}$
G.$-\frac{22}{3}$
H.$1$
I.$11$
\testStop
\kluczStart
A
\kluczStop



\zadStart{Przykład z Wikieł P 4.2b moja wersja nr 153}
Obliczyć granicę $\lim\limits_{x\to\ 11}\frac{x^{2}-11^{2}}{(x-11)(x-15)}$.
\zadStop
\rozwStart{Patryk Wirkus}{Martyna Czarnobaj}
$$\frac{x^{2}-11^{2}}{(x-11)(x-15)}=\frac{x+11}{x-15}$$

$$\lim\limits_{x\to\ 11}\frac{x^{2}-11^{2}}{(x-11)(x-15)}=[\frac{0}{0}]=\lim\limits_{x\to\ 11}\frac{x+11}{x-15}=2 \cdot \frac{11}{11-15} = \frac{22}{-4}$$
\rozwStop
\odpStart
$\frac{22}{-4}$
\odpStop
\testStart
A.$\frac{22}{-4}$
B.$\infty$
C.$-\infty$
D.$0$
E.$\frac{22}{4}$
F.$\frac{11}{15}$
G.$-\frac{22}{4}$
H.$1$
I.$11$
\testStop
\kluczStart
A
\kluczStop



\zadStart{Przykład z Wikieł P 4.2b moja wersja nr 154}
Obliczyć granicę $\lim\limits_{x\to\ 11}\frac{x^{2}-11^{2}}{(x-11)(x-16)}$.
\zadStop
\rozwStart{Patryk Wirkus}{Martyna Czarnobaj}
$$\frac{x^{2}-11^{2}}{(x-11)(x-16)}=\frac{x+11}{x-16}$$

$$\lim\limits_{x\to\ 11}\frac{x^{2}-11^{2}}{(x-11)(x-16)}=[\frac{0}{0}]=\lim\limits_{x\to\ 11}\frac{x+11}{x-16}=2 \cdot \frac{11}{11-16} = \frac{22}{-5}$$
\rozwStop
\odpStart
$\frac{22}{-5}$
\odpStop
\testStart
A.$\frac{22}{-5}$
B.$\infty$
C.$-\infty$
D.$0$
E.$\frac{22}{5}$
F.$\frac{11}{16}$
G.$-\frac{22}{5}$
H.$1$
I.$11$
\testStop
\kluczStart
A
\kluczStop



\zadStart{Przykład z Wikieł P 4.2b moja wersja nr 155}
Obliczyć granicę $\lim\limits_{x\to\ 11}\frac{x^{2}-11^{2}}{(x-11)(x-17)}$.
\zadStop
\rozwStart{Patryk Wirkus}{Martyna Czarnobaj}
$$\frac{x^{2}-11^{2}}{(x-11)(x-17)}=\frac{x+11}{x-17}$$

$$\lim\limits_{x\to\ 11}\frac{x^{2}-11^{2}}{(x-11)(x-17)}=[\frac{0}{0}]=\lim\limits_{x\to\ 11}\frac{x+11}{x-17}=2 \cdot \frac{11}{11-17} = \frac{22}{-6}$$
\rozwStop
\odpStart
$\frac{22}{-6}$
\odpStop
\testStart
A.$\frac{22}{-6}$
B.$\infty$
C.$-\infty$
D.$0$
E.$\frac{22}{6}$
F.$\frac{11}{17}$
G.$-\frac{22}{6}$
H.$1$
I.$11$
\testStop
\kluczStart
A
\kluczStop



\zadStart{Przykład z Wikieł P 4.2b moja wersja nr 156}
Obliczyć granicę $\lim\limits_{x\to\ 11}\frac{x^{2}-11^{2}}{(x-11)(x-18)}$.
\zadStop
\rozwStart{Patryk Wirkus}{Martyna Czarnobaj}
$$\frac{x^{2}-11^{2}}{(x-11)(x-18)}=\frac{x+11}{x-18}$$

$$\lim\limits_{x\to\ 11}\frac{x^{2}-11^{2}}{(x-11)(x-18)}=[\frac{0}{0}]=\lim\limits_{x\to\ 11}\frac{x+11}{x-18}=2 \cdot \frac{11}{11-18} = \frac{22}{-7}$$
\rozwStop
\odpStart
$\frac{22}{-7}$
\odpStop
\testStart
A.$\frac{22}{-7}$
B.$\infty$
C.$-\infty$
D.$0$
E.$\frac{22}{7}$
F.$\frac{11}{18}$
G.$-\frac{22}{7}$
H.$1$
I.$11$
\testStop
\kluczStart
A
\kluczStop



\zadStart{Przykład z Wikieł P 4.2b moja wersja nr 157}
Obliczyć granicę $\lim\limits_{x\to\ 11}\frac{x^{2}-11^{2}}{(x-11)(x-19)}$.
\zadStop
\rozwStart{Patryk Wirkus}{Martyna Czarnobaj}
$$\frac{x^{2}-11^{2}}{(x-11)(x-19)}=\frac{x+11}{x-19}$$

$$\lim\limits_{x\to\ 11}\frac{x^{2}-11^{2}}{(x-11)(x-19)}=[\frac{0}{0}]=\lim\limits_{x\to\ 11}\frac{x+11}{x-19}=2 \cdot \frac{11}{11-19} = \frac{22}{-8}$$
\rozwStop
\odpStart
$\frac{22}{-8}$
\odpStop
\testStart
A.$\frac{22}{-8}$
B.$\infty$
C.$-\infty$
D.$0$
E.$\frac{22}{8}$
F.$\frac{11}{19}$
G.$-\frac{22}{8}$
H.$1$
I.$11$
\testStop
\kluczStart
A
\kluczStop



\zadStart{Przykład z Wikieł P 4.2b moja wersja nr 158}
Obliczyć granicę $\lim\limits_{x\to\ 11}\frac{x^{2}-11^{2}}{(x-11)(x-20)}$.
\zadStop
\rozwStart{Patryk Wirkus}{Martyna Czarnobaj}
$$\frac{x^{2}-11^{2}}{(x-11)(x-20)}=\frac{x+11}{x-20}$$

$$\lim\limits_{x\to\ 11}\frac{x^{2}-11^{2}}{(x-11)(x-20)}=[\frac{0}{0}]=\lim\limits_{x\to\ 11}\frac{x+11}{x-20}=2 \cdot \frac{11}{11-20} = \frac{22}{-9}$$
\rozwStop
\odpStart
$\frac{22}{-9}$
\odpStop
\testStart
A.$\frac{22}{-9}$
B.$\infty$
C.$-\infty$
D.$0$
E.$\frac{22}{9}$
F.$\frac{11}{20}$
G.$-\frac{22}{9}$
H.$1$
I.$11$
\testStop
\kluczStart
A
\kluczStop



\zadStart{Przykład z Wikieł P 4.2b moja wersja nr 159}
Obliczyć granicę $\lim\limits_{x\to\ 11}\frac{x^{2}-11^{2}}{(x-11)(x-21)}$.
\zadStop
\rozwStart{Patryk Wirkus}{Martyna Czarnobaj}
$$\frac{x^{2}-11^{2}}{(x-11)(x-21)}=\frac{x+11}{x-21}$$

$$\lim\limits_{x\to\ 11}\frac{x^{2}-11^{2}}{(x-11)(x-21)}=[\frac{0}{0}]=\lim\limits_{x\to\ 11}\frac{x+11}{x-21}=2 \cdot \frac{11}{11-21} = \frac{22}{-10}$$
\rozwStop
\odpStart
$\frac{22}{-10}$
\odpStop
\testStart
A.$\frac{22}{-10}$
B.$\infty$
C.$-\infty$
D.$0$
E.$\frac{22}{10}$
F.$\frac{11}{21}$
G.$-\frac{22}{10}$
H.$1$
I.$11$
\testStop
\kluczStart
A
\kluczStop



\zadStart{Przykład z Wikieł P 4.2b moja wersja nr 160}
Obliczyć granicę $\lim\limits_{x\to\ 11}\frac{x^{2}-11^{2}}{(x-11)(x-23)}$.
\zadStop
\rozwStart{Patryk Wirkus}{Martyna Czarnobaj}
$$\frac{x^{2}-11^{2}}{(x-11)(x-23)}=\frac{x+11}{x-23}$$

$$\lim\limits_{x\to\ 11}\frac{x^{2}-11^{2}}{(x-11)(x-23)}=[\frac{0}{0}]=\lim\limits_{x\to\ 11}\frac{x+11}{x-23}=2 \cdot \frac{11}{11-23} = \frac{22}{-12}$$
\rozwStop
\odpStart
$\frac{22}{-12}$
\odpStop
\testStart
A.$\frac{22}{-12}$
B.$\infty$
C.$-\infty$
D.$0$
E.$\frac{22}{12}$
F.$\frac{11}{23}$
G.$-\frac{22}{12}$
H.$1$
I.$11$
\testStop
\kluczStart
A
\kluczStop



\zadStart{Przykład z Wikieł P 4.2b moja wersja nr 161}
Obliczyć granicę $\lim\limits_{x\to\ 11}\frac{x^{2}-11^{2}}{(x-11)(x-24)}$.
\zadStop
\rozwStart{Patryk Wirkus}{Martyna Czarnobaj}
$$\frac{x^{2}-11^{2}}{(x-11)(x-24)}=\frac{x+11}{x-24}$$

$$\lim\limits_{x\to\ 11}\frac{x^{2}-11^{2}}{(x-11)(x-24)}=[\frac{0}{0}]=\lim\limits_{x\to\ 11}\frac{x+11}{x-24}=2 \cdot \frac{11}{11-24} = \frac{22}{-13}$$
\rozwStop
\odpStart
$\frac{22}{-13}$
\odpStop
\testStart
A.$\frac{22}{-13}$
B.$\infty$
C.$-\infty$
D.$0$
E.$\frac{22}{13}$
F.$\frac{11}{24}$
G.$-\frac{22}{13}$
H.$1$
I.$11$
\testStop
\kluczStart
A
\kluczStop



\zadStart{Przykład z Wikieł P 4.2b moja wersja nr 162}
Obliczyć granicę $\lim\limits_{x\to\ 11}\frac{x^{2}-11^{2}}{(x-11)(x-25)}$.
\zadStop
\rozwStart{Patryk Wirkus}{Martyna Czarnobaj}
$$\frac{x^{2}-11^{2}}{(x-11)(x-25)}=\frac{x+11}{x-25}$$

$$\lim\limits_{x\to\ 11}\frac{x^{2}-11^{2}}{(x-11)(x-25)}=[\frac{0}{0}]=\lim\limits_{x\to\ 11}\frac{x+11}{x-25}=2 \cdot \frac{11}{11-25} = \frac{22}{-14}$$
\rozwStop
\odpStart
$\frac{22}{-14}$
\odpStop
\testStart
A.$\frac{22}{-14}$
B.$\infty$
C.$-\infty$
D.$0$
E.$\frac{22}{14}$
F.$\frac{11}{25}$
G.$-\frac{22}{14}$
H.$1$
I.$11$
\testStop
\kluczStart
A
\kluczStop



\zadStart{Przykład z Wikieł P 4.2b moja wersja nr 163}
Obliczyć granicę $\lim\limits_{x\to\ 11}\frac{x^{2}-11^{2}}{(x-11)(x-26)}$.
\zadStop
\rozwStart{Patryk Wirkus}{Martyna Czarnobaj}
$$\frac{x^{2}-11^{2}}{(x-11)(x-26)}=\frac{x+11}{x-26}$$

$$\lim\limits_{x\to\ 11}\frac{x^{2}-11^{2}}{(x-11)(x-26)}=[\frac{0}{0}]=\lim\limits_{x\to\ 11}\frac{x+11}{x-26}=2 \cdot \frac{11}{11-26} = \frac{22}{-15}$$
\rozwStop
\odpStart
$\frac{22}{-15}$
\odpStop
\testStart
A.$\frac{22}{-15}$
B.$\infty$
C.$-\infty$
D.$0$
E.$\frac{22}{15}$
F.$\frac{11}{26}$
G.$-\frac{22}{15}$
H.$1$
I.$11$
\testStop
\kluczStart
A
\kluczStop



\zadStart{Przykład z Wikieł P 4.2b moja wersja nr 164}
Obliczyć granicę $\lim\limits_{x\to\ 11}\frac{x^{2}-11^{2}}{(x-11)(x-27)}$.
\zadStop
\rozwStart{Patryk Wirkus}{Martyna Czarnobaj}
$$\frac{x^{2}-11^{2}}{(x-11)(x-27)}=\frac{x+11}{x-27}$$

$$\lim\limits_{x\to\ 11}\frac{x^{2}-11^{2}}{(x-11)(x-27)}=[\frac{0}{0}]=\lim\limits_{x\to\ 11}\frac{x+11}{x-27}=2 \cdot \frac{11}{11-27} = \frac{22}{-16}$$
\rozwStop
\odpStart
$\frac{22}{-16}$
\odpStop
\testStart
A.$\frac{22}{-16}$
B.$\infty$
C.$-\infty$
D.$0$
E.$\frac{22}{16}$
F.$\frac{11}{27}$
G.$-\frac{22}{16}$
H.$1$
I.$11$
\testStop
\kluczStart
A
\kluczStop



\zadStart{Przykład z Wikieł P 4.2b moja wersja nr 165}
Obliczyć granicę $\lim\limits_{x\to\ 11}\frac{x^{2}-11^{2}}{(x-11)(x-28)}$.
\zadStop
\rozwStart{Patryk Wirkus}{Martyna Czarnobaj}
$$\frac{x^{2}-11^{2}}{(x-11)(x-28)}=\frac{x+11}{x-28}$$

$$\lim\limits_{x\to\ 11}\frac{x^{2}-11^{2}}{(x-11)(x-28)}=[\frac{0}{0}]=\lim\limits_{x\to\ 11}\frac{x+11}{x-28}=2 \cdot \frac{11}{11-28} = \frac{22}{-17}$$
\rozwStop
\odpStart
$\frac{22}{-17}$
\odpStop
\testStart
A.$\frac{22}{-17}$
B.$\infty$
C.$-\infty$
D.$0$
E.$\frac{22}{17}$
F.$\frac{11}{28}$
G.$-\frac{22}{17}$
H.$1$
I.$11$
\testStop
\kluczStart
A
\kluczStop



\zadStart{Przykład z Wikieł P 4.2b moja wersja nr 166}
Obliczyć granicę $\lim\limits_{x\to\ 11}\frac{x^{2}-11^{2}}{(x-11)(x-29)}$.
\zadStop
\rozwStart{Patryk Wirkus}{Martyna Czarnobaj}
$$\frac{x^{2}-11^{2}}{(x-11)(x-29)}=\frac{x+11}{x-29}$$

$$\lim\limits_{x\to\ 11}\frac{x^{2}-11^{2}}{(x-11)(x-29)}=[\frac{0}{0}]=\lim\limits_{x\to\ 11}\frac{x+11}{x-29}=2 \cdot \frac{11}{11-29} = \frac{22}{-18}$$
\rozwStop
\odpStart
$\frac{22}{-18}$
\odpStop
\testStart
A.$\frac{22}{-18}$
B.$\infty$
C.$-\infty$
D.$0$
E.$\frac{22}{18}$
F.$\frac{11}{29}$
G.$-\frac{22}{18}$
H.$1$
I.$11$
\testStop
\kluczStart
A
\kluczStop



\zadStart{Przykład z Wikieł P 4.2b moja wersja nr 167}
Obliczyć granicę $\lim\limits_{x\to\ 11}\frac{x^{2}-11^{2}}{(x-11)(x-30)}$.
\zadStop
\rozwStart{Patryk Wirkus}{Martyna Czarnobaj}
$$\frac{x^{2}-11^{2}}{(x-11)(x-30)}=\frac{x+11}{x-30}$$

$$\lim\limits_{x\to\ 11}\frac{x^{2}-11^{2}}{(x-11)(x-30)}=[\frac{0}{0}]=\lim\limits_{x\to\ 11}\frac{x+11}{x-30}=2 \cdot \frac{11}{11-30} = \frac{22}{-19}$$
\rozwStop
\odpStart
$\frac{22}{-19}$
\odpStop
\testStart
A.$\frac{22}{-19}$
B.$\infty$
C.$-\infty$
D.$0$
E.$\frac{22}{19}$
F.$\frac{11}{30}$
G.$-\frac{22}{19}$
H.$1$
I.$11$
\testStop
\kluczStart
A
\kluczStop



\zadStart{Przykład z Wikieł P 4.2b moja wersja nr 168}
Obliczyć granicę $\lim\limits_{x\to\ 11}\frac{x^{2}-11^{2}}{(x-11)(x-31)}$.
\zadStop
\rozwStart{Patryk Wirkus}{Martyna Czarnobaj}
$$\frac{x^{2}-11^{2}}{(x-11)(x-31)}=\frac{x+11}{x-31}$$

$$\lim\limits_{x\to\ 11}\frac{x^{2}-11^{2}}{(x-11)(x-31)}=[\frac{0}{0}]=\lim\limits_{x\to\ 11}\frac{x+11}{x-31}=2 \cdot \frac{11}{11-31} = \frac{22}{-20}$$
\rozwStop
\odpStart
$\frac{22}{-20}$
\odpStop
\testStart
A.$\frac{22}{-20}$
B.$\infty$
C.$-\infty$
D.$0$
E.$\frac{22}{20}$
F.$\frac{11}{31}$
G.$-\frac{22}{20}$
H.$1$
I.$11$
\testStop
\kluczStart
A
\kluczStop



\zadStart{Przykład z Wikieł P 4.2b moja wersja nr 169}
Obliczyć granicę $\lim\limits_{x\to\ 11}\frac{x^{2}-11^{2}}{(x-11)(x-32)}$.
\zadStop
\rozwStart{Patryk Wirkus}{Martyna Czarnobaj}
$$\frac{x^{2}-11^{2}}{(x-11)(x-32)}=\frac{x+11}{x-32}$$

$$\lim\limits_{x\to\ 11}\frac{x^{2}-11^{2}}{(x-11)(x-32)}=[\frac{0}{0}]=\lim\limits_{x\to\ 11}\frac{x+11}{x-32}=2 \cdot \frac{11}{11-32} = \frac{22}{-21}$$
\rozwStop
\odpStart
$\frac{22}{-21}$
\odpStop
\testStart
A.$\frac{22}{-21}$
B.$\infty$
C.$-\infty$
D.$0$
E.$\frac{22}{21}$
F.$\frac{11}{32}$
G.$-\frac{22}{21}$
H.$1$
I.$11$
\testStop
\kluczStart
A
\kluczStop



\zadStart{Przykład z Wikieł P 4.2b moja wersja nr 170}
Obliczyć granicę $\lim\limits_{x\to\ 11}\frac{x^{2}-11^{2}}{(x-11)(x-34)}$.
\zadStop
\rozwStart{Patryk Wirkus}{Martyna Czarnobaj}
$$\frac{x^{2}-11^{2}}{(x-11)(x-34)}=\frac{x+11}{x-34}$$

$$\lim\limits_{x\to\ 11}\frac{x^{2}-11^{2}}{(x-11)(x-34)}=[\frac{0}{0}]=\lim\limits_{x\to\ 11}\frac{x+11}{x-34}=2 \cdot \frac{11}{11-34} = \frac{22}{-23}$$
\rozwStop
\odpStart
$\frac{22}{-23}$
\odpStop
\testStart
A.$\frac{22}{-23}$
B.$\infty$
C.$-\infty$
D.$0$
E.$\frac{22}{23}$
F.$\frac{11}{34}$
G.$-\frac{22}{23}$
H.$1$
I.$11$
\testStop
\kluczStart
A
\kluczStop



\zadStart{Przykład z Wikieł P 4.2b moja wersja nr 171}
Obliczyć granicę $\lim\limits_{x\to\ 11}\frac{x^{2}-11^{2}}{(x-11)(x-35)}$.
\zadStop
\rozwStart{Patryk Wirkus}{Martyna Czarnobaj}
$$\frac{x^{2}-11^{2}}{(x-11)(x-35)}=\frac{x+11}{x-35}$$

$$\lim\limits_{x\to\ 11}\frac{x^{2}-11^{2}}{(x-11)(x-35)}=[\frac{0}{0}]=\lim\limits_{x\to\ 11}\frac{x+11}{x-35}=2 \cdot \frac{11}{11-35} = \frac{22}{-24}$$
\rozwStop
\odpStart
$\frac{22}{-24}$
\odpStop
\testStart
A.$\frac{22}{-24}$
B.$\infty$
C.$-\infty$
D.$0$
E.$\frac{22}{24}$
F.$\frac{11}{35}$
G.$-\frac{22}{24}$
H.$1$
I.$11$
\testStop
\kluczStart
A
\kluczStop



\zadStart{Przykład z Wikieł P 4.2b moja wersja nr 172}
Obliczyć granicę $\lim\limits_{x\to\ 11}\frac{x^{2}-11^{2}}{(x-11)(x-36)}$.
\zadStop
\rozwStart{Patryk Wirkus}{Martyna Czarnobaj}
$$\frac{x^{2}-11^{2}}{(x-11)(x-36)}=\frac{x+11}{x-36}$$

$$\lim\limits_{x\to\ 11}\frac{x^{2}-11^{2}}{(x-11)(x-36)}=[\frac{0}{0}]=\lim\limits_{x\to\ 11}\frac{x+11}{x-36}=2 \cdot \frac{11}{11-36} = \frac{22}{-25}$$
\rozwStop
\odpStart
$\frac{22}{-25}$
\odpStop
\testStart
A.$\frac{22}{-25}$
B.$\infty$
C.$-\infty$
D.$0$
E.$\frac{22}{25}$
F.$\frac{11}{36}$
G.$-\frac{22}{25}$
H.$1$
I.$11$
\testStop
\kluczStart
A
\kluczStop



\zadStart{Przykład z Wikieł P 4.2b moja wersja nr 173}
Obliczyć granicę $\lim\limits_{x\to\ 11}\frac{x^{2}-11^{2}}{(x-11)(x-37)}$.
\zadStop
\rozwStart{Patryk Wirkus}{Martyna Czarnobaj}
$$\frac{x^{2}-11^{2}}{(x-11)(x-37)}=\frac{x+11}{x-37}$$

$$\lim\limits_{x\to\ 11}\frac{x^{2}-11^{2}}{(x-11)(x-37)}=[\frac{0}{0}]=\lim\limits_{x\to\ 11}\frac{x+11}{x-37}=2 \cdot \frac{11}{11-37} = \frac{22}{-26}$$
\rozwStop
\odpStart
$\frac{22}{-26}$
\odpStop
\testStart
A.$\frac{22}{-26}$
B.$\infty$
C.$-\infty$
D.$0$
E.$\frac{22}{26}$
F.$\frac{11}{37}$
G.$-\frac{22}{26}$
H.$1$
I.$11$
\testStop
\kluczStart
A
\kluczStop



\zadStart{Przykład z Wikieł P 4.2b moja wersja nr 174}
Obliczyć granicę $\lim\limits_{x\to\ 11}\frac{x^{2}-11^{2}}{(x-11)(x-38)}$.
\zadStop
\rozwStart{Patryk Wirkus}{Martyna Czarnobaj}
$$\frac{x^{2}-11^{2}}{(x-11)(x-38)}=\frac{x+11}{x-38}$$

$$\lim\limits_{x\to\ 11}\frac{x^{2}-11^{2}}{(x-11)(x-38)}=[\frac{0}{0}]=\lim\limits_{x\to\ 11}\frac{x+11}{x-38}=2 \cdot \frac{11}{11-38} = \frac{22}{-27}$$
\rozwStop
\odpStart
$\frac{22}{-27}$
\odpStop
\testStart
A.$\frac{22}{-27}$
B.$\infty$
C.$-\infty$
D.$0$
E.$\frac{22}{27}$
F.$\frac{11}{38}$
G.$-\frac{22}{27}$
H.$1$
I.$11$
\testStop
\kluczStart
A
\kluczStop



\zadStart{Przykład z Wikieł P 4.2b moja wersja nr 175}
Obliczyć granicę $\lim\limits_{x\to\ 11}\frac{x^{2}-11^{2}}{(x-11)(x-39)}$.
\zadStop
\rozwStart{Patryk Wirkus}{Martyna Czarnobaj}
$$\frac{x^{2}-11^{2}}{(x-11)(x-39)}=\frac{x+11}{x-39}$$

$$\lim\limits_{x\to\ 11}\frac{x^{2}-11^{2}}{(x-11)(x-39)}=[\frac{0}{0}]=\lim\limits_{x\to\ 11}\frac{x+11}{x-39}=2 \cdot \frac{11}{11-39} = \frac{22}{-28}$$
\rozwStop
\odpStart
$\frac{22}{-28}$
\odpStop
\testStart
A.$\frac{22}{-28}$
B.$\infty$
C.$-\infty$
D.$0$
E.$\frac{22}{28}$
F.$\frac{11}{39}$
G.$-\frac{22}{28}$
H.$1$
I.$11$
\testStop
\kluczStart
A
\kluczStop



\zadStart{Przykład z Wikieł P 4.2b moja wersja nr 176}
Obliczyć granicę $\lim\limits_{x\to\ 11}\frac{x^{2}-11^{2}}{(x-11)(x-40)}$.
\zadStop
\rozwStart{Patryk Wirkus}{Martyna Czarnobaj}
$$\frac{x^{2}-11^{2}}{(x-11)(x-40)}=\frac{x+11}{x-40}$$

$$\lim\limits_{x\to\ 11}\frac{x^{2}-11^{2}}{(x-11)(x-40)}=[\frac{0}{0}]=\lim\limits_{x\to\ 11}\frac{x+11}{x-40}=2 \cdot \frac{11}{11-40} = \frac{22}{-29}$$
\rozwStop
\odpStart
$\frac{22}{-29}$
\odpStop
\testStart
A.$\frac{22}{-29}$
B.$\infty$
C.$-\infty$
D.$0$
E.$\frac{22}{29}$
F.$\frac{11}{40}$
G.$-\frac{22}{29}$
H.$1$
I.$11$
\testStop
\kluczStart
A
\kluczStop



\zadStart{Przykład z Wikieł P 4.2b moja wersja nr 177}
Obliczyć granicę $\lim\limits_{x\to\ 12}\frac{x^{2}-12^{2}}{(x-12)(x-5)}$.
\zadStop
\rozwStart{Patryk Wirkus}{Martyna Czarnobaj}
$$\frac{x^{2}-12^{2}}{(x-12)(x-5)}=\frac{x+12}{x-5}$$

$$\lim\limits_{x\to\ 12}\frac{x^{2}-12^{2}}{(x-12)(x-5)}=[\frac{0}{0}]=\lim\limits_{x\to\ 12}\frac{x+12}{x-5}=2 \cdot \frac{12}{12-5} = \frac{24}{7}$$
\rozwStop
\odpStart
$\frac{24}{7}$
\odpStop
\testStart
A.$\frac{24}{7}$
B.$\infty$
C.$-\infty$
D.$0$
E.$\frac{24}{-7}$
F.$\frac{12}{5}$
G.$-\frac{24}{-7}$
H.$1$
I.$12$
\testStop
\kluczStart
A
\kluczStop



\zadStart{Przykład z Wikieł P 4.2b moja wersja nr 178}
Obliczyć granicę $\lim\limits_{x\to\ 12}\frac{x^{2}-12^{2}}{(x-12)(x-7)}$.
\zadStop
\rozwStart{Patryk Wirkus}{Martyna Czarnobaj}
$$\frac{x^{2}-12^{2}}{(x-12)(x-7)}=\frac{x+12}{x-7}$$

$$\lim\limits_{x\to\ 12}\frac{x^{2}-12^{2}}{(x-12)(x-7)}=[\frac{0}{0}]=\lim\limits_{x\to\ 12}\frac{x+12}{x-7}=2 \cdot \frac{12}{12-7} = \frac{24}{5}$$
\rozwStop
\odpStart
$\frac{24}{5}$
\odpStop
\testStart
A.$\frac{24}{5}$
B.$\infty$
C.$-\infty$
D.$0$
E.$\frac{24}{-5}$
F.$\frac{12}{7}$
G.$-\frac{24}{-5}$
H.$1$
I.$12$
\testStop
\kluczStart
A
\kluczStop



\zadStart{Przykład z Wikieł P 4.2b moja wersja nr 179}
Obliczyć granicę $\lim\limits_{x\to\ 12}\frac{x^{2}-12^{2}}{(x-12)(x-17)}$.
\zadStop
\rozwStart{Patryk Wirkus}{Martyna Czarnobaj}
$$\frac{x^{2}-12^{2}}{(x-12)(x-17)}=\frac{x+12}{x-17}$$

$$\lim\limits_{x\to\ 12}\frac{x^{2}-12^{2}}{(x-12)(x-17)}=[\frac{0}{0}]=\lim\limits_{x\to\ 12}\frac{x+12}{x-17}=2 \cdot \frac{12}{12-17} = \frac{24}{-5}$$
\rozwStop
\odpStart
$\frac{24}{-5}$
\odpStop
\testStart
A.$\frac{24}{-5}$
B.$\infty$
C.$-\infty$
D.$0$
E.$\frac{24}{5}$
F.$\frac{12}{17}$
G.$-\frac{24}{5}$
H.$1$
I.$12$
\testStop
\kluczStart
A
\kluczStop



\zadStart{Przykład z Wikieł P 4.2b moja wersja nr 180}
Obliczyć granicę $\lim\limits_{x\to\ 12}\frac{x^{2}-12^{2}}{(x-12)(x-19)}$.
\zadStop
\rozwStart{Patryk Wirkus}{Martyna Czarnobaj}
$$\frac{x^{2}-12^{2}}{(x-12)(x-19)}=\frac{x+12}{x-19}$$

$$\lim\limits_{x\to\ 12}\frac{x^{2}-12^{2}}{(x-12)(x-19)}=[\frac{0}{0}]=\lim\limits_{x\to\ 12}\frac{x+12}{x-19}=2 \cdot \frac{12}{12-19} = \frac{24}{-7}$$
\rozwStop
\odpStart
$\frac{24}{-7}$
\odpStop
\testStart
A.$\frac{24}{-7}$
B.$\infty$
C.$-\infty$
D.$0$
E.$\frac{24}{7}$
F.$\frac{12}{19}$
G.$-\frac{24}{7}$
H.$1$
I.$12$
\testStop
\kluczStart
A
\kluczStop



\zadStart{Przykład z Wikieł P 4.2b moja wersja nr 181}
Obliczyć granicę $\lim\limits_{x\to\ 12}\frac{x^{2}-12^{2}}{(x-12)(x-23)}$.
\zadStop
\rozwStart{Patryk Wirkus}{Martyna Czarnobaj}
$$\frac{x^{2}-12^{2}}{(x-12)(x-23)}=\frac{x+12}{x-23}$$

$$\lim\limits_{x\to\ 12}\frac{x^{2}-12^{2}}{(x-12)(x-23)}=[\frac{0}{0}]=\lim\limits_{x\to\ 12}\frac{x+12}{x-23}=2 \cdot \frac{12}{12-23} = \frac{24}{-11}$$
\rozwStop
\odpStart
$\frac{24}{-11}$
\odpStop
\testStart
A.$\frac{24}{-11}$
B.$\infty$
C.$-\infty$
D.$0$
E.$\frac{24}{11}$
F.$\frac{12}{23}$
G.$-\frac{24}{11}$
H.$1$
I.$12$
\testStop
\kluczStart
A
\kluczStop



\zadStart{Przykład z Wikieł P 4.2b moja wersja nr 182}
Obliczyć granicę $\lim\limits_{x\to\ 12}\frac{x^{2}-12^{2}}{(x-12)(x-25)}$.
\zadStop
\rozwStart{Patryk Wirkus}{Martyna Czarnobaj}
$$\frac{x^{2}-12^{2}}{(x-12)(x-25)}=\frac{x+12}{x-25}$$

$$\lim\limits_{x\to\ 12}\frac{x^{2}-12^{2}}{(x-12)(x-25)}=[\frac{0}{0}]=\lim\limits_{x\to\ 12}\frac{x+12}{x-25}=2 \cdot \frac{12}{12-25} = \frac{24}{-13}$$
\rozwStop
\odpStart
$\frac{24}{-13}$
\odpStop
\testStart
A.$\frac{24}{-13}$
B.$\infty$
C.$-\infty$
D.$0$
E.$\frac{24}{13}$
F.$\frac{12}{25}$
G.$-\frac{24}{13}$
H.$1$
I.$12$
\testStop
\kluczStart
A
\kluczStop



\zadStart{Przykład z Wikieł P 4.2b moja wersja nr 183}
Obliczyć granicę $\lim\limits_{x\to\ 12}\frac{x^{2}-12^{2}}{(x-12)(x-29)}$.
\zadStop
\rozwStart{Patryk Wirkus}{Martyna Czarnobaj}
$$\frac{x^{2}-12^{2}}{(x-12)(x-29)}=\frac{x+12}{x-29}$$

$$\lim\limits_{x\to\ 12}\frac{x^{2}-12^{2}}{(x-12)(x-29)}=[\frac{0}{0}]=\lim\limits_{x\to\ 12}\frac{x+12}{x-29}=2 \cdot \frac{12}{12-29} = \frac{24}{-17}$$
\rozwStop
\odpStart
$\frac{24}{-17}$
\odpStop
\testStart
A.$\frac{24}{-17}$
B.$\infty$
C.$-\infty$
D.$0$
E.$\frac{24}{17}$
F.$\frac{12}{29}$
G.$-\frac{24}{17}$
H.$1$
I.$12$
\testStop
\kluczStart
A
\kluczStop



\zadStart{Przykład z Wikieł P 4.2b moja wersja nr 184}
Obliczyć granicę $\lim\limits_{x\to\ 12}\frac{x^{2}-12^{2}}{(x-12)(x-31)}$.
\zadStop
\rozwStart{Patryk Wirkus}{Martyna Czarnobaj}
$$\frac{x^{2}-12^{2}}{(x-12)(x-31)}=\frac{x+12}{x-31}$$

$$\lim\limits_{x\to\ 12}\frac{x^{2}-12^{2}}{(x-12)(x-31)}=[\frac{0}{0}]=\lim\limits_{x\to\ 12}\frac{x+12}{x-31}=2 \cdot \frac{12}{12-31} = \frac{24}{-19}$$
\rozwStop
\odpStart
$\frac{24}{-19}$
\odpStop
\testStart
A.$\frac{24}{-19}$
B.$\infty$
C.$-\infty$
D.$0$
E.$\frac{24}{19}$
F.$\frac{12}{31}$
G.$-\frac{24}{19}$
H.$1$
I.$12$
\testStop
\kluczStart
A
\kluczStop



\zadStart{Przykład z Wikieł P 4.2b moja wersja nr 185}
Obliczyć granicę $\lim\limits_{x\to\ 12}\frac{x^{2}-12^{2}}{(x-12)(x-35)}$.
\zadStop
\rozwStart{Patryk Wirkus}{Martyna Czarnobaj}
$$\frac{x^{2}-12^{2}}{(x-12)(x-35)}=\frac{x+12}{x-35}$$

$$\lim\limits_{x\to\ 12}\frac{x^{2}-12^{2}}{(x-12)(x-35)}=[\frac{0}{0}]=\lim\limits_{x\to\ 12}\frac{x+12}{x-35}=2 \cdot \frac{12}{12-35} = \frac{24}{-23}$$
\rozwStop
\odpStart
$\frac{24}{-23}$
\odpStop
\testStart
A.$\frac{24}{-23}$
B.$\infty$
C.$-\infty$
D.$0$
E.$\frac{24}{23}$
F.$\frac{12}{35}$
G.$-\frac{24}{23}$
H.$1$
I.$12$
\testStop
\kluczStart
A
\kluczStop



\zadStart{Przykład z Wikieł P 4.2b moja wersja nr 186}
Obliczyć granicę $\lim\limits_{x\to\ 12}\frac{x^{2}-12^{2}}{(x-12)(x-37)}$.
\zadStop
\rozwStart{Patryk Wirkus}{Martyna Czarnobaj}
$$\frac{x^{2}-12^{2}}{(x-12)(x-37)}=\frac{x+12}{x-37}$$

$$\lim\limits_{x\to\ 12}\frac{x^{2}-12^{2}}{(x-12)(x-37)}=[\frac{0}{0}]=\lim\limits_{x\to\ 12}\frac{x+12}{x-37}=2 \cdot \frac{12}{12-37} = \frac{24}{-25}$$
\rozwStop
\odpStart
$\frac{24}{-25}$
\odpStop
\testStart
A.$\frac{24}{-25}$
B.$\infty$
C.$-\infty$
D.$0$
E.$\frac{24}{25}$
F.$\frac{12}{37}$
G.$-\frac{24}{25}$
H.$1$
I.$12$
\testStop
\kluczStart
A
\kluczStop



\zadStart{Przykład z Wikieł P 4.2b moja wersja nr 187}
Obliczyć granicę $\lim\limits_{x\to\ 13}\frac{x^{2}-13^{2}}{(x-13)(x-2)}$.
\zadStop
\rozwStart{Patryk Wirkus}{Martyna Czarnobaj}
$$\frac{x^{2}-13^{2}}{(x-13)(x-2)}=\frac{x+13}{x-2}$$

$$\lim\limits_{x\to\ 13}\frac{x^{2}-13^{2}}{(x-13)(x-2)}=[\frac{0}{0}]=\lim\limits_{x\to\ 13}\frac{x+13}{x-2}=2 \cdot \frac{13}{13-2} = \frac{26}{11}$$
\rozwStop
\odpStart
$\frac{26}{11}$
\odpStop
\testStart
A.$\frac{26}{11}$
B.$\infty$
C.$-\infty$
D.$0$
E.$\frac{26}{-11}$
F.$\frac{13}{2}$
G.$-\frac{26}{-11}$
H.$1$
I.$13$
\testStop
\kluczStart
A
\kluczStop



\zadStart{Przykład z Wikieł P 4.2b moja wersja nr 188}
Obliczyć granicę $\lim\limits_{x\to\ 13}\frac{x^{2}-13^{2}}{(x-13)(x-3)}$.
\zadStop
\rozwStart{Patryk Wirkus}{Martyna Czarnobaj}
$$\frac{x^{2}-13^{2}}{(x-13)(x-3)}=\frac{x+13}{x-3}$$

$$\lim\limits_{x\to\ 13}\frac{x^{2}-13^{2}}{(x-13)(x-3)}=[\frac{0}{0}]=\lim\limits_{x\to\ 13}\frac{x+13}{x-3}=2 \cdot \frac{13}{13-3} = \frac{26}{10}$$
\rozwStop
\odpStart
$\frac{26}{10}$
\odpStop
\testStart
A.$\frac{26}{10}$
B.$\infty$
C.$-\infty$
D.$0$
E.$\frac{26}{-10}$
F.$\frac{13}{3}$
G.$-\frac{26}{-10}$
H.$1$
I.$13$
\testStop
\kluczStart
A
\kluczStop



\zadStart{Przykład z Wikieł P 4.2b moja wersja nr 189}
Obliczyć granicę $\lim\limits_{x\to\ 13}\frac{x^{2}-13^{2}}{(x-13)(x-4)}$.
\zadStop
\rozwStart{Patryk Wirkus}{Martyna Czarnobaj}
$$\frac{x^{2}-13^{2}}{(x-13)(x-4)}=\frac{x+13}{x-4}$$

$$\lim\limits_{x\to\ 13}\frac{x^{2}-13^{2}}{(x-13)(x-4)}=[\frac{0}{0}]=\lim\limits_{x\to\ 13}\frac{x+13}{x-4}=2 \cdot \frac{13}{13-4} = \frac{26}{9}$$
\rozwStop
\odpStart
$\frac{26}{9}$
\odpStop
\testStart
A.$\frac{26}{9}$
B.$\infty$
C.$-\infty$
D.$0$
E.$\frac{26}{-9}$
F.$\frac{13}{4}$
G.$-\frac{26}{-9}$
H.$1$
I.$13$
\testStop
\kluczStart
A
\kluczStop



\zadStart{Przykład z Wikieł P 4.2b moja wersja nr 190}
Obliczyć granicę $\lim\limits_{x\to\ 13}\frac{x^{2}-13^{2}}{(x-13)(x-5)}$.
\zadStop
\rozwStart{Patryk Wirkus}{Martyna Czarnobaj}
$$\frac{x^{2}-13^{2}}{(x-13)(x-5)}=\frac{x+13}{x-5}$$

$$\lim\limits_{x\to\ 13}\frac{x^{2}-13^{2}}{(x-13)(x-5)}=[\frac{0}{0}]=\lim\limits_{x\to\ 13}\frac{x+13}{x-5}=2 \cdot \frac{13}{13-5} = \frac{26}{8}$$
\rozwStop
\odpStart
$\frac{26}{8}$
\odpStop
\testStart
A.$\frac{26}{8}$
B.$\infty$
C.$-\infty$
D.$0$
E.$\frac{26}{-8}$
F.$\frac{13}{5}$
G.$-\frac{26}{-8}$
H.$1$
I.$13$
\testStop
\kluczStart
A
\kluczStop



\zadStart{Przykład z Wikieł P 4.2b moja wersja nr 191}
Obliczyć granicę $\lim\limits_{x\to\ 13}\frac{x^{2}-13^{2}}{(x-13)(x-6)}$.
\zadStop
\rozwStart{Patryk Wirkus}{Martyna Czarnobaj}
$$\frac{x^{2}-13^{2}}{(x-13)(x-6)}=\frac{x+13}{x-6}$$

$$\lim\limits_{x\to\ 13}\frac{x^{2}-13^{2}}{(x-13)(x-6)}=[\frac{0}{0}]=\lim\limits_{x\to\ 13}\frac{x+13}{x-6}=2 \cdot \frac{13}{13-6} = \frac{26}{7}$$
\rozwStop
\odpStart
$\frac{26}{7}$
\odpStop
\testStart
A.$\frac{26}{7}$
B.$\infty$
C.$-\infty$
D.$0$
E.$\frac{26}{-7}$
F.$\frac{13}{6}$
G.$-\frac{26}{-7}$
H.$1$
I.$13$
\testStop
\kluczStart
A
\kluczStop



\zadStart{Przykład z Wikieł P 4.2b moja wersja nr 192}
Obliczyć granicę $\lim\limits_{x\to\ 13}\frac{x^{2}-13^{2}}{(x-13)(x-7)}$.
\zadStop
\rozwStart{Patryk Wirkus}{Martyna Czarnobaj}
$$\frac{x^{2}-13^{2}}{(x-13)(x-7)}=\frac{x+13}{x-7}$$

$$\lim\limits_{x\to\ 13}\frac{x^{2}-13^{2}}{(x-13)(x-7)}=[\frac{0}{0}]=\lim\limits_{x\to\ 13}\frac{x+13}{x-7}=2 \cdot \frac{13}{13-7} = \frac{26}{6}$$
\rozwStop
\odpStart
$\frac{26}{6}$
\odpStop
\testStart
A.$\frac{26}{6}$
B.$\infty$
C.$-\infty$
D.$0$
E.$\frac{26}{-6}$
F.$\frac{13}{7}$
G.$-\frac{26}{-6}$
H.$1$
I.$13$
\testStop
\kluczStart
A
\kluczStop



\zadStart{Przykład z Wikieł P 4.2b moja wersja nr 193}
Obliczyć granicę $\lim\limits_{x\to\ 13}\frac{x^{2}-13^{2}}{(x-13)(x-8)}$.
\zadStop
\rozwStart{Patryk Wirkus}{Martyna Czarnobaj}
$$\frac{x^{2}-13^{2}}{(x-13)(x-8)}=\frac{x+13}{x-8}$$

$$\lim\limits_{x\to\ 13}\frac{x^{2}-13^{2}}{(x-13)(x-8)}=[\frac{0}{0}]=\lim\limits_{x\to\ 13}\frac{x+13}{x-8}=2 \cdot \frac{13}{13-8} = \frac{26}{5}$$
\rozwStop
\odpStart
$\frac{26}{5}$
\odpStop
\testStart
A.$\frac{26}{5}$
B.$\infty$
C.$-\infty$
D.$0$
E.$\frac{26}{-5}$
F.$\frac{13}{8}$
G.$-\frac{26}{-5}$
H.$1$
I.$13$
\testStop
\kluczStart
A
\kluczStop



\zadStart{Przykład z Wikieł P 4.2b moja wersja nr 194}
Obliczyć granicę $\lim\limits_{x\to\ 13}\frac{x^{2}-13^{2}}{(x-13)(x-9)}$.
\zadStop
\rozwStart{Patryk Wirkus}{Martyna Czarnobaj}
$$\frac{x^{2}-13^{2}}{(x-13)(x-9)}=\frac{x+13}{x-9}$$

$$\lim\limits_{x\to\ 13}\frac{x^{2}-13^{2}}{(x-13)(x-9)}=[\frac{0}{0}]=\lim\limits_{x\to\ 13}\frac{x+13}{x-9}=2 \cdot \frac{13}{13-9} = \frac{26}{4}$$
\rozwStop
\odpStart
$\frac{26}{4}$
\odpStop
\testStart
A.$\frac{26}{4}$
B.$\infty$
C.$-\infty$
D.$0$
E.$\frac{26}{-4}$
F.$\frac{13}{9}$
G.$-\frac{26}{-4}$
H.$1$
I.$13$
\testStop
\kluczStart
A
\kluczStop



\zadStart{Przykład z Wikieł P 4.2b moja wersja nr 195}
Obliczyć granicę $\lim\limits_{x\to\ 13}\frac{x^{2}-13^{2}}{(x-13)(x-10)}$.
\zadStop
\rozwStart{Patryk Wirkus}{Martyna Czarnobaj}
$$\frac{x^{2}-13^{2}}{(x-13)(x-10)}=\frac{x+13}{x-10}$$

$$\lim\limits_{x\to\ 13}\frac{x^{2}-13^{2}}{(x-13)(x-10)}=[\frac{0}{0}]=\lim\limits_{x\to\ 13}\frac{x+13}{x-10}=2 \cdot \frac{13}{13-10} = \frac{26}{3}$$
\rozwStop
\odpStart
$\frac{26}{3}$
\odpStop
\testStart
A.$\frac{26}{3}$
B.$\infty$
C.$-\infty$
D.$0$
E.$\frac{26}{-3}$
F.$\frac{13}{10}$
G.$-\frac{26}{-3}$
H.$1$
I.$13$
\testStop
\kluczStart
A
\kluczStop



\zadStart{Przykład z Wikieł P 4.2b moja wersja nr 196}
Obliczyć granicę $\lim\limits_{x\to\ 13}\frac{x^{2}-13^{2}}{(x-13)(x-11)}$.
\zadStop
\rozwStart{Patryk Wirkus}{Martyna Czarnobaj}
$$\frac{x^{2}-13^{2}}{(x-13)(x-11)}=\frac{x+13}{x-11}$$

$$\lim\limits_{x\to\ 13}\frac{x^{2}-13^{2}}{(x-13)(x-11)}=[\frac{0}{0}]=\lim\limits_{x\to\ 13}\frac{x+13}{x-11}=2 \cdot \frac{13}{13-11} = \frac{26}{2}$$
\rozwStop
\odpStart
$\frac{26}{2}$
\odpStop
\testStart
A.$\frac{26}{2}$
B.$\infty$
C.$-\infty$
D.$0$
E.$\frac{26}{-2}$
F.$\frac{13}{11}$
G.$-\frac{26}{-2}$
H.$1$
I.$13$
\testStop
\kluczStart
A
\kluczStop



\zadStart{Przykład z Wikieł P 4.2b moja wersja nr 197}
Obliczyć granicę $\lim\limits_{x\to\ 13}\frac{x^{2}-13^{2}}{(x-13)(x-15)}$.
\zadStop
\rozwStart{Patryk Wirkus}{Martyna Czarnobaj}
$$\frac{x^{2}-13^{2}}{(x-13)(x-15)}=\frac{x+13}{x-15}$$

$$\lim\limits_{x\to\ 13}\frac{x^{2}-13^{2}}{(x-13)(x-15)}=[\frac{0}{0}]=\lim\limits_{x\to\ 13}\frac{x+13}{x-15}=2 \cdot \frac{13}{13-15} = \frac{26}{-2}$$
\rozwStop
\odpStart
$\frac{26}{-2}$
\odpStop
\testStart
A.$\frac{26}{-2}$
B.$\infty$
C.$-\infty$
D.$0$
E.$\frac{26}{2}$
F.$\frac{13}{15}$
G.$-\frac{26}{2}$
H.$1$
I.$13$
\testStop
\kluczStart
A
\kluczStop



\zadStart{Przykład z Wikieł P 4.2b moja wersja nr 198}
Obliczyć granicę $\lim\limits_{x\to\ 13}\frac{x^{2}-13^{2}}{(x-13)(x-16)}$.
\zadStop
\rozwStart{Patryk Wirkus}{Martyna Czarnobaj}
$$\frac{x^{2}-13^{2}}{(x-13)(x-16)}=\frac{x+13}{x-16}$$

$$\lim\limits_{x\to\ 13}\frac{x^{2}-13^{2}}{(x-13)(x-16)}=[\frac{0}{0}]=\lim\limits_{x\to\ 13}\frac{x+13}{x-16}=2 \cdot \frac{13}{13-16} = \frac{26}{-3}$$
\rozwStop
\odpStart
$\frac{26}{-3}$
\odpStop
\testStart
A.$\frac{26}{-3}$
B.$\infty$
C.$-\infty$
D.$0$
E.$\frac{26}{3}$
F.$\frac{13}{16}$
G.$-\frac{26}{3}$
H.$1$
I.$13$
\testStop
\kluczStart
A
\kluczStop



\zadStart{Przykład z Wikieł P 4.2b moja wersja nr 199}
Obliczyć granicę $\lim\limits_{x\to\ 13}\frac{x^{2}-13^{2}}{(x-13)(x-17)}$.
\zadStop
\rozwStart{Patryk Wirkus}{Martyna Czarnobaj}
$$\frac{x^{2}-13^{2}}{(x-13)(x-17)}=\frac{x+13}{x-17}$$

$$\lim\limits_{x\to\ 13}\frac{x^{2}-13^{2}}{(x-13)(x-17)}=[\frac{0}{0}]=\lim\limits_{x\to\ 13}\frac{x+13}{x-17}=2 \cdot \frac{13}{13-17} = \frac{26}{-4}$$
\rozwStop
\odpStart
$\frac{26}{-4}$
\odpStop
\testStart
A.$\frac{26}{-4}$
B.$\infty$
C.$-\infty$
D.$0$
E.$\frac{26}{4}$
F.$\frac{13}{17}$
G.$-\frac{26}{4}$
H.$1$
I.$13$
\testStop
\kluczStart
A
\kluczStop



\zadStart{Przykład z Wikieł P 4.2b moja wersja nr 200}
Obliczyć granicę $\lim\limits_{x\to\ 13}\frac{x^{2}-13^{2}}{(x-13)(x-18)}$.
\zadStop
\rozwStart{Patryk Wirkus}{Martyna Czarnobaj}
$$\frac{x^{2}-13^{2}}{(x-13)(x-18)}=\frac{x+13}{x-18}$$

$$\lim\limits_{x\to\ 13}\frac{x^{2}-13^{2}}{(x-13)(x-18)}=[\frac{0}{0}]=\lim\limits_{x\to\ 13}\frac{x+13}{x-18}=2 \cdot \frac{13}{13-18} = \frac{26}{-5}$$
\rozwStop
\odpStart
$\frac{26}{-5}$
\odpStop
\testStart
A.$\frac{26}{-5}$
B.$\infty$
C.$-\infty$
D.$0$
E.$\frac{26}{5}$
F.$\frac{13}{18}$
G.$-\frac{26}{5}$
H.$1$
I.$13$
\testStop
\kluczStart
A
\kluczStop



\zadStart{Przykład z Wikieł P 4.2b moja wersja nr 201}
Obliczyć granicę $\lim\limits_{x\to\ 13}\frac{x^{2}-13^{2}}{(x-13)(x-19)}$.
\zadStop
\rozwStart{Patryk Wirkus}{Martyna Czarnobaj}
$$\frac{x^{2}-13^{2}}{(x-13)(x-19)}=\frac{x+13}{x-19}$$

$$\lim\limits_{x\to\ 13}\frac{x^{2}-13^{2}}{(x-13)(x-19)}=[\frac{0}{0}]=\lim\limits_{x\to\ 13}\frac{x+13}{x-19}=2 \cdot \frac{13}{13-19} = \frac{26}{-6}$$
\rozwStop
\odpStart
$\frac{26}{-6}$
\odpStop
\testStart
A.$\frac{26}{-6}$
B.$\infty$
C.$-\infty$
D.$0$
E.$\frac{26}{6}$
F.$\frac{13}{19}$
G.$-\frac{26}{6}$
H.$1$
I.$13$
\testStop
\kluczStart
A
\kluczStop



\zadStart{Przykład z Wikieł P 4.2b moja wersja nr 202}
Obliczyć granicę $\lim\limits_{x\to\ 13}\frac{x^{2}-13^{2}}{(x-13)(x-20)}$.
\zadStop
\rozwStart{Patryk Wirkus}{Martyna Czarnobaj}
$$\frac{x^{2}-13^{2}}{(x-13)(x-20)}=\frac{x+13}{x-20}$$

$$\lim\limits_{x\to\ 13}\frac{x^{2}-13^{2}}{(x-13)(x-20)}=[\frac{0}{0}]=\lim\limits_{x\to\ 13}\frac{x+13}{x-20}=2 \cdot \frac{13}{13-20} = \frac{26}{-7}$$
\rozwStop
\odpStart
$\frac{26}{-7}$
\odpStop
\testStart
A.$\frac{26}{-7}$
B.$\infty$
C.$-\infty$
D.$0$
E.$\frac{26}{7}$
F.$\frac{13}{20}$
G.$-\frac{26}{7}$
H.$1$
I.$13$
\testStop
\kluczStart
A
\kluczStop



\zadStart{Przykład z Wikieł P 4.2b moja wersja nr 203}
Obliczyć granicę $\lim\limits_{x\to\ 13}\frac{x^{2}-13^{2}}{(x-13)(x-21)}$.
\zadStop
\rozwStart{Patryk Wirkus}{Martyna Czarnobaj}
$$\frac{x^{2}-13^{2}}{(x-13)(x-21)}=\frac{x+13}{x-21}$$

$$\lim\limits_{x\to\ 13}\frac{x^{2}-13^{2}}{(x-13)(x-21)}=[\frac{0}{0}]=\lim\limits_{x\to\ 13}\frac{x+13}{x-21}=2 \cdot \frac{13}{13-21} = \frac{26}{-8}$$
\rozwStop
\odpStart
$\frac{26}{-8}$
\odpStop
\testStart
A.$\frac{26}{-8}$
B.$\infty$
C.$-\infty$
D.$0$
E.$\frac{26}{8}$
F.$\frac{13}{21}$
G.$-\frac{26}{8}$
H.$1$
I.$13$
\testStop
\kluczStart
A
\kluczStop



\zadStart{Przykład z Wikieł P 4.2b moja wersja nr 204}
Obliczyć granicę $\lim\limits_{x\to\ 13}\frac{x^{2}-13^{2}}{(x-13)(x-22)}$.
\zadStop
\rozwStart{Patryk Wirkus}{Martyna Czarnobaj}
$$\frac{x^{2}-13^{2}}{(x-13)(x-22)}=\frac{x+13}{x-22}$$

$$\lim\limits_{x\to\ 13}\frac{x^{2}-13^{2}}{(x-13)(x-22)}=[\frac{0}{0}]=\lim\limits_{x\to\ 13}\frac{x+13}{x-22}=2 \cdot \frac{13}{13-22} = \frac{26}{-9}$$
\rozwStop
\odpStart
$\frac{26}{-9}$
\odpStop
\testStart
A.$\frac{26}{-9}$
B.$\infty$
C.$-\infty$
D.$0$
E.$\frac{26}{9}$
F.$\frac{13}{22}$
G.$-\frac{26}{9}$
H.$1$
I.$13$
\testStop
\kluczStart
A
\kluczStop



\zadStart{Przykład z Wikieł P 4.2b moja wersja nr 205}
Obliczyć granicę $\lim\limits_{x\to\ 13}\frac{x^{2}-13^{2}}{(x-13)(x-23)}$.
\zadStop
\rozwStart{Patryk Wirkus}{Martyna Czarnobaj}
$$\frac{x^{2}-13^{2}}{(x-13)(x-23)}=\frac{x+13}{x-23}$$

$$\lim\limits_{x\to\ 13}\frac{x^{2}-13^{2}}{(x-13)(x-23)}=[\frac{0}{0}]=\lim\limits_{x\to\ 13}\frac{x+13}{x-23}=2 \cdot \frac{13}{13-23} = \frac{26}{-10}$$
\rozwStop
\odpStart
$\frac{26}{-10}$
\odpStop
\testStart
A.$\frac{26}{-10}$
B.$\infty$
C.$-\infty$
D.$0$
E.$\frac{26}{10}$
F.$\frac{13}{23}$
G.$-\frac{26}{10}$
H.$1$
I.$13$
\testStop
\kluczStart
A
\kluczStop



\zadStart{Przykład z Wikieł P 4.2b moja wersja nr 206}
Obliczyć granicę $\lim\limits_{x\to\ 13}\frac{x^{2}-13^{2}}{(x-13)(x-24)}$.
\zadStop
\rozwStart{Patryk Wirkus}{Martyna Czarnobaj}
$$\frac{x^{2}-13^{2}}{(x-13)(x-24)}=\frac{x+13}{x-24}$$

$$\lim\limits_{x\to\ 13}\frac{x^{2}-13^{2}}{(x-13)(x-24)}=[\frac{0}{0}]=\lim\limits_{x\to\ 13}\frac{x+13}{x-24}=2 \cdot \frac{13}{13-24} = \frac{26}{-11}$$
\rozwStop
\odpStart
$\frac{26}{-11}$
\odpStop
\testStart
A.$\frac{26}{-11}$
B.$\infty$
C.$-\infty$
D.$0$
E.$\frac{26}{11}$
F.$\frac{13}{24}$
G.$-\frac{26}{11}$
H.$1$
I.$13$
\testStop
\kluczStart
A
\kluczStop



\zadStart{Przykład z Wikieł P 4.2b moja wersja nr 207}
Obliczyć granicę $\lim\limits_{x\to\ 13}\frac{x^{2}-13^{2}}{(x-13)(x-25)}$.
\zadStop
\rozwStart{Patryk Wirkus}{Martyna Czarnobaj}
$$\frac{x^{2}-13^{2}}{(x-13)(x-25)}=\frac{x+13}{x-25}$$

$$\lim\limits_{x\to\ 13}\frac{x^{2}-13^{2}}{(x-13)(x-25)}=[\frac{0}{0}]=\lim\limits_{x\to\ 13}\frac{x+13}{x-25}=2 \cdot \frac{13}{13-25} = \frac{26}{-12}$$
\rozwStop
\odpStart
$\frac{26}{-12}$
\odpStop
\testStart
A.$\frac{26}{-12}$
B.$\infty$
C.$-\infty$
D.$0$
E.$\frac{26}{12}$
F.$\frac{13}{25}$
G.$-\frac{26}{12}$
H.$1$
I.$13$
\testStop
\kluczStart
A
\kluczStop



\zadStart{Przykład z Wikieł P 4.2b moja wersja nr 208}
Obliczyć granicę $\lim\limits_{x\to\ 13}\frac{x^{2}-13^{2}}{(x-13)(x-27)}$.
\zadStop
\rozwStart{Patryk Wirkus}{Martyna Czarnobaj}
$$\frac{x^{2}-13^{2}}{(x-13)(x-27)}=\frac{x+13}{x-27}$$

$$\lim\limits_{x\to\ 13}\frac{x^{2}-13^{2}}{(x-13)(x-27)}=[\frac{0}{0}]=\lim\limits_{x\to\ 13}\frac{x+13}{x-27}=2 \cdot \frac{13}{13-27} = \frac{26}{-14}$$
\rozwStop
\odpStart
$\frac{26}{-14}$
\odpStop
\testStart
A.$\frac{26}{-14}$
B.$\infty$
C.$-\infty$
D.$0$
E.$\frac{26}{14}$
F.$\frac{13}{27}$
G.$-\frac{26}{14}$
H.$1$
I.$13$
\testStop
\kluczStart
A
\kluczStop



\zadStart{Przykład z Wikieł P 4.2b moja wersja nr 209}
Obliczyć granicę $\lim\limits_{x\to\ 13}\frac{x^{2}-13^{2}}{(x-13)(x-28)}$.
\zadStop
\rozwStart{Patryk Wirkus}{Martyna Czarnobaj}
$$\frac{x^{2}-13^{2}}{(x-13)(x-28)}=\frac{x+13}{x-28}$$

$$\lim\limits_{x\to\ 13}\frac{x^{2}-13^{2}}{(x-13)(x-28)}=[\frac{0}{0}]=\lim\limits_{x\to\ 13}\frac{x+13}{x-28}=2 \cdot \frac{13}{13-28} = \frac{26}{-15}$$
\rozwStop
\odpStart
$\frac{26}{-15}$
\odpStop
\testStart
A.$\frac{26}{-15}$
B.$\infty$
C.$-\infty$
D.$0$
E.$\frac{26}{15}$
F.$\frac{13}{28}$
G.$-\frac{26}{15}$
H.$1$
I.$13$
\testStop
\kluczStart
A
\kluczStop



\zadStart{Przykład z Wikieł P 4.2b moja wersja nr 210}
Obliczyć granicę $\lim\limits_{x\to\ 13}\frac{x^{2}-13^{2}}{(x-13)(x-29)}$.
\zadStop
\rozwStart{Patryk Wirkus}{Martyna Czarnobaj}
$$\frac{x^{2}-13^{2}}{(x-13)(x-29)}=\frac{x+13}{x-29}$$

$$\lim\limits_{x\to\ 13}\frac{x^{2}-13^{2}}{(x-13)(x-29)}=[\frac{0}{0}]=\lim\limits_{x\to\ 13}\frac{x+13}{x-29}=2 \cdot \frac{13}{13-29} = \frac{26}{-16}$$
\rozwStop
\odpStart
$\frac{26}{-16}$
\odpStop
\testStart
A.$\frac{26}{-16}$
B.$\infty$
C.$-\infty$
D.$0$
E.$\frac{26}{16}$
F.$\frac{13}{29}$
G.$-\frac{26}{16}$
H.$1$
I.$13$
\testStop
\kluczStart
A
\kluczStop



\zadStart{Przykład z Wikieł P 4.2b moja wersja nr 211}
Obliczyć granicę $\lim\limits_{x\to\ 13}\frac{x^{2}-13^{2}}{(x-13)(x-30)}$.
\zadStop
\rozwStart{Patryk Wirkus}{Martyna Czarnobaj}
$$\frac{x^{2}-13^{2}}{(x-13)(x-30)}=\frac{x+13}{x-30}$$

$$\lim\limits_{x\to\ 13}\frac{x^{2}-13^{2}}{(x-13)(x-30)}=[\frac{0}{0}]=\lim\limits_{x\to\ 13}\frac{x+13}{x-30}=2 \cdot \frac{13}{13-30} = \frac{26}{-17}$$
\rozwStop
\odpStart
$\frac{26}{-17}$
\odpStop
\testStart
A.$\frac{26}{-17}$
B.$\infty$
C.$-\infty$
D.$0$
E.$\frac{26}{17}$
F.$\frac{13}{30}$
G.$-\frac{26}{17}$
H.$1$
I.$13$
\testStop
\kluczStart
A
\kluczStop



\zadStart{Przykład z Wikieł P 4.2b moja wersja nr 212}
Obliczyć granicę $\lim\limits_{x\to\ 13}\frac{x^{2}-13^{2}}{(x-13)(x-31)}$.
\zadStop
\rozwStart{Patryk Wirkus}{Martyna Czarnobaj}
$$\frac{x^{2}-13^{2}}{(x-13)(x-31)}=\frac{x+13}{x-31}$$

$$\lim\limits_{x\to\ 13}\frac{x^{2}-13^{2}}{(x-13)(x-31)}=[\frac{0}{0}]=\lim\limits_{x\to\ 13}\frac{x+13}{x-31}=2 \cdot \frac{13}{13-31} = \frac{26}{-18}$$
\rozwStop
\odpStart
$\frac{26}{-18}$
\odpStop
\testStart
A.$\frac{26}{-18}$
B.$\infty$
C.$-\infty$
D.$0$
E.$\frac{26}{18}$
F.$\frac{13}{31}$
G.$-\frac{26}{18}$
H.$1$
I.$13$
\testStop
\kluczStart
A
\kluczStop



\zadStart{Przykład z Wikieł P 4.2b moja wersja nr 213}
Obliczyć granicę $\lim\limits_{x\to\ 13}\frac{x^{2}-13^{2}}{(x-13)(x-32)}$.
\zadStop
\rozwStart{Patryk Wirkus}{Martyna Czarnobaj}
$$\frac{x^{2}-13^{2}}{(x-13)(x-32)}=\frac{x+13}{x-32}$$

$$\lim\limits_{x\to\ 13}\frac{x^{2}-13^{2}}{(x-13)(x-32)}=[\frac{0}{0}]=\lim\limits_{x\to\ 13}\frac{x+13}{x-32}=2 \cdot \frac{13}{13-32} = \frac{26}{-19}$$
\rozwStop
\odpStart
$\frac{26}{-19}$
\odpStop
\testStart
A.$\frac{26}{-19}$
B.$\infty$
C.$-\infty$
D.$0$
E.$\frac{26}{19}$
F.$\frac{13}{32}$
G.$-\frac{26}{19}$
H.$1$
I.$13$
\testStop
\kluczStart
A
\kluczStop



\zadStart{Przykład z Wikieł P 4.2b moja wersja nr 214}
Obliczyć granicę $\lim\limits_{x\to\ 13}\frac{x^{2}-13^{2}}{(x-13)(x-33)}$.
\zadStop
\rozwStart{Patryk Wirkus}{Martyna Czarnobaj}
$$\frac{x^{2}-13^{2}}{(x-13)(x-33)}=\frac{x+13}{x-33}$$

$$\lim\limits_{x\to\ 13}\frac{x^{2}-13^{2}}{(x-13)(x-33)}=[\frac{0}{0}]=\lim\limits_{x\to\ 13}\frac{x+13}{x-33}=2 \cdot \frac{13}{13-33} = \frac{26}{-20}$$
\rozwStop
\odpStart
$\frac{26}{-20}$
\odpStop
\testStart
A.$\frac{26}{-20}$
B.$\infty$
C.$-\infty$
D.$0$
E.$\frac{26}{20}$
F.$\frac{13}{33}$
G.$-\frac{26}{20}$
H.$1$
I.$13$
\testStop
\kluczStart
A
\kluczStop



\zadStart{Przykład z Wikieł P 4.2b moja wersja nr 215}
Obliczyć granicę $\lim\limits_{x\to\ 13}\frac{x^{2}-13^{2}}{(x-13)(x-34)}$.
\zadStop
\rozwStart{Patryk Wirkus}{Martyna Czarnobaj}
$$\frac{x^{2}-13^{2}}{(x-13)(x-34)}=\frac{x+13}{x-34}$$

$$\lim\limits_{x\to\ 13}\frac{x^{2}-13^{2}}{(x-13)(x-34)}=[\frac{0}{0}]=\lim\limits_{x\to\ 13}\frac{x+13}{x-34}=2 \cdot \frac{13}{13-34} = \frac{26}{-21}$$
\rozwStop
\odpStart
$\frac{26}{-21}$
\odpStop
\testStart
A.$\frac{26}{-21}$
B.$\infty$
C.$-\infty$
D.$0$
E.$\frac{26}{21}$
F.$\frac{13}{34}$
G.$-\frac{26}{21}$
H.$1$
I.$13$
\testStop
\kluczStart
A
\kluczStop



\zadStart{Przykład z Wikieł P 4.2b moja wersja nr 216}
Obliczyć granicę $\lim\limits_{x\to\ 13}\frac{x^{2}-13^{2}}{(x-13)(x-35)}$.
\zadStop
\rozwStart{Patryk Wirkus}{Martyna Czarnobaj}
$$\frac{x^{2}-13^{2}}{(x-13)(x-35)}=\frac{x+13}{x-35}$$

$$\lim\limits_{x\to\ 13}\frac{x^{2}-13^{2}}{(x-13)(x-35)}=[\frac{0}{0}]=\lim\limits_{x\to\ 13}\frac{x+13}{x-35}=2 \cdot \frac{13}{13-35} = \frac{26}{-22}$$
\rozwStop
\odpStart
$\frac{26}{-22}$
\odpStop
\testStart
A.$\frac{26}{-22}$
B.$\infty$
C.$-\infty$
D.$0$
E.$\frac{26}{22}$
F.$\frac{13}{35}$
G.$-\frac{26}{22}$
H.$1$
I.$13$
\testStop
\kluczStart
A
\kluczStop



\zadStart{Przykład z Wikieł P 4.2b moja wersja nr 217}
Obliczyć granicę $\lim\limits_{x\to\ 13}\frac{x^{2}-13^{2}}{(x-13)(x-36)}$.
\zadStop
\rozwStart{Patryk Wirkus}{Martyna Czarnobaj}
$$\frac{x^{2}-13^{2}}{(x-13)(x-36)}=\frac{x+13}{x-36}$$

$$\lim\limits_{x\to\ 13}\frac{x^{2}-13^{2}}{(x-13)(x-36)}=[\frac{0}{0}]=\lim\limits_{x\to\ 13}\frac{x+13}{x-36}=2 \cdot \frac{13}{13-36} = \frac{26}{-23}$$
\rozwStop
\odpStart
$\frac{26}{-23}$
\odpStop
\testStart
A.$\frac{26}{-23}$
B.$\infty$
C.$-\infty$
D.$0$
E.$\frac{26}{23}$
F.$\frac{13}{36}$
G.$-\frac{26}{23}$
H.$1$
I.$13$
\testStop
\kluczStart
A
\kluczStop



\zadStart{Przykład z Wikieł P 4.2b moja wersja nr 218}
Obliczyć granicę $\lim\limits_{x\to\ 13}\frac{x^{2}-13^{2}}{(x-13)(x-37)}$.
\zadStop
\rozwStart{Patryk Wirkus}{Martyna Czarnobaj}
$$\frac{x^{2}-13^{2}}{(x-13)(x-37)}=\frac{x+13}{x-37}$$

$$\lim\limits_{x\to\ 13}\frac{x^{2}-13^{2}}{(x-13)(x-37)}=[\frac{0}{0}]=\lim\limits_{x\to\ 13}\frac{x+13}{x-37}=2 \cdot \frac{13}{13-37} = \frac{26}{-24}$$
\rozwStop
\odpStart
$\frac{26}{-24}$
\odpStop
\testStart
A.$\frac{26}{-24}$
B.$\infty$
C.$-\infty$
D.$0$
E.$\frac{26}{24}$
F.$\frac{13}{37}$
G.$-\frac{26}{24}$
H.$1$
I.$13$
\testStop
\kluczStart
A
\kluczStop



\zadStart{Przykład z Wikieł P 4.2b moja wersja nr 219}
Obliczyć granicę $\lim\limits_{x\to\ 13}\frac{x^{2}-13^{2}}{(x-13)(x-38)}$.
\zadStop
\rozwStart{Patryk Wirkus}{Martyna Czarnobaj}
$$\frac{x^{2}-13^{2}}{(x-13)(x-38)}=\frac{x+13}{x-38}$$

$$\lim\limits_{x\to\ 13}\frac{x^{2}-13^{2}}{(x-13)(x-38)}=[\frac{0}{0}]=\lim\limits_{x\to\ 13}\frac{x+13}{x-38}=2 \cdot \frac{13}{13-38} = \frac{26}{-25}$$
\rozwStop
\odpStart
$\frac{26}{-25}$
\odpStop
\testStart
A.$\frac{26}{-25}$
B.$\infty$
C.$-\infty$
D.$0$
E.$\frac{26}{25}$
F.$\frac{13}{38}$
G.$-\frac{26}{25}$
H.$1$
I.$13$
\testStop
\kluczStart
A
\kluczStop



\zadStart{Przykład z Wikieł P 4.2b moja wersja nr 220}
Obliczyć granicę $\lim\limits_{x\to\ 13}\frac{x^{2}-13^{2}}{(x-13)(x-40)}$.
\zadStop
\rozwStart{Patryk Wirkus}{Martyna Czarnobaj}
$$\frac{x^{2}-13^{2}}{(x-13)(x-40)}=\frac{x+13}{x-40}$$

$$\lim\limits_{x\to\ 13}\frac{x^{2}-13^{2}}{(x-13)(x-40)}=[\frac{0}{0}]=\lim\limits_{x\to\ 13}\frac{x+13}{x-40}=2 \cdot \frac{13}{13-40} = \frac{26}{-27}$$
\rozwStop
\odpStart
$\frac{26}{-27}$
\odpStop
\testStart
A.$\frac{26}{-27}$
B.$\infty$
C.$-\infty$
D.$0$
E.$\frac{26}{27}$
F.$\frac{13}{40}$
G.$-\frac{26}{27}$
H.$1$
I.$13$
\testStop
\kluczStart
A
\kluczStop



\zadStart{Przykład z Wikieł P 4.2b moja wersja nr 221}
Obliczyć granicę $\lim\limits_{x\to\ 14}\frac{x^{2}-14^{2}}{(x-14)(x-3)}$.
\zadStop
\rozwStart{Patryk Wirkus}{Martyna Czarnobaj}
$$\frac{x^{2}-14^{2}}{(x-14)(x-3)}=\frac{x+14}{x-3}$$

$$\lim\limits_{x\to\ 14}\frac{x^{2}-14^{2}}{(x-14)(x-3)}=[\frac{0}{0}]=\lim\limits_{x\to\ 14}\frac{x+14}{x-3}=2 \cdot \frac{14}{14-3} = \frac{28}{11}$$
\rozwStop
\odpStart
$\frac{28}{11}$
\odpStop
\testStart
A.$\frac{28}{11}$
B.$\infty$
C.$-\infty$
D.$0$
E.$\frac{28}{-11}$
F.$\frac{14}{3}$
G.$-\frac{28}{-11}$
H.$1$
I.$14$
\testStop
\kluczStart
A
\kluczStop



\zadStart{Przykład z Wikieł P 4.2b moja wersja nr 222}
Obliczyć granicę $\lim\limits_{x\to\ 14}\frac{x^{2}-14^{2}}{(x-14)(x-5)}$.
\zadStop
\rozwStart{Patryk Wirkus}{Martyna Czarnobaj}
$$\frac{x^{2}-14^{2}}{(x-14)(x-5)}=\frac{x+14}{x-5}$$

$$\lim\limits_{x\to\ 14}\frac{x^{2}-14^{2}}{(x-14)(x-5)}=[\frac{0}{0}]=\lim\limits_{x\to\ 14}\frac{x+14}{x-5}=2 \cdot \frac{14}{14-5} = \frac{28}{9}$$
\rozwStop
\odpStart
$\frac{28}{9}$
\odpStop
\testStart
A.$\frac{28}{9}$
B.$\infty$
C.$-\infty$
D.$0$
E.$\frac{28}{-9}$
F.$\frac{14}{5}$
G.$-\frac{28}{-9}$
H.$1$
I.$14$
\testStop
\kluczStart
A
\kluczStop



\zadStart{Przykład z Wikieł P 4.2b moja wersja nr 223}
Obliczyć granicę $\lim\limits_{x\to\ 14}\frac{x^{2}-14^{2}}{(x-14)(x-9)}$.
\zadStop
\rozwStart{Patryk Wirkus}{Martyna Czarnobaj}
$$\frac{x^{2}-14^{2}}{(x-14)(x-9)}=\frac{x+14}{x-9}$$

$$\lim\limits_{x\to\ 14}\frac{x^{2}-14^{2}}{(x-14)(x-9)}=[\frac{0}{0}]=\lim\limits_{x\to\ 14}\frac{x+14}{x-9}=2 \cdot \frac{14}{14-9} = \frac{28}{5}$$
\rozwStop
\odpStart
$\frac{28}{5}$
\odpStop
\testStart
A.$\frac{28}{5}$
B.$\infty$
C.$-\infty$
D.$0$
E.$\frac{28}{-5}$
F.$\frac{14}{9}$
G.$-\frac{28}{-5}$
H.$1$
I.$14$
\testStop
\kluczStart
A
\kluczStop



\zadStart{Przykład z Wikieł P 4.2b moja wersja nr 224}
Obliczyć granicę $\lim\limits_{x\to\ 14}\frac{x^{2}-14^{2}}{(x-14)(x-11)}$.
\zadStop
\rozwStart{Patryk Wirkus}{Martyna Czarnobaj}
$$\frac{x^{2}-14^{2}}{(x-14)(x-11)}=\frac{x+14}{x-11}$$

$$\lim\limits_{x\to\ 14}\frac{x^{2}-14^{2}}{(x-14)(x-11)}=[\frac{0}{0}]=\lim\limits_{x\to\ 14}\frac{x+14}{x-11}=2 \cdot \frac{14}{14-11} = \frac{28}{3}$$
\rozwStop
\odpStart
$\frac{28}{3}$
\odpStop
\testStart
A.$\frac{28}{3}$
B.$\infty$
C.$-\infty$
D.$0$
E.$\frac{28}{-3}$
F.$\frac{14}{11}$
G.$-\frac{28}{-3}$
H.$1$
I.$14$
\testStop
\kluczStart
A
\kluczStop



\zadStart{Przykład z Wikieł P 4.2b moja wersja nr 225}
Obliczyć granicę $\lim\limits_{x\to\ 14}\frac{x^{2}-14^{2}}{(x-14)(x-17)}$.
\zadStop
\rozwStart{Patryk Wirkus}{Martyna Czarnobaj}
$$\frac{x^{2}-14^{2}}{(x-14)(x-17)}=\frac{x+14}{x-17}$$

$$\lim\limits_{x\to\ 14}\frac{x^{2}-14^{2}}{(x-14)(x-17)}=[\frac{0}{0}]=\lim\limits_{x\to\ 14}\frac{x+14}{x-17}=2 \cdot \frac{14}{14-17} = \frac{28}{-3}$$
\rozwStop
\odpStart
$\frac{28}{-3}$
\odpStop
\testStart
A.$\frac{28}{-3}$
B.$\infty$
C.$-\infty$
D.$0$
E.$\frac{28}{3}$
F.$\frac{14}{17}$
G.$-\frac{28}{3}$
H.$1$
I.$14$
\testStop
\kluczStart
A
\kluczStop



\zadStart{Przykład z Wikieł P 4.2b moja wersja nr 226}
Obliczyć granicę $\lim\limits_{x\to\ 14}\frac{x^{2}-14^{2}}{(x-14)(x-19)}$.
\zadStop
\rozwStart{Patryk Wirkus}{Martyna Czarnobaj}
$$\frac{x^{2}-14^{2}}{(x-14)(x-19)}=\frac{x+14}{x-19}$$

$$\lim\limits_{x\to\ 14}\frac{x^{2}-14^{2}}{(x-14)(x-19)}=[\frac{0}{0}]=\lim\limits_{x\to\ 14}\frac{x+14}{x-19}=2 \cdot \frac{14}{14-19} = \frac{28}{-5}$$
\rozwStop
\odpStart
$\frac{28}{-5}$
\odpStop
\testStart
A.$\frac{28}{-5}$
B.$\infty$
C.$-\infty$
D.$0$
E.$\frac{28}{5}$
F.$\frac{14}{19}$
G.$-\frac{28}{5}$
H.$1$
I.$14$
\testStop
\kluczStart
A
\kluczStop



\zadStart{Przykład z Wikieł P 4.2b moja wersja nr 227}
Obliczyć granicę $\lim\limits_{x\to\ 14}\frac{x^{2}-14^{2}}{(x-14)(x-23)}$.
\zadStop
\rozwStart{Patryk Wirkus}{Martyna Czarnobaj}
$$\frac{x^{2}-14^{2}}{(x-14)(x-23)}=\frac{x+14}{x-23}$$

$$\lim\limits_{x\to\ 14}\frac{x^{2}-14^{2}}{(x-14)(x-23)}=[\frac{0}{0}]=\lim\limits_{x\to\ 14}\frac{x+14}{x-23}=2 \cdot \frac{14}{14-23} = \frac{28}{-9}$$
\rozwStop
\odpStart
$\frac{28}{-9}$
\odpStop
\testStart
A.$\frac{28}{-9}$
B.$\infty$
C.$-\infty$
D.$0$
E.$\frac{28}{9}$
F.$\frac{14}{23}$
G.$-\frac{28}{9}$
H.$1$
I.$14$
\testStop
\kluczStart
A
\kluczStop



\zadStart{Przykład z Wikieł P 4.2b moja wersja nr 228}
Obliczyć granicę $\lim\limits_{x\to\ 14}\frac{x^{2}-14^{2}}{(x-14)(x-25)}$.
\zadStop
\rozwStart{Patryk Wirkus}{Martyna Czarnobaj}
$$\frac{x^{2}-14^{2}}{(x-14)(x-25)}=\frac{x+14}{x-25}$$

$$\lim\limits_{x\to\ 14}\frac{x^{2}-14^{2}}{(x-14)(x-25)}=[\frac{0}{0}]=\lim\limits_{x\to\ 14}\frac{x+14}{x-25}=2 \cdot \frac{14}{14-25} = \frac{28}{-11}$$
\rozwStop
\odpStart
$\frac{28}{-11}$
\odpStop
\testStart
A.$\frac{28}{-11}$
B.$\infty$
C.$-\infty$
D.$0$
E.$\frac{28}{11}$
F.$\frac{14}{25}$
G.$-\frac{28}{11}$
H.$1$
I.$14$
\testStop
\kluczStart
A
\kluczStop



\zadStart{Przykład z Wikieł P 4.2b moja wersja nr 229}
Obliczyć granicę $\lim\limits_{x\to\ 14}\frac{x^{2}-14^{2}}{(x-14)(x-27)}$.
\zadStop
\rozwStart{Patryk Wirkus}{Martyna Czarnobaj}
$$\frac{x^{2}-14^{2}}{(x-14)(x-27)}=\frac{x+14}{x-27}$$

$$\lim\limits_{x\to\ 14}\frac{x^{2}-14^{2}}{(x-14)(x-27)}=[\frac{0}{0}]=\lim\limits_{x\to\ 14}\frac{x+14}{x-27}=2 \cdot \frac{14}{14-27} = \frac{28}{-13}$$
\rozwStop
\odpStart
$\frac{28}{-13}$
\odpStop
\testStart
A.$\frac{28}{-13}$
B.$\infty$
C.$-\infty$
D.$0$
E.$\frac{28}{13}$
F.$\frac{14}{27}$
G.$-\frac{28}{13}$
H.$1$
I.$14$
\testStop
\kluczStart
A
\kluczStop



\zadStart{Przykład z Wikieł P 4.2b moja wersja nr 230}
Obliczyć granicę $\lim\limits_{x\to\ 14}\frac{x^{2}-14^{2}}{(x-14)(x-29)}$.
\zadStop
\rozwStart{Patryk Wirkus}{Martyna Czarnobaj}
$$\frac{x^{2}-14^{2}}{(x-14)(x-29)}=\frac{x+14}{x-29}$$

$$\lim\limits_{x\to\ 14}\frac{x^{2}-14^{2}}{(x-14)(x-29)}=[\frac{0}{0}]=\lim\limits_{x\to\ 14}\frac{x+14}{x-29}=2 \cdot \frac{14}{14-29} = \frac{28}{-15}$$
\rozwStop
\odpStart
$\frac{28}{-15}$
\odpStop
\testStart
A.$\frac{28}{-15}$
B.$\infty$
C.$-\infty$
D.$0$
E.$\frac{28}{15}$
F.$\frac{14}{29}$
G.$-\frac{28}{15}$
H.$1$
I.$14$
\testStop
\kluczStart
A
\kluczStop



\zadStart{Przykład z Wikieł P 4.2b moja wersja nr 231}
Obliczyć granicę $\lim\limits_{x\to\ 14}\frac{x^{2}-14^{2}}{(x-14)(x-31)}$.
\zadStop
\rozwStart{Patryk Wirkus}{Martyna Czarnobaj}
$$\frac{x^{2}-14^{2}}{(x-14)(x-31)}=\frac{x+14}{x-31}$$

$$\lim\limits_{x\to\ 14}\frac{x^{2}-14^{2}}{(x-14)(x-31)}=[\frac{0}{0}]=\lim\limits_{x\to\ 14}\frac{x+14}{x-31}=2 \cdot \frac{14}{14-31} = \frac{28}{-17}$$
\rozwStop
\odpStart
$\frac{28}{-17}$
\odpStop
\testStart
A.$\frac{28}{-17}$
B.$\infty$
C.$-\infty$
D.$0$
E.$\frac{28}{17}$
F.$\frac{14}{31}$
G.$-\frac{28}{17}$
H.$1$
I.$14$
\testStop
\kluczStart
A
\kluczStop



\zadStart{Przykład z Wikieł P 4.2b moja wersja nr 232}
Obliczyć granicę $\lim\limits_{x\to\ 14}\frac{x^{2}-14^{2}}{(x-14)(x-33)}$.
\zadStop
\rozwStart{Patryk Wirkus}{Martyna Czarnobaj}
$$\frac{x^{2}-14^{2}}{(x-14)(x-33)}=\frac{x+14}{x-33}$$

$$\lim\limits_{x\to\ 14}\frac{x^{2}-14^{2}}{(x-14)(x-33)}=[\frac{0}{0}]=\lim\limits_{x\to\ 14}\frac{x+14}{x-33}=2 \cdot \frac{14}{14-33} = \frac{28}{-19}$$
\rozwStop
\odpStart
$\frac{28}{-19}$
\odpStop
\testStart
A.$\frac{28}{-19}$
B.$\infty$
C.$-\infty$
D.$0$
E.$\frac{28}{19}$
F.$\frac{14}{33}$
G.$-\frac{28}{19}$
H.$1$
I.$14$
\testStop
\kluczStart
A
\kluczStop



\zadStart{Przykład z Wikieł P 4.2b moja wersja nr 233}
Obliczyć granicę $\lim\limits_{x\to\ 14}\frac{x^{2}-14^{2}}{(x-14)(x-37)}$.
\zadStop
\rozwStart{Patryk Wirkus}{Martyna Czarnobaj}
$$\frac{x^{2}-14^{2}}{(x-14)(x-37)}=\frac{x+14}{x-37}$$

$$\lim\limits_{x\to\ 14}\frac{x^{2}-14^{2}}{(x-14)(x-37)}=[\frac{0}{0}]=\lim\limits_{x\to\ 14}\frac{x+14}{x-37}=2 \cdot \frac{14}{14-37} = \frac{28}{-23}$$
\rozwStop
\odpStart
$\frac{28}{-23}$
\odpStop
\testStart
A.$\frac{28}{-23}$
B.$\infty$
C.$-\infty$
D.$0$
E.$\frac{28}{23}$
F.$\frac{14}{37}$
G.$-\frac{28}{23}$
H.$1$
I.$14$
\testStop
\kluczStart
A
\kluczStop



\zadStart{Przykład z Wikieł P 4.2b moja wersja nr 234}
Obliczyć granicę $\lim\limits_{x\to\ 14}\frac{x^{2}-14^{2}}{(x-14)(x-39)}$.
\zadStop
\rozwStart{Patryk Wirkus}{Martyna Czarnobaj}
$$\frac{x^{2}-14^{2}}{(x-14)(x-39)}=\frac{x+14}{x-39}$$

$$\lim\limits_{x\to\ 14}\frac{x^{2}-14^{2}}{(x-14)(x-39)}=[\frac{0}{0}]=\lim\limits_{x\to\ 14}\frac{x+14}{x-39}=2 \cdot \frac{14}{14-39} = \frac{28}{-25}$$
\rozwStop
\odpStart
$\frac{28}{-25}$
\odpStop
\testStart
A.$\frac{28}{-25}$
B.$\infty$
C.$-\infty$
D.$0$
E.$\frac{28}{25}$
F.$\frac{14}{39}$
G.$-\frac{28}{25}$
H.$1$
I.$14$
\testStop
\kluczStart
A
\kluczStop



\zadStart{Przykład z Wikieł P 4.2b moja wersja nr 235}
Obliczyć granicę $\lim\limits_{x\to\ 15}\frac{x^{2}-15^{2}}{(x-15)(x-2)}$.
\zadStop
\rozwStart{Patryk Wirkus}{Martyna Czarnobaj}
$$\frac{x^{2}-15^{2}}{(x-15)(x-2)}=\frac{x+15}{x-2}$$

$$\lim\limits_{x\to\ 15}\frac{x^{2}-15^{2}}{(x-15)(x-2)}=[\frac{0}{0}]=\lim\limits_{x\to\ 15}\frac{x+15}{x-2}=2 \cdot \frac{15}{15-2} = \frac{30}{13}$$
\rozwStop
\odpStart
$\frac{30}{13}$
\odpStop
\testStart
A.$\frac{30}{13}$
B.$\infty$
C.$-\infty$
D.$0$
E.$\frac{30}{-13}$
F.$\frac{15}{2}$
G.$-\frac{30}{-13}$
H.$1$
I.$15$
\testStop
\kluczStart
A
\kluczStop



\zadStart{Przykład z Wikieł P 4.2b moja wersja nr 236}
Obliczyć granicę $\lim\limits_{x\to\ 15}\frac{x^{2}-15^{2}}{(x-15)(x-4)}$.
\zadStop
\rozwStart{Patryk Wirkus}{Martyna Czarnobaj}
$$\frac{x^{2}-15^{2}}{(x-15)(x-4)}=\frac{x+15}{x-4}$$

$$\lim\limits_{x\to\ 15}\frac{x^{2}-15^{2}}{(x-15)(x-4)}=[\frac{0}{0}]=\lim\limits_{x\to\ 15}\frac{x+15}{x-4}=2 \cdot \frac{15}{15-4} = \frac{30}{11}$$
\rozwStop
\odpStart
$\frac{30}{11}$
\odpStop
\testStart
A.$\frac{30}{11}$
B.$\infty$
C.$-\infty$
D.$0$
E.$\frac{30}{-11}$
F.$\frac{15}{4}$
G.$-\frac{30}{-11}$
H.$1$
I.$15$
\testStop
\kluczStart
A
\kluczStop



\zadStart{Przykład z Wikieł P 4.2b moja wersja nr 237}
Obliczyć granicę $\lim\limits_{x\to\ 15}\frac{x^{2}-15^{2}}{(x-15)(x-7)}$.
\zadStop
\rozwStart{Patryk Wirkus}{Martyna Czarnobaj}
$$\frac{x^{2}-15^{2}}{(x-15)(x-7)}=\frac{x+15}{x-7}$$

$$\lim\limits_{x\to\ 15}\frac{x^{2}-15^{2}}{(x-15)(x-7)}=[\frac{0}{0}]=\lim\limits_{x\to\ 15}\frac{x+15}{x-7}=2 \cdot \frac{15}{15-7} = \frac{30}{8}$$
\rozwStop
\odpStart
$\frac{30}{8}$
\odpStop
\testStart
A.$\frac{30}{8}$
B.$\infty$
C.$-\infty$
D.$0$
E.$\frac{30}{-8}$
F.$\frac{15}{7}$
G.$-\frac{30}{-8}$
H.$1$
I.$15$
\testStop
\kluczStart
A
\kluczStop



\zadStart{Przykład z Wikieł P 4.2b moja wersja nr 238}
Obliczyć granicę $\lim\limits_{x\to\ 15}\frac{x^{2}-15^{2}}{(x-15)(x-8)}$.
\zadStop
\rozwStart{Patryk Wirkus}{Martyna Czarnobaj}
$$\frac{x^{2}-15^{2}}{(x-15)(x-8)}=\frac{x+15}{x-8}$$

$$\lim\limits_{x\to\ 15}\frac{x^{2}-15^{2}}{(x-15)(x-8)}=[\frac{0}{0}]=\lim\limits_{x\to\ 15}\frac{x+15}{x-8}=2 \cdot \frac{15}{15-8} = \frac{30}{7}$$
\rozwStop
\odpStart
$\frac{30}{7}$
\odpStop
\testStart
A.$\frac{30}{7}$
B.$\infty$
C.$-\infty$
D.$0$
E.$\frac{30}{-7}$
F.$\frac{15}{8}$
G.$-\frac{30}{-7}$
H.$1$
I.$15$
\testStop
\kluczStart
A
\kluczStop



\zadStart{Przykład z Wikieł P 4.2b moja wersja nr 239}
Obliczyć granicę $\lim\limits_{x\to\ 15}\frac{x^{2}-15^{2}}{(x-15)(x-11)}$.
\zadStop
\rozwStart{Patryk Wirkus}{Martyna Czarnobaj}
$$\frac{x^{2}-15^{2}}{(x-15)(x-11)}=\frac{x+15}{x-11}$$

$$\lim\limits_{x\to\ 15}\frac{x^{2}-15^{2}}{(x-15)(x-11)}=[\frac{0}{0}]=\lim\limits_{x\to\ 15}\frac{x+15}{x-11}=2 \cdot \frac{15}{15-11} = \frac{30}{4}$$
\rozwStop
\odpStart
$\frac{30}{4}$
\odpStop
\testStart
A.$\frac{30}{4}$
B.$\infty$
C.$-\infty$
D.$0$
E.$\frac{30}{-4}$
F.$\frac{15}{11}$
G.$-\frac{30}{-4}$
H.$1$
I.$15$
\testStop
\kluczStart
A
\kluczStop



\zadStart{Przykład z Wikieł P 4.2b moja wersja nr 240}
Obliczyć granicę $\lim\limits_{x\to\ 15}\frac{x^{2}-15^{2}}{(x-15)(x-13)}$.
\zadStop
\rozwStart{Patryk Wirkus}{Martyna Czarnobaj}
$$\frac{x^{2}-15^{2}}{(x-15)(x-13)}=\frac{x+15}{x-13}$$

$$\lim\limits_{x\to\ 15}\frac{x^{2}-15^{2}}{(x-15)(x-13)}=[\frac{0}{0}]=\lim\limits_{x\to\ 15}\frac{x+15}{x-13}=2 \cdot \frac{15}{15-13} = \frac{30}{2}$$
\rozwStop
\odpStart
$\frac{30}{2}$
\odpStop
\testStart
A.$\frac{30}{2}$
B.$\infty$
C.$-\infty$
D.$0$
E.$\frac{30}{-2}$
F.$\frac{15}{13}$
G.$-\frac{30}{-2}$
H.$1$
I.$15$
\testStop
\kluczStart
A
\kluczStop



\zadStart{Przykład z Wikieł P 4.2b moja wersja nr 241}
Obliczyć granicę $\lim\limits_{x\to\ 15}\frac{x^{2}-15^{2}}{(x-15)(x-17)}$.
\zadStop
\rozwStart{Patryk Wirkus}{Martyna Czarnobaj}
$$\frac{x^{2}-15^{2}}{(x-15)(x-17)}=\frac{x+15}{x-17}$$

$$\lim\limits_{x\to\ 15}\frac{x^{2}-15^{2}}{(x-15)(x-17)}=[\frac{0}{0}]=\lim\limits_{x\to\ 15}\frac{x+15}{x-17}=2 \cdot \frac{15}{15-17} = \frac{30}{-2}$$
\rozwStop
\odpStart
$\frac{30}{-2}$
\odpStop
\testStart
A.$\frac{30}{-2}$
B.$\infty$
C.$-\infty$
D.$0$
E.$\frac{30}{2}$
F.$\frac{15}{17}$
G.$-\frac{30}{2}$
H.$1$
I.$15$
\testStop
\kluczStart
A
\kluczStop



\zadStart{Przykład z Wikieł P 4.2b moja wersja nr 242}
Obliczyć granicę $\lim\limits_{x\to\ 15}\frac{x^{2}-15^{2}}{(x-15)(x-19)}$.
\zadStop
\rozwStart{Patryk Wirkus}{Martyna Czarnobaj}
$$\frac{x^{2}-15^{2}}{(x-15)(x-19)}=\frac{x+15}{x-19}$$

$$\lim\limits_{x\to\ 15}\frac{x^{2}-15^{2}}{(x-15)(x-19)}=[\frac{0}{0}]=\lim\limits_{x\to\ 15}\frac{x+15}{x-19}=2 \cdot \frac{15}{15-19} = \frac{30}{-4}$$
\rozwStop
\odpStart
$\frac{30}{-4}$
\odpStop
\testStart
A.$\frac{30}{-4}$
B.$\infty$
C.$-\infty$
D.$0$
E.$\frac{30}{4}$
F.$\frac{15}{19}$
G.$-\frac{30}{4}$
H.$1$
I.$15$
\testStop
\kluczStart
A
\kluczStop



\zadStart{Przykład z Wikieł P 4.2b moja wersja nr 243}
Obliczyć granicę $\lim\limits_{x\to\ 15}\frac{x^{2}-15^{2}}{(x-15)(x-22)}$.
\zadStop
\rozwStart{Patryk Wirkus}{Martyna Czarnobaj}
$$\frac{x^{2}-15^{2}}{(x-15)(x-22)}=\frac{x+15}{x-22}$$

$$\lim\limits_{x\to\ 15}\frac{x^{2}-15^{2}}{(x-15)(x-22)}=[\frac{0}{0}]=\lim\limits_{x\to\ 15}\frac{x+15}{x-22}=2 \cdot \frac{15}{15-22} = \frac{30}{-7}$$
\rozwStop
\odpStart
$\frac{30}{-7}$
\odpStop
\testStart
A.$\frac{30}{-7}$
B.$\infty$
C.$-\infty$
D.$0$
E.$\frac{30}{7}$
F.$\frac{15}{22}$
G.$-\frac{30}{7}$
H.$1$
I.$15$
\testStop
\kluczStart
A
\kluczStop



\zadStart{Przykład z Wikieł P 4.2b moja wersja nr 244}
Obliczyć granicę $\lim\limits_{x\to\ 15}\frac{x^{2}-15^{2}}{(x-15)(x-23)}$.
\zadStop
\rozwStart{Patryk Wirkus}{Martyna Czarnobaj}
$$\frac{x^{2}-15^{2}}{(x-15)(x-23)}=\frac{x+15}{x-23}$$

$$\lim\limits_{x\to\ 15}\frac{x^{2}-15^{2}}{(x-15)(x-23)}=[\frac{0}{0}]=\lim\limits_{x\to\ 15}\frac{x+15}{x-23}=2 \cdot \frac{15}{15-23} = \frac{30}{-8}$$
\rozwStop
\odpStart
$\frac{30}{-8}$
\odpStop
\testStart
A.$\frac{30}{-8}$
B.$\infty$
C.$-\infty$
D.$0$
E.$\frac{30}{8}$
F.$\frac{15}{23}$
G.$-\frac{30}{8}$
H.$1$
I.$15$
\testStop
\kluczStart
A
\kluczStop



\zadStart{Przykład z Wikieł P 4.2b moja wersja nr 245}
Obliczyć granicę $\lim\limits_{x\to\ 15}\frac{x^{2}-15^{2}}{(x-15)(x-26)}$.
\zadStop
\rozwStart{Patryk Wirkus}{Martyna Czarnobaj}
$$\frac{x^{2}-15^{2}}{(x-15)(x-26)}=\frac{x+15}{x-26}$$

$$\lim\limits_{x\to\ 15}\frac{x^{2}-15^{2}}{(x-15)(x-26)}=[\frac{0}{0}]=\lim\limits_{x\to\ 15}\frac{x+15}{x-26}=2 \cdot \frac{15}{15-26} = \frac{30}{-11}$$
\rozwStop
\odpStart
$\frac{30}{-11}$
\odpStop
\testStart
A.$\frac{30}{-11}$
B.$\infty$
C.$-\infty$
D.$0$
E.$\frac{30}{11}$
F.$\frac{15}{26}$
G.$-\frac{30}{11}$
H.$1$
I.$15$
\testStop
\kluczStart
A
\kluczStop



\zadStart{Przykład z Wikieł P 4.2b moja wersja nr 246}
Obliczyć granicę $\lim\limits_{x\to\ 15}\frac{x^{2}-15^{2}}{(x-15)(x-28)}$.
\zadStop
\rozwStart{Patryk Wirkus}{Martyna Czarnobaj}
$$\frac{x^{2}-15^{2}}{(x-15)(x-28)}=\frac{x+15}{x-28}$$

$$\lim\limits_{x\to\ 15}\frac{x^{2}-15^{2}}{(x-15)(x-28)}=[\frac{0}{0}]=\lim\limits_{x\to\ 15}\frac{x+15}{x-28}=2 \cdot \frac{15}{15-28} = \frac{30}{-13}$$
\rozwStop
\odpStart
$\frac{30}{-13}$
\odpStop
\testStart
A.$\frac{30}{-13}$
B.$\infty$
C.$-\infty$
D.$0$
E.$\frac{30}{13}$
F.$\frac{15}{28}$
G.$-\frac{30}{13}$
H.$1$
I.$15$
\testStop
\kluczStart
A
\kluczStop



\zadStart{Przykład z Wikieł P 4.2b moja wersja nr 247}
Obliczyć granicę $\lim\limits_{x\to\ 15}\frac{x^{2}-15^{2}}{(x-15)(x-29)}$.
\zadStop
\rozwStart{Patryk Wirkus}{Martyna Czarnobaj}
$$\frac{x^{2}-15^{2}}{(x-15)(x-29)}=\frac{x+15}{x-29}$$

$$\lim\limits_{x\to\ 15}\frac{x^{2}-15^{2}}{(x-15)(x-29)}=[\frac{0}{0}]=\lim\limits_{x\to\ 15}\frac{x+15}{x-29}=2 \cdot \frac{15}{15-29} = \frac{30}{-14}$$
\rozwStop
\odpStart
$\frac{30}{-14}$
\odpStop
\testStart
A.$\frac{30}{-14}$
B.$\infty$
C.$-\infty$
D.$0$
E.$\frac{30}{14}$
F.$\frac{15}{29}$
G.$-\frac{30}{14}$
H.$1$
I.$15$
\testStop
\kluczStart
A
\kluczStop



\zadStart{Przykład z Wikieł P 4.2b moja wersja nr 248}
Obliczyć granicę $\lim\limits_{x\to\ 15}\frac{x^{2}-15^{2}}{(x-15)(x-31)}$.
\zadStop
\rozwStart{Patryk Wirkus}{Martyna Czarnobaj}
$$\frac{x^{2}-15^{2}}{(x-15)(x-31)}=\frac{x+15}{x-31}$$

$$\lim\limits_{x\to\ 15}\frac{x^{2}-15^{2}}{(x-15)(x-31)}=[\frac{0}{0}]=\lim\limits_{x\to\ 15}\frac{x+15}{x-31}=2 \cdot \frac{15}{15-31} = \frac{30}{-16}$$
\rozwStop
\odpStart
$\frac{30}{-16}$
\odpStop
\testStart
A.$\frac{30}{-16}$
B.$\infty$
C.$-\infty$
D.$0$
E.$\frac{30}{16}$
F.$\frac{15}{31}$
G.$-\frac{30}{16}$
H.$1$
I.$15$
\testStop
\kluczStart
A
\kluczStop



\zadStart{Przykład z Wikieł P 4.2b moja wersja nr 249}
Obliczyć granicę $\lim\limits_{x\to\ 15}\frac{x^{2}-15^{2}}{(x-15)(x-32)}$.
\zadStop
\rozwStart{Patryk Wirkus}{Martyna Czarnobaj}
$$\frac{x^{2}-15^{2}}{(x-15)(x-32)}=\frac{x+15}{x-32}$$

$$\lim\limits_{x\to\ 15}\frac{x^{2}-15^{2}}{(x-15)(x-32)}=[\frac{0}{0}]=\lim\limits_{x\to\ 15}\frac{x+15}{x-32}=2 \cdot \frac{15}{15-32} = \frac{30}{-17}$$
\rozwStop
\odpStart
$\frac{30}{-17}$
\odpStop
\testStart
A.$\frac{30}{-17}$
B.$\infty$
C.$-\infty$
D.$0$
E.$\frac{30}{17}$
F.$\frac{15}{32}$
G.$-\frac{30}{17}$
H.$1$
I.$15$
\testStop
\kluczStart
A
\kluczStop



\zadStart{Przykład z Wikieł P 4.2b moja wersja nr 250}
Obliczyć granicę $\lim\limits_{x\to\ 15}\frac{x^{2}-15^{2}}{(x-15)(x-34)}$.
\zadStop
\rozwStart{Patryk Wirkus}{Martyna Czarnobaj}
$$\frac{x^{2}-15^{2}}{(x-15)(x-34)}=\frac{x+15}{x-34}$$

$$\lim\limits_{x\to\ 15}\frac{x^{2}-15^{2}}{(x-15)(x-34)}=[\frac{0}{0}]=\lim\limits_{x\to\ 15}\frac{x+15}{x-34}=2 \cdot \frac{15}{15-34} = \frac{30}{-19}$$
\rozwStop
\odpStart
$\frac{30}{-19}$
\odpStop
\testStart
A.$\frac{30}{-19}$
B.$\infty$
C.$-\infty$
D.$0$
E.$\frac{30}{19}$
F.$\frac{15}{34}$
G.$-\frac{30}{19}$
H.$1$
I.$15$
\testStop
\kluczStart
A
\kluczStop



\zadStart{Przykład z Wikieł P 4.2b moja wersja nr 251}
Obliczyć granicę $\lim\limits_{x\to\ 15}\frac{x^{2}-15^{2}}{(x-15)(x-37)}$.
\zadStop
\rozwStart{Patryk Wirkus}{Martyna Czarnobaj}
$$\frac{x^{2}-15^{2}}{(x-15)(x-37)}=\frac{x+15}{x-37}$$

$$\lim\limits_{x\to\ 15}\frac{x^{2}-15^{2}}{(x-15)(x-37)}=[\frac{0}{0}]=\lim\limits_{x\to\ 15}\frac{x+15}{x-37}=2 \cdot \frac{15}{15-37} = \frac{30}{-22}$$
\rozwStop
\odpStart
$\frac{30}{-22}$
\odpStop
\testStart
A.$\frac{30}{-22}$
B.$\infty$
C.$-\infty$
D.$0$
E.$\frac{30}{22}$
F.$\frac{15}{37}$
G.$-\frac{30}{22}$
H.$1$
I.$15$
\testStop
\kluczStart
A
\kluczStop



\zadStart{Przykład z Wikieł P 4.2b moja wersja nr 252}
Obliczyć granicę $\lim\limits_{x\to\ 15}\frac{x^{2}-15^{2}}{(x-15)(x-38)}$.
\zadStop
\rozwStart{Patryk Wirkus}{Martyna Czarnobaj}
$$\frac{x^{2}-15^{2}}{(x-15)(x-38)}=\frac{x+15}{x-38}$$

$$\lim\limits_{x\to\ 15}\frac{x^{2}-15^{2}}{(x-15)(x-38)}=[\frac{0}{0}]=\lim\limits_{x\to\ 15}\frac{x+15}{x-38}=2 \cdot \frac{15}{15-38} = \frac{30}{-23}$$
\rozwStop
\odpStart
$\frac{30}{-23}$
\odpStop
\testStart
A.$\frac{30}{-23}$
B.$\infty$
C.$-\infty$
D.$0$
E.$\frac{30}{23}$
F.$\frac{15}{38}$
G.$-\frac{30}{23}$
H.$1$
I.$15$
\testStop
\kluczStart
A
\kluczStop



\zadStart{Przykład z Wikieł P 4.2b moja wersja nr 253}
Obliczyć granicę $\lim\limits_{x\to\ 16}\frac{x^{2}-16^{2}}{(x-16)(x-3)}$.
\zadStop
\rozwStart{Patryk Wirkus}{Martyna Czarnobaj}
$$\frac{x^{2}-16^{2}}{(x-16)(x-3)}=\frac{x+16}{x-3}$$

$$\lim\limits_{x\to\ 16}\frac{x^{2}-16^{2}}{(x-16)(x-3)}=[\frac{0}{0}]=\lim\limits_{x\to\ 16}\frac{x+16}{x-3}=2 \cdot \frac{16}{16-3} = \frac{32}{13}$$
\rozwStop
\odpStart
$\frac{32}{13}$
\odpStop
\testStart
A.$\frac{32}{13}$
B.$\infty$
C.$-\infty$
D.$0$
E.$\frac{32}{-13}$
F.$\frac{16}{3}$
G.$-\frac{32}{-13}$
H.$1$
I.$16$
\testStop
\kluczStart
A
\kluczStop



\zadStart{Przykład z Wikieł P 4.2b moja wersja nr 254}
Obliczyć granicę $\lim\limits_{x\to\ 16}\frac{x^{2}-16^{2}}{(x-16)(x-5)}$.
\zadStop
\rozwStart{Patryk Wirkus}{Martyna Czarnobaj}
$$\frac{x^{2}-16^{2}}{(x-16)(x-5)}=\frac{x+16}{x-5}$$

$$\lim\limits_{x\to\ 16}\frac{x^{2}-16^{2}}{(x-16)(x-5)}=[\frac{0}{0}]=\lim\limits_{x\to\ 16}\frac{x+16}{x-5}=2 \cdot \frac{16}{16-5} = \frac{32}{11}$$
\rozwStop
\odpStart
$\frac{32}{11}$
\odpStop
\testStart
A.$\frac{32}{11}$
B.$\infty$
C.$-\infty$
D.$0$
E.$\frac{32}{-11}$
F.$\frac{16}{5}$
G.$-\frac{32}{-11}$
H.$1$
I.$16$
\testStop
\kluczStart
A
\kluczStop



\zadStart{Przykład z Wikieł P 4.2b moja wersja nr 255}
Obliczyć granicę $\lim\limits_{x\to\ 16}\frac{x^{2}-16^{2}}{(x-16)(x-7)}$.
\zadStop
\rozwStart{Patryk Wirkus}{Martyna Czarnobaj}
$$\frac{x^{2}-16^{2}}{(x-16)(x-7)}=\frac{x+16}{x-7}$$

$$\lim\limits_{x\to\ 16}\frac{x^{2}-16^{2}}{(x-16)(x-7)}=[\frac{0}{0}]=\lim\limits_{x\to\ 16}\frac{x+16}{x-7}=2 \cdot \frac{16}{16-7} = \frac{32}{9}$$
\rozwStop
\odpStart
$\frac{32}{9}$
\odpStop
\testStart
A.$\frac{32}{9}$
B.$\infty$
C.$-\infty$
D.$0$
E.$\frac{32}{-9}$
F.$\frac{16}{7}$
G.$-\frac{32}{-9}$
H.$1$
I.$16$
\testStop
\kluczStart
A
\kluczStop



\zadStart{Przykład z Wikieł P 4.2b moja wersja nr 256}
Obliczyć granicę $\lim\limits_{x\to\ 16}\frac{x^{2}-16^{2}}{(x-16)(x-9)}$.
\zadStop
\rozwStart{Patryk Wirkus}{Martyna Czarnobaj}
$$\frac{x^{2}-16^{2}}{(x-16)(x-9)}=\frac{x+16}{x-9}$$

$$\lim\limits_{x\to\ 16}\frac{x^{2}-16^{2}}{(x-16)(x-9)}=[\frac{0}{0}]=\lim\limits_{x\to\ 16}\frac{x+16}{x-9}=2 \cdot \frac{16}{16-9} = \frac{32}{7}$$
\rozwStop
\odpStart
$\frac{32}{7}$
\odpStop
\testStart
A.$\frac{32}{7}$
B.$\infty$
C.$-\infty$
D.$0$
E.$\frac{32}{-7}$
F.$\frac{16}{9}$
G.$-\frac{32}{-7}$
H.$1$
I.$16$
\testStop
\kluczStart
A
\kluczStop



\zadStart{Przykład z Wikieł P 4.2b moja wersja nr 257}
Obliczyć granicę $\lim\limits_{x\to\ 16}\frac{x^{2}-16^{2}}{(x-16)(x-11)}$.
\zadStop
\rozwStart{Patryk Wirkus}{Martyna Czarnobaj}
$$\frac{x^{2}-16^{2}}{(x-16)(x-11)}=\frac{x+16}{x-11}$$

$$\lim\limits_{x\to\ 16}\frac{x^{2}-16^{2}}{(x-16)(x-11)}=[\frac{0}{0}]=\lim\limits_{x\to\ 16}\frac{x+16}{x-11}=2 \cdot \frac{16}{16-11} = \frac{32}{5}$$
\rozwStop
\odpStart
$\frac{32}{5}$
\odpStop
\testStart
A.$\frac{32}{5}$
B.$\infty$
C.$-\infty$
D.$0$
E.$\frac{32}{-5}$
F.$\frac{16}{11}$
G.$-\frac{32}{-5}$
H.$1$
I.$16$
\testStop
\kluczStart
A
\kluczStop



\zadStart{Przykład z Wikieł P 4.2b moja wersja nr 258}
Obliczyć granicę $\lim\limits_{x\to\ 16}\frac{x^{2}-16^{2}}{(x-16)(x-13)}$.
\zadStop
\rozwStart{Patryk Wirkus}{Martyna Czarnobaj}
$$\frac{x^{2}-16^{2}}{(x-16)(x-13)}=\frac{x+16}{x-13}$$

$$\lim\limits_{x\to\ 16}\frac{x^{2}-16^{2}}{(x-16)(x-13)}=[\frac{0}{0}]=\lim\limits_{x\to\ 16}\frac{x+16}{x-13}=2 \cdot \frac{16}{16-13} = \frac{32}{3}$$
\rozwStop
\odpStart
$\frac{32}{3}$
\odpStop
\testStart
A.$\frac{32}{3}$
B.$\infty$
C.$-\infty$
D.$0$
E.$\frac{32}{-3}$
F.$\frac{16}{13}$
G.$-\frac{32}{-3}$
H.$1$
I.$16$
\testStop
\kluczStart
A
\kluczStop



\zadStart{Przykład z Wikieł P 4.2b moja wersja nr 259}
Obliczyć granicę $\lim\limits_{x\to\ 16}\frac{x^{2}-16^{2}}{(x-16)(x-19)}$.
\zadStop
\rozwStart{Patryk Wirkus}{Martyna Czarnobaj}
$$\frac{x^{2}-16^{2}}{(x-16)(x-19)}=\frac{x+16}{x-19}$$

$$\lim\limits_{x\to\ 16}\frac{x^{2}-16^{2}}{(x-16)(x-19)}=[\frac{0}{0}]=\lim\limits_{x\to\ 16}\frac{x+16}{x-19}=2 \cdot \frac{16}{16-19} = \frac{32}{-3}$$
\rozwStop
\odpStart
$\frac{32}{-3}$
\odpStop
\testStart
A.$\frac{32}{-3}$
B.$\infty$
C.$-\infty$
D.$0$
E.$\frac{32}{3}$
F.$\frac{16}{19}$
G.$-\frac{32}{3}$
H.$1$
I.$16$
\testStop
\kluczStart
A
\kluczStop



\zadStart{Przykład z Wikieł P 4.2b moja wersja nr 260}
Obliczyć granicę $\lim\limits_{x\to\ 16}\frac{x^{2}-16^{2}}{(x-16)(x-21)}$.
\zadStop
\rozwStart{Patryk Wirkus}{Martyna Czarnobaj}
$$\frac{x^{2}-16^{2}}{(x-16)(x-21)}=\frac{x+16}{x-21}$$

$$\lim\limits_{x\to\ 16}\frac{x^{2}-16^{2}}{(x-16)(x-21)}=[\frac{0}{0}]=\lim\limits_{x\to\ 16}\frac{x+16}{x-21}=2 \cdot \frac{16}{16-21} = \frac{32}{-5}$$
\rozwStop
\odpStart
$\frac{32}{-5}$
\odpStop
\testStart
A.$\frac{32}{-5}$
B.$\infty$
C.$-\infty$
D.$0$
E.$\frac{32}{5}$
F.$\frac{16}{21}$
G.$-\frac{32}{5}$
H.$1$
I.$16$
\testStop
\kluczStart
A
\kluczStop



\zadStart{Przykład z Wikieł P 4.2b moja wersja nr 261}
Obliczyć granicę $\lim\limits_{x\to\ 16}\frac{x^{2}-16^{2}}{(x-16)(x-23)}$.
\zadStop
\rozwStart{Patryk Wirkus}{Martyna Czarnobaj}
$$\frac{x^{2}-16^{2}}{(x-16)(x-23)}=\frac{x+16}{x-23}$$

$$\lim\limits_{x\to\ 16}\frac{x^{2}-16^{2}}{(x-16)(x-23)}=[\frac{0}{0}]=\lim\limits_{x\to\ 16}\frac{x+16}{x-23}=2 \cdot \frac{16}{16-23} = \frac{32}{-7}$$
\rozwStop
\odpStart
$\frac{32}{-7}$
\odpStop
\testStart
A.$\frac{32}{-7}$
B.$\infty$
C.$-\infty$
D.$0$
E.$\frac{32}{7}$
F.$\frac{16}{23}$
G.$-\frac{32}{7}$
H.$1$
I.$16$
\testStop
\kluczStart
A
\kluczStop



\zadStart{Przykład z Wikieł P 4.2b moja wersja nr 262}
Obliczyć granicę $\lim\limits_{x\to\ 16}\frac{x^{2}-16^{2}}{(x-16)(x-25)}$.
\zadStop
\rozwStart{Patryk Wirkus}{Martyna Czarnobaj}
$$\frac{x^{2}-16^{2}}{(x-16)(x-25)}=\frac{x+16}{x-25}$$

$$\lim\limits_{x\to\ 16}\frac{x^{2}-16^{2}}{(x-16)(x-25)}=[\frac{0}{0}]=\lim\limits_{x\to\ 16}\frac{x+16}{x-25}=2 \cdot \frac{16}{16-25} = \frac{32}{-9}$$
\rozwStop
\odpStart
$\frac{32}{-9}$
\odpStop
\testStart
A.$\frac{32}{-9}$
B.$\infty$
C.$-\infty$
D.$0$
E.$\frac{32}{9}$
F.$\frac{16}{25}$
G.$-\frac{32}{9}$
H.$1$
I.$16$
\testStop
\kluczStart
A
\kluczStop



\zadStart{Przykład z Wikieł P 4.2b moja wersja nr 263}
Obliczyć granicę $\lim\limits_{x\to\ 16}\frac{x^{2}-16^{2}}{(x-16)(x-27)}$.
\zadStop
\rozwStart{Patryk Wirkus}{Martyna Czarnobaj}
$$\frac{x^{2}-16^{2}}{(x-16)(x-27)}=\frac{x+16}{x-27}$$

$$\lim\limits_{x\to\ 16}\frac{x^{2}-16^{2}}{(x-16)(x-27)}=[\frac{0}{0}]=\lim\limits_{x\to\ 16}\frac{x+16}{x-27}=2 \cdot \frac{16}{16-27} = \frac{32}{-11}$$
\rozwStop
\odpStart
$\frac{32}{-11}$
\odpStop
\testStart
A.$\frac{32}{-11}$
B.$\infty$
C.$-\infty$
D.$0$
E.$\frac{32}{11}$
F.$\frac{16}{27}$
G.$-\frac{32}{11}$
H.$1$
I.$16$
\testStop
\kluczStart
A
\kluczStop



\zadStart{Przykład z Wikieł P 4.2b moja wersja nr 264}
Obliczyć granicę $\lim\limits_{x\to\ 16}\frac{x^{2}-16^{2}}{(x-16)(x-29)}$.
\zadStop
\rozwStart{Patryk Wirkus}{Martyna Czarnobaj}
$$\frac{x^{2}-16^{2}}{(x-16)(x-29)}=\frac{x+16}{x-29}$$

$$\lim\limits_{x\to\ 16}\frac{x^{2}-16^{2}}{(x-16)(x-29)}=[\frac{0}{0}]=\lim\limits_{x\to\ 16}\frac{x+16}{x-29}=2 \cdot \frac{16}{16-29} = \frac{32}{-13}$$
\rozwStop
\odpStart
$\frac{32}{-13}$
\odpStop
\testStart
A.$\frac{32}{-13}$
B.$\infty$
C.$-\infty$
D.$0$
E.$\frac{32}{13}$
F.$\frac{16}{29}$
G.$-\frac{32}{13}$
H.$1$
I.$16$
\testStop
\kluczStart
A
\kluczStop



\zadStart{Przykład z Wikieł P 4.2b moja wersja nr 265}
Obliczyć granicę $\lim\limits_{x\to\ 16}\frac{x^{2}-16^{2}}{(x-16)(x-31)}$.
\zadStop
\rozwStart{Patryk Wirkus}{Martyna Czarnobaj}
$$\frac{x^{2}-16^{2}}{(x-16)(x-31)}=\frac{x+16}{x-31}$$

$$\lim\limits_{x\to\ 16}\frac{x^{2}-16^{2}}{(x-16)(x-31)}=[\frac{0}{0}]=\lim\limits_{x\to\ 16}\frac{x+16}{x-31}=2 \cdot \frac{16}{16-31} = \frac{32}{-15}$$
\rozwStop
\odpStart
$\frac{32}{-15}$
\odpStop
\testStart
A.$\frac{32}{-15}$
B.$\infty$
C.$-\infty$
D.$0$
E.$\frac{32}{15}$
F.$\frac{16}{31}$
G.$-\frac{32}{15}$
H.$1$
I.$16$
\testStop
\kluczStart
A
\kluczStop



\zadStart{Przykład z Wikieł P 4.2b moja wersja nr 266}
Obliczyć granicę $\lim\limits_{x\to\ 16}\frac{x^{2}-16^{2}}{(x-16)(x-33)}$.
\zadStop
\rozwStart{Patryk Wirkus}{Martyna Czarnobaj}
$$\frac{x^{2}-16^{2}}{(x-16)(x-33)}=\frac{x+16}{x-33}$$

$$\lim\limits_{x\to\ 16}\frac{x^{2}-16^{2}}{(x-16)(x-33)}=[\frac{0}{0}]=\lim\limits_{x\to\ 16}\frac{x+16}{x-33}=2 \cdot \frac{16}{16-33} = \frac{32}{-17}$$
\rozwStop
\odpStart
$\frac{32}{-17}$
\odpStop
\testStart
A.$\frac{32}{-17}$
B.$\infty$
C.$-\infty$
D.$0$
E.$\frac{32}{17}$
F.$\frac{16}{33}$
G.$-\frac{32}{17}$
H.$1$
I.$16$
\testStop
\kluczStart
A
\kluczStop



\zadStart{Przykład z Wikieł P 4.2b moja wersja nr 267}
Obliczyć granicę $\lim\limits_{x\to\ 16}\frac{x^{2}-16^{2}}{(x-16)(x-35)}$.
\zadStop
\rozwStart{Patryk Wirkus}{Martyna Czarnobaj}
$$\frac{x^{2}-16^{2}}{(x-16)(x-35)}=\frac{x+16}{x-35}$$

$$\lim\limits_{x\to\ 16}\frac{x^{2}-16^{2}}{(x-16)(x-35)}=[\frac{0}{0}]=\lim\limits_{x\to\ 16}\frac{x+16}{x-35}=2 \cdot \frac{16}{16-35} = \frac{32}{-19}$$
\rozwStop
\odpStart
$\frac{32}{-19}$
\odpStop
\testStart
A.$\frac{32}{-19}$
B.$\infty$
C.$-\infty$
D.$0$
E.$\frac{32}{19}$
F.$\frac{16}{35}$
G.$-\frac{32}{19}$
H.$1$
I.$16$
\testStop
\kluczStart
A
\kluczStop



\zadStart{Przykład z Wikieł P 4.2b moja wersja nr 268}
Obliczyć granicę $\lim\limits_{x\to\ 16}\frac{x^{2}-16^{2}}{(x-16)(x-37)}$.
\zadStop
\rozwStart{Patryk Wirkus}{Martyna Czarnobaj}
$$\frac{x^{2}-16^{2}}{(x-16)(x-37)}=\frac{x+16}{x-37}$$

$$\lim\limits_{x\to\ 16}\frac{x^{2}-16^{2}}{(x-16)(x-37)}=[\frac{0}{0}]=\lim\limits_{x\to\ 16}\frac{x+16}{x-37}=2 \cdot \frac{16}{16-37} = \frac{32}{-21}$$
\rozwStop
\odpStart
$\frac{32}{-21}$
\odpStop
\testStart
A.$\frac{32}{-21}$
B.$\infty$
C.$-\infty$
D.$0$
E.$\frac{32}{21}$
F.$\frac{16}{37}$
G.$-\frac{32}{21}$
H.$1$
I.$16$
\testStop
\kluczStart
A
\kluczStop



\zadStart{Przykład z Wikieł P 4.2b moja wersja nr 269}
Obliczyć granicę $\lim\limits_{x\to\ 16}\frac{x^{2}-16^{2}}{(x-16)(x-39)}$.
\zadStop
\rozwStart{Patryk Wirkus}{Martyna Czarnobaj}
$$\frac{x^{2}-16^{2}}{(x-16)(x-39)}=\frac{x+16}{x-39}$$

$$\lim\limits_{x\to\ 16}\frac{x^{2}-16^{2}}{(x-16)(x-39)}=[\frac{0}{0}]=\lim\limits_{x\to\ 16}\frac{x+16}{x-39}=2 \cdot \frac{16}{16-39} = \frac{32}{-23}$$
\rozwStop
\odpStart
$\frac{32}{-23}$
\odpStop
\testStart
A.$\frac{32}{-23}$
B.$\infty$
C.$-\infty$
D.$0$
E.$\frac{32}{23}$
F.$\frac{16}{39}$
G.$-\frac{32}{23}$
H.$1$
I.$16$
\testStop
\kluczStart
A
\kluczStop



\zadStart{Przykład z Wikieł P 4.2b moja wersja nr 270}
Obliczyć granicę $\lim\limits_{x\to\ 17}\frac{x^{2}-17^{2}}{(x-17)(x-2)}$.
\zadStop
\rozwStart{Patryk Wirkus}{Martyna Czarnobaj}
$$\frac{x^{2}-17^{2}}{(x-17)(x-2)}=\frac{x+17}{x-2}$$

$$\lim\limits_{x\to\ 17}\frac{x^{2}-17^{2}}{(x-17)(x-2)}=[\frac{0}{0}]=\lim\limits_{x\to\ 17}\frac{x+17}{x-2}=2 \cdot \frac{17}{17-2} = \frac{34}{15}$$
\rozwStop
\odpStart
$\frac{34}{15}$
\odpStop
\testStart
A.$\frac{34}{15}$
B.$\infty$
C.$-\infty$
D.$0$
E.$\frac{34}{-15}$
F.$\frac{17}{2}$
G.$-\frac{34}{-15}$
H.$1$
I.$17$
\testStop
\kluczStart
A
\kluczStop



\zadStart{Przykład z Wikieł P 4.2b moja wersja nr 271}
Obliczyć granicę $\lim\limits_{x\to\ 17}\frac{x^{2}-17^{2}}{(x-17)(x-3)}$.
\zadStop
\rozwStart{Patryk Wirkus}{Martyna Czarnobaj}
$$\frac{x^{2}-17^{2}}{(x-17)(x-3)}=\frac{x+17}{x-3}$$

$$\lim\limits_{x\to\ 17}\frac{x^{2}-17^{2}}{(x-17)(x-3)}=[\frac{0}{0}]=\lim\limits_{x\to\ 17}\frac{x+17}{x-3}=2 \cdot \frac{17}{17-3} = \frac{34}{14}$$
\rozwStop
\odpStart
$\frac{34}{14}$
\odpStop
\testStart
A.$\frac{34}{14}$
B.$\infty$
C.$-\infty$
D.$0$
E.$\frac{34}{-14}$
F.$\frac{17}{3}$
G.$-\frac{34}{-14}$
H.$1$
I.$17$
\testStop
\kluczStart
A
\kluczStop



\zadStart{Przykład z Wikieł P 4.2b moja wersja nr 272}
Obliczyć granicę $\lim\limits_{x\to\ 17}\frac{x^{2}-17^{2}}{(x-17)(x-4)}$.
\zadStop
\rozwStart{Patryk Wirkus}{Martyna Czarnobaj}
$$\frac{x^{2}-17^{2}}{(x-17)(x-4)}=\frac{x+17}{x-4}$$

$$\lim\limits_{x\to\ 17}\frac{x^{2}-17^{2}}{(x-17)(x-4)}=[\frac{0}{0}]=\lim\limits_{x\to\ 17}\frac{x+17}{x-4}=2 \cdot \frac{17}{17-4} = \frac{34}{13}$$
\rozwStop
\odpStart
$\frac{34}{13}$
\odpStop
\testStart
A.$\frac{34}{13}$
B.$\infty$
C.$-\infty$
D.$0$
E.$\frac{34}{-13}$
F.$\frac{17}{4}$
G.$-\frac{34}{-13}$
H.$1$
I.$17$
\testStop
\kluczStart
A
\kluczStop



\zadStart{Przykład z Wikieł P 4.2b moja wersja nr 273}
Obliczyć granicę $\lim\limits_{x\to\ 17}\frac{x^{2}-17^{2}}{(x-17)(x-5)}$.
\zadStop
\rozwStart{Patryk Wirkus}{Martyna Czarnobaj}
$$\frac{x^{2}-17^{2}}{(x-17)(x-5)}=\frac{x+17}{x-5}$$

$$\lim\limits_{x\to\ 17}\frac{x^{2}-17^{2}}{(x-17)(x-5)}=[\frac{0}{0}]=\lim\limits_{x\to\ 17}\frac{x+17}{x-5}=2 \cdot \frac{17}{17-5} = \frac{34}{12}$$
\rozwStop
\odpStart
$\frac{34}{12}$
\odpStop
\testStart
A.$\frac{34}{12}$
B.$\infty$
C.$-\infty$
D.$0$
E.$\frac{34}{-12}$
F.$\frac{17}{5}$
G.$-\frac{34}{-12}$
H.$1$
I.$17$
\testStop
\kluczStart
A
\kluczStop



\zadStart{Przykład z Wikieł P 4.2b moja wersja nr 274}
Obliczyć granicę $\lim\limits_{x\to\ 17}\frac{x^{2}-17^{2}}{(x-17)(x-6)}$.
\zadStop
\rozwStart{Patryk Wirkus}{Martyna Czarnobaj}
$$\frac{x^{2}-17^{2}}{(x-17)(x-6)}=\frac{x+17}{x-6}$$

$$\lim\limits_{x\to\ 17}\frac{x^{2}-17^{2}}{(x-17)(x-6)}=[\frac{0}{0}]=\lim\limits_{x\to\ 17}\frac{x+17}{x-6}=2 \cdot \frac{17}{17-6} = \frac{34}{11}$$
\rozwStop
\odpStart
$\frac{34}{11}$
\odpStop
\testStart
A.$\frac{34}{11}$
B.$\infty$
C.$-\infty$
D.$0$
E.$\frac{34}{-11}$
F.$\frac{17}{6}$
G.$-\frac{34}{-11}$
H.$1$
I.$17$
\testStop
\kluczStart
A
\kluczStop



\zadStart{Przykład z Wikieł P 4.2b moja wersja nr 275}
Obliczyć granicę $\lim\limits_{x\to\ 17}\frac{x^{2}-17^{2}}{(x-17)(x-7)}$.
\zadStop
\rozwStart{Patryk Wirkus}{Martyna Czarnobaj}
$$\frac{x^{2}-17^{2}}{(x-17)(x-7)}=\frac{x+17}{x-7}$$

$$\lim\limits_{x\to\ 17}\frac{x^{2}-17^{2}}{(x-17)(x-7)}=[\frac{0}{0}]=\lim\limits_{x\to\ 17}\frac{x+17}{x-7}=2 \cdot \frac{17}{17-7} = \frac{34}{10}$$
\rozwStop
\odpStart
$\frac{34}{10}$
\odpStop
\testStart
A.$\frac{34}{10}$
B.$\infty$
C.$-\infty$
D.$0$
E.$\frac{34}{-10}$
F.$\frac{17}{7}$
G.$-\frac{34}{-10}$
H.$1$
I.$17$
\testStop
\kluczStart
A
\kluczStop



\zadStart{Przykład z Wikieł P 4.2b moja wersja nr 276}
Obliczyć granicę $\lim\limits_{x\to\ 17}\frac{x^{2}-17^{2}}{(x-17)(x-8)}$.
\zadStop
\rozwStart{Patryk Wirkus}{Martyna Czarnobaj}
$$\frac{x^{2}-17^{2}}{(x-17)(x-8)}=\frac{x+17}{x-8}$$

$$\lim\limits_{x\to\ 17}\frac{x^{2}-17^{2}}{(x-17)(x-8)}=[\frac{0}{0}]=\lim\limits_{x\to\ 17}\frac{x+17}{x-8}=2 \cdot \frac{17}{17-8} = \frac{34}{9}$$
\rozwStop
\odpStart
$\frac{34}{9}$
\odpStop
\testStart
A.$\frac{34}{9}$
B.$\infty$
C.$-\infty$
D.$0$
E.$\frac{34}{-9}$
F.$\frac{17}{8}$
G.$-\frac{34}{-9}$
H.$1$
I.$17$
\testStop
\kluczStart
A
\kluczStop



\zadStart{Przykład z Wikieł P 4.2b moja wersja nr 277}
Obliczyć granicę $\lim\limits_{x\to\ 17}\frac{x^{2}-17^{2}}{(x-17)(x-9)}$.
\zadStop
\rozwStart{Patryk Wirkus}{Martyna Czarnobaj}
$$\frac{x^{2}-17^{2}}{(x-17)(x-9)}=\frac{x+17}{x-9}$$

$$\lim\limits_{x\to\ 17}\frac{x^{2}-17^{2}}{(x-17)(x-9)}=[\frac{0}{0}]=\lim\limits_{x\to\ 17}\frac{x+17}{x-9}=2 \cdot \frac{17}{17-9} = \frac{34}{8}$$
\rozwStop
\odpStart
$\frac{34}{8}$
\odpStop
\testStart
A.$\frac{34}{8}$
B.$\infty$
C.$-\infty$
D.$0$
E.$\frac{34}{-8}$
F.$\frac{17}{9}$
G.$-\frac{34}{-8}$
H.$1$
I.$17$
\testStop
\kluczStart
A
\kluczStop



\zadStart{Przykład z Wikieł P 4.2b moja wersja nr 278}
Obliczyć granicę $\lim\limits_{x\to\ 17}\frac{x^{2}-17^{2}}{(x-17)(x-10)}$.
\zadStop
\rozwStart{Patryk Wirkus}{Martyna Czarnobaj}
$$\frac{x^{2}-17^{2}}{(x-17)(x-10)}=\frac{x+17}{x-10}$$

$$\lim\limits_{x\to\ 17}\frac{x^{2}-17^{2}}{(x-17)(x-10)}=[\frac{0}{0}]=\lim\limits_{x\to\ 17}\frac{x+17}{x-10}=2 \cdot \frac{17}{17-10} = \frac{34}{7}$$
\rozwStop
\odpStart
$\frac{34}{7}$
\odpStop
\testStart
A.$\frac{34}{7}$
B.$\infty$
C.$-\infty$
D.$0$
E.$\frac{34}{-7}$
F.$\frac{17}{10}$
G.$-\frac{34}{-7}$
H.$1$
I.$17$
\testStop
\kluczStart
A
\kluczStop



\zadStart{Przykład z Wikieł P 4.2b moja wersja nr 279}
Obliczyć granicę $\lim\limits_{x\to\ 17}\frac{x^{2}-17^{2}}{(x-17)(x-11)}$.
\zadStop
\rozwStart{Patryk Wirkus}{Martyna Czarnobaj}
$$\frac{x^{2}-17^{2}}{(x-17)(x-11)}=\frac{x+17}{x-11}$$

$$\lim\limits_{x\to\ 17}\frac{x^{2}-17^{2}}{(x-17)(x-11)}=[\frac{0}{0}]=\lim\limits_{x\to\ 17}\frac{x+17}{x-11}=2 \cdot \frac{17}{17-11} = \frac{34}{6}$$
\rozwStop
\odpStart
$\frac{34}{6}$
\odpStop
\testStart
A.$\frac{34}{6}$
B.$\infty$
C.$-\infty$
D.$0$
E.$\frac{34}{-6}$
F.$\frac{17}{11}$
G.$-\frac{34}{-6}$
H.$1$
I.$17$
\testStop
\kluczStart
A
\kluczStop



\zadStart{Przykład z Wikieł P 4.2b moja wersja nr 280}
Obliczyć granicę $\lim\limits_{x\to\ 17}\frac{x^{2}-17^{2}}{(x-17)(x-12)}$.
\zadStop
\rozwStart{Patryk Wirkus}{Martyna Czarnobaj}
$$\frac{x^{2}-17^{2}}{(x-17)(x-12)}=\frac{x+17}{x-12}$$

$$\lim\limits_{x\to\ 17}\frac{x^{2}-17^{2}}{(x-17)(x-12)}=[\frac{0}{0}]=\lim\limits_{x\to\ 17}\frac{x+17}{x-12}=2 \cdot \frac{17}{17-12} = \frac{34}{5}$$
\rozwStop
\odpStart
$\frac{34}{5}$
\odpStop
\testStart
A.$\frac{34}{5}$
B.$\infty$
C.$-\infty$
D.$0$
E.$\frac{34}{-5}$
F.$\frac{17}{12}$
G.$-\frac{34}{-5}$
H.$1$
I.$17$
\testStop
\kluczStart
A
\kluczStop



\zadStart{Przykład z Wikieł P 4.2b moja wersja nr 281}
Obliczyć granicę $\lim\limits_{x\to\ 17}\frac{x^{2}-17^{2}}{(x-17)(x-13)}$.
\zadStop
\rozwStart{Patryk Wirkus}{Martyna Czarnobaj}
$$\frac{x^{2}-17^{2}}{(x-17)(x-13)}=\frac{x+17}{x-13}$$

$$\lim\limits_{x\to\ 17}\frac{x^{2}-17^{2}}{(x-17)(x-13)}=[\frac{0}{0}]=\lim\limits_{x\to\ 17}\frac{x+17}{x-13}=2 \cdot \frac{17}{17-13} = \frac{34}{4}$$
\rozwStop
\odpStart
$\frac{34}{4}$
\odpStop
\testStart
A.$\frac{34}{4}$
B.$\infty$
C.$-\infty$
D.$0$
E.$\frac{34}{-4}$
F.$\frac{17}{13}$
G.$-\frac{34}{-4}$
H.$1$
I.$17$
\testStop
\kluczStart
A
\kluczStop



\zadStart{Przykład z Wikieł P 4.2b moja wersja nr 282}
Obliczyć granicę $\lim\limits_{x\to\ 17}\frac{x^{2}-17^{2}}{(x-17)(x-14)}$.
\zadStop
\rozwStart{Patryk Wirkus}{Martyna Czarnobaj}
$$\frac{x^{2}-17^{2}}{(x-17)(x-14)}=\frac{x+17}{x-14}$$

$$\lim\limits_{x\to\ 17}\frac{x^{2}-17^{2}}{(x-17)(x-14)}=[\frac{0}{0}]=\lim\limits_{x\to\ 17}\frac{x+17}{x-14}=2 \cdot \frac{17}{17-14} = \frac{34}{3}$$
\rozwStop
\odpStart
$\frac{34}{3}$
\odpStop
\testStart
A.$\frac{34}{3}$
B.$\infty$
C.$-\infty$
D.$0$
E.$\frac{34}{-3}$
F.$\frac{17}{14}$
G.$-\frac{34}{-3}$
H.$1$
I.$17$
\testStop
\kluczStart
A
\kluczStop



\zadStart{Przykład z Wikieł P 4.2b moja wersja nr 283}
Obliczyć granicę $\lim\limits_{x\to\ 17}\frac{x^{2}-17^{2}}{(x-17)(x-15)}$.
\zadStop
\rozwStart{Patryk Wirkus}{Martyna Czarnobaj}
$$\frac{x^{2}-17^{2}}{(x-17)(x-15)}=\frac{x+17}{x-15}$$

$$\lim\limits_{x\to\ 17}\frac{x^{2}-17^{2}}{(x-17)(x-15)}=[\frac{0}{0}]=\lim\limits_{x\to\ 17}\frac{x+17}{x-15}=2 \cdot \frac{17}{17-15} = \frac{34}{2}$$
\rozwStop
\odpStart
$\frac{34}{2}$
\odpStop
\testStart
A.$\frac{34}{2}$
B.$\infty$
C.$-\infty$
D.$0$
E.$\frac{34}{-2}$
F.$\frac{17}{15}$
G.$-\frac{34}{-2}$
H.$1$
I.$17$
\testStop
\kluczStart
A
\kluczStop



\zadStart{Przykład z Wikieł P 4.2b moja wersja nr 284}
Obliczyć granicę $\lim\limits_{x\to\ 17}\frac{x^{2}-17^{2}}{(x-17)(x-19)}$.
\zadStop
\rozwStart{Patryk Wirkus}{Martyna Czarnobaj}
$$\frac{x^{2}-17^{2}}{(x-17)(x-19)}=\frac{x+17}{x-19}$$

$$\lim\limits_{x\to\ 17}\frac{x^{2}-17^{2}}{(x-17)(x-19)}=[\frac{0}{0}]=\lim\limits_{x\to\ 17}\frac{x+17}{x-19}=2 \cdot \frac{17}{17-19} = \frac{34}{-2}$$
\rozwStop
\odpStart
$\frac{34}{-2}$
\odpStop
\testStart
A.$\frac{34}{-2}$
B.$\infty$
C.$-\infty$
D.$0$
E.$\frac{34}{2}$
F.$\frac{17}{19}$
G.$-\frac{34}{2}$
H.$1$
I.$17$
\testStop
\kluczStart
A
\kluczStop



\zadStart{Przykład z Wikieł P 4.2b moja wersja nr 285}
Obliczyć granicę $\lim\limits_{x\to\ 17}\frac{x^{2}-17^{2}}{(x-17)(x-20)}$.
\zadStop
\rozwStart{Patryk Wirkus}{Martyna Czarnobaj}
$$\frac{x^{2}-17^{2}}{(x-17)(x-20)}=\frac{x+17}{x-20}$$

$$\lim\limits_{x\to\ 17}\frac{x^{2}-17^{2}}{(x-17)(x-20)}=[\frac{0}{0}]=\lim\limits_{x\to\ 17}\frac{x+17}{x-20}=2 \cdot \frac{17}{17-20} = \frac{34}{-3}$$
\rozwStop
\odpStart
$\frac{34}{-3}$
\odpStop
\testStart
A.$\frac{34}{-3}$
B.$\infty$
C.$-\infty$
D.$0$
E.$\frac{34}{3}$
F.$\frac{17}{20}$
G.$-\frac{34}{3}$
H.$1$
I.$17$
\testStop
\kluczStart
A
\kluczStop



\zadStart{Przykład z Wikieł P 4.2b moja wersja nr 286}
Obliczyć granicę $\lim\limits_{x\to\ 17}\frac{x^{2}-17^{2}}{(x-17)(x-21)}$.
\zadStop
\rozwStart{Patryk Wirkus}{Martyna Czarnobaj}
$$\frac{x^{2}-17^{2}}{(x-17)(x-21)}=\frac{x+17}{x-21}$$

$$\lim\limits_{x\to\ 17}\frac{x^{2}-17^{2}}{(x-17)(x-21)}=[\frac{0}{0}]=\lim\limits_{x\to\ 17}\frac{x+17}{x-21}=2 \cdot \frac{17}{17-21} = \frac{34}{-4}$$
\rozwStop
\odpStart
$\frac{34}{-4}$
\odpStop
\testStart
A.$\frac{34}{-4}$
B.$\infty$
C.$-\infty$
D.$0$
E.$\frac{34}{4}$
F.$\frac{17}{21}$
G.$-\frac{34}{4}$
H.$1$
I.$17$
\testStop
\kluczStart
A
\kluczStop



\zadStart{Przykład z Wikieł P 4.2b moja wersja nr 287}
Obliczyć granicę $\lim\limits_{x\to\ 17}\frac{x^{2}-17^{2}}{(x-17)(x-22)}$.
\zadStop
\rozwStart{Patryk Wirkus}{Martyna Czarnobaj}
$$\frac{x^{2}-17^{2}}{(x-17)(x-22)}=\frac{x+17}{x-22}$$

$$\lim\limits_{x\to\ 17}\frac{x^{2}-17^{2}}{(x-17)(x-22)}=[\frac{0}{0}]=\lim\limits_{x\to\ 17}\frac{x+17}{x-22}=2 \cdot \frac{17}{17-22} = \frac{34}{-5}$$
\rozwStop
\odpStart
$\frac{34}{-5}$
\odpStop
\testStart
A.$\frac{34}{-5}$
B.$\infty$
C.$-\infty$
D.$0$
E.$\frac{34}{5}$
F.$\frac{17}{22}$
G.$-\frac{34}{5}$
H.$1$
I.$17$
\testStop
\kluczStart
A
\kluczStop



\zadStart{Przykład z Wikieł P 4.2b moja wersja nr 288}
Obliczyć granicę $\lim\limits_{x\to\ 17}\frac{x^{2}-17^{2}}{(x-17)(x-23)}$.
\zadStop
\rozwStart{Patryk Wirkus}{Martyna Czarnobaj}
$$\frac{x^{2}-17^{2}}{(x-17)(x-23)}=\frac{x+17}{x-23}$$

$$\lim\limits_{x\to\ 17}\frac{x^{2}-17^{2}}{(x-17)(x-23)}=[\frac{0}{0}]=\lim\limits_{x\to\ 17}\frac{x+17}{x-23}=2 \cdot \frac{17}{17-23} = \frac{34}{-6}$$
\rozwStop
\odpStart
$\frac{34}{-6}$
\odpStop
\testStart
A.$\frac{34}{-6}$
B.$\infty$
C.$-\infty$
D.$0$
E.$\frac{34}{6}$
F.$\frac{17}{23}$
G.$-\frac{34}{6}$
H.$1$
I.$17$
\testStop
\kluczStart
A
\kluczStop



\zadStart{Przykład z Wikieł P 4.2b moja wersja nr 289}
Obliczyć granicę $\lim\limits_{x\to\ 17}\frac{x^{2}-17^{2}}{(x-17)(x-24)}$.
\zadStop
\rozwStart{Patryk Wirkus}{Martyna Czarnobaj}
$$\frac{x^{2}-17^{2}}{(x-17)(x-24)}=\frac{x+17}{x-24}$$

$$\lim\limits_{x\to\ 17}\frac{x^{2}-17^{2}}{(x-17)(x-24)}=[\frac{0}{0}]=\lim\limits_{x\to\ 17}\frac{x+17}{x-24}=2 \cdot \frac{17}{17-24} = \frac{34}{-7}$$
\rozwStop
\odpStart
$\frac{34}{-7}$
\odpStop
\testStart
A.$\frac{34}{-7}$
B.$\infty$
C.$-\infty$
D.$0$
E.$\frac{34}{7}$
F.$\frac{17}{24}$
G.$-\frac{34}{7}$
H.$1$
I.$17$
\testStop
\kluczStart
A
\kluczStop



\zadStart{Przykład z Wikieł P 4.2b moja wersja nr 290}
Obliczyć granicę $\lim\limits_{x\to\ 17}\frac{x^{2}-17^{2}}{(x-17)(x-25)}$.
\zadStop
\rozwStart{Patryk Wirkus}{Martyna Czarnobaj}
$$\frac{x^{2}-17^{2}}{(x-17)(x-25)}=\frac{x+17}{x-25}$$

$$\lim\limits_{x\to\ 17}\frac{x^{2}-17^{2}}{(x-17)(x-25)}=[\frac{0}{0}]=\lim\limits_{x\to\ 17}\frac{x+17}{x-25}=2 \cdot \frac{17}{17-25} = \frac{34}{-8}$$
\rozwStop
\odpStart
$\frac{34}{-8}$
\odpStop
\testStart
A.$\frac{34}{-8}$
B.$\infty$
C.$-\infty$
D.$0$
E.$\frac{34}{8}$
F.$\frac{17}{25}$
G.$-\frac{34}{8}$
H.$1$
I.$17$
\testStop
\kluczStart
A
\kluczStop



\zadStart{Przykład z Wikieł P 4.2b moja wersja nr 291}
Obliczyć granicę $\lim\limits_{x\to\ 17}\frac{x^{2}-17^{2}}{(x-17)(x-26)}$.
\zadStop
\rozwStart{Patryk Wirkus}{Martyna Czarnobaj}
$$\frac{x^{2}-17^{2}}{(x-17)(x-26)}=\frac{x+17}{x-26}$$

$$\lim\limits_{x\to\ 17}\frac{x^{2}-17^{2}}{(x-17)(x-26)}=[\frac{0}{0}]=\lim\limits_{x\to\ 17}\frac{x+17}{x-26}=2 \cdot \frac{17}{17-26} = \frac{34}{-9}$$
\rozwStop
\odpStart
$\frac{34}{-9}$
\odpStop
\testStart
A.$\frac{34}{-9}$
B.$\infty$
C.$-\infty$
D.$0$
E.$\frac{34}{9}$
F.$\frac{17}{26}$
G.$-\frac{34}{9}$
H.$1$
I.$17$
\testStop
\kluczStart
A
\kluczStop



\zadStart{Przykład z Wikieł P 4.2b moja wersja nr 292}
Obliczyć granicę $\lim\limits_{x\to\ 17}\frac{x^{2}-17^{2}}{(x-17)(x-27)}$.
\zadStop
\rozwStart{Patryk Wirkus}{Martyna Czarnobaj}
$$\frac{x^{2}-17^{2}}{(x-17)(x-27)}=\frac{x+17}{x-27}$$

$$\lim\limits_{x\to\ 17}\frac{x^{2}-17^{2}}{(x-17)(x-27)}=[\frac{0}{0}]=\lim\limits_{x\to\ 17}\frac{x+17}{x-27}=2 \cdot \frac{17}{17-27} = \frac{34}{-10}$$
\rozwStop
\odpStart
$\frac{34}{-10}$
\odpStop
\testStart
A.$\frac{34}{-10}$
B.$\infty$
C.$-\infty$
D.$0$
E.$\frac{34}{10}$
F.$\frac{17}{27}$
G.$-\frac{34}{10}$
H.$1$
I.$17$
\testStop
\kluczStart
A
\kluczStop



\zadStart{Przykład z Wikieł P 4.2b moja wersja nr 293}
Obliczyć granicę $\lim\limits_{x\to\ 17}\frac{x^{2}-17^{2}}{(x-17)(x-28)}$.
\zadStop
\rozwStart{Patryk Wirkus}{Martyna Czarnobaj}
$$\frac{x^{2}-17^{2}}{(x-17)(x-28)}=\frac{x+17}{x-28}$$

$$\lim\limits_{x\to\ 17}\frac{x^{2}-17^{2}}{(x-17)(x-28)}=[\frac{0}{0}]=\lim\limits_{x\to\ 17}\frac{x+17}{x-28}=2 \cdot \frac{17}{17-28} = \frac{34}{-11}$$
\rozwStop
\odpStart
$\frac{34}{-11}$
\odpStop
\testStart
A.$\frac{34}{-11}$
B.$\infty$
C.$-\infty$
D.$0$
E.$\frac{34}{11}$
F.$\frac{17}{28}$
G.$-\frac{34}{11}$
H.$1$
I.$17$
\testStop
\kluczStart
A
\kluczStop



\zadStart{Przykład z Wikieł P 4.2b moja wersja nr 294}
Obliczyć granicę $\lim\limits_{x\to\ 17}\frac{x^{2}-17^{2}}{(x-17)(x-29)}$.
\zadStop
\rozwStart{Patryk Wirkus}{Martyna Czarnobaj}
$$\frac{x^{2}-17^{2}}{(x-17)(x-29)}=\frac{x+17}{x-29}$$

$$\lim\limits_{x\to\ 17}\frac{x^{2}-17^{2}}{(x-17)(x-29)}=[\frac{0}{0}]=\lim\limits_{x\to\ 17}\frac{x+17}{x-29}=2 \cdot \frac{17}{17-29} = \frac{34}{-12}$$
\rozwStop
\odpStart
$\frac{34}{-12}$
\odpStop
\testStart
A.$\frac{34}{-12}$
B.$\infty$
C.$-\infty$
D.$0$
E.$\frac{34}{12}$
F.$\frac{17}{29}$
G.$-\frac{34}{12}$
H.$1$
I.$17$
\testStop
\kluczStart
A
\kluczStop



\zadStart{Przykład z Wikieł P 4.2b moja wersja nr 295}
Obliczyć granicę $\lim\limits_{x\to\ 17}\frac{x^{2}-17^{2}}{(x-17)(x-30)}$.
\zadStop
\rozwStart{Patryk Wirkus}{Martyna Czarnobaj}
$$\frac{x^{2}-17^{2}}{(x-17)(x-30)}=\frac{x+17}{x-30}$$

$$\lim\limits_{x\to\ 17}\frac{x^{2}-17^{2}}{(x-17)(x-30)}=[\frac{0}{0}]=\lim\limits_{x\to\ 17}\frac{x+17}{x-30}=2 \cdot \frac{17}{17-30} = \frac{34}{-13}$$
\rozwStop
\odpStart
$\frac{34}{-13}$
\odpStop
\testStart
A.$\frac{34}{-13}$
B.$\infty$
C.$-\infty$
D.$0$
E.$\frac{34}{13}$
F.$\frac{17}{30}$
G.$-\frac{34}{13}$
H.$1$
I.$17$
\testStop
\kluczStart
A
\kluczStop



\zadStart{Przykład z Wikieł P 4.2b moja wersja nr 296}
Obliczyć granicę $\lim\limits_{x\to\ 17}\frac{x^{2}-17^{2}}{(x-17)(x-31)}$.
\zadStop
\rozwStart{Patryk Wirkus}{Martyna Czarnobaj}
$$\frac{x^{2}-17^{2}}{(x-17)(x-31)}=\frac{x+17}{x-31}$$

$$\lim\limits_{x\to\ 17}\frac{x^{2}-17^{2}}{(x-17)(x-31)}=[\frac{0}{0}]=\lim\limits_{x\to\ 17}\frac{x+17}{x-31}=2 \cdot \frac{17}{17-31} = \frac{34}{-14}$$
\rozwStop
\odpStart
$\frac{34}{-14}$
\odpStop
\testStart
A.$\frac{34}{-14}$
B.$\infty$
C.$-\infty$
D.$0$
E.$\frac{34}{14}$
F.$\frac{17}{31}$
G.$-\frac{34}{14}$
H.$1$
I.$17$
\testStop
\kluczStart
A
\kluczStop



\zadStart{Przykład z Wikieł P 4.2b moja wersja nr 297}
Obliczyć granicę $\lim\limits_{x\to\ 17}\frac{x^{2}-17^{2}}{(x-17)(x-32)}$.
\zadStop
\rozwStart{Patryk Wirkus}{Martyna Czarnobaj}
$$\frac{x^{2}-17^{2}}{(x-17)(x-32)}=\frac{x+17}{x-32}$$

$$\lim\limits_{x\to\ 17}\frac{x^{2}-17^{2}}{(x-17)(x-32)}=[\frac{0}{0}]=\lim\limits_{x\to\ 17}\frac{x+17}{x-32}=2 \cdot \frac{17}{17-32} = \frac{34}{-15}$$
\rozwStop
\odpStart
$\frac{34}{-15}$
\odpStop
\testStart
A.$\frac{34}{-15}$
B.$\infty$
C.$-\infty$
D.$0$
E.$\frac{34}{15}$
F.$\frac{17}{32}$
G.$-\frac{34}{15}$
H.$1$
I.$17$
\testStop
\kluczStart
A
\kluczStop



\zadStart{Przykład z Wikieł P 4.2b moja wersja nr 298}
Obliczyć granicę $\lim\limits_{x\to\ 17}\frac{x^{2}-17^{2}}{(x-17)(x-33)}$.
\zadStop
\rozwStart{Patryk Wirkus}{Martyna Czarnobaj}
$$\frac{x^{2}-17^{2}}{(x-17)(x-33)}=\frac{x+17}{x-33}$$

$$\lim\limits_{x\to\ 17}\frac{x^{2}-17^{2}}{(x-17)(x-33)}=[\frac{0}{0}]=\lim\limits_{x\to\ 17}\frac{x+17}{x-33}=2 \cdot \frac{17}{17-33} = \frac{34}{-16}$$
\rozwStop
\odpStart
$\frac{34}{-16}$
\odpStop
\testStart
A.$\frac{34}{-16}$
B.$\infty$
C.$-\infty$
D.$0$
E.$\frac{34}{16}$
F.$\frac{17}{33}$
G.$-\frac{34}{16}$
H.$1$
I.$17$
\testStop
\kluczStart
A
\kluczStop



\zadStart{Przykład z Wikieł P 4.2b moja wersja nr 299}
Obliczyć granicę $\lim\limits_{x\to\ 17}\frac{x^{2}-17^{2}}{(x-17)(x-35)}$.
\zadStop
\rozwStart{Patryk Wirkus}{Martyna Czarnobaj}
$$\frac{x^{2}-17^{2}}{(x-17)(x-35)}=\frac{x+17}{x-35}$$

$$\lim\limits_{x\to\ 17}\frac{x^{2}-17^{2}}{(x-17)(x-35)}=[\frac{0}{0}]=\lim\limits_{x\to\ 17}\frac{x+17}{x-35}=2 \cdot \frac{17}{17-35} = \frac{34}{-18}$$
\rozwStop
\odpStart
$\frac{34}{-18}$
\odpStop
\testStart
A.$\frac{34}{-18}$
B.$\infty$
C.$-\infty$
D.$0$
E.$\frac{34}{18}$
F.$\frac{17}{35}$
G.$-\frac{34}{18}$
H.$1$
I.$17$
\testStop
\kluczStart
A
\kluczStop



\zadStart{Przykład z Wikieł P 4.2b moja wersja nr 300}
Obliczyć granicę $\lim\limits_{x\to\ 17}\frac{x^{2}-17^{2}}{(x-17)(x-36)}$.
\zadStop
\rozwStart{Patryk Wirkus}{Martyna Czarnobaj}
$$\frac{x^{2}-17^{2}}{(x-17)(x-36)}=\frac{x+17}{x-36}$$

$$\lim\limits_{x\to\ 17}\frac{x^{2}-17^{2}}{(x-17)(x-36)}=[\frac{0}{0}]=\lim\limits_{x\to\ 17}\frac{x+17}{x-36}=2 \cdot \frac{17}{17-36} = \frac{34}{-19}$$
\rozwStop
\odpStart
$\frac{34}{-19}$
\odpStop
\testStart
A.$\frac{34}{-19}$
B.$\infty$
C.$-\infty$
D.$0$
E.$\frac{34}{19}$
F.$\frac{17}{36}$
G.$-\frac{34}{19}$
H.$1$
I.$17$
\testStop
\kluczStart
A
\kluczStop



\zadStart{Przykład z Wikieł P 4.2b moja wersja nr 301}
Obliczyć granicę $\lim\limits_{x\to\ 17}\frac{x^{2}-17^{2}}{(x-17)(x-37)}$.
\zadStop
\rozwStart{Patryk Wirkus}{Martyna Czarnobaj}
$$\frac{x^{2}-17^{2}}{(x-17)(x-37)}=\frac{x+17}{x-37}$$

$$\lim\limits_{x\to\ 17}\frac{x^{2}-17^{2}}{(x-17)(x-37)}=[\frac{0}{0}]=\lim\limits_{x\to\ 17}\frac{x+17}{x-37}=2 \cdot \frac{17}{17-37} = \frac{34}{-20}$$
\rozwStop
\odpStart
$\frac{34}{-20}$
\odpStop
\testStart
A.$\frac{34}{-20}$
B.$\infty$
C.$-\infty$
D.$0$
E.$\frac{34}{20}$
F.$\frac{17}{37}$
G.$-\frac{34}{20}$
H.$1$
I.$17$
\testStop
\kluczStart
A
\kluczStop



\zadStart{Przykład z Wikieł P 4.2b moja wersja nr 302}
Obliczyć granicę $\lim\limits_{x\to\ 17}\frac{x^{2}-17^{2}}{(x-17)(x-38)}$.
\zadStop
\rozwStart{Patryk Wirkus}{Martyna Czarnobaj}
$$\frac{x^{2}-17^{2}}{(x-17)(x-38)}=\frac{x+17}{x-38}$$

$$\lim\limits_{x\to\ 17}\frac{x^{2}-17^{2}}{(x-17)(x-38)}=[\frac{0}{0}]=\lim\limits_{x\to\ 17}\frac{x+17}{x-38}=2 \cdot \frac{17}{17-38} = \frac{34}{-21}$$
\rozwStop
\odpStart
$\frac{34}{-21}$
\odpStop
\testStart
A.$\frac{34}{-21}$
B.$\infty$
C.$-\infty$
D.$0$
E.$\frac{34}{21}$
F.$\frac{17}{38}$
G.$-\frac{34}{21}$
H.$1$
I.$17$
\testStop
\kluczStart
A
\kluczStop



\zadStart{Przykład z Wikieł P 4.2b moja wersja nr 303}
Obliczyć granicę $\lim\limits_{x\to\ 17}\frac{x^{2}-17^{2}}{(x-17)(x-39)}$.
\zadStop
\rozwStart{Patryk Wirkus}{Martyna Czarnobaj}
$$\frac{x^{2}-17^{2}}{(x-17)(x-39)}=\frac{x+17}{x-39}$$

$$\lim\limits_{x\to\ 17}\frac{x^{2}-17^{2}}{(x-17)(x-39)}=[\frac{0}{0}]=\lim\limits_{x\to\ 17}\frac{x+17}{x-39}=2 \cdot \frac{17}{17-39} = \frac{34}{-22}$$
\rozwStop
\odpStart
$\frac{34}{-22}$
\odpStop
\testStart
A.$\frac{34}{-22}$
B.$\infty$
C.$-\infty$
D.$0$
E.$\frac{34}{22}$
F.$\frac{17}{39}$
G.$-\frac{34}{22}$
H.$1$
I.$17$
\testStop
\kluczStart
A
\kluczStop



\zadStart{Przykład z Wikieł P 4.2b moja wersja nr 304}
Obliczyć granicę $\lim\limits_{x\to\ 17}\frac{x^{2}-17^{2}}{(x-17)(x-40)}$.
\zadStop
\rozwStart{Patryk Wirkus}{Martyna Czarnobaj}
$$\frac{x^{2}-17^{2}}{(x-17)(x-40)}=\frac{x+17}{x-40}$$

$$\lim\limits_{x\to\ 17}\frac{x^{2}-17^{2}}{(x-17)(x-40)}=[\frac{0}{0}]=\lim\limits_{x\to\ 17}\frac{x+17}{x-40}=2 \cdot \frac{17}{17-40} = \frac{34}{-23}$$
\rozwStop
\odpStart
$\frac{34}{-23}$
\odpStop
\testStart
A.$\frac{34}{-23}$
B.$\infty$
C.$-\infty$
D.$0$
E.$\frac{34}{23}$
F.$\frac{17}{40}$
G.$-\frac{34}{23}$
H.$1$
I.$17$
\testStop
\kluczStart
A
\kluczStop



\zadStart{Przykład z Wikieł P 4.2b moja wersja nr 305}
Obliczyć granicę $\lim\limits_{x\to\ 18}\frac{x^{2}-18^{2}}{(x-18)(x-5)}$.
\zadStop
\rozwStart{Patryk Wirkus}{Martyna Czarnobaj}
$$\frac{x^{2}-18^{2}}{(x-18)(x-5)}=\frac{x+18}{x-5}$$

$$\lim\limits_{x\to\ 18}\frac{x^{2}-18^{2}}{(x-18)(x-5)}=[\frac{0}{0}]=\lim\limits_{x\to\ 18}\frac{x+18}{x-5}=2 \cdot \frac{18}{18-5} = \frac{36}{13}$$
\rozwStop
\odpStart
$\frac{36}{13}$
\odpStop
\testStart
A.$\frac{36}{13}$
B.$\infty$
C.$-\infty$
D.$0$
E.$\frac{36}{-13}$
F.$\frac{18}{5}$
G.$-\frac{36}{-13}$
H.$1$
I.$18$
\testStop
\kluczStart
A
\kluczStop



\zadStart{Przykład z Wikieł P 4.2b moja wersja nr 306}
Obliczyć granicę $\lim\limits_{x\to\ 18}\frac{x^{2}-18^{2}}{(x-18)(x-7)}$.
\zadStop
\rozwStart{Patryk Wirkus}{Martyna Czarnobaj}
$$\frac{x^{2}-18^{2}}{(x-18)(x-7)}=\frac{x+18}{x-7}$$

$$\lim\limits_{x\to\ 18}\frac{x^{2}-18^{2}}{(x-18)(x-7)}=[\frac{0}{0}]=\lim\limits_{x\to\ 18}\frac{x+18}{x-7}=2 \cdot \frac{18}{18-7} = \frac{36}{11}$$
\rozwStop
\odpStart
$\frac{36}{11}$
\odpStop
\testStart
A.$\frac{36}{11}$
B.$\infty$
C.$-\infty$
D.$0$
E.$\frac{36}{-11}$
F.$\frac{18}{7}$
G.$-\frac{36}{-11}$
H.$1$
I.$18$
\testStop
\kluczStart
A
\kluczStop



\zadStart{Przykład z Wikieł P 4.2b moja wersja nr 307}
Obliczyć granicę $\lim\limits_{x\to\ 18}\frac{x^{2}-18^{2}}{(x-18)(x-11)}$.
\zadStop
\rozwStart{Patryk Wirkus}{Martyna Czarnobaj}
$$\frac{x^{2}-18^{2}}{(x-18)(x-11)}=\frac{x+18}{x-11}$$

$$\lim\limits_{x\to\ 18}\frac{x^{2}-18^{2}}{(x-18)(x-11)}=[\frac{0}{0}]=\lim\limits_{x\to\ 18}\frac{x+18}{x-11}=2 \cdot \frac{18}{18-11} = \frac{36}{7}$$
\rozwStop
\odpStart
$\frac{36}{7}$
\odpStop
\testStart
A.$\frac{36}{7}$
B.$\infty$
C.$-\infty$
D.$0$
E.$\frac{36}{-7}$
F.$\frac{18}{11}$
G.$-\frac{36}{-7}$
H.$1$
I.$18$
\testStop
\kluczStart
A
\kluczStop



\zadStart{Przykład z Wikieł P 4.2b moja wersja nr 308}
Obliczyć granicę $\lim\limits_{x\to\ 18}\frac{x^{2}-18^{2}}{(x-18)(x-13)}$.
\zadStop
\rozwStart{Patryk Wirkus}{Martyna Czarnobaj}
$$\frac{x^{2}-18^{2}}{(x-18)(x-13)}=\frac{x+18}{x-13}$$

$$\lim\limits_{x\to\ 18}\frac{x^{2}-18^{2}}{(x-18)(x-13)}=[\frac{0}{0}]=\lim\limits_{x\to\ 18}\frac{x+18}{x-13}=2 \cdot \frac{18}{18-13} = \frac{36}{5}$$
\rozwStop
\odpStart
$\frac{36}{5}$
\odpStop
\testStart
A.$\frac{36}{5}$
B.$\infty$
C.$-\infty$
D.$0$
E.$\frac{36}{-5}$
F.$\frac{18}{13}$
G.$-\frac{36}{-5}$
H.$1$
I.$18$
\testStop
\kluczStart
A
\kluczStop



\zadStart{Przykład z Wikieł P 4.2b moja wersja nr 309}
Obliczyć granicę $\lim\limits_{x\to\ 18}\frac{x^{2}-18^{2}}{(x-18)(x-23)}$.
\zadStop
\rozwStart{Patryk Wirkus}{Martyna Czarnobaj}
$$\frac{x^{2}-18^{2}}{(x-18)(x-23)}=\frac{x+18}{x-23}$$

$$\lim\limits_{x\to\ 18}\frac{x^{2}-18^{2}}{(x-18)(x-23)}=[\frac{0}{0}]=\lim\limits_{x\to\ 18}\frac{x+18}{x-23}=2 \cdot \frac{18}{18-23} = \frac{36}{-5}$$
\rozwStop
\odpStart
$\frac{36}{-5}$
\odpStop
\testStart
A.$\frac{36}{-5}$
B.$\infty$
C.$-\infty$
D.$0$
E.$\frac{36}{5}$
F.$\frac{18}{23}$
G.$-\frac{36}{5}$
H.$1$
I.$18$
\testStop
\kluczStart
A
\kluczStop



\zadStart{Przykład z Wikieł P 4.2b moja wersja nr 310}
Obliczyć granicę $\lim\limits_{x\to\ 18}\frac{x^{2}-18^{2}}{(x-18)(x-25)}$.
\zadStop
\rozwStart{Patryk Wirkus}{Martyna Czarnobaj}
$$\frac{x^{2}-18^{2}}{(x-18)(x-25)}=\frac{x+18}{x-25}$$

$$\lim\limits_{x\to\ 18}\frac{x^{2}-18^{2}}{(x-18)(x-25)}=[\frac{0}{0}]=\lim\limits_{x\to\ 18}\frac{x+18}{x-25}=2 \cdot \frac{18}{18-25} = \frac{36}{-7}$$
\rozwStop
\odpStart
$\frac{36}{-7}$
\odpStop
\testStart
A.$\frac{36}{-7}$
B.$\infty$
C.$-\infty$
D.$0$
E.$\frac{36}{7}$
F.$\frac{18}{25}$
G.$-\frac{36}{7}$
H.$1$
I.$18$
\testStop
\kluczStart
A
\kluczStop



\zadStart{Przykład z Wikieł P 4.2b moja wersja nr 311}
Obliczyć granicę $\lim\limits_{x\to\ 18}\frac{x^{2}-18^{2}}{(x-18)(x-29)}$.
\zadStop
\rozwStart{Patryk Wirkus}{Martyna Czarnobaj}
$$\frac{x^{2}-18^{2}}{(x-18)(x-29)}=\frac{x+18}{x-29}$$

$$\lim\limits_{x\to\ 18}\frac{x^{2}-18^{2}}{(x-18)(x-29)}=[\frac{0}{0}]=\lim\limits_{x\to\ 18}\frac{x+18}{x-29}=2 \cdot \frac{18}{18-29} = \frac{36}{-11}$$
\rozwStop
\odpStart
$\frac{36}{-11}$
\odpStop
\testStart
A.$\frac{36}{-11}$
B.$\infty$
C.$-\infty$
D.$0$
E.$\frac{36}{11}$
F.$\frac{18}{29}$
G.$-\frac{36}{11}$
H.$1$
I.$18$
\testStop
\kluczStart
A
\kluczStop



\zadStart{Przykład z Wikieł P 4.2b moja wersja nr 312}
Obliczyć granicę $\lim\limits_{x\to\ 18}\frac{x^{2}-18^{2}}{(x-18)(x-31)}$.
\zadStop
\rozwStart{Patryk Wirkus}{Martyna Czarnobaj}
$$\frac{x^{2}-18^{2}}{(x-18)(x-31)}=\frac{x+18}{x-31}$$

$$\lim\limits_{x\to\ 18}\frac{x^{2}-18^{2}}{(x-18)(x-31)}=[\frac{0}{0}]=\lim\limits_{x\to\ 18}\frac{x+18}{x-31}=2 \cdot \frac{18}{18-31} = \frac{36}{-13}$$
\rozwStop
\odpStart
$\frac{36}{-13}$
\odpStop
\testStart
A.$\frac{36}{-13}$
B.$\infty$
C.$-\infty$
D.$0$
E.$\frac{36}{13}$
F.$\frac{18}{31}$
G.$-\frac{36}{13}$
H.$1$
I.$18$
\testStop
\kluczStart
A
\kluczStop



\zadStart{Przykład z Wikieł P 4.2b moja wersja nr 313}
Obliczyć granicę $\lim\limits_{x\to\ 18}\frac{x^{2}-18^{2}}{(x-18)(x-35)}$.
\zadStop
\rozwStart{Patryk Wirkus}{Martyna Czarnobaj}
$$\frac{x^{2}-18^{2}}{(x-18)(x-35)}=\frac{x+18}{x-35}$$

$$\lim\limits_{x\to\ 18}\frac{x^{2}-18^{2}}{(x-18)(x-35)}=[\frac{0}{0}]=\lim\limits_{x\to\ 18}\frac{x+18}{x-35}=2 \cdot \frac{18}{18-35} = \frac{36}{-17}$$
\rozwStop
\odpStart
$\frac{36}{-17}$
\odpStop
\testStart
A.$\frac{36}{-17}$
B.$\infty$
C.$-\infty$
D.$0$
E.$\frac{36}{17}$
F.$\frac{18}{35}$
G.$-\frac{36}{17}$
H.$1$
I.$18$
\testStop
\kluczStart
A
\kluczStop



\zadStart{Przykład z Wikieł P 4.2b moja wersja nr 314}
Obliczyć granicę $\lim\limits_{x\to\ 18}\frac{x^{2}-18^{2}}{(x-18)(x-37)}$.
\zadStop
\rozwStart{Patryk Wirkus}{Martyna Czarnobaj}
$$\frac{x^{2}-18^{2}}{(x-18)(x-37)}=\frac{x+18}{x-37}$$

$$\lim\limits_{x\to\ 18}\frac{x^{2}-18^{2}}{(x-18)(x-37)}=[\frac{0}{0}]=\lim\limits_{x\to\ 18}\frac{x+18}{x-37}=2 \cdot \frac{18}{18-37} = \frac{36}{-19}$$
\rozwStop
\odpStart
$\frac{36}{-19}$
\odpStop
\testStart
A.$\frac{36}{-19}$
B.$\infty$
C.$-\infty$
D.$0$
E.$\frac{36}{19}$
F.$\frac{18}{37}$
G.$-\frac{36}{19}$
H.$1$
I.$18$
\testStop
\kluczStart
A
\kluczStop



\zadStart{Przykład z Wikieł P 4.2b moja wersja nr 315}
Obliczyć granicę $\lim\limits_{x\to\ 19}\frac{x^{2}-19^{2}}{(x-19)(x-2)}$.
\zadStop
\rozwStart{Patryk Wirkus}{Martyna Czarnobaj}
$$\frac{x^{2}-19^{2}}{(x-19)(x-2)}=\frac{x+19}{x-2}$$

$$\lim\limits_{x\to\ 19}\frac{x^{2}-19^{2}}{(x-19)(x-2)}=[\frac{0}{0}]=\lim\limits_{x\to\ 19}\frac{x+19}{x-2}=2 \cdot \frac{19}{19-2} = \frac{38}{17}$$
\rozwStop
\odpStart
$\frac{38}{17}$
\odpStop
\testStart
A.$\frac{38}{17}$
B.$\infty$
C.$-\infty$
D.$0$
E.$\frac{38}{-17}$
F.$\frac{19}{2}$
G.$-\frac{38}{-17}$
H.$1$
I.$19$
\testStop
\kluczStart
A
\kluczStop



\zadStart{Przykład z Wikieł P 4.2b moja wersja nr 316}
Obliczyć granicę $\lim\limits_{x\to\ 19}\frac{x^{2}-19^{2}}{(x-19)(x-3)}$.
\zadStop
\rozwStart{Patryk Wirkus}{Martyna Czarnobaj}
$$\frac{x^{2}-19^{2}}{(x-19)(x-3)}=\frac{x+19}{x-3}$$

$$\lim\limits_{x\to\ 19}\frac{x^{2}-19^{2}}{(x-19)(x-3)}=[\frac{0}{0}]=\lim\limits_{x\to\ 19}\frac{x+19}{x-3}=2 \cdot \frac{19}{19-3} = \frac{38}{16}$$
\rozwStop
\odpStart
$\frac{38}{16}$
\odpStop
\testStart
A.$\frac{38}{16}$
B.$\infty$
C.$-\infty$
D.$0$
E.$\frac{38}{-16}$
F.$\frac{19}{3}$
G.$-\frac{38}{-16}$
H.$1$
I.$19$
\testStop
\kluczStart
A
\kluczStop



\zadStart{Przykład z Wikieł P 4.2b moja wersja nr 317}
Obliczyć granicę $\lim\limits_{x\to\ 19}\frac{x^{2}-19^{2}}{(x-19)(x-4)}$.
\zadStop
\rozwStart{Patryk Wirkus}{Martyna Czarnobaj}
$$\frac{x^{2}-19^{2}}{(x-19)(x-4)}=\frac{x+19}{x-4}$$

$$\lim\limits_{x\to\ 19}\frac{x^{2}-19^{2}}{(x-19)(x-4)}=[\frac{0}{0}]=\lim\limits_{x\to\ 19}\frac{x+19}{x-4}=2 \cdot \frac{19}{19-4} = \frac{38}{15}$$
\rozwStop
\odpStart
$\frac{38}{15}$
\odpStop
\testStart
A.$\frac{38}{15}$
B.$\infty$
C.$-\infty$
D.$0$
E.$\frac{38}{-15}$
F.$\frac{19}{4}$
G.$-\frac{38}{-15}$
H.$1$
I.$19$
\testStop
\kluczStart
A
\kluczStop



\zadStart{Przykład z Wikieł P 4.2b moja wersja nr 318}
Obliczyć granicę $\lim\limits_{x\to\ 19}\frac{x^{2}-19^{2}}{(x-19)(x-5)}$.
\zadStop
\rozwStart{Patryk Wirkus}{Martyna Czarnobaj}
$$\frac{x^{2}-19^{2}}{(x-19)(x-5)}=\frac{x+19}{x-5}$$

$$\lim\limits_{x\to\ 19}\frac{x^{2}-19^{2}}{(x-19)(x-5)}=[\frac{0}{0}]=\lim\limits_{x\to\ 19}\frac{x+19}{x-5}=2 \cdot \frac{19}{19-5} = \frac{38}{14}$$
\rozwStop
\odpStart
$\frac{38}{14}$
\odpStop
\testStart
A.$\frac{38}{14}$
B.$\infty$
C.$-\infty$
D.$0$
E.$\frac{38}{-14}$
F.$\frac{19}{5}$
G.$-\frac{38}{-14}$
H.$1$
I.$19$
\testStop
\kluczStart
A
\kluczStop



\zadStart{Przykład z Wikieł P 4.2b moja wersja nr 319}
Obliczyć granicę $\lim\limits_{x\to\ 19}\frac{x^{2}-19^{2}}{(x-19)(x-6)}$.
\zadStop
\rozwStart{Patryk Wirkus}{Martyna Czarnobaj}
$$\frac{x^{2}-19^{2}}{(x-19)(x-6)}=\frac{x+19}{x-6}$$

$$\lim\limits_{x\to\ 19}\frac{x^{2}-19^{2}}{(x-19)(x-6)}=[\frac{0}{0}]=\lim\limits_{x\to\ 19}\frac{x+19}{x-6}=2 \cdot \frac{19}{19-6} = \frac{38}{13}$$
\rozwStop
\odpStart
$\frac{38}{13}$
\odpStop
\testStart
A.$\frac{38}{13}$
B.$\infty$
C.$-\infty$
D.$0$
E.$\frac{38}{-13}$
F.$\frac{19}{6}$
G.$-\frac{38}{-13}$
H.$1$
I.$19$
\testStop
\kluczStart
A
\kluczStop



\zadStart{Przykład z Wikieł P 4.2b moja wersja nr 320}
Obliczyć granicę $\lim\limits_{x\to\ 19}\frac{x^{2}-19^{2}}{(x-19)(x-7)}$.
\zadStop
\rozwStart{Patryk Wirkus}{Martyna Czarnobaj}
$$\frac{x^{2}-19^{2}}{(x-19)(x-7)}=\frac{x+19}{x-7}$$

$$\lim\limits_{x\to\ 19}\frac{x^{2}-19^{2}}{(x-19)(x-7)}=[\frac{0}{0}]=\lim\limits_{x\to\ 19}\frac{x+19}{x-7}=2 \cdot \frac{19}{19-7} = \frac{38}{12}$$
\rozwStop
\odpStart
$\frac{38}{12}$
\odpStop
\testStart
A.$\frac{38}{12}$
B.$\infty$
C.$-\infty$
D.$0$
E.$\frac{38}{-12}$
F.$\frac{19}{7}$
G.$-\frac{38}{-12}$
H.$1$
I.$19$
\testStop
\kluczStart
A
\kluczStop



\zadStart{Przykład z Wikieł P 4.2b moja wersja nr 321}
Obliczyć granicę $\lim\limits_{x\to\ 19}\frac{x^{2}-19^{2}}{(x-19)(x-8)}$.
\zadStop
\rozwStart{Patryk Wirkus}{Martyna Czarnobaj}
$$\frac{x^{2}-19^{2}}{(x-19)(x-8)}=\frac{x+19}{x-8}$$

$$\lim\limits_{x\to\ 19}\frac{x^{2}-19^{2}}{(x-19)(x-8)}=[\frac{0}{0}]=\lim\limits_{x\to\ 19}\frac{x+19}{x-8}=2 \cdot \frac{19}{19-8} = \frac{38}{11}$$
\rozwStop
\odpStart
$\frac{38}{11}$
\odpStop
\testStart
A.$\frac{38}{11}$
B.$\infty$
C.$-\infty$
D.$0$
E.$\frac{38}{-11}$
F.$\frac{19}{8}$
G.$-\frac{38}{-11}$
H.$1$
I.$19$
\testStop
\kluczStart
A
\kluczStop



\zadStart{Przykład z Wikieł P 4.2b moja wersja nr 322}
Obliczyć granicę $\lim\limits_{x\to\ 19}\frac{x^{2}-19^{2}}{(x-19)(x-9)}$.
\zadStop
\rozwStart{Patryk Wirkus}{Martyna Czarnobaj}
$$\frac{x^{2}-19^{2}}{(x-19)(x-9)}=\frac{x+19}{x-9}$$

$$\lim\limits_{x\to\ 19}\frac{x^{2}-19^{2}}{(x-19)(x-9)}=[\frac{0}{0}]=\lim\limits_{x\to\ 19}\frac{x+19}{x-9}=2 \cdot \frac{19}{19-9} = \frac{38}{10}$$
\rozwStop
\odpStart
$\frac{38}{10}$
\odpStop
\testStart
A.$\frac{38}{10}$
B.$\infty$
C.$-\infty$
D.$0$
E.$\frac{38}{-10}$
F.$\frac{19}{9}$
G.$-\frac{38}{-10}$
H.$1$
I.$19$
\testStop
\kluczStart
A
\kluczStop



\zadStart{Przykład z Wikieł P 4.2b moja wersja nr 323}
Obliczyć granicę $\lim\limits_{x\to\ 19}\frac{x^{2}-19^{2}}{(x-19)(x-10)}$.
\zadStop
\rozwStart{Patryk Wirkus}{Martyna Czarnobaj}
$$\frac{x^{2}-19^{2}}{(x-19)(x-10)}=\frac{x+19}{x-10}$$

$$\lim\limits_{x\to\ 19}\frac{x^{2}-19^{2}}{(x-19)(x-10)}=[\frac{0}{0}]=\lim\limits_{x\to\ 19}\frac{x+19}{x-10}=2 \cdot \frac{19}{19-10} = \frac{38}{9}$$
\rozwStop
\odpStart
$\frac{38}{9}$
\odpStop
\testStart
A.$\frac{38}{9}$
B.$\infty$
C.$-\infty$
D.$0$
E.$\frac{38}{-9}$
F.$\frac{19}{10}$
G.$-\frac{38}{-9}$
H.$1$
I.$19$
\testStop
\kluczStart
A
\kluczStop



\zadStart{Przykład z Wikieł P 4.2b moja wersja nr 324}
Obliczyć granicę $\lim\limits_{x\to\ 19}\frac{x^{2}-19^{2}}{(x-19)(x-11)}$.
\zadStop
\rozwStart{Patryk Wirkus}{Martyna Czarnobaj}
$$\frac{x^{2}-19^{2}}{(x-19)(x-11)}=\frac{x+19}{x-11}$$

$$\lim\limits_{x\to\ 19}\frac{x^{2}-19^{2}}{(x-19)(x-11)}=[\frac{0}{0}]=\lim\limits_{x\to\ 19}\frac{x+19}{x-11}=2 \cdot \frac{19}{19-11} = \frac{38}{8}$$
\rozwStop
\odpStart
$\frac{38}{8}$
\odpStop
\testStart
A.$\frac{38}{8}$
B.$\infty$
C.$-\infty$
D.$0$
E.$\frac{38}{-8}$
F.$\frac{19}{11}$
G.$-\frac{38}{-8}$
H.$1$
I.$19$
\testStop
\kluczStart
A
\kluczStop



\zadStart{Przykład z Wikieł P 4.2b moja wersja nr 325}
Obliczyć granicę $\lim\limits_{x\to\ 19}\frac{x^{2}-19^{2}}{(x-19)(x-12)}$.
\zadStop
\rozwStart{Patryk Wirkus}{Martyna Czarnobaj}
$$\frac{x^{2}-19^{2}}{(x-19)(x-12)}=\frac{x+19}{x-12}$$

$$\lim\limits_{x\to\ 19}\frac{x^{2}-19^{2}}{(x-19)(x-12)}=[\frac{0}{0}]=\lim\limits_{x\to\ 19}\frac{x+19}{x-12}=2 \cdot \frac{19}{19-12} = \frac{38}{7}$$
\rozwStop
\odpStart
$\frac{38}{7}$
\odpStop
\testStart
A.$\frac{38}{7}$
B.$\infty$
C.$-\infty$
D.$0$
E.$\frac{38}{-7}$
F.$\frac{19}{12}$
G.$-\frac{38}{-7}$
H.$1$
I.$19$
\testStop
\kluczStart
A
\kluczStop



\zadStart{Przykład z Wikieł P 4.2b moja wersja nr 326}
Obliczyć granicę $\lim\limits_{x\to\ 19}\frac{x^{2}-19^{2}}{(x-19)(x-13)}$.
\zadStop
\rozwStart{Patryk Wirkus}{Martyna Czarnobaj}
$$\frac{x^{2}-19^{2}}{(x-19)(x-13)}=\frac{x+19}{x-13}$$

$$\lim\limits_{x\to\ 19}\frac{x^{2}-19^{2}}{(x-19)(x-13)}=[\frac{0}{0}]=\lim\limits_{x\to\ 19}\frac{x+19}{x-13}=2 \cdot \frac{19}{19-13} = \frac{38}{6}$$
\rozwStop
\odpStart
$\frac{38}{6}$
\odpStop
\testStart
A.$\frac{38}{6}$
B.$\infty$
C.$-\infty$
D.$0$
E.$\frac{38}{-6}$
F.$\frac{19}{13}$
G.$-\frac{38}{-6}$
H.$1$
I.$19$
\testStop
\kluczStart
A
\kluczStop



\zadStart{Przykład z Wikieł P 4.2b moja wersja nr 327}
Obliczyć granicę $\lim\limits_{x\to\ 19}\frac{x^{2}-19^{2}}{(x-19)(x-14)}$.
\zadStop
\rozwStart{Patryk Wirkus}{Martyna Czarnobaj}
$$\frac{x^{2}-19^{2}}{(x-19)(x-14)}=\frac{x+19}{x-14}$$

$$\lim\limits_{x\to\ 19}\frac{x^{2}-19^{2}}{(x-19)(x-14)}=[\frac{0}{0}]=\lim\limits_{x\to\ 19}\frac{x+19}{x-14}=2 \cdot \frac{19}{19-14} = \frac{38}{5}$$
\rozwStop
\odpStart
$\frac{38}{5}$
\odpStop
\testStart
A.$\frac{38}{5}$
B.$\infty$
C.$-\infty$
D.$0$
E.$\frac{38}{-5}$
F.$\frac{19}{14}$
G.$-\frac{38}{-5}$
H.$1$
I.$19$
\testStop
\kluczStart
A
\kluczStop



\zadStart{Przykład z Wikieł P 4.2b moja wersja nr 328}
Obliczyć granicę $\lim\limits_{x\to\ 19}\frac{x^{2}-19^{2}}{(x-19)(x-15)}$.
\zadStop
\rozwStart{Patryk Wirkus}{Martyna Czarnobaj}
$$\frac{x^{2}-19^{2}}{(x-19)(x-15)}=\frac{x+19}{x-15}$$

$$\lim\limits_{x\to\ 19}\frac{x^{2}-19^{2}}{(x-19)(x-15)}=[\frac{0}{0}]=\lim\limits_{x\to\ 19}\frac{x+19}{x-15}=2 \cdot \frac{19}{19-15} = \frac{38}{4}$$
\rozwStop
\odpStart
$\frac{38}{4}$
\odpStop
\testStart
A.$\frac{38}{4}$
B.$\infty$
C.$-\infty$
D.$0$
E.$\frac{38}{-4}$
F.$\frac{19}{15}$
G.$-\frac{38}{-4}$
H.$1$
I.$19$
\testStop
\kluczStart
A
\kluczStop



\zadStart{Przykład z Wikieł P 4.2b moja wersja nr 329}
Obliczyć granicę $\lim\limits_{x\to\ 19}\frac{x^{2}-19^{2}}{(x-19)(x-16)}$.
\zadStop
\rozwStart{Patryk Wirkus}{Martyna Czarnobaj}
$$\frac{x^{2}-19^{2}}{(x-19)(x-16)}=\frac{x+19}{x-16}$$

$$\lim\limits_{x\to\ 19}\frac{x^{2}-19^{2}}{(x-19)(x-16)}=[\frac{0}{0}]=\lim\limits_{x\to\ 19}\frac{x+19}{x-16}=2 \cdot \frac{19}{19-16} = \frac{38}{3}$$
\rozwStop
\odpStart
$\frac{38}{3}$
\odpStop
\testStart
A.$\frac{38}{3}$
B.$\infty$
C.$-\infty$
D.$0$
E.$\frac{38}{-3}$
F.$\frac{19}{16}$
G.$-\frac{38}{-3}$
H.$1$
I.$19$
\testStop
\kluczStart
A
\kluczStop



\zadStart{Przykład z Wikieł P 4.2b moja wersja nr 330}
Obliczyć granicę $\lim\limits_{x\to\ 19}\frac{x^{2}-19^{2}}{(x-19)(x-17)}$.
\zadStop
\rozwStart{Patryk Wirkus}{Martyna Czarnobaj}
$$\frac{x^{2}-19^{2}}{(x-19)(x-17)}=\frac{x+19}{x-17}$$

$$\lim\limits_{x\to\ 19}\frac{x^{2}-19^{2}}{(x-19)(x-17)}=[\frac{0}{0}]=\lim\limits_{x\to\ 19}\frac{x+19}{x-17}=2 \cdot \frac{19}{19-17} = \frac{38}{2}$$
\rozwStop
\odpStart
$\frac{38}{2}$
\odpStop
\testStart
A.$\frac{38}{2}$
B.$\infty$
C.$-\infty$
D.$0$
E.$\frac{38}{-2}$
F.$\frac{19}{17}$
G.$-\frac{38}{-2}$
H.$1$
I.$19$
\testStop
\kluczStart
A
\kluczStop



\zadStart{Przykład z Wikieł P 4.2b moja wersja nr 331}
Obliczyć granicę $\lim\limits_{x\to\ 19}\frac{x^{2}-19^{2}}{(x-19)(x-21)}$.
\zadStop
\rozwStart{Patryk Wirkus}{Martyna Czarnobaj}
$$\frac{x^{2}-19^{2}}{(x-19)(x-21)}=\frac{x+19}{x-21}$$

$$\lim\limits_{x\to\ 19}\frac{x^{2}-19^{2}}{(x-19)(x-21)}=[\frac{0}{0}]=\lim\limits_{x\to\ 19}\frac{x+19}{x-21}=2 \cdot \frac{19}{19-21} = \frac{38}{-2}$$
\rozwStop
\odpStart
$\frac{38}{-2}$
\odpStop
\testStart
A.$\frac{38}{-2}$
B.$\infty$
C.$-\infty$
D.$0$
E.$\frac{38}{2}$
F.$\frac{19}{21}$
G.$-\frac{38}{2}$
H.$1$
I.$19$
\testStop
\kluczStart
A
\kluczStop



\zadStart{Przykład z Wikieł P 4.2b moja wersja nr 332}
Obliczyć granicę $\lim\limits_{x\to\ 19}\frac{x^{2}-19^{2}}{(x-19)(x-22)}$.
\zadStop
\rozwStart{Patryk Wirkus}{Martyna Czarnobaj}
$$\frac{x^{2}-19^{2}}{(x-19)(x-22)}=\frac{x+19}{x-22}$$

$$\lim\limits_{x\to\ 19}\frac{x^{2}-19^{2}}{(x-19)(x-22)}=[\frac{0}{0}]=\lim\limits_{x\to\ 19}\frac{x+19}{x-22}=2 \cdot \frac{19}{19-22} = \frac{38}{-3}$$
\rozwStop
\odpStart
$\frac{38}{-3}$
\odpStop
\testStart
A.$\frac{38}{-3}$
B.$\infty$
C.$-\infty$
D.$0$
E.$\frac{38}{3}$
F.$\frac{19}{22}$
G.$-\frac{38}{3}$
H.$1$
I.$19$
\testStop
\kluczStart
A
\kluczStop



\zadStart{Przykład z Wikieł P 4.2b moja wersja nr 333}
Obliczyć granicę $\lim\limits_{x\to\ 19}\frac{x^{2}-19^{2}}{(x-19)(x-23)}$.
\zadStop
\rozwStart{Patryk Wirkus}{Martyna Czarnobaj}
$$\frac{x^{2}-19^{2}}{(x-19)(x-23)}=\frac{x+19}{x-23}$$

$$\lim\limits_{x\to\ 19}\frac{x^{2}-19^{2}}{(x-19)(x-23)}=[\frac{0}{0}]=\lim\limits_{x\to\ 19}\frac{x+19}{x-23}=2 \cdot \frac{19}{19-23} = \frac{38}{-4}$$
\rozwStop
\odpStart
$\frac{38}{-4}$
\odpStop
\testStart
A.$\frac{38}{-4}$
B.$\infty$
C.$-\infty$
D.$0$
E.$\frac{38}{4}$
F.$\frac{19}{23}$
G.$-\frac{38}{4}$
H.$1$
I.$19$
\testStop
\kluczStart
A
\kluczStop



\zadStart{Przykład z Wikieł P 4.2b moja wersja nr 334}
Obliczyć granicę $\lim\limits_{x\to\ 19}\frac{x^{2}-19^{2}}{(x-19)(x-24)}$.
\zadStop
\rozwStart{Patryk Wirkus}{Martyna Czarnobaj}
$$\frac{x^{2}-19^{2}}{(x-19)(x-24)}=\frac{x+19}{x-24}$$

$$\lim\limits_{x\to\ 19}\frac{x^{2}-19^{2}}{(x-19)(x-24)}=[\frac{0}{0}]=\lim\limits_{x\to\ 19}\frac{x+19}{x-24}=2 \cdot \frac{19}{19-24} = \frac{38}{-5}$$
\rozwStop
\odpStart
$\frac{38}{-5}$
\odpStop
\testStart
A.$\frac{38}{-5}$
B.$\infty$
C.$-\infty$
D.$0$
E.$\frac{38}{5}$
F.$\frac{19}{24}$
G.$-\frac{38}{5}$
H.$1$
I.$19$
\testStop
\kluczStart
A
\kluczStop



\zadStart{Przykład z Wikieł P 4.2b moja wersja nr 335}
Obliczyć granicę $\lim\limits_{x\to\ 19}\frac{x^{2}-19^{2}}{(x-19)(x-25)}$.
\zadStop
\rozwStart{Patryk Wirkus}{Martyna Czarnobaj}
$$\frac{x^{2}-19^{2}}{(x-19)(x-25)}=\frac{x+19}{x-25}$$

$$\lim\limits_{x\to\ 19}\frac{x^{2}-19^{2}}{(x-19)(x-25)}=[\frac{0}{0}]=\lim\limits_{x\to\ 19}\frac{x+19}{x-25}=2 \cdot \frac{19}{19-25} = \frac{38}{-6}$$
\rozwStop
\odpStart
$\frac{38}{-6}$
\odpStop
\testStart
A.$\frac{38}{-6}$
B.$\infty$
C.$-\infty$
D.$0$
E.$\frac{38}{6}$
F.$\frac{19}{25}$
G.$-\frac{38}{6}$
H.$1$
I.$19$
\testStop
\kluczStart
A
\kluczStop



\zadStart{Przykład z Wikieł P 4.2b moja wersja nr 336}
Obliczyć granicę $\lim\limits_{x\to\ 19}\frac{x^{2}-19^{2}}{(x-19)(x-26)}$.
\zadStop
\rozwStart{Patryk Wirkus}{Martyna Czarnobaj}
$$\frac{x^{2}-19^{2}}{(x-19)(x-26)}=\frac{x+19}{x-26}$$

$$\lim\limits_{x\to\ 19}\frac{x^{2}-19^{2}}{(x-19)(x-26)}=[\frac{0}{0}]=\lim\limits_{x\to\ 19}\frac{x+19}{x-26}=2 \cdot \frac{19}{19-26} = \frac{38}{-7}$$
\rozwStop
\odpStart
$\frac{38}{-7}$
\odpStop
\testStart
A.$\frac{38}{-7}$
B.$\infty$
C.$-\infty$
D.$0$
E.$\frac{38}{7}$
F.$\frac{19}{26}$
G.$-\frac{38}{7}$
H.$1$
I.$19$
\testStop
\kluczStart
A
\kluczStop



\zadStart{Przykład z Wikieł P 4.2b moja wersja nr 337}
Obliczyć granicę $\lim\limits_{x\to\ 19}\frac{x^{2}-19^{2}}{(x-19)(x-27)}$.
\zadStop
\rozwStart{Patryk Wirkus}{Martyna Czarnobaj}
$$\frac{x^{2}-19^{2}}{(x-19)(x-27)}=\frac{x+19}{x-27}$$

$$\lim\limits_{x\to\ 19}\frac{x^{2}-19^{2}}{(x-19)(x-27)}=[\frac{0}{0}]=\lim\limits_{x\to\ 19}\frac{x+19}{x-27}=2 \cdot \frac{19}{19-27} = \frac{38}{-8}$$
\rozwStop
\odpStart
$\frac{38}{-8}$
\odpStop
\testStart
A.$\frac{38}{-8}$
B.$\infty$
C.$-\infty$
D.$0$
E.$\frac{38}{8}$
F.$\frac{19}{27}$
G.$-\frac{38}{8}$
H.$1$
I.$19$
\testStop
\kluczStart
A
\kluczStop



\zadStart{Przykład z Wikieł P 4.2b moja wersja nr 338}
Obliczyć granicę $\lim\limits_{x\to\ 19}\frac{x^{2}-19^{2}}{(x-19)(x-28)}$.
\zadStop
\rozwStart{Patryk Wirkus}{Martyna Czarnobaj}
$$\frac{x^{2}-19^{2}}{(x-19)(x-28)}=\frac{x+19}{x-28}$$

$$\lim\limits_{x\to\ 19}\frac{x^{2}-19^{2}}{(x-19)(x-28)}=[\frac{0}{0}]=\lim\limits_{x\to\ 19}\frac{x+19}{x-28}=2 \cdot \frac{19}{19-28} = \frac{38}{-9}$$
\rozwStop
\odpStart
$\frac{38}{-9}$
\odpStop
\testStart
A.$\frac{38}{-9}$
B.$\infty$
C.$-\infty$
D.$0$
E.$\frac{38}{9}$
F.$\frac{19}{28}$
G.$-\frac{38}{9}$
H.$1$
I.$19$
\testStop
\kluczStart
A
\kluczStop



\zadStart{Przykład z Wikieł P 4.2b moja wersja nr 339}
Obliczyć granicę $\lim\limits_{x\to\ 19}\frac{x^{2}-19^{2}}{(x-19)(x-29)}$.
\zadStop
\rozwStart{Patryk Wirkus}{Martyna Czarnobaj}
$$\frac{x^{2}-19^{2}}{(x-19)(x-29)}=\frac{x+19}{x-29}$$

$$\lim\limits_{x\to\ 19}\frac{x^{2}-19^{2}}{(x-19)(x-29)}=[\frac{0}{0}]=\lim\limits_{x\to\ 19}\frac{x+19}{x-29}=2 \cdot \frac{19}{19-29} = \frac{38}{-10}$$
\rozwStop
\odpStart
$\frac{38}{-10}$
\odpStop
\testStart
A.$\frac{38}{-10}$
B.$\infty$
C.$-\infty$
D.$0$
E.$\frac{38}{10}$
F.$\frac{19}{29}$
G.$-\frac{38}{10}$
H.$1$
I.$19$
\testStop
\kluczStart
A
\kluczStop



\zadStart{Przykład z Wikieł P 4.2b moja wersja nr 340}
Obliczyć granicę $\lim\limits_{x\to\ 19}\frac{x^{2}-19^{2}}{(x-19)(x-30)}$.
\zadStop
\rozwStart{Patryk Wirkus}{Martyna Czarnobaj}
$$\frac{x^{2}-19^{2}}{(x-19)(x-30)}=\frac{x+19}{x-30}$$

$$\lim\limits_{x\to\ 19}\frac{x^{2}-19^{2}}{(x-19)(x-30)}=[\frac{0}{0}]=\lim\limits_{x\to\ 19}\frac{x+19}{x-30}=2 \cdot \frac{19}{19-30} = \frac{38}{-11}$$
\rozwStop
\odpStart
$\frac{38}{-11}$
\odpStop
\testStart
A.$\frac{38}{-11}$
B.$\infty$
C.$-\infty$
D.$0$
E.$\frac{38}{11}$
F.$\frac{19}{30}$
G.$-\frac{38}{11}$
H.$1$
I.$19$
\testStop
\kluczStart
A
\kluczStop



\zadStart{Przykład z Wikieł P 4.2b moja wersja nr 341}
Obliczyć granicę $\lim\limits_{x\to\ 19}\frac{x^{2}-19^{2}}{(x-19)(x-31)}$.
\zadStop
\rozwStart{Patryk Wirkus}{Martyna Czarnobaj}
$$\frac{x^{2}-19^{2}}{(x-19)(x-31)}=\frac{x+19}{x-31}$$

$$\lim\limits_{x\to\ 19}\frac{x^{2}-19^{2}}{(x-19)(x-31)}=[\frac{0}{0}]=\lim\limits_{x\to\ 19}\frac{x+19}{x-31}=2 \cdot \frac{19}{19-31} = \frac{38}{-12}$$
\rozwStop
\odpStart
$\frac{38}{-12}$
\odpStop
\testStart
A.$\frac{38}{-12}$
B.$\infty$
C.$-\infty$
D.$0$
E.$\frac{38}{12}$
F.$\frac{19}{31}$
G.$-\frac{38}{12}$
H.$1$
I.$19$
\testStop
\kluczStart
A
\kluczStop



\zadStart{Przykład z Wikieł P 4.2b moja wersja nr 342}
Obliczyć granicę $\lim\limits_{x\to\ 19}\frac{x^{2}-19^{2}}{(x-19)(x-32)}$.
\zadStop
\rozwStart{Patryk Wirkus}{Martyna Czarnobaj}
$$\frac{x^{2}-19^{2}}{(x-19)(x-32)}=\frac{x+19}{x-32}$$

$$\lim\limits_{x\to\ 19}\frac{x^{2}-19^{2}}{(x-19)(x-32)}=[\frac{0}{0}]=\lim\limits_{x\to\ 19}\frac{x+19}{x-32}=2 \cdot \frac{19}{19-32} = \frac{38}{-13}$$
\rozwStop
\odpStart
$\frac{38}{-13}$
\odpStop
\testStart
A.$\frac{38}{-13}$
B.$\infty$
C.$-\infty$
D.$0$
E.$\frac{38}{13}$
F.$\frac{19}{32}$
G.$-\frac{38}{13}$
H.$1$
I.$19$
\testStop
\kluczStart
A
\kluczStop



\zadStart{Przykład z Wikieł P 4.2b moja wersja nr 343}
Obliczyć granicę $\lim\limits_{x\to\ 19}\frac{x^{2}-19^{2}}{(x-19)(x-33)}$.
\zadStop
\rozwStart{Patryk Wirkus}{Martyna Czarnobaj}
$$\frac{x^{2}-19^{2}}{(x-19)(x-33)}=\frac{x+19}{x-33}$$

$$\lim\limits_{x\to\ 19}\frac{x^{2}-19^{2}}{(x-19)(x-33)}=[\frac{0}{0}]=\lim\limits_{x\to\ 19}\frac{x+19}{x-33}=2 \cdot \frac{19}{19-33} = \frac{38}{-14}$$
\rozwStop
\odpStart
$\frac{38}{-14}$
\odpStop
\testStart
A.$\frac{38}{-14}$
B.$\infty$
C.$-\infty$
D.$0$
E.$\frac{38}{14}$
F.$\frac{19}{33}$
G.$-\frac{38}{14}$
H.$1$
I.$19$
\testStop
\kluczStart
A
\kluczStop



\zadStart{Przykład z Wikieł P 4.2b moja wersja nr 344}
Obliczyć granicę $\lim\limits_{x\to\ 19}\frac{x^{2}-19^{2}}{(x-19)(x-34)}$.
\zadStop
\rozwStart{Patryk Wirkus}{Martyna Czarnobaj}
$$\frac{x^{2}-19^{2}}{(x-19)(x-34)}=\frac{x+19}{x-34}$$

$$\lim\limits_{x\to\ 19}\frac{x^{2}-19^{2}}{(x-19)(x-34)}=[\frac{0}{0}]=\lim\limits_{x\to\ 19}\frac{x+19}{x-34}=2 \cdot \frac{19}{19-34} = \frac{38}{-15}$$
\rozwStop
\odpStart
$\frac{38}{-15}$
\odpStop
\testStart
A.$\frac{38}{-15}$
B.$\infty$
C.$-\infty$
D.$0$
E.$\frac{38}{15}$
F.$\frac{19}{34}$
G.$-\frac{38}{15}$
H.$1$
I.$19$
\testStop
\kluczStart
A
\kluczStop



\zadStart{Przykład z Wikieł P 4.2b moja wersja nr 345}
Obliczyć granicę $\lim\limits_{x\to\ 19}\frac{x^{2}-19^{2}}{(x-19)(x-35)}$.
\zadStop
\rozwStart{Patryk Wirkus}{Martyna Czarnobaj}
$$\frac{x^{2}-19^{2}}{(x-19)(x-35)}=\frac{x+19}{x-35}$$

$$\lim\limits_{x\to\ 19}\frac{x^{2}-19^{2}}{(x-19)(x-35)}=[\frac{0}{0}]=\lim\limits_{x\to\ 19}\frac{x+19}{x-35}=2 \cdot \frac{19}{19-35} = \frac{38}{-16}$$
\rozwStop
\odpStart
$\frac{38}{-16}$
\odpStop
\testStart
A.$\frac{38}{-16}$
B.$\infty$
C.$-\infty$
D.$0$
E.$\frac{38}{16}$
F.$\frac{19}{35}$
G.$-\frac{38}{16}$
H.$1$
I.$19$
\testStop
\kluczStart
A
\kluczStop



\zadStart{Przykład z Wikieł P 4.2b moja wersja nr 346}
Obliczyć granicę $\lim\limits_{x\to\ 19}\frac{x^{2}-19^{2}}{(x-19)(x-36)}$.
\zadStop
\rozwStart{Patryk Wirkus}{Martyna Czarnobaj}
$$\frac{x^{2}-19^{2}}{(x-19)(x-36)}=\frac{x+19}{x-36}$$

$$\lim\limits_{x\to\ 19}\frac{x^{2}-19^{2}}{(x-19)(x-36)}=[\frac{0}{0}]=\lim\limits_{x\to\ 19}\frac{x+19}{x-36}=2 \cdot \frac{19}{19-36} = \frac{38}{-17}$$
\rozwStop
\odpStart
$\frac{38}{-17}$
\odpStop
\testStart
A.$\frac{38}{-17}$
B.$\infty$
C.$-\infty$
D.$0$
E.$\frac{38}{17}$
F.$\frac{19}{36}$
G.$-\frac{38}{17}$
H.$1$
I.$19$
\testStop
\kluczStart
A
\kluczStop



\zadStart{Przykład z Wikieł P 4.2b moja wersja nr 347}
Obliczyć granicę $\lim\limits_{x\to\ 19}\frac{x^{2}-19^{2}}{(x-19)(x-37)}$.
\zadStop
\rozwStart{Patryk Wirkus}{Martyna Czarnobaj}
$$\frac{x^{2}-19^{2}}{(x-19)(x-37)}=\frac{x+19}{x-37}$$

$$\lim\limits_{x\to\ 19}\frac{x^{2}-19^{2}}{(x-19)(x-37)}=[\frac{0}{0}]=\lim\limits_{x\to\ 19}\frac{x+19}{x-37}=2 \cdot \frac{19}{19-37} = \frac{38}{-18}$$
\rozwStop
\odpStart
$\frac{38}{-18}$
\odpStop
\testStart
A.$\frac{38}{-18}$
B.$\infty$
C.$-\infty$
D.$0$
E.$\frac{38}{18}$
F.$\frac{19}{37}$
G.$-\frac{38}{18}$
H.$1$
I.$19$
\testStop
\kluczStart
A
\kluczStop



\zadStart{Przykład z Wikieł P 4.2b moja wersja nr 348}
Obliczyć granicę $\lim\limits_{x\to\ 19}\frac{x^{2}-19^{2}}{(x-19)(x-39)}$.
\zadStop
\rozwStart{Patryk Wirkus}{Martyna Czarnobaj}
$$\frac{x^{2}-19^{2}}{(x-19)(x-39)}=\frac{x+19}{x-39}$$

$$\lim\limits_{x\to\ 19}\frac{x^{2}-19^{2}}{(x-19)(x-39)}=[\frac{0}{0}]=\lim\limits_{x\to\ 19}\frac{x+19}{x-39}=2 \cdot \frac{19}{19-39} = \frac{38}{-20}$$
\rozwStop
\odpStart
$\frac{38}{-20}$
\odpStop
\testStart
A.$\frac{38}{-20}$
B.$\infty$
C.$-\infty$
D.$0$
E.$\frac{38}{20}$
F.$\frac{19}{39}$
G.$-\frac{38}{20}$
H.$1$
I.$19$
\testStop
\kluczStart
A
\kluczStop



\zadStart{Przykład z Wikieł P 4.2b moja wersja nr 349}
Obliczyć granicę $\lim\limits_{x\to\ 19}\frac{x^{2}-19^{2}}{(x-19)(x-40)}$.
\zadStop
\rozwStart{Patryk Wirkus}{Martyna Czarnobaj}
$$\frac{x^{2}-19^{2}}{(x-19)(x-40)}=\frac{x+19}{x-40}$$

$$\lim\limits_{x\to\ 19}\frac{x^{2}-19^{2}}{(x-19)(x-40)}=[\frac{0}{0}]=\lim\limits_{x\to\ 19}\frac{x+19}{x-40}=2 \cdot \frac{19}{19-40} = \frac{38}{-21}$$
\rozwStop
\odpStart
$\frac{38}{-21}$
\odpStop
\testStart
A.$\frac{38}{-21}$
B.$\infty$
C.$-\infty$
D.$0$
E.$\frac{38}{21}$
F.$\frac{19}{40}$
G.$-\frac{38}{21}$
H.$1$
I.$19$
\testStop
\kluczStart
A
\kluczStop



\zadStart{Przykład z Wikieł P 4.2b moja wersja nr 350}
Obliczyć granicę $\lim\limits_{x\to\ 20}\frac{x^{2}-20^{2}}{(x-20)(x-3)}$.
\zadStop
\rozwStart{Patryk Wirkus}{Martyna Czarnobaj}
$$\frac{x^{2}-20^{2}}{(x-20)(x-3)}=\frac{x+20}{x-3}$$

$$\lim\limits_{x\to\ 20}\frac{x^{2}-20^{2}}{(x-20)(x-3)}=[\frac{0}{0}]=\lim\limits_{x\to\ 20}\frac{x+20}{x-3}=2 \cdot \frac{20}{20-3} = \frac{40}{17}$$
\rozwStop
\odpStart
$\frac{40}{17}$
\odpStop
\testStart
A.$\frac{40}{17}$
B.$\infty$
C.$-\infty$
D.$0$
E.$\frac{40}{-17}$
F.$\frac{20}{3}$
G.$-\frac{40}{-17}$
H.$1$
I.$20$
\testStop
\kluczStart
A
\kluczStop



\zadStart{Przykład z Wikieł P 4.2b moja wersja nr 351}
Obliczyć granicę $\lim\limits_{x\to\ 20}\frac{x^{2}-20^{2}}{(x-20)(x-7)}$.
\zadStop
\rozwStart{Patryk Wirkus}{Martyna Czarnobaj}
$$\frac{x^{2}-20^{2}}{(x-20)(x-7)}=\frac{x+20}{x-7}$$

$$\lim\limits_{x\to\ 20}\frac{x^{2}-20^{2}}{(x-20)(x-7)}=[\frac{0}{0}]=\lim\limits_{x\to\ 20}\frac{x+20}{x-7}=2 \cdot \frac{20}{20-7} = \frac{40}{13}$$
\rozwStop
\odpStart
$\frac{40}{13}$
\odpStop
\testStart
A.$\frac{40}{13}$
B.$\infty$
C.$-\infty$
D.$0$
E.$\frac{40}{-13}$
F.$\frac{20}{7}$
G.$-\frac{40}{-13}$
H.$1$
I.$20$
\testStop
\kluczStart
A
\kluczStop



\zadStart{Przykład z Wikieł P 4.2b moja wersja nr 352}
Obliczyć granicę $\lim\limits_{x\to\ 20}\frac{x^{2}-20^{2}}{(x-20)(x-9)}$.
\zadStop
\rozwStart{Patryk Wirkus}{Martyna Czarnobaj}
$$\frac{x^{2}-20^{2}}{(x-20)(x-9)}=\frac{x+20}{x-9}$$

$$\lim\limits_{x\to\ 20}\frac{x^{2}-20^{2}}{(x-20)(x-9)}=[\frac{0}{0}]=\lim\limits_{x\to\ 20}\frac{x+20}{x-9}=2 \cdot \frac{20}{20-9} = \frac{40}{11}$$
\rozwStop
\odpStart
$\frac{40}{11}$
\odpStop
\testStart
A.$\frac{40}{11}$
B.$\infty$
C.$-\infty$
D.$0$
E.$\frac{40}{-11}$
F.$\frac{20}{9}$
G.$-\frac{40}{-11}$
H.$1$
I.$20$
\testStop
\kluczStart
A
\kluczStop



\zadStart{Przykład z Wikieł P 4.2b moja wersja nr 353}
Obliczyć granicę $\lim\limits_{x\to\ 20}\frac{x^{2}-20^{2}}{(x-20)(x-11)}$.
\zadStop
\rozwStart{Patryk Wirkus}{Martyna Czarnobaj}
$$\frac{x^{2}-20^{2}}{(x-20)(x-11)}=\frac{x+20}{x-11}$$

$$\lim\limits_{x\to\ 20}\frac{x^{2}-20^{2}}{(x-20)(x-11)}=[\frac{0}{0}]=\lim\limits_{x\to\ 20}\frac{x+20}{x-11}=2 \cdot \frac{20}{20-11} = \frac{40}{9}$$
\rozwStop
\odpStart
$\frac{40}{9}$
\odpStop
\testStart
A.$\frac{40}{9}$
B.$\infty$
C.$-\infty$
D.$0$
E.$\frac{40}{-9}$
F.$\frac{20}{11}$
G.$-\frac{40}{-9}$
H.$1$
I.$20$
\testStop
\kluczStart
A
\kluczStop



\zadStart{Przykład z Wikieł P 4.2b moja wersja nr 354}
Obliczyć granicę $\lim\limits_{x\to\ 20}\frac{x^{2}-20^{2}}{(x-20)(x-13)}$.
\zadStop
\rozwStart{Patryk Wirkus}{Martyna Czarnobaj}
$$\frac{x^{2}-20^{2}}{(x-20)(x-13)}=\frac{x+20}{x-13}$$

$$\lim\limits_{x\to\ 20}\frac{x^{2}-20^{2}}{(x-20)(x-13)}=[\frac{0}{0}]=\lim\limits_{x\to\ 20}\frac{x+20}{x-13}=2 \cdot \frac{20}{20-13} = \frac{40}{7}$$
\rozwStop
\odpStart
$\frac{40}{7}$
\odpStop
\testStart
A.$\frac{40}{7}$
B.$\infty$
C.$-\infty$
D.$0$
E.$\frac{40}{-7}$
F.$\frac{20}{13}$
G.$-\frac{40}{-7}$
H.$1$
I.$20$
\testStop
\kluczStart
A
\kluczStop



\zadStart{Przykład z Wikieł P 4.2b moja wersja nr 355}
Obliczyć granicę $\lim\limits_{x\to\ 20}\frac{x^{2}-20^{2}}{(x-20)(x-17)}$.
\zadStop
\rozwStart{Patryk Wirkus}{Martyna Czarnobaj}
$$\frac{x^{2}-20^{2}}{(x-20)(x-17)}=\frac{x+20}{x-17}$$

$$\lim\limits_{x\to\ 20}\frac{x^{2}-20^{2}}{(x-20)(x-17)}=[\frac{0}{0}]=\lim\limits_{x\to\ 20}\frac{x+20}{x-17}=2 \cdot \frac{20}{20-17} = \frac{40}{3}$$
\rozwStop
\odpStart
$\frac{40}{3}$
\odpStop
\testStart
A.$\frac{40}{3}$
B.$\infty$
C.$-\infty$
D.$0$
E.$\frac{40}{-3}$
F.$\frac{20}{17}$
G.$-\frac{40}{-3}$
H.$1$
I.$20$
\testStop
\kluczStart
A
\kluczStop



\zadStart{Przykład z Wikieł P 4.2b moja wersja nr 356}
Obliczyć granicę $\lim\limits_{x\to\ 20}\frac{x^{2}-20^{2}}{(x-20)(x-23)}$.
\zadStop
\rozwStart{Patryk Wirkus}{Martyna Czarnobaj}
$$\frac{x^{2}-20^{2}}{(x-20)(x-23)}=\frac{x+20}{x-23}$$

$$\lim\limits_{x\to\ 20}\frac{x^{2}-20^{2}}{(x-20)(x-23)}=[\frac{0}{0}]=\lim\limits_{x\to\ 20}\frac{x+20}{x-23}=2 \cdot \frac{20}{20-23} = \frac{40}{-3}$$
\rozwStop
\odpStart
$\frac{40}{-3}$
\odpStop
\testStart
A.$\frac{40}{-3}$
B.$\infty$
C.$-\infty$
D.$0$
E.$\frac{40}{3}$
F.$\frac{20}{23}$
G.$-\frac{40}{3}$
H.$1$
I.$20$
\testStop
\kluczStart
A
\kluczStop



\zadStart{Przykład z Wikieł P 4.2b moja wersja nr 357}
Obliczyć granicę $\lim\limits_{x\to\ 20}\frac{x^{2}-20^{2}}{(x-20)(x-27)}$.
\zadStop
\rozwStart{Patryk Wirkus}{Martyna Czarnobaj}
$$\frac{x^{2}-20^{2}}{(x-20)(x-27)}=\frac{x+20}{x-27}$$

$$\lim\limits_{x\to\ 20}\frac{x^{2}-20^{2}}{(x-20)(x-27)}=[\frac{0}{0}]=\lim\limits_{x\to\ 20}\frac{x+20}{x-27}=2 \cdot \frac{20}{20-27} = \frac{40}{-7}$$
\rozwStop
\odpStart
$\frac{40}{-7}$
\odpStop
\testStart
A.$\frac{40}{-7}$
B.$\infty$
C.$-\infty$
D.$0$
E.$\frac{40}{7}$
F.$\frac{20}{27}$
G.$-\frac{40}{7}$
H.$1$
I.$20$
\testStop
\kluczStart
A
\kluczStop



\zadStart{Przykład z Wikieł P 4.2b moja wersja nr 358}
Obliczyć granicę $\lim\limits_{x\to\ 20}\frac{x^{2}-20^{2}}{(x-20)(x-29)}$.
\zadStop
\rozwStart{Patryk Wirkus}{Martyna Czarnobaj}
$$\frac{x^{2}-20^{2}}{(x-20)(x-29)}=\frac{x+20}{x-29}$$

$$\lim\limits_{x\to\ 20}\frac{x^{2}-20^{2}}{(x-20)(x-29)}=[\frac{0}{0}]=\lim\limits_{x\to\ 20}\frac{x+20}{x-29}=2 \cdot \frac{20}{20-29} = \frac{40}{-9}$$
\rozwStop
\odpStart
$\frac{40}{-9}$
\odpStop
\testStart
A.$\frac{40}{-9}$
B.$\infty$
C.$-\infty$
D.$0$
E.$\frac{40}{9}$
F.$\frac{20}{29}$
G.$-\frac{40}{9}$
H.$1$
I.$20$
\testStop
\kluczStart
A
\kluczStop



\zadStart{Przykład z Wikieł P 4.2b moja wersja nr 359}
Obliczyć granicę $\lim\limits_{x\to\ 20}\frac{x^{2}-20^{2}}{(x-20)(x-31)}$.
\zadStop
\rozwStart{Patryk Wirkus}{Martyna Czarnobaj}
$$\frac{x^{2}-20^{2}}{(x-20)(x-31)}=\frac{x+20}{x-31}$$

$$\lim\limits_{x\to\ 20}\frac{x^{2}-20^{2}}{(x-20)(x-31)}=[\frac{0}{0}]=\lim\limits_{x\to\ 20}\frac{x+20}{x-31}=2 \cdot \frac{20}{20-31} = \frac{40}{-11}$$
\rozwStop
\odpStart
$\frac{40}{-11}$
\odpStop
\testStart
A.$\frac{40}{-11}$
B.$\infty$
C.$-\infty$
D.$0$
E.$\frac{40}{11}$
F.$\frac{20}{31}$
G.$-\frac{40}{11}$
H.$1$
I.$20$
\testStop
\kluczStart
A
\kluczStop



\zadStart{Przykład z Wikieł P 4.2b moja wersja nr 360}
Obliczyć granicę $\lim\limits_{x\to\ 20}\frac{x^{2}-20^{2}}{(x-20)(x-33)}$.
\zadStop
\rozwStart{Patryk Wirkus}{Martyna Czarnobaj}
$$\frac{x^{2}-20^{2}}{(x-20)(x-33)}=\frac{x+20}{x-33}$$

$$\lim\limits_{x\to\ 20}\frac{x^{2}-20^{2}}{(x-20)(x-33)}=[\frac{0}{0}]=\lim\limits_{x\to\ 20}\frac{x+20}{x-33}=2 \cdot \frac{20}{20-33} = \frac{40}{-13}$$
\rozwStop
\odpStart
$\frac{40}{-13}$
\odpStop
\testStart
A.$\frac{40}{-13}$
B.$\infty$
C.$-\infty$
D.$0$
E.$\frac{40}{13}$
F.$\frac{20}{33}$
G.$-\frac{40}{13}$
H.$1$
I.$20$
\testStop
\kluczStart
A
\kluczStop



\zadStart{Przykład z Wikieł P 4.2b moja wersja nr 361}
Obliczyć granicę $\lim\limits_{x\to\ 20}\frac{x^{2}-20^{2}}{(x-20)(x-37)}$.
\zadStop
\rozwStart{Patryk Wirkus}{Martyna Czarnobaj}
$$\frac{x^{2}-20^{2}}{(x-20)(x-37)}=\frac{x+20}{x-37}$$

$$\lim\limits_{x\to\ 20}\frac{x^{2}-20^{2}}{(x-20)(x-37)}=[\frac{0}{0}]=\lim\limits_{x\to\ 20}\frac{x+20}{x-37}=2 \cdot \frac{20}{20-37} = \frac{40}{-17}$$
\rozwStop
\odpStart
$\frac{40}{-17}$
\odpStop
\testStart
A.$\frac{40}{-17}$
B.$\infty$
C.$-\infty$
D.$0$
E.$\frac{40}{17}$
F.$\frac{20}{37}$
G.$-\frac{40}{17}$
H.$1$
I.$20$
\testStop
\kluczStart
A
\kluczStop



\zadStart{Przykład z Wikieł P 4.2b moja wersja nr 362}
Obliczyć granicę $\lim\limits_{x\to\ 20}\frac{x^{2}-20^{2}}{(x-20)(x-39)}$.
\zadStop
\rozwStart{Patryk Wirkus}{Martyna Czarnobaj}
$$\frac{x^{2}-20^{2}}{(x-20)(x-39)}=\frac{x+20}{x-39}$$

$$\lim\limits_{x\to\ 20}\frac{x^{2}-20^{2}}{(x-20)(x-39)}=[\frac{0}{0}]=\lim\limits_{x\to\ 20}\frac{x+20}{x-39}=2 \cdot \frac{20}{20-39} = \frac{40}{-19}$$
\rozwStop
\odpStart
$\frac{40}{-19}$
\odpStop
\testStart
A.$\frac{40}{-19}$
B.$\infty$
C.$-\infty$
D.$0$
E.$\frac{40}{19}$
F.$\frac{20}{39}$
G.$-\frac{40}{19}$
H.$1$
I.$20$
\testStop
\kluczStart
A
\kluczStop



\zadStart{Przykład z Wikieł P 4.2b moja wersja nr 363}
Obliczyć granicę $\lim\limits_{x\to\ 21}\frac{x^{2}-21^{2}}{(x-21)(x-2)}$.
\zadStop
\rozwStart{Patryk Wirkus}{Martyna Czarnobaj}
$$\frac{x^{2}-21^{2}}{(x-21)(x-2)}=\frac{x+21}{x-2}$$

$$\lim\limits_{x\to\ 21}\frac{x^{2}-21^{2}}{(x-21)(x-2)}=[\frac{0}{0}]=\lim\limits_{x\to\ 21}\frac{x+21}{x-2}=2 \cdot \frac{21}{21-2} = \frac{42}{19}$$
\rozwStop
\odpStart
$\frac{42}{19}$
\odpStop
\testStart
A.$\frac{42}{19}$
B.$\infty$
C.$-\infty$
D.$0$
E.$\frac{42}{-19}$
F.$\frac{21}{2}$
G.$-\frac{42}{-19}$
H.$1$
I.$21$
\testStop
\kluczStart
A
\kluczStop



\zadStart{Przykład z Wikieł P 4.2b moja wersja nr 364}
Obliczyć granicę $\lim\limits_{x\to\ 21}\frac{x^{2}-21^{2}}{(x-21)(x-4)}$.
\zadStop
\rozwStart{Patryk Wirkus}{Martyna Czarnobaj}
$$\frac{x^{2}-21^{2}}{(x-21)(x-4)}=\frac{x+21}{x-4}$$

$$\lim\limits_{x\to\ 21}\frac{x^{2}-21^{2}}{(x-21)(x-4)}=[\frac{0}{0}]=\lim\limits_{x\to\ 21}\frac{x+21}{x-4}=2 \cdot \frac{21}{21-4} = \frac{42}{17}$$
\rozwStop
\odpStart
$\frac{42}{17}$
\odpStop
\testStart
A.$\frac{42}{17}$
B.$\infty$
C.$-\infty$
D.$0$
E.$\frac{42}{-17}$
F.$\frac{21}{4}$
G.$-\frac{42}{-17}$
H.$1$
I.$21$
\testStop
\kluczStart
A
\kluczStop



\zadStart{Przykład z Wikieł P 4.2b moja wersja nr 365}
Obliczyć granicę $\lim\limits_{x\to\ 21}\frac{x^{2}-21^{2}}{(x-21)(x-5)}$.
\zadStop
\rozwStart{Patryk Wirkus}{Martyna Czarnobaj}
$$\frac{x^{2}-21^{2}}{(x-21)(x-5)}=\frac{x+21}{x-5}$$

$$\lim\limits_{x\to\ 21}\frac{x^{2}-21^{2}}{(x-21)(x-5)}=[\frac{0}{0}]=\lim\limits_{x\to\ 21}\frac{x+21}{x-5}=2 \cdot \frac{21}{21-5} = \frac{42}{16}$$
\rozwStop
\odpStart
$\frac{42}{16}$
\odpStop
\testStart
A.$\frac{42}{16}$
B.$\infty$
C.$-\infty$
D.$0$
E.$\frac{42}{-16}$
F.$\frac{21}{5}$
G.$-\frac{42}{-16}$
H.$1$
I.$21$
\testStop
\kluczStart
A
\kluczStop



\zadStart{Przykład z Wikieł P 4.2b moja wersja nr 366}
Obliczyć granicę $\lim\limits_{x\to\ 21}\frac{x^{2}-21^{2}}{(x-21)(x-8)}$.
\zadStop
\rozwStart{Patryk Wirkus}{Martyna Czarnobaj}
$$\frac{x^{2}-21^{2}}{(x-21)(x-8)}=\frac{x+21}{x-8}$$

$$\lim\limits_{x\to\ 21}\frac{x^{2}-21^{2}}{(x-21)(x-8)}=[\frac{0}{0}]=\lim\limits_{x\to\ 21}\frac{x+21}{x-8}=2 \cdot \frac{21}{21-8} = \frac{42}{13}$$
\rozwStop
\odpStart
$\frac{42}{13}$
\odpStop
\testStart
A.$\frac{42}{13}$
B.$\infty$
C.$-\infty$
D.$0$
E.$\frac{42}{-13}$
F.$\frac{21}{8}$
G.$-\frac{42}{-13}$
H.$1$
I.$21$
\testStop
\kluczStart
A
\kluczStop



\zadStart{Przykład z Wikieł P 4.2b moja wersja nr 367}
Obliczyć granicę $\lim\limits_{x\to\ 21}\frac{x^{2}-21^{2}}{(x-21)(x-10)}$.
\zadStop
\rozwStart{Patryk Wirkus}{Martyna Czarnobaj}
$$\frac{x^{2}-21^{2}}{(x-21)(x-10)}=\frac{x+21}{x-10}$$

$$\lim\limits_{x\to\ 21}\frac{x^{2}-21^{2}}{(x-21)(x-10)}=[\frac{0}{0}]=\lim\limits_{x\to\ 21}\frac{x+21}{x-10}=2 \cdot \frac{21}{21-10} = \frac{42}{11}$$
\rozwStop
\odpStart
$\frac{42}{11}$
\odpStop
\testStart
A.$\frac{42}{11}$
B.$\infty$
C.$-\infty$
D.$0$
E.$\frac{42}{-11}$
F.$\frac{21}{10}$
G.$-\frac{42}{-11}$
H.$1$
I.$21$
\testStop
\kluczStart
A
\kluczStop



\zadStart{Przykład z Wikieł P 4.2b moja wersja nr 368}
Obliczyć granicę $\lim\limits_{x\to\ 21}\frac{x^{2}-21^{2}}{(x-21)(x-11)}$.
\zadStop
\rozwStart{Patryk Wirkus}{Martyna Czarnobaj}
$$\frac{x^{2}-21^{2}}{(x-21)(x-11)}=\frac{x+21}{x-11}$$

$$\lim\limits_{x\to\ 21}\frac{x^{2}-21^{2}}{(x-21)(x-11)}=[\frac{0}{0}]=\lim\limits_{x\to\ 21}\frac{x+21}{x-11}=2 \cdot \frac{21}{21-11} = \frac{42}{10}$$
\rozwStop
\odpStart
$\frac{42}{10}$
\odpStop
\testStart
A.$\frac{42}{10}$
B.$\infty$
C.$-\infty$
D.$0$
E.$\frac{42}{-10}$
F.$\frac{21}{11}$
G.$-\frac{42}{-10}$
H.$1$
I.$21$
\testStop
\kluczStart
A
\kluczStop



\zadStart{Przykład z Wikieł P 4.2b moja wersja nr 369}
Obliczyć granicę $\lim\limits_{x\to\ 21}\frac{x^{2}-21^{2}}{(x-21)(x-13)}$.
\zadStop
\rozwStart{Patryk Wirkus}{Martyna Czarnobaj}
$$\frac{x^{2}-21^{2}}{(x-21)(x-13)}=\frac{x+21}{x-13}$$

$$\lim\limits_{x\to\ 21}\frac{x^{2}-21^{2}}{(x-21)(x-13)}=[\frac{0}{0}]=\lim\limits_{x\to\ 21}\frac{x+21}{x-13}=2 \cdot \frac{21}{21-13} = \frac{42}{8}$$
\rozwStop
\odpStart
$\frac{42}{8}$
\odpStop
\testStart
A.$\frac{42}{8}$
B.$\infty$
C.$-\infty$
D.$0$
E.$\frac{42}{-8}$
F.$\frac{21}{13}$
G.$-\frac{42}{-8}$
H.$1$
I.$21$
\testStop
\kluczStart
A
\kluczStop



\zadStart{Przykład z Wikieł P 4.2b moja wersja nr 370}
Obliczyć granicę $\lim\limits_{x\to\ 21}\frac{x^{2}-21^{2}}{(x-21)(x-16)}$.
\zadStop
\rozwStart{Patryk Wirkus}{Martyna Czarnobaj}
$$\frac{x^{2}-21^{2}}{(x-21)(x-16)}=\frac{x+21}{x-16}$$

$$\lim\limits_{x\to\ 21}\frac{x^{2}-21^{2}}{(x-21)(x-16)}=[\frac{0}{0}]=\lim\limits_{x\to\ 21}\frac{x+21}{x-16}=2 \cdot \frac{21}{21-16} = \frac{42}{5}$$
\rozwStop
\odpStart
$\frac{42}{5}$
\odpStop
\testStart
A.$\frac{42}{5}$
B.$\infty$
C.$-\infty$
D.$0$
E.$\frac{42}{-5}$
F.$\frac{21}{16}$
G.$-\frac{42}{-5}$
H.$1$
I.$21$
\testStop
\kluczStart
A
\kluczStop



\zadStart{Przykład z Wikieł P 4.2b moja wersja nr 371}
Obliczyć granicę $\lim\limits_{x\to\ 21}\frac{x^{2}-21^{2}}{(x-21)(x-17)}$.
\zadStop
\rozwStart{Patryk Wirkus}{Martyna Czarnobaj}
$$\frac{x^{2}-21^{2}}{(x-21)(x-17)}=\frac{x+21}{x-17}$$

$$\lim\limits_{x\to\ 21}\frac{x^{2}-21^{2}}{(x-21)(x-17)}=[\frac{0}{0}]=\lim\limits_{x\to\ 21}\frac{x+21}{x-17}=2 \cdot \frac{21}{21-17} = \frac{42}{4}$$
\rozwStop
\odpStart
$\frac{42}{4}$
\odpStop
\testStart
A.$\frac{42}{4}$
B.$\infty$
C.$-\infty$
D.$0$
E.$\frac{42}{-4}$
F.$\frac{21}{17}$
G.$-\frac{42}{-4}$
H.$1$
I.$21$
\testStop
\kluczStart
A
\kluczStop



\zadStart{Przykład z Wikieł P 4.2b moja wersja nr 372}
Obliczyć granicę $\lim\limits_{x\to\ 21}\frac{x^{2}-21^{2}}{(x-21)(x-19)}$.
\zadStop
\rozwStart{Patryk Wirkus}{Martyna Czarnobaj}
$$\frac{x^{2}-21^{2}}{(x-21)(x-19)}=\frac{x+21}{x-19}$$

$$\lim\limits_{x\to\ 21}\frac{x^{2}-21^{2}}{(x-21)(x-19)}=[\frac{0}{0}]=\lim\limits_{x\to\ 21}\frac{x+21}{x-19}=2 \cdot \frac{21}{21-19} = \frac{42}{2}$$
\rozwStop
\odpStart
$\frac{42}{2}$
\odpStop
\testStart
A.$\frac{42}{2}$
B.$\infty$
C.$-\infty$
D.$0$
E.$\frac{42}{-2}$
F.$\frac{21}{19}$
G.$-\frac{42}{-2}$
H.$1$
I.$21$
\testStop
\kluczStart
A
\kluczStop



\zadStart{Przykład z Wikieł P 4.2b moja wersja nr 373}
Obliczyć granicę $\lim\limits_{x\to\ 21}\frac{x^{2}-21^{2}}{(x-21)(x-23)}$.
\zadStop
\rozwStart{Patryk Wirkus}{Martyna Czarnobaj}
$$\frac{x^{2}-21^{2}}{(x-21)(x-23)}=\frac{x+21}{x-23}$$

$$\lim\limits_{x\to\ 21}\frac{x^{2}-21^{2}}{(x-21)(x-23)}=[\frac{0}{0}]=\lim\limits_{x\to\ 21}\frac{x+21}{x-23}=2 \cdot \frac{21}{21-23} = \frac{42}{-2}$$
\rozwStop
\odpStart
$\frac{42}{-2}$
\odpStop
\testStart
A.$\frac{42}{-2}$
B.$\infty$
C.$-\infty$
D.$0$
E.$\frac{42}{2}$
F.$\frac{21}{23}$
G.$-\frac{42}{2}$
H.$1$
I.$21$
\testStop
\kluczStart
A
\kluczStop



\zadStart{Przykład z Wikieł P 4.2b moja wersja nr 374}
Obliczyć granicę $\lim\limits_{x\to\ 21}\frac{x^{2}-21^{2}}{(x-21)(x-25)}$.
\zadStop
\rozwStart{Patryk Wirkus}{Martyna Czarnobaj}
$$\frac{x^{2}-21^{2}}{(x-21)(x-25)}=\frac{x+21}{x-25}$$

$$\lim\limits_{x\to\ 21}\frac{x^{2}-21^{2}}{(x-21)(x-25)}=[\frac{0}{0}]=\lim\limits_{x\to\ 21}\frac{x+21}{x-25}=2 \cdot \frac{21}{21-25} = \frac{42}{-4}$$
\rozwStop
\odpStart
$\frac{42}{-4}$
\odpStop
\testStart
A.$\frac{42}{-4}$
B.$\infty$
C.$-\infty$
D.$0$
E.$\frac{42}{4}$
F.$\frac{21}{25}$
G.$-\frac{42}{4}$
H.$1$
I.$21$
\testStop
\kluczStart
A
\kluczStop



\zadStart{Przykład z Wikieł P 4.2b moja wersja nr 375}
Obliczyć granicę $\lim\limits_{x\to\ 21}\frac{x^{2}-21^{2}}{(x-21)(x-26)}$.
\zadStop
\rozwStart{Patryk Wirkus}{Martyna Czarnobaj}
$$\frac{x^{2}-21^{2}}{(x-21)(x-26)}=\frac{x+21}{x-26}$$

$$\lim\limits_{x\to\ 21}\frac{x^{2}-21^{2}}{(x-21)(x-26)}=[\frac{0}{0}]=\lim\limits_{x\to\ 21}\frac{x+21}{x-26}=2 \cdot \frac{21}{21-26} = \frac{42}{-5}$$
\rozwStop
\odpStart
$\frac{42}{-5}$
\odpStop
\testStart
A.$\frac{42}{-5}$
B.$\infty$
C.$-\infty$
D.$0$
E.$\frac{42}{5}$
F.$\frac{21}{26}$
G.$-\frac{42}{5}$
H.$1$
I.$21$
\testStop
\kluczStart
A
\kluczStop



\zadStart{Przykład z Wikieł P 4.2b moja wersja nr 376}
Obliczyć granicę $\lim\limits_{x\to\ 21}\frac{x^{2}-21^{2}}{(x-21)(x-29)}$.
\zadStop
\rozwStart{Patryk Wirkus}{Martyna Czarnobaj}
$$\frac{x^{2}-21^{2}}{(x-21)(x-29)}=\frac{x+21}{x-29}$$

$$\lim\limits_{x\to\ 21}\frac{x^{2}-21^{2}}{(x-21)(x-29)}=[\frac{0}{0}]=\lim\limits_{x\to\ 21}\frac{x+21}{x-29}=2 \cdot \frac{21}{21-29} = \frac{42}{-8}$$
\rozwStop
\odpStart
$\frac{42}{-8}$
\odpStop
\testStart
A.$\frac{42}{-8}$
B.$\infty$
C.$-\infty$
D.$0$
E.$\frac{42}{8}$
F.$\frac{21}{29}$
G.$-\frac{42}{8}$
H.$1$
I.$21$
\testStop
\kluczStart
A
\kluczStop



\zadStart{Przykład z Wikieł P 4.2b moja wersja nr 377}
Obliczyć granicę $\lim\limits_{x\to\ 21}\frac{x^{2}-21^{2}}{(x-21)(x-31)}$.
\zadStop
\rozwStart{Patryk Wirkus}{Martyna Czarnobaj}
$$\frac{x^{2}-21^{2}}{(x-21)(x-31)}=\frac{x+21}{x-31}$$

$$\lim\limits_{x\to\ 21}\frac{x^{2}-21^{2}}{(x-21)(x-31)}=[\frac{0}{0}]=\lim\limits_{x\to\ 21}\frac{x+21}{x-31}=2 \cdot \frac{21}{21-31} = \frac{42}{-10}$$
\rozwStop
\odpStart
$\frac{42}{-10}$
\odpStop
\testStart
A.$\frac{42}{-10}$
B.$\infty$
C.$-\infty$
D.$0$
E.$\frac{42}{10}$
F.$\frac{21}{31}$
G.$-\frac{42}{10}$
H.$1$
I.$21$
\testStop
\kluczStart
A
\kluczStop



\zadStart{Przykład z Wikieł P 4.2b moja wersja nr 378}
Obliczyć granicę $\lim\limits_{x\to\ 21}\frac{x^{2}-21^{2}}{(x-21)(x-32)}$.
\zadStop
\rozwStart{Patryk Wirkus}{Martyna Czarnobaj}
$$\frac{x^{2}-21^{2}}{(x-21)(x-32)}=\frac{x+21}{x-32}$$

$$\lim\limits_{x\to\ 21}\frac{x^{2}-21^{2}}{(x-21)(x-32)}=[\frac{0}{0}]=\lim\limits_{x\to\ 21}\frac{x+21}{x-32}=2 \cdot \frac{21}{21-32} = \frac{42}{-11}$$
\rozwStop
\odpStart
$\frac{42}{-11}$
\odpStop
\testStart
A.$\frac{42}{-11}$
B.$\infty$
C.$-\infty$
D.$0$
E.$\frac{42}{11}$
F.$\frac{21}{32}$
G.$-\frac{42}{11}$
H.$1$
I.$21$
\testStop
\kluczStart
A
\kluczStop



\zadStart{Przykład z Wikieł P 4.2b moja wersja nr 379}
Obliczyć granicę $\lim\limits_{x\to\ 21}\frac{x^{2}-21^{2}}{(x-21)(x-34)}$.
\zadStop
\rozwStart{Patryk Wirkus}{Martyna Czarnobaj}
$$\frac{x^{2}-21^{2}}{(x-21)(x-34)}=\frac{x+21}{x-34}$$

$$\lim\limits_{x\to\ 21}\frac{x^{2}-21^{2}}{(x-21)(x-34)}=[\frac{0}{0}]=\lim\limits_{x\to\ 21}\frac{x+21}{x-34}=2 \cdot \frac{21}{21-34} = \frac{42}{-13}$$
\rozwStop
\odpStart
$\frac{42}{-13}$
\odpStop
\testStart
A.$\frac{42}{-13}$
B.$\infty$
C.$-\infty$
D.$0$
E.$\frac{42}{13}$
F.$\frac{21}{34}$
G.$-\frac{42}{13}$
H.$1$
I.$21$
\testStop
\kluczStart
A
\kluczStop



\zadStart{Przykład z Wikieł P 4.2b moja wersja nr 380}
Obliczyć granicę $\lim\limits_{x\to\ 21}\frac{x^{2}-21^{2}}{(x-21)(x-37)}$.
\zadStop
\rozwStart{Patryk Wirkus}{Martyna Czarnobaj}
$$\frac{x^{2}-21^{2}}{(x-21)(x-37)}=\frac{x+21}{x-37}$$

$$\lim\limits_{x\to\ 21}\frac{x^{2}-21^{2}}{(x-21)(x-37)}=[\frac{0}{0}]=\lim\limits_{x\to\ 21}\frac{x+21}{x-37}=2 \cdot \frac{21}{21-37} = \frac{42}{-16}$$
\rozwStop
\odpStart
$\frac{42}{-16}$
\odpStop
\testStart
A.$\frac{42}{-16}$
B.$\infty$
C.$-\infty$
D.$0$
E.$\frac{42}{16}$
F.$\frac{21}{37}$
G.$-\frac{42}{16}$
H.$1$
I.$21$
\testStop
\kluczStart
A
\kluczStop



\zadStart{Przykład z Wikieł P 4.2b moja wersja nr 381}
Obliczyć granicę $\lim\limits_{x\to\ 21}\frac{x^{2}-21^{2}}{(x-21)(x-38)}$.
\zadStop
\rozwStart{Patryk Wirkus}{Martyna Czarnobaj}
$$\frac{x^{2}-21^{2}}{(x-21)(x-38)}=\frac{x+21}{x-38}$$

$$\lim\limits_{x\to\ 21}\frac{x^{2}-21^{2}}{(x-21)(x-38)}=[\frac{0}{0}]=\lim\limits_{x\to\ 21}\frac{x+21}{x-38}=2 \cdot \frac{21}{21-38} = \frac{42}{-17}$$
\rozwStop
\odpStart
$\frac{42}{-17}$
\odpStop
\testStart
A.$\frac{42}{-17}$
B.$\infty$
C.$-\infty$
D.$0$
E.$\frac{42}{17}$
F.$\frac{21}{38}$
G.$-\frac{42}{17}$
H.$1$
I.$21$
\testStop
\kluczStart
A
\kluczStop



\zadStart{Przykład z Wikieł P 4.2b moja wersja nr 382}
Obliczyć granicę $\lim\limits_{x\to\ 21}\frac{x^{2}-21^{2}}{(x-21)(x-40)}$.
\zadStop
\rozwStart{Patryk Wirkus}{Martyna Czarnobaj}
$$\frac{x^{2}-21^{2}}{(x-21)(x-40)}=\frac{x+21}{x-40}$$

$$\lim\limits_{x\to\ 21}\frac{x^{2}-21^{2}}{(x-21)(x-40)}=[\frac{0}{0}]=\lim\limits_{x\to\ 21}\frac{x+21}{x-40}=2 \cdot \frac{21}{21-40} = \frac{42}{-19}$$
\rozwStop
\odpStart
$\frac{42}{-19}$
\odpStop
\testStart
A.$\frac{42}{-19}$
B.$\infty$
C.$-\infty$
D.$0$
E.$\frac{42}{19}$
F.$\frac{21}{40}$
G.$-\frac{42}{19}$
H.$1$
I.$21$
\testStop
\kluczStart
A
\kluczStop



\zadStart{Przykład z Wikieł P 4.2b moja wersja nr 383}
Obliczyć granicę $\lim\limits_{x\to\ 22}\frac{x^{2}-22^{2}}{(x-22)(x-3)}$.
\zadStop
\rozwStart{Patryk Wirkus}{Martyna Czarnobaj}
$$\frac{x^{2}-22^{2}}{(x-22)(x-3)}=\frac{x+22}{x-3}$$

$$\lim\limits_{x\to\ 22}\frac{x^{2}-22^{2}}{(x-22)(x-3)}=[\frac{0}{0}]=\lim\limits_{x\to\ 22}\frac{x+22}{x-3}=2 \cdot \frac{22}{22-3} = \frac{44}{19}$$
\rozwStop
\odpStart
$\frac{44}{19}$
\odpStop
\testStart
A.$\frac{44}{19}$
B.$\infty$
C.$-\infty$
D.$0$
E.$\frac{44}{-19}$
F.$\frac{22}{3}$
G.$-\frac{44}{-19}$
H.$1$
I.$22$
\testStop
\kluczStart
A
\kluczStop



\zadStart{Przykład z Wikieł P 4.2b moja wersja nr 384}
Obliczyć granicę $\lim\limits_{x\to\ 22}\frac{x^{2}-22^{2}}{(x-22)(x-5)}$.
\zadStop
\rozwStart{Patryk Wirkus}{Martyna Czarnobaj}
$$\frac{x^{2}-22^{2}}{(x-22)(x-5)}=\frac{x+22}{x-5}$$

$$\lim\limits_{x\to\ 22}\frac{x^{2}-22^{2}}{(x-22)(x-5)}=[\frac{0}{0}]=\lim\limits_{x\to\ 22}\frac{x+22}{x-5}=2 \cdot \frac{22}{22-5} = \frac{44}{17}$$
\rozwStop
\odpStart
$\frac{44}{17}$
\odpStop
\testStart
A.$\frac{44}{17}$
B.$\infty$
C.$-\infty$
D.$0$
E.$\frac{44}{-17}$
F.$\frac{22}{5}$
G.$-\frac{44}{-17}$
H.$1$
I.$22$
\testStop
\kluczStart
A
\kluczStop



\zadStart{Przykład z Wikieł P 4.2b moja wersja nr 385}
Obliczyć granicę $\lim\limits_{x\to\ 22}\frac{x^{2}-22^{2}}{(x-22)(x-7)}$.
\zadStop
\rozwStart{Patryk Wirkus}{Martyna Czarnobaj}
$$\frac{x^{2}-22^{2}}{(x-22)(x-7)}=\frac{x+22}{x-7}$$

$$\lim\limits_{x\to\ 22}\frac{x^{2}-22^{2}}{(x-22)(x-7)}=[\frac{0}{0}]=\lim\limits_{x\to\ 22}\frac{x+22}{x-7}=2 \cdot \frac{22}{22-7} = \frac{44}{15}$$
\rozwStop
\odpStart
$\frac{44}{15}$
\odpStop
\testStart
A.$\frac{44}{15}$
B.$\infty$
C.$-\infty$
D.$0$
E.$\frac{44}{-15}$
F.$\frac{22}{7}$
G.$-\frac{44}{-15}$
H.$1$
I.$22$
\testStop
\kluczStart
A
\kluczStop



\zadStart{Przykład z Wikieł P 4.2b moja wersja nr 386}
Obliczyć granicę $\lim\limits_{x\to\ 22}\frac{x^{2}-22^{2}}{(x-22)(x-9)}$.
\zadStop
\rozwStart{Patryk Wirkus}{Martyna Czarnobaj}
$$\frac{x^{2}-22^{2}}{(x-22)(x-9)}=\frac{x+22}{x-9}$$

$$\lim\limits_{x\to\ 22}\frac{x^{2}-22^{2}}{(x-22)(x-9)}=[\frac{0}{0}]=\lim\limits_{x\to\ 22}\frac{x+22}{x-9}=2 \cdot \frac{22}{22-9} = \frac{44}{13}$$
\rozwStop
\odpStart
$\frac{44}{13}$
\odpStop
\testStart
A.$\frac{44}{13}$
B.$\infty$
C.$-\infty$
D.$0$
E.$\frac{44}{-13}$
F.$\frac{22}{9}$
G.$-\frac{44}{-13}$
H.$1$
I.$22$
\testStop
\kluczStart
A
\kluczStop



\zadStart{Przykład z Wikieł P 4.2b moja wersja nr 387}
Obliczyć granicę $\lim\limits_{x\to\ 22}\frac{x^{2}-22^{2}}{(x-22)(x-13)}$.
\zadStop
\rozwStart{Patryk Wirkus}{Martyna Czarnobaj}
$$\frac{x^{2}-22^{2}}{(x-22)(x-13)}=\frac{x+22}{x-13}$$

$$\lim\limits_{x\to\ 22}\frac{x^{2}-22^{2}}{(x-22)(x-13)}=[\frac{0}{0}]=\lim\limits_{x\to\ 22}\frac{x+22}{x-13}=2 \cdot \frac{22}{22-13} = \frac{44}{9}$$
\rozwStop
\odpStart
$\frac{44}{9}$
\odpStop
\testStart
A.$\frac{44}{9}$
B.$\infty$
C.$-\infty$
D.$0$
E.$\frac{44}{-9}$
F.$\frac{22}{13}$
G.$-\frac{44}{-9}$
H.$1$
I.$22$
\testStop
\kluczStart
A
\kluczStop



\zadStart{Przykład z Wikieł P 4.2b moja wersja nr 388}
Obliczyć granicę $\lim\limits_{x\to\ 22}\frac{x^{2}-22^{2}}{(x-22)(x-15)}$.
\zadStop
\rozwStart{Patryk Wirkus}{Martyna Czarnobaj}
$$\frac{x^{2}-22^{2}}{(x-22)(x-15)}=\frac{x+22}{x-15}$$

$$\lim\limits_{x\to\ 22}\frac{x^{2}-22^{2}}{(x-22)(x-15)}=[\frac{0}{0}]=\lim\limits_{x\to\ 22}\frac{x+22}{x-15}=2 \cdot \frac{22}{22-15} = \frac{44}{7}$$
\rozwStop
\odpStart
$\frac{44}{7}$
\odpStop
\testStart
A.$\frac{44}{7}$
B.$\infty$
C.$-\infty$
D.$0$
E.$\frac{44}{-7}$
F.$\frac{22}{15}$
G.$-\frac{44}{-7}$
H.$1$
I.$22$
\testStop
\kluczStart
A
\kluczStop



\zadStart{Przykład z Wikieł P 4.2b moja wersja nr 389}
Obliczyć granicę $\lim\limits_{x\to\ 22}\frac{x^{2}-22^{2}}{(x-22)(x-17)}$.
\zadStop
\rozwStart{Patryk Wirkus}{Martyna Czarnobaj}
$$\frac{x^{2}-22^{2}}{(x-22)(x-17)}=\frac{x+22}{x-17}$$

$$\lim\limits_{x\to\ 22}\frac{x^{2}-22^{2}}{(x-22)(x-17)}=[\frac{0}{0}]=\lim\limits_{x\to\ 22}\frac{x+22}{x-17}=2 \cdot \frac{22}{22-17} = \frac{44}{5}$$
\rozwStop
\odpStart
$\frac{44}{5}$
\odpStop
\testStart
A.$\frac{44}{5}$
B.$\infty$
C.$-\infty$
D.$0$
E.$\frac{44}{-5}$
F.$\frac{22}{17}$
G.$-\frac{44}{-5}$
H.$1$
I.$22$
\testStop
\kluczStart
A
\kluczStop



\zadStart{Przykład z Wikieł P 4.2b moja wersja nr 390}
Obliczyć granicę $\lim\limits_{x\to\ 22}\frac{x^{2}-22^{2}}{(x-22)(x-19)}$.
\zadStop
\rozwStart{Patryk Wirkus}{Martyna Czarnobaj}
$$\frac{x^{2}-22^{2}}{(x-22)(x-19)}=\frac{x+22}{x-19}$$

$$\lim\limits_{x\to\ 22}\frac{x^{2}-22^{2}}{(x-22)(x-19)}=[\frac{0}{0}]=\lim\limits_{x\to\ 22}\frac{x+22}{x-19}=2 \cdot \frac{22}{22-19} = \frac{44}{3}$$
\rozwStop
\odpStart
$\frac{44}{3}$
\odpStop
\testStart
A.$\frac{44}{3}$
B.$\infty$
C.$-\infty$
D.$0$
E.$\frac{44}{-3}$
F.$\frac{22}{19}$
G.$-\frac{44}{-3}$
H.$1$
I.$22$
\testStop
\kluczStart
A
\kluczStop



\zadStart{Przykład z Wikieł P 4.2b moja wersja nr 391}
Obliczyć granicę $\lim\limits_{x\to\ 22}\frac{x^{2}-22^{2}}{(x-22)(x-25)}$.
\zadStop
\rozwStart{Patryk Wirkus}{Martyna Czarnobaj}
$$\frac{x^{2}-22^{2}}{(x-22)(x-25)}=\frac{x+22}{x-25}$$

$$\lim\limits_{x\to\ 22}\frac{x^{2}-22^{2}}{(x-22)(x-25)}=[\frac{0}{0}]=\lim\limits_{x\to\ 22}\frac{x+22}{x-25}=2 \cdot \frac{22}{22-25} = \frac{44}{-3}$$
\rozwStop
\odpStart
$\frac{44}{-3}$
\odpStop
\testStart
A.$\frac{44}{-3}$
B.$\infty$
C.$-\infty$
D.$0$
E.$\frac{44}{3}$
F.$\frac{22}{25}$
G.$-\frac{44}{3}$
H.$1$
I.$22$
\testStop
\kluczStart
A
\kluczStop



\zadStart{Przykład z Wikieł P 4.2b moja wersja nr 392}
Obliczyć granicę $\lim\limits_{x\to\ 22}\frac{x^{2}-22^{2}}{(x-22)(x-27)}$.
\zadStop
\rozwStart{Patryk Wirkus}{Martyna Czarnobaj}
$$\frac{x^{2}-22^{2}}{(x-22)(x-27)}=\frac{x+22}{x-27}$$

$$\lim\limits_{x\to\ 22}\frac{x^{2}-22^{2}}{(x-22)(x-27)}=[\frac{0}{0}]=\lim\limits_{x\to\ 22}\frac{x+22}{x-27}=2 \cdot \frac{22}{22-27} = \frac{44}{-5}$$
\rozwStop
\odpStart
$\frac{44}{-5}$
\odpStop
\testStart
A.$\frac{44}{-5}$
B.$\infty$
C.$-\infty$
D.$0$
E.$\frac{44}{5}$
F.$\frac{22}{27}$
G.$-\frac{44}{5}$
H.$1$
I.$22$
\testStop
\kluczStart
A
\kluczStop



\zadStart{Przykład z Wikieł P 4.2b moja wersja nr 393}
Obliczyć granicę $\lim\limits_{x\to\ 22}\frac{x^{2}-22^{2}}{(x-22)(x-29)}$.
\zadStop
\rozwStart{Patryk Wirkus}{Martyna Czarnobaj}
$$\frac{x^{2}-22^{2}}{(x-22)(x-29)}=\frac{x+22}{x-29}$$

$$\lim\limits_{x\to\ 22}\frac{x^{2}-22^{2}}{(x-22)(x-29)}=[\frac{0}{0}]=\lim\limits_{x\to\ 22}\frac{x+22}{x-29}=2 \cdot \frac{22}{22-29} = \frac{44}{-7}$$
\rozwStop
\odpStart
$\frac{44}{-7}$
\odpStop
\testStart
A.$\frac{44}{-7}$
B.$\infty$
C.$-\infty$
D.$0$
E.$\frac{44}{7}$
F.$\frac{22}{29}$
G.$-\frac{44}{7}$
H.$1$
I.$22$
\testStop
\kluczStart
A
\kluczStop



\zadStart{Przykład z Wikieł P 4.2b moja wersja nr 394}
Obliczyć granicę $\lim\limits_{x\to\ 22}\frac{x^{2}-22^{2}}{(x-22)(x-31)}$.
\zadStop
\rozwStart{Patryk Wirkus}{Martyna Czarnobaj}
$$\frac{x^{2}-22^{2}}{(x-22)(x-31)}=\frac{x+22}{x-31}$$

$$\lim\limits_{x\to\ 22}\frac{x^{2}-22^{2}}{(x-22)(x-31)}=[\frac{0}{0}]=\lim\limits_{x\to\ 22}\frac{x+22}{x-31}=2 \cdot \frac{22}{22-31} = \frac{44}{-9}$$
\rozwStop
\odpStart
$\frac{44}{-9}$
\odpStop
\testStart
A.$\frac{44}{-9}$
B.$\infty$
C.$-\infty$
D.$0$
E.$\frac{44}{9}$
F.$\frac{22}{31}$
G.$-\frac{44}{9}$
H.$1$
I.$22$
\testStop
\kluczStart
A
\kluczStop



\zadStart{Przykład z Wikieł P 4.2b moja wersja nr 395}
Obliczyć granicę $\lim\limits_{x\to\ 22}\frac{x^{2}-22^{2}}{(x-22)(x-35)}$.
\zadStop
\rozwStart{Patryk Wirkus}{Martyna Czarnobaj}
$$\frac{x^{2}-22^{2}}{(x-22)(x-35)}=\frac{x+22}{x-35}$$

$$\lim\limits_{x\to\ 22}\frac{x^{2}-22^{2}}{(x-22)(x-35)}=[\frac{0}{0}]=\lim\limits_{x\to\ 22}\frac{x+22}{x-35}=2 \cdot \frac{22}{22-35} = \frac{44}{-13}$$
\rozwStop
\odpStart
$\frac{44}{-13}$
\odpStop
\testStart
A.$\frac{44}{-13}$
B.$\infty$
C.$-\infty$
D.$0$
E.$\frac{44}{13}$
F.$\frac{22}{35}$
G.$-\frac{44}{13}$
H.$1$
I.$22$
\testStop
\kluczStart
A
\kluczStop



\zadStart{Przykład z Wikieł P 4.2b moja wersja nr 396}
Obliczyć granicę $\lim\limits_{x\to\ 22}\frac{x^{2}-22^{2}}{(x-22)(x-37)}$.
\zadStop
\rozwStart{Patryk Wirkus}{Martyna Czarnobaj}
$$\frac{x^{2}-22^{2}}{(x-22)(x-37)}=\frac{x+22}{x-37}$$

$$\lim\limits_{x\to\ 22}\frac{x^{2}-22^{2}}{(x-22)(x-37)}=[\frac{0}{0}]=\lim\limits_{x\to\ 22}\frac{x+22}{x-37}=2 \cdot \frac{22}{22-37} = \frac{44}{-15}$$
\rozwStop
\odpStart
$\frac{44}{-15}$
\odpStop
\testStart
A.$\frac{44}{-15}$
B.$\infty$
C.$-\infty$
D.$0$
E.$\frac{44}{15}$
F.$\frac{22}{37}$
G.$-\frac{44}{15}$
H.$1$
I.$22$
\testStop
\kluczStart
A
\kluczStop



\zadStart{Przykład z Wikieł P 4.2b moja wersja nr 397}
Obliczyć granicę $\lim\limits_{x\to\ 22}\frac{x^{2}-22^{2}}{(x-22)(x-39)}$.
\zadStop
\rozwStart{Patryk Wirkus}{Martyna Czarnobaj}
$$\frac{x^{2}-22^{2}}{(x-22)(x-39)}=\frac{x+22}{x-39}$$

$$\lim\limits_{x\to\ 22}\frac{x^{2}-22^{2}}{(x-22)(x-39)}=[\frac{0}{0}]=\lim\limits_{x\to\ 22}\frac{x+22}{x-39}=2 \cdot \frac{22}{22-39} = \frac{44}{-17}$$
\rozwStop
\odpStart
$\frac{44}{-17}$
\odpStop
\testStart
A.$\frac{44}{-17}$
B.$\infty$
C.$-\infty$
D.$0$
E.$\frac{44}{17}$
F.$\frac{22}{39}$
G.$-\frac{44}{17}$
H.$1$
I.$22$
\testStop
\kluczStart
A
\kluczStop



\zadStart{Przykład z Wikieł P 4.2b moja wersja nr 398}
Obliczyć granicę $\lim\limits_{x\to\ 23}\frac{x^{2}-23^{2}}{(x-23)(x-2)}$.
\zadStop
\rozwStart{Patryk Wirkus}{Martyna Czarnobaj}
$$\frac{x^{2}-23^{2}}{(x-23)(x-2)}=\frac{x+23}{x-2}$$

$$\lim\limits_{x\to\ 23}\frac{x^{2}-23^{2}}{(x-23)(x-2)}=[\frac{0}{0}]=\lim\limits_{x\to\ 23}\frac{x+23}{x-2}=2 \cdot \frac{23}{23-2} = \frac{46}{21}$$
\rozwStop
\odpStart
$\frac{46}{21}$
\odpStop
\testStart
A.$\frac{46}{21}$
B.$\infty$
C.$-\infty$
D.$0$
E.$\frac{46}{-21}$
F.$\frac{23}{2}$
G.$-\frac{46}{-21}$
H.$1$
I.$23$
\testStop
\kluczStart
A
\kluczStop



\zadStart{Przykład z Wikieł P 4.2b moja wersja nr 399}
Obliczyć granicę $\lim\limits_{x\to\ 23}\frac{x^{2}-23^{2}}{(x-23)(x-3)}$.
\zadStop
\rozwStart{Patryk Wirkus}{Martyna Czarnobaj}
$$\frac{x^{2}-23^{2}}{(x-23)(x-3)}=\frac{x+23}{x-3}$$

$$\lim\limits_{x\to\ 23}\frac{x^{2}-23^{2}}{(x-23)(x-3)}=[\frac{0}{0}]=\lim\limits_{x\to\ 23}\frac{x+23}{x-3}=2 \cdot \frac{23}{23-3} = \frac{46}{20}$$
\rozwStop
\odpStart
$\frac{46}{20}$
\odpStop
\testStart
A.$\frac{46}{20}$
B.$\infty$
C.$-\infty$
D.$0$
E.$\frac{46}{-20}$
F.$\frac{23}{3}$
G.$-\frac{46}{-20}$
H.$1$
I.$23$
\testStop
\kluczStart
A
\kluczStop



\zadStart{Przykład z Wikieł P 4.2b moja wersja nr 400}
Obliczyć granicę $\lim\limits_{x\to\ 23}\frac{x^{2}-23^{2}}{(x-23)(x-4)}$.
\zadStop
\rozwStart{Patryk Wirkus}{Martyna Czarnobaj}
$$\frac{x^{2}-23^{2}}{(x-23)(x-4)}=\frac{x+23}{x-4}$$

$$\lim\limits_{x\to\ 23}\frac{x^{2}-23^{2}}{(x-23)(x-4)}=[\frac{0}{0}]=\lim\limits_{x\to\ 23}\frac{x+23}{x-4}=2 \cdot \frac{23}{23-4} = \frac{46}{19}$$
\rozwStop
\odpStart
$\frac{46}{19}$
\odpStop
\testStart
A.$\frac{46}{19}$
B.$\infty$
C.$-\infty$
D.$0$
E.$\frac{46}{-19}$
F.$\frac{23}{4}$
G.$-\frac{46}{-19}$
H.$1$
I.$23$
\testStop
\kluczStart
A
\kluczStop



\zadStart{Przykład z Wikieł P 4.2b moja wersja nr 401}
Obliczyć granicę $\lim\limits_{x\to\ 23}\frac{x^{2}-23^{2}}{(x-23)(x-5)}$.
\zadStop
\rozwStart{Patryk Wirkus}{Martyna Czarnobaj}
$$\frac{x^{2}-23^{2}}{(x-23)(x-5)}=\frac{x+23}{x-5}$$

$$\lim\limits_{x\to\ 23}\frac{x^{2}-23^{2}}{(x-23)(x-5)}=[\frac{0}{0}]=\lim\limits_{x\to\ 23}\frac{x+23}{x-5}=2 \cdot \frac{23}{23-5} = \frac{46}{18}$$
\rozwStop
\odpStart
$\frac{46}{18}$
\odpStop
\testStart
A.$\frac{46}{18}$
B.$\infty$
C.$-\infty$
D.$0$
E.$\frac{46}{-18}$
F.$\frac{23}{5}$
G.$-\frac{46}{-18}$
H.$1$
I.$23$
\testStop
\kluczStart
A
\kluczStop



\zadStart{Przykład z Wikieł P 4.2b moja wersja nr 402}
Obliczyć granicę $\lim\limits_{x\to\ 23}\frac{x^{2}-23^{2}}{(x-23)(x-6)}$.
\zadStop
\rozwStart{Patryk Wirkus}{Martyna Czarnobaj}
$$\frac{x^{2}-23^{2}}{(x-23)(x-6)}=\frac{x+23}{x-6}$$

$$\lim\limits_{x\to\ 23}\frac{x^{2}-23^{2}}{(x-23)(x-6)}=[\frac{0}{0}]=\lim\limits_{x\to\ 23}\frac{x+23}{x-6}=2 \cdot \frac{23}{23-6} = \frac{46}{17}$$
\rozwStop
\odpStart
$\frac{46}{17}$
\odpStop
\testStart
A.$\frac{46}{17}$
B.$\infty$
C.$-\infty$
D.$0$
E.$\frac{46}{-17}$
F.$\frac{23}{6}$
G.$-\frac{46}{-17}$
H.$1$
I.$23$
\testStop
\kluczStart
A
\kluczStop



\zadStart{Przykład z Wikieł P 4.2b moja wersja nr 403}
Obliczyć granicę $\lim\limits_{x\to\ 23}\frac{x^{2}-23^{2}}{(x-23)(x-7)}$.
\zadStop
\rozwStart{Patryk Wirkus}{Martyna Czarnobaj}
$$\frac{x^{2}-23^{2}}{(x-23)(x-7)}=\frac{x+23}{x-7}$$

$$\lim\limits_{x\to\ 23}\frac{x^{2}-23^{2}}{(x-23)(x-7)}=[\frac{0}{0}]=\lim\limits_{x\to\ 23}\frac{x+23}{x-7}=2 \cdot \frac{23}{23-7} = \frac{46}{16}$$
\rozwStop
\odpStart
$\frac{46}{16}$
\odpStop
\testStart
A.$\frac{46}{16}$
B.$\infty$
C.$-\infty$
D.$0$
E.$\frac{46}{-16}$
F.$\frac{23}{7}$
G.$-\frac{46}{-16}$
H.$1$
I.$23$
\testStop
\kluczStart
A
\kluczStop



\zadStart{Przykład z Wikieł P 4.2b moja wersja nr 404}
Obliczyć granicę $\lim\limits_{x\to\ 23}\frac{x^{2}-23^{2}}{(x-23)(x-8)}$.
\zadStop
\rozwStart{Patryk Wirkus}{Martyna Czarnobaj}
$$\frac{x^{2}-23^{2}}{(x-23)(x-8)}=\frac{x+23}{x-8}$$

$$\lim\limits_{x\to\ 23}\frac{x^{2}-23^{2}}{(x-23)(x-8)}=[\frac{0}{0}]=\lim\limits_{x\to\ 23}\frac{x+23}{x-8}=2 \cdot \frac{23}{23-8} = \frac{46}{15}$$
\rozwStop
\odpStart
$\frac{46}{15}$
\odpStop
\testStart
A.$\frac{46}{15}$
B.$\infty$
C.$-\infty$
D.$0$
E.$\frac{46}{-15}$
F.$\frac{23}{8}$
G.$-\frac{46}{-15}$
H.$1$
I.$23$
\testStop
\kluczStart
A
\kluczStop



\zadStart{Przykład z Wikieł P 4.2b moja wersja nr 405}
Obliczyć granicę $\lim\limits_{x\to\ 23}\frac{x^{2}-23^{2}}{(x-23)(x-9)}$.
\zadStop
\rozwStart{Patryk Wirkus}{Martyna Czarnobaj}
$$\frac{x^{2}-23^{2}}{(x-23)(x-9)}=\frac{x+23}{x-9}$$

$$\lim\limits_{x\to\ 23}\frac{x^{2}-23^{2}}{(x-23)(x-9)}=[\frac{0}{0}]=\lim\limits_{x\to\ 23}\frac{x+23}{x-9}=2 \cdot \frac{23}{23-9} = \frac{46}{14}$$
\rozwStop
\odpStart
$\frac{46}{14}$
\odpStop
\testStart
A.$\frac{46}{14}$
B.$\infty$
C.$-\infty$
D.$0$
E.$\frac{46}{-14}$
F.$\frac{23}{9}$
G.$-\frac{46}{-14}$
H.$1$
I.$23$
\testStop
\kluczStart
A
\kluczStop



\zadStart{Przykład z Wikieł P 4.2b moja wersja nr 406}
Obliczyć granicę $\lim\limits_{x\to\ 23}\frac{x^{2}-23^{2}}{(x-23)(x-10)}$.
\zadStop
\rozwStart{Patryk Wirkus}{Martyna Czarnobaj}
$$\frac{x^{2}-23^{2}}{(x-23)(x-10)}=\frac{x+23}{x-10}$$

$$\lim\limits_{x\to\ 23}\frac{x^{2}-23^{2}}{(x-23)(x-10)}=[\frac{0}{0}]=\lim\limits_{x\to\ 23}\frac{x+23}{x-10}=2 \cdot \frac{23}{23-10} = \frac{46}{13}$$
\rozwStop
\odpStart
$\frac{46}{13}$
\odpStop
\testStart
A.$\frac{46}{13}$
B.$\infty$
C.$-\infty$
D.$0$
E.$\frac{46}{-13}$
F.$\frac{23}{10}$
G.$-\frac{46}{-13}$
H.$1$
I.$23$
\testStop
\kluczStart
A
\kluczStop



\zadStart{Przykład z Wikieł P 4.2b moja wersja nr 407}
Obliczyć granicę $\lim\limits_{x\to\ 23}\frac{x^{2}-23^{2}}{(x-23)(x-11)}$.
\zadStop
\rozwStart{Patryk Wirkus}{Martyna Czarnobaj}
$$\frac{x^{2}-23^{2}}{(x-23)(x-11)}=\frac{x+23}{x-11}$$

$$\lim\limits_{x\to\ 23}\frac{x^{2}-23^{2}}{(x-23)(x-11)}=[\frac{0}{0}]=\lim\limits_{x\to\ 23}\frac{x+23}{x-11}=2 \cdot \frac{23}{23-11} = \frac{46}{12}$$
\rozwStop
\odpStart
$\frac{46}{12}$
\odpStop
\testStart
A.$\frac{46}{12}$
B.$\infty$
C.$-\infty$
D.$0$
E.$\frac{46}{-12}$
F.$\frac{23}{11}$
G.$-\frac{46}{-12}$
H.$1$
I.$23$
\testStop
\kluczStart
A
\kluczStop



\zadStart{Przykład z Wikieł P 4.2b moja wersja nr 408}
Obliczyć granicę $\lim\limits_{x\to\ 23}\frac{x^{2}-23^{2}}{(x-23)(x-12)}$.
\zadStop
\rozwStart{Patryk Wirkus}{Martyna Czarnobaj}
$$\frac{x^{2}-23^{2}}{(x-23)(x-12)}=\frac{x+23}{x-12}$$

$$\lim\limits_{x\to\ 23}\frac{x^{2}-23^{2}}{(x-23)(x-12)}=[\frac{0}{0}]=\lim\limits_{x\to\ 23}\frac{x+23}{x-12}=2 \cdot \frac{23}{23-12} = \frac{46}{11}$$
\rozwStop
\odpStart
$\frac{46}{11}$
\odpStop
\testStart
A.$\frac{46}{11}$
B.$\infty$
C.$-\infty$
D.$0$
E.$\frac{46}{-11}$
F.$\frac{23}{12}$
G.$-\frac{46}{-11}$
H.$1$
I.$23$
\testStop
\kluczStart
A
\kluczStop



\zadStart{Przykład z Wikieł P 4.2b moja wersja nr 409}
Obliczyć granicę $\lim\limits_{x\to\ 23}\frac{x^{2}-23^{2}}{(x-23)(x-13)}$.
\zadStop
\rozwStart{Patryk Wirkus}{Martyna Czarnobaj}
$$\frac{x^{2}-23^{2}}{(x-23)(x-13)}=\frac{x+23}{x-13}$$

$$\lim\limits_{x\to\ 23}\frac{x^{2}-23^{2}}{(x-23)(x-13)}=[\frac{0}{0}]=\lim\limits_{x\to\ 23}\frac{x+23}{x-13}=2 \cdot \frac{23}{23-13} = \frac{46}{10}$$
\rozwStop
\odpStart
$\frac{46}{10}$
\odpStop
\testStart
A.$\frac{46}{10}$
B.$\infty$
C.$-\infty$
D.$0$
E.$\frac{46}{-10}$
F.$\frac{23}{13}$
G.$-\frac{46}{-10}$
H.$1$
I.$23$
\testStop
\kluczStart
A
\kluczStop



\zadStart{Przykład z Wikieł P 4.2b moja wersja nr 410}
Obliczyć granicę $\lim\limits_{x\to\ 23}\frac{x^{2}-23^{2}}{(x-23)(x-14)}$.
\zadStop
\rozwStart{Patryk Wirkus}{Martyna Czarnobaj}
$$\frac{x^{2}-23^{2}}{(x-23)(x-14)}=\frac{x+23}{x-14}$$

$$\lim\limits_{x\to\ 23}\frac{x^{2}-23^{2}}{(x-23)(x-14)}=[\frac{0}{0}]=\lim\limits_{x\to\ 23}\frac{x+23}{x-14}=2 \cdot \frac{23}{23-14} = \frac{46}{9}$$
\rozwStop
\odpStart
$\frac{46}{9}$
\odpStop
\testStart
A.$\frac{46}{9}$
B.$\infty$
C.$-\infty$
D.$0$
E.$\frac{46}{-9}$
F.$\frac{23}{14}$
G.$-\frac{46}{-9}$
H.$1$
I.$23$
\testStop
\kluczStart
A
\kluczStop



\zadStart{Przykład z Wikieł P 4.2b moja wersja nr 411}
Obliczyć granicę $\lim\limits_{x\to\ 23}\frac{x^{2}-23^{2}}{(x-23)(x-15)}$.
\zadStop
\rozwStart{Patryk Wirkus}{Martyna Czarnobaj}
$$\frac{x^{2}-23^{2}}{(x-23)(x-15)}=\frac{x+23}{x-15}$$

$$\lim\limits_{x\to\ 23}\frac{x^{2}-23^{2}}{(x-23)(x-15)}=[\frac{0}{0}]=\lim\limits_{x\to\ 23}\frac{x+23}{x-15}=2 \cdot \frac{23}{23-15} = \frac{46}{8}$$
\rozwStop
\odpStart
$\frac{46}{8}$
\odpStop
\testStart
A.$\frac{46}{8}$
B.$\infty$
C.$-\infty$
D.$0$
E.$\frac{46}{-8}$
F.$\frac{23}{15}$
G.$-\frac{46}{-8}$
H.$1$
I.$23$
\testStop
\kluczStart
A
\kluczStop



\zadStart{Przykład z Wikieł P 4.2b moja wersja nr 412}
Obliczyć granicę $\lim\limits_{x\to\ 23}\frac{x^{2}-23^{2}}{(x-23)(x-16)}$.
\zadStop
\rozwStart{Patryk Wirkus}{Martyna Czarnobaj}
$$\frac{x^{2}-23^{2}}{(x-23)(x-16)}=\frac{x+23}{x-16}$$

$$\lim\limits_{x\to\ 23}\frac{x^{2}-23^{2}}{(x-23)(x-16)}=[\frac{0}{0}]=\lim\limits_{x\to\ 23}\frac{x+23}{x-16}=2 \cdot \frac{23}{23-16} = \frac{46}{7}$$
\rozwStop
\odpStart
$\frac{46}{7}$
\odpStop
\testStart
A.$\frac{46}{7}$
B.$\infty$
C.$-\infty$
D.$0$
E.$\frac{46}{-7}$
F.$\frac{23}{16}$
G.$-\frac{46}{-7}$
H.$1$
I.$23$
\testStop
\kluczStart
A
\kluczStop



\zadStart{Przykład z Wikieł P 4.2b moja wersja nr 413}
Obliczyć granicę $\lim\limits_{x\to\ 23}\frac{x^{2}-23^{2}}{(x-23)(x-17)}$.
\zadStop
\rozwStart{Patryk Wirkus}{Martyna Czarnobaj}
$$\frac{x^{2}-23^{2}}{(x-23)(x-17)}=\frac{x+23}{x-17}$$

$$\lim\limits_{x\to\ 23}\frac{x^{2}-23^{2}}{(x-23)(x-17)}=[\frac{0}{0}]=\lim\limits_{x\to\ 23}\frac{x+23}{x-17}=2 \cdot \frac{23}{23-17} = \frac{46}{6}$$
\rozwStop
\odpStart
$\frac{46}{6}$
\odpStop
\testStart
A.$\frac{46}{6}$
B.$\infty$
C.$-\infty$
D.$0$
E.$\frac{46}{-6}$
F.$\frac{23}{17}$
G.$-\frac{46}{-6}$
H.$1$
I.$23$
\testStop
\kluczStart
A
\kluczStop



\zadStart{Przykład z Wikieł P 4.2b moja wersja nr 414}
Obliczyć granicę $\lim\limits_{x\to\ 23}\frac{x^{2}-23^{2}}{(x-23)(x-18)}$.
\zadStop
\rozwStart{Patryk Wirkus}{Martyna Czarnobaj}
$$\frac{x^{2}-23^{2}}{(x-23)(x-18)}=\frac{x+23}{x-18}$$

$$\lim\limits_{x\to\ 23}\frac{x^{2}-23^{2}}{(x-23)(x-18)}=[\frac{0}{0}]=\lim\limits_{x\to\ 23}\frac{x+23}{x-18}=2 \cdot \frac{23}{23-18} = \frac{46}{5}$$
\rozwStop
\odpStart
$\frac{46}{5}$
\odpStop
\testStart
A.$\frac{46}{5}$
B.$\infty$
C.$-\infty$
D.$0$
E.$\frac{46}{-5}$
F.$\frac{23}{18}$
G.$-\frac{46}{-5}$
H.$1$
I.$23$
\testStop
\kluczStart
A
\kluczStop



\zadStart{Przykład z Wikieł P 4.2b moja wersja nr 415}
Obliczyć granicę $\lim\limits_{x\to\ 23}\frac{x^{2}-23^{2}}{(x-23)(x-19)}$.
\zadStop
\rozwStart{Patryk Wirkus}{Martyna Czarnobaj}
$$\frac{x^{2}-23^{2}}{(x-23)(x-19)}=\frac{x+23}{x-19}$$

$$\lim\limits_{x\to\ 23}\frac{x^{2}-23^{2}}{(x-23)(x-19)}=[\frac{0}{0}]=\lim\limits_{x\to\ 23}\frac{x+23}{x-19}=2 \cdot \frac{23}{23-19} = \frac{46}{4}$$
\rozwStop
\odpStart
$\frac{46}{4}$
\odpStop
\testStart
A.$\frac{46}{4}$
B.$\infty$
C.$-\infty$
D.$0$
E.$\frac{46}{-4}$
F.$\frac{23}{19}$
G.$-\frac{46}{-4}$
H.$1$
I.$23$
\testStop
\kluczStart
A
\kluczStop



\zadStart{Przykład z Wikieł P 4.2b moja wersja nr 416}
Obliczyć granicę $\lim\limits_{x\to\ 23}\frac{x^{2}-23^{2}}{(x-23)(x-20)}$.
\zadStop
\rozwStart{Patryk Wirkus}{Martyna Czarnobaj}
$$\frac{x^{2}-23^{2}}{(x-23)(x-20)}=\frac{x+23}{x-20}$$

$$\lim\limits_{x\to\ 23}\frac{x^{2}-23^{2}}{(x-23)(x-20)}=[\frac{0}{0}]=\lim\limits_{x\to\ 23}\frac{x+23}{x-20}=2 \cdot \frac{23}{23-20} = \frac{46}{3}$$
\rozwStop
\odpStart
$\frac{46}{3}$
\odpStop
\testStart
A.$\frac{46}{3}$
B.$\infty$
C.$-\infty$
D.$0$
E.$\frac{46}{-3}$
F.$\frac{23}{20}$
G.$-\frac{46}{-3}$
H.$1$
I.$23$
\testStop
\kluczStart
A
\kluczStop



\zadStart{Przykład z Wikieł P 4.2b moja wersja nr 417}
Obliczyć granicę $\lim\limits_{x\to\ 23}\frac{x^{2}-23^{2}}{(x-23)(x-21)}$.
\zadStop
\rozwStart{Patryk Wirkus}{Martyna Czarnobaj}
$$\frac{x^{2}-23^{2}}{(x-23)(x-21)}=\frac{x+23}{x-21}$$

$$\lim\limits_{x\to\ 23}\frac{x^{2}-23^{2}}{(x-23)(x-21)}=[\frac{0}{0}]=\lim\limits_{x\to\ 23}\frac{x+23}{x-21}=2 \cdot \frac{23}{23-21} = \frac{46}{2}$$
\rozwStop
\odpStart
$\frac{46}{2}$
\odpStop
\testStart
A.$\frac{46}{2}$
B.$\infty$
C.$-\infty$
D.$0$
E.$\frac{46}{-2}$
F.$\frac{23}{21}$
G.$-\frac{46}{-2}$
H.$1$
I.$23$
\testStop
\kluczStart
A
\kluczStop



\zadStart{Przykład z Wikieł P 4.2b moja wersja nr 418}
Obliczyć granicę $\lim\limits_{x\to\ 23}\frac{x^{2}-23^{2}}{(x-23)(x-25)}$.
\zadStop
\rozwStart{Patryk Wirkus}{Martyna Czarnobaj}
$$\frac{x^{2}-23^{2}}{(x-23)(x-25)}=\frac{x+23}{x-25}$$

$$\lim\limits_{x\to\ 23}\frac{x^{2}-23^{2}}{(x-23)(x-25)}=[\frac{0}{0}]=\lim\limits_{x\to\ 23}\frac{x+23}{x-25}=2 \cdot \frac{23}{23-25} = \frac{46}{-2}$$
\rozwStop
\odpStart
$\frac{46}{-2}$
\odpStop
\testStart
A.$\frac{46}{-2}$
B.$\infty$
C.$-\infty$
D.$0$
E.$\frac{46}{2}$
F.$\frac{23}{25}$
G.$-\frac{46}{2}$
H.$1$
I.$23$
\testStop
\kluczStart
A
\kluczStop



\zadStart{Przykład z Wikieł P 4.2b moja wersja nr 419}
Obliczyć granicę $\lim\limits_{x\to\ 23}\frac{x^{2}-23^{2}}{(x-23)(x-26)}$.
\zadStop
\rozwStart{Patryk Wirkus}{Martyna Czarnobaj}
$$\frac{x^{2}-23^{2}}{(x-23)(x-26)}=\frac{x+23}{x-26}$$

$$\lim\limits_{x\to\ 23}\frac{x^{2}-23^{2}}{(x-23)(x-26)}=[\frac{0}{0}]=\lim\limits_{x\to\ 23}\frac{x+23}{x-26}=2 \cdot \frac{23}{23-26} = \frac{46}{-3}$$
\rozwStop
\odpStart
$\frac{46}{-3}$
\odpStop
\testStart
A.$\frac{46}{-3}$
B.$\infty$
C.$-\infty$
D.$0$
E.$\frac{46}{3}$
F.$\frac{23}{26}$
G.$-\frac{46}{3}$
H.$1$
I.$23$
\testStop
\kluczStart
A
\kluczStop



\zadStart{Przykład z Wikieł P 4.2b moja wersja nr 420}
Obliczyć granicę $\lim\limits_{x\to\ 23}\frac{x^{2}-23^{2}}{(x-23)(x-27)}$.
\zadStop
\rozwStart{Patryk Wirkus}{Martyna Czarnobaj}
$$\frac{x^{2}-23^{2}}{(x-23)(x-27)}=\frac{x+23}{x-27}$$

$$\lim\limits_{x\to\ 23}\frac{x^{2}-23^{2}}{(x-23)(x-27)}=[\frac{0}{0}]=\lim\limits_{x\to\ 23}\frac{x+23}{x-27}=2 \cdot \frac{23}{23-27} = \frac{46}{-4}$$
\rozwStop
\odpStart
$\frac{46}{-4}$
\odpStop
\testStart
A.$\frac{46}{-4}$
B.$\infty$
C.$-\infty$
D.$0$
E.$\frac{46}{4}$
F.$\frac{23}{27}$
G.$-\frac{46}{4}$
H.$1$
I.$23$
\testStop
\kluczStart
A
\kluczStop



\zadStart{Przykład z Wikieł P 4.2b moja wersja nr 421}
Obliczyć granicę $\lim\limits_{x\to\ 23}\frac{x^{2}-23^{2}}{(x-23)(x-28)}$.
\zadStop
\rozwStart{Patryk Wirkus}{Martyna Czarnobaj}
$$\frac{x^{2}-23^{2}}{(x-23)(x-28)}=\frac{x+23}{x-28}$$

$$\lim\limits_{x\to\ 23}\frac{x^{2}-23^{2}}{(x-23)(x-28)}=[\frac{0}{0}]=\lim\limits_{x\to\ 23}\frac{x+23}{x-28}=2 \cdot \frac{23}{23-28} = \frac{46}{-5}$$
\rozwStop
\odpStart
$\frac{46}{-5}$
\odpStop
\testStart
A.$\frac{46}{-5}$
B.$\infty$
C.$-\infty$
D.$0$
E.$\frac{46}{5}$
F.$\frac{23}{28}$
G.$-\frac{46}{5}$
H.$1$
I.$23$
\testStop
\kluczStart
A
\kluczStop



\zadStart{Przykład z Wikieł P 4.2b moja wersja nr 422}
Obliczyć granicę $\lim\limits_{x\to\ 23}\frac{x^{2}-23^{2}}{(x-23)(x-29)}$.
\zadStop
\rozwStart{Patryk Wirkus}{Martyna Czarnobaj}
$$\frac{x^{2}-23^{2}}{(x-23)(x-29)}=\frac{x+23}{x-29}$$

$$\lim\limits_{x\to\ 23}\frac{x^{2}-23^{2}}{(x-23)(x-29)}=[\frac{0}{0}]=\lim\limits_{x\to\ 23}\frac{x+23}{x-29}=2 \cdot \frac{23}{23-29} = \frac{46}{-6}$$
\rozwStop
\odpStart
$\frac{46}{-6}$
\odpStop
\testStart
A.$\frac{46}{-6}$
B.$\infty$
C.$-\infty$
D.$0$
E.$\frac{46}{6}$
F.$\frac{23}{29}$
G.$-\frac{46}{6}$
H.$1$
I.$23$
\testStop
\kluczStart
A
\kluczStop



\zadStart{Przykład z Wikieł P 4.2b moja wersja nr 423}
Obliczyć granicę $\lim\limits_{x\to\ 23}\frac{x^{2}-23^{2}}{(x-23)(x-30)}$.
\zadStop
\rozwStart{Patryk Wirkus}{Martyna Czarnobaj}
$$\frac{x^{2}-23^{2}}{(x-23)(x-30)}=\frac{x+23}{x-30}$$

$$\lim\limits_{x\to\ 23}\frac{x^{2}-23^{2}}{(x-23)(x-30)}=[\frac{0}{0}]=\lim\limits_{x\to\ 23}\frac{x+23}{x-30}=2 \cdot \frac{23}{23-30} = \frac{46}{-7}$$
\rozwStop
\odpStart
$\frac{46}{-7}$
\odpStop
\testStart
A.$\frac{46}{-7}$
B.$\infty$
C.$-\infty$
D.$0$
E.$\frac{46}{7}$
F.$\frac{23}{30}$
G.$-\frac{46}{7}$
H.$1$
I.$23$
\testStop
\kluczStart
A
\kluczStop



\zadStart{Przykład z Wikieł P 4.2b moja wersja nr 424}
Obliczyć granicę $\lim\limits_{x\to\ 23}\frac{x^{2}-23^{2}}{(x-23)(x-31)}$.
\zadStop
\rozwStart{Patryk Wirkus}{Martyna Czarnobaj}
$$\frac{x^{2}-23^{2}}{(x-23)(x-31)}=\frac{x+23}{x-31}$$

$$\lim\limits_{x\to\ 23}\frac{x^{2}-23^{2}}{(x-23)(x-31)}=[\frac{0}{0}]=\lim\limits_{x\to\ 23}\frac{x+23}{x-31}=2 \cdot \frac{23}{23-31} = \frac{46}{-8}$$
\rozwStop
\odpStart
$\frac{46}{-8}$
\odpStop
\testStart
A.$\frac{46}{-8}$
B.$\infty$
C.$-\infty$
D.$0$
E.$\frac{46}{8}$
F.$\frac{23}{31}$
G.$-\frac{46}{8}$
H.$1$
I.$23$
\testStop
\kluczStart
A
\kluczStop



\zadStart{Przykład z Wikieł P 4.2b moja wersja nr 425}
Obliczyć granicę $\lim\limits_{x\to\ 23}\frac{x^{2}-23^{2}}{(x-23)(x-32)}$.
\zadStop
\rozwStart{Patryk Wirkus}{Martyna Czarnobaj}
$$\frac{x^{2}-23^{2}}{(x-23)(x-32)}=\frac{x+23}{x-32}$$

$$\lim\limits_{x\to\ 23}\frac{x^{2}-23^{2}}{(x-23)(x-32)}=[\frac{0}{0}]=\lim\limits_{x\to\ 23}\frac{x+23}{x-32}=2 \cdot \frac{23}{23-32} = \frac{46}{-9}$$
\rozwStop
\odpStart
$\frac{46}{-9}$
\odpStop
\testStart
A.$\frac{46}{-9}$
B.$\infty$
C.$-\infty$
D.$0$
E.$\frac{46}{9}$
F.$\frac{23}{32}$
G.$-\frac{46}{9}$
H.$1$
I.$23$
\testStop
\kluczStart
A
\kluczStop



\zadStart{Przykład z Wikieł P 4.2b moja wersja nr 426}
Obliczyć granicę $\lim\limits_{x\to\ 23}\frac{x^{2}-23^{2}}{(x-23)(x-33)}$.
\zadStop
\rozwStart{Patryk Wirkus}{Martyna Czarnobaj}
$$\frac{x^{2}-23^{2}}{(x-23)(x-33)}=\frac{x+23}{x-33}$$

$$\lim\limits_{x\to\ 23}\frac{x^{2}-23^{2}}{(x-23)(x-33)}=[\frac{0}{0}]=\lim\limits_{x\to\ 23}\frac{x+23}{x-33}=2 \cdot \frac{23}{23-33} = \frac{46}{-10}$$
\rozwStop
\odpStart
$\frac{46}{-10}$
\odpStop
\testStart
A.$\frac{46}{-10}$
B.$\infty$
C.$-\infty$
D.$0$
E.$\frac{46}{10}$
F.$\frac{23}{33}$
G.$-\frac{46}{10}$
H.$1$
I.$23$
\testStop
\kluczStart
A
\kluczStop



\zadStart{Przykład z Wikieł P 4.2b moja wersja nr 427}
Obliczyć granicę $\lim\limits_{x\to\ 23}\frac{x^{2}-23^{2}}{(x-23)(x-34)}$.
\zadStop
\rozwStart{Patryk Wirkus}{Martyna Czarnobaj}
$$\frac{x^{2}-23^{2}}{(x-23)(x-34)}=\frac{x+23}{x-34}$$

$$\lim\limits_{x\to\ 23}\frac{x^{2}-23^{2}}{(x-23)(x-34)}=[\frac{0}{0}]=\lim\limits_{x\to\ 23}\frac{x+23}{x-34}=2 \cdot \frac{23}{23-34} = \frac{46}{-11}$$
\rozwStop
\odpStart
$\frac{46}{-11}$
\odpStop
\testStart
A.$\frac{46}{-11}$
B.$\infty$
C.$-\infty$
D.$0$
E.$\frac{46}{11}$
F.$\frac{23}{34}$
G.$-\frac{46}{11}$
H.$1$
I.$23$
\testStop
\kluczStart
A
\kluczStop



\zadStart{Przykład z Wikieł P 4.2b moja wersja nr 428}
Obliczyć granicę $\lim\limits_{x\to\ 23}\frac{x^{2}-23^{2}}{(x-23)(x-35)}$.
\zadStop
\rozwStart{Patryk Wirkus}{Martyna Czarnobaj}
$$\frac{x^{2}-23^{2}}{(x-23)(x-35)}=\frac{x+23}{x-35}$$

$$\lim\limits_{x\to\ 23}\frac{x^{2}-23^{2}}{(x-23)(x-35)}=[\frac{0}{0}]=\lim\limits_{x\to\ 23}\frac{x+23}{x-35}=2 \cdot \frac{23}{23-35} = \frac{46}{-12}$$
\rozwStop
\odpStart
$\frac{46}{-12}$
\odpStop
\testStart
A.$\frac{46}{-12}$
B.$\infty$
C.$-\infty$
D.$0$
E.$\frac{46}{12}$
F.$\frac{23}{35}$
G.$-\frac{46}{12}$
H.$1$
I.$23$
\testStop
\kluczStart
A
\kluczStop



\zadStart{Przykład z Wikieł P 4.2b moja wersja nr 429}
Obliczyć granicę $\lim\limits_{x\to\ 23}\frac{x^{2}-23^{2}}{(x-23)(x-36)}$.
\zadStop
\rozwStart{Patryk Wirkus}{Martyna Czarnobaj}
$$\frac{x^{2}-23^{2}}{(x-23)(x-36)}=\frac{x+23}{x-36}$$

$$\lim\limits_{x\to\ 23}\frac{x^{2}-23^{2}}{(x-23)(x-36)}=[\frac{0}{0}]=\lim\limits_{x\to\ 23}\frac{x+23}{x-36}=2 \cdot \frac{23}{23-36} = \frac{46}{-13}$$
\rozwStop
\odpStart
$\frac{46}{-13}$
\odpStop
\testStart
A.$\frac{46}{-13}$
B.$\infty$
C.$-\infty$
D.$0$
E.$\frac{46}{13}$
F.$\frac{23}{36}$
G.$-\frac{46}{13}$
H.$1$
I.$23$
\testStop
\kluczStart
A
\kluczStop



\zadStart{Przykład z Wikieł P 4.2b moja wersja nr 430}
Obliczyć granicę $\lim\limits_{x\to\ 23}\frac{x^{2}-23^{2}}{(x-23)(x-37)}$.
\zadStop
\rozwStart{Patryk Wirkus}{Martyna Czarnobaj}
$$\frac{x^{2}-23^{2}}{(x-23)(x-37)}=\frac{x+23}{x-37}$$

$$\lim\limits_{x\to\ 23}\frac{x^{2}-23^{2}}{(x-23)(x-37)}=[\frac{0}{0}]=\lim\limits_{x\to\ 23}\frac{x+23}{x-37}=2 \cdot \frac{23}{23-37} = \frac{46}{-14}$$
\rozwStop
\odpStart
$\frac{46}{-14}$
\odpStop
\testStart
A.$\frac{46}{-14}$
B.$\infty$
C.$-\infty$
D.$0$
E.$\frac{46}{14}$
F.$\frac{23}{37}$
G.$-\frac{46}{14}$
H.$1$
I.$23$
\testStop
\kluczStart
A
\kluczStop



\zadStart{Przykład z Wikieł P 4.2b moja wersja nr 431}
Obliczyć granicę $\lim\limits_{x\to\ 23}\frac{x^{2}-23^{2}}{(x-23)(x-38)}$.
\zadStop
\rozwStart{Patryk Wirkus}{Martyna Czarnobaj}
$$\frac{x^{2}-23^{2}}{(x-23)(x-38)}=\frac{x+23}{x-38}$$

$$\lim\limits_{x\to\ 23}\frac{x^{2}-23^{2}}{(x-23)(x-38)}=[\frac{0}{0}]=\lim\limits_{x\to\ 23}\frac{x+23}{x-38}=2 \cdot \frac{23}{23-38} = \frac{46}{-15}$$
\rozwStop
\odpStart
$\frac{46}{-15}$
\odpStop
\testStart
A.$\frac{46}{-15}$
B.$\infty$
C.$-\infty$
D.$0$
E.$\frac{46}{15}$
F.$\frac{23}{38}$
G.$-\frac{46}{15}$
H.$1$
I.$23$
\testStop
\kluczStart
A
\kluczStop



\zadStart{Przykład z Wikieł P 4.2b moja wersja nr 432}
Obliczyć granicę $\lim\limits_{x\to\ 23}\frac{x^{2}-23^{2}}{(x-23)(x-39)}$.
\zadStop
\rozwStart{Patryk Wirkus}{Martyna Czarnobaj}
$$\frac{x^{2}-23^{2}}{(x-23)(x-39)}=\frac{x+23}{x-39}$$

$$\lim\limits_{x\to\ 23}\frac{x^{2}-23^{2}}{(x-23)(x-39)}=[\frac{0}{0}]=\lim\limits_{x\to\ 23}\frac{x+23}{x-39}=2 \cdot \frac{23}{23-39} = \frac{46}{-16}$$
\rozwStop
\odpStart
$\frac{46}{-16}$
\odpStop
\testStart
A.$\frac{46}{-16}$
B.$\infty$
C.$-\infty$
D.$0$
E.$\frac{46}{16}$
F.$\frac{23}{39}$
G.$-\frac{46}{16}$
H.$1$
I.$23$
\testStop
\kluczStart
A
\kluczStop



\zadStart{Przykład z Wikieł P 4.2b moja wersja nr 433}
Obliczyć granicę $\lim\limits_{x\to\ 23}\frac{x^{2}-23^{2}}{(x-23)(x-40)}$.
\zadStop
\rozwStart{Patryk Wirkus}{Martyna Czarnobaj}
$$\frac{x^{2}-23^{2}}{(x-23)(x-40)}=\frac{x+23}{x-40}$$

$$\lim\limits_{x\to\ 23}\frac{x^{2}-23^{2}}{(x-23)(x-40)}=[\frac{0}{0}]=\lim\limits_{x\to\ 23}\frac{x+23}{x-40}=2 \cdot \frac{23}{23-40} = \frac{46}{-17}$$
\rozwStop
\odpStart
$\frac{46}{-17}$
\odpStop
\testStart
A.$\frac{46}{-17}$
B.$\infty$
C.$-\infty$
D.$0$
E.$\frac{46}{17}$
F.$\frac{23}{40}$
G.$-\frac{46}{17}$
H.$1$
I.$23$
\testStop
\kluczStart
A
\kluczStop



\zadStart{Przykład z Wikieł P 4.2b moja wersja nr 434}
Obliczyć granicę $\lim\limits_{x\to\ 24}\frac{x^{2}-24^{2}}{(x-24)(x-5)}$.
\zadStop
\rozwStart{Patryk Wirkus}{Martyna Czarnobaj}
$$\frac{x^{2}-24^{2}}{(x-24)(x-5)}=\frac{x+24}{x-5}$$

$$\lim\limits_{x\to\ 24}\frac{x^{2}-24^{2}}{(x-24)(x-5)}=[\frac{0}{0}]=\lim\limits_{x\to\ 24}\frac{x+24}{x-5}=2 \cdot \frac{24}{24-5} = \frac{48}{19}$$
\rozwStop
\odpStart
$\frac{48}{19}$
\odpStop
\testStart
A.$\frac{48}{19}$
B.$\infty$
C.$-\infty$
D.$0$
E.$\frac{48}{-19}$
F.$\frac{24}{5}$
G.$-\frac{48}{-19}$
H.$1$
I.$24$
\testStop
\kluczStart
A
\kluczStop



\zadStart{Przykład z Wikieł P 4.2b moja wersja nr 435}
Obliczyć granicę $\lim\limits_{x\to\ 24}\frac{x^{2}-24^{2}}{(x-24)(x-7)}$.
\zadStop
\rozwStart{Patryk Wirkus}{Martyna Czarnobaj}
$$\frac{x^{2}-24^{2}}{(x-24)(x-7)}=\frac{x+24}{x-7}$$

$$\lim\limits_{x\to\ 24}\frac{x^{2}-24^{2}}{(x-24)(x-7)}=[\frac{0}{0}]=\lim\limits_{x\to\ 24}\frac{x+24}{x-7}=2 \cdot \frac{24}{24-7} = \frac{48}{17}$$
\rozwStop
\odpStart
$\frac{48}{17}$
\odpStop
\testStart
A.$\frac{48}{17}$
B.$\infty$
C.$-\infty$
D.$0$
E.$\frac{48}{-17}$
F.$\frac{24}{7}$
G.$-\frac{48}{-17}$
H.$1$
I.$24$
\testStop
\kluczStart
A
\kluczStop



\zadStart{Przykład z Wikieł P 4.2b moja wersja nr 436}
Obliczyć granicę $\lim\limits_{x\to\ 24}\frac{x^{2}-24^{2}}{(x-24)(x-11)}$.
\zadStop
\rozwStart{Patryk Wirkus}{Martyna Czarnobaj}
$$\frac{x^{2}-24^{2}}{(x-24)(x-11)}=\frac{x+24}{x-11}$$

$$\lim\limits_{x\to\ 24}\frac{x^{2}-24^{2}}{(x-24)(x-11)}=[\frac{0}{0}]=\lim\limits_{x\to\ 24}\frac{x+24}{x-11}=2 \cdot \frac{24}{24-11} = \frac{48}{13}$$
\rozwStop
\odpStart
$\frac{48}{13}$
\odpStop
\testStart
A.$\frac{48}{13}$
B.$\infty$
C.$-\infty$
D.$0$
E.$\frac{48}{-13}$
F.$\frac{24}{11}$
G.$-\frac{48}{-13}$
H.$1$
I.$24$
\testStop
\kluczStart
A
\kluczStop



\zadStart{Przykład z Wikieł P 4.2b moja wersja nr 437}
Obliczyć granicę $\lim\limits_{x\to\ 24}\frac{x^{2}-24^{2}}{(x-24)(x-13)}$.
\zadStop
\rozwStart{Patryk Wirkus}{Martyna Czarnobaj}
$$\frac{x^{2}-24^{2}}{(x-24)(x-13)}=\frac{x+24}{x-13}$$

$$\lim\limits_{x\to\ 24}\frac{x^{2}-24^{2}}{(x-24)(x-13)}=[\frac{0}{0}]=\lim\limits_{x\to\ 24}\frac{x+24}{x-13}=2 \cdot \frac{24}{24-13} = \frac{48}{11}$$
\rozwStop
\odpStart
$\frac{48}{11}$
\odpStop
\testStart
A.$\frac{48}{11}$
B.$\infty$
C.$-\infty$
D.$0$
E.$\frac{48}{-11}$
F.$\frac{24}{13}$
G.$-\frac{48}{-11}$
H.$1$
I.$24$
\testStop
\kluczStart
A
\kluczStop



\zadStart{Przykład z Wikieł P 4.2b moja wersja nr 438}
Obliczyć granicę $\lim\limits_{x\to\ 24}\frac{x^{2}-24^{2}}{(x-24)(x-17)}$.
\zadStop
\rozwStart{Patryk Wirkus}{Martyna Czarnobaj}
$$\frac{x^{2}-24^{2}}{(x-24)(x-17)}=\frac{x+24}{x-17}$$

$$\lim\limits_{x\to\ 24}\frac{x^{2}-24^{2}}{(x-24)(x-17)}=[\frac{0}{0}]=\lim\limits_{x\to\ 24}\frac{x+24}{x-17}=2 \cdot \frac{24}{24-17} = \frac{48}{7}$$
\rozwStop
\odpStart
$\frac{48}{7}$
\odpStop
\testStart
A.$\frac{48}{7}$
B.$\infty$
C.$-\infty$
D.$0$
E.$\frac{48}{-7}$
F.$\frac{24}{17}$
G.$-\frac{48}{-7}$
H.$1$
I.$24$
\testStop
\kluczStart
A
\kluczStop



\zadStart{Przykład z Wikieł P 4.2b moja wersja nr 439}
Obliczyć granicę $\lim\limits_{x\to\ 24}\frac{x^{2}-24^{2}}{(x-24)(x-19)}$.
\zadStop
\rozwStart{Patryk Wirkus}{Martyna Czarnobaj}
$$\frac{x^{2}-24^{2}}{(x-24)(x-19)}=\frac{x+24}{x-19}$$

$$\lim\limits_{x\to\ 24}\frac{x^{2}-24^{2}}{(x-24)(x-19)}=[\frac{0}{0}]=\lim\limits_{x\to\ 24}\frac{x+24}{x-19}=2 \cdot \frac{24}{24-19} = \frac{48}{5}$$
\rozwStop
\odpStart
$\frac{48}{5}$
\odpStop
\testStart
A.$\frac{48}{5}$
B.$\infty$
C.$-\infty$
D.$0$
E.$\frac{48}{-5}$
F.$\frac{24}{19}$
G.$-\frac{48}{-5}$
H.$1$
I.$24$
\testStop
\kluczStart
A
\kluczStop



\zadStart{Przykład z Wikieł P 4.2b moja wersja nr 440}
Obliczyć granicę $\lim\limits_{x\to\ 24}\frac{x^{2}-24^{2}}{(x-24)(x-29)}$.
\zadStop
\rozwStart{Patryk Wirkus}{Martyna Czarnobaj}
$$\frac{x^{2}-24^{2}}{(x-24)(x-29)}=\frac{x+24}{x-29}$$

$$\lim\limits_{x\to\ 24}\frac{x^{2}-24^{2}}{(x-24)(x-29)}=[\frac{0}{0}]=\lim\limits_{x\to\ 24}\frac{x+24}{x-29}=2 \cdot \frac{24}{24-29} = \frac{48}{-5}$$
\rozwStop
\odpStart
$\frac{48}{-5}$
\odpStop
\testStart
A.$\frac{48}{-5}$
B.$\infty$
C.$-\infty$
D.$0$
E.$\frac{48}{5}$
F.$\frac{24}{29}$
G.$-\frac{48}{5}$
H.$1$
I.$24$
\testStop
\kluczStart
A
\kluczStop



\zadStart{Przykład z Wikieł P 4.2b moja wersja nr 441}
Obliczyć granicę $\lim\limits_{x\to\ 24}\frac{x^{2}-24^{2}}{(x-24)(x-31)}$.
\zadStop
\rozwStart{Patryk Wirkus}{Martyna Czarnobaj}
$$\frac{x^{2}-24^{2}}{(x-24)(x-31)}=\frac{x+24}{x-31}$$

$$\lim\limits_{x\to\ 24}\frac{x^{2}-24^{2}}{(x-24)(x-31)}=[\frac{0}{0}]=\lim\limits_{x\to\ 24}\frac{x+24}{x-31}=2 \cdot \frac{24}{24-31} = \frac{48}{-7}$$
\rozwStop
\odpStart
$\frac{48}{-7}$
\odpStop
\testStart
A.$\frac{48}{-7}$
B.$\infty$
C.$-\infty$
D.$0$
E.$\frac{48}{7}$
F.$\frac{24}{31}$
G.$-\frac{48}{7}$
H.$1$
I.$24$
\testStop
\kluczStart
A
\kluczStop



\zadStart{Przykład z Wikieł P 4.2b moja wersja nr 442}
Obliczyć granicę $\lim\limits_{x\to\ 24}\frac{x^{2}-24^{2}}{(x-24)(x-35)}$.
\zadStop
\rozwStart{Patryk Wirkus}{Martyna Czarnobaj}
$$\frac{x^{2}-24^{2}}{(x-24)(x-35)}=\frac{x+24}{x-35}$$

$$\lim\limits_{x\to\ 24}\frac{x^{2}-24^{2}}{(x-24)(x-35)}=[\frac{0}{0}]=\lim\limits_{x\to\ 24}\frac{x+24}{x-35}=2 \cdot \frac{24}{24-35} = \frac{48}{-11}$$
\rozwStop
\odpStart
$\frac{48}{-11}$
\odpStop
\testStart
A.$\frac{48}{-11}$
B.$\infty$
C.$-\infty$
D.$0$
E.$\frac{48}{11}$
F.$\frac{24}{35}$
G.$-\frac{48}{11}$
H.$1$
I.$24$
\testStop
\kluczStart
A
\kluczStop



\zadStart{Przykład z Wikieł P 4.2b moja wersja nr 443}
Obliczyć granicę $\lim\limits_{x\to\ 24}\frac{x^{2}-24^{2}}{(x-24)(x-37)}$.
\zadStop
\rozwStart{Patryk Wirkus}{Martyna Czarnobaj}
$$\frac{x^{2}-24^{2}}{(x-24)(x-37)}=\frac{x+24}{x-37}$$

$$\lim\limits_{x\to\ 24}\frac{x^{2}-24^{2}}{(x-24)(x-37)}=[\frac{0}{0}]=\lim\limits_{x\to\ 24}\frac{x+24}{x-37}=2 \cdot \frac{24}{24-37} = \frac{48}{-13}$$
\rozwStop
\odpStart
$\frac{48}{-13}$
\odpStop
\testStart
A.$\frac{48}{-13}$
B.$\infty$
C.$-\infty$
D.$0$
E.$\frac{48}{13}$
F.$\frac{24}{37}$
G.$-\frac{48}{13}$
H.$1$
I.$24$
\testStop
\kluczStart
A
\kluczStop



\zadStart{Przykład z Wikieł P 4.2b moja wersja nr 444}
Obliczyć granicę $\lim\limits_{x\to\ 25}\frac{x^{2}-25^{2}}{(x-25)(x-2)}$.
\zadStop
\rozwStart{Patryk Wirkus}{Martyna Czarnobaj}
$$\frac{x^{2}-25^{2}}{(x-25)(x-2)}=\frac{x+25}{x-2}$$

$$\lim\limits_{x\to\ 25}\frac{x^{2}-25^{2}}{(x-25)(x-2)}=[\frac{0}{0}]=\lim\limits_{x\to\ 25}\frac{x+25}{x-2}=2 \cdot \frac{25}{25-2} = \frac{50}{23}$$
\rozwStop
\odpStart
$\frac{50}{23}$
\odpStop
\testStart
A.$\frac{50}{23}$
B.$\infty$
C.$-\infty$
D.$0$
E.$\frac{50}{-23}$
F.$\frac{25}{2}$
G.$-\frac{50}{-23}$
H.$1$
I.$25$
\testStop
\kluczStart
A
\kluczStop



\zadStart{Przykład z Wikieł P 4.2b moja wersja nr 445}
Obliczyć granicę $\lim\limits_{x\to\ 25}\frac{x^{2}-25^{2}}{(x-25)(x-3)}$.
\zadStop
\rozwStart{Patryk Wirkus}{Martyna Czarnobaj}
$$\frac{x^{2}-25^{2}}{(x-25)(x-3)}=\frac{x+25}{x-3}$$

$$\lim\limits_{x\to\ 25}\frac{x^{2}-25^{2}}{(x-25)(x-3)}=[\frac{0}{0}]=\lim\limits_{x\to\ 25}\frac{x+25}{x-3}=2 \cdot \frac{25}{25-3} = \frac{50}{22}$$
\rozwStop
\odpStart
$\frac{50}{22}$
\odpStop
\testStart
A.$\frac{50}{22}$
B.$\infty$
C.$-\infty$
D.$0$
E.$\frac{50}{-22}$
F.$\frac{25}{3}$
G.$-\frac{50}{-22}$
H.$1$
I.$25$
\testStop
\kluczStart
A
\kluczStop



\zadStart{Przykład z Wikieł P 4.2b moja wersja nr 446}
Obliczyć granicę $\lim\limits_{x\to\ 25}\frac{x^{2}-25^{2}}{(x-25)(x-4)}$.
\zadStop
\rozwStart{Patryk Wirkus}{Martyna Czarnobaj}
$$\frac{x^{2}-25^{2}}{(x-25)(x-4)}=\frac{x+25}{x-4}$$

$$\lim\limits_{x\to\ 25}\frac{x^{2}-25^{2}}{(x-25)(x-4)}=[\frac{0}{0}]=\lim\limits_{x\to\ 25}\frac{x+25}{x-4}=2 \cdot \frac{25}{25-4} = \frac{50}{21}$$
\rozwStop
\odpStart
$\frac{50}{21}$
\odpStop
\testStart
A.$\frac{50}{21}$
B.$\infty$
C.$-\infty$
D.$0$
E.$\frac{50}{-21}$
F.$\frac{25}{4}$
G.$-\frac{50}{-21}$
H.$1$
I.$25$
\testStop
\kluczStart
A
\kluczStop



\zadStart{Przykład z Wikieł P 4.2b moja wersja nr 447}
Obliczyć granicę $\lim\limits_{x\to\ 25}\frac{x^{2}-25^{2}}{(x-25)(x-6)}$.
\zadStop
\rozwStart{Patryk Wirkus}{Martyna Czarnobaj}
$$\frac{x^{2}-25^{2}}{(x-25)(x-6)}=\frac{x+25}{x-6}$$

$$\lim\limits_{x\to\ 25}\frac{x^{2}-25^{2}}{(x-25)(x-6)}=[\frac{0}{0}]=\lim\limits_{x\to\ 25}\frac{x+25}{x-6}=2 \cdot \frac{25}{25-6} = \frac{50}{19}$$
\rozwStop
\odpStart
$\frac{50}{19}$
\odpStop
\testStart
A.$\frac{50}{19}$
B.$\infty$
C.$-\infty$
D.$0$
E.$\frac{50}{-19}$
F.$\frac{25}{6}$
G.$-\frac{50}{-19}$
H.$1$
I.$25$
\testStop
\kluczStart
A
\kluczStop



\zadStart{Przykład z Wikieł P 4.2b moja wersja nr 448}
Obliczyć granicę $\lim\limits_{x\to\ 25}\frac{x^{2}-25^{2}}{(x-25)(x-7)}$.
\zadStop
\rozwStart{Patryk Wirkus}{Martyna Czarnobaj}
$$\frac{x^{2}-25^{2}}{(x-25)(x-7)}=\frac{x+25}{x-7}$$

$$\lim\limits_{x\to\ 25}\frac{x^{2}-25^{2}}{(x-25)(x-7)}=[\frac{0}{0}]=\lim\limits_{x\to\ 25}\frac{x+25}{x-7}=2 \cdot \frac{25}{25-7} = \frac{50}{18}$$
\rozwStop
\odpStart
$\frac{50}{18}$
\odpStop
\testStart
A.$\frac{50}{18}$
B.$\infty$
C.$-\infty$
D.$0$
E.$\frac{50}{-18}$
F.$\frac{25}{7}$
G.$-\frac{50}{-18}$
H.$1$
I.$25$
\testStop
\kluczStart
A
\kluczStop



\zadStart{Przykład z Wikieł P 4.2b moja wersja nr 449}
Obliczyć granicę $\lim\limits_{x\to\ 25}\frac{x^{2}-25^{2}}{(x-25)(x-8)}$.
\zadStop
\rozwStart{Patryk Wirkus}{Martyna Czarnobaj}
$$\frac{x^{2}-25^{2}}{(x-25)(x-8)}=\frac{x+25}{x-8}$$

$$\lim\limits_{x\to\ 25}\frac{x^{2}-25^{2}}{(x-25)(x-8)}=[\frac{0}{0}]=\lim\limits_{x\to\ 25}\frac{x+25}{x-8}=2 \cdot \frac{25}{25-8} = \frac{50}{17}$$
\rozwStop
\odpStart
$\frac{50}{17}$
\odpStop
\testStart
A.$\frac{50}{17}$
B.$\infty$
C.$-\infty$
D.$0$
E.$\frac{50}{-17}$
F.$\frac{25}{8}$
G.$-\frac{50}{-17}$
H.$1$
I.$25$
\testStop
\kluczStart
A
\kluczStop



\zadStart{Przykład z Wikieł P 4.2b moja wersja nr 450}
Obliczyć granicę $\lim\limits_{x\to\ 25}\frac{x^{2}-25^{2}}{(x-25)(x-9)}$.
\zadStop
\rozwStart{Patryk Wirkus}{Martyna Czarnobaj}
$$\frac{x^{2}-25^{2}}{(x-25)(x-9)}=\frac{x+25}{x-9}$$

$$\lim\limits_{x\to\ 25}\frac{x^{2}-25^{2}}{(x-25)(x-9)}=[\frac{0}{0}]=\lim\limits_{x\to\ 25}\frac{x+25}{x-9}=2 \cdot \frac{25}{25-9} = \frac{50}{16}$$
\rozwStop
\odpStart
$\frac{50}{16}$
\odpStop
\testStart
A.$\frac{50}{16}$
B.$\infty$
C.$-\infty$
D.$0$
E.$\frac{50}{-16}$
F.$\frac{25}{9}$
G.$-\frac{50}{-16}$
H.$1$
I.$25$
\testStop
\kluczStart
A
\kluczStop



\zadStart{Przykład z Wikieł P 4.2b moja wersja nr 451}
Obliczyć granicę $\lim\limits_{x\to\ 25}\frac{x^{2}-25^{2}}{(x-25)(x-11)}$.
\zadStop
\rozwStart{Patryk Wirkus}{Martyna Czarnobaj}
$$\frac{x^{2}-25^{2}}{(x-25)(x-11)}=\frac{x+25}{x-11}$$

$$\lim\limits_{x\to\ 25}\frac{x^{2}-25^{2}}{(x-25)(x-11)}=[\frac{0}{0}]=\lim\limits_{x\to\ 25}\frac{x+25}{x-11}=2 \cdot \frac{25}{25-11} = \frac{50}{14}$$
\rozwStop
\odpStart
$\frac{50}{14}$
\odpStop
\testStart
A.$\frac{50}{14}$
B.$\infty$
C.$-\infty$
D.$0$
E.$\frac{50}{-14}$
F.$\frac{25}{11}$
G.$-\frac{50}{-14}$
H.$1$
I.$25$
\testStop
\kluczStart
A
\kluczStop



\zadStart{Przykład z Wikieł P 4.2b moja wersja nr 452}
Obliczyć granicę $\lim\limits_{x\to\ 25}\frac{x^{2}-25^{2}}{(x-25)(x-12)}$.
\zadStop
\rozwStart{Patryk Wirkus}{Martyna Czarnobaj}
$$\frac{x^{2}-25^{2}}{(x-25)(x-12)}=\frac{x+25}{x-12}$$

$$\lim\limits_{x\to\ 25}\frac{x^{2}-25^{2}}{(x-25)(x-12)}=[\frac{0}{0}]=\lim\limits_{x\to\ 25}\frac{x+25}{x-12}=2 \cdot \frac{25}{25-12} = \frac{50}{13}$$
\rozwStop
\odpStart
$\frac{50}{13}$
\odpStop
\testStart
A.$\frac{50}{13}$
B.$\infty$
C.$-\infty$
D.$0$
E.$\frac{50}{-13}$
F.$\frac{25}{12}$
G.$-\frac{50}{-13}$
H.$1$
I.$25$
\testStop
\kluczStart
A
\kluczStop



\zadStart{Przykład z Wikieł P 4.2b moja wersja nr 453}
Obliczyć granicę $\lim\limits_{x\to\ 25}\frac{x^{2}-25^{2}}{(x-25)(x-13)}$.
\zadStop
\rozwStart{Patryk Wirkus}{Martyna Czarnobaj}
$$\frac{x^{2}-25^{2}}{(x-25)(x-13)}=\frac{x+25}{x-13}$$

$$\lim\limits_{x\to\ 25}\frac{x^{2}-25^{2}}{(x-25)(x-13)}=[\frac{0}{0}]=\lim\limits_{x\to\ 25}\frac{x+25}{x-13}=2 \cdot \frac{25}{25-13} = \frac{50}{12}$$
\rozwStop
\odpStart
$\frac{50}{12}$
\odpStop
\testStart
A.$\frac{50}{12}$
B.$\infty$
C.$-\infty$
D.$0$
E.$\frac{50}{-12}$
F.$\frac{25}{13}$
G.$-\frac{50}{-12}$
H.$1$
I.$25$
\testStop
\kluczStart
A
\kluczStop



\zadStart{Przykład z Wikieł P 4.2b moja wersja nr 454}
Obliczyć granicę $\lim\limits_{x\to\ 25}\frac{x^{2}-25^{2}}{(x-25)(x-14)}$.
\zadStop
\rozwStart{Patryk Wirkus}{Martyna Czarnobaj}
$$\frac{x^{2}-25^{2}}{(x-25)(x-14)}=\frac{x+25}{x-14}$$

$$\lim\limits_{x\to\ 25}\frac{x^{2}-25^{2}}{(x-25)(x-14)}=[\frac{0}{0}]=\lim\limits_{x\to\ 25}\frac{x+25}{x-14}=2 \cdot \frac{25}{25-14} = \frac{50}{11}$$
\rozwStop
\odpStart
$\frac{50}{11}$
\odpStop
\testStart
A.$\frac{50}{11}$
B.$\infty$
C.$-\infty$
D.$0$
E.$\frac{50}{-11}$
F.$\frac{25}{14}$
G.$-\frac{50}{-11}$
H.$1$
I.$25$
\testStop
\kluczStart
A
\kluczStop



\zadStart{Przykład z Wikieł P 4.2b moja wersja nr 455}
Obliczyć granicę $\lim\limits_{x\to\ 25}\frac{x^{2}-25^{2}}{(x-25)(x-16)}$.
\zadStop
\rozwStart{Patryk Wirkus}{Martyna Czarnobaj}
$$\frac{x^{2}-25^{2}}{(x-25)(x-16)}=\frac{x+25}{x-16}$$

$$\lim\limits_{x\to\ 25}\frac{x^{2}-25^{2}}{(x-25)(x-16)}=[\frac{0}{0}]=\lim\limits_{x\to\ 25}\frac{x+25}{x-16}=2 \cdot \frac{25}{25-16} = \frac{50}{9}$$
\rozwStop
\odpStart
$\frac{50}{9}$
\odpStop
\testStart
A.$\frac{50}{9}$
B.$\infty$
C.$-\infty$
D.$0$
E.$\frac{50}{-9}$
F.$\frac{25}{16}$
G.$-\frac{50}{-9}$
H.$1$
I.$25$
\testStop
\kluczStart
A
\kluczStop



\zadStart{Przykład z Wikieł P 4.2b moja wersja nr 456}
Obliczyć granicę $\lim\limits_{x\to\ 25}\frac{x^{2}-25^{2}}{(x-25)(x-17)}$.
\zadStop
\rozwStart{Patryk Wirkus}{Martyna Czarnobaj}
$$\frac{x^{2}-25^{2}}{(x-25)(x-17)}=\frac{x+25}{x-17}$$

$$\lim\limits_{x\to\ 25}\frac{x^{2}-25^{2}}{(x-25)(x-17)}=[\frac{0}{0}]=\lim\limits_{x\to\ 25}\frac{x+25}{x-17}=2 \cdot \frac{25}{25-17} = \frac{50}{8}$$
\rozwStop
\odpStart
$\frac{50}{8}$
\odpStop
\testStart
A.$\frac{50}{8}$
B.$\infty$
C.$-\infty$
D.$0$
E.$\frac{50}{-8}$
F.$\frac{25}{17}$
G.$-\frac{50}{-8}$
H.$1$
I.$25$
\testStop
\kluczStart
A
\kluczStop



\zadStart{Przykład z Wikieł P 4.2b moja wersja nr 457}
Obliczyć granicę $\lim\limits_{x\to\ 25}\frac{x^{2}-25^{2}}{(x-25)(x-18)}$.
\zadStop
\rozwStart{Patryk Wirkus}{Martyna Czarnobaj}
$$\frac{x^{2}-25^{2}}{(x-25)(x-18)}=\frac{x+25}{x-18}$$

$$\lim\limits_{x\to\ 25}\frac{x^{2}-25^{2}}{(x-25)(x-18)}=[\frac{0}{0}]=\lim\limits_{x\to\ 25}\frac{x+25}{x-18}=2 \cdot \frac{25}{25-18} = \frac{50}{7}$$
\rozwStop
\odpStart
$\frac{50}{7}$
\odpStop
\testStart
A.$\frac{50}{7}$
B.$\infty$
C.$-\infty$
D.$0$
E.$\frac{50}{-7}$
F.$\frac{25}{18}$
G.$-\frac{50}{-7}$
H.$1$
I.$25$
\testStop
\kluczStart
A
\kluczStop



\zadStart{Przykład z Wikieł P 4.2b moja wersja nr 458}
Obliczyć granicę $\lim\limits_{x\to\ 25}\frac{x^{2}-25^{2}}{(x-25)(x-19)}$.
\zadStop
\rozwStart{Patryk Wirkus}{Martyna Czarnobaj}
$$\frac{x^{2}-25^{2}}{(x-25)(x-19)}=\frac{x+25}{x-19}$$

$$\lim\limits_{x\to\ 25}\frac{x^{2}-25^{2}}{(x-25)(x-19)}=[\frac{0}{0}]=\lim\limits_{x\to\ 25}\frac{x+25}{x-19}=2 \cdot \frac{25}{25-19} = \frac{50}{6}$$
\rozwStop
\odpStart
$\frac{50}{6}$
\odpStop
\testStart
A.$\frac{50}{6}$
B.$\infty$
C.$-\infty$
D.$0$
E.$\frac{50}{-6}$
F.$\frac{25}{19}$
G.$-\frac{50}{-6}$
H.$1$
I.$25$
\testStop
\kluczStart
A
\kluczStop



\zadStart{Przykład z Wikieł P 4.2b moja wersja nr 459}
Obliczyć granicę $\lim\limits_{x\to\ 25}\frac{x^{2}-25^{2}}{(x-25)(x-21)}$.
\zadStop
\rozwStart{Patryk Wirkus}{Martyna Czarnobaj}
$$\frac{x^{2}-25^{2}}{(x-25)(x-21)}=\frac{x+25}{x-21}$$

$$\lim\limits_{x\to\ 25}\frac{x^{2}-25^{2}}{(x-25)(x-21)}=[\frac{0}{0}]=\lim\limits_{x\to\ 25}\frac{x+25}{x-21}=2 \cdot \frac{25}{25-21} = \frac{50}{4}$$
\rozwStop
\odpStart
$\frac{50}{4}$
\odpStop
\testStart
A.$\frac{50}{4}$
B.$\infty$
C.$-\infty$
D.$0$
E.$\frac{50}{-4}$
F.$\frac{25}{21}$
G.$-\frac{50}{-4}$
H.$1$
I.$25$
\testStop
\kluczStart
A
\kluczStop



\zadStart{Przykład z Wikieł P 4.2b moja wersja nr 460}
Obliczyć granicę $\lim\limits_{x\to\ 25}\frac{x^{2}-25^{2}}{(x-25)(x-22)}$.
\zadStop
\rozwStart{Patryk Wirkus}{Martyna Czarnobaj}
$$\frac{x^{2}-25^{2}}{(x-25)(x-22)}=\frac{x+25}{x-22}$$

$$\lim\limits_{x\to\ 25}\frac{x^{2}-25^{2}}{(x-25)(x-22)}=[\frac{0}{0}]=\lim\limits_{x\to\ 25}\frac{x+25}{x-22}=2 \cdot \frac{25}{25-22} = \frac{50}{3}$$
\rozwStop
\odpStart
$\frac{50}{3}$
\odpStop
\testStart
A.$\frac{50}{3}$
B.$\infty$
C.$-\infty$
D.$0$
E.$\frac{50}{-3}$
F.$\frac{25}{22}$
G.$-\frac{50}{-3}$
H.$1$
I.$25$
\testStop
\kluczStart
A
\kluczStop



\zadStart{Przykład z Wikieł P 4.2b moja wersja nr 461}
Obliczyć granicę $\lim\limits_{x\to\ 25}\frac{x^{2}-25^{2}}{(x-25)(x-23)}$.
\zadStop
\rozwStart{Patryk Wirkus}{Martyna Czarnobaj}
$$\frac{x^{2}-25^{2}}{(x-25)(x-23)}=\frac{x+25}{x-23}$$

$$\lim\limits_{x\to\ 25}\frac{x^{2}-25^{2}}{(x-25)(x-23)}=[\frac{0}{0}]=\lim\limits_{x\to\ 25}\frac{x+25}{x-23}=2 \cdot \frac{25}{25-23} = \frac{50}{2}$$
\rozwStop
\odpStart
$\frac{50}{2}$
\odpStop
\testStart
A.$\frac{50}{2}$
B.$\infty$
C.$-\infty$
D.$0$
E.$\frac{50}{-2}$
F.$\frac{25}{23}$
G.$-\frac{50}{-2}$
H.$1$
I.$25$
\testStop
\kluczStart
A
\kluczStop



\zadStart{Przykład z Wikieł P 4.2b moja wersja nr 462}
Obliczyć granicę $\lim\limits_{x\to\ 25}\frac{x^{2}-25^{2}}{(x-25)(x-27)}$.
\zadStop
\rozwStart{Patryk Wirkus}{Martyna Czarnobaj}
$$\frac{x^{2}-25^{2}}{(x-25)(x-27)}=\frac{x+25}{x-27}$$

$$\lim\limits_{x\to\ 25}\frac{x^{2}-25^{2}}{(x-25)(x-27)}=[\frac{0}{0}]=\lim\limits_{x\to\ 25}\frac{x+25}{x-27}=2 \cdot \frac{25}{25-27} = \frac{50}{-2}$$
\rozwStop
\odpStart
$\frac{50}{-2}$
\odpStop
\testStart
A.$\frac{50}{-2}$
B.$\infty$
C.$-\infty$
D.$0$
E.$\frac{50}{2}$
F.$\frac{25}{27}$
G.$-\frac{50}{2}$
H.$1$
I.$25$
\testStop
\kluczStart
A
\kluczStop



\zadStart{Przykład z Wikieł P 4.2b moja wersja nr 463}
Obliczyć granicę $\lim\limits_{x\to\ 25}\frac{x^{2}-25^{2}}{(x-25)(x-28)}$.
\zadStop
\rozwStart{Patryk Wirkus}{Martyna Czarnobaj}
$$\frac{x^{2}-25^{2}}{(x-25)(x-28)}=\frac{x+25}{x-28}$$

$$\lim\limits_{x\to\ 25}\frac{x^{2}-25^{2}}{(x-25)(x-28)}=[\frac{0}{0}]=\lim\limits_{x\to\ 25}\frac{x+25}{x-28}=2 \cdot \frac{25}{25-28} = \frac{50}{-3}$$
\rozwStop
\odpStart
$\frac{50}{-3}$
\odpStop
\testStart
A.$\frac{50}{-3}$
B.$\infty$
C.$-\infty$
D.$0$
E.$\frac{50}{3}$
F.$\frac{25}{28}$
G.$-\frac{50}{3}$
H.$1$
I.$25$
\testStop
\kluczStart
A
\kluczStop



\zadStart{Przykład z Wikieł P 4.2b moja wersja nr 464}
Obliczyć granicę $\lim\limits_{x\to\ 25}\frac{x^{2}-25^{2}}{(x-25)(x-29)}$.
\zadStop
\rozwStart{Patryk Wirkus}{Martyna Czarnobaj}
$$\frac{x^{2}-25^{2}}{(x-25)(x-29)}=\frac{x+25}{x-29}$$

$$\lim\limits_{x\to\ 25}\frac{x^{2}-25^{2}}{(x-25)(x-29)}=[\frac{0}{0}]=\lim\limits_{x\to\ 25}\frac{x+25}{x-29}=2 \cdot \frac{25}{25-29} = \frac{50}{-4}$$
\rozwStop
\odpStart
$\frac{50}{-4}$
\odpStop
\testStart
A.$\frac{50}{-4}$
B.$\infty$
C.$-\infty$
D.$0$
E.$\frac{50}{4}$
F.$\frac{25}{29}$
G.$-\frac{50}{4}$
H.$1$
I.$25$
\testStop
\kluczStart
A
\kluczStop



\zadStart{Przykład z Wikieł P 4.2b moja wersja nr 465}
Obliczyć granicę $\lim\limits_{x\to\ 25}\frac{x^{2}-25^{2}}{(x-25)(x-31)}$.
\zadStop
\rozwStart{Patryk Wirkus}{Martyna Czarnobaj}
$$\frac{x^{2}-25^{2}}{(x-25)(x-31)}=\frac{x+25}{x-31}$$

$$\lim\limits_{x\to\ 25}\frac{x^{2}-25^{2}}{(x-25)(x-31)}=[\frac{0}{0}]=\lim\limits_{x\to\ 25}\frac{x+25}{x-31}=2 \cdot \frac{25}{25-31} = \frac{50}{-6}$$
\rozwStop
\odpStart
$\frac{50}{-6}$
\odpStop
\testStart
A.$\frac{50}{-6}$
B.$\infty$
C.$-\infty$
D.$0$
E.$\frac{50}{6}$
F.$\frac{25}{31}$
G.$-\frac{50}{6}$
H.$1$
I.$25$
\testStop
\kluczStart
A
\kluczStop



\zadStart{Przykład z Wikieł P 4.2b moja wersja nr 466}
Obliczyć granicę $\lim\limits_{x\to\ 25}\frac{x^{2}-25^{2}}{(x-25)(x-32)}$.
\zadStop
\rozwStart{Patryk Wirkus}{Martyna Czarnobaj}
$$\frac{x^{2}-25^{2}}{(x-25)(x-32)}=\frac{x+25}{x-32}$$

$$\lim\limits_{x\to\ 25}\frac{x^{2}-25^{2}}{(x-25)(x-32)}=[\frac{0}{0}]=\lim\limits_{x\to\ 25}\frac{x+25}{x-32}=2 \cdot \frac{25}{25-32} = \frac{50}{-7}$$
\rozwStop
\odpStart
$\frac{50}{-7}$
\odpStop
\testStart
A.$\frac{50}{-7}$
B.$\infty$
C.$-\infty$
D.$0$
E.$\frac{50}{7}$
F.$\frac{25}{32}$
G.$-\frac{50}{7}$
H.$1$
I.$25$
\testStop
\kluczStart
A
\kluczStop



\zadStart{Przykład z Wikieł P 4.2b moja wersja nr 467}
Obliczyć granicę $\lim\limits_{x\to\ 25}\frac{x^{2}-25^{2}}{(x-25)(x-33)}$.
\zadStop
\rozwStart{Patryk Wirkus}{Martyna Czarnobaj}
$$\frac{x^{2}-25^{2}}{(x-25)(x-33)}=\frac{x+25}{x-33}$$

$$\lim\limits_{x\to\ 25}\frac{x^{2}-25^{2}}{(x-25)(x-33)}=[\frac{0}{0}]=\lim\limits_{x\to\ 25}\frac{x+25}{x-33}=2 \cdot \frac{25}{25-33} = \frac{50}{-8}$$
\rozwStop
\odpStart
$\frac{50}{-8}$
\odpStop
\testStart
A.$\frac{50}{-8}$
B.$\infty$
C.$-\infty$
D.$0$
E.$\frac{50}{8}$
F.$\frac{25}{33}$
G.$-\frac{50}{8}$
H.$1$
I.$25$
\testStop
\kluczStart
A
\kluczStop



\zadStart{Przykład z Wikieł P 4.2b moja wersja nr 468}
Obliczyć granicę $\lim\limits_{x\to\ 25}\frac{x^{2}-25^{2}}{(x-25)(x-34)}$.
\zadStop
\rozwStart{Patryk Wirkus}{Martyna Czarnobaj}
$$\frac{x^{2}-25^{2}}{(x-25)(x-34)}=\frac{x+25}{x-34}$$

$$\lim\limits_{x\to\ 25}\frac{x^{2}-25^{2}}{(x-25)(x-34)}=[\frac{0}{0}]=\lim\limits_{x\to\ 25}\frac{x+25}{x-34}=2 \cdot \frac{25}{25-34} = \frac{50}{-9}$$
\rozwStop
\odpStart
$\frac{50}{-9}$
\odpStop
\testStart
A.$\frac{50}{-9}$
B.$\infty$
C.$-\infty$
D.$0$
E.$\frac{50}{9}$
F.$\frac{25}{34}$
G.$-\frac{50}{9}$
H.$1$
I.$25$
\testStop
\kluczStart
A
\kluczStop



\zadStart{Przykład z Wikieł P 4.2b moja wersja nr 469}
Obliczyć granicę $\lim\limits_{x\to\ 25}\frac{x^{2}-25^{2}}{(x-25)(x-36)}$.
\zadStop
\rozwStart{Patryk Wirkus}{Martyna Czarnobaj}
$$\frac{x^{2}-25^{2}}{(x-25)(x-36)}=\frac{x+25}{x-36}$$

$$\lim\limits_{x\to\ 25}\frac{x^{2}-25^{2}}{(x-25)(x-36)}=[\frac{0}{0}]=\lim\limits_{x\to\ 25}\frac{x+25}{x-36}=2 \cdot \frac{25}{25-36} = \frac{50}{-11}$$
\rozwStop
\odpStart
$\frac{50}{-11}$
\odpStop
\testStart
A.$\frac{50}{-11}$
B.$\infty$
C.$-\infty$
D.$0$
E.$\frac{50}{11}$
F.$\frac{25}{36}$
G.$-\frac{50}{11}$
H.$1$
I.$25$
\testStop
\kluczStart
A
\kluczStop



\zadStart{Przykład z Wikieł P 4.2b moja wersja nr 470}
Obliczyć granicę $\lim\limits_{x\to\ 25}\frac{x^{2}-25^{2}}{(x-25)(x-37)}$.
\zadStop
\rozwStart{Patryk Wirkus}{Martyna Czarnobaj}
$$\frac{x^{2}-25^{2}}{(x-25)(x-37)}=\frac{x+25}{x-37}$$

$$\lim\limits_{x\to\ 25}\frac{x^{2}-25^{2}}{(x-25)(x-37)}=[\frac{0}{0}]=\lim\limits_{x\to\ 25}\frac{x+25}{x-37}=2 \cdot \frac{25}{25-37} = \frac{50}{-12}$$
\rozwStop
\odpStart
$\frac{50}{-12}$
\odpStop
\testStart
A.$\frac{50}{-12}$
B.$\infty$
C.$-\infty$
D.$0$
E.$\frac{50}{12}$
F.$\frac{25}{37}$
G.$-\frac{50}{12}$
H.$1$
I.$25$
\testStop
\kluczStart
A
\kluczStop



\zadStart{Przykład z Wikieł P 4.2b moja wersja nr 471}
Obliczyć granicę $\lim\limits_{x\to\ 25}\frac{x^{2}-25^{2}}{(x-25)(x-38)}$.
\zadStop
\rozwStart{Patryk Wirkus}{Martyna Czarnobaj}
$$\frac{x^{2}-25^{2}}{(x-25)(x-38)}=\frac{x+25}{x-38}$$

$$\lim\limits_{x\to\ 25}\frac{x^{2}-25^{2}}{(x-25)(x-38)}=[\frac{0}{0}]=\lim\limits_{x\to\ 25}\frac{x+25}{x-38}=2 \cdot \frac{25}{25-38} = \frac{50}{-13}$$
\rozwStop
\odpStart
$\frac{50}{-13}$
\odpStop
\testStart
A.$\frac{50}{-13}$
B.$\infty$
C.$-\infty$
D.$0$
E.$\frac{50}{13}$
F.$\frac{25}{38}$
G.$-\frac{50}{13}$
H.$1$
I.$25$
\testStop
\kluczStart
A
\kluczStop



\zadStart{Przykład z Wikieł P 4.2b moja wersja nr 472}
Obliczyć granicę $\lim\limits_{x\to\ 25}\frac{x^{2}-25^{2}}{(x-25)(x-39)}$.
\zadStop
\rozwStart{Patryk Wirkus}{Martyna Czarnobaj}
$$\frac{x^{2}-25^{2}}{(x-25)(x-39)}=\frac{x+25}{x-39}$$

$$\lim\limits_{x\to\ 25}\frac{x^{2}-25^{2}}{(x-25)(x-39)}=[\frac{0}{0}]=\lim\limits_{x\to\ 25}\frac{x+25}{x-39}=2 \cdot \frac{25}{25-39} = \frac{50}{-14}$$
\rozwStop
\odpStart
$\frac{50}{-14}$
\odpStop
\testStart
A.$\frac{50}{-14}$
B.$\infty$
C.$-\infty$
D.$0$
E.$\frac{50}{14}$
F.$\frac{25}{39}$
G.$-\frac{50}{14}$
H.$1$
I.$25$
\testStop
\kluczStart
A
\kluczStop



\zadStart{Przykład z Wikieł P 4.2b moja wersja nr 473}
Obliczyć granicę $\lim\limits_{x\to\ 26}\frac{x^{2}-26^{2}}{(x-26)(x-3)}$.
\zadStop
\rozwStart{Patryk Wirkus}{Martyna Czarnobaj}
$$\frac{x^{2}-26^{2}}{(x-26)(x-3)}=\frac{x+26}{x-3}$$

$$\lim\limits_{x\to\ 26}\frac{x^{2}-26^{2}}{(x-26)(x-3)}=[\frac{0}{0}]=\lim\limits_{x\to\ 26}\frac{x+26}{x-3}=2 \cdot \frac{26}{26-3} = \frac{52}{23}$$
\rozwStop
\odpStart
$\frac{52}{23}$
\odpStop
\testStart
A.$\frac{52}{23}$
B.$\infty$
C.$-\infty$
D.$0$
E.$\frac{52}{-23}$
F.$\frac{26}{3}$
G.$-\frac{52}{-23}$
H.$1$
I.$26$
\testStop
\kluczStart
A
\kluczStop



\zadStart{Przykład z Wikieł P 4.2b moja wersja nr 474}
Obliczyć granicę $\lim\limits_{x\to\ 26}\frac{x^{2}-26^{2}}{(x-26)(x-5)}$.
\zadStop
\rozwStart{Patryk Wirkus}{Martyna Czarnobaj}
$$\frac{x^{2}-26^{2}}{(x-26)(x-5)}=\frac{x+26}{x-5}$$

$$\lim\limits_{x\to\ 26}\frac{x^{2}-26^{2}}{(x-26)(x-5)}=[\frac{0}{0}]=\lim\limits_{x\to\ 26}\frac{x+26}{x-5}=2 \cdot \frac{26}{26-5} = \frac{52}{21}$$
\rozwStop
\odpStart
$\frac{52}{21}$
\odpStop
\testStart
A.$\frac{52}{21}$
B.$\infty$
C.$-\infty$
D.$0$
E.$\frac{52}{-21}$
F.$\frac{26}{5}$
G.$-\frac{52}{-21}$
H.$1$
I.$26$
\testStop
\kluczStart
A
\kluczStop



\zadStart{Przykład z Wikieł P 4.2b moja wersja nr 475}
Obliczyć granicę $\lim\limits_{x\to\ 26}\frac{x^{2}-26^{2}}{(x-26)(x-7)}$.
\zadStop
\rozwStart{Patryk Wirkus}{Martyna Czarnobaj}
$$\frac{x^{2}-26^{2}}{(x-26)(x-7)}=\frac{x+26}{x-7}$$

$$\lim\limits_{x\to\ 26}\frac{x^{2}-26^{2}}{(x-26)(x-7)}=[\frac{0}{0}]=\lim\limits_{x\to\ 26}\frac{x+26}{x-7}=2 \cdot \frac{26}{26-7} = \frac{52}{19}$$
\rozwStop
\odpStart
$\frac{52}{19}$
\odpStop
\testStart
A.$\frac{52}{19}$
B.$\infty$
C.$-\infty$
D.$0$
E.$\frac{52}{-19}$
F.$\frac{26}{7}$
G.$-\frac{52}{-19}$
H.$1$
I.$26$
\testStop
\kluczStart
A
\kluczStop



\zadStart{Przykład z Wikieł P 4.2b moja wersja nr 476}
Obliczyć granicę $\lim\limits_{x\to\ 26}\frac{x^{2}-26^{2}}{(x-26)(x-9)}$.
\zadStop
\rozwStart{Patryk Wirkus}{Martyna Czarnobaj}
$$\frac{x^{2}-26^{2}}{(x-26)(x-9)}=\frac{x+26}{x-9}$$

$$\lim\limits_{x\to\ 26}\frac{x^{2}-26^{2}}{(x-26)(x-9)}=[\frac{0}{0}]=\lim\limits_{x\to\ 26}\frac{x+26}{x-9}=2 \cdot \frac{26}{26-9} = \frac{52}{17}$$
\rozwStop
\odpStart
$\frac{52}{17}$
\odpStop
\testStart
A.$\frac{52}{17}$
B.$\infty$
C.$-\infty$
D.$0$
E.$\frac{52}{-17}$
F.$\frac{26}{9}$
G.$-\frac{52}{-17}$
H.$1$
I.$26$
\testStop
\kluczStart
A
\kluczStop



\zadStart{Przykład z Wikieł P 4.2b moja wersja nr 477}
Obliczyć granicę $\lim\limits_{x\to\ 26}\frac{x^{2}-26^{2}}{(x-26)(x-11)}$.
\zadStop
\rozwStart{Patryk Wirkus}{Martyna Czarnobaj}
$$\frac{x^{2}-26^{2}}{(x-26)(x-11)}=\frac{x+26}{x-11}$$

$$\lim\limits_{x\to\ 26}\frac{x^{2}-26^{2}}{(x-26)(x-11)}=[\frac{0}{0}]=\lim\limits_{x\to\ 26}\frac{x+26}{x-11}=2 \cdot \frac{26}{26-11} = \frac{52}{15}$$
\rozwStop
\odpStart
$\frac{52}{15}$
\odpStop
\testStart
A.$\frac{52}{15}$
B.$\infty$
C.$-\infty$
D.$0$
E.$\frac{52}{-15}$
F.$\frac{26}{11}$
G.$-\frac{52}{-15}$
H.$1$
I.$26$
\testStop
\kluczStart
A
\kluczStop



\zadStart{Przykład z Wikieł P 4.2b moja wersja nr 478}
Obliczyć granicę $\lim\limits_{x\to\ 26}\frac{x^{2}-26^{2}}{(x-26)(x-15)}$.
\zadStop
\rozwStart{Patryk Wirkus}{Martyna Czarnobaj}
$$\frac{x^{2}-26^{2}}{(x-26)(x-15)}=\frac{x+26}{x-15}$$

$$\lim\limits_{x\to\ 26}\frac{x^{2}-26^{2}}{(x-26)(x-15)}=[\frac{0}{0}]=\lim\limits_{x\to\ 26}\frac{x+26}{x-15}=2 \cdot \frac{26}{26-15} = \frac{52}{11}$$
\rozwStop
\odpStart
$\frac{52}{11}$
\odpStop
\testStart
A.$\frac{52}{11}$
B.$\infty$
C.$-\infty$
D.$0$
E.$\frac{52}{-11}$
F.$\frac{26}{15}$
G.$-\frac{52}{-11}$
H.$1$
I.$26$
\testStop
\kluczStart
A
\kluczStop



\zadStart{Przykład z Wikieł P 4.2b moja wersja nr 479}
Obliczyć granicę $\lim\limits_{x\to\ 26}\frac{x^{2}-26^{2}}{(x-26)(x-17)}$.
\zadStop
\rozwStart{Patryk Wirkus}{Martyna Czarnobaj}
$$\frac{x^{2}-26^{2}}{(x-26)(x-17)}=\frac{x+26}{x-17}$$

$$\lim\limits_{x\to\ 26}\frac{x^{2}-26^{2}}{(x-26)(x-17)}=[\frac{0}{0}]=\lim\limits_{x\to\ 26}\frac{x+26}{x-17}=2 \cdot \frac{26}{26-17} = \frac{52}{9}$$
\rozwStop
\odpStart
$\frac{52}{9}$
\odpStop
\testStart
A.$\frac{52}{9}$
B.$\infty$
C.$-\infty$
D.$0$
E.$\frac{52}{-9}$
F.$\frac{26}{17}$
G.$-\frac{52}{-9}$
H.$1$
I.$26$
\testStop
\kluczStart
A
\kluczStop



\zadStart{Przykład z Wikieł P 4.2b moja wersja nr 480}
Obliczyć granicę $\lim\limits_{x\to\ 26}\frac{x^{2}-26^{2}}{(x-26)(x-19)}$.
\zadStop
\rozwStart{Patryk Wirkus}{Martyna Czarnobaj}
$$\frac{x^{2}-26^{2}}{(x-26)(x-19)}=\frac{x+26}{x-19}$$

$$\lim\limits_{x\to\ 26}\frac{x^{2}-26^{2}}{(x-26)(x-19)}=[\frac{0}{0}]=\lim\limits_{x\to\ 26}\frac{x+26}{x-19}=2 \cdot \frac{26}{26-19} = \frac{52}{7}$$
\rozwStop
\odpStart
$\frac{52}{7}$
\odpStop
\testStart
A.$\frac{52}{7}$
B.$\infty$
C.$-\infty$
D.$0$
E.$\frac{52}{-7}$
F.$\frac{26}{19}$
G.$-\frac{52}{-7}$
H.$1$
I.$26$
\testStop
\kluczStart
A
\kluczStop



\zadStart{Przykład z Wikieł P 4.2b moja wersja nr 481}
Obliczyć granicę $\lim\limits_{x\to\ 26}\frac{x^{2}-26^{2}}{(x-26)(x-21)}$.
\zadStop
\rozwStart{Patryk Wirkus}{Martyna Czarnobaj}
$$\frac{x^{2}-26^{2}}{(x-26)(x-21)}=\frac{x+26}{x-21}$$

$$\lim\limits_{x\to\ 26}\frac{x^{2}-26^{2}}{(x-26)(x-21)}=[\frac{0}{0}]=\lim\limits_{x\to\ 26}\frac{x+26}{x-21}=2 \cdot \frac{26}{26-21} = \frac{52}{5}$$
\rozwStop
\odpStart
$\frac{52}{5}$
\odpStop
\testStart
A.$\frac{52}{5}$
B.$\infty$
C.$-\infty$
D.$0$
E.$\frac{52}{-5}$
F.$\frac{26}{21}$
G.$-\frac{52}{-5}$
H.$1$
I.$26$
\testStop
\kluczStart
A
\kluczStop



\zadStart{Przykład z Wikieł P 4.2b moja wersja nr 482}
Obliczyć granicę $\lim\limits_{x\to\ 26}\frac{x^{2}-26^{2}}{(x-26)(x-23)}$.
\zadStop
\rozwStart{Patryk Wirkus}{Martyna Czarnobaj}
$$\frac{x^{2}-26^{2}}{(x-26)(x-23)}=\frac{x+26}{x-23}$$

$$\lim\limits_{x\to\ 26}\frac{x^{2}-26^{2}}{(x-26)(x-23)}=[\frac{0}{0}]=\lim\limits_{x\to\ 26}\frac{x+26}{x-23}=2 \cdot \frac{26}{26-23} = \frac{52}{3}$$
\rozwStop
\odpStart
$\frac{52}{3}$
\odpStop
\testStart
A.$\frac{52}{3}$
B.$\infty$
C.$-\infty$
D.$0$
E.$\frac{52}{-3}$
F.$\frac{26}{23}$
G.$-\frac{52}{-3}$
H.$1$
I.$26$
\testStop
\kluczStart
A
\kluczStop



\zadStart{Przykład z Wikieł P 4.2b moja wersja nr 483}
Obliczyć granicę $\lim\limits_{x\to\ 26}\frac{x^{2}-26^{2}}{(x-26)(x-29)}$.
\zadStop
\rozwStart{Patryk Wirkus}{Martyna Czarnobaj}
$$\frac{x^{2}-26^{2}}{(x-26)(x-29)}=\frac{x+26}{x-29}$$

$$\lim\limits_{x\to\ 26}\frac{x^{2}-26^{2}}{(x-26)(x-29)}=[\frac{0}{0}]=\lim\limits_{x\to\ 26}\frac{x+26}{x-29}=2 \cdot \frac{26}{26-29} = \frac{52}{-3}$$
\rozwStop
\odpStart
$\frac{52}{-3}$
\odpStop
\testStart
A.$\frac{52}{-3}$
B.$\infty$
C.$-\infty$
D.$0$
E.$\frac{52}{3}$
F.$\frac{26}{29}$
G.$-\frac{52}{3}$
H.$1$
I.$26$
\testStop
\kluczStart
A
\kluczStop



\zadStart{Przykład z Wikieł P 4.2b moja wersja nr 484}
Obliczyć granicę $\lim\limits_{x\to\ 26}\frac{x^{2}-26^{2}}{(x-26)(x-31)}$.
\zadStop
\rozwStart{Patryk Wirkus}{Martyna Czarnobaj}
$$\frac{x^{2}-26^{2}}{(x-26)(x-31)}=\frac{x+26}{x-31}$$

$$\lim\limits_{x\to\ 26}\frac{x^{2}-26^{2}}{(x-26)(x-31)}=[\frac{0}{0}]=\lim\limits_{x\to\ 26}\frac{x+26}{x-31}=2 \cdot \frac{26}{26-31} = \frac{52}{-5}$$
\rozwStop
\odpStart
$\frac{52}{-5}$
\odpStop
\testStart
A.$\frac{52}{-5}$
B.$\infty$
C.$-\infty$
D.$0$
E.$\frac{52}{5}$
F.$\frac{26}{31}$
G.$-\frac{52}{5}$
H.$1$
I.$26$
\testStop
\kluczStart
A
\kluczStop



\zadStart{Przykład z Wikieł P 4.2b moja wersja nr 485}
Obliczyć granicę $\lim\limits_{x\to\ 26}\frac{x^{2}-26^{2}}{(x-26)(x-33)}$.
\zadStop
\rozwStart{Patryk Wirkus}{Martyna Czarnobaj}
$$\frac{x^{2}-26^{2}}{(x-26)(x-33)}=\frac{x+26}{x-33}$$

$$\lim\limits_{x\to\ 26}\frac{x^{2}-26^{2}}{(x-26)(x-33)}=[\frac{0}{0}]=\lim\limits_{x\to\ 26}\frac{x+26}{x-33}=2 \cdot \frac{26}{26-33} = \frac{52}{-7}$$
\rozwStop
\odpStart
$\frac{52}{-7}$
\odpStop
\testStart
A.$\frac{52}{-7}$
B.$\infty$
C.$-\infty$
D.$0$
E.$\frac{52}{7}$
F.$\frac{26}{33}$
G.$-\frac{52}{7}$
H.$1$
I.$26$
\testStop
\kluczStart
A
\kluczStop



\zadStart{Przykład z Wikieł P 4.2b moja wersja nr 486}
Obliczyć granicę $\lim\limits_{x\to\ 26}\frac{x^{2}-26^{2}}{(x-26)(x-35)}$.
\zadStop
\rozwStart{Patryk Wirkus}{Martyna Czarnobaj}
$$\frac{x^{2}-26^{2}}{(x-26)(x-35)}=\frac{x+26}{x-35}$$

$$\lim\limits_{x\to\ 26}\frac{x^{2}-26^{2}}{(x-26)(x-35)}=[\frac{0}{0}]=\lim\limits_{x\to\ 26}\frac{x+26}{x-35}=2 \cdot \frac{26}{26-35} = \frac{52}{-9}$$
\rozwStop
\odpStart
$\frac{52}{-9}$
\odpStop
\testStart
A.$\frac{52}{-9}$
B.$\infty$
C.$-\infty$
D.$0$
E.$\frac{52}{9}$
F.$\frac{26}{35}$
G.$-\frac{52}{9}$
H.$1$
I.$26$
\testStop
\kluczStart
A
\kluczStop



\zadStart{Przykład z Wikieł P 4.2b moja wersja nr 487}
Obliczyć granicę $\lim\limits_{x\to\ 26}\frac{x^{2}-26^{2}}{(x-26)(x-37)}$.
\zadStop
\rozwStart{Patryk Wirkus}{Martyna Czarnobaj}
$$\frac{x^{2}-26^{2}}{(x-26)(x-37)}=\frac{x+26}{x-37}$$

$$\lim\limits_{x\to\ 26}\frac{x^{2}-26^{2}}{(x-26)(x-37)}=[\frac{0}{0}]=\lim\limits_{x\to\ 26}\frac{x+26}{x-37}=2 \cdot \frac{26}{26-37} = \frac{52}{-11}$$
\rozwStop
\odpStart
$\frac{52}{-11}$
\odpStop
\testStart
A.$\frac{52}{-11}$
B.$\infty$
C.$-\infty$
D.$0$
E.$\frac{52}{11}$
F.$\frac{26}{37}$
G.$-\frac{52}{11}$
H.$1$
I.$26$
\testStop
\kluczStart
A
\kluczStop



\zadStart{Przykład z Wikieł P 4.2b moja wersja nr 488}
Obliczyć granicę $\lim\limits_{x\to\ 27}\frac{x^{2}-27^{2}}{(x-27)(x-2)}$.
\zadStop
\rozwStart{Patryk Wirkus}{Martyna Czarnobaj}
$$\frac{x^{2}-27^{2}}{(x-27)(x-2)}=\frac{x+27}{x-2}$$

$$\lim\limits_{x\to\ 27}\frac{x^{2}-27^{2}}{(x-27)(x-2)}=[\frac{0}{0}]=\lim\limits_{x\to\ 27}\frac{x+27}{x-2}=2 \cdot \frac{27}{27-2} = \frac{54}{25}$$
\rozwStop
\odpStart
$\frac{54}{25}$
\odpStop
\testStart
A.$\frac{54}{25}$
B.$\infty$
C.$-\infty$
D.$0$
E.$\frac{54}{-25}$
F.$\frac{27}{2}$
G.$-\frac{54}{-25}$
H.$1$
I.$27$
\testStop
\kluczStart
A
\kluczStop



\zadStart{Przykład z Wikieł P 4.2b moja wersja nr 489}
Obliczyć granicę $\lim\limits_{x\to\ 27}\frac{x^{2}-27^{2}}{(x-27)(x-4)}$.
\zadStop
\rozwStart{Patryk Wirkus}{Martyna Czarnobaj}
$$\frac{x^{2}-27^{2}}{(x-27)(x-4)}=\frac{x+27}{x-4}$$

$$\lim\limits_{x\to\ 27}\frac{x^{2}-27^{2}}{(x-27)(x-4)}=[\frac{0}{0}]=\lim\limits_{x\to\ 27}\frac{x+27}{x-4}=2 \cdot \frac{27}{27-4} = \frac{54}{23}$$
\rozwStop
\odpStart
$\frac{54}{23}$
\odpStop
\testStart
A.$\frac{54}{23}$
B.$\infty$
C.$-\infty$
D.$0$
E.$\frac{54}{-23}$
F.$\frac{27}{4}$
G.$-\frac{54}{-23}$
H.$1$
I.$27$
\testStop
\kluczStart
A
\kluczStop



\zadStart{Przykład z Wikieł P 4.2b moja wersja nr 490}
Obliczyć granicę $\lim\limits_{x\to\ 27}\frac{x^{2}-27^{2}}{(x-27)(x-5)}$.
\zadStop
\rozwStart{Patryk Wirkus}{Martyna Czarnobaj}
$$\frac{x^{2}-27^{2}}{(x-27)(x-5)}=\frac{x+27}{x-5}$$

$$\lim\limits_{x\to\ 27}\frac{x^{2}-27^{2}}{(x-27)(x-5)}=[\frac{0}{0}]=\lim\limits_{x\to\ 27}\frac{x+27}{x-5}=2 \cdot \frac{27}{27-5} = \frac{54}{22}$$
\rozwStop
\odpStart
$\frac{54}{22}$
\odpStop
\testStart
A.$\frac{54}{22}$
B.$\infty$
C.$-\infty$
D.$0$
E.$\frac{54}{-22}$
F.$\frac{27}{5}$
G.$-\frac{54}{-22}$
H.$1$
I.$27$
\testStop
\kluczStart
A
\kluczStop



\zadStart{Przykład z Wikieł P 4.2b moja wersja nr 491}
Obliczyć granicę $\lim\limits_{x\to\ 27}\frac{x^{2}-27^{2}}{(x-27)(x-7)}$.
\zadStop
\rozwStart{Patryk Wirkus}{Martyna Czarnobaj}
$$\frac{x^{2}-27^{2}}{(x-27)(x-7)}=\frac{x+27}{x-7}$$

$$\lim\limits_{x\to\ 27}\frac{x^{2}-27^{2}}{(x-27)(x-7)}=[\frac{0}{0}]=\lim\limits_{x\to\ 27}\frac{x+27}{x-7}=2 \cdot \frac{27}{27-7} = \frac{54}{20}$$
\rozwStop
\odpStart
$\frac{54}{20}$
\odpStop
\testStart
A.$\frac{54}{20}$
B.$\infty$
C.$-\infty$
D.$0$
E.$\frac{54}{-20}$
F.$\frac{27}{7}$
G.$-\frac{54}{-20}$
H.$1$
I.$27$
\testStop
\kluczStart
A
\kluczStop



\zadStart{Przykład z Wikieł P 4.2b moja wersja nr 492}
Obliczyć granicę $\lim\limits_{x\to\ 27}\frac{x^{2}-27^{2}}{(x-27)(x-8)}$.
\zadStop
\rozwStart{Patryk Wirkus}{Martyna Czarnobaj}
$$\frac{x^{2}-27^{2}}{(x-27)(x-8)}=\frac{x+27}{x-8}$$

$$\lim\limits_{x\to\ 27}\frac{x^{2}-27^{2}}{(x-27)(x-8)}=[\frac{0}{0}]=\lim\limits_{x\to\ 27}\frac{x+27}{x-8}=2 \cdot \frac{27}{27-8} = \frac{54}{19}$$
\rozwStop
\odpStart
$\frac{54}{19}$
\odpStop
\testStart
A.$\frac{54}{19}$
B.$\infty$
C.$-\infty$
D.$0$
E.$\frac{54}{-19}$
F.$\frac{27}{8}$
G.$-\frac{54}{-19}$
H.$1$
I.$27$
\testStop
\kluczStart
A
\kluczStop



\zadStart{Przykład z Wikieł P 4.2b moja wersja nr 493}
Obliczyć granicę $\lim\limits_{x\to\ 27}\frac{x^{2}-27^{2}}{(x-27)(x-10)}$.
\zadStop
\rozwStart{Patryk Wirkus}{Martyna Czarnobaj}
$$\frac{x^{2}-27^{2}}{(x-27)(x-10)}=\frac{x+27}{x-10}$$

$$\lim\limits_{x\to\ 27}\frac{x^{2}-27^{2}}{(x-27)(x-10)}=[\frac{0}{0}]=\lim\limits_{x\to\ 27}\frac{x+27}{x-10}=2 \cdot \frac{27}{27-10} = \frac{54}{17}$$
\rozwStop
\odpStart
$\frac{54}{17}$
\odpStop
\testStart
A.$\frac{54}{17}$
B.$\infty$
C.$-\infty$
D.$0$
E.$\frac{54}{-17}$
F.$\frac{27}{10}$
G.$-\frac{54}{-17}$
H.$1$
I.$27$
\testStop
\kluczStart
A
\kluczStop



\zadStart{Przykład z Wikieł P 4.2b moja wersja nr 494}
Obliczyć granicę $\lim\limits_{x\to\ 27}\frac{x^{2}-27^{2}}{(x-27)(x-11)}$.
\zadStop
\rozwStart{Patryk Wirkus}{Martyna Czarnobaj}
$$\frac{x^{2}-27^{2}}{(x-27)(x-11)}=\frac{x+27}{x-11}$$

$$\lim\limits_{x\to\ 27}\frac{x^{2}-27^{2}}{(x-27)(x-11)}=[\frac{0}{0}]=\lim\limits_{x\to\ 27}\frac{x+27}{x-11}=2 \cdot \frac{27}{27-11} = \frac{54}{16}$$
\rozwStop
\odpStart
$\frac{54}{16}$
\odpStop
\testStart
A.$\frac{54}{16}$
B.$\infty$
C.$-\infty$
D.$0$
E.$\frac{54}{-16}$
F.$\frac{27}{11}$
G.$-\frac{54}{-16}$
H.$1$
I.$27$
\testStop
\kluczStart
A
\kluczStop



\zadStart{Przykład z Wikieł P 4.2b moja wersja nr 495}
Obliczyć granicę $\lim\limits_{x\to\ 27}\frac{x^{2}-27^{2}}{(x-27)(x-13)}$.
\zadStop
\rozwStart{Patryk Wirkus}{Martyna Czarnobaj}
$$\frac{x^{2}-27^{2}}{(x-27)(x-13)}=\frac{x+27}{x-13}$$

$$\lim\limits_{x\to\ 27}\frac{x^{2}-27^{2}}{(x-27)(x-13)}=[\frac{0}{0}]=\lim\limits_{x\to\ 27}\frac{x+27}{x-13}=2 \cdot \frac{27}{27-13} = \frac{54}{14}$$
\rozwStop
\odpStart
$\frac{54}{14}$
\odpStop
\testStart
A.$\frac{54}{14}$
B.$\infty$
C.$-\infty$
D.$0$
E.$\frac{54}{-14}$
F.$\frac{27}{13}$
G.$-\frac{54}{-14}$
H.$1$
I.$27$
\testStop
\kluczStart
A
\kluczStop



\zadStart{Przykład z Wikieł P 4.2b moja wersja nr 496}
Obliczyć granicę $\lim\limits_{x\to\ 27}\frac{x^{2}-27^{2}}{(x-27)(x-14)}$.
\zadStop
\rozwStart{Patryk Wirkus}{Martyna Czarnobaj}
$$\frac{x^{2}-27^{2}}{(x-27)(x-14)}=\frac{x+27}{x-14}$$

$$\lim\limits_{x\to\ 27}\frac{x^{2}-27^{2}}{(x-27)(x-14)}=[\frac{0}{0}]=\lim\limits_{x\to\ 27}\frac{x+27}{x-14}=2 \cdot \frac{27}{27-14} = \frac{54}{13}$$
\rozwStop
\odpStart
$\frac{54}{13}$
\odpStop
\testStart
A.$\frac{54}{13}$
B.$\infty$
C.$-\infty$
D.$0$
E.$\frac{54}{-13}$
F.$\frac{27}{14}$
G.$-\frac{54}{-13}$
H.$1$
I.$27$
\testStop
\kluczStart
A
\kluczStop



\zadStart{Przykład z Wikieł P 4.2b moja wersja nr 497}
Obliczyć granicę $\lim\limits_{x\to\ 27}\frac{x^{2}-27^{2}}{(x-27)(x-16)}$.
\zadStop
\rozwStart{Patryk Wirkus}{Martyna Czarnobaj}
$$\frac{x^{2}-27^{2}}{(x-27)(x-16)}=\frac{x+27}{x-16}$$

$$\lim\limits_{x\to\ 27}\frac{x^{2}-27^{2}}{(x-27)(x-16)}=[\frac{0}{0}]=\lim\limits_{x\to\ 27}\frac{x+27}{x-16}=2 \cdot \frac{27}{27-16} = \frac{54}{11}$$
\rozwStop
\odpStart
$\frac{54}{11}$
\odpStop
\testStart
A.$\frac{54}{11}$
B.$\infty$
C.$-\infty$
D.$0$
E.$\frac{54}{-11}$
F.$\frac{27}{16}$
G.$-\frac{54}{-11}$
H.$1$
I.$27$
\testStop
\kluczStart
A
\kluczStop



\zadStart{Przykład z Wikieł P 4.2b moja wersja nr 498}
Obliczyć granicę $\lim\limits_{x\to\ 27}\frac{x^{2}-27^{2}}{(x-27)(x-17)}$.
\zadStop
\rozwStart{Patryk Wirkus}{Martyna Czarnobaj}
$$\frac{x^{2}-27^{2}}{(x-27)(x-17)}=\frac{x+27}{x-17}$$

$$\lim\limits_{x\to\ 27}\frac{x^{2}-27^{2}}{(x-27)(x-17)}=[\frac{0}{0}]=\lim\limits_{x\to\ 27}\frac{x+27}{x-17}=2 \cdot \frac{27}{27-17} = \frac{54}{10}$$
\rozwStop
\odpStart
$\frac{54}{10}$
\odpStop
\testStart
A.$\frac{54}{10}$
B.$\infty$
C.$-\infty$
D.$0$
E.$\frac{54}{-10}$
F.$\frac{27}{17}$
G.$-\frac{54}{-10}$
H.$1$
I.$27$
\testStop
\kluczStart
A
\kluczStop



\zadStart{Przykład z Wikieł P 4.2b moja wersja nr 499}
Obliczyć granicę $\lim\limits_{x\to\ 27}\frac{x^{2}-27^{2}}{(x-27)(x-19)}$.
\zadStop
\rozwStart{Patryk Wirkus}{Martyna Czarnobaj}
$$\frac{x^{2}-27^{2}}{(x-27)(x-19)}=\frac{x+27}{x-19}$$

$$\lim\limits_{x\to\ 27}\frac{x^{2}-27^{2}}{(x-27)(x-19)}=[\frac{0}{0}]=\lim\limits_{x\to\ 27}\frac{x+27}{x-19}=2 \cdot \frac{27}{27-19} = \frac{54}{8}$$
\rozwStop
\odpStart
$\frac{54}{8}$
\odpStop
\testStart
A.$\frac{54}{8}$
B.$\infty$
C.$-\infty$
D.$0$
E.$\frac{54}{-8}$
F.$\frac{27}{19}$
G.$-\frac{54}{-8}$
H.$1$
I.$27$
\testStop
\kluczStart
A
\kluczStop



\zadStart{Przykład z Wikieł P 4.2b moja wersja nr 500}
Obliczyć granicę $\lim\limits_{x\to\ 27}\frac{x^{2}-27^{2}}{(x-27)(x-20)}$.
\zadStop
\rozwStart{Patryk Wirkus}{Martyna Czarnobaj}
$$\frac{x^{2}-27^{2}}{(x-27)(x-20)}=\frac{x+27}{x-20}$$

$$\lim\limits_{x\to\ 27}\frac{x^{2}-27^{2}}{(x-27)(x-20)}=[\frac{0}{0}]=\lim\limits_{x\to\ 27}\frac{x+27}{x-20}=2 \cdot \frac{27}{27-20} = \frac{54}{7}$$
\rozwStop
\odpStart
$\frac{54}{7}$
\odpStop
\testStart
A.$\frac{54}{7}$
B.$\infty$
C.$-\infty$
D.$0$
E.$\frac{54}{-7}$
F.$\frac{27}{20}$
G.$-\frac{54}{-7}$
H.$1$
I.$27$
\testStop
\kluczStart
A
\kluczStop



\zadStart{Przykład z Wikieł P 4.2b moja wersja nr 501}
Obliczyć granicę $\lim\limits_{x\to\ 27}\frac{x^{2}-27^{2}}{(x-27)(x-22)}$.
\zadStop
\rozwStart{Patryk Wirkus}{Martyna Czarnobaj}
$$\frac{x^{2}-27^{2}}{(x-27)(x-22)}=\frac{x+27}{x-22}$$

$$\lim\limits_{x\to\ 27}\frac{x^{2}-27^{2}}{(x-27)(x-22)}=[\frac{0}{0}]=\lim\limits_{x\to\ 27}\frac{x+27}{x-22}=2 \cdot \frac{27}{27-22} = \frac{54}{5}$$
\rozwStop
\odpStart
$\frac{54}{5}$
\odpStop
\testStart
A.$\frac{54}{5}$
B.$\infty$
C.$-\infty$
D.$0$
E.$\frac{54}{-5}$
F.$\frac{27}{22}$
G.$-\frac{54}{-5}$
H.$1$
I.$27$
\testStop
\kluczStart
A
\kluczStop



\zadStart{Przykład z Wikieł P 4.2b moja wersja nr 502}
Obliczyć granicę $\lim\limits_{x\to\ 27}\frac{x^{2}-27^{2}}{(x-27)(x-23)}$.
\zadStop
\rozwStart{Patryk Wirkus}{Martyna Czarnobaj}
$$\frac{x^{2}-27^{2}}{(x-27)(x-23)}=\frac{x+27}{x-23}$$

$$\lim\limits_{x\to\ 27}\frac{x^{2}-27^{2}}{(x-27)(x-23)}=[\frac{0}{0}]=\lim\limits_{x\to\ 27}\frac{x+27}{x-23}=2 \cdot \frac{27}{27-23} = \frac{54}{4}$$
\rozwStop
\odpStart
$\frac{54}{4}$
\odpStop
\testStart
A.$\frac{54}{4}$
B.$\infty$
C.$-\infty$
D.$0$
E.$\frac{54}{-4}$
F.$\frac{27}{23}$
G.$-\frac{54}{-4}$
H.$1$
I.$27$
\testStop
\kluczStart
A
\kluczStop



\zadStart{Przykład z Wikieł P 4.2b moja wersja nr 503}
Obliczyć granicę $\lim\limits_{x\to\ 27}\frac{x^{2}-27^{2}}{(x-27)(x-25)}$.
\zadStop
\rozwStart{Patryk Wirkus}{Martyna Czarnobaj}
$$\frac{x^{2}-27^{2}}{(x-27)(x-25)}=\frac{x+27}{x-25}$$

$$\lim\limits_{x\to\ 27}\frac{x^{2}-27^{2}}{(x-27)(x-25)}=[\frac{0}{0}]=\lim\limits_{x\to\ 27}\frac{x+27}{x-25}=2 \cdot \frac{27}{27-25} = \frac{54}{2}$$
\rozwStop
\odpStart
$\frac{54}{2}$
\odpStop
\testStart
A.$\frac{54}{2}$
B.$\infty$
C.$-\infty$
D.$0$
E.$\frac{54}{-2}$
F.$\frac{27}{25}$
G.$-\frac{54}{-2}$
H.$1$
I.$27$
\testStop
\kluczStart
A
\kluczStop



\zadStart{Przykład z Wikieł P 4.2b moja wersja nr 504}
Obliczyć granicę $\lim\limits_{x\to\ 27}\frac{x^{2}-27^{2}}{(x-27)(x-29)}$.
\zadStop
\rozwStart{Patryk Wirkus}{Martyna Czarnobaj}
$$\frac{x^{2}-27^{2}}{(x-27)(x-29)}=\frac{x+27}{x-29}$$

$$\lim\limits_{x\to\ 27}\frac{x^{2}-27^{2}}{(x-27)(x-29)}=[\frac{0}{0}]=\lim\limits_{x\to\ 27}\frac{x+27}{x-29}=2 \cdot \frac{27}{27-29} = \frac{54}{-2}$$
\rozwStop
\odpStart
$\frac{54}{-2}$
\odpStop
\testStart
A.$\frac{54}{-2}$
B.$\infty$
C.$-\infty$
D.$0$
E.$\frac{54}{2}$
F.$\frac{27}{29}$
G.$-\frac{54}{2}$
H.$1$
I.$27$
\testStop
\kluczStart
A
\kluczStop



\zadStart{Przykład z Wikieł P 4.2b moja wersja nr 505}
Obliczyć granicę $\lim\limits_{x\to\ 27}\frac{x^{2}-27^{2}}{(x-27)(x-31)}$.
\zadStop
\rozwStart{Patryk Wirkus}{Martyna Czarnobaj}
$$\frac{x^{2}-27^{2}}{(x-27)(x-31)}=\frac{x+27}{x-31}$$

$$\lim\limits_{x\to\ 27}\frac{x^{2}-27^{2}}{(x-27)(x-31)}=[\frac{0}{0}]=\lim\limits_{x\to\ 27}\frac{x+27}{x-31}=2 \cdot \frac{27}{27-31} = \frac{54}{-4}$$
\rozwStop
\odpStart
$\frac{54}{-4}$
\odpStop
\testStart
A.$\frac{54}{-4}$
B.$\infty$
C.$-\infty$
D.$0$
E.$\frac{54}{4}$
F.$\frac{27}{31}$
G.$-\frac{54}{4}$
H.$1$
I.$27$
\testStop
\kluczStart
A
\kluczStop



\zadStart{Przykład z Wikieł P 4.2b moja wersja nr 506}
Obliczyć granicę $\lim\limits_{x\to\ 27}\frac{x^{2}-27^{2}}{(x-27)(x-32)}$.
\zadStop
\rozwStart{Patryk Wirkus}{Martyna Czarnobaj}
$$\frac{x^{2}-27^{2}}{(x-27)(x-32)}=\frac{x+27}{x-32}$$

$$\lim\limits_{x\to\ 27}\frac{x^{2}-27^{2}}{(x-27)(x-32)}=[\frac{0}{0}]=\lim\limits_{x\to\ 27}\frac{x+27}{x-32}=2 \cdot \frac{27}{27-32} = \frac{54}{-5}$$
\rozwStop
\odpStart
$\frac{54}{-5}$
\odpStop
\testStart
A.$\frac{54}{-5}$
B.$\infty$
C.$-\infty$
D.$0$
E.$\frac{54}{5}$
F.$\frac{27}{32}$
G.$-\frac{54}{5}$
H.$1$
I.$27$
\testStop
\kluczStart
A
\kluczStop



\zadStart{Przykład z Wikieł P 4.2b moja wersja nr 507}
Obliczyć granicę $\lim\limits_{x\to\ 27}\frac{x^{2}-27^{2}}{(x-27)(x-34)}$.
\zadStop
\rozwStart{Patryk Wirkus}{Martyna Czarnobaj}
$$\frac{x^{2}-27^{2}}{(x-27)(x-34)}=\frac{x+27}{x-34}$$

$$\lim\limits_{x\to\ 27}\frac{x^{2}-27^{2}}{(x-27)(x-34)}=[\frac{0}{0}]=\lim\limits_{x\to\ 27}\frac{x+27}{x-34}=2 \cdot \frac{27}{27-34} = \frac{54}{-7}$$
\rozwStop
\odpStart
$\frac{54}{-7}$
\odpStop
\testStart
A.$\frac{54}{-7}$
B.$\infty$
C.$-\infty$
D.$0$
E.$\frac{54}{7}$
F.$\frac{27}{34}$
G.$-\frac{54}{7}$
H.$1$
I.$27$
\testStop
\kluczStart
A
\kluczStop



\zadStart{Przykład z Wikieł P 4.2b moja wersja nr 508}
Obliczyć granicę $\lim\limits_{x\to\ 27}\frac{x^{2}-27^{2}}{(x-27)(x-35)}$.
\zadStop
\rozwStart{Patryk Wirkus}{Martyna Czarnobaj}
$$\frac{x^{2}-27^{2}}{(x-27)(x-35)}=\frac{x+27}{x-35}$$

$$\lim\limits_{x\to\ 27}\frac{x^{2}-27^{2}}{(x-27)(x-35)}=[\frac{0}{0}]=\lim\limits_{x\to\ 27}\frac{x+27}{x-35}=2 \cdot \frac{27}{27-35} = \frac{54}{-8}$$
\rozwStop
\odpStart
$\frac{54}{-8}$
\odpStop
\testStart
A.$\frac{54}{-8}$
B.$\infty$
C.$-\infty$
D.$0$
E.$\frac{54}{8}$
F.$\frac{27}{35}$
G.$-\frac{54}{8}$
H.$1$
I.$27$
\testStop
\kluczStart
A
\kluczStop



\zadStart{Przykład z Wikieł P 4.2b moja wersja nr 509}
Obliczyć granicę $\lim\limits_{x\to\ 27}\frac{x^{2}-27^{2}}{(x-27)(x-37)}$.
\zadStop
\rozwStart{Patryk Wirkus}{Martyna Czarnobaj}
$$\frac{x^{2}-27^{2}}{(x-27)(x-37)}=\frac{x+27}{x-37}$$

$$\lim\limits_{x\to\ 27}\frac{x^{2}-27^{2}}{(x-27)(x-37)}=[\frac{0}{0}]=\lim\limits_{x\to\ 27}\frac{x+27}{x-37}=2 \cdot \frac{27}{27-37} = \frac{54}{-10}$$
\rozwStop
\odpStart
$\frac{54}{-10}$
\odpStop
\testStart
A.$\frac{54}{-10}$
B.$\infty$
C.$-\infty$
D.$0$
E.$\frac{54}{10}$
F.$\frac{27}{37}$
G.$-\frac{54}{10}$
H.$1$
I.$27$
\testStop
\kluczStart
A
\kluczStop



\zadStart{Przykład z Wikieł P 4.2b moja wersja nr 510}
Obliczyć granicę $\lim\limits_{x\to\ 27}\frac{x^{2}-27^{2}}{(x-27)(x-38)}$.
\zadStop
\rozwStart{Patryk Wirkus}{Martyna Czarnobaj}
$$\frac{x^{2}-27^{2}}{(x-27)(x-38)}=\frac{x+27}{x-38}$$

$$\lim\limits_{x\to\ 27}\frac{x^{2}-27^{2}}{(x-27)(x-38)}=[\frac{0}{0}]=\lim\limits_{x\to\ 27}\frac{x+27}{x-38}=2 \cdot \frac{27}{27-38} = \frac{54}{-11}$$
\rozwStop
\odpStart
$\frac{54}{-11}$
\odpStop
\testStart
A.$\frac{54}{-11}$
B.$\infty$
C.$-\infty$
D.$0$
E.$\frac{54}{11}$
F.$\frac{27}{38}$
G.$-\frac{54}{11}$
H.$1$
I.$27$
\testStop
\kluczStart
A
\kluczStop



\zadStart{Przykład z Wikieł P 4.2b moja wersja nr 511}
Obliczyć granicę $\lim\limits_{x\to\ 27}\frac{x^{2}-27^{2}}{(x-27)(x-40)}$.
\zadStop
\rozwStart{Patryk Wirkus}{Martyna Czarnobaj}
$$\frac{x^{2}-27^{2}}{(x-27)(x-40)}=\frac{x+27}{x-40}$$

$$\lim\limits_{x\to\ 27}\frac{x^{2}-27^{2}}{(x-27)(x-40)}=[\frac{0}{0}]=\lim\limits_{x\to\ 27}\frac{x+27}{x-40}=2 \cdot \frac{27}{27-40} = \frac{54}{-13}$$
\rozwStop
\odpStart
$\frac{54}{-13}$
\odpStop
\testStart
A.$\frac{54}{-13}$
B.$\infty$
C.$-\infty$
D.$0$
E.$\frac{54}{13}$
F.$\frac{27}{40}$
G.$-\frac{54}{13}$
H.$1$
I.$27$
\testStop
\kluczStart
A
\kluczStop



\zadStart{Przykład z Wikieł P 4.2b moja wersja nr 512}
Obliczyć granicę $\lim\limits_{x\to\ 28}\frac{x^{2}-28^{2}}{(x-28)(x-3)}$.
\zadStop
\rozwStart{Patryk Wirkus}{Martyna Czarnobaj}
$$\frac{x^{2}-28^{2}}{(x-28)(x-3)}=\frac{x+28}{x-3}$$

$$\lim\limits_{x\to\ 28}\frac{x^{2}-28^{2}}{(x-28)(x-3)}=[\frac{0}{0}]=\lim\limits_{x\to\ 28}\frac{x+28}{x-3}=2 \cdot \frac{28}{28-3} = \frac{56}{25}$$
\rozwStop
\odpStart
$\frac{56}{25}$
\odpStop
\testStart
A.$\frac{56}{25}$
B.$\infty$
C.$-\infty$
D.$0$
E.$\frac{56}{-25}$
F.$\frac{28}{3}$
G.$-\frac{56}{-25}$
H.$1$
I.$28$
\testStop
\kluczStart
A
\kluczStop



\zadStart{Przykład z Wikieł P 4.2b moja wersja nr 513}
Obliczyć granicę $\lim\limits_{x\to\ 28}\frac{x^{2}-28^{2}}{(x-28)(x-5)}$.
\zadStop
\rozwStart{Patryk Wirkus}{Martyna Czarnobaj}
$$\frac{x^{2}-28^{2}}{(x-28)(x-5)}=\frac{x+28}{x-5}$$

$$\lim\limits_{x\to\ 28}\frac{x^{2}-28^{2}}{(x-28)(x-5)}=[\frac{0}{0}]=\lim\limits_{x\to\ 28}\frac{x+28}{x-5}=2 \cdot \frac{28}{28-5} = \frac{56}{23}$$
\rozwStop
\odpStart
$\frac{56}{23}$
\odpStop
\testStart
A.$\frac{56}{23}$
B.$\infty$
C.$-\infty$
D.$0$
E.$\frac{56}{-23}$
F.$\frac{28}{5}$
G.$-\frac{56}{-23}$
H.$1$
I.$28$
\testStop
\kluczStart
A
\kluczStop



\zadStart{Przykład z Wikieł P 4.2b moja wersja nr 514}
Obliczyć granicę $\lim\limits_{x\to\ 28}\frac{x^{2}-28^{2}}{(x-28)(x-9)}$.
\zadStop
\rozwStart{Patryk Wirkus}{Martyna Czarnobaj}
$$\frac{x^{2}-28^{2}}{(x-28)(x-9)}=\frac{x+28}{x-9}$$

$$\lim\limits_{x\to\ 28}\frac{x^{2}-28^{2}}{(x-28)(x-9)}=[\frac{0}{0}]=\lim\limits_{x\to\ 28}\frac{x+28}{x-9}=2 \cdot \frac{28}{28-9} = \frac{56}{19}$$
\rozwStop
\odpStart
$\frac{56}{19}$
\odpStop
\testStart
A.$\frac{56}{19}$
B.$\infty$
C.$-\infty$
D.$0$
E.$\frac{56}{-19}$
F.$\frac{28}{9}$
G.$-\frac{56}{-19}$
H.$1$
I.$28$
\testStop
\kluczStart
A
\kluczStop



\zadStart{Przykład z Wikieł P 4.2b moja wersja nr 515}
Obliczyć granicę $\lim\limits_{x\to\ 28}\frac{x^{2}-28^{2}}{(x-28)(x-11)}$.
\zadStop
\rozwStart{Patryk Wirkus}{Martyna Czarnobaj}
$$\frac{x^{2}-28^{2}}{(x-28)(x-11)}=\frac{x+28}{x-11}$$

$$\lim\limits_{x\to\ 28}\frac{x^{2}-28^{2}}{(x-28)(x-11)}=[\frac{0}{0}]=\lim\limits_{x\to\ 28}\frac{x+28}{x-11}=2 \cdot \frac{28}{28-11} = \frac{56}{17}$$
\rozwStop
\odpStart
$\frac{56}{17}$
\odpStop
\testStart
A.$\frac{56}{17}$
B.$\infty$
C.$-\infty$
D.$0$
E.$\frac{56}{-17}$
F.$\frac{28}{11}$
G.$-\frac{56}{-17}$
H.$1$
I.$28$
\testStop
\kluczStart
A
\kluczStop



\zadStart{Przykład z Wikieł P 4.2b moja wersja nr 516}
Obliczyć granicę $\lim\limits_{x\to\ 28}\frac{x^{2}-28^{2}}{(x-28)(x-13)}$.
\zadStop
\rozwStart{Patryk Wirkus}{Martyna Czarnobaj}
$$\frac{x^{2}-28^{2}}{(x-28)(x-13)}=\frac{x+28}{x-13}$$

$$\lim\limits_{x\to\ 28}\frac{x^{2}-28^{2}}{(x-28)(x-13)}=[\frac{0}{0}]=\lim\limits_{x\to\ 28}\frac{x+28}{x-13}=2 \cdot \frac{28}{28-13} = \frac{56}{15}$$
\rozwStop
\odpStart
$\frac{56}{15}$
\odpStop
\testStart
A.$\frac{56}{15}$
B.$\infty$
C.$-\infty$
D.$0$
E.$\frac{56}{-15}$
F.$\frac{28}{13}$
G.$-\frac{56}{-15}$
H.$1$
I.$28$
\testStop
\kluczStart
A
\kluczStop



\zadStart{Przykład z Wikieł P 4.2b moja wersja nr 517}
Obliczyć granicę $\lim\limits_{x\to\ 28}\frac{x^{2}-28^{2}}{(x-28)(x-15)}$.
\zadStop
\rozwStart{Patryk Wirkus}{Martyna Czarnobaj}
$$\frac{x^{2}-28^{2}}{(x-28)(x-15)}=\frac{x+28}{x-15}$$

$$\lim\limits_{x\to\ 28}\frac{x^{2}-28^{2}}{(x-28)(x-15)}=[\frac{0}{0}]=\lim\limits_{x\to\ 28}\frac{x+28}{x-15}=2 \cdot \frac{28}{28-15} = \frac{56}{13}$$
\rozwStop
\odpStart
$\frac{56}{13}$
\odpStop
\testStart
A.$\frac{56}{13}$
B.$\infty$
C.$-\infty$
D.$0$
E.$\frac{56}{-13}$
F.$\frac{28}{15}$
G.$-\frac{56}{-13}$
H.$1$
I.$28$
\testStop
\kluczStart
A
\kluczStop



\zadStart{Przykład z Wikieł P 4.2b moja wersja nr 518}
Obliczyć granicę $\lim\limits_{x\to\ 28}\frac{x^{2}-28^{2}}{(x-28)(x-17)}$.
\zadStop
\rozwStart{Patryk Wirkus}{Martyna Czarnobaj}
$$\frac{x^{2}-28^{2}}{(x-28)(x-17)}=\frac{x+28}{x-17}$$

$$\lim\limits_{x\to\ 28}\frac{x^{2}-28^{2}}{(x-28)(x-17)}=[\frac{0}{0}]=\lim\limits_{x\to\ 28}\frac{x+28}{x-17}=2 \cdot \frac{28}{28-17} = \frac{56}{11}$$
\rozwStop
\odpStart
$\frac{56}{11}$
\odpStop
\testStart
A.$\frac{56}{11}$
B.$\infty$
C.$-\infty$
D.$0$
E.$\frac{56}{-11}$
F.$\frac{28}{17}$
G.$-\frac{56}{-11}$
H.$1$
I.$28$
\testStop
\kluczStart
A
\kluczStop



\zadStart{Przykład z Wikieł P 4.2b moja wersja nr 519}
Obliczyć granicę $\lim\limits_{x\to\ 28}\frac{x^{2}-28^{2}}{(x-28)(x-19)}$.
\zadStop
\rozwStart{Patryk Wirkus}{Martyna Czarnobaj}
$$\frac{x^{2}-28^{2}}{(x-28)(x-19)}=\frac{x+28}{x-19}$$

$$\lim\limits_{x\to\ 28}\frac{x^{2}-28^{2}}{(x-28)(x-19)}=[\frac{0}{0}]=\lim\limits_{x\to\ 28}\frac{x+28}{x-19}=2 \cdot \frac{28}{28-19} = \frac{56}{9}$$
\rozwStop
\odpStart
$\frac{56}{9}$
\odpStop
\testStart
A.$\frac{56}{9}$
B.$\infty$
C.$-\infty$
D.$0$
E.$\frac{56}{-9}$
F.$\frac{28}{19}$
G.$-\frac{56}{-9}$
H.$1$
I.$28$
\testStop
\kluczStart
A
\kluczStop



\zadStart{Przykład z Wikieł P 4.2b moja wersja nr 520}
Obliczyć granicę $\lim\limits_{x\to\ 28}\frac{x^{2}-28^{2}}{(x-28)(x-23)}$.
\zadStop
\rozwStart{Patryk Wirkus}{Martyna Czarnobaj}
$$\frac{x^{2}-28^{2}}{(x-28)(x-23)}=\frac{x+28}{x-23}$$

$$\lim\limits_{x\to\ 28}\frac{x^{2}-28^{2}}{(x-28)(x-23)}=[\frac{0}{0}]=\lim\limits_{x\to\ 28}\frac{x+28}{x-23}=2 \cdot \frac{28}{28-23} = \frac{56}{5}$$
\rozwStop
\odpStart
$\frac{56}{5}$
\odpStop
\testStart
A.$\frac{56}{5}$
B.$\infty$
C.$-\infty$
D.$0$
E.$\frac{56}{-5}$
F.$\frac{28}{23}$
G.$-\frac{56}{-5}$
H.$1$
I.$28$
\testStop
\kluczStart
A
\kluczStop



\zadStart{Przykład z Wikieł P 4.2b moja wersja nr 521}
Obliczyć granicę $\lim\limits_{x\to\ 28}\frac{x^{2}-28^{2}}{(x-28)(x-25)}$.
\zadStop
\rozwStart{Patryk Wirkus}{Martyna Czarnobaj}
$$\frac{x^{2}-28^{2}}{(x-28)(x-25)}=\frac{x+28}{x-25}$$

$$\lim\limits_{x\to\ 28}\frac{x^{2}-28^{2}}{(x-28)(x-25)}=[\frac{0}{0}]=\lim\limits_{x\to\ 28}\frac{x+28}{x-25}=2 \cdot \frac{28}{28-25} = \frac{56}{3}$$
\rozwStop
\odpStart
$\frac{56}{3}$
\odpStop
\testStart
A.$\frac{56}{3}$
B.$\infty$
C.$-\infty$
D.$0$
E.$\frac{56}{-3}$
F.$\frac{28}{25}$
G.$-\frac{56}{-3}$
H.$1$
I.$28$
\testStop
\kluczStart
A
\kluczStop



\zadStart{Przykład z Wikieł P 4.2b moja wersja nr 522}
Obliczyć granicę $\lim\limits_{x\to\ 28}\frac{x^{2}-28^{2}}{(x-28)(x-31)}$.
\zadStop
\rozwStart{Patryk Wirkus}{Martyna Czarnobaj}
$$\frac{x^{2}-28^{2}}{(x-28)(x-31)}=\frac{x+28}{x-31}$$

$$\lim\limits_{x\to\ 28}\frac{x^{2}-28^{2}}{(x-28)(x-31)}=[\frac{0}{0}]=\lim\limits_{x\to\ 28}\frac{x+28}{x-31}=2 \cdot \frac{28}{28-31} = \frac{56}{-3}$$
\rozwStop
\odpStart
$\frac{56}{-3}$
\odpStop
\testStart
A.$\frac{56}{-3}$
B.$\infty$
C.$-\infty$
D.$0$
E.$\frac{56}{3}$
F.$\frac{28}{31}$
G.$-\frac{56}{3}$
H.$1$
I.$28$
\testStop
\kluczStart
A
\kluczStop



\zadStart{Przykład z Wikieł P 4.2b moja wersja nr 523}
Obliczyć granicę $\lim\limits_{x\to\ 28}\frac{x^{2}-28^{2}}{(x-28)(x-33)}$.
\zadStop
\rozwStart{Patryk Wirkus}{Martyna Czarnobaj}
$$\frac{x^{2}-28^{2}}{(x-28)(x-33)}=\frac{x+28}{x-33}$$

$$\lim\limits_{x\to\ 28}\frac{x^{2}-28^{2}}{(x-28)(x-33)}=[\frac{0}{0}]=\lim\limits_{x\to\ 28}\frac{x+28}{x-33}=2 \cdot \frac{28}{28-33} = \frac{56}{-5}$$
\rozwStop
\odpStart
$\frac{56}{-5}$
\odpStop
\testStart
A.$\frac{56}{-5}$
B.$\infty$
C.$-\infty$
D.$0$
E.$\frac{56}{5}$
F.$\frac{28}{33}$
G.$-\frac{56}{5}$
H.$1$
I.$28$
\testStop
\kluczStart
A
\kluczStop



\zadStart{Przykład z Wikieł P 4.2b moja wersja nr 524}
Obliczyć granicę $\lim\limits_{x\to\ 28}\frac{x^{2}-28^{2}}{(x-28)(x-37)}$.
\zadStop
\rozwStart{Patryk Wirkus}{Martyna Czarnobaj}
$$\frac{x^{2}-28^{2}}{(x-28)(x-37)}=\frac{x+28}{x-37}$$

$$\lim\limits_{x\to\ 28}\frac{x^{2}-28^{2}}{(x-28)(x-37)}=[\frac{0}{0}]=\lim\limits_{x\to\ 28}\frac{x+28}{x-37}=2 \cdot \frac{28}{28-37} = \frac{56}{-9}$$
\rozwStop
\odpStart
$\frac{56}{-9}$
\odpStop
\testStart
A.$\frac{56}{-9}$
B.$\infty$
C.$-\infty$
D.$0$
E.$\frac{56}{9}$
F.$\frac{28}{37}$
G.$-\frac{56}{9}$
H.$1$
I.$28$
\testStop
\kluczStart
A
\kluczStop



\zadStart{Przykład z Wikieł P 4.2b moja wersja nr 525}
Obliczyć granicę $\lim\limits_{x\to\ 28}\frac{x^{2}-28^{2}}{(x-28)(x-39)}$.
\zadStop
\rozwStart{Patryk Wirkus}{Martyna Czarnobaj}
$$\frac{x^{2}-28^{2}}{(x-28)(x-39)}=\frac{x+28}{x-39}$$

$$\lim\limits_{x\to\ 28}\frac{x^{2}-28^{2}}{(x-28)(x-39)}=[\frac{0}{0}]=\lim\limits_{x\to\ 28}\frac{x+28}{x-39}=2 \cdot \frac{28}{28-39} = \frac{56}{-11}$$
\rozwStop
\odpStart
$\frac{56}{-11}$
\odpStop
\testStart
A.$\frac{56}{-11}$
B.$\infty$
C.$-\infty$
D.$0$
E.$\frac{56}{11}$
F.$\frac{28}{39}$
G.$-\frac{56}{11}$
H.$1$
I.$28$
\testStop
\kluczStart
A
\kluczStop



\zadStart{Przykład z Wikieł P 4.2b moja wersja nr 526}
Obliczyć granicę $\lim\limits_{x\to\ 29}\frac{x^{2}-29^{2}}{(x-29)(x-2)}$.
\zadStop
\rozwStart{Patryk Wirkus}{Martyna Czarnobaj}
$$\frac{x^{2}-29^{2}}{(x-29)(x-2)}=\frac{x+29}{x-2}$$

$$\lim\limits_{x\to\ 29}\frac{x^{2}-29^{2}}{(x-29)(x-2)}=[\frac{0}{0}]=\lim\limits_{x\to\ 29}\frac{x+29}{x-2}=2 \cdot \frac{29}{29-2} = \frac{58}{27}$$
\rozwStop
\odpStart
$\frac{58}{27}$
\odpStop
\testStart
A.$\frac{58}{27}$
B.$\infty$
C.$-\infty$
D.$0$
E.$\frac{58}{-27}$
F.$\frac{29}{2}$
G.$-\frac{58}{-27}$
H.$1$
I.$29$
\testStop
\kluczStart
A
\kluczStop



\zadStart{Przykład z Wikieł P 4.2b moja wersja nr 527}
Obliczyć granicę $\lim\limits_{x\to\ 29}\frac{x^{2}-29^{2}}{(x-29)(x-3)}$.
\zadStop
\rozwStart{Patryk Wirkus}{Martyna Czarnobaj}
$$\frac{x^{2}-29^{2}}{(x-29)(x-3)}=\frac{x+29}{x-3}$$

$$\lim\limits_{x\to\ 29}\frac{x^{2}-29^{2}}{(x-29)(x-3)}=[\frac{0}{0}]=\lim\limits_{x\to\ 29}\frac{x+29}{x-3}=2 \cdot \frac{29}{29-3} = \frac{58}{26}$$
\rozwStop
\odpStart
$\frac{58}{26}$
\odpStop
\testStart
A.$\frac{58}{26}$
B.$\infty$
C.$-\infty$
D.$0$
E.$\frac{58}{-26}$
F.$\frac{29}{3}$
G.$-\frac{58}{-26}$
H.$1$
I.$29$
\testStop
\kluczStart
A
\kluczStop



\zadStart{Przykład z Wikieł P 4.2b moja wersja nr 528}
Obliczyć granicę $\lim\limits_{x\to\ 29}\frac{x^{2}-29^{2}}{(x-29)(x-4)}$.
\zadStop
\rozwStart{Patryk Wirkus}{Martyna Czarnobaj}
$$\frac{x^{2}-29^{2}}{(x-29)(x-4)}=\frac{x+29}{x-4}$$

$$\lim\limits_{x\to\ 29}\frac{x^{2}-29^{2}}{(x-29)(x-4)}=[\frac{0}{0}]=\lim\limits_{x\to\ 29}\frac{x+29}{x-4}=2 \cdot \frac{29}{29-4} = \frac{58}{25}$$
\rozwStop
\odpStart
$\frac{58}{25}$
\odpStop
\testStart
A.$\frac{58}{25}$
B.$\infty$
C.$-\infty$
D.$0$
E.$\frac{58}{-25}$
F.$\frac{29}{4}$
G.$-\frac{58}{-25}$
H.$1$
I.$29$
\testStop
\kluczStart
A
\kluczStop



\zadStart{Przykład z Wikieł P 4.2b moja wersja nr 529}
Obliczyć granicę $\lim\limits_{x\to\ 29}\frac{x^{2}-29^{2}}{(x-29)(x-5)}$.
\zadStop
\rozwStart{Patryk Wirkus}{Martyna Czarnobaj}
$$\frac{x^{2}-29^{2}}{(x-29)(x-5)}=\frac{x+29}{x-5}$$

$$\lim\limits_{x\to\ 29}\frac{x^{2}-29^{2}}{(x-29)(x-5)}=[\frac{0}{0}]=\lim\limits_{x\to\ 29}\frac{x+29}{x-5}=2 \cdot \frac{29}{29-5} = \frac{58}{24}$$
\rozwStop
\odpStart
$\frac{58}{24}$
\odpStop
\testStart
A.$\frac{58}{24}$
B.$\infty$
C.$-\infty$
D.$0$
E.$\frac{58}{-24}$
F.$\frac{29}{5}$
G.$-\frac{58}{-24}$
H.$1$
I.$29$
\testStop
\kluczStart
A
\kluczStop



\zadStart{Przykład z Wikieł P 4.2b moja wersja nr 530}
Obliczyć granicę $\lim\limits_{x\to\ 29}\frac{x^{2}-29^{2}}{(x-29)(x-6)}$.
\zadStop
\rozwStart{Patryk Wirkus}{Martyna Czarnobaj}
$$\frac{x^{2}-29^{2}}{(x-29)(x-6)}=\frac{x+29}{x-6}$$

$$\lim\limits_{x\to\ 29}\frac{x^{2}-29^{2}}{(x-29)(x-6)}=[\frac{0}{0}]=\lim\limits_{x\to\ 29}\frac{x+29}{x-6}=2 \cdot \frac{29}{29-6} = \frac{58}{23}$$
\rozwStop
\odpStart
$\frac{58}{23}$
\odpStop
\testStart
A.$\frac{58}{23}$
B.$\infty$
C.$-\infty$
D.$0$
E.$\frac{58}{-23}$
F.$\frac{29}{6}$
G.$-\frac{58}{-23}$
H.$1$
I.$29$
\testStop
\kluczStart
A
\kluczStop



\zadStart{Przykład z Wikieł P 4.2b moja wersja nr 531}
Obliczyć granicę $\lim\limits_{x\to\ 29}\frac{x^{2}-29^{2}}{(x-29)(x-7)}$.
\zadStop
\rozwStart{Patryk Wirkus}{Martyna Czarnobaj}
$$\frac{x^{2}-29^{2}}{(x-29)(x-7)}=\frac{x+29}{x-7}$$

$$\lim\limits_{x\to\ 29}\frac{x^{2}-29^{2}}{(x-29)(x-7)}=[\frac{0}{0}]=\lim\limits_{x\to\ 29}\frac{x+29}{x-7}=2 \cdot \frac{29}{29-7} = \frac{58}{22}$$
\rozwStop
\odpStart
$\frac{58}{22}$
\odpStop
\testStart
A.$\frac{58}{22}$
B.$\infty$
C.$-\infty$
D.$0$
E.$\frac{58}{-22}$
F.$\frac{29}{7}$
G.$-\frac{58}{-22}$
H.$1$
I.$29$
\testStop
\kluczStart
A
\kluczStop



\zadStart{Przykład z Wikieł P 4.2b moja wersja nr 532}
Obliczyć granicę $\lim\limits_{x\to\ 29}\frac{x^{2}-29^{2}}{(x-29)(x-8)}$.
\zadStop
\rozwStart{Patryk Wirkus}{Martyna Czarnobaj}
$$\frac{x^{2}-29^{2}}{(x-29)(x-8)}=\frac{x+29}{x-8}$$

$$\lim\limits_{x\to\ 29}\frac{x^{2}-29^{2}}{(x-29)(x-8)}=[\frac{0}{0}]=\lim\limits_{x\to\ 29}\frac{x+29}{x-8}=2 \cdot \frac{29}{29-8} = \frac{58}{21}$$
\rozwStop
\odpStart
$\frac{58}{21}$
\odpStop
\testStart
A.$\frac{58}{21}$
B.$\infty$
C.$-\infty$
D.$0$
E.$\frac{58}{-21}$
F.$\frac{29}{8}$
G.$-\frac{58}{-21}$
H.$1$
I.$29$
\testStop
\kluczStart
A
\kluczStop



\zadStart{Przykład z Wikieł P 4.2b moja wersja nr 533}
Obliczyć granicę $\lim\limits_{x\to\ 29}\frac{x^{2}-29^{2}}{(x-29)(x-9)}$.
\zadStop
\rozwStart{Patryk Wirkus}{Martyna Czarnobaj}
$$\frac{x^{2}-29^{2}}{(x-29)(x-9)}=\frac{x+29}{x-9}$$

$$\lim\limits_{x\to\ 29}\frac{x^{2}-29^{2}}{(x-29)(x-9)}=[\frac{0}{0}]=\lim\limits_{x\to\ 29}\frac{x+29}{x-9}=2 \cdot \frac{29}{29-9} = \frac{58}{20}$$
\rozwStop
\odpStart
$\frac{58}{20}$
\odpStop
\testStart
A.$\frac{58}{20}$
B.$\infty$
C.$-\infty$
D.$0$
E.$\frac{58}{-20}$
F.$\frac{29}{9}$
G.$-\frac{58}{-20}$
H.$1$
I.$29$
\testStop
\kluczStart
A
\kluczStop



\zadStart{Przykład z Wikieł P 4.2b moja wersja nr 534}
Obliczyć granicę $\lim\limits_{x\to\ 29}\frac{x^{2}-29^{2}}{(x-29)(x-10)}$.
\zadStop
\rozwStart{Patryk Wirkus}{Martyna Czarnobaj}
$$\frac{x^{2}-29^{2}}{(x-29)(x-10)}=\frac{x+29}{x-10}$$

$$\lim\limits_{x\to\ 29}\frac{x^{2}-29^{2}}{(x-29)(x-10)}=[\frac{0}{0}]=\lim\limits_{x\to\ 29}\frac{x+29}{x-10}=2 \cdot \frac{29}{29-10} = \frac{58}{19}$$
\rozwStop
\odpStart
$\frac{58}{19}$
\odpStop
\testStart
A.$\frac{58}{19}$
B.$\infty$
C.$-\infty$
D.$0$
E.$\frac{58}{-19}$
F.$\frac{29}{10}$
G.$-\frac{58}{-19}$
H.$1$
I.$29$
\testStop
\kluczStart
A
\kluczStop



\zadStart{Przykład z Wikieł P 4.2b moja wersja nr 535}
Obliczyć granicę $\lim\limits_{x\to\ 29}\frac{x^{2}-29^{2}}{(x-29)(x-11)}$.
\zadStop
\rozwStart{Patryk Wirkus}{Martyna Czarnobaj}
$$\frac{x^{2}-29^{2}}{(x-29)(x-11)}=\frac{x+29}{x-11}$$

$$\lim\limits_{x\to\ 29}\frac{x^{2}-29^{2}}{(x-29)(x-11)}=[\frac{0}{0}]=\lim\limits_{x\to\ 29}\frac{x+29}{x-11}=2 \cdot \frac{29}{29-11} = \frac{58}{18}$$
\rozwStop
\odpStart
$\frac{58}{18}$
\odpStop
\testStart
A.$\frac{58}{18}$
B.$\infty$
C.$-\infty$
D.$0$
E.$\frac{58}{-18}$
F.$\frac{29}{11}$
G.$-\frac{58}{-18}$
H.$1$
I.$29$
\testStop
\kluczStart
A
\kluczStop



\zadStart{Przykład z Wikieł P 4.2b moja wersja nr 536}
Obliczyć granicę $\lim\limits_{x\to\ 29}\frac{x^{2}-29^{2}}{(x-29)(x-12)}$.
\zadStop
\rozwStart{Patryk Wirkus}{Martyna Czarnobaj}
$$\frac{x^{2}-29^{2}}{(x-29)(x-12)}=\frac{x+29}{x-12}$$

$$\lim\limits_{x\to\ 29}\frac{x^{2}-29^{2}}{(x-29)(x-12)}=[\frac{0}{0}]=\lim\limits_{x\to\ 29}\frac{x+29}{x-12}=2 \cdot \frac{29}{29-12} = \frac{58}{17}$$
\rozwStop
\odpStart
$\frac{58}{17}$
\odpStop
\testStart
A.$\frac{58}{17}$
B.$\infty$
C.$-\infty$
D.$0$
E.$\frac{58}{-17}$
F.$\frac{29}{12}$
G.$-\frac{58}{-17}$
H.$1$
I.$29$
\testStop
\kluczStart
A
\kluczStop



\zadStart{Przykład z Wikieł P 4.2b moja wersja nr 537}
Obliczyć granicę $\lim\limits_{x\to\ 29}\frac{x^{2}-29^{2}}{(x-29)(x-13)}$.
\zadStop
\rozwStart{Patryk Wirkus}{Martyna Czarnobaj}
$$\frac{x^{2}-29^{2}}{(x-29)(x-13)}=\frac{x+29}{x-13}$$

$$\lim\limits_{x\to\ 29}\frac{x^{2}-29^{2}}{(x-29)(x-13)}=[\frac{0}{0}]=\lim\limits_{x\to\ 29}\frac{x+29}{x-13}=2 \cdot \frac{29}{29-13} = \frac{58}{16}$$
\rozwStop
\odpStart
$\frac{58}{16}$
\odpStop
\testStart
A.$\frac{58}{16}$
B.$\infty$
C.$-\infty$
D.$0$
E.$\frac{58}{-16}$
F.$\frac{29}{13}$
G.$-\frac{58}{-16}$
H.$1$
I.$29$
\testStop
\kluczStart
A
\kluczStop



\zadStart{Przykład z Wikieł P 4.2b moja wersja nr 538}
Obliczyć granicę $\lim\limits_{x\to\ 29}\frac{x^{2}-29^{2}}{(x-29)(x-14)}$.
\zadStop
\rozwStart{Patryk Wirkus}{Martyna Czarnobaj}
$$\frac{x^{2}-29^{2}}{(x-29)(x-14)}=\frac{x+29}{x-14}$$

$$\lim\limits_{x\to\ 29}\frac{x^{2}-29^{2}}{(x-29)(x-14)}=[\frac{0}{0}]=\lim\limits_{x\to\ 29}\frac{x+29}{x-14}=2 \cdot \frac{29}{29-14} = \frac{58}{15}$$
\rozwStop
\odpStart
$\frac{58}{15}$
\odpStop
\testStart
A.$\frac{58}{15}$
B.$\infty$
C.$-\infty$
D.$0$
E.$\frac{58}{-15}$
F.$\frac{29}{14}$
G.$-\frac{58}{-15}$
H.$1$
I.$29$
\testStop
\kluczStart
A
\kluczStop



\zadStart{Przykład z Wikieł P 4.2b moja wersja nr 539}
Obliczyć granicę $\lim\limits_{x\to\ 29}\frac{x^{2}-29^{2}}{(x-29)(x-15)}$.
\zadStop
\rozwStart{Patryk Wirkus}{Martyna Czarnobaj}
$$\frac{x^{2}-29^{2}}{(x-29)(x-15)}=\frac{x+29}{x-15}$$

$$\lim\limits_{x\to\ 29}\frac{x^{2}-29^{2}}{(x-29)(x-15)}=[\frac{0}{0}]=\lim\limits_{x\to\ 29}\frac{x+29}{x-15}=2 \cdot \frac{29}{29-15} = \frac{58}{14}$$
\rozwStop
\odpStart
$\frac{58}{14}$
\odpStop
\testStart
A.$\frac{58}{14}$
B.$\infty$
C.$-\infty$
D.$0$
E.$\frac{58}{-14}$
F.$\frac{29}{15}$
G.$-\frac{58}{-14}$
H.$1$
I.$29$
\testStop
\kluczStart
A
\kluczStop



\zadStart{Przykład z Wikieł P 4.2b moja wersja nr 540}
Obliczyć granicę $\lim\limits_{x\to\ 29}\frac{x^{2}-29^{2}}{(x-29)(x-16)}$.
\zadStop
\rozwStart{Patryk Wirkus}{Martyna Czarnobaj}
$$\frac{x^{2}-29^{2}}{(x-29)(x-16)}=\frac{x+29}{x-16}$$

$$\lim\limits_{x\to\ 29}\frac{x^{2}-29^{2}}{(x-29)(x-16)}=[\frac{0}{0}]=\lim\limits_{x\to\ 29}\frac{x+29}{x-16}=2 \cdot \frac{29}{29-16} = \frac{58}{13}$$
\rozwStop
\odpStart
$\frac{58}{13}$
\odpStop
\testStart
A.$\frac{58}{13}$
B.$\infty$
C.$-\infty$
D.$0$
E.$\frac{58}{-13}$
F.$\frac{29}{16}$
G.$-\frac{58}{-13}$
H.$1$
I.$29$
\testStop
\kluczStart
A
\kluczStop



\zadStart{Przykład z Wikieł P 4.2b moja wersja nr 541}
Obliczyć granicę $\lim\limits_{x\to\ 29}\frac{x^{2}-29^{2}}{(x-29)(x-17)}$.
\zadStop
\rozwStart{Patryk Wirkus}{Martyna Czarnobaj}
$$\frac{x^{2}-29^{2}}{(x-29)(x-17)}=\frac{x+29}{x-17}$$

$$\lim\limits_{x\to\ 29}\frac{x^{2}-29^{2}}{(x-29)(x-17)}=[\frac{0}{0}]=\lim\limits_{x\to\ 29}\frac{x+29}{x-17}=2 \cdot \frac{29}{29-17} = \frac{58}{12}$$
\rozwStop
\odpStart
$\frac{58}{12}$
\odpStop
\testStart
A.$\frac{58}{12}$
B.$\infty$
C.$-\infty$
D.$0$
E.$\frac{58}{-12}$
F.$\frac{29}{17}$
G.$-\frac{58}{-12}$
H.$1$
I.$29$
\testStop
\kluczStart
A
\kluczStop



\zadStart{Przykład z Wikieł P 4.2b moja wersja nr 542}
Obliczyć granicę $\lim\limits_{x\to\ 29}\frac{x^{2}-29^{2}}{(x-29)(x-18)}$.
\zadStop
\rozwStart{Patryk Wirkus}{Martyna Czarnobaj}
$$\frac{x^{2}-29^{2}}{(x-29)(x-18)}=\frac{x+29}{x-18}$$

$$\lim\limits_{x\to\ 29}\frac{x^{2}-29^{2}}{(x-29)(x-18)}=[\frac{0}{0}]=\lim\limits_{x\to\ 29}\frac{x+29}{x-18}=2 \cdot \frac{29}{29-18} = \frac{58}{11}$$
\rozwStop
\odpStart
$\frac{58}{11}$
\odpStop
\testStart
A.$\frac{58}{11}$
B.$\infty$
C.$-\infty$
D.$0$
E.$\frac{58}{-11}$
F.$\frac{29}{18}$
G.$-\frac{58}{-11}$
H.$1$
I.$29$
\testStop
\kluczStart
A
\kluczStop



\zadStart{Przykład z Wikieł P 4.2b moja wersja nr 543}
Obliczyć granicę $\lim\limits_{x\to\ 29}\frac{x^{2}-29^{2}}{(x-29)(x-19)}$.
\zadStop
\rozwStart{Patryk Wirkus}{Martyna Czarnobaj}
$$\frac{x^{2}-29^{2}}{(x-29)(x-19)}=\frac{x+29}{x-19}$$

$$\lim\limits_{x\to\ 29}\frac{x^{2}-29^{2}}{(x-29)(x-19)}=[\frac{0}{0}]=\lim\limits_{x\to\ 29}\frac{x+29}{x-19}=2 \cdot \frac{29}{29-19} = \frac{58}{10}$$
\rozwStop
\odpStart
$\frac{58}{10}$
\odpStop
\testStart
A.$\frac{58}{10}$
B.$\infty$
C.$-\infty$
D.$0$
E.$\frac{58}{-10}$
F.$\frac{29}{19}$
G.$-\frac{58}{-10}$
H.$1$
I.$29$
\testStop
\kluczStart
A
\kluczStop



\zadStart{Przykład z Wikieł P 4.2b moja wersja nr 544}
Obliczyć granicę $\lim\limits_{x\to\ 29}\frac{x^{2}-29^{2}}{(x-29)(x-20)}$.
\zadStop
\rozwStart{Patryk Wirkus}{Martyna Czarnobaj}
$$\frac{x^{2}-29^{2}}{(x-29)(x-20)}=\frac{x+29}{x-20}$$

$$\lim\limits_{x\to\ 29}\frac{x^{2}-29^{2}}{(x-29)(x-20)}=[\frac{0}{0}]=\lim\limits_{x\to\ 29}\frac{x+29}{x-20}=2 \cdot \frac{29}{29-20} = \frac{58}{9}$$
\rozwStop
\odpStart
$\frac{58}{9}$
\odpStop
\testStart
A.$\frac{58}{9}$
B.$\infty$
C.$-\infty$
D.$0$
E.$\frac{58}{-9}$
F.$\frac{29}{20}$
G.$-\frac{58}{-9}$
H.$1$
I.$29$
\testStop
\kluczStart
A
\kluczStop



\zadStart{Przykład z Wikieł P 4.2b moja wersja nr 545}
Obliczyć granicę $\lim\limits_{x\to\ 29}\frac{x^{2}-29^{2}}{(x-29)(x-21)}$.
\zadStop
\rozwStart{Patryk Wirkus}{Martyna Czarnobaj}
$$\frac{x^{2}-29^{2}}{(x-29)(x-21)}=\frac{x+29}{x-21}$$

$$\lim\limits_{x\to\ 29}\frac{x^{2}-29^{2}}{(x-29)(x-21)}=[\frac{0}{0}]=\lim\limits_{x\to\ 29}\frac{x+29}{x-21}=2 \cdot \frac{29}{29-21} = \frac{58}{8}$$
\rozwStop
\odpStart
$\frac{58}{8}$
\odpStop
\testStart
A.$\frac{58}{8}$
B.$\infty$
C.$-\infty$
D.$0$
E.$\frac{58}{-8}$
F.$\frac{29}{21}$
G.$-\frac{58}{-8}$
H.$1$
I.$29$
\testStop
\kluczStart
A
\kluczStop



\zadStart{Przykład z Wikieł P 4.2b moja wersja nr 546}
Obliczyć granicę $\lim\limits_{x\to\ 29}\frac{x^{2}-29^{2}}{(x-29)(x-22)}$.
\zadStop
\rozwStart{Patryk Wirkus}{Martyna Czarnobaj}
$$\frac{x^{2}-29^{2}}{(x-29)(x-22)}=\frac{x+29}{x-22}$$

$$\lim\limits_{x\to\ 29}\frac{x^{2}-29^{2}}{(x-29)(x-22)}=[\frac{0}{0}]=\lim\limits_{x\to\ 29}\frac{x+29}{x-22}=2 \cdot \frac{29}{29-22} = \frac{58}{7}$$
\rozwStop
\odpStart
$\frac{58}{7}$
\odpStop
\testStart
A.$\frac{58}{7}$
B.$\infty$
C.$-\infty$
D.$0$
E.$\frac{58}{-7}$
F.$\frac{29}{22}$
G.$-\frac{58}{-7}$
H.$1$
I.$29$
\testStop
\kluczStart
A
\kluczStop



\zadStart{Przykład z Wikieł P 4.2b moja wersja nr 547}
Obliczyć granicę $\lim\limits_{x\to\ 29}\frac{x^{2}-29^{2}}{(x-29)(x-23)}$.
\zadStop
\rozwStart{Patryk Wirkus}{Martyna Czarnobaj}
$$\frac{x^{2}-29^{2}}{(x-29)(x-23)}=\frac{x+29}{x-23}$$

$$\lim\limits_{x\to\ 29}\frac{x^{2}-29^{2}}{(x-29)(x-23)}=[\frac{0}{0}]=\lim\limits_{x\to\ 29}\frac{x+29}{x-23}=2 \cdot \frac{29}{29-23} = \frac{58}{6}$$
\rozwStop
\odpStart
$\frac{58}{6}$
\odpStop
\testStart
A.$\frac{58}{6}$
B.$\infty$
C.$-\infty$
D.$0$
E.$\frac{58}{-6}$
F.$\frac{29}{23}$
G.$-\frac{58}{-6}$
H.$1$
I.$29$
\testStop
\kluczStart
A
\kluczStop



\zadStart{Przykład z Wikieł P 4.2b moja wersja nr 548}
Obliczyć granicę $\lim\limits_{x\to\ 29}\frac{x^{2}-29^{2}}{(x-29)(x-24)}$.
\zadStop
\rozwStart{Patryk Wirkus}{Martyna Czarnobaj}
$$\frac{x^{2}-29^{2}}{(x-29)(x-24)}=\frac{x+29}{x-24}$$

$$\lim\limits_{x\to\ 29}\frac{x^{2}-29^{2}}{(x-29)(x-24)}=[\frac{0}{0}]=\lim\limits_{x\to\ 29}\frac{x+29}{x-24}=2 \cdot \frac{29}{29-24} = \frac{58}{5}$$
\rozwStop
\odpStart
$\frac{58}{5}$
\odpStop
\testStart
A.$\frac{58}{5}$
B.$\infty$
C.$-\infty$
D.$0$
E.$\frac{58}{-5}$
F.$\frac{29}{24}$
G.$-\frac{58}{-5}$
H.$1$
I.$29$
\testStop
\kluczStart
A
\kluczStop



\zadStart{Przykład z Wikieł P 4.2b moja wersja nr 549}
Obliczyć granicę $\lim\limits_{x\to\ 29}\frac{x^{2}-29^{2}}{(x-29)(x-25)}$.
\zadStop
\rozwStart{Patryk Wirkus}{Martyna Czarnobaj}
$$\frac{x^{2}-29^{2}}{(x-29)(x-25)}=\frac{x+29}{x-25}$$

$$\lim\limits_{x\to\ 29}\frac{x^{2}-29^{2}}{(x-29)(x-25)}=[\frac{0}{0}]=\lim\limits_{x\to\ 29}\frac{x+29}{x-25}=2 \cdot \frac{29}{29-25} = \frac{58}{4}$$
\rozwStop
\odpStart
$\frac{58}{4}$
\odpStop
\testStart
A.$\frac{58}{4}$
B.$\infty$
C.$-\infty$
D.$0$
E.$\frac{58}{-4}$
F.$\frac{29}{25}$
G.$-\frac{58}{-4}$
H.$1$
I.$29$
\testStop
\kluczStart
A
\kluczStop



\zadStart{Przykład z Wikieł P 4.2b moja wersja nr 550}
Obliczyć granicę $\lim\limits_{x\to\ 29}\frac{x^{2}-29^{2}}{(x-29)(x-26)}$.
\zadStop
\rozwStart{Patryk Wirkus}{Martyna Czarnobaj}
$$\frac{x^{2}-29^{2}}{(x-29)(x-26)}=\frac{x+29}{x-26}$$

$$\lim\limits_{x\to\ 29}\frac{x^{2}-29^{2}}{(x-29)(x-26)}=[\frac{0}{0}]=\lim\limits_{x\to\ 29}\frac{x+29}{x-26}=2 \cdot \frac{29}{29-26} = \frac{58}{3}$$
\rozwStop
\odpStart
$\frac{58}{3}$
\odpStop
\testStart
A.$\frac{58}{3}$
B.$\infty$
C.$-\infty$
D.$0$
E.$\frac{58}{-3}$
F.$\frac{29}{26}$
G.$-\frac{58}{-3}$
H.$1$
I.$29$
\testStop
\kluczStart
A
\kluczStop



\zadStart{Przykład z Wikieł P 4.2b moja wersja nr 551}
Obliczyć granicę $\lim\limits_{x\to\ 29}\frac{x^{2}-29^{2}}{(x-29)(x-27)}$.
\zadStop
\rozwStart{Patryk Wirkus}{Martyna Czarnobaj}
$$\frac{x^{2}-29^{2}}{(x-29)(x-27)}=\frac{x+29}{x-27}$$

$$\lim\limits_{x\to\ 29}\frac{x^{2}-29^{2}}{(x-29)(x-27)}=[\frac{0}{0}]=\lim\limits_{x\to\ 29}\frac{x+29}{x-27}=2 \cdot \frac{29}{29-27} = \frac{58}{2}$$
\rozwStop
\odpStart
$\frac{58}{2}$
\odpStop
\testStart
A.$\frac{58}{2}$
B.$\infty$
C.$-\infty$
D.$0$
E.$\frac{58}{-2}$
F.$\frac{29}{27}$
G.$-\frac{58}{-2}$
H.$1$
I.$29$
\testStop
\kluczStart
A
\kluczStop



\zadStart{Przykład z Wikieł P 4.2b moja wersja nr 552}
Obliczyć granicę $\lim\limits_{x\to\ 29}\frac{x^{2}-29^{2}}{(x-29)(x-31)}$.
\zadStop
\rozwStart{Patryk Wirkus}{Martyna Czarnobaj}
$$\frac{x^{2}-29^{2}}{(x-29)(x-31)}=\frac{x+29}{x-31}$$

$$\lim\limits_{x\to\ 29}\frac{x^{2}-29^{2}}{(x-29)(x-31)}=[\frac{0}{0}]=\lim\limits_{x\to\ 29}\frac{x+29}{x-31}=2 \cdot \frac{29}{29-31} = \frac{58}{-2}$$
\rozwStop
\odpStart
$\frac{58}{-2}$
\odpStop
\testStart
A.$\frac{58}{-2}$
B.$\infty$
C.$-\infty$
D.$0$
E.$\frac{58}{2}$
F.$\frac{29}{31}$
G.$-\frac{58}{2}$
H.$1$
I.$29$
\testStop
\kluczStart
A
\kluczStop



\zadStart{Przykład z Wikieł P 4.2b moja wersja nr 553}
Obliczyć granicę $\lim\limits_{x\to\ 29}\frac{x^{2}-29^{2}}{(x-29)(x-32)}$.
\zadStop
\rozwStart{Patryk Wirkus}{Martyna Czarnobaj}
$$\frac{x^{2}-29^{2}}{(x-29)(x-32)}=\frac{x+29}{x-32}$$

$$\lim\limits_{x\to\ 29}\frac{x^{2}-29^{2}}{(x-29)(x-32)}=[\frac{0}{0}]=\lim\limits_{x\to\ 29}\frac{x+29}{x-32}=2 \cdot \frac{29}{29-32} = \frac{58}{-3}$$
\rozwStop
\odpStart
$\frac{58}{-3}$
\odpStop
\testStart
A.$\frac{58}{-3}$
B.$\infty$
C.$-\infty$
D.$0$
E.$\frac{58}{3}$
F.$\frac{29}{32}$
G.$-\frac{58}{3}$
H.$1$
I.$29$
\testStop
\kluczStart
A
\kluczStop



\zadStart{Przykład z Wikieł P 4.2b moja wersja nr 554}
Obliczyć granicę $\lim\limits_{x\to\ 29}\frac{x^{2}-29^{2}}{(x-29)(x-33)}$.
\zadStop
\rozwStart{Patryk Wirkus}{Martyna Czarnobaj}
$$\frac{x^{2}-29^{2}}{(x-29)(x-33)}=\frac{x+29}{x-33}$$

$$\lim\limits_{x\to\ 29}\frac{x^{2}-29^{2}}{(x-29)(x-33)}=[\frac{0}{0}]=\lim\limits_{x\to\ 29}\frac{x+29}{x-33}=2 \cdot \frac{29}{29-33} = \frac{58}{-4}$$
\rozwStop
\odpStart
$\frac{58}{-4}$
\odpStop
\testStart
A.$\frac{58}{-4}$
B.$\infty$
C.$-\infty$
D.$0$
E.$\frac{58}{4}$
F.$\frac{29}{33}$
G.$-\frac{58}{4}$
H.$1$
I.$29$
\testStop
\kluczStart
A
\kluczStop



\zadStart{Przykład z Wikieł P 4.2b moja wersja nr 555}
Obliczyć granicę $\lim\limits_{x\to\ 29}\frac{x^{2}-29^{2}}{(x-29)(x-34)}$.
\zadStop
\rozwStart{Patryk Wirkus}{Martyna Czarnobaj}
$$\frac{x^{2}-29^{2}}{(x-29)(x-34)}=\frac{x+29}{x-34}$$

$$\lim\limits_{x\to\ 29}\frac{x^{2}-29^{2}}{(x-29)(x-34)}=[\frac{0}{0}]=\lim\limits_{x\to\ 29}\frac{x+29}{x-34}=2 \cdot \frac{29}{29-34} = \frac{58}{-5}$$
\rozwStop
\odpStart
$\frac{58}{-5}$
\odpStop
\testStart
A.$\frac{58}{-5}$
B.$\infty$
C.$-\infty$
D.$0$
E.$\frac{58}{5}$
F.$\frac{29}{34}$
G.$-\frac{58}{5}$
H.$1$
I.$29$
\testStop
\kluczStart
A
\kluczStop



\zadStart{Przykład z Wikieł P 4.2b moja wersja nr 556}
Obliczyć granicę $\lim\limits_{x\to\ 29}\frac{x^{2}-29^{2}}{(x-29)(x-35)}$.
\zadStop
\rozwStart{Patryk Wirkus}{Martyna Czarnobaj}
$$\frac{x^{2}-29^{2}}{(x-29)(x-35)}=\frac{x+29}{x-35}$$

$$\lim\limits_{x\to\ 29}\frac{x^{2}-29^{2}}{(x-29)(x-35)}=[\frac{0}{0}]=\lim\limits_{x\to\ 29}\frac{x+29}{x-35}=2 \cdot \frac{29}{29-35} = \frac{58}{-6}$$
\rozwStop
\odpStart
$\frac{58}{-6}$
\odpStop
\testStart
A.$\frac{58}{-6}$
B.$\infty$
C.$-\infty$
D.$0$
E.$\frac{58}{6}$
F.$\frac{29}{35}$
G.$-\frac{58}{6}$
H.$1$
I.$29$
\testStop
\kluczStart
A
\kluczStop



\zadStart{Przykład z Wikieł P 4.2b moja wersja nr 557}
Obliczyć granicę $\lim\limits_{x\to\ 29}\frac{x^{2}-29^{2}}{(x-29)(x-36)}$.
\zadStop
\rozwStart{Patryk Wirkus}{Martyna Czarnobaj}
$$\frac{x^{2}-29^{2}}{(x-29)(x-36)}=\frac{x+29}{x-36}$$

$$\lim\limits_{x\to\ 29}\frac{x^{2}-29^{2}}{(x-29)(x-36)}=[\frac{0}{0}]=\lim\limits_{x\to\ 29}\frac{x+29}{x-36}=2 \cdot \frac{29}{29-36} = \frac{58}{-7}$$
\rozwStop
\odpStart
$\frac{58}{-7}$
\odpStop
\testStart
A.$\frac{58}{-7}$
B.$\infty$
C.$-\infty$
D.$0$
E.$\frac{58}{7}$
F.$\frac{29}{36}$
G.$-\frac{58}{7}$
H.$1$
I.$29$
\testStop
\kluczStart
A
\kluczStop



\zadStart{Przykład z Wikieł P 4.2b moja wersja nr 558}
Obliczyć granicę $\lim\limits_{x\to\ 29}\frac{x^{2}-29^{2}}{(x-29)(x-37)}$.
\zadStop
\rozwStart{Patryk Wirkus}{Martyna Czarnobaj}
$$\frac{x^{2}-29^{2}}{(x-29)(x-37)}=\frac{x+29}{x-37}$$

$$\lim\limits_{x\to\ 29}\frac{x^{2}-29^{2}}{(x-29)(x-37)}=[\frac{0}{0}]=\lim\limits_{x\to\ 29}\frac{x+29}{x-37}=2 \cdot \frac{29}{29-37} = \frac{58}{-8}$$
\rozwStop
\odpStart
$\frac{58}{-8}$
\odpStop
\testStart
A.$\frac{58}{-8}$
B.$\infty$
C.$-\infty$
D.$0$
E.$\frac{58}{8}$
F.$\frac{29}{37}$
G.$-\frac{58}{8}$
H.$1$
I.$29$
\testStop
\kluczStart
A
\kluczStop



\zadStart{Przykład z Wikieł P 4.2b moja wersja nr 559}
Obliczyć granicę $\lim\limits_{x\to\ 29}\frac{x^{2}-29^{2}}{(x-29)(x-38)}$.
\zadStop
\rozwStart{Patryk Wirkus}{Martyna Czarnobaj}
$$\frac{x^{2}-29^{2}}{(x-29)(x-38)}=\frac{x+29}{x-38}$$

$$\lim\limits_{x\to\ 29}\frac{x^{2}-29^{2}}{(x-29)(x-38)}=[\frac{0}{0}]=\lim\limits_{x\to\ 29}\frac{x+29}{x-38}=2 \cdot \frac{29}{29-38} = \frac{58}{-9}$$
\rozwStop
\odpStart
$\frac{58}{-9}$
\odpStop
\testStart
A.$\frac{58}{-9}$
B.$\infty$
C.$-\infty$
D.$0$
E.$\frac{58}{9}$
F.$\frac{29}{38}$
G.$-\frac{58}{9}$
H.$1$
I.$29$
\testStop
\kluczStart
A
\kluczStop



\zadStart{Przykład z Wikieł P 4.2b moja wersja nr 560}
Obliczyć granicę $\lim\limits_{x\to\ 29}\frac{x^{2}-29^{2}}{(x-29)(x-39)}$.
\zadStop
\rozwStart{Patryk Wirkus}{Martyna Czarnobaj}
$$\frac{x^{2}-29^{2}}{(x-29)(x-39)}=\frac{x+29}{x-39}$$

$$\lim\limits_{x\to\ 29}\frac{x^{2}-29^{2}}{(x-29)(x-39)}=[\frac{0}{0}]=\lim\limits_{x\to\ 29}\frac{x+29}{x-39}=2 \cdot \frac{29}{29-39} = \frac{58}{-10}$$
\rozwStop
\odpStart
$\frac{58}{-10}$
\odpStop
\testStart
A.$\frac{58}{-10}$
B.$\infty$
C.$-\infty$
D.$0$
E.$\frac{58}{10}$
F.$\frac{29}{39}$
G.$-\frac{58}{10}$
H.$1$
I.$29$
\testStop
\kluczStart
A
\kluczStop



\zadStart{Przykład z Wikieł P 4.2b moja wersja nr 561}
Obliczyć granicę $\lim\limits_{x\to\ 29}\frac{x^{2}-29^{2}}{(x-29)(x-40)}$.
\zadStop
\rozwStart{Patryk Wirkus}{Martyna Czarnobaj}
$$\frac{x^{2}-29^{2}}{(x-29)(x-40)}=\frac{x+29}{x-40}$$

$$\lim\limits_{x\to\ 29}\frac{x^{2}-29^{2}}{(x-29)(x-40)}=[\frac{0}{0}]=\lim\limits_{x\to\ 29}\frac{x+29}{x-40}=2 \cdot \frac{29}{29-40} = \frac{58}{-11}$$
\rozwStop
\odpStart
$\frac{58}{-11}$
\odpStop
\testStart
A.$\frac{58}{-11}$
B.$\infty$
C.$-\infty$
D.$0$
E.$\frac{58}{11}$
F.$\frac{29}{40}$
G.$-\frac{58}{11}$
H.$1$
I.$29$
\testStop
\kluczStart
A
\kluczStop



\zadStart{Przykład z Wikieł P 4.2b moja wersja nr 562}
Obliczyć granicę $\lim\limits_{x\to\ 30}\frac{x^{2}-30^{2}}{(x-30)(x-7)}$.
\zadStop
\rozwStart{Patryk Wirkus}{Martyna Czarnobaj}
$$\frac{x^{2}-30^{2}}{(x-30)(x-7)}=\frac{x+30}{x-7}$$

$$\lim\limits_{x\to\ 30}\frac{x^{2}-30^{2}}{(x-30)(x-7)}=[\frac{0}{0}]=\lim\limits_{x\to\ 30}\frac{x+30}{x-7}=2 \cdot \frac{30}{30-7} = \frac{60}{23}$$
\rozwStop
\odpStart
$\frac{60}{23}$
\odpStop
\testStart
A.$\frac{60}{23}$
B.$\infty$
C.$-\infty$
D.$0$
E.$\frac{60}{-23}$
F.$\frac{30}{7}$
G.$-\frac{60}{-23}$
H.$1$
I.$30$
\testStop
\kluczStart
A
\kluczStop



\zadStart{Przykład z Wikieł P 4.2b moja wersja nr 563}
Obliczyć granicę $\lim\limits_{x\to\ 30}\frac{x^{2}-30^{2}}{(x-30)(x-11)}$.
\zadStop
\rozwStart{Patryk Wirkus}{Martyna Czarnobaj}
$$\frac{x^{2}-30^{2}}{(x-30)(x-11)}=\frac{x+30}{x-11}$$

$$\lim\limits_{x\to\ 30}\frac{x^{2}-30^{2}}{(x-30)(x-11)}=[\frac{0}{0}]=\lim\limits_{x\to\ 30}\frac{x+30}{x-11}=2 \cdot \frac{30}{30-11} = \frac{60}{19}$$
\rozwStop
\odpStart
$\frac{60}{19}$
\odpStop
\testStart
A.$\frac{60}{19}$
B.$\infty$
C.$-\infty$
D.$0$
E.$\frac{60}{-19}$
F.$\frac{30}{11}$
G.$-\frac{60}{-19}$
H.$1$
I.$30$
\testStop
\kluczStart
A
\kluczStop



\zadStart{Przykład z Wikieł P 4.2b moja wersja nr 564}
Obliczyć granicę $\lim\limits_{x\to\ 30}\frac{x^{2}-30^{2}}{(x-30)(x-13)}$.
\zadStop
\rozwStart{Patryk Wirkus}{Martyna Czarnobaj}
$$\frac{x^{2}-30^{2}}{(x-30)(x-13)}=\frac{x+30}{x-13}$$

$$\lim\limits_{x\to\ 30}\frac{x^{2}-30^{2}}{(x-30)(x-13)}=[\frac{0}{0}]=\lim\limits_{x\to\ 30}\frac{x+30}{x-13}=2 \cdot \frac{30}{30-13} = \frac{60}{17}$$
\rozwStop
\odpStart
$\frac{60}{17}$
\odpStop
\testStart
A.$\frac{60}{17}$
B.$\infty$
C.$-\infty$
D.$0$
E.$\frac{60}{-17}$
F.$\frac{30}{13}$
G.$-\frac{60}{-17}$
H.$1$
I.$30$
\testStop
\kluczStart
A
\kluczStop



\zadStart{Przykład z Wikieł P 4.2b moja wersja nr 565}
Obliczyć granicę $\lim\limits_{x\to\ 30}\frac{x^{2}-30^{2}}{(x-30)(x-17)}$.
\zadStop
\rozwStart{Patryk Wirkus}{Martyna Czarnobaj}
$$\frac{x^{2}-30^{2}}{(x-30)(x-17)}=\frac{x+30}{x-17}$$

$$\lim\limits_{x\to\ 30}\frac{x^{2}-30^{2}}{(x-30)(x-17)}=[\frac{0}{0}]=\lim\limits_{x\to\ 30}\frac{x+30}{x-17}=2 \cdot \frac{30}{30-17} = \frac{60}{13}$$
\rozwStop
\odpStart
$\frac{60}{13}$
\odpStop
\testStart
A.$\frac{60}{13}$
B.$\infty$
C.$-\infty$
D.$0$
E.$\frac{60}{-13}$
F.$\frac{30}{17}$
G.$-\frac{60}{-13}$
H.$1$
I.$30$
\testStop
\kluczStart
A
\kluczStop



\zadStart{Przykład z Wikieł P 4.2b moja wersja nr 566}
Obliczyć granicę $\lim\limits_{x\to\ 30}\frac{x^{2}-30^{2}}{(x-30)(x-19)}$.
\zadStop
\rozwStart{Patryk Wirkus}{Martyna Czarnobaj}
$$\frac{x^{2}-30^{2}}{(x-30)(x-19)}=\frac{x+30}{x-19}$$

$$\lim\limits_{x\to\ 30}\frac{x^{2}-30^{2}}{(x-30)(x-19)}=[\frac{0}{0}]=\lim\limits_{x\to\ 30}\frac{x+30}{x-19}=2 \cdot \frac{30}{30-19} = \frac{60}{11}$$
\rozwStop
\odpStart
$\frac{60}{11}$
\odpStop
\testStart
A.$\frac{60}{11}$
B.$\infty$
C.$-\infty$
D.$0$
E.$\frac{60}{-11}$
F.$\frac{30}{19}$
G.$-\frac{60}{-11}$
H.$1$
I.$30$
\testStop
\kluczStart
A
\kluczStop



\zadStart{Przykład z Wikieł P 4.2b moja wersja nr 567}
Obliczyć granicę $\lim\limits_{x\to\ 30}\frac{x^{2}-30^{2}}{(x-30)(x-23)}$.
\zadStop
\rozwStart{Patryk Wirkus}{Martyna Czarnobaj}
$$\frac{x^{2}-30^{2}}{(x-30)(x-23)}=\frac{x+30}{x-23}$$

$$\lim\limits_{x\to\ 30}\frac{x^{2}-30^{2}}{(x-30)(x-23)}=[\frac{0}{0}]=\lim\limits_{x\to\ 30}\frac{x+30}{x-23}=2 \cdot \frac{30}{30-23} = \frac{60}{7}$$
\rozwStop
\odpStart
$\frac{60}{7}$
\odpStop
\testStart
A.$\frac{60}{7}$
B.$\infty$
C.$-\infty$
D.$0$
E.$\frac{60}{-7}$
F.$\frac{30}{23}$
G.$-\frac{60}{-7}$
H.$1$
I.$30$
\testStop
\kluczStart
A
\kluczStop



\zadStart{Przykład z Wikieł P 4.2b moja wersja nr 568}
Obliczyć granicę $\lim\limits_{x\to\ 30}\frac{x^{2}-30^{2}}{(x-30)(x-37)}$.
\zadStop
\rozwStart{Patryk Wirkus}{Martyna Czarnobaj}
$$\frac{x^{2}-30^{2}}{(x-30)(x-37)}=\frac{x+30}{x-37}$$

$$\lim\limits_{x\to\ 30}\frac{x^{2}-30^{2}}{(x-30)(x-37)}=[\frac{0}{0}]=\lim\limits_{x\to\ 30}\frac{x+30}{x-37}=2 \cdot \frac{30}{30-37} = \frac{60}{-7}$$
\rozwStop
\odpStart
$\frac{60}{-7}$
\odpStop
\testStart
A.$\frac{60}{-7}$
B.$\infty$
C.$-\infty$
D.$0$
E.$\frac{60}{7}$
F.$\frac{30}{37}$
G.$-\frac{60}{7}$
H.$1$
I.$30$
\testStop
\kluczStart
A
\kluczStop



\zadStart{Przykład z Wikieł P 4.2b moja wersja nr 569}
Obliczyć granicę $\lim\limits_{x\to\ 31}\frac{x^{2}-31^{2}}{(x-31)(x-2)}$.
\zadStop
\rozwStart{Patryk Wirkus}{Martyna Czarnobaj}
$$\frac{x^{2}-31^{2}}{(x-31)(x-2)}=\frac{x+31}{x-2}$$

$$\lim\limits_{x\to\ 31}\frac{x^{2}-31^{2}}{(x-31)(x-2)}=[\frac{0}{0}]=\lim\limits_{x\to\ 31}\frac{x+31}{x-2}=2 \cdot \frac{31}{31-2} = \frac{62}{29}$$
\rozwStop
\odpStart
$\frac{62}{29}$
\odpStop
\testStart
A.$\frac{62}{29}$
B.$\infty$
C.$-\infty$
D.$0$
E.$\frac{62}{-29}$
F.$\frac{31}{2}$
G.$-\frac{62}{-29}$
H.$1$
I.$31$
\testStop
\kluczStart
A
\kluczStop



\zadStart{Przykład z Wikieł P 4.2b moja wersja nr 570}
Obliczyć granicę $\lim\limits_{x\to\ 31}\frac{x^{2}-31^{2}}{(x-31)(x-3)}$.
\zadStop
\rozwStart{Patryk Wirkus}{Martyna Czarnobaj}
$$\frac{x^{2}-31^{2}}{(x-31)(x-3)}=\frac{x+31}{x-3}$$

$$\lim\limits_{x\to\ 31}\frac{x^{2}-31^{2}}{(x-31)(x-3)}=[\frac{0}{0}]=\lim\limits_{x\to\ 31}\frac{x+31}{x-3}=2 \cdot \frac{31}{31-3} = \frac{62}{28}$$
\rozwStop
\odpStart
$\frac{62}{28}$
\odpStop
\testStart
A.$\frac{62}{28}$
B.$\infty$
C.$-\infty$
D.$0$
E.$\frac{62}{-28}$
F.$\frac{31}{3}$
G.$-\frac{62}{-28}$
H.$1$
I.$31$
\testStop
\kluczStart
A
\kluczStop



\zadStart{Przykład z Wikieł P 4.2b moja wersja nr 571}
Obliczyć granicę $\lim\limits_{x\to\ 31}\frac{x^{2}-31^{2}}{(x-31)(x-4)}$.
\zadStop
\rozwStart{Patryk Wirkus}{Martyna Czarnobaj}
$$\frac{x^{2}-31^{2}}{(x-31)(x-4)}=\frac{x+31}{x-4}$$

$$\lim\limits_{x\to\ 31}\frac{x^{2}-31^{2}}{(x-31)(x-4)}=[\frac{0}{0}]=\lim\limits_{x\to\ 31}\frac{x+31}{x-4}=2 \cdot \frac{31}{31-4} = \frac{62}{27}$$
\rozwStop
\odpStart
$\frac{62}{27}$
\odpStop
\testStart
A.$\frac{62}{27}$
B.$\infty$
C.$-\infty$
D.$0$
E.$\frac{62}{-27}$
F.$\frac{31}{4}$
G.$-\frac{62}{-27}$
H.$1$
I.$31$
\testStop
\kluczStart
A
\kluczStop



\zadStart{Przykład z Wikieł P 4.2b moja wersja nr 572}
Obliczyć granicę $\lim\limits_{x\to\ 31}\frac{x^{2}-31^{2}}{(x-31)(x-5)}$.
\zadStop
\rozwStart{Patryk Wirkus}{Martyna Czarnobaj}
$$\frac{x^{2}-31^{2}}{(x-31)(x-5)}=\frac{x+31}{x-5}$$

$$\lim\limits_{x\to\ 31}\frac{x^{2}-31^{2}}{(x-31)(x-5)}=[\frac{0}{0}]=\lim\limits_{x\to\ 31}\frac{x+31}{x-5}=2 \cdot \frac{31}{31-5} = \frac{62}{26}$$
\rozwStop
\odpStart
$\frac{62}{26}$
\odpStop
\testStart
A.$\frac{62}{26}$
B.$\infty$
C.$-\infty$
D.$0$
E.$\frac{62}{-26}$
F.$\frac{31}{5}$
G.$-\frac{62}{-26}$
H.$1$
I.$31$
\testStop
\kluczStart
A
\kluczStop



\zadStart{Przykład z Wikieł P 4.2b moja wersja nr 573}
Obliczyć granicę $\lim\limits_{x\to\ 31}\frac{x^{2}-31^{2}}{(x-31)(x-6)}$.
\zadStop
\rozwStart{Patryk Wirkus}{Martyna Czarnobaj}
$$\frac{x^{2}-31^{2}}{(x-31)(x-6)}=\frac{x+31}{x-6}$$

$$\lim\limits_{x\to\ 31}\frac{x^{2}-31^{2}}{(x-31)(x-6)}=[\frac{0}{0}]=\lim\limits_{x\to\ 31}\frac{x+31}{x-6}=2 \cdot \frac{31}{31-6} = \frac{62}{25}$$
\rozwStop
\odpStart
$\frac{62}{25}$
\odpStop
\testStart
A.$\frac{62}{25}$
B.$\infty$
C.$-\infty$
D.$0$
E.$\frac{62}{-25}$
F.$\frac{31}{6}$
G.$-\frac{62}{-25}$
H.$1$
I.$31$
\testStop
\kluczStart
A
\kluczStop



\zadStart{Przykład z Wikieł P 4.2b moja wersja nr 574}
Obliczyć granicę $\lim\limits_{x\to\ 31}\frac{x^{2}-31^{2}}{(x-31)(x-7)}$.
\zadStop
\rozwStart{Patryk Wirkus}{Martyna Czarnobaj}
$$\frac{x^{2}-31^{2}}{(x-31)(x-7)}=\frac{x+31}{x-7}$$

$$\lim\limits_{x\to\ 31}\frac{x^{2}-31^{2}}{(x-31)(x-7)}=[\frac{0}{0}]=\lim\limits_{x\to\ 31}\frac{x+31}{x-7}=2 \cdot \frac{31}{31-7} = \frac{62}{24}$$
\rozwStop
\odpStart
$\frac{62}{24}$
\odpStop
\testStart
A.$\frac{62}{24}$
B.$\infty$
C.$-\infty$
D.$0$
E.$\frac{62}{-24}$
F.$\frac{31}{7}$
G.$-\frac{62}{-24}$
H.$1$
I.$31$
\testStop
\kluczStart
A
\kluczStop



\zadStart{Przykład z Wikieł P 4.2b moja wersja nr 575}
Obliczyć granicę $\lim\limits_{x\to\ 31}\frac{x^{2}-31^{2}}{(x-31)(x-8)}$.
\zadStop
\rozwStart{Patryk Wirkus}{Martyna Czarnobaj}
$$\frac{x^{2}-31^{2}}{(x-31)(x-8)}=\frac{x+31}{x-8}$$

$$\lim\limits_{x\to\ 31}\frac{x^{2}-31^{2}}{(x-31)(x-8)}=[\frac{0}{0}]=\lim\limits_{x\to\ 31}\frac{x+31}{x-8}=2 \cdot \frac{31}{31-8} = \frac{62}{23}$$
\rozwStop
\odpStart
$\frac{62}{23}$
\odpStop
\testStart
A.$\frac{62}{23}$
B.$\infty$
C.$-\infty$
D.$0$
E.$\frac{62}{-23}$
F.$\frac{31}{8}$
G.$-\frac{62}{-23}$
H.$1$
I.$31$
\testStop
\kluczStart
A
\kluczStop



\zadStart{Przykład z Wikieł P 4.2b moja wersja nr 576}
Obliczyć granicę $\lim\limits_{x\to\ 31}\frac{x^{2}-31^{2}}{(x-31)(x-9)}$.
\zadStop
\rozwStart{Patryk Wirkus}{Martyna Czarnobaj}
$$\frac{x^{2}-31^{2}}{(x-31)(x-9)}=\frac{x+31}{x-9}$$

$$\lim\limits_{x\to\ 31}\frac{x^{2}-31^{2}}{(x-31)(x-9)}=[\frac{0}{0}]=\lim\limits_{x\to\ 31}\frac{x+31}{x-9}=2 \cdot \frac{31}{31-9} = \frac{62}{22}$$
\rozwStop
\odpStart
$\frac{62}{22}$
\odpStop
\testStart
A.$\frac{62}{22}$
B.$\infty$
C.$-\infty$
D.$0$
E.$\frac{62}{-22}$
F.$\frac{31}{9}$
G.$-\frac{62}{-22}$
H.$1$
I.$31$
\testStop
\kluczStart
A
\kluczStop



\zadStart{Przykład z Wikieł P 4.2b moja wersja nr 577}
Obliczyć granicę $\lim\limits_{x\to\ 31}\frac{x^{2}-31^{2}}{(x-31)(x-10)}$.
\zadStop
\rozwStart{Patryk Wirkus}{Martyna Czarnobaj}
$$\frac{x^{2}-31^{2}}{(x-31)(x-10)}=\frac{x+31}{x-10}$$

$$\lim\limits_{x\to\ 31}\frac{x^{2}-31^{2}}{(x-31)(x-10)}=[\frac{0}{0}]=\lim\limits_{x\to\ 31}\frac{x+31}{x-10}=2 \cdot \frac{31}{31-10} = \frac{62}{21}$$
\rozwStop
\odpStart
$\frac{62}{21}$
\odpStop
\testStart
A.$\frac{62}{21}$
B.$\infty$
C.$-\infty$
D.$0$
E.$\frac{62}{-21}$
F.$\frac{31}{10}$
G.$-\frac{62}{-21}$
H.$1$
I.$31$
\testStop
\kluczStart
A
\kluczStop



\zadStart{Przykład z Wikieł P 4.2b moja wersja nr 578}
Obliczyć granicę $\lim\limits_{x\to\ 31}\frac{x^{2}-31^{2}}{(x-31)(x-11)}$.
\zadStop
\rozwStart{Patryk Wirkus}{Martyna Czarnobaj}
$$\frac{x^{2}-31^{2}}{(x-31)(x-11)}=\frac{x+31}{x-11}$$

$$\lim\limits_{x\to\ 31}\frac{x^{2}-31^{2}}{(x-31)(x-11)}=[\frac{0}{0}]=\lim\limits_{x\to\ 31}\frac{x+31}{x-11}=2 \cdot \frac{31}{31-11} = \frac{62}{20}$$
\rozwStop
\odpStart
$\frac{62}{20}$
\odpStop
\testStart
A.$\frac{62}{20}$
B.$\infty$
C.$-\infty$
D.$0$
E.$\frac{62}{-20}$
F.$\frac{31}{11}$
G.$-\frac{62}{-20}$
H.$1$
I.$31$
\testStop
\kluczStart
A
\kluczStop



\zadStart{Przykład z Wikieł P 4.2b moja wersja nr 579}
Obliczyć granicę $\lim\limits_{x\to\ 31}\frac{x^{2}-31^{2}}{(x-31)(x-12)}$.
\zadStop
\rozwStart{Patryk Wirkus}{Martyna Czarnobaj}
$$\frac{x^{2}-31^{2}}{(x-31)(x-12)}=\frac{x+31}{x-12}$$

$$\lim\limits_{x\to\ 31}\frac{x^{2}-31^{2}}{(x-31)(x-12)}=[\frac{0}{0}]=\lim\limits_{x\to\ 31}\frac{x+31}{x-12}=2 \cdot \frac{31}{31-12} = \frac{62}{19}$$
\rozwStop
\odpStart
$\frac{62}{19}$
\odpStop
\testStart
A.$\frac{62}{19}$
B.$\infty$
C.$-\infty$
D.$0$
E.$\frac{62}{-19}$
F.$\frac{31}{12}$
G.$-\frac{62}{-19}$
H.$1$
I.$31$
\testStop
\kluczStart
A
\kluczStop



\zadStart{Przykład z Wikieł P 4.2b moja wersja nr 580}
Obliczyć granicę $\lim\limits_{x\to\ 31}\frac{x^{2}-31^{2}}{(x-31)(x-13)}$.
\zadStop
\rozwStart{Patryk Wirkus}{Martyna Czarnobaj}
$$\frac{x^{2}-31^{2}}{(x-31)(x-13)}=\frac{x+31}{x-13}$$

$$\lim\limits_{x\to\ 31}\frac{x^{2}-31^{2}}{(x-31)(x-13)}=[\frac{0}{0}]=\lim\limits_{x\to\ 31}\frac{x+31}{x-13}=2 \cdot \frac{31}{31-13} = \frac{62}{18}$$
\rozwStop
\odpStart
$\frac{62}{18}$
\odpStop
\testStart
A.$\frac{62}{18}$
B.$\infty$
C.$-\infty$
D.$0$
E.$\frac{62}{-18}$
F.$\frac{31}{13}$
G.$-\frac{62}{-18}$
H.$1$
I.$31$
\testStop
\kluczStart
A
\kluczStop



\zadStart{Przykład z Wikieł P 4.2b moja wersja nr 581}
Obliczyć granicę $\lim\limits_{x\to\ 31}\frac{x^{2}-31^{2}}{(x-31)(x-14)}$.
\zadStop
\rozwStart{Patryk Wirkus}{Martyna Czarnobaj}
$$\frac{x^{2}-31^{2}}{(x-31)(x-14)}=\frac{x+31}{x-14}$$

$$\lim\limits_{x\to\ 31}\frac{x^{2}-31^{2}}{(x-31)(x-14)}=[\frac{0}{0}]=\lim\limits_{x\to\ 31}\frac{x+31}{x-14}=2 \cdot \frac{31}{31-14} = \frac{62}{17}$$
\rozwStop
\odpStart
$\frac{62}{17}$
\odpStop
\testStart
A.$\frac{62}{17}$
B.$\infty$
C.$-\infty$
D.$0$
E.$\frac{62}{-17}$
F.$\frac{31}{14}$
G.$-\frac{62}{-17}$
H.$1$
I.$31$
\testStop
\kluczStart
A
\kluczStop



\zadStart{Przykład z Wikieł P 4.2b moja wersja nr 582}
Obliczyć granicę $\lim\limits_{x\to\ 31}\frac{x^{2}-31^{2}}{(x-31)(x-15)}$.
\zadStop
\rozwStart{Patryk Wirkus}{Martyna Czarnobaj}
$$\frac{x^{2}-31^{2}}{(x-31)(x-15)}=\frac{x+31}{x-15}$$

$$\lim\limits_{x\to\ 31}\frac{x^{2}-31^{2}}{(x-31)(x-15)}=[\frac{0}{0}]=\lim\limits_{x\to\ 31}\frac{x+31}{x-15}=2 \cdot \frac{31}{31-15} = \frac{62}{16}$$
\rozwStop
\odpStart
$\frac{62}{16}$
\odpStop
\testStart
A.$\frac{62}{16}$
B.$\infty$
C.$-\infty$
D.$0$
E.$\frac{62}{-16}$
F.$\frac{31}{15}$
G.$-\frac{62}{-16}$
H.$1$
I.$31$
\testStop
\kluczStart
A
\kluczStop



\zadStart{Przykład z Wikieł P 4.2b moja wersja nr 583}
Obliczyć granicę $\lim\limits_{x\to\ 31}\frac{x^{2}-31^{2}}{(x-31)(x-16)}$.
\zadStop
\rozwStart{Patryk Wirkus}{Martyna Czarnobaj}
$$\frac{x^{2}-31^{2}}{(x-31)(x-16)}=\frac{x+31}{x-16}$$

$$\lim\limits_{x\to\ 31}\frac{x^{2}-31^{2}}{(x-31)(x-16)}=[\frac{0}{0}]=\lim\limits_{x\to\ 31}\frac{x+31}{x-16}=2 \cdot \frac{31}{31-16} = \frac{62}{15}$$
\rozwStop
\odpStart
$\frac{62}{15}$
\odpStop
\testStart
A.$\frac{62}{15}$
B.$\infty$
C.$-\infty$
D.$0$
E.$\frac{62}{-15}$
F.$\frac{31}{16}$
G.$-\frac{62}{-15}$
H.$1$
I.$31$
\testStop
\kluczStart
A
\kluczStop



\zadStart{Przykład z Wikieł P 4.2b moja wersja nr 584}
Obliczyć granicę $\lim\limits_{x\to\ 31}\frac{x^{2}-31^{2}}{(x-31)(x-17)}$.
\zadStop
\rozwStart{Patryk Wirkus}{Martyna Czarnobaj}
$$\frac{x^{2}-31^{2}}{(x-31)(x-17)}=\frac{x+31}{x-17}$$

$$\lim\limits_{x\to\ 31}\frac{x^{2}-31^{2}}{(x-31)(x-17)}=[\frac{0}{0}]=\lim\limits_{x\to\ 31}\frac{x+31}{x-17}=2 \cdot \frac{31}{31-17} = \frac{62}{14}$$
\rozwStop
\odpStart
$\frac{62}{14}$
\odpStop
\testStart
A.$\frac{62}{14}$
B.$\infty$
C.$-\infty$
D.$0$
E.$\frac{62}{-14}$
F.$\frac{31}{17}$
G.$-\frac{62}{-14}$
H.$1$
I.$31$
\testStop
\kluczStart
A
\kluczStop



\zadStart{Przykład z Wikieł P 4.2b moja wersja nr 585}
Obliczyć granicę $\lim\limits_{x\to\ 31}\frac{x^{2}-31^{2}}{(x-31)(x-18)}$.
\zadStop
\rozwStart{Patryk Wirkus}{Martyna Czarnobaj}
$$\frac{x^{2}-31^{2}}{(x-31)(x-18)}=\frac{x+31}{x-18}$$

$$\lim\limits_{x\to\ 31}\frac{x^{2}-31^{2}}{(x-31)(x-18)}=[\frac{0}{0}]=\lim\limits_{x\to\ 31}\frac{x+31}{x-18}=2 \cdot \frac{31}{31-18} = \frac{62}{13}$$
\rozwStop
\odpStart
$\frac{62}{13}$
\odpStop
\testStart
A.$\frac{62}{13}$
B.$\infty$
C.$-\infty$
D.$0$
E.$\frac{62}{-13}$
F.$\frac{31}{18}$
G.$-\frac{62}{-13}$
H.$1$
I.$31$
\testStop
\kluczStart
A
\kluczStop



\zadStart{Przykład z Wikieł P 4.2b moja wersja nr 586}
Obliczyć granicę $\lim\limits_{x\to\ 31}\frac{x^{2}-31^{2}}{(x-31)(x-19)}$.
\zadStop
\rozwStart{Patryk Wirkus}{Martyna Czarnobaj}
$$\frac{x^{2}-31^{2}}{(x-31)(x-19)}=\frac{x+31}{x-19}$$

$$\lim\limits_{x\to\ 31}\frac{x^{2}-31^{2}}{(x-31)(x-19)}=[\frac{0}{0}]=\lim\limits_{x\to\ 31}\frac{x+31}{x-19}=2 \cdot \frac{31}{31-19} = \frac{62}{12}$$
\rozwStop
\odpStart
$\frac{62}{12}$
\odpStop
\testStart
A.$\frac{62}{12}$
B.$\infty$
C.$-\infty$
D.$0$
E.$\frac{62}{-12}$
F.$\frac{31}{19}$
G.$-\frac{62}{-12}$
H.$1$
I.$31$
\testStop
\kluczStart
A
\kluczStop



\zadStart{Przykład z Wikieł P 4.2b moja wersja nr 587}
Obliczyć granicę $\lim\limits_{x\to\ 31}\frac{x^{2}-31^{2}}{(x-31)(x-20)}$.
\zadStop
\rozwStart{Patryk Wirkus}{Martyna Czarnobaj}
$$\frac{x^{2}-31^{2}}{(x-31)(x-20)}=\frac{x+31}{x-20}$$

$$\lim\limits_{x\to\ 31}\frac{x^{2}-31^{2}}{(x-31)(x-20)}=[\frac{0}{0}]=\lim\limits_{x\to\ 31}\frac{x+31}{x-20}=2 \cdot \frac{31}{31-20} = \frac{62}{11}$$
\rozwStop
\odpStart
$\frac{62}{11}$
\odpStop
\testStart
A.$\frac{62}{11}$
B.$\infty$
C.$-\infty$
D.$0$
E.$\frac{62}{-11}$
F.$\frac{31}{20}$
G.$-\frac{62}{-11}$
H.$1$
I.$31$
\testStop
\kluczStart
A
\kluczStop



\zadStart{Przykład z Wikieł P 4.2b moja wersja nr 588}
Obliczyć granicę $\lim\limits_{x\to\ 31}\frac{x^{2}-31^{2}}{(x-31)(x-21)}$.
\zadStop
\rozwStart{Patryk Wirkus}{Martyna Czarnobaj}
$$\frac{x^{2}-31^{2}}{(x-31)(x-21)}=\frac{x+31}{x-21}$$

$$\lim\limits_{x\to\ 31}\frac{x^{2}-31^{2}}{(x-31)(x-21)}=[\frac{0}{0}]=\lim\limits_{x\to\ 31}\frac{x+31}{x-21}=2 \cdot \frac{31}{31-21} = \frac{62}{10}$$
\rozwStop
\odpStart
$\frac{62}{10}$
\odpStop
\testStart
A.$\frac{62}{10}$
B.$\infty$
C.$-\infty$
D.$0$
E.$\frac{62}{-10}$
F.$\frac{31}{21}$
G.$-\frac{62}{-10}$
H.$1$
I.$31$
\testStop
\kluczStart
A
\kluczStop



\zadStart{Przykład z Wikieł P 4.2b moja wersja nr 589}
Obliczyć granicę $\lim\limits_{x\to\ 31}\frac{x^{2}-31^{2}}{(x-31)(x-22)}$.
\zadStop
\rozwStart{Patryk Wirkus}{Martyna Czarnobaj}
$$\frac{x^{2}-31^{2}}{(x-31)(x-22)}=\frac{x+31}{x-22}$$

$$\lim\limits_{x\to\ 31}\frac{x^{2}-31^{2}}{(x-31)(x-22)}=[\frac{0}{0}]=\lim\limits_{x\to\ 31}\frac{x+31}{x-22}=2 \cdot \frac{31}{31-22} = \frac{62}{9}$$
\rozwStop
\odpStart
$\frac{62}{9}$
\odpStop
\testStart
A.$\frac{62}{9}$
B.$\infty$
C.$-\infty$
D.$0$
E.$\frac{62}{-9}$
F.$\frac{31}{22}$
G.$-\frac{62}{-9}$
H.$1$
I.$31$
\testStop
\kluczStart
A
\kluczStop



\zadStart{Przykład z Wikieł P 4.2b moja wersja nr 590}
Obliczyć granicę $\lim\limits_{x\to\ 31}\frac{x^{2}-31^{2}}{(x-31)(x-23)}$.
\zadStop
\rozwStart{Patryk Wirkus}{Martyna Czarnobaj}
$$\frac{x^{2}-31^{2}}{(x-31)(x-23)}=\frac{x+31}{x-23}$$

$$\lim\limits_{x\to\ 31}\frac{x^{2}-31^{2}}{(x-31)(x-23)}=[\frac{0}{0}]=\lim\limits_{x\to\ 31}\frac{x+31}{x-23}=2 \cdot \frac{31}{31-23} = \frac{62}{8}$$
\rozwStop
\odpStart
$\frac{62}{8}$
\odpStop
\testStart
A.$\frac{62}{8}$
B.$\infty$
C.$-\infty$
D.$0$
E.$\frac{62}{-8}$
F.$\frac{31}{23}$
G.$-\frac{62}{-8}$
H.$1$
I.$31$
\testStop
\kluczStart
A
\kluczStop



\zadStart{Przykład z Wikieł P 4.2b moja wersja nr 591}
Obliczyć granicę $\lim\limits_{x\to\ 31}\frac{x^{2}-31^{2}}{(x-31)(x-24)}$.
\zadStop
\rozwStart{Patryk Wirkus}{Martyna Czarnobaj}
$$\frac{x^{2}-31^{2}}{(x-31)(x-24)}=\frac{x+31}{x-24}$$

$$\lim\limits_{x\to\ 31}\frac{x^{2}-31^{2}}{(x-31)(x-24)}=[\frac{0}{0}]=\lim\limits_{x\to\ 31}\frac{x+31}{x-24}=2 \cdot \frac{31}{31-24} = \frac{62}{7}$$
\rozwStop
\odpStart
$\frac{62}{7}$
\odpStop
\testStart
A.$\frac{62}{7}$
B.$\infty$
C.$-\infty$
D.$0$
E.$\frac{62}{-7}$
F.$\frac{31}{24}$
G.$-\frac{62}{-7}$
H.$1$
I.$31$
\testStop
\kluczStart
A
\kluczStop



\zadStart{Przykład z Wikieł P 4.2b moja wersja nr 592}
Obliczyć granicę $\lim\limits_{x\to\ 31}\frac{x^{2}-31^{2}}{(x-31)(x-25)}$.
\zadStop
\rozwStart{Patryk Wirkus}{Martyna Czarnobaj}
$$\frac{x^{2}-31^{2}}{(x-31)(x-25)}=\frac{x+31}{x-25}$$

$$\lim\limits_{x\to\ 31}\frac{x^{2}-31^{2}}{(x-31)(x-25)}=[\frac{0}{0}]=\lim\limits_{x\to\ 31}\frac{x+31}{x-25}=2 \cdot \frac{31}{31-25} = \frac{62}{6}$$
\rozwStop
\odpStart
$\frac{62}{6}$
\odpStop
\testStart
A.$\frac{62}{6}$
B.$\infty$
C.$-\infty$
D.$0$
E.$\frac{62}{-6}$
F.$\frac{31}{25}$
G.$-\frac{62}{-6}$
H.$1$
I.$31$
\testStop
\kluczStart
A
\kluczStop



\zadStart{Przykład z Wikieł P 4.2b moja wersja nr 593}
Obliczyć granicę $\lim\limits_{x\to\ 31}\frac{x^{2}-31^{2}}{(x-31)(x-26)}$.
\zadStop
\rozwStart{Patryk Wirkus}{Martyna Czarnobaj}
$$\frac{x^{2}-31^{2}}{(x-31)(x-26)}=\frac{x+31}{x-26}$$

$$\lim\limits_{x\to\ 31}\frac{x^{2}-31^{2}}{(x-31)(x-26)}=[\frac{0}{0}]=\lim\limits_{x\to\ 31}\frac{x+31}{x-26}=2 \cdot \frac{31}{31-26} = \frac{62}{5}$$
\rozwStop
\odpStart
$\frac{62}{5}$
\odpStop
\testStart
A.$\frac{62}{5}$
B.$\infty$
C.$-\infty$
D.$0$
E.$\frac{62}{-5}$
F.$\frac{31}{26}$
G.$-\frac{62}{-5}$
H.$1$
I.$31$
\testStop
\kluczStart
A
\kluczStop



\zadStart{Przykład z Wikieł P 4.2b moja wersja nr 594}
Obliczyć granicę $\lim\limits_{x\to\ 31}\frac{x^{2}-31^{2}}{(x-31)(x-27)}$.
\zadStop
\rozwStart{Patryk Wirkus}{Martyna Czarnobaj}
$$\frac{x^{2}-31^{2}}{(x-31)(x-27)}=\frac{x+31}{x-27}$$

$$\lim\limits_{x\to\ 31}\frac{x^{2}-31^{2}}{(x-31)(x-27)}=[\frac{0}{0}]=\lim\limits_{x\to\ 31}\frac{x+31}{x-27}=2 \cdot \frac{31}{31-27} = \frac{62}{4}$$
\rozwStop
\odpStart
$\frac{62}{4}$
\odpStop
\testStart
A.$\frac{62}{4}$
B.$\infty$
C.$-\infty$
D.$0$
E.$\frac{62}{-4}$
F.$\frac{31}{27}$
G.$-\frac{62}{-4}$
H.$1$
I.$31$
\testStop
\kluczStart
A
\kluczStop



\zadStart{Przykład z Wikieł P 4.2b moja wersja nr 595}
Obliczyć granicę $\lim\limits_{x\to\ 31}\frac{x^{2}-31^{2}}{(x-31)(x-28)}$.
\zadStop
\rozwStart{Patryk Wirkus}{Martyna Czarnobaj}
$$\frac{x^{2}-31^{2}}{(x-31)(x-28)}=\frac{x+31}{x-28}$$

$$\lim\limits_{x\to\ 31}\frac{x^{2}-31^{2}}{(x-31)(x-28)}=[\frac{0}{0}]=\lim\limits_{x\to\ 31}\frac{x+31}{x-28}=2 \cdot \frac{31}{31-28} = \frac{62}{3}$$
\rozwStop
\odpStart
$\frac{62}{3}$
\odpStop
\testStart
A.$\frac{62}{3}$
B.$\infty$
C.$-\infty$
D.$0$
E.$\frac{62}{-3}$
F.$\frac{31}{28}$
G.$-\frac{62}{-3}$
H.$1$
I.$31$
\testStop
\kluczStart
A
\kluczStop



\zadStart{Przykład z Wikieł P 4.2b moja wersja nr 596}
Obliczyć granicę $\lim\limits_{x\to\ 31}\frac{x^{2}-31^{2}}{(x-31)(x-29)}$.
\zadStop
\rozwStart{Patryk Wirkus}{Martyna Czarnobaj}
$$\frac{x^{2}-31^{2}}{(x-31)(x-29)}=\frac{x+31}{x-29}$$

$$\lim\limits_{x\to\ 31}\frac{x^{2}-31^{2}}{(x-31)(x-29)}=[\frac{0}{0}]=\lim\limits_{x\to\ 31}\frac{x+31}{x-29}=2 \cdot \frac{31}{31-29} = \frac{62}{2}$$
\rozwStop
\odpStart
$\frac{62}{2}$
\odpStop
\testStart
A.$\frac{62}{2}$
B.$\infty$
C.$-\infty$
D.$0$
E.$\frac{62}{-2}$
F.$\frac{31}{29}$
G.$-\frac{62}{-2}$
H.$1$
I.$31$
\testStop
\kluczStart
A
\kluczStop



\zadStart{Przykład z Wikieł P 4.2b moja wersja nr 597}
Obliczyć granicę $\lim\limits_{x\to\ 31}\frac{x^{2}-31^{2}}{(x-31)(x-33)}$.
\zadStop
\rozwStart{Patryk Wirkus}{Martyna Czarnobaj}
$$\frac{x^{2}-31^{2}}{(x-31)(x-33)}=\frac{x+31}{x-33}$$

$$\lim\limits_{x\to\ 31}\frac{x^{2}-31^{2}}{(x-31)(x-33)}=[\frac{0}{0}]=\lim\limits_{x\to\ 31}\frac{x+31}{x-33}=2 \cdot \frac{31}{31-33} = \frac{62}{-2}$$
\rozwStop
\odpStart
$\frac{62}{-2}$
\odpStop
\testStart
A.$\frac{62}{-2}$
B.$\infty$
C.$-\infty$
D.$0$
E.$\frac{62}{2}$
F.$\frac{31}{33}$
G.$-\frac{62}{2}$
H.$1$
I.$31$
\testStop
\kluczStart
A
\kluczStop



\zadStart{Przykład z Wikieł P 4.2b moja wersja nr 598}
Obliczyć granicę $\lim\limits_{x\to\ 31}\frac{x^{2}-31^{2}}{(x-31)(x-34)}$.
\zadStop
\rozwStart{Patryk Wirkus}{Martyna Czarnobaj}
$$\frac{x^{2}-31^{2}}{(x-31)(x-34)}=\frac{x+31}{x-34}$$

$$\lim\limits_{x\to\ 31}\frac{x^{2}-31^{2}}{(x-31)(x-34)}=[\frac{0}{0}]=\lim\limits_{x\to\ 31}\frac{x+31}{x-34}=2 \cdot \frac{31}{31-34} = \frac{62}{-3}$$
\rozwStop
\odpStart
$\frac{62}{-3}$
\odpStop
\testStart
A.$\frac{62}{-3}$
B.$\infty$
C.$-\infty$
D.$0$
E.$\frac{62}{3}$
F.$\frac{31}{34}$
G.$-\frac{62}{3}$
H.$1$
I.$31$
\testStop
\kluczStart
A
\kluczStop



\zadStart{Przykład z Wikieł P 4.2b moja wersja nr 599}
Obliczyć granicę $\lim\limits_{x\to\ 31}\frac{x^{2}-31^{2}}{(x-31)(x-35)}$.
\zadStop
\rozwStart{Patryk Wirkus}{Martyna Czarnobaj}
$$\frac{x^{2}-31^{2}}{(x-31)(x-35)}=\frac{x+31}{x-35}$$

$$\lim\limits_{x\to\ 31}\frac{x^{2}-31^{2}}{(x-31)(x-35)}=[\frac{0}{0}]=\lim\limits_{x\to\ 31}\frac{x+31}{x-35}=2 \cdot \frac{31}{31-35} = \frac{62}{-4}$$
\rozwStop
\odpStart
$\frac{62}{-4}$
\odpStop
\testStart
A.$\frac{62}{-4}$
B.$\infty$
C.$-\infty$
D.$0$
E.$\frac{62}{4}$
F.$\frac{31}{35}$
G.$-\frac{62}{4}$
H.$1$
I.$31$
\testStop
\kluczStart
A
\kluczStop



\zadStart{Przykład z Wikieł P 4.2b moja wersja nr 600}
Obliczyć granicę $\lim\limits_{x\to\ 31}\frac{x^{2}-31^{2}}{(x-31)(x-36)}$.
\zadStop
\rozwStart{Patryk Wirkus}{Martyna Czarnobaj}
$$\frac{x^{2}-31^{2}}{(x-31)(x-36)}=\frac{x+31}{x-36}$$

$$\lim\limits_{x\to\ 31}\frac{x^{2}-31^{2}}{(x-31)(x-36)}=[\frac{0}{0}]=\lim\limits_{x\to\ 31}\frac{x+31}{x-36}=2 \cdot \frac{31}{31-36} = \frac{62}{-5}$$
\rozwStop
\odpStart
$\frac{62}{-5}$
\odpStop
\testStart
A.$\frac{62}{-5}$
B.$\infty$
C.$-\infty$
D.$0$
E.$\frac{62}{5}$
F.$\frac{31}{36}$
G.$-\frac{62}{5}$
H.$1$
I.$31$
\testStop
\kluczStart
A
\kluczStop



\zadStart{Przykład z Wikieł P 4.2b moja wersja nr 601}
Obliczyć granicę $\lim\limits_{x\to\ 31}\frac{x^{2}-31^{2}}{(x-31)(x-37)}$.
\zadStop
\rozwStart{Patryk Wirkus}{Martyna Czarnobaj}
$$\frac{x^{2}-31^{2}}{(x-31)(x-37)}=\frac{x+31}{x-37}$$

$$\lim\limits_{x\to\ 31}\frac{x^{2}-31^{2}}{(x-31)(x-37)}=[\frac{0}{0}]=\lim\limits_{x\to\ 31}\frac{x+31}{x-37}=2 \cdot \frac{31}{31-37} = \frac{62}{-6}$$
\rozwStop
\odpStart
$\frac{62}{-6}$
\odpStop
\testStart
A.$\frac{62}{-6}$
B.$\infty$
C.$-\infty$
D.$0$
E.$\frac{62}{6}$
F.$\frac{31}{37}$
G.$-\frac{62}{6}$
H.$1$
I.$31$
\testStop
\kluczStart
A
\kluczStop



\zadStart{Przykład z Wikieł P 4.2b moja wersja nr 602}
Obliczyć granicę $\lim\limits_{x\to\ 31}\frac{x^{2}-31^{2}}{(x-31)(x-38)}$.
\zadStop
\rozwStart{Patryk Wirkus}{Martyna Czarnobaj}
$$\frac{x^{2}-31^{2}}{(x-31)(x-38)}=\frac{x+31}{x-38}$$

$$\lim\limits_{x\to\ 31}\frac{x^{2}-31^{2}}{(x-31)(x-38)}=[\frac{0}{0}]=\lim\limits_{x\to\ 31}\frac{x+31}{x-38}=2 \cdot \frac{31}{31-38} = \frac{62}{-7}$$
\rozwStop
\odpStart
$\frac{62}{-7}$
\odpStop
\testStart
A.$\frac{62}{-7}$
B.$\infty$
C.$-\infty$
D.$0$
E.$\frac{62}{7}$
F.$\frac{31}{38}$
G.$-\frac{62}{7}$
H.$1$
I.$31$
\testStop
\kluczStart
A
\kluczStop



\zadStart{Przykład z Wikieł P 4.2b moja wersja nr 603}
Obliczyć granicę $\lim\limits_{x\to\ 31}\frac{x^{2}-31^{2}}{(x-31)(x-39)}$.
\zadStop
\rozwStart{Patryk Wirkus}{Martyna Czarnobaj}
$$\frac{x^{2}-31^{2}}{(x-31)(x-39)}=\frac{x+31}{x-39}$$

$$\lim\limits_{x\to\ 31}\frac{x^{2}-31^{2}}{(x-31)(x-39)}=[\frac{0}{0}]=\lim\limits_{x\to\ 31}\frac{x+31}{x-39}=2 \cdot \frac{31}{31-39} = \frac{62}{-8}$$
\rozwStop
\odpStart
$\frac{62}{-8}$
\odpStop
\testStart
A.$\frac{62}{-8}$
B.$\infty$
C.$-\infty$
D.$0$
E.$\frac{62}{8}$
F.$\frac{31}{39}$
G.$-\frac{62}{8}$
H.$1$
I.$31$
\testStop
\kluczStart
A
\kluczStop



\zadStart{Przykład z Wikieł P 4.2b moja wersja nr 604}
Obliczyć granicę $\lim\limits_{x\to\ 31}\frac{x^{2}-31^{2}}{(x-31)(x-40)}$.
\zadStop
\rozwStart{Patryk Wirkus}{Martyna Czarnobaj}
$$\frac{x^{2}-31^{2}}{(x-31)(x-40)}=\frac{x+31}{x-40}$$

$$\lim\limits_{x\to\ 31}\frac{x^{2}-31^{2}}{(x-31)(x-40)}=[\frac{0}{0}]=\lim\limits_{x\to\ 31}\frac{x+31}{x-40}=2 \cdot \frac{31}{31-40} = \frac{62}{-9}$$
\rozwStop
\odpStart
$\frac{62}{-9}$
\odpStop
\testStart
A.$\frac{62}{-9}$
B.$\infty$
C.$-\infty$
D.$0$
E.$\frac{62}{9}$
F.$\frac{31}{40}$
G.$-\frac{62}{9}$
H.$1$
I.$31$
\testStop
\kluczStart
A
\kluczStop



\zadStart{Przykład z Wikieł P 4.2b moja wersja nr 605}
Obliczyć granicę $\lim\limits_{x\to\ 32}\frac{x^{2}-32^{2}}{(x-32)(x-3)}$.
\zadStop
\rozwStart{Patryk Wirkus}{Martyna Czarnobaj}
$$\frac{x^{2}-32^{2}}{(x-32)(x-3)}=\frac{x+32}{x-3}$$

$$\lim\limits_{x\to\ 32}\frac{x^{2}-32^{2}}{(x-32)(x-3)}=[\frac{0}{0}]=\lim\limits_{x\to\ 32}\frac{x+32}{x-3}=2 \cdot \frac{32}{32-3} = \frac{64}{29}$$
\rozwStop
\odpStart
$\frac{64}{29}$
\odpStop
\testStart
A.$\frac{64}{29}$
B.$\infty$
C.$-\infty$
D.$0$
E.$\frac{64}{-29}$
F.$\frac{32}{3}$
G.$-\frac{64}{-29}$
H.$1$
I.$32$
\testStop
\kluczStart
A
\kluczStop



\zadStart{Przykład z Wikieł P 4.2b moja wersja nr 606}
Obliczyć granicę $\lim\limits_{x\to\ 32}\frac{x^{2}-32^{2}}{(x-32)(x-5)}$.
\zadStop
\rozwStart{Patryk Wirkus}{Martyna Czarnobaj}
$$\frac{x^{2}-32^{2}}{(x-32)(x-5)}=\frac{x+32}{x-5}$$

$$\lim\limits_{x\to\ 32}\frac{x^{2}-32^{2}}{(x-32)(x-5)}=[\frac{0}{0}]=\lim\limits_{x\to\ 32}\frac{x+32}{x-5}=2 \cdot \frac{32}{32-5} = \frac{64}{27}$$
\rozwStop
\odpStart
$\frac{64}{27}$
\odpStop
\testStart
A.$\frac{64}{27}$
B.$\infty$
C.$-\infty$
D.$0$
E.$\frac{64}{-27}$
F.$\frac{32}{5}$
G.$-\frac{64}{-27}$
H.$1$
I.$32$
\testStop
\kluczStart
A
\kluczStop



\zadStart{Przykład z Wikieł P 4.2b moja wersja nr 607}
Obliczyć granicę $\lim\limits_{x\to\ 32}\frac{x^{2}-32^{2}}{(x-32)(x-7)}$.
\zadStop
\rozwStart{Patryk Wirkus}{Martyna Czarnobaj}
$$\frac{x^{2}-32^{2}}{(x-32)(x-7)}=\frac{x+32}{x-7}$$

$$\lim\limits_{x\to\ 32}\frac{x^{2}-32^{2}}{(x-32)(x-7)}=[\frac{0}{0}]=\lim\limits_{x\to\ 32}\frac{x+32}{x-7}=2 \cdot \frac{32}{32-7} = \frac{64}{25}$$
\rozwStop
\odpStart
$\frac{64}{25}$
\odpStop
\testStart
A.$\frac{64}{25}$
B.$\infty$
C.$-\infty$
D.$0$
E.$\frac{64}{-25}$
F.$\frac{32}{7}$
G.$-\frac{64}{-25}$
H.$1$
I.$32$
\testStop
\kluczStart
A
\kluczStop



\zadStart{Przykład z Wikieł P 4.2b moja wersja nr 608}
Obliczyć granicę $\lim\limits_{x\to\ 32}\frac{x^{2}-32^{2}}{(x-32)(x-9)}$.
\zadStop
\rozwStart{Patryk Wirkus}{Martyna Czarnobaj}
$$\frac{x^{2}-32^{2}}{(x-32)(x-9)}=\frac{x+32}{x-9}$$

$$\lim\limits_{x\to\ 32}\frac{x^{2}-32^{2}}{(x-32)(x-9)}=[\frac{0}{0}]=\lim\limits_{x\to\ 32}\frac{x+32}{x-9}=2 \cdot \frac{32}{32-9} = \frac{64}{23}$$
\rozwStop
\odpStart
$\frac{64}{23}$
\odpStop
\testStart
A.$\frac{64}{23}$
B.$\infty$
C.$-\infty$
D.$0$
E.$\frac{64}{-23}$
F.$\frac{32}{9}$
G.$-\frac{64}{-23}$
H.$1$
I.$32$
\testStop
\kluczStart
A
\kluczStop



\zadStart{Przykład z Wikieł P 4.2b moja wersja nr 609}
Obliczyć granicę $\lim\limits_{x\to\ 32}\frac{x^{2}-32^{2}}{(x-32)(x-11)}$.
\zadStop
\rozwStart{Patryk Wirkus}{Martyna Czarnobaj}
$$\frac{x^{2}-32^{2}}{(x-32)(x-11)}=\frac{x+32}{x-11}$$

$$\lim\limits_{x\to\ 32}\frac{x^{2}-32^{2}}{(x-32)(x-11)}=[\frac{0}{0}]=\lim\limits_{x\to\ 32}\frac{x+32}{x-11}=2 \cdot \frac{32}{32-11} = \frac{64}{21}$$
\rozwStop
\odpStart
$\frac{64}{21}$
\odpStop
\testStart
A.$\frac{64}{21}$
B.$\infty$
C.$-\infty$
D.$0$
E.$\frac{64}{-21}$
F.$\frac{32}{11}$
G.$-\frac{64}{-21}$
H.$1$
I.$32$
\testStop
\kluczStart
A
\kluczStop



\zadStart{Przykład z Wikieł P 4.2b moja wersja nr 610}
Obliczyć granicę $\lim\limits_{x\to\ 32}\frac{x^{2}-32^{2}}{(x-32)(x-13)}$.
\zadStop
\rozwStart{Patryk Wirkus}{Martyna Czarnobaj}
$$\frac{x^{2}-32^{2}}{(x-32)(x-13)}=\frac{x+32}{x-13}$$

$$\lim\limits_{x\to\ 32}\frac{x^{2}-32^{2}}{(x-32)(x-13)}=[\frac{0}{0}]=\lim\limits_{x\to\ 32}\frac{x+32}{x-13}=2 \cdot \frac{32}{32-13} = \frac{64}{19}$$
\rozwStop
\odpStart
$\frac{64}{19}$
\odpStop
\testStart
A.$\frac{64}{19}$
B.$\infty$
C.$-\infty$
D.$0$
E.$\frac{64}{-19}$
F.$\frac{32}{13}$
G.$-\frac{64}{-19}$
H.$1$
I.$32$
\testStop
\kluczStart
A
\kluczStop



\zadStart{Przykład z Wikieł P 4.2b moja wersja nr 611}
Obliczyć granicę $\lim\limits_{x\to\ 32}\frac{x^{2}-32^{2}}{(x-32)(x-15)}$.
\zadStop
\rozwStart{Patryk Wirkus}{Martyna Czarnobaj}
$$\frac{x^{2}-32^{2}}{(x-32)(x-15)}=\frac{x+32}{x-15}$$

$$\lim\limits_{x\to\ 32}\frac{x^{2}-32^{2}}{(x-32)(x-15)}=[\frac{0}{0}]=\lim\limits_{x\to\ 32}\frac{x+32}{x-15}=2 \cdot \frac{32}{32-15} = \frac{64}{17}$$
\rozwStop
\odpStart
$\frac{64}{17}$
\odpStop
\testStart
A.$\frac{64}{17}$
B.$\infty$
C.$-\infty$
D.$0$
E.$\frac{64}{-17}$
F.$\frac{32}{15}$
G.$-\frac{64}{-17}$
H.$1$
I.$32$
\testStop
\kluczStart
A
\kluczStop



\zadStart{Przykład z Wikieł P 4.2b moja wersja nr 612}
Obliczyć granicę $\lim\limits_{x\to\ 32}\frac{x^{2}-32^{2}}{(x-32)(x-17)}$.
\zadStop
\rozwStart{Patryk Wirkus}{Martyna Czarnobaj}
$$\frac{x^{2}-32^{2}}{(x-32)(x-17)}=\frac{x+32}{x-17}$$

$$\lim\limits_{x\to\ 32}\frac{x^{2}-32^{2}}{(x-32)(x-17)}=[\frac{0}{0}]=\lim\limits_{x\to\ 32}\frac{x+32}{x-17}=2 \cdot \frac{32}{32-17} = \frac{64}{15}$$
\rozwStop
\odpStart
$\frac{64}{15}$
\odpStop
\testStart
A.$\frac{64}{15}$
B.$\infty$
C.$-\infty$
D.$0$
E.$\frac{64}{-15}$
F.$\frac{32}{17}$
G.$-\frac{64}{-15}$
H.$1$
I.$32$
\testStop
\kluczStart
A
\kluczStop



\zadStart{Przykład z Wikieł P 4.2b moja wersja nr 613}
Obliczyć granicę $\lim\limits_{x\to\ 32}\frac{x^{2}-32^{2}}{(x-32)(x-19)}$.
\zadStop
\rozwStart{Patryk Wirkus}{Martyna Czarnobaj}
$$\frac{x^{2}-32^{2}}{(x-32)(x-19)}=\frac{x+32}{x-19}$$

$$\lim\limits_{x\to\ 32}\frac{x^{2}-32^{2}}{(x-32)(x-19)}=[\frac{0}{0}]=\lim\limits_{x\to\ 32}\frac{x+32}{x-19}=2 \cdot \frac{32}{32-19} = \frac{64}{13}$$
\rozwStop
\odpStart
$\frac{64}{13}$
\odpStop
\testStart
A.$\frac{64}{13}$
B.$\infty$
C.$-\infty$
D.$0$
E.$\frac{64}{-13}$
F.$\frac{32}{19}$
G.$-\frac{64}{-13}$
H.$1$
I.$32$
\testStop
\kluczStart
A
\kluczStop



\zadStart{Przykład z Wikieł P 4.2b moja wersja nr 614}
Obliczyć granicę $\lim\limits_{x\to\ 32}\frac{x^{2}-32^{2}}{(x-32)(x-21)}$.
\zadStop
\rozwStart{Patryk Wirkus}{Martyna Czarnobaj}
$$\frac{x^{2}-32^{2}}{(x-32)(x-21)}=\frac{x+32}{x-21}$$

$$\lim\limits_{x\to\ 32}\frac{x^{2}-32^{2}}{(x-32)(x-21)}=[\frac{0}{0}]=\lim\limits_{x\to\ 32}\frac{x+32}{x-21}=2 \cdot \frac{32}{32-21} = \frac{64}{11}$$
\rozwStop
\odpStart
$\frac{64}{11}$
\odpStop
\testStart
A.$\frac{64}{11}$
B.$\infty$
C.$-\infty$
D.$0$
E.$\frac{64}{-11}$
F.$\frac{32}{21}$
G.$-\frac{64}{-11}$
H.$1$
I.$32$
\testStop
\kluczStart
A
\kluczStop



\zadStart{Przykład z Wikieł P 4.2b moja wersja nr 615}
Obliczyć granicę $\lim\limits_{x\to\ 32}\frac{x^{2}-32^{2}}{(x-32)(x-23)}$.
\zadStop
\rozwStart{Patryk Wirkus}{Martyna Czarnobaj}
$$\frac{x^{2}-32^{2}}{(x-32)(x-23)}=\frac{x+32}{x-23}$$

$$\lim\limits_{x\to\ 32}\frac{x^{2}-32^{2}}{(x-32)(x-23)}=[\frac{0}{0}]=\lim\limits_{x\to\ 32}\frac{x+32}{x-23}=2 \cdot \frac{32}{32-23} = \frac{64}{9}$$
\rozwStop
\odpStart
$\frac{64}{9}$
\odpStop
\testStart
A.$\frac{64}{9}$
B.$\infty$
C.$-\infty$
D.$0$
E.$\frac{64}{-9}$
F.$\frac{32}{23}$
G.$-\frac{64}{-9}$
H.$1$
I.$32$
\testStop
\kluczStart
A
\kluczStop



\zadStart{Przykład z Wikieł P 4.2b moja wersja nr 616}
Obliczyć granicę $\lim\limits_{x\to\ 32}\frac{x^{2}-32^{2}}{(x-32)(x-25)}$.
\zadStop
\rozwStart{Patryk Wirkus}{Martyna Czarnobaj}
$$\frac{x^{2}-32^{2}}{(x-32)(x-25)}=\frac{x+32}{x-25}$$

$$\lim\limits_{x\to\ 32}\frac{x^{2}-32^{2}}{(x-32)(x-25)}=[\frac{0}{0}]=\lim\limits_{x\to\ 32}\frac{x+32}{x-25}=2 \cdot \frac{32}{32-25} = \frac{64}{7}$$
\rozwStop
\odpStart
$\frac{64}{7}$
\odpStop
\testStart
A.$\frac{64}{7}$
B.$\infty$
C.$-\infty$
D.$0$
E.$\frac{64}{-7}$
F.$\frac{32}{25}$
G.$-\frac{64}{-7}$
H.$1$
I.$32$
\testStop
\kluczStart
A
\kluczStop



\zadStart{Przykład z Wikieł P 4.2b moja wersja nr 617}
Obliczyć granicę $\lim\limits_{x\to\ 32}\frac{x^{2}-32^{2}}{(x-32)(x-27)}$.
\zadStop
\rozwStart{Patryk Wirkus}{Martyna Czarnobaj}
$$\frac{x^{2}-32^{2}}{(x-32)(x-27)}=\frac{x+32}{x-27}$$

$$\lim\limits_{x\to\ 32}\frac{x^{2}-32^{2}}{(x-32)(x-27)}=[\frac{0}{0}]=\lim\limits_{x\to\ 32}\frac{x+32}{x-27}=2 \cdot \frac{32}{32-27} = \frac{64}{5}$$
\rozwStop
\odpStart
$\frac{64}{5}$
\odpStop
\testStart
A.$\frac{64}{5}$
B.$\infty$
C.$-\infty$
D.$0$
E.$\frac{64}{-5}$
F.$\frac{32}{27}$
G.$-\frac{64}{-5}$
H.$1$
I.$32$
\testStop
\kluczStart
A
\kluczStop



\zadStart{Przykład z Wikieł P 4.2b moja wersja nr 618}
Obliczyć granicę $\lim\limits_{x\to\ 32}\frac{x^{2}-32^{2}}{(x-32)(x-29)}$.
\zadStop
\rozwStart{Patryk Wirkus}{Martyna Czarnobaj}
$$\frac{x^{2}-32^{2}}{(x-32)(x-29)}=\frac{x+32}{x-29}$$

$$\lim\limits_{x\to\ 32}\frac{x^{2}-32^{2}}{(x-32)(x-29)}=[\frac{0}{0}]=\lim\limits_{x\to\ 32}\frac{x+32}{x-29}=2 \cdot \frac{32}{32-29} = \frac{64}{3}$$
\rozwStop
\odpStart
$\frac{64}{3}$
\odpStop
\testStart
A.$\frac{64}{3}$
B.$\infty$
C.$-\infty$
D.$0$
E.$\frac{64}{-3}$
F.$\frac{32}{29}$
G.$-\frac{64}{-3}$
H.$1$
I.$32$
\testStop
\kluczStart
A
\kluczStop



\zadStart{Przykład z Wikieł P 4.2b moja wersja nr 619}
Obliczyć granicę $\lim\limits_{x\to\ 32}\frac{x^{2}-32^{2}}{(x-32)(x-35)}$.
\zadStop
\rozwStart{Patryk Wirkus}{Martyna Czarnobaj}
$$\frac{x^{2}-32^{2}}{(x-32)(x-35)}=\frac{x+32}{x-35}$$

$$\lim\limits_{x\to\ 32}\frac{x^{2}-32^{2}}{(x-32)(x-35)}=[\frac{0}{0}]=\lim\limits_{x\to\ 32}\frac{x+32}{x-35}=2 \cdot \frac{32}{32-35} = \frac{64}{-3}$$
\rozwStop
\odpStart
$\frac{64}{-3}$
\odpStop
\testStart
A.$\frac{64}{-3}$
B.$\infty$
C.$-\infty$
D.$0$
E.$\frac{64}{3}$
F.$\frac{32}{35}$
G.$-\frac{64}{3}$
H.$1$
I.$32$
\testStop
\kluczStart
A
\kluczStop



\zadStart{Przykład z Wikieł P 4.2b moja wersja nr 620}
Obliczyć granicę $\lim\limits_{x\to\ 32}\frac{x^{2}-32^{2}}{(x-32)(x-37)}$.
\zadStop
\rozwStart{Patryk Wirkus}{Martyna Czarnobaj}
$$\frac{x^{2}-32^{2}}{(x-32)(x-37)}=\frac{x+32}{x-37}$$

$$\lim\limits_{x\to\ 32}\frac{x^{2}-32^{2}}{(x-32)(x-37)}=[\frac{0}{0}]=\lim\limits_{x\to\ 32}\frac{x+32}{x-37}=2 \cdot \frac{32}{32-37} = \frac{64}{-5}$$
\rozwStop
\odpStart
$\frac{64}{-5}$
\odpStop
\testStart
A.$\frac{64}{-5}$
B.$\infty$
C.$-\infty$
D.$0$
E.$\frac{64}{5}$
F.$\frac{32}{37}$
G.$-\frac{64}{5}$
H.$1$
I.$32$
\testStop
\kluczStart
A
\kluczStop



\zadStart{Przykład z Wikieł P 4.2b moja wersja nr 621}
Obliczyć granicę $\lim\limits_{x\to\ 32}\frac{x^{2}-32^{2}}{(x-32)(x-39)}$.
\zadStop
\rozwStart{Patryk Wirkus}{Martyna Czarnobaj}
$$\frac{x^{2}-32^{2}}{(x-32)(x-39)}=\frac{x+32}{x-39}$$

$$\lim\limits_{x\to\ 32}\frac{x^{2}-32^{2}}{(x-32)(x-39)}=[\frac{0}{0}]=\lim\limits_{x\to\ 32}\frac{x+32}{x-39}=2 \cdot \frac{32}{32-39} = \frac{64}{-7}$$
\rozwStop
\odpStart
$\frac{64}{-7}$
\odpStop
\testStart
A.$\frac{64}{-7}$
B.$\infty$
C.$-\infty$
D.$0$
E.$\frac{64}{7}$
F.$\frac{32}{39}$
G.$-\frac{64}{7}$
H.$1$
I.$32$
\testStop
\kluczStart
A
\kluczStop



\zadStart{Przykład z Wikieł P 4.2b moja wersja nr 622}
Obliczyć granicę $\lim\limits_{x\to\ 33}\frac{x^{2}-33^{2}}{(x-33)(x-2)}$.
\zadStop
\rozwStart{Patryk Wirkus}{Martyna Czarnobaj}
$$\frac{x^{2}-33^{2}}{(x-33)(x-2)}=\frac{x+33}{x-2}$$

$$\lim\limits_{x\to\ 33}\frac{x^{2}-33^{2}}{(x-33)(x-2)}=[\frac{0}{0}]=\lim\limits_{x\to\ 33}\frac{x+33}{x-2}=2 \cdot \frac{33}{33-2} = \frac{66}{31}$$
\rozwStop
\odpStart
$\frac{66}{31}$
\odpStop
\testStart
A.$\frac{66}{31}$
B.$\infty$
C.$-\infty$
D.$0$
E.$\frac{66}{-31}$
F.$\frac{33}{2}$
G.$-\frac{66}{-31}$
H.$1$
I.$33$
\testStop
\kluczStart
A
\kluczStop



\zadStart{Przykład z Wikieł P 4.2b moja wersja nr 623}
Obliczyć granicę $\lim\limits_{x\to\ 33}\frac{x^{2}-33^{2}}{(x-33)(x-4)}$.
\zadStop
\rozwStart{Patryk Wirkus}{Martyna Czarnobaj}
$$\frac{x^{2}-33^{2}}{(x-33)(x-4)}=\frac{x+33}{x-4}$$

$$\lim\limits_{x\to\ 33}\frac{x^{2}-33^{2}}{(x-33)(x-4)}=[\frac{0}{0}]=\lim\limits_{x\to\ 33}\frac{x+33}{x-4}=2 \cdot \frac{33}{33-4} = \frac{66}{29}$$
\rozwStop
\odpStart
$\frac{66}{29}$
\odpStop
\testStart
A.$\frac{66}{29}$
B.$\infty$
C.$-\infty$
D.$0$
E.$\frac{66}{-29}$
F.$\frac{33}{4}$
G.$-\frac{66}{-29}$
H.$1$
I.$33$
\testStop
\kluczStart
A
\kluczStop



\zadStart{Przykład z Wikieł P 4.2b moja wersja nr 624}
Obliczyć granicę $\lim\limits_{x\to\ 33}\frac{x^{2}-33^{2}}{(x-33)(x-5)}$.
\zadStop
\rozwStart{Patryk Wirkus}{Martyna Czarnobaj}
$$\frac{x^{2}-33^{2}}{(x-33)(x-5)}=\frac{x+33}{x-5}$$

$$\lim\limits_{x\to\ 33}\frac{x^{2}-33^{2}}{(x-33)(x-5)}=[\frac{0}{0}]=\lim\limits_{x\to\ 33}\frac{x+33}{x-5}=2 \cdot \frac{33}{33-5} = \frac{66}{28}$$
\rozwStop
\odpStart
$\frac{66}{28}$
\odpStop
\testStart
A.$\frac{66}{28}$
B.$\infty$
C.$-\infty$
D.$0$
E.$\frac{66}{-28}$
F.$\frac{33}{5}$
G.$-\frac{66}{-28}$
H.$1$
I.$33$
\testStop
\kluczStart
A
\kluczStop



\zadStart{Przykład z Wikieł P 4.2b moja wersja nr 625}
Obliczyć granicę $\lim\limits_{x\to\ 33}\frac{x^{2}-33^{2}}{(x-33)(x-7)}$.
\zadStop
\rozwStart{Patryk Wirkus}{Martyna Czarnobaj}
$$\frac{x^{2}-33^{2}}{(x-33)(x-7)}=\frac{x+33}{x-7}$$

$$\lim\limits_{x\to\ 33}\frac{x^{2}-33^{2}}{(x-33)(x-7)}=[\frac{0}{0}]=\lim\limits_{x\to\ 33}\frac{x+33}{x-7}=2 \cdot \frac{33}{33-7} = \frac{66}{26}$$
\rozwStop
\odpStart
$\frac{66}{26}$
\odpStop
\testStart
A.$\frac{66}{26}$
B.$\infty$
C.$-\infty$
D.$0$
E.$\frac{66}{-26}$
F.$\frac{33}{7}$
G.$-\frac{66}{-26}$
H.$1$
I.$33$
\testStop
\kluczStart
A
\kluczStop



\zadStart{Przykład z Wikieł P 4.2b moja wersja nr 626}
Obliczyć granicę $\lim\limits_{x\to\ 33}\frac{x^{2}-33^{2}}{(x-33)(x-8)}$.
\zadStop
\rozwStart{Patryk Wirkus}{Martyna Czarnobaj}
$$\frac{x^{2}-33^{2}}{(x-33)(x-8)}=\frac{x+33}{x-8}$$

$$\lim\limits_{x\to\ 33}\frac{x^{2}-33^{2}}{(x-33)(x-8)}=[\frac{0}{0}]=\lim\limits_{x\to\ 33}\frac{x+33}{x-8}=2 \cdot \frac{33}{33-8} = \frac{66}{25}$$
\rozwStop
\odpStart
$\frac{66}{25}$
\odpStop
\testStart
A.$\frac{66}{25}$
B.$\infty$
C.$-\infty$
D.$0$
E.$\frac{66}{-25}$
F.$\frac{33}{8}$
G.$-\frac{66}{-25}$
H.$1$
I.$33$
\testStop
\kluczStart
A
\kluczStop



\zadStart{Przykład z Wikieł P 4.2b moja wersja nr 627}
Obliczyć granicę $\lim\limits_{x\to\ 33}\frac{x^{2}-33^{2}}{(x-33)(x-10)}$.
\zadStop
\rozwStart{Patryk Wirkus}{Martyna Czarnobaj}
$$\frac{x^{2}-33^{2}}{(x-33)(x-10)}=\frac{x+33}{x-10}$$

$$\lim\limits_{x\to\ 33}\frac{x^{2}-33^{2}}{(x-33)(x-10)}=[\frac{0}{0}]=\lim\limits_{x\to\ 33}\frac{x+33}{x-10}=2 \cdot \frac{33}{33-10} = \frac{66}{23}$$
\rozwStop
\odpStart
$\frac{66}{23}$
\odpStop
\testStart
A.$\frac{66}{23}$
B.$\infty$
C.$-\infty$
D.$0$
E.$\frac{66}{-23}$
F.$\frac{33}{10}$
G.$-\frac{66}{-23}$
H.$1$
I.$33$
\testStop
\kluczStart
A
\kluczStop



\zadStart{Przykład z Wikieł P 4.2b moja wersja nr 628}
Obliczyć granicę $\lim\limits_{x\to\ 33}\frac{x^{2}-33^{2}}{(x-33)(x-13)}$.
\zadStop
\rozwStart{Patryk Wirkus}{Martyna Czarnobaj}
$$\frac{x^{2}-33^{2}}{(x-33)(x-13)}=\frac{x+33}{x-13}$$

$$\lim\limits_{x\to\ 33}\frac{x^{2}-33^{2}}{(x-33)(x-13)}=[\frac{0}{0}]=\lim\limits_{x\to\ 33}\frac{x+33}{x-13}=2 \cdot \frac{33}{33-13} = \frac{66}{20}$$
\rozwStop
\odpStart
$\frac{66}{20}$
\odpStop
\testStart
A.$\frac{66}{20}$
B.$\infty$
C.$-\infty$
D.$0$
E.$\frac{66}{-20}$
F.$\frac{33}{13}$
G.$-\frac{66}{-20}$
H.$1$
I.$33$
\testStop
\kluczStart
A
\kluczStop



\zadStart{Przykład z Wikieł P 4.2b moja wersja nr 629}
Obliczyć granicę $\lim\limits_{x\to\ 33}\frac{x^{2}-33^{2}}{(x-33)(x-14)}$.
\zadStop
\rozwStart{Patryk Wirkus}{Martyna Czarnobaj}
$$\frac{x^{2}-33^{2}}{(x-33)(x-14)}=\frac{x+33}{x-14}$$

$$\lim\limits_{x\to\ 33}\frac{x^{2}-33^{2}}{(x-33)(x-14)}=[\frac{0}{0}]=\lim\limits_{x\to\ 33}\frac{x+33}{x-14}=2 \cdot \frac{33}{33-14} = \frac{66}{19}$$
\rozwStop
\odpStart
$\frac{66}{19}$
\odpStop
\testStart
A.$\frac{66}{19}$
B.$\infty$
C.$-\infty$
D.$0$
E.$\frac{66}{-19}$
F.$\frac{33}{14}$
G.$-\frac{66}{-19}$
H.$1$
I.$33$
\testStop
\kluczStart
A
\kluczStop



\zadStart{Przykład z Wikieł P 4.2b moja wersja nr 630}
Obliczyć granicę $\lim\limits_{x\to\ 33}\frac{x^{2}-33^{2}}{(x-33)(x-16)}$.
\zadStop
\rozwStart{Patryk Wirkus}{Martyna Czarnobaj}
$$\frac{x^{2}-33^{2}}{(x-33)(x-16)}=\frac{x+33}{x-16}$$

$$\lim\limits_{x\to\ 33}\frac{x^{2}-33^{2}}{(x-33)(x-16)}=[\frac{0}{0}]=\lim\limits_{x\to\ 33}\frac{x+33}{x-16}=2 \cdot \frac{33}{33-16} = \frac{66}{17}$$
\rozwStop
\odpStart
$\frac{66}{17}$
\odpStop
\testStart
A.$\frac{66}{17}$
B.$\infty$
C.$-\infty$
D.$0$
E.$\frac{66}{-17}$
F.$\frac{33}{16}$
G.$-\frac{66}{-17}$
H.$1$
I.$33$
\testStop
\kluczStart
A
\kluczStop



\zadStart{Przykład z Wikieł P 4.2b moja wersja nr 631}
Obliczyć granicę $\lim\limits_{x\to\ 33}\frac{x^{2}-33^{2}}{(x-33)(x-17)}$.
\zadStop
\rozwStart{Patryk Wirkus}{Martyna Czarnobaj}
$$\frac{x^{2}-33^{2}}{(x-33)(x-17)}=\frac{x+33}{x-17}$$

$$\lim\limits_{x\to\ 33}\frac{x^{2}-33^{2}}{(x-33)(x-17)}=[\frac{0}{0}]=\lim\limits_{x\to\ 33}\frac{x+33}{x-17}=2 \cdot \frac{33}{33-17} = \frac{66}{16}$$
\rozwStop
\odpStart
$\frac{66}{16}$
\odpStop
\testStart
A.$\frac{66}{16}$
B.$\infty$
C.$-\infty$
D.$0$
E.$\frac{66}{-16}$
F.$\frac{33}{17}$
G.$-\frac{66}{-16}$
H.$1$
I.$33$
\testStop
\kluczStart
A
\kluczStop



\zadStart{Przykład z Wikieł P 4.2b moja wersja nr 632}
Obliczyć granicę $\lim\limits_{x\to\ 33}\frac{x^{2}-33^{2}}{(x-33)(x-19)}$.
\zadStop
\rozwStart{Patryk Wirkus}{Martyna Czarnobaj}
$$\frac{x^{2}-33^{2}}{(x-33)(x-19)}=\frac{x+33}{x-19}$$

$$\lim\limits_{x\to\ 33}\frac{x^{2}-33^{2}}{(x-33)(x-19)}=[\frac{0}{0}]=\lim\limits_{x\to\ 33}\frac{x+33}{x-19}=2 \cdot \frac{33}{33-19} = \frac{66}{14}$$
\rozwStop
\odpStart
$\frac{66}{14}$
\odpStop
\testStart
A.$\frac{66}{14}$
B.$\infty$
C.$-\infty$
D.$0$
E.$\frac{66}{-14}$
F.$\frac{33}{19}$
G.$-\frac{66}{-14}$
H.$1$
I.$33$
\testStop
\kluczStart
A
\kluczStop



\zadStart{Przykład z Wikieł P 4.2b moja wersja nr 633}
Obliczyć granicę $\lim\limits_{x\to\ 33}\frac{x^{2}-33^{2}}{(x-33)(x-20)}$.
\zadStop
\rozwStart{Patryk Wirkus}{Martyna Czarnobaj}
$$\frac{x^{2}-33^{2}}{(x-33)(x-20)}=\frac{x+33}{x-20}$$

$$\lim\limits_{x\to\ 33}\frac{x^{2}-33^{2}}{(x-33)(x-20)}=[\frac{0}{0}]=\lim\limits_{x\to\ 33}\frac{x+33}{x-20}=2 \cdot \frac{33}{33-20} = \frac{66}{13}$$
\rozwStop
\odpStart
$\frac{66}{13}$
\odpStop
\testStart
A.$\frac{66}{13}$
B.$\infty$
C.$-\infty$
D.$0$
E.$\frac{66}{-13}$
F.$\frac{33}{20}$
G.$-\frac{66}{-13}$
H.$1$
I.$33$
\testStop
\kluczStart
A
\kluczStop



\zadStart{Przykład z Wikieł P 4.2b moja wersja nr 634}
Obliczyć granicę $\lim\limits_{x\to\ 33}\frac{x^{2}-33^{2}}{(x-33)(x-23)}$.
\zadStop
\rozwStart{Patryk Wirkus}{Martyna Czarnobaj}
$$\frac{x^{2}-33^{2}}{(x-33)(x-23)}=\frac{x+33}{x-23}$$

$$\lim\limits_{x\to\ 33}\frac{x^{2}-33^{2}}{(x-33)(x-23)}=[\frac{0}{0}]=\lim\limits_{x\to\ 33}\frac{x+33}{x-23}=2 \cdot \frac{33}{33-23} = \frac{66}{10}$$
\rozwStop
\odpStart
$\frac{66}{10}$
\odpStop
\testStart
A.$\frac{66}{10}$
B.$\infty$
C.$-\infty$
D.$0$
E.$\frac{66}{-10}$
F.$\frac{33}{23}$
G.$-\frac{66}{-10}$
H.$1$
I.$33$
\testStop
\kluczStart
A
\kluczStop



\zadStart{Przykład z Wikieł P 4.2b moja wersja nr 635}
Obliczyć granicę $\lim\limits_{x\to\ 33}\frac{x^{2}-33^{2}}{(x-33)(x-25)}$.
\zadStop
\rozwStart{Patryk Wirkus}{Martyna Czarnobaj}
$$\frac{x^{2}-33^{2}}{(x-33)(x-25)}=\frac{x+33}{x-25}$$

$$\lim\limits_{x\to\ 33}\frac{x^{2}-33^{2}}{(x-33)(x-25)}=[\frac{0}{0}]=\lim\limits_{x\to\ 33}\frac{x+33}{x-25}=2 \cdot \frac{33}{33-25} = \frac{66}{8}$$
\rozwStop
\odpStart
$\frac{66}{8}$
\odpStop
\testStart
A.$\frac{66}{8}$
B.$\infty$
C.$-\infty$
D.$0$
E.$\frac{66}{-8}$
F.$\frac{33}{25}$
G.$-\frac{66}{-8}$
H.$1$
I.$33$
\testStop
\kluczStart
A
\kluczStop



\zadStart{Przykład z Wikieł P 4.2b moja wersja nr 636}
Obliczyć granicę $\lim\limits_{x\to\ 33}\frac{x^{2}-33^{2}}{(x-33)(x-26)}$.
\zadStop
\rozwStart{Patryk Wirkus}{Martyna Czarnobaj}
$$\frac{x^{2}-33^{2}}{(x-33)(x-26)}=\frac{x+33}{x-26}$$

$$\lim\limits_{x\to\ 33}\frac{x^{2}-33^{2}}{(x-33)(x-26)}=[\frac{0}{0}]=\lim\limits_{x\to\ 33}\frac{x+33}{x-26}=2 \cdot \frac{33}{33-26} = \frac{66}{7}$$
\rozwStop
\odpStart
$\frac{66}{7}$
\odpStop
\testStart
A.$\frac{66}{7}$
B.$\infty$
C.$-\infty$
D.$0$
E.$\frac{66}{-7}$
F.$\frac{33}{26}$
G.$-\frac{66}{-7}$
H.$1$
I.$33$
\testStop
\kluczStart
A
\kluczStop



\zadStart{Przykład z Wikieł P 4.2b moja wersja nr 637}
Obliczyć granicę $\lim\limits_{x\to\ 33}\frac{x^{2}-33^{2}}{(x-33)(x-28)}$.
\zadStop
\rozwStart{Patryk Wirkus}{Martyna Czarnobaj}
$$\frac{x^{2}-33^{2}}{(x-33)(x-28)}=\frac{x+33}{x-28}$$

$$\lim\limits_{x\to\ 33}\frac{x^{2}-33^{2}}{(x-33)(x-28)}=[\frac{0}{0}]=\lim\limits_{x\to\ 33}\frac{x+33}{x-28}=2 \cdot \frac{33}{33-28} = \frac{66}{5}$$
\rozwStop
\odpStart
$\frac{66}{5}$
\odpStop
\testStart
A.$\frac{66}{5}$
B.$\infty$
C.$-\infty$
D.$0$
E.$\frac{66}{-5}$
F.$\frac{33}{28}$
G.$-\frac{66}{-5}$
H.$1$
I.$33$
\testStop
\kluczStart
A
\kluczStop



\zadStart{Przykład z Wikieł P 4.2b moja wersja nr 638}
Obliczyć granicę $\lim\limits_{x\to\ 33}\frac{x^{2}-33^{2}}{(x-33)(x-29)}$.
\zadStop
\rozwStart{Patryk Wirkus}{Martyna Czarnobaj}
$$\frac{x^{2}-33^{2}}{(x-33)(x-29)}=\frac{x+33}{x-29}$$

$$\lim\limits_{x\to\ 33}\frac{x^{2}-33^{2}}{(x-33)(x-29)}=[\frac{0}{0}]=\lim\limits_{x\to\ 33}\frac{x+33}{x-29}=2 \cdot \frac{33}{33-29} = \frac{66}{4}$$
\rozwStop
\odpStart
$\frac{66}{4}$
\odpStop
\testStart
A.$\frac{66}{4}$
B.$\infty$
C.$-\infty$
D.$0$
E.$\frac{66}{-4}$
F.$\frac{33}{29}$
G.$-\frac{66}{-4}$
H.$1$
I.$33$
\testStop
\kluczStart
A
\kluczStop



\zadStart{Przykład z Wikieł P 4.2b moja wersja nr 639}
Obliczyć granicę $\lim\limits_{x\to\ 33}\frac{x^{2}-33^{2}}{(x-33)(x-31)}$.
\zadStop
\rozwStart{Patryk Wirkus}{Martyna Czarnobaj}
$$\frac{x^{2}-33^{2}}{(x-33)(x-31)}=\frac{x+33}{x-31}$$

$$\lim\limits_{x\to\ 33}\frac{x^{2}-33^{2}}{(x-33)(x-31)}=[\frac{0}{0}]=\lim\limits_{x\to\ 33}\frac{x+33}{x-31}=2 \cdot \frac{33}{33-31} = \frac{66}{2}$$
\rozwStop
\odpStart
$\frac{66}{2}$
\odpStop
\testStart
A.$\frac{66}{2}$
B.$\infty$
C.$-\infty$
D.$0$
E.$\frac{66}{-2}$
F.$\frac{33}{31}$
G.$-\frac{66}{-2}$
H.$1$
I.$33$
\testStop
\kluczStart
A
\kluczStop



\zadStart{Przykład z Wikieł P 4.2b moja wersja nr 640}
Obliczyć granicę $\lim\limits_{x\to\ 33}\frac{x^{2}-33^{2}}{(x-33)(x-35)}$.
\zadStop
\rozwStart{Patryk Wirkus}{Martyna Czarnobaj}
$$\frac{x^{2}-33^{2}}{(x-33)(x-35)}=\frac{x+33}{x-35}$$

$$\lim\limits_{x\to\ 33}\frac{x^{2}-33^{2}}{(x-33)(x-35)}=[\frac{0}{0}]=\lim\limits_{x\to\ 33}\frac{x+33}{x-35}=2 \cdot \frac{33}{33-35} = \frac{66}{-2}$$
\rozwStop
\odpStart
$\frac{66}{-2}$
\odpStop
\testStart
A.$\frac{66}{-2}$
B.$\infty$
C.$-\infty$
D.$0$
E.$\frac{66}{2}$
F.$\frac{33}{35}$
G.$-\frac{66}{2}$
H.$1$
I.$33$
\testStop
\kluczStart
A
\kluczStop



\zadStart{Przykład z Wikieł P 4.2b moja wersja nr 641}
Obliczyć granicę $\lim\limits_{x\to\ 33}\frac{x^{2}-33^{2}}{(x-33)(x-37)}$.
\zadStop
\rozwStart{Patryk Wirkus}{Martyna Czarnobaj}
$$\frac{x^{2}-33^{2}}{(x-33)(x-37)}=\frac{x+33}{x-37}$$

$$\lim\limits_{x\to\ 33}\frac{x^{2}-33^{2}}{(x-33)(x-37)}=[\frac{0}{0}]=\lim\limits_{x\to\ 33}\frac{x+33}{x-37}=2 \cdot \frac{33}{33-37} = \frac{66}{-4}$$
\rozwStop
\odpStart
$\frac{66}{-4}$
\odpStop
\testStart
A.$\frac{66}{-4}$
B.$\infty$
C.$-\infty$
D.$0$
E.$\frac{66}{4}$
F.$\frac{33}{37}$
G.$-\frac{66}{4}$
H.$1$
I.$33$
\testStop
\kluczStart
A
\kluczStop



\zadStart{Przykład z Wikieł P 4.2b moja wersja nr 642}
Obliczyć granicę $\lim\limits_{x\to\ 33}\frac{x^{2}-33^{2}}{(x-33)(x-38)}$.
\zadStop
\rozwStart{Patryk Wirkus}{Martyna Czarnobaj}
$$\frac{x^{2}-33^{2}}{(x-33)(x-38)}=\frac{x+33}{x-38}$$

$$\lim\limits_{x\to\ 33}\frac{x^{2}-33^{2}}{(x-33)(x-38)}=[\frac{0}{0}]=\lim\limits_{x\to\ 33}\frac{x+33}{x-38}=2 \cdot \frac{33}{33-38} = \frac{66}{-5}$$
\rozwStop
\odpStart
$\frac{66}{-5}$
\odpStop
\testStart
A.$\frac{66}{-5}$
B.$\infty$
C.$-\infty$
D.$0$
E.$\frac{66}{5}$
F.$\frac{33}{38}$
G.$-\frac{66}{5}$
H.$1$
I.$33$
\testStop
\kluczStart
A
\kluczStop



\zadStart{Przykład z Wikieł P 4.2b moja wersja nr 643}
Obliczyć granicę $\lim\limits_{x\to\ 33}\frac{x^{2}-33^{2}}{(x-33)(x-40)}$.
\zadStop
\rozwStart{Patryk Wirkus}{Martyna Czarnobaj}
$$\frac{x^{2}-33^{2}}{(x-33)(x-40)}=\frac{x+33}{x-40}$$

$$\lim\limits_{x\to\ 33}\frac{x^{2}-33^{2}}{(x-33)(x-40)}=[\frac{0}{0}]=\lim\limits_{x\to\ 33}\frac{x+33}{x-40}=2 \cdot \frac{33}{33-40} = \frac{66}{-7}$$
\rozwStop
\odpStart
$\frac{66}{-7}$
\odpStop
\testStart
A.$\frac{66}{-7}$
B.$\infty$
C.$-\infty$
D.$0$
E.$\frac{66}{7}$
F.$\frac{33}{40}$
G.$-\frac{66}{7}$
H.$1$
I.$33$
\testStop
\kluczStart
A
\kluczStop



\zadStart{Przykład z Wikieł P 4.2b moja wersja nr 644}
Obliczyć granicę $\lim\limits_{x\to\ 34}\frac{x^{2}-34^{2}}{(x-34)(x-3)}$.
\zadStop
\rozwStart{Patryk Wirkus}{Martyna Czarnobaj}
$$\frac{x^{2}-34^{2}}{(x-34)(x-3)}=\frac{x+34}{x-3}$$

$$\lim\limits_{x\to\ 34}\frac{x^{2}-34^{2}}{(x-34)(x-3)}=[\frac{0}{0}]=\lim\limits_{x\to\ 34}\frac{x+34}{x-3}=2 \cdot \frac{34}{34-3} = \frac{68}{31}$$
\rozwStop
\odpStart
$\frac{68}{31}$
\odpStop
\testStart
A.$\frac{68}{31}$
B.$\infty$
C.$-\infty$
D.$0$
E.$\frac{68}{-31}$
F.$\frac{34}{3}$
G.$-\frac{68}{-31}$
H.$1$
I.$34$
\testStop
\kluczStart
A
\kluczStop



\zadStart{Przykład z Wikieł P 4.2b moja wersja nr 645}
Obliczyć granicę $\lim\limits_{x\to\ 34}\frac{x^{2}-34^{2}}{(x-34)(x-5)}$.
\zadStop
\rozwStart{Patryk Wirkus}{Martyna Czarnobaj}
$$\frac{x^{2}-34^{2}}{(x-34)(x-5)}=\frac{x+34}{x-5}$$

$$\lim\limits_{x\to\ 34}\frac{x^{2}-34^{2}}{(x-34)(x-5)}=[\frac{0}{0}]=\lim\limits_{x\to\ 34}\frac{x+34}{x-5}=2 \cdot \frac{34}{34-5} = \frac{68}{29}$$
\rozwStop
\odpStart
$\frac{68}{29}$
\odpStop
\testStart
A.$\frac{68}{29}$
B.$\infty$
C.$-\infty$
D.$0$
E.$\frac{68}{-29}$
F.$\frac{34}{5}$
G.$-\frac{68}{-29}$
H.$1$
I.$34$
\testStop
\kluczStart
A
\kluczStop



\zadStart{Przykład z Wikieł P 4.2b moja wersja nr 646}
Obliczyć granicę $\lim\limits_{x\to\ 34}\frac{x^{2}-34^{2}}{(x-34)(x-7)}$.
\zadStop
\rozwStart{Patryk Wirkus}{Martyna Czarnobaj}
$$\frac{x^{2}-34^{2}}{(x-34)(x-7)}=\frac{x+34}{x-7}$$

$$\lim\limits_{x\to\ 34}\frac{x^{2}-34^{2}}{(x-34)(x-7)}=[\frac{0}{0}]=\lim\limits_{x\to\ 34}\frac{x+34}{x-7}=2 \cdot \frac{34}{34-7} = \frac{68}{27}$$
\rozwStop
\odpStart
$\frac{68}{27}$
\odpStop
\testStart
A.$\frac{68}{27}$
B.$\infty$
C.$-\infty$
D.$0$
E.$\frac{68}{-27}$
F.$\frac{34}{7}$
G.$-\frac{68}{-27}$
H.$1$
I.$34$
\testStop
\kluczStart
A
\kluczStop



\zadStart{Przykład z Wikieł P 4.2b moja wersja nr 647}
Obliczyć granicę $\lim\limits_{x\to\ 34}\frac{x^{2}-34^{2}}{(x-34)(x-9)}$.
\zadStop
\rozwStart{Patryk Wirkus}{Martyna Czarnobaj}
$$\frac{x^{2}-34^{2}}{(x-34)(x-9)}=\frac{x+34}{x-9}$$

$$\lim\limits_{x\to\ 34}\frac{x^{2}-34^{2}}{(x-34)(x-9)}=[\frac{0}{0}]=\lim\limits_{x\to\ 34}\frac{x+34}{x-9}=2 \cdot \frac{34}{34-9} = \frac{68}{25}$$
\rozwStop
\odpStart
$\frac{68}{25}$
\odpStop
\testStart
A.$\frac{68}{25}$
B.$\infty$
C.$-\infty$
D.$0$
E.$\frac{68}{-25}$
F.$\frac{34}{9}$
G.$-\frac{68}{-25}$
H.$1$
I.$34$
\testStop
\kluczStart
A
\kluczStop



\zadStart{Przykład z Wikieł P 4.2b moja wersja nr 648}
Obliczyć granicę $\lim\limits_{x\to\ 34}\frac{x^{2}-34^{2}}{(x-34)(x-11)}$.
\zadStop
\rozwStart{Patryk Wirkus}{Martyna Czarnobaj}
$$\frac{x^{2}-34^{2}}{(x-34)(x-11)}=\frac{x+34}{x-11}$$

$$\lim\limits_{x\to\ 34}\frac{x^{2}-34^{2}}{(x-34)(x-11)}=[\frac{0}{0}]=\lim\limits_{x\to\ 34}\frac{x+34}{x-11}=2 \cdot \frac{34}{34-11} = \frac{68}{23}$$
\rozwStop
\odpStart
$\frac{68}{23}$
\odpStop
\testStart
A.$\frac{68}{23}$
B.$\infty$
C.$-\infty$
D.$0$
E.$\frac{68}{-23}$
F.$\frac{34}{11}$
G.$-\frac{68}{-23}$
H.$1$
I.$34$
\testStop
\kluczStart
A
\kluczStop



\zadStart{Przykład z Wikieł P 4.2b moja wersja nr 649}
Obliczyć granicę $\lim\limits_{x\to\ 34}\frac{x^{2}-34^{2}}{(x-34)(x-13)}$.
\zadStop
\rozwStart{Patryk Wirkus}{Martyna Czarnobaj}
$$\frac{x^{2}-34^{2}}{(x-34)(x-13)}=\frac{x+34}{x-13}$$

$$\lim\limits_{x\to\ 34}\frac{x^{2}-34^{2}}{(x-34)(x-13)}=[\frac{0}{0}]=\lim\limits_{x\to\ 34}\frac{x+34}{x-13}=2 \cdot \frac{34}{34-13} = \frac{68}{21}$$
\rozwStop
\odpStart
$\frac{68}{21}$
\odpStop
\testStart
A.$\frac{68}{21}$
B.$\infty$
C.$-\infty$
D.$0$
E.$\frac{68}{-21}$
F.$\frac{34}{13}$
G.$-\frac{68}{-21}$
H.$1$
I.$34$
\testStop
\kluczStart
A
\kluczStop



\zadStart{Przykład z Wikieł P 4.2b moja wersja nr 650}
Obliczyć granicę $\lim\limits_{x\to\ 34}\frac{x^{2}-34^{2}}{(x-34)(x-15)}$.
\zadStop
\rozwStart{Patryk Wirkus}{Martyna Czarnobaj}
$$\frac{x^{2}-34^{2}}{(x-34)(x-15)}=\frac{x+34}{x-15}$$

$$\lim\limits_{x\to\ 34}\frac{x^{2}-34^{2}}{(x-34)(x-15)}=[\frac{0}{0}]=\lim\limits_{x\to\ 34}\frac{x+34}{x-15}=2 \cdot \frac{34}{34-15} = \frac{68}{19}$$
\rozwStop
\odpStart
$\frac{68}{19}$
\odpStop
\testStart
A.$\frac{68}{19}$
B.$\infty$
C.$-\infty$
D.$0$
E.$\frac{68}{-19}$
F.$\frac{34}{15}$
G.$-\frac{68}{-19}$
H.$1$
I.$34$
\testStop
\kluczStart
A
\kluczStop



\zadStart{Przykład z Wikieł P 4.2b moja wersja nr 651}
Obliczyć granicę $\lim\limits_{x\to\ 34}\frac{x^{2}-34^{2}}{(x-34)(x-19)}$.
\zadStop
\rozwStart{Patryk Wirkus}{Martyna Czarnobaj}
$$\frac{x^{2}-34^{2}}{(x-34)(x-19)}=\frac{x+34}{x-19}$$

$$\lim\limits_{x\to\ 34}\frac{x^{2}-34^{2}}{(x-34)(x-19)}=[\frac{0}{0}]=\lim\limits_{x\to\ 34}\frac{x+34}{x-19}=2 \cdot \frac{34}{34-19} = \frac{68}{15}$$
\rozwStop
\odpStart
$\frac{68}{15}$
\odpStop
\testStart
A.$\frac{68}{15}$
B.$\infty$
C.$-\infty$
D.$0$
E.$\frac{68}{-15}$
F.$\frac{34}{19}$
G.$-\frac{68}{-15}$
H.$1$
I.$34$
\testStop
\kluczStart
A
\kluczStop



\zadStart{Przykład z Wikieł P 4.2b moja wersja nr 652}
Obliczyć granicę $\lim\limits_{x\to\ 34}\frac{x^{2}-34^{2}}{(x-34)(x-21)}$.
\zadStop
\rozwStart{Patryk Wirkus}{Martyna Czarnobaj}
$$\frac{x^{2}-34^{2}}{(x-34)(x-21)}=\frac{x+34}{x-21}$$

$$\lim\limits_{x\to\ 34}\frac{x^{2}-34^{2}}{(x-34)(x-21)}=[\frac{0}{0}]=\lim\limits_{x\to\ 34}\frac{x+34}{x-21}=2 \cdot \frac{34}{34-21} = \frac{68}{13}$$
\rozwStop
\odpStart
$\frac{68}{13}$
\odpStop
\testStart
A.$\frac{68}{13}$
B.$\infty$
C.$-\infty$
D.$0$
E.$\frac{68}{-13}$
F.$\frac{34}{21}$
G.$-\frac{68}{-13}$
H.$1$
I.$34$
\testStop
\kluczStart
A
\kluczStop



\zadStart{Przykład z Wikieł P 4.2b moja wersja nr 653}
Obliczyć granicę $\lim\limits_{x\to\ 34}\frac{x^{2}-34^{2}}{(x-34)(x-23)}$.
\zadStop
\rozwStart{Patryk Wirkus}{Martyna Czarnobaj}
$$\frac{x^{2}-34^{2}}{(x-34)(x-23)}=\frac{x+34}{x-23}$$

$$\lim\limits_{x\to\ 34}\frac{x^{2}-34^{2}}{(x-34)(x-23)}=[\frac{0}{0}]=\lim\limits_{x\to\ 34}\frac{x+34}{x-23}=2 \cdot \frac{34}{34-23} = \frac{68}{11}$$
\rozwStop
\odpStart
$\frac{68}{11}$
\odpStop
\testStart
A.$\frac{68}{11}$
B.$\infty$
C.$-\infty$
D.$0$
E.$\frac{68}{-11}$
F.$\frac{34}{23}$
G.$-\frac{68}{-11}$
H.$1$
I.$34$
\testStop
\kluczStart
A
\kluczStop



\zadStart{Przykład z Wikieł P 4.2b moja wersja nr 654}
Obliczyć granicę $\lim\limits_{x\to\ 34}\frac{x^{2}-34^{2}}{(x-34)(x-25)}$.
\zadStop
\rozwStart{Patryk Wirkus}{Martyna Czarnobaj}
$$\frac{x^{2}-34^{2}}{(x-34)(x-25)}=\frac{x+34}{x-25}$$

$$\lim\limits_{x\to\ 34}\frac{x^{2}-34^{2}}{(x-34)(x-25)}=[\frac{0}{0}]=\lim\limits_{x\to\ 34}\frac{x+34}{x-25}=2 \cdot \frac{34}{34-25} = \frac{68}{9}$$
\rozwStop
\odpStart
$\frac{68}{9}$
\odpStop
\testStart
A.$\frac{68}{9}$
B.$\infty$
C.$-\infty$
D.$0$
E.$\frac{68}{-9}$
F.$\frac{34}{25}$
G.$-\frac{68}{-9}$
H.$1$
I.$34$
\testStop
\kluczStart
A
\kluczStop



\zadStart{Przykład z Wikieł P 4.2b moja wersja nr 655}
Obliczyć granicę $\lim\limits_{x\to\ 34}\frac{x^{2}-34^{2}}{(x-34)(x-27)}$.
\zadStop
\rozwStart{Patryk Wirkus}{Martyna Czarnobaj}
$$\frac{x^{2}-34^{2}}{(x-34)(x-27)}=\frac{x+34}{x-27}$$

$$\lim\limits_{x\to\ 34}\frac{x^{2}-34^{2}}{(x-34)(x-27)}=[\frac{0}{0}]=\lim\limits_{x\to\ 34}\frac{x+34}{x-27}=2 \cdot \frac{34}{34-27} = \frac{68}{7}$$
\rozwStop
\odpStart
$\frac{68}{7}$
\odpStop
\testStart
A.$\frac{68}{7}$
B.$\infty$
C.$-\infty$
D.$0$
E.$\frac{68}{-7}$
F.$\frac{34}{27}$
G.$-\frac{68}{-7}$
H.$1$
I.$34$
\testStop
\kluczStart
A
\kluczStop



\zadStart{Przykład z Wikieł P 4.2b moja wersja nr 656}
Obliczyć granicę $\lim\limits_{x\to\ 34}\frac{x^{2}-34^{2}}{(x-34)(x-29)}$.
\zadStop
\rozwStart{Patryk Wirkus}{Martyna Czarnobaj}
$$\frac{x^{2}-34^{2}}{(x-34)(x-29)}=\frac{x+34}{x-29}$$

$$\lim\limits_{x\to\ 34}\frac{x^{2}-34^{2}}{(x-34)(x-29)}=[\frac{0}{0}]=\lim\limits_{x\to\ 34}\frac{x+34}{x-29}=2 \cdot \frac{34}{34-29} = \frac{68}{5}$$
\rozwStop
\odpStart
$\frac{68}{5}$
\odpStop
\testStart
A.$\frac{68}{5}$
B.$\infty$
C.$-\infty$
D.$0$
E.$\frac{68}{-5}$
F.$\frac{34}{29}$
G.$-\frac{68}{-5}$
H.$1$
I.$34$
\testStop
\kluczStart
A
\kluczStop



\zadStart{Przykład z Wikieł P 4.2b moja wersja nr 657}
Obliczyć granicę $\lim\limits_{x\to\ 34}\frac{x^{2}-34^{2}}{(x-34)(x-31)}$.
\zadStop
\rozwStart{Patryk Wirkus}{Martyna Czarnobaj}
$$\frac{x^{2}-34^{2}}{(x-34)(x-31)}=\frac{x+34}{x-31}$$

$$\lim\limits_{x\to\ 34}\frac{x^{2}-34^{2}}{(x-34)(x-31)}=[\frac{0}{0}]=\lim\limits_{x\to\ 34}\frac{x+34}{x-31}=2 \cdot \frac{34}{34-31} = \frac{68}{3}$$
\rozwStop
\odpStart
$\frac{68}{3}$
\odpStop
\testStart
A.$\frac{68}{3}$
B.$\infty$
C.$-\infty$
D.$0$
E.$\frac{68}{-3}$
F.$\frac{34}{31}$
G.$-\frac{68}{-3}$
H.$1$
I.$34$
\testStop
\kluczStart
A
\kluczStop



\zadStart{Przykład z Wikieł P 4.2b moja wersja nr 658}
Obliczyć granicę $\lim\limits_{x\to\ 34}\frac{x^{2}-34^{2}}{(x-34)(x-37)}$.
\zadStop
\rozwStart{Patryk Wirkus}{Martyna Czarnobaj}
$$\frac{x^{2}-34^{2}}{(x-34)(x-37)}=\frac{x+34}{x-37}$$

$$\lim\limits_{x\to\ 34}\frac{x^{2}-34^{2}}{(x-34)(x-37)}=[\frac{0}{0}]=\lim\limits_{x\to\ 34}\frac{x+34}{x-37}=2 \cdot \frac{34}{34-37} = \frac{68}{-3}$$
\rozwStop
\odpStart
$\frac{68}{-3}$
\odpStop
\testStart
A.$\frac{68}{-3}$
B.$\infty$
C.$-\infty$
D.$0$
E.$\frac{68}{3}$
F.$\frac{34}{37}$
G.$-\frac{68}{3}$
H.$1$
I.$34$
\testStop
\kluczStart
A
\kluczStop



\zadStart{Przykład z Wikieł P 4.2b moja wersja nr 659}
Obliczyć granicę $\lim\limits_{x\to\ 34}\frac{x^{2}-34^{2}}{(x-34)(x-39)}$.
\zadStop
\rozwStart{Patryk Wirkus}{Martyna Czarnobaj}
$$\frac{x^{2}-34^{2}}{(x-34)(x-39)}=\frac{x+34}{x-39}$$

$$\lim\limits_{x\to\ 34}\frac{x^{2}-34^{2}}{(x-34)(x-39)}=[\frac{0}{0}]=\lim\limits_{x\to\ 34}\frac{x+34}{x-39}=2 \cdot \frac{34}{34-39} = \frac{68}{-5}$$
\rozwStop
\odpStart
$\frac{68}{-5}$
\odpStop
\testStart
A.$\frac{68}{-5}$
B.$\infty$
C.$-\infty$
D.$0$
E.$\frac{68}{5}$
F.$\frac{34}{39}$
G.$-\frac{68}{5}$
H.$1$
I.$34$
\testStop
\kluczStart
A
\kluczStop



\zadStart{Przykład z Wikieł P 4.2b moja wersja nr 660}
Obliczyć granicę $\lim\limits_{x\to\ 35}\frac{x^{2}-35^{2}}{(x-35)(x-2)}$.
\zadStop
\rozwStart{Patryk Wirkus}{Martyna Czarnobaj}
$$\frac{x^{2}-35^{2}}{(x-35)(x-2)}=\frac{x+35}{x-2}$$

$$\lim\limits_{x\to\ 35}\frac{x^{2}-35^{2}}{(x-35)(x-2)}=[\frac{0}{0}]=\lim\limits_{x\to\ 35}\frac{x+35}{x-2}=2 \cdot \frac{35}{35-2} = \frac{70}{33}$$
\rozwStop
\odpStart
$\frac{70}{33}$
\odpStop
\testStart
A.$\frac{70}{33}$
B.$\infty$
C.$-\infty$
D.$0$
E.$\frac{70}{-33}$
F.$\frac{35}{2}$
G.$-\frac{70}{-33}$
H.$1$
I.$35$
\testStop
\kluczStart
A
\kluczStop



\zadStart{Przykład z Wikieł P 4.2b moja wersja nr 661}
Obliczyć granicę $\lim\limits_{x\to\ 35}\frac{x^{2}-35^{2}}{(x-35)(x-3)}$.
\zadStop
\rozwStart{Patryk Wirkus}{Martyna Czarnobaj}
$$\frac{x^{2}-35^{2}}{(x-35)(x-3)}=\frac{x+35}{x-3}$$

$$\lim\limits_{x\to\ 35}\frac{x^{2}-35^{2}}{(x-35)(x-3)}=[\frac{0}{0}]=\lim\limits_{x\to\ 35}\frac{x+35}{x-3}=2 \cdot \frac{35}{35-3} = \frac{70}{32}$$
\rozwStop
\odpStart
$\frac{70}{32}$
\odpStop
\testStart
A.$\frac{70}{32}$
B.$\infty$
C.$-\infty$
D.$0$
E.$\frac{70}{-32}$
F.$\frac{35}{3}$
G.$-\frac{70}{-32}$
H.$1$
I.$35$
\testStop
\kluczStart
A
\kluczStop



\zadStart{Przykład z Wikieł P 4.2b moja wersja nr 662}
Obliczyć granicę $\lim\limits_{x\to\ 35}\frac{x^{2}-35^{2}}{(x-35)(x-4)}$.
\zadStop
\rozwStart{Patryk Wirkus}{Martyna Czarnobaj}
$$\frac{x^{2}-35^{2}}{(x-35)(x-4)}=\frac{x+35}{x-4}$$

$$\lim\limits_{x\to\ 35}\frac{x^{2}-35^{2}}{(x-35)(x-4)}=[\frac{0}{0}]=\lim\limits_{x\to\ 35}\frac{x+35}{x-4}=2 \cdot \frac{35}{35-4} = \frac{70}{31}$$
\rozwStop
\odpStart
$\frac{70}{31}$
\odpStop
\testStart
A.$\frac{70}{31}$
B.$\infty$
C.$-\infty$
D.$0$
E.$\frac{70}{-31}$
F.$\frac{35}{4}$
G.$-\frac{70}{-31}$
H.$1$
I.$35$
\testStop
\kluczStart
A
\kluczStop



\zadStart{Przykład z Wikieł P 4.2b moja wersja nr 663}
Obliczyć granicę $\lim\limits_{x\to\ 35}\frac{x^{2}-35^{2}}{(x-35)(x-6)}$.
\zadStop
\rozwStart{Patryk Wirkus}{Martyna Czarnobaj}
$$\frac{x^{2}-35^{2}}{(x-35)(x-6)}=\frac{x+35}{x-6}$$

$$\lim\limits_{x\to\ 35}\frac{x^{2}-35^{2}}{(x-35)(x-6)}=[\frac{0}{0}]=\lim\limits_{x\to\ 35}\frac{x+35}{x-6}=2 \cdot \frac{35}{35-6} = \frac{70}{29}$$
\rozwStop
\odpStart
$\frac{70}{29}$
\odpStop
\testStart
A.$\frac{70}{29}$
B.$\infty$
C.$-\infty$
D.$0$
E.$\frac{70}{-29}$
F.$\frac{35}{6}$
G.$-\frac{70}{-29}$
H.$1$
I.$35$
\testStop
\kluczStart
A
\kluczStop



\zadStart{Przykład z Wikieł P 4.2b moja wersja nr 664}
Obliczyć granicę $\lim\limits_{x\to\ 35}\frac{x^{2}-35^{2}}{(x-35)(x-8)}$.
\zadStop
\rozwStart{Patryk Wirkus}{Martyna Czarnobaj}
$$\frac{x^{2}-35^{2}}{(x-35)(x-8)}=\frac{x+35}{x-8}$$

$$\lim\limits_{x\to\ 35}\frac{x^{2}-35^{2}}{(x-35)(x-8)}=[\frac{0}{0}]=\lim\limits_{x\to\ 35}\frac{x+35}{x-8}=2 \cdot \frac{35}{35-8} = \frac{70}{27}$$
\rozwStop
\odpStart
$\frac{70}{27}$
\odpStop
\testStart
A.$\frac{70}{27}$
B.$\infty$
C.$-\infty$
D.$0$
E.$\frac{70}{-27}$
F.$\frac{35}{8}$
G.$-\frac{70}{-27}$
H.$1$
I.$35$
\testStop
\kluczStart
A
\kluczStop



\zadStart{Przykład z Wikieł P 4.2b moja wersja nr 665}
Obliczyć granicę $\lim\limits_{x\to\ 35}\frac{x^{2}-35^{2}}{(x-35)(x-9)}$.
\zadStop
\rozwStart{Patryk Wirkus}{Martyna Czarnobaj}
$$\frac{x^{2}-35^{2}}{(x-35)(x-9)}=\frac{x+35}{x-9}$$

$$\lim\limits_{x\to\ 35}\frac{x^{2}-35^{2}}{(x-35)(x-9)}=[\frac{0}{0}]=\lim\limits_{x\to\ 35}\frac{x+35}{x-9}=2 \cdot \frac{35}{35-9} = \frac{70}{26}$$
\rozwStop
\odpStart
$\frac{70}{26}$
\odpStop
\testStart
A.$\frac{70}{26}$
B.$\infty$
C.$-\infty$
D.$0$
E.$\frac{70}{-26}$
F.$\frac{35}{9}$
G.$-\frac{70}{-26}$
H.$1$
I.$35$
\testStop
\kluczStart
A
\kluczStop



\zadStart{Przykład z Wikieł P 4.2b moja wersja nr 666}
Obliczyć granicę $\lim\limits_{x\to\ 35}\frac{x^{2}-35^{2}}{(x-35)(x-11)}$.
\zadStop
\rozwStart{Patryk Wirkus}{Martyna Czarnobaj}
$$\frac{x^{2}-35^{2}}{(x-35)(x-11)}=\frac{x+35}{x-11}$$

$$\lim\limits_{x\to\ 35}\frac{x^{2}-35^{2}}{(x-35)(x-11)}=[\frac{0}{0}]=\lim\limits_{x\to\ 35}\frac{x+35}{x-11}=2 \cdot \frac{35}{35-11} = \frac{70}{24}$$
\rozwStop
\odpStart
$\frac{70}{24}$
\odpStop
\testStart
A.$\frac{70}{24}$
B.$\infty$
C.$-\infty$
D.$0$
E.$\frac{70}{-24}$
F.$\frac{35}{11}$
G.$-\frac{70}{-24}$
H.$1$
I.$35$
\testStop
\kluczStart
A
\kluczStop



\zadStart{Przykład z Wikieł P 4.2b moja wersja nr 667}
Obliczyć granicę $\lim\limits_{x\to\ 35}\frac{x^{2}-35^{2}}{(x-35)(x-12)}$.
\zadStop
\rozwStart{Patryk Wirkus}{Martyna Czarnobaj}
$$\frac{x^{2}-35^{2}}{(x-35)(x-12)}=\frac{x+35}{x-12}$$

$$\lim\limits_{x\to\ 35}\frac{x^{2}-35^{2}}{(x-35)(x-12)}=[\frac{0}{0}]=\lim\limits_{x\to\ 35}\frac{x+35}{x-12}=2 \cdot \frac{35}{35-12} = \frac{70}{23}$$
\rozwStop
\odpStart
$\frac{70}{23}$
\odpStop
\testStart
A.$\frac{70}{23}$
B.$\infty$
C.$-\infty$
D.$0$
E.$\frac{70}{-23}$
F.$\frac{35}{12}$
G.$-\frac{70}{-23}$
H.$1$
I.$35$
\testStop
\kluczStart
A
\kluczStop



\zadStart{Przykład z Wikieł P 4.2b moja wersja nr 668}
Obliczyć granicę $\lim\limits_{x\to\ 35}\frac{x^{2}-35^{2}}{(x-35)(x-13)}$.
\zadStop
\rozwStart{Patryk Wirkus}{Martyna Czarnobaj}
$$\frac{x^{2}-35^{2}}{(x-35)(x-13)}=\frac{x+35}{x-13}$$

$$\lim\limits_{x\to\ 35}\frac{x^{2}-35^{2}}{(x-35)(x-13)}=[\frac{0}{0}]=\lim\limits_{x\to\ 35}\frac{x+35}{x-13}=2 \cdot \frac{35}{35-13} = \frac{70}{22}$$
\rozwStop
\odpStart
$\frac{70}{22}$
\odpStop
\testStart
A.$\frac{70}{22}$
B.$\infty$
C.$-\infty$
D.$0$
E.$\frac{70}{-22}$
F.$\frac{35}{13}$
G.$-\frac{70}{-22}$
H.$1$
I.$35$
\testStop
\kluczStart
A
\kluczStop



\zadStart{Przykład z Wikieł P 4.2b moja wersja nr 669}
Obliczyć granicę $\lim\limits_{x\to\ 35}\frac{x^{2}-35^{2}}{(x-35)(x-16)}$.
\zadStop
\rozwStart{Patryk Wirkus}{Martyna Czarnobaj}
$$\frac{x^{2}-35^{2}}{(x-35)(x-16)}=\frac{x+35}{x-16}$$

$$\lim\limits_{x\to\ 35}\frac{x^{2}-35^{2}}{(x-35)(x-16)}=[\frac{0}{0}]=\lim\limits_{x\to\ 35}\frac{x+35}{x-16}=2 \cdot \frac{35}{35-16} = \frac{70}{19}$$
\rozwStop
\odpStart
$\frac{70}{19}$
\odpStop
\testStart
A.$\frac{70}{19}$
B.$\infty$
C.$-\infty$
D.$0$
E.$\frac{70}{-19}$
F.$\frac{35}{16}$
G.$-\frac{70}{-19}$
H.$1$
I.$35$
\testStop
\kluczStart
A
\kluczStop



\zadStart{Przykład z Wikieł P 4.2b moja wersja nr 670}
Obliczyć granicę $\lim\limits_{x\to\ 35}\frac{x^{2}-35^{2}}{(x-35)(x-17)}$.
\zadStop
\rozwStart{Patryk Wirkus}{Martyna Czarnobaj}
$$\frac{x^{2}-35^{2}}{(x-35)(x-17)}=\frac{x+35}{x-17}$$

$$\lim\limits_{x\to\ 35}\frac{x^{2}-35^{2}}{(x-35)(x-17)}=[\frac{0}{0}]=\lim\limits_{x\to\ 35}\frac{x+35}{x-17}=2 \cdot \frac{35}{35-17} = \frac{70}{18}$$
\rozwStop
\odpStart
$\frac{70}{18}$
\odpStop
\testStart
A.$\frac{70}{18}$
B.$\infty$
C.$-\infty$
D.$0$
E.$\frac{70}{-18}$
F.$\frac{35}{17}$
G.$-\frac{70}{-18}$
H.$1$
I.$35$
\testStop
\kluczStart
A
\kluczStop



\zadStart{Przykład z Wikieł P 4.2b moja wersja nr 671}
Obliczyć granicę $\lim\limits_{x\to\ 35}\frac{x^{2}-35^{2}}{(x-35)(x-18)}$.
\zadStop
\rozwStart{Patryk Wirkus}{Martyna Czarnobaj}
$$\frac{x^{2}-35^{2}}{(x-35)(x-18)}=\frac{x+35}{x-18}$$

$$\lim\limits_{x\to\ 35}\frac{x^{2}-35^{2}}{(x-35)(x-18)}=[\frac{0}{0}]=\lim\limits_{x\to\ 35}\frac{x+35}{x-18}=2 \cdot \frac{35}{35-18} = \frac{70}{17}$$
\rozwStop
\odpStart
$\frac{70}{17}$
\odpStop
\testStart
A.$\frac{70}{17}$
B.$\infty$
C.$-\infty$
D.$0$
E.$\frac{70}{-17}$
F.$\frac{35}{18}$
G.$-\frac{70}{-17}$
H.$1$
I.$35$
\testStop
\kluczStart
A
\kluczStop



\zadStart{Przykład z Wikieł P 4.2b moja wersja nr 672}
Obliczyć granicę $\lim\limits_{x\to\ 35}\frac{x^{2}-35^{2}}{(x-35)(x-19)}$.
\zadStop
\rozwStart{Patryk Wirkus}{Martyna Czarnobaj}
$$\frac{x^{2}-35^{2}}{(x-35)(x-19)}=\frac{x+35}{x-19}$$

$$\lim\limits_{x\to\ 35}\frac{x^{2}-35^{2}}{(x-35)(x-19)}=[\frac{0}{0}]=\lim\limits_{x\to\ 35}\frac{x+35}{x-19}=2 \cdot \frac{35}{35-19} = \frac{70}{16}$$
\rozwStop
\odpStart
$\frac{70}{16}$
\odpStop
\testStart
A.$\frac{70}{16}$
B.$\infty$
C.$-\infty$
D.$0$
E.$\frac{70}{-16}$
F.$\frac{35}{19}$
G.$-\frac{70}{-16}$
H.$1$
I.$35$
\testStop
\kluczStart
A
\kluczStop



\zadStart{Przykład z Wikieł P 4.2b moja wersja nr 673}
Obliczyć granicę $\lim\limits_{x\to\ 35}\frac{x^{2}-35^{2}}{(x-35)(x-22)}$.
\zadStop
\rozwStart{Patryk Wirkus}{Martyna Czarnobaj}
$$\frac{x^{2}-35^{2}}{(x-35)(x-22)}=\frac{x+35}{x-22}$$

$$\lim\limits_{x\to\ 35}\frac{x^{2}-35^{2}}{(x-35)(x-22)}=[\frac{0}{0}]=\lim\limits_{x\to\ 35}\frac{x+35}{x-22}=2 \cdot \frac{35}{35-22} = \frac{70}{13}$$
\rozwStop
\odpStart
$\frac{70}{13}$
\odpStop
\testStart
A.$\frac{70}{13}$
B.$\infty$
C.$-\infty$
D.$0$
E.$\frac{70}{-13}$
F.$\frac{35}{22}$
G.$-\frac{70}{-13}$
H.$1$
I.$35$
\testStop
\kluczStart
A
\kluczStop



\zadStart{Przykład z Wikieł P 4.2b moja wersja nr 674}
Obliczyć granicę $\lim\limits_{x\to\ 35}\frac{x^{2}-35^{2}}{(x-35)(x-23)}$.
\zadStop
\rozwStart{Patryk Wirkus}{Martyna Czarnobaj}
$$\frac{x^{2}-35^{2}}{(x-35)(x-23)}=\frac{x+35}{x-23}$$

$$\lim\limits_{x\to\ 35}\frac{x^{2}-35^{2}}{(x-35)(x-23)}=[\frac{0}{0}]=\lim\limits_{x\to\ 35}\frac{x+35}{x-23}=2 \cdot \frac{35}{35-23} = \frac{70}{12}$$
\rozwStop
\odpStart
$\frac{70}{12}$
\odpStop
\testStart
A.$\frac{70}{12}$
B.$\infty$
C.$-\infty$
D.$0$
E.$\frac{70}{-12}$
F.$\frac{35}{23}$
G.$-\frac{70}{-12}$
H.$1$
I.$35$
\testStop
\kluczStart
A
\kluczStop



\zadStart{Przykład z Wikieł P 4.2b moja wersja nr 675}
Obliczyć granicę $\lim\limits_{x\to\ 35}\frac{x^{2}-35^{2}}{(x-35)(x-24)}$.
\zadStop
\rozwStart{Patryk Wirkus}{Martyna Czarnobaj}
$$\frac{x^{2}-35^{2}}{(x-35)(x-24)}=\frac{x+35}{x-24}$$

$$\lim\limits_{x\to\ 35}\frac{x^{2}-35^{2}}{(x-35)(x-24)}=[\frac{0}{0}]=\lim\limits_{x\to\ 35}\frac{x+35}{x-24}=2 \cdot \frac{35}{35-24} = \frac{70}{11}$$
\rozwStop
\odpStart
$\frac{70}{11}$
\odpStop
\testStart
A.$\frac{70}{11}$
B.$\infty$
C.$-\infty$
D.$0$
E.$\frac{70}{-11}$
F.$\frac{35}{24}$
G.$-\frac{70}{-11}$
H.$1$
I.$35$
\testStop
\kluczStart
A
\kluczStop



\zadStart{Przykład z Wikieł P 4.2b moja wersja nr 676}
Obliczyć granicę $\lim\limits_{x\to\ 35}\frac{x^{2}-35^{2}}{(x-35)(x-26)}$.
\zadStop
\rozwStart{Patryk Wirkus}{Martyna Czarnobaj}
$$\frac{x^{2}-35^{2}}{(x-35)(x-26)}=\frac{x+35}{x-26}$$

$$\lim\limits_{x\to\ 35}\frac{x^{2}-35^{2}}{(x-35)(x-26)}=[\frac{0}{0}]=\lim\limits_{x\to\ 35}\frac{x+35}{x-26}=2 \cdot \frac{35}{35-26} = \frac{70}{9}$$
\rozwStop
\odpStart
$\frac{70}{9}$
\odpStop
\testStart
A.$\frac{70}{9}$
B.$\infty$
C.$-\infty$
D.$0$
E.$\frac{70}{-9}$
F.$\frac{35}{26}$
G.$-\frac{70}{-9}$
H.$1$
I.$35$
\testStop
\kluczStart
A
\kluczStop



\zadStart{Przykład z Wikieł P 4.2b moja wersja nr 677}
Obliczyć granicę $\lim\limits_{x\to\ 35}\frac{x^{2}-35^{2}}{(x-35)(x-27)}$.
\zadStop
\rozwStart{Patryk Wirkus}{Martyna Czarnobaj}
$$\frac{x^{2}-35^{2}}{(x-35)(x-27)}=\frac{x+35}{x-27}$$

$$\lim\limits_{x\to\ 35}\frac{x^{2}-35^{2}}{(x-35)(x-27)}=[\frac{0}{0}]=\lim\limits_{x\to\ 35}\frac{x+35}{x-27}=2 \cdot \frac{35}{35-27} = \frac{70}{8}$$
\rozwStop
\odpStart
$\frac{70}{8}$
\odpStop
\testStart
A.$\frac{70}{8}$
B.$\infty$
C.$-\infty$
D.$0$
E.$\frac{70}{-8}$
F.$\frac{35}{27}$
G.$-\frac{70}{-8}$
H.$1$
I.$35$
\testStop
\kluczStart
A
\kluczStop



\zadStart{Przykład z Wikieł P 4.2b moja wersja nr 678}
Obliczyć granicę $\lim\limits_{x\to\ 35}\frac{x^{2}-35^{2}}{(x-35)(x-29)}$.
\zadStop
\rozwStart{Patryk Wirkus}{Martyna Czarnobaj}
$$\frac{x^{2}-35^{2}}{(x-35)(x-29)}=\frac{x+35}{x-29}$$

$$\lim\limits_{x\to\ 35}\frac{x^{2}-35^{2}}{(x-35)(x-29)}=[\frac{0}{0}]=\lim\limits_{x\to\ 35}\frac{x+35}{x-29}=2 \cdot \frac{35}{35-29} = \frac{70}{6}$$
\rozwStop
\odpStart
$\frac{70}{6}$
\odpStop
\testStart
A.$\frac{70}{6}$
B.$\infty$
C.$-\infty$
D.$0$
E.$\frac{70}{-6}$
F.$\frac{35}{29}$
G.$-\frac{70}{-6}$
H.$1$
I.$35$
\testStop
\kluczStart
A
\kluczStop



\zadStart{Przykład z Wikieł P 4.2b moja wersja nr 679}
Obliczyć granicę $\lim\limits_{x\to\ 35}\frac{x^{2}-35^{2}}{(x-35)(x-31)}$.
\zadStop
\rozwStart{Patryk Wirkus}{Martyna Czarnobaj}
$$\frac{x^{2}-35^{2}}{(x-35)(x-31)}=\frac{x+35}{x-31}$$

$$\lim\limits_{x\to\ 35}\frac{x^{2}-35^{2}}{(x-35)(x-31)}=[\frac{0}{0}]=\lim\limits_{x\to\ 35}\frac{x+35}{x-31}=2 \cdot \frac{35}{35-31} = \frac{70}{4}$$
\rozwStop
\odpStart
$\frac{70}{4}$
\odpStop
\testStart
A.$\frac{70}{4}$
B.$\infty$
C.$-\infty$
D.$0$
E.$\frac{70}{-4}$
F.$\frac{35}{31}$
G.$-\frac{70}{-4}$
H.$1$
I.$35$
\testStop
\kluczStart
A
\kluczStop



\zadStart{Przykład z Wikieł P 4.2b moja wersja nr 680}
Obliczyć granicę $\lim\limits_{x\to\ 35}\frac{x^{2}-35^{2}}{(x-35)(x-32)}$.
\zadStop
\rozwStart{Patryk Wirkus}{Martyna Czarnobaj}
$$\frac{x^{2}-35^{2}}{(x-35)(x-32)}=\frac{x+35}{x-32}$$

$$\lim\limits_{x\to\ 35}\frac{x^{2}-35^{2}}{(x-35)(x-32)}=[\frac{0}{0}]=\lim\limits_{x\to\ 35}\frac{x+35}{x-32}=2 \cdot \frac{35}{35-32} = \frac{70}{3}$$
\rozwStop
\odpStart
$\frac{70}{3}$
\odpStop
\testStart
A.$\frac{70}{3}$
B.$\infty$
C.$-\infty$
D.$0$
E.$\frac{70}{-3}$
F.$\frac{35}{32}$
G.$-\frac{70}{-3}$
H.$1$
I.$35$
\testStop
\kluczStart
A
\kluczStop



\zadStart{Przykład z Wikieł P 4.2b moja wersja nr 681}
Obliczyć granicę $\lim\limits_{x\to\ 35}\frac{x^{2}-35^{2}}{(x-35)(x-33)}$.
\zadStop
\rozwStart{Patryk Wirkus}{Martyna Czarnobaj}
$$\frac{x^{2}-35^{2}}{(x-35)(x-33)}=\frac{x+35}{x-33}$$

$$\lim\limits_{x\to\ 35}\frac{x^{2}-35^{2}}{(x-35)(x-33)}=[\frac{0}{0}]=\lim\limits_{x\to\ 35}\frac{x+35}{x-33}=2 \cdot \frac{35}{35-33} = \frac{70}{2}$$
\rozwStop
\odpStart
$\frac{70}{2}$
\odpStop
\testStart
A.$\frac{70}{2}$
B.$\infty$
C.$-\infty$
D.$0$
E.$\frac{70}{-2}$
F.$\frac{35}{33}$
G.$-\frac{70}{-2}$
H.$1$
I.$35$
\testStop
\kluczStart
A
\kluczStop



\zadStart{Przykład z Wikieł P 4.2b moja wersja nr 682}
Obliczyć granicę $\lim\limits_{x\to\ 35}\frac{x^{2}-35^{2}}{(x-35)(x-37)}$.
\zadStop
\rozwStart{Patryk Wirkus}{Martyna Czarnobaj}
$$\frac{x^{2}-35^{2}}{(x-35)(x-37)}=\frac{x+35}{x-37}$$

$$\lim\limits_{x\to\ 35}\frac{x^{2}-35^{2}}{(x-35)(x-37)}=[\frac{0}{0}]=\lim\limits_{x\to\ 35}\frac{x+35}{x-37}=2 \cdot \frac{35}{35-37} = \frac{70}{-2}$$
\rozwStop
\odpStart
$\frac{70}{-2}$
\odpStop
\testStart
A.$\frac{70}{-2}$
B.$\infty$
C.$-\infty$
D.$0$
E.$\frac{70}{2}$
F.$\frac{35}{37}$
G.$-\frac{70}{2}$
H.$1$
I.$35$
\testStop
\kluczStart
A
\kluczStop



\zadStart{Przykład z Wikieł P 4.2b moja wersja nr 683}
Obliczyć granicę $\lim\limits_{x\to\ 35}\frac{x^{2}-35^{2}}{(x-35)(x-38)}$.
\zadStop
\rozwStart{Patryk Wirkus}{Martyna Czarnobaj}
$$\frac{x^{2}-35^{2}}{(x-35)(x-38)}=\frac{x+35}{x-38}$$

$$\lim\limits_{x\to\ 35}\frac{x^{2}-35^{2}}{(x-35)(x-38)}=[\frac{0}{0}]=\lim\limits_{x\to\ 35}\frac{x+35}{x-38}=2 \cdot \frac{35}{35-38} = \frac{70}{-3}$$
\rozwStop
\odpStart
$\frac{70}{-3}$
\odpStop
\testStart
A.$\frac{70}{-3}$
B.$\infty$
C.$-\infty$
D.$0$
E.$\frac{70}{3}$
F.$\frac{35}{38}$
G.$-\frac{70}{3}$
H.$1$
I.$35$
\testStop
\kluczStart
A
\kluczStop



\zadStart{Przykład z Wikieł P 4.2b moja wersja nr 684}
Obliczyć granicę $\lim\limits_{x\to\ 35}\frac{x^{2}-35^{2}}{(x-35)(x-39)}$.
\zadStop
\rozwStart{Patryk Wirkus}{Martyna Czarnobaj}
$$\frac{x^{2}-35^{2}}{(x-35)(x-39)}=\frac{x+35}{x-39}$$

$$\lim\limits_{x\to\ 35}\frac{x^{2}-35^{2}}{(x-35)(x-39)}=[\frac{0}{0}]=\lim\limits_{x\to\ 35}\frac{x+35}{x-39}=2 \cdot \frac{35}{35-39} = \frac{70}{-4}$$
\rozwStop
\odpStart
$\frac{70}{-4}$
\odpStop
\testStart
A.$\frac{70}{-4}$
B.$\infty$
C.$-\infty$
D.$0$
E.$\frac{70}{4}$
F.$\frac{35}{39}$
G.$-\frac{70}{4}$
H.$1$
I.$35$
\testStop
\kluczStart
A
\kluczStop



\zadStart{Przykład z Wikieł P 4.2b moja wersja nr 685}
Obliczyć granicę $\lim\limits_{x\to\ 36}\frac{x^{2}-36^{2}}{(x-36)(x-5)}$.
\zadStop
\rozwStart{Patryk Wirkus}{Martyna Czarnobaj}
$$\frac{x^{2}-36^{2}}{(x-36)(x-5)}=\frac{x+36}{x-5}$$

$$\lim\limits_{x\to\ 36}\frac{x^{2}-36^{2}}{(x-36)(x-5)}=[\frac{0}{0}]=\lim\limits_{x\to\ 36}\frac{x+36}{x-5}=2 \cdot \frac{36}{36-5} = \frac{72}{31}$$
\rozwStop
\odpStart
$\frac{72}{31}$
\odpStop
\testStart
A.$\frac{72}{31}$
B.$\infty$
C.$-\infty$
D.$0$
E.$\frac{72}{-31}$
F.$\frac{36}{5}$
G.$-\frac{72}{-31}$
H.$1$
I.$36$
\testStop
\kluczStart
A
\kluczStop



\zadStart{Przykład z Wikieł P 4.2b moja wersja nr 686}
Obliczyć granicę $\lim\limits_{x\to\ 36}\frac{x^{2}-36^{2}}{(x-36)(x-7)}$.
\zadStop
\rozwStart{Patryk Wirkus}{Martyna Czarnobaj}
$$\frac{x^{2}-36^{2}}{(x-36)(x-7)}=\frac{x+36}{x-7}$$

$$\lim\limits_{x\to\ 36}\frac{x^{2}-36^{2}}{(x-36)(x-7)}=[\frac{0}{0}]=\lim\limits_{x\to\ 36}\frac{x+36}{x-7}=2 \cdot \frac{36}{36-7} = \frac{72}{29}$$
\rozwStop
\odpStart
$\frac{72}{29}$
\odpStop
\testStart
A.$\frac{72}{29}$
B.$\infty$
C.$-\infty$
D.$0$
E.$\frac{72}{-29}$
F.$\frac{36}{7}$
G.$-\frac{72}{-29}$
H.$1$
I.$36$
\testStop
\kluczStart
A
\kluczStop



\zadStart{Przykład z Wikieł P 4.2b moja wersja nr 687}
Obliczyć granicę $\lim\limits_{x\to\ 36}\frac{x^{2}-36^{2}}{(x-36)(x-11)}$.
\zadStop
\rozwStart{Patryk Wirkus}{Martyna Czarnobaj}
$$\frac{x^{2}-36^{2}}{(x-36)(x-11)}=\frac{x+36}{x-11}$$

$$\lim\limits_{x\to\ 36}\frac{x^{2}-36^{2}}{(x-36)(x-11)}=[\frac{0}{0}]=\lim\limits_{x\to\ 36}\frac{x+36}{x-11}=2 \cdot \frac{36}{36-11} = \frac{72}{25}$$
\rozwStop
\odpStart
$\frac{72}{25}$
\odpStop
\testStart
A.$\frac{72}{25}$
B.$\infty$
C.$-\infty$
D.$0$
E.$\frac{72}{-25}$
F.$\frac{36}{11}$
G.$-\frac{72}{-25}$
H.$1$
I.$36$
\testStop
\kluczStart
A
\kluczStop



\zadStart{Przykład z Wikieł P 4.2b moja wersja nr 688}
Obliczyć granicę $\lim\limits_{x\to\ 36}\frac{x^{2}-36^{2}}{(x-36)(x-13)}$.
\zadStop
\rozwStart{Patryk Wirkus}{Martyna Czarnobaj}
$$\frac{x^{2}-36^{2}}{(x-36)(x-13)}=\frac{x+36}{x-13}$$

$$\lim\limits_{x\to\ 36}\frac{x^{2}-36^{2}}{(x-36)(x-13)}=[\frac{0}{0}]=\lim\limits_{x\to\ 36}\frac{x+36}{x-13}=2 \cdot \frac{36}{36-13} = \frac{72}{23}$$
\rozwStop
\odpStart
$\frac{72}{23}$
\odpStop
\testStart
A.$\frac{72}{23}$
B.$\infty$
C.$-\infty$
D.$0$
E.$\frac{72}{-23}$
F.$\frac{36}{13}$
G.$-\frac{72}{-23}$
H.$1$
I.$36$
\testStop
\kluczStart
A
\kluczStop



\zadStart{Przykład z Wikieł P 4.2b moja wersja nr 689}
Obliczyć granicę $\lim\limits_{x\to\ 36}\frac{x^{2}-36^{2}}{(x-36)(x-17)}$.
\zadStop
\rozwStart{Patryk Wirkus}{Martyna Czarnobaj}
$$\frac{x^{2}-36^{2}}{(x-36)(x-17)}=\frac{x+36}{x-17}$$

$$\lim\limits_{x\to\ 36}\frac{x^{2}-36^{2}}{(x-36)(x-17)}=[\frac{0}{0}]=\lim\limits_{x\to\ 36}\frac{x+36}{x-17}=2 \cdot \frac{36}{36-17} = \frac{72}{19}$$
\rozwStop
\odpStart
$\frac{72}{19}$
\odpStop
\testStart
A.$\frac{72}{19}$
B.$\infty$
C.$-\infty$
D.$0$
E.$\frac{72}{-19}$
F.$\frac{36}{17}$
G.$-\frac{72}{-19}$
H.$1$
I.$36$
\testStop
\kluczStart
A
\kluczStop



\zadStart{Przykład z Wikieł P 4.2b moja wersja nr 690}
Obliczyć granicę $\lim\limits_{x\to\ 36}\frac{x^{2}-36^{2}}{(x-36)(x-19)}$.
\zadStop
\rozwStart{Patryk Wirkus}{Martyna Czarnobaj}
$$\frac{x^{2}-36^{2}}{(x-36)(x-19)}=\frac{x+36}{x-19}$$

$$\lim\limits_{x\to\ 36}\frac{x^{2}-36^{2}}{(x-36)(x-19)}=[\frac{0}{0}]=\lim\limits_{x\to\ 36}\frac{x+36}{x-19}=2 \cdot \frac{36}{36-19} = \frac{72}{17}$$
\rozwStop
\odpStart
$\frac{72}{17}$
\odpStop
\testStart
A.$\frac{72}{17}$
B.$\infty$
C.$-\infty$
D.$0$
E.$\frac{72}{-17}$
F.$\frac{36}{19}$
G.$-\frac{72}{-17}$
H.$1$
I.$36$
\testStop
\kluczStart
A
\kluczStop



\zadStart{Przykład z Wikieł P 4.2b moja wersja nr 691}
Obliczyć granicę $\lim\limits_{x\to\ 36}\frac{x^{2}-36^{2}}{(x-36)(x-23)}$.
\zadStop
\rozwStart{Patryk Wirkus}{Martyna Czarnobaj}
$$\frac{x^{2}-36^{2}}{(x-36)(x-23)}=\frac{x+36}{x-23}$$

$$\lim\limits_{x\to\ 36}\frac{x^{2}-36^{2}}{(x-36)(x-23)}=[\frac{0}{0}]=\lim\limits_{x\to\ 36}\frac{x+36}{x-23}=2 \cdot \frac{36}{36-23} = \frac{72}{13}$$
\rozwStop
\odpStart
$\frac{72}{13}$
\odpStop
\testStart
A.$\frac{72}{13}$
B.$\infty$
C.$-\infty$
D.$0$
E.$\frac{72}{-13}$
F.$\frac{36}{23}$
G.$-\frac{72}{-13}$
H.$1$
I.$36$
\testStop
\kluczStart
A
\kluczStop



\zadStart{Przykład z Wikieł P 4.2b moja wersja nr 692}
Obliczyć granicę $\lim\limits_{x\to\ 36}\frac{x^{2}-36^{2}}{(x-36)(x-25)}$.
\zadStop
\rozwStart{Patryk Wirkus}{Martyna Czarnobaj}
$$\frac{x^{2}-36^{2}}{(x-36)(x-25)}=\frac{x+36}{x-25}$$

$$\lim\limits_{x\to\ 36}\frac{x^{2}-36^{2}}{(x-36)(x-25)}=[\frac{0}{0}]=\lim\limits_{x\to\ 36}\frac{x+36}{x-25}=2 \cdot \frac{36}{36-25} = \frac{72}{11}$$
\rozwStop
\odpStart
$\frac{72}{11}$
\odpStop
\testStart
A.$\frac{72}{11}$
B.$\infty$
C.$-\infty$
D.$0$
E.$\frac{72}{-11}$
F.$\frac{36}{25}$
G.$-\frac{72}{-11}$
H.$1$
I.$36$
\testStop
\kluczStart
A
\kluczStop



\zadStart{Przykład z Wikieł P 4.2b moja wersja nr 693}
Obliczyć granicę $\lim\limits_{x\to\ 36}\frac{x^{2}-36^{2}}{(x-36)(x-29)}$.
\zadStop
\rozwStart{Patryk Wirkus}{Martyna Czarnobaj}
$$\frac{x^{2}-36^{2}}{(x-36)(x-29)}=\frac{x+36}{x-29}$$

$$\lim\limits_{x\to\ 36}\frac{x^{2}-36^{2}}{(x-36)(x-29)}=[\frac{0}{0}]=\lim\limits_{x\to\ 36}\frac{x+36}{x-29}=2 \cdot \frac{36}{36-29} = \frac{72}{7}$$
\rozwStop
\odpStart
$\frac{72}{7}$
\odpStop
\testStart
A.$\frac{72}{7}$
B.$\infty$
C.$-\infty$
D.$0$
E.$\frac{72}{-7}$
F.$\frac{36}{29}$
G.$-\frac{72}{-7}$
H.$1$
I.$36$
\testStop
\kluczStart
A
\kluczStop



\zadStart{Przykład z Wikieł P 4.2b moja wersja nr 694}
Obliczyć granicę $\lim\limits_{x\to\ 36}\frac{x^{2}-36^{2}}{(x-36)(x-31)}$.
\zadStop
\rozwStart{Patryk Wirkus}{Martyna Czarnobaj}
$$\frac{x^{2}-36^{2}}{(x-36)(x-31)}=\frac{x+36}{x-31}$$

$$\lim\limits_{x\to\ 36}\frac{x^{2}-36^{2}}{(x-36)(x-31)}=[\frac{0}{0}]=\lim\limits_{x\to\ 36}\frac{x+36}{x-31}=2 \cdot \frac{36}{36-31} = \frac{72}{5}$$
\rozwStop
\odpStart
$\frac{72}{5}$
\odpStop
\testStart
A.$\frac{72}{5}$
B.$\infty$
C.$-\infty$
D.$0$
E.$\frac{72}{-5}$
F.$\frac{36}{31}$
G.$-\frac{72}{-5}$
H.$1$
I.$36$
\testStop
\kluczStart
A
\kluczStop



\zadStart{Przykład z Wikieł P 4.2b moja wersja nr 695}
Obliczyć granicę $\lim\limits_{x\to\ 37}\frac{x^{2}-37^{2}}{(x-37)(x-2)}$.
\zadStop
\rozwStart{Patryk Wirkus}{Martyna Czarnobaj}
$$\frac{x^{2}-37^{2}}{(x-37)(x-2)}=\frac{x+37}{x-2}$$

$$\lim\limits_{x\to\ 37}\frac{x^{2}-37^{2}}{(x-37)(x-2)}=[\frac{0}{0}]=\lim\limits_{x\to\ 37}\frac{x+37}{x-2}=2 \cdot \frac{37}{37-2} = \frac{74}{35}$$
\rozwStop
\odpStart
$\frac{74}{35}$
\odpStop
\testStart
A.$\frac{74}{35}$
B.$\infty$
C.$-\infty$
D.$0$
E.$\frac{74}{-35}$
F.$\frac{37}{2}$
G.$-\frac{74}{-35}$
H.$1$
I.$37$
\testStop
\kluczStart
A
\kluczStop



\zadStart{Przykład z Wikieł P 4.2b moja wersja nr 696}
Obliczyć granicę $\lim\limits_{x\to\ 37}\frac{x^{2}-37^{2}}{(x-37)(x-3)}$.
\zadStop
\rozwStart{Patryk Wirkus}{Martyna Czarnobaj}
$$\frac{x^{2}-37^{2}}{(x-37)(x-3)}=\frac{x+37}{x-3}$$

$$\lim\limits_{x\to\ 37}\frac{x^{2}-37^{2}}{(x-37)(x-3)}=[\frac{0}{0}]=\lim\limits_{x\to\ 37}\frac{x+37}{x-3}=2 \cdot \frac{37}{37-3} = \frac{74}{34}$$
\rozwStop
\odpStart
$\frac{74}{34}$
\odpStop
\testStart
A.$\frac{74}{34}$
B.$\infty$
C.$-\infty$
D.$0$
E.$\frac{74}{-34}$
F.$\frac{37}{3}$
G.$-\frac{74}{-34}$
H.$1$
I.$37$
\testStop
\kluczStart
A
\kluczStop



\zadStart{Przykład z Wikieł P 4.2b moja wersja nr 697}
Obliczyć granicę $\lim\limits_{x\to\ 37}\frac{x^{2}-37^{2}}{(x-37)(x-4)}$.
\zadStop
\rozwStart{Patryk Wirkus}{Martyna Czarnobaj}
$$\frac{x^{2}-37^{2}}{(x-37)(x-4)}=\frac{x+37}{x-4}$$

$$\lim\limits_{x\to\ 37}\frac{x^{2}-37^{2}}{(x-37)(x-4)}=[\frac{0}{0}]=\lim\limits_{x\to\ 37}\frac{x+37}{x-4}=2 \cdot \frac{37}{37-4} = \frac{74}{33}$$
\rozwStop
\odpStart
$\frac{74}{33}$
\odpStop
\testStart
A.$\frac{74}{33}$
B.$\infty$
C.$-\infty$
D.$0$
E.$\frac{74}{-33}$
F.$\frac{37}{4}$
G.$-\frac{74}{-33}$
H.$1$
I.$37$
\testStop
\kluczStart
A
\kluczStop



\zadStart{Przykład z Wikieł P 4.2b moja wersja nr 698}
Obliczyć granicę $\lim\limits_{x\to\ 37}\frac{x^{2}-37^{2}}{(x-37)(x-5)}$.
\zadStop
\rozwStart{Patryk Wirkus}{Martyna Czarnobaj}
$$\frac{x^{2}-37^{2}}{(x-37)(x-5)}=\frac{x+37}{x-5}$$

$$\lim\limits_{x\to\ 37}\frac{x^{2}-37^{2}}{(x-37)(x-5)}=[\frac{0}{0}]=\lim\limits_{x\to\ 37}\frac{x+37}{x-5}=2 \cdot \frac{37}{37-5} = \frac{74}{32}$$
\rozwStop
\odpStart
$\frac{74}{32}$
\odpStop
\testStart
A.$\frac{74}{32}$
B.$\infty$
C.$-\infty$
D.$0$
E.$\frac{74}{-32}$
F.$\frac{37}{5}$
G.$-\frac{74}{-32}$
H.$1$
I.$37$
\testStop
\kluczStart
A
\kluczStop



\zadStart{Przykład z Wikieł P 4.2b moja wersja nr 699}
Obliczyć granicę $\lim\limits_{x\to\ 37}\frac{x^{2}-37^{2}}{(x-37)(x-6)}$.
\zadStop
\rozwStart{Patryk Wirkus}{Martyna Czarnobaj}
$$\frac{x^{2}-37^{2}}{(x-37)(x-6)}=\frac{x+37}{x-6}$$

$$\lim\limits_{x\to\ 37}\frac{x^{2}-37^{2}}{(x-37)(x-6)}=[\frac{0}{0}]=\lim\limits_{x\to\ 37}\frac{x+37}{x-6}=2 \cdot \frac{37}{37-6} = \frac{74}{31}$$
\rozwStop
\odpStart
$\frac{74}{31}$
\odpStop
\testStart
A.$\frac{74}{31}$
B.$\infty$
C.$-\infty$
D.$0$
E.$\frac{74}{-31}$
F.$\frac{37}{6}$
G.$-\frac{74}{-31}$
H.$1$
I.$37$
\testStop
\kluczStart
A
\kluczStop



\zadStart{Przykład z Wikieł P 4.2b moja wersja nr 700}
Obliczyć granicę $\lim\limits_{x\to\ 37}\frac{x^{2}-37^{2}}{(x-37)(x-7)}$.
\zadStop
\rozwStart{Patryk Wirkus}{Martyna Czarnobaj}
$$\frac{x^{2}-37^{2}}{(x-37)(x-7)}=\frac{x+37}{x-7}$$

$$\lim\limits_{x\to\ 37}\frac{x^{2}-37^{2}}{(x-37)(x-7)}=[\frac{0}{0}]=\lim\limits_{x\to\ 37}\frac{x+37}{x-7}=2 \cdot \frac{37}{37-7} = \frac{74}{30}$$
\rozwStop
\odpStart
$\frac{74}{30}$
\odpStop
\testStart
A.$\frac{74}{30}$
B.$\infty$
C.$-\infty$
D.$0$
E.$\frac{74}{-30}$
F.$\frac{37}{7}$
G.$-\frac{74}{-30}$
H.$1$
I.$37$
\testStop
\kluczStart
A
\kluczStop



\zadStart{Przykład z Wikieł P 4.2b moja wersja nr 701}
Obliczyć granicę $\lim\limits_{x\to\ 37}\frac{x^{2}-37^{2}}{(x-37)(x-8)}$.
\zadStop
\rozwStart{Patryk Wirkus}{Martyna Czarnobaj}
$$\frac{x^{2}-37^{2}}{(x-37)(x-8)}=\frac{x+37}{x-8}$$

$$\lim\limits_{x\to\ 37}\frac{x^{2}-37^{2}}{(x-37)(x-8)}=[\frac{0}{0}]=\lim\limits_{x\to\ 37}\frac{x+37}{x-8}=2 \cdot \frac{37}{37-8} = \frac{74}{29}$$
\rozwStop
\odpStart
$\frac{74}{29}$
\odpStop
\testStart
A.$\frac{74}{29}$
B.$\infty$
C.$-\infty$
D.$0$
E.$\frac{74}{-29}$
F.$\frac{37}{8}$
G.$-\frac{74}{-29}$
H.$1$
I.$37$
\testStop
\kluczStart
A
\kluczStop



\zadStart{Przykład z Wikieł P 4.2b moja wersja nr 702}
Obliczyć granicę $\lim\limits_{x\to\ 37}\frac{x^{2}-37^{2}}{(x-37)(x-9)}$.
\zadStop
\rozwStart{Patryk Wirkus}{Martyna Czarnobaj}
$$\frac{x^{2}-37^{2}}{(x-37)(x-9)}=\frac{x+37}{x-9}$$

$$\lim\limits_{x\to\ 37}\frac{x^{2}-37^{2}}{(x-37)(x-9)}=[\frac{0}{0}]=\lim\limits_{x\to\ 37}\frac{x+37}{x-9}=2 \cdot \frac{37}{37-9} = \frac{74}{28}$$
\rozwStop
\odpStart
$\frac{74}{28}$
\odpStop
\testStart
A.$\frac{74}{28}$
B.$\infty$
C.$-\infty$
D.$0$
E.$\frac{74}{-28}$
F.$\frac{37}{9}$
G.$-\frac{74}{-28}$
H.$1$
I.$37$
\testStop
\kluczStart
A
\kluczStop



\zadStart{Przykład z Wikieł P 4.2b moja wersja nr 703}
Obliczyć granicę $\lim\limits_{x\to\ 37}\frac{x^{2}-37^{2}}{(x-37)(x-10)}$.
\zadStop
\rozwStart{Patryk Wirkus}{Martyna Czarnobaj}
$$\frac{x^{2}-37^{2}}{(x-37)(x-10)}=\frac{x+37}{x-10}$$

$$\lim\limits_{x\to\ 37}\frac{x^{2}-37^{2}}{(x-37)(x-10)}=[\frac{0}{0}]=\lim\limits_{x\to\ 37}\frac{x+37}{x-10}=2 \cdot \frac{37}{37-10} = \frac{74}{27}$$
\rozwStop
\odpStart
$\frac{74}{27}$
\odpStop
\testStart
A.$\frac{74}{27}$
B.$\infty$
C.$-\infty$
D.$0$
E.$\frac{74}{-27}$
F.$\frac{37}{10}$
G.$-\frac{74}{-27}$
H.$1$
I.$37$
\testStop
\kluczStart
A
\kluczStop



\zadStart{Przykład z Wikieł P 4.2b moja wersja nr 704}
Obliczyć granicę $\lim\limits_{x\to\ 37}\frac{x^{2}-37^{2}}{(x-37)(x-11)}$.
\zadStop
\rozwStart{Patryk Wirkus}{Martyna Czarnobaj}
$$\frac{x^{2}-37^{2}}{(x-37)(x-11)}=\frac{x+37}{x-11}$$

$$\lim\limits_{x\to\ 37}\frac{x^{2}-37^{2}}{(x-37)(x-11)}=[\frac{0}{0}]=\lim\limits_{x\to\ 37}\frac{x+37}{x-11}=2 \cdot \frac{37}{37-11} = \frac{74}{26}$$
\rozwStop
\odpStart
$\frac{74}{26}$
\odpStop
\testStart
A.$\frac{74}{26}$
B.$\infty$
C.$-\infty$
D.$0$
E.$\frac{74}{-26}$
F.$\frac{37}{11}$
G.$-\frac{74}{-26}$
H.$1$
I.$37$
\testStop
\kluczStart
A
\kluczStop



\zadStart{Przykład z Wikieł P 4.2b moja wersja nr 705}
Obliczyć granicę $\lim\limits_{x\to\ 37}\frac{x^{2}-37^{2}}{(x-37)(x-12)}$.
\zadStop
\rozwStart{Patryk Wirkus}{Martyna Czarnobaj}
$$\frac{x^{2}-37^{2}}{(x-37)(x-12)}=\frac{x+37}{x-12}$$

$$\lim\limits_{x\to\ 37}\frac{x^{2}-37^{2}}{(x-37)(x-12)}=[\frac{0}{0}]=\lim\limits_{x\to\ 37}\frac{x+37}{x-12}=2 \cdot \frac{37}{37-12} = \frac{74}{25}$$
\rozwStop
\odpStart
$\frac{74}{25}$
\odpStop
\testStart
A.$\frac{74}{25}$
B.$\infty$
C.$-\infty$
D.$0$
E.$\frac{74}{-25}$
F.$\frac{37}{12}$
G.$-\frac{74}{-25}$
H.$1$
I.$37$
\testStop
\kluczStart
A
\kluczStop



\zadStart{Przykład z Wikieł P 4.2b moja wersja nr 706}
Obliczyć granicę $\lim\limits_{x\to\ 37}\frac{x^{2}-37^{2}}{(x-37)(x-13)}$.
\zadStop
\rozwStart{Patryk Wirkus}{Martyna Czarnobaj}
$$\frac{x^{2}-37^{2}}{(x-37)(x-13)}=\frac{x+37}{x-13}$$

$$\lim\limits_{x\to\ 37}\frac{x^{2}-37^{2}}{(x-37)(x-13)}=[\frac{0}{0}]=\lim\limits_{x\to\ 37}\frac{x+37}{x-13}=2 \cdot \frac{37}{37-13} = \frac{74}{24}$$
\rozwStop
\odpStart
$\frac{74}{24}$
\odpStop
\testStart
A.$\frac{74}{24}$
B.$\infty$
C.$-\infty$
D.$0$
E.$\frac{74}{-24}$
F.$\frac{37}{13}$
G.$-\frac{74}{-24}$
H.$1$
I.$37$
\testStop
\kluczStart
A
\kluczStop



\zadStart{Przykład z Wikieł P 4.2b moja wersja nr 707}
Obliczyć granicę $\lim\limits_{x\to\ 37}\frac{x^{2}-37^{2}}{(x-37)(x-14)}$.
\zadStop
\rozwStart{Patryk Wirkus}{Martyna Czarnobaj}
$$\frac{x^{2}-37^{2}}{(x-37)(x-14)}=\frac{x+37}{x-14}$$

$$\lim\limits_{x\to\ 37}\frac{x^{2}-37^{2}}{(x-37)(x-14)}=[\frac{0}{0}]=\lim\limits_{x\to\ 37}\frac{x+37}{x-14}=2 \cdot \frac{37}{37-14} = \frac{74}{23}$$
\rozwStop
\odpStart
$\frac{74}{23}$
\odpStop
\testStart
A.$\frac{74}{23}$
B.$\infty$
C.$-\infty$
D.$0$
E.$\frac{74}{-23}$
F.$\frac{37}{14}$
G.$-\frac{74}{-23}$
H.$1$
I.$37$
\testStop
\kluczStart
A
\kluczStop



\zadStart{Przykład z Wikieł P 4.2b moja wersja nr 708}
Obliczyć granicę $\lim\limits_{x\to\ 37}\frac{x^{2}-37^{2}}{(x-37)(x-15)}$.
\zadStop
\rozwStart{Patryk Wirkus}{Martyna Czarnobaj}
$$\frac{x^{2}-37^{2}}{(x-37)(x-15)}=\frac{x+37}{x-15}$$

$$\lim\limits_{x\to\ 37}\frac{x^{2}-37^{2}}{(x-37)(x-15)}=[\frac{0}{0}]=\lim\limits_{x\to\ 37}\frac{x+37}{x-15}=2 \cdot \frac{37}{37-15} = \frac{74}{22}$$
\rozwStop
\odpStart
$\frac{74}{22}$
\odpStop
\testStart
A.$\frac{74}{22}$
B.$\infty$
C.$-\infty$
D.$0$
E.$\frac{74}{-22}$
F.$\frac{37}{15}$
G.$-\frac{74}{-22}$
H.$1$
I.$37$
\testStop
\kluczStart
A
\kluczStop



\zadStart{Przykład z Wikieł P 4.2b moja wersja nr 709}
Obliczyć granicę $\lim\limits_{x\to\ 37}\frac{x^{2}-37^{2}}{(x-37)(x-16)}$.
\zadStop
\rozwStart{Patryk Wirkus}{Martyna Czarnobaj}
$$\frac{x^{2}-37^{2}}{(x-37)(x-16)}=\frac{x+37}{x-16}$$

$$\lim\limits_{x\to\ 37}\frac{x^{2}-37^{2}}{(x-37)(x-16)}=[\frac{0}{0}]=\lim\limits_{x\to\ 37}\frac{x+37}{x-16}=2 \cdot \frac{37}{37-16} = \frac{74}{21}$$
\rozwStop
\odpStart
$\frac{74}{21}$
\odpStop
\testStart
A.$\frac{74}{21}$
B.$\infty$
C.$-\infty$
D.$0$
E.$\frac{74}{-21}$
F.$\frac{37}{16}$
G.$-\frac{74}{-21}$
H.$1$
I.$37$
\testStop
\kluczStart
A
\kluczStop



\zadStart{Przykład z Wikieł P 4.2b moja wersja nr 710}
Obliczyć granicę $\lim\limits_{x\to\ 37}\frac{x^{2}-37^{2}}{(x-37)(x-17)}$.
\zadStop
\rozwStart{Patryk Wirkus}{Martyna Czarnobaj}
$$\frac{x^{2}-37^{2}}{(x-37)(x-17)}=\frac{x+37}{x-17}$$

$$\lim\limits_{x\to\ 37}\frac{x^{2}-37^{2}}{(x-37)(x-17)}=[\frac{0}{0}]=\lim\limits_{x\to\ 37}\frac{x+37}{x-17}=2 \cdot \frac{37}{37-17} = \frac{74}{20}$$
\rozwStop
\odpStart
$\frac{74}{20}$
\odpStop
\testStart
A.$\frac{74}{20}$
B.$\infty$
C.$-\infty$
D.$0$
E.$\frac{74}{-20}$
F.$\frac{37}{17}$
G.$-\frac{74}{-20}$
H.$1$
I.$37$
\testStop
\kluczStart
A
\kluczStop



\zadStart{Przykład z Wikieł P 4.2b moja wersja nr 711}
Obliczyć granicę $\lim\limits_{x\to\ 37}\frac{x^{2}-37^{2}}{(x-37)(x-18)}$.
\zadStop
\rozwStart{Patryk Wirkus}{Martyna Czarnobaj}
$$\frac{x^{2}-37^{2}}{(x-37)(x-18)}=\frac{x+37}{x-18}$$

$$\lim\limits_{x\to\ 37}\frac{x^{2}-37^{2}}{(x-37)(x-18)}=[\frac{0}{0}]=\lim\limits_{x\to\ 37}\frac{x+37}{x-18}=2 \cdot \frac{37}{37-18} = \frac{74}{19}$$
\rozwStop
\odpStart
$\frac{74}{19}$
\odpStop
\testStart
A.$\frac{74}{19}$
B.$\infty$
C.$-\infty$
D.$0$
E.$\frac{74}{-19}$
F.$\frac{37}{18}$
G.$-\frac{74}{-19}$
H.$1$
I.$37$
\testStop
\kluczStart
A
\kluczStop



\zadStart{Przykład z Wikieł P 4.2b moja wersja nr 712}
Obliczyć granicę $\lim\limits_{x\to\ 37}\frac{x^{2}-37^{2}}{(x-37)(x-19)}$.
\zadStop
\rozwStart{Patryk Wirkus}{Martyna Czarnobaj}
$$\frac{x^{2}-37^{2}}{(x-37)(x-19)}=\frac{x+37}{x-19}$$

$$\lim\limits_{x\to\ 37}\frac{x^{2}-37^{2}}{(x-37)(x-19)}=[\frac{0}{0}]=\lim\limits_{x\to\ 37}\frac{x+37}{x-19}=2 \cdot \frac{37}{37-19} = \frac{74}{18}$$
\rozwStop
\odpStart
$\frac{74}{18}$
\odpStop
\testStart
A.$\frac{74}{18}$
B.$\infty$
C.$-\infty$
D.$0$
E.$\frac{74}{-18}$
F.$\frac{37}{19}$
G.$-\frac{74}{-18}$
H.$1$
I.$37$
\testStop
\kluczStart
A
\kluczStop



\zadStart{Przykład z Wikieł P 4.2b moja wersja nr 713}
Obliczyć granicę $\lim\limits_{x\to\ 37}\frac{x^{2}-37^{2}}{(x-37)(x-20)}$.
\zadStop
\rozwStart{Patryk Wirkus}{Martyna Czarnobaj}
$$\frac{x^{2}-37^{2}}{(x-37)(x-20)}=\frac{x+37}{x-20}$$

$$\lim\limits_{x\to\ 37}\frac{x^{2}-37^{2}}{(x-37)(x-20)}=[\frac{0}{0}]=\lim\limits_{x\to\ 37}\frac{x+37}{x-20}=2 \cdot \frac{37}{37-20} = \frac{74}{17}$$
\rozwStop
\odpStart
$\frac{74}{17}$
\odpStop
\testStart
A.$\frac{74}{17}$
B.$\infty$
C.$-\infty$
D.$0$
E.$\frac{74}{-17}$
F.$\frac{37}{20}$
G.$-\frac{74}{-17}$
H.$1$
I.$37$
\testStop
\kluczStart
A
\kluczStop



\zadStart{Przykład z Wikieł P 4.2b moja wersja nr 714}
Obliczyć granicę $\lim\limits_{x\to\ 37}\frac{x^{2}-37^{2}}{(x-37)(x-21)}$.
\zadStop
\rozwStart{Patryk Wirkus}{Martyna Czarnobaj}
$$\frac{x^{2}-37^{2}}{(x-37)(x-21)}=\frac{x+37}{x-21}$$

$$\lim\limits_{x\to\ 37}\frac{x^{2}-37^{2}}{(x-37)(x-21)}=[\frac{0}{0}]=\lim\limits_{x\to\ 37}\frac{x+37}{x-21}=2 \cdot \frac{37}{37-21} = \frac{74}{16}$$
\rozwStop
\odpStart
$\frac{74}{16}$
\odpStop
\testStart
A.$\frac{74}{16}$
B.$\infty$
C.$-\infty$
D.$0$
E.$\frac{74}{-16}$
F.$\frac{37}{21}$
G.$-\frac{74}{-16}$
H.$1$
I.$37$
\testStop
\kluczStart
A
\kluczStop



\zadStart{Przykład z Wikieł P 4.2b moja wersja nr 715}
Obliczyć granicę $\lim\limits_{x\to\ 37}\frac{x^{2}-37^{2}}{(x-37)(x-22)}$.
\zadStop
\rozwStart{Patryk Wirkus}{Martyna Czarnobaj}
$$\frac{x^{2}-37^{2}}{(x-37)(x-22)}=\frac{x+37}{x-22}$$

$$\lim\limits_{x\to\ 37}\frac{x^{2}-37^{2}}{(x-37)(x-22)}=[\frac{0}{0}]=\lim\limits_{x\to\ 37}\frac{x+37}{x-22}=2 \cdot \frac{37}{37-22} = \frac{74}{15}$$
\rozwStop
\odpStart
$\frac{74}{15}$
\odpStop
\testStart
A.$\frac{74}{15}$
B.$\infty$
C.$-\infty$
D.$0$
E.$\frac{74}{-15}$
F.$\frac{37}{22}$
G.$-\frac{74}{-15}$
H.$1$
I.$37$
\testStop
\kluczStart
A
\kluczStop



\zadStart{Przykład z Wikieł P 4.2b moja wersja nr 716}
Obliczyć granicę $\lim\limits_{x\to\ 37}\frac{x^{2}-37^{2}}{(x-37)(x-23)}$.
\zadStop
\rozwStart{Patryk Wirkus}{Martyna Czarnobaj}
$$\frac{x^{2}-37^{2}}{(x-37)(x-23)}=\frac{x+37}{x-23}$$

$$\lim\limits_{x\to\ 37}\frac{x^{2}-37^{2}}{(x-37)(x-23)}=[\frac{0}{0}]=\lim\limits_{x\to\ 37}\frac{x+37}{x-23}=2 \cdot \frac{37}{37-23} = \frac{74}{14}$$
\rozwStop
\odpStart
$\frac{74}{14}$
\odpStop
\testStart
A.$\frac{74}{14}$
B.$\infty$
C.$-\infty$
D.$0$
E.$\frac{74}{-14}$
F.$\frac{37}{23}$
G.$-\frac{74}{-14}$
H.$1$
I.$37$
\testStop
\kluczStart
A
\kluczStop



\zadStart{Przykład z Wikieł P 4.2b moja wersja nr 717}
Obliczyć granicę $\lim\limits_{x\to\ 37}\frac{x^{2}-37^{2}}{(x-37)(x-24)}$.
\zadStop
\rozwStart{Patryk Wirkus}{Martyna Czarnobaj}
$$\frac{x^{2}-37^{2}}{(x-37)(x-24)}=\frac{x+37}{x-24}$$

$$\lim\limits_{x\to\ 37}\frac{x^{2}-37^{2}}{(x-37)(x-24)}=[\frac{0}{0}]=\lim\limits_{x\to\ 37}\frac{x+37}{x-24}=2 \cdot \frac{37}{37-24} = \frac{74}{13}$$
\rozwStop
\odpStart
$\frac{74}{13}$
\odpStop
\testStart
A.$\frac{74}{13}$
B.$\infty$
C.$-\infty$
D.$0$
E.$\frac{74}{-13}$
F.$\frac{37}{24}$
G.$-\frac{74}{-13}$
H.$1$
I.$37$
\testStop
\kluczStart
A
\kluczStop



\zadStart{Przykład z Wikieł P 4.2b moja wersja nr 718}
Obliczyć granicę $\lim\limits_{x\to\ 37}\frac{x^{2}-37^{2}}{(x-37)(x-25)}$.
\zadStop
\rozwStart{Patryk Wirkus}{Martyna Czarnobaj}
$$\frac{x^{2}-37^{2}}{(x-37)(x-25)}=\frac{x+37}{x-25}$$

$$\lim\limits_{x\to\ 37}\frac{x^{2}-37^{2}}{(x-37)(x-25)}=[\frac{0}{0}]=\lim\limits_{x\to\ 37}\frac{x+37}{x-25}=2 \cdot \frac{37}{37-25} = \frac{74}{12}$$
\rozwStop
\odpStart
$\frac{74}{12}$
\odpStop
\testStart
A.$\frac{74}{12}$
B.$\infty$
C.$-\infty$
D.$0$
E.$\frac{74}{-12}$
F.$\frac{37}{25}$
G.$-\frac{74}{-12}$
H.$1$
I.$37$
\testStop
\kluczStart
A
\kluczStop



\zadStart{Przykład z Wikieł P 4.2b moja wersja nr 719}
Obliczyć granicę $\lim\limits_{x\to\ 37}\frac{x^{2}-37^{2}}{(x-37)(x-26)}$.
\zadStop
\rozwStart{Patryk Wirkus}{Martyna Czarnobaj}
$$\frac{x^{2}-37^{2}}{(x-37)(x-26)}=\frac{x+37}{x-26}$$

$$\lim\limits_{x\to\ 37}\frac{x^{2}-37^{2}}{(x-37)(x-26)}=[\frac{0}{0}]=\lim\limits_{x\to\ 37}\frac{x+37}{x-26}=2 \cdot \frac{37}{37-26} = \frac{74}{11}$$
\rozwStop
\odpStart
$\frac{74}{11}$
\odpStop
\testStart
A.$\frac{74}{11}$
B.$\infty$
C.$-\infty$
D.$0$
E.$\frac{74}{-11}$
F.$\frac{37}{26}$
G.$-\frac{74}{-11}$
H.$1$
I.$37$
\testStop
\kluczStart
A
\kluczStop



\zadStart{Przykład z Wikieł P 4.2b moja wersja nr 720}
Obliczyć granicę $\lim\limits_{x\to\ 37}\frac{x^{2}-37^{2}}{(x-37)(x-27)}$.
\zadStop
\rozwStart{Patryk Wirkus}{Martyna Czarnobaj}
$$\frac{x^{2}-37^{2}}{(x-37)(x-27)}=\frac{x+37}{x-27}$$

$$\lim\limits_{x\to\ 37}\frac{x^{2}-37^{2}}{(x-37)(x-27)}=[\frac{0}{0}]=\lim\limits_{x\to\ 37}\frac{x+37}{x-27}=2 \cdot \frac{37}{37-27} = \frac{74}{10}$$
\rozwStop
\odpStart
$\frac{74}{10}$
\odpStop
\testStart
A.$\frac{74}{10}$
B.$\infty$
C.$-\infty$
D.$0$
E.$\frac{74}{-10}$
F.$\frac{37}{27}$
G.$-\frac{74}{-10}$
H.$1$
I.$37$
\testStop
\kluczStart
A
\kluczStop



\zadStart{Przykład z Wikieł P 4.2b moja wersja nr 721}
Obliczyć granicę $\lim\limits_{x\to\ 37}\frac{x^{2}-37^{2}}{(x-37)(x-28)}$.
\zadStop
\rozwStart{Patryk Wirkus}{Martyna Czarnobaj}
$$\frac{x^{2}-37^{2}}{(x-37)(x-28)}=\frac{x+37}{x-28}$$

$$\lim\limits_{x\to\ 37}\frac{x^{2}-37^{2}}{(x-37)(x-28)}=[\frac{0}{0}]=\lim\limits_{x\to\ 37}\frac{x+37}{x-28}=2 \cdot \frac{37}{37-28} = \frac{74}{9}$$
\rozwStop
\odpStart
$\frac{74}{9}$
\odpStop
\testStart
A.$\frac{74}{9}$
B.$\infty$
C.$-\infty$
D.$0$
E.$\frac{74}{-9}$
F.$\frac{37}{28}$
G.$-\frac{74}{-9}$
H.$1$
I.$37$
\testStop
\kluczStart
A
\kluczStop



\zadStart{Przykład z Wikieł P 4.2b moja wersja nr 722}
Obliczyć granicę $\lim\limits_{x\to\ 37}\frac{x^{2}-37^{2}}{(x-37)(x-29)}$.
\zadStop
\rozwStart{Patryk Wirkus}{Martyna Czarnobaj}
$$\frac{x^{2}-37^{2}}{(x-37)(x-29)}=\frac{x+37}{x-29}$$

$$\lim\limits_{x\to\ 37}\frac{x^{2}-37^{2}}{(x-37)(x-29)}=[\frac{0}{0}]=\lim\limits_{x\to\ 37}\frac{x+37}{x-29}=2 \cdot \frac{37}{37-29} = \frac{74}{8}$$
\rozwStop
\odpStart
$\frac{74}{8}$
\odpStop
\testStart
A.$\frac{74}{8}$
B.$\infty$
C.$-\infty$
D.$0$
E.$\frac{74}{-8}$
F.$\frac{37}{29}$
G.$-\frac{74}{-8}$
H.$1$
I.$37$
\testStop
\kluczStart
A
\kluczStop



\zadStart{Przykład z Wikieł P 4.2b moja wersja nr 723}
Obliczyć granicę $\lim\limits_{x\to\ 37}\frac{x^{2}-37^{2}}{(x-37)(x-30)}$.
\zadStop
\rozwStart{Patryk Wirkus}{Martyna Czarnobaj}
$$\frac{x^{2}-37^{2}}{(x-37)(x-30)}=\frac{x+37}{x-30}$$

$$\lim\limits_{x\to\ 37}\frac{x^{2}-37^{2}}{(x-37)(x-30)}=[\frac{0}{0}]=\lim\limits_{x\to\ 37}\frac{x+37}{x-30}=2 \cdot \frac{37}{37-30} = \frac{74}{7}$$
\rozwStop
\odpStart
$\frac{74}{7}$
\odpStop
\testStart
A.$\frac{74}{7}$
B.$\infty$
C.$-\infty$
D.$0$
E.$\frac{74}{-7}$
F.$\frac{37}{30}$
G.$-\frac{74}{-7}$
H.$1$
I.$37$
\testStop
\kluczStart
A
\kluczStop



\zadStart{Przykład z Wikieł P 4.2b moja wersja nr 724}
Obliczyć granicę $\lim\limits_{x\to\ 37}\frac{x^{2}-37^{2}}{(x-37)(x-31)}$.
\zadStop
\rozwStart{Patryk Wirkus}{Martyna Czarnobaj}
$$\frac{x^{2}-37^{2}}{(x-37)(x-31)}=\frac{x+37}{x-31}$$

$$\lim\limits_{x\to\ 37}\frac{x^{2}-37^{2}}{(x-37)(x-31)}=[\frac{0}{0}]=\lim\limits_{x\to\ 37}\frac{x+37}{x-31}=2 \cdot \frac{37}{37-31} = \frac{74}{6}$$
\rozwStop
\odpStart
$\frac{74}{6}$
\odpStop
\testStart
A.$\frac{74}{6}$
B.$\infty$
C.$-\infty$
D.$0$
E.$\frac{74}{-6}$
F.$\frac{37}{31}$
G.$-\frac{74}{-6}$
H.$1$
I.$37$
\testStop
\kluczStart
A
\kluczStop



\zadStart{Przykład z Wikieł P 4.2b moja wersja nr 725}
Obliczyć granicę $\lim\limits_{x\to\ 37}\frac{x^{2}-37^{2}}{(x-37)(x-32)}$.
\zadStop
\rozwStart{Patryk Wirkus}{Martyna Czarnobaj}
$$\frac{x^{2}-37^{2}}{(x-37)(x-32)}=\frac{x+37}{x-32}$$

$$\lim\limits_{x\to\ 37}\frac{x^{2}-37^{2}}{(x-37)(x-32)}=[\frac{0}{0}]=\lim\limits_{x\to\ 37}\frac{x+37}{x-32}=2 \cdot \frac{37}{37-32} = \frac{74}{5}$$
\rozwStop
\odpStart
$\frac{74}{5}$
\odpStop
\testStart
A.$\frac{74}{5}$
B.$\infty$
C.$-\infty$
D.$0$
E.$\frac{74}{-5}$
F.$\frac{37}{32}$
G.$-\frac{74}{-5}$
H.$1$
I.$37$
\testStop
\kluczStart
A
\kluczStop



\zadStart{Przykład z Wikieł P 4.2b moja wersja nr 726}
Obliczyć granicę $\lim\limits_{x\to\ 37}\frac{x^{2}-37^{2}}{(x-37)(x-33)}$.
\zadStop
\rozwStart{Patryk Wirkus}{Martyna Czarnobaj}
$$\frac{x^{2}-37^{2}}{(x-37)(x-33)}=\frac{x+37}{x-33}$$

$$\lim\limits_{x\to\ 37}\frac{x^{2}-37^{2}}{(x-37)(x-33)}=[\frac{0}{0}]=\lim\limits_{x\to\ 37}\frac{x+37}{x-33}=2 \cdot \frac{37}{37-33} = \frac{74}{4}$$
\rozwStop
\odpStart
$\frac{74}{4}$
\odpStop
\testStart
A.$\frac{74}{4}$
B.$\infty$
C.$-\infty$
D.$0$
E.$\frac{74}{-4}$
F.$\frac{37}{33}$
G.$-\frac{74}{-4}$
H.$1$
I.$37$
\testStop
\kluczStart
A
\kluczStop



\zadStart{Przykład z Wikieł P 4.2b moja wersja nr 727}
Obliczyć granicę $\lim\limits_{x\to\ 37}\frac{x^{2}-37^{2}}{(x-37)(x-34)}$.
\zadStop
\rozwStart{Patryk Wirkus}{Martyna Czarnobaj}
$$\frac{x^{2}-37^{2}}{(x-37)(x-34)}=\frac{x+37}{x-34}$$

$$\lim\limits_{x\to\ 37}\frac{x^{2}-37^{2}}{(x-37)(x-34)}=[\frac{0}{0}]=\lim\limits_{x\to\ 37}\frac{x+37}{x-34}=2 \cdot \frac{37}{37-34} = \frac{74}{3}$$
\rozwStop
\odpStart
$\frac{74}{3}$
\odpStop
\testStart
A.$\frac{74}{3}$
B.$\infty$
C.$-\infty$
D.$0$
E.$\frac{74}{-3}$
F.$\frac{37}{34}$
G.$-\frac{74}{-3}$
H.$1$
I.$37$
\testStop
\kluczStart
A
\kluczStop



\zadStart{Przykład z Wikieł P 4.2b moja wersja nr 728}
Obliczyć granicę $\lim\limits_{x\to\ 37}\frac{x^{2}-37^{2}}{(x-37)(x-35)}$.
\zadStop
\rozwStart{Patryk Wirkus}{Martyna Czarnobaj}
$$\frac{x^{2}-37^{2}}{(x-37)(x-35)}=\frac{x+37}{x-35}$$

$$\lim\limits_{x\to\ 37}\frac{x^{2}-37^{2}}{(x-37)(x-35)}=[\frac{0}{0}]=\lim\limits_{x\to\ 37}\frac{x+37}{x-35}=2 \cdot \frac{37}{37-35} = \frac{74}{2}$$
\rozwStop
\odpStart
$\frac{74}{2}$
\odpStop
\testStart
A.$\frac{74}{2}$
B.$\infty$
C.$-\infty$
D.$0$
E.$\frac{74}{-2}$
F.$\frac{37}{35}$
G.$-\frac{74}{-2}$
H.$1$
I.$37$
\testStop
\kluczStart
A
\kluczStop



\zadStart{Przykład z Wikieł P 4.2b moja wersja nr 729}
Obliczyć granicę $\lim\limits_{x\to\ 37}\frac{x^{2}-37^{2}}{(x-37)(x-39)}$.
\zadStop
\rozwStart{Patryk Wirkus}{Martyna Czarnobaj}
$$\frac{x^{2}-37^{2}}{(x-37)(x-39)}=\frac{x+37}{x-39}$$

$$\lim\limits_{x\to\ 37}\frac{x^{2}-37^{2}}{(x-37)(x-39)}=[\frac{0}{0}]=\lim\limits_{x\to\ 37}\frac{x+37}{x-39}=2 \cdot \frac{37}{37-39} = \frac{74}{-2}$$
\rozwStop
\odpStart
$\frac{74}{-2}$
\odpStop
\testStart
A.$\frac{74}{-2}$
B.$\infty$
C.$-\infty$
D.$0$
E.$\frac{74}{2}$
F.$\frac{37}{39}$
G.$-\frac{74}{2}$
H.$1$
I.$37$
\testStop
\kluczStart
A
\kluczStop



\zadStart{Przykład z Wikieł P 4.2b moja wersja nr 730}
Obliczyć granicę $\lim\limits_{x\to\ 37}\frac{x^{2}-37^{2}}{(x-37)(x-40)}$.
\zadStop
\rozwStart{Patryk Wirkus}{Martyna Czarnobaj}
$$\frac{x^{2}-37^{2}}{(x-37)(x-40)}=\frac{x+37}{x-40}$$

$$\lim\limits_{x\to\ 37}\frac{x^{2}-37^{2}}{(x-37)(x-40)}=[\frac{0}{0}]=\lim\limits_{x\to\ 37}\frac{x+37}{x-40}=2 \cdot \frac{37}{37-40} = \frac{74}{-3}$$
\rozwStop
\odpStart
$\frac{74}{-3}$
\odpStop
\testStart
A.$\frac{74}{-3}$
B.$\infty$
C.$-\infty$
D.$0$
E.$\frac{74}{3}$
F.$\frac{37}{40}$
G.$-\frac{74}{3}$
H.$1$
I.$37$
\testStop
\kluczStart
A
\kluczStop



\zadStart{Przykład z Wikieł P 4.2b moja wersja nr 731}
Obliczyć granicę $\lim\limits_{x\to\ 38}\frac{x^{2}-38^{2}}{(x-38)(x-3)}$.
\zadStop
\rozwStart{Patryk Wirkus}{Martyna Czarnobaj}
$$\frac{x^{2}-38^{2}}{(x-38)(x-3)}=\frac{x+38}{x-3}$$

$$\lim\limits_{x\to\ 38}\frac{x^{2}-38^{2}}{(x-38)(x-3)}=[\frac{0}{0}]=\lim\limits_{x\to\ 38}\frac{x+38}{x-3}=2 \cdot \frac{38}{38-3} = \frac{76}{35}$$
\rozwStop
\odpStart
$\frac{76}{35}$
\odpStop
\testStart
A.$\frac{76}{35}$
B.$\infty$
C.$-\infty$
D.$0$
E.$\frac{76}{-35}$
F.$\frac{38}{3}$
G.$-\frac{76}{-35}$
H.$1$
I.$38$
\testStop
\kluczStart
A
\kluczStop



\zadStart{Przykład z Wikieł P 4.2b moja wersja nr 732}
Obliczyć granicę $\lim\limits_{x\to\ 38}\frac{x^{2}-38^{2}}{(x-38)(x-5)}$.
\zadStop
\rozwStart{Patryk Wirkus}{Martyna Czarnobaj}
$$\frac{x^{2}-38^{2}}{(x-38)(x-5)}=\frac{x+38}{x-5}$$

$$\lim\limits_{x\to\ 38}\frac{x^{2}-38^{2}}{(x-38)(x-5)}=[\frac{0}{0}]=\lim\limits_{x\to\ 38}\frac{x+38}{x-5}=2 \cdot \frac{38}{38-5} = \frac{76}{33}$$
\rozwStop
\odpStart
$\frac{76}{33}$
\odpStop
\testStart
A.$\frac{76}{33}$
B.$\infty$
C.$-\infty$
D.$0$
E.$\frac{76}{-33}$
F.$\frac{38}{5}$
G.$-\frac{76}{-33}$
H.$1$
I.$38$
\testStop
\kluczStart
A
\kluczStop



\zadStart{Przykład z Wikieł P 4.2b moja wersja nr 733}
Obliczyć granicę $\lim\limits_{x\to\ 38}\frac{x^{2}-38^{2}}{(x-38)(x-7)}$.
\zadStop
\rozwStart{Patryk Wirkus}{Martyna Czarnobaj}
$$\frac{x^{2}-38^{2}}{(x-38)(x-7)}=\frac{x+38}{x-7}$$

$$\lim\limits_{x\to\ 38}\frac{x^{2}-38^{2}}{(x-38)(x-7)}=[\frac{0}{0}]=\lim\limits_{x\to\ 38}\frac{x+38}{x-7}=2 \cdot \frac{38}{38-7} = \frac{76}{31}$$
\rozwStop
\odpStart
$\frac{76}{31}$
\odpStop
\testStart
A.$\frac{76}{31}$
B.$\infty$
C.$-\infty$
D.$0$
E.$\frac{76}{-31}$
F.$\frac{38}{7}$
G.$-\frac{76}{-31}$
H.$1$
I.$38$
\testStop
\kluczStart
A
\kluczStop



\zadStart{Przykład z Wikieł P 4.2b moja wersja nr 734}
Obliczyć granicę $\lim\limits_{x\to\ 38}\frac{x^{2}-38^{2}}{(x-38)(x-9)}$.
\zadStop
\rozwStart{Patryk Wirkus}{Martyna Czarnobaj}
$$\frac{x^{2}-38^{2}}{(x-38)(x-9)}=\frac{x+38}{x-9}$$

$$\lim\limits_{x\to\ 38}\frac{x^{2}-38^{2}}{(x-38)(x-9)}=[\frac{0}{0}]=\lim\limits_{x\to\ 38}\frac{x+38}{x-9}=2 \cdot \frac{38}{38-9} = \frac{76}{29}$$
\rozwStop
\odpStart
$\frac{76}{29}$
\odpStop
\testStart
A.$\frac{76}{29}$
B.$\infty$
C.$-\infty$
D.$0$
E.$\frac{76}{-29}$
F.$\frac{38}{9}$
G.$-\frac{76}{-29}$
H.$1$
I.$38$
\testStop
\kluczStart
A
\kluczStop



\zadStart{Przykład z Wikieł P 4.2b moja wersja nr 735}
Obliczyć granicę $\lim\limits_{x\to\ 38}\frac{x^{2}-38^{2}}{(x-38)(x-11)}$.
\zadStop
\rozwStart{Patryk Wirkus}{Martyna Czarnobaj}
$$\frac{x^{2}-38^{2}}{(x-38)(x-11)}=\frac{x+38}{x-11}$$

$$\lim\limits_{x\to\ 38}\frac{x^{2}-38^{2}}{(x-38)(x-11)}=[\frac{0}{0}]=\lim\limits_{x\to\ 38}\frac{x+38}{x-11}=2 \cdot \frac{38}{38-11} = \frac{76}{27}$$
\rozwStop
\odpStart
$\frac{76}{27}$
\odpStop
\testStart
A.$\frac{76}{27}$
B.$\infty$
C.$-\infty$
D.$0$
E.$\frac{76}{-27}$
F.$\frac{38}{11}$
G.$-\frac{76}{-27}$
H.$1$
I.$38$
\testStop
\kluczStart
A
\kluczStop



\zadStart{Przykład z Wikieł P 4.2b moja wersja nr 736}
Obliczyć granicę $\lim\limits_{x\to\ 38}\frac{x^{2}-38^{2}}{(x-38)(x-13)}$.
\zadStop
\rozwStart{Patryk Wirkus}{Martyna Czarnobaj}
$$\frac{x^{2}-38^{2}}{(x-38)(x-13)}=\frac{x+38}{x-13}$$

$$\lim\limits_{x\to\ 38}\frac{x^{2}-38^{2}}{(x-38)(x-13)}=[\frac{0}{0}]=\lim\limits_{x\to\ 38}\frac{x+38}{x-13}=2 \cdot \frac{38}{38-13} = \frac{76}{25}$$
\rozwStop
\odpStart
$\frac{76}{25}$
\odpStop
\testStart
A.$\frac{76}{25}$
B.$\infty$
C.$-\infty$
D.$0$
E.$\frac{76}{-25}$
F.$\frac{38}{13}$
G.$-\frac{76}{-25}$
H.$1$
I.$38$
\testStop
\kluczStart
A
\kluczStop



\zadStart{Przykład z Wikieł P 4.2b moja wersja nr 737}
Obliczyć granicę $\lim\limits_{x\to\ 38}\frac{x^{2}-38^{2}}{(x-38)(x-15)}$.
\zadStop
\rozwStart{Patryk Wirkus}{Martyna Czarnobaj}
$$\frac{x^{2}-38^{2}}{(x-38)(x-15)}=\frac{x+38}{x-15}$$

$$\lim\limits_{x\to\ 38}\frac{x^{2}-38^{2}}{(x-38)(x-15)}=[\frac{0}{0}]=\lim\limits_{x\to\ 38}\frac{x+38}{x-15}=2 \cdot \frac{38}{38-15} = \frac{76}{23}$$
\rozwStop
\odpStart
$\frac{76}{23}$
\odpStop
\testStart
A.$\frac{76}{23}$
B.$\infty$
C.$-\infty$
D.$0$
E.$\frac{76}{-23}$
F.$\frac{38}{15}$
G.$-\frac{76}{-23}$
H.$1$
I.$38$
\testStop
\kluczStart
A
\kluczStop



\zadStart{Przykład z Wikieł P 4.2b moja wersja nr 738}
Obliczyć granicę $\lim\limits_{x\to\ 38}\frac{x^{2}-38^{2}}{(x-38)(x-17)}$.
\zadStop
\rozwStart{Patryk Wirkus}{Martyna Czarnobaj}
$$\frac{x^{2}-38^{2}}{(x-38)(x-17)}=\frac{x+38}{x-17}$$

$$\lim\limits_{x\to\ 38}\frac{x^{2}-38^{2}}{(x-38)(x-17)}=[\frac{0}{0}]=\lim\limits_{x\to\ 38}\frac{x+38}{x-17}=2 \cdot \frac{38}{38-17} = \frac{76}{21}$$
\rozwStop
\odpStart
$\frac{76}{21}$
\odpStop
\testStart
A.$\frac{76}{21}$
B.$\infty$
C.$-\infty$
D.$0$
E.$\frac{76}{-21}$
F.$\frac{38}{17}$
G.$-\frac{76}{-21}$
H.$1$
I.$38$
\testStop
\kluczStart
A
\kluczStop



\zadStart{Przykład z Wikieł P 4.2b moja wersja nr 739}
Obliczyć granicę $\lim\limits_{x\to\ 38}\frac{x^{2}-38^{2}}{(x-38)(x-21)}$.
\zadStop
\rozwStart{Patryk Wirkus}{Martyna Czarnobaj}
$$\frac{x^{2}-38^{2}}{(x-38)(x-21)}=\frac{x+38}{x-21}$$

$$\lim\limits_{x\to\ 38}\frac{x^{2}-38^{2}}{(x-38)(x-21)}=[\frac{0}{0}]=\lim\limits_{x\to\ 38}\frac{x+38}{x-21}=2 \cdot \frac{38}{38-21} = \frac{76}{17}$$
\rozwStop
\odpStart
$\frac{76}{17}$
\odpStop
\testStart
A.$\frac{76}{17}$
B.$\infty$
C.$-\infty$
D.$0$
E.$\frac{76}{-17}$
F.$\frac{38}{21}$
G.$-\frac{76}{-17}$
H.$1$
I.$38$
\testStop
\kluczStart
A
\kluczStop



\zadStart{Przykład z Wikieł P 4.2b moja wersja nr 740}
Obliczyć granicę $\lim\limits_{x\to\ 38}\frac{x^{2}-38^{2}}{(x-38)(x-23)}$.
\zadStop
\rozwStart{Patryk Wirkus}{Martyna Czarnobaj}
$$\frac{x^{2}-38^{2}}{(x-38)(x-23)}=\frac{x+38}{x-23}$$

$$\lim\limits_{x\to\ 38}\frac{x^{2}-38^{2}}{(x-38)(x-23)}=[\frac{0}{0}]=\lim\limits_{x\to\ 38}\frac{x+38}{x-23}=2 \cdot \frac{38}{38-23} = \frac{76}{15}$$
\rozwStop
\odpStart
$\frac{76}{15}$
\odpStop
\testStart
A.$\frac{76}{15}$
B.$\infty$
C.$-\infty$
D.$0$
E.$\frac{76}{-15}$
F.$\frac{38}{23}$
G.$-\frac{76}{-15}$
H.$1$
I.$38$
\testStop
\kluczStart
A
\kluczStop



\zadStart{Przykład z Wikieł P 4.2b moja wersja nr 741}
Obliczyć granicę $\lim\limits_{x\to\ 38}\frac{x^{2}-38^{2}}{(x-38)(x-25)}$.
\zadStop
\rozwStart{Patryk Wirkus}{Martyna Czarnobaj}
$$\frac{x^{2}-38^{2}}{(x-38)(x-25)}=\frac{x+38}{x-25}$$

$$\lim\limits_{x\to\ 38}\frac{x^{2}-38^{2}}{(x-38)(x-25)}=[\frac{0}{0}]=\lim\limits_{x\to\ 38}\frac{x+38}{x-25}=2 \cdot \frac{38}{38-25} = \frac{76}{13}$$
\rozwStop
\odpStart
$\frac{76}{13}$
\odpStop
\testStart
A.$\frac{76}{13}$
B.$\infty$
C.$-\infty$
D.$0$
E.$\frac{76}{-13}$
F.$\frac{38}{25}$
G.$-\frac{76}{-13}$
H.$1$
I.$38$
\testStop
\kluczStart
A
\kluczStop



\zadStart{Przykład z Wikieł P 4.2b moja wersja nr 742}
Obliczyć granicę $\lim\limits_{x\to\ 38}\frac{x^{2}-38^{2}}{(x-38)(x-27)}$.
\zadStop
\rozwStart{Patryk Wirkus}{Martyna Czarnobaj}
$$\frac{x^{2}-38^{2}}{(x-38)(x-27)}=\frac{x+38}{x-27}$$

$$\lim\limits_{x\to\ 38}\frac{x^{2}-38^{2}}{(x-38)(x-27)}=[\frac{0}{0}]=\lim\limits_{x\to\ 38}\frac{x+38}{x-27}=2 \cdot \frac{38}{38-27} = \frac{76}{11}$$
\rozwStop
\odpStart
$\frac{76}{11}$
\odpStop
\testStart
A.$\frac{76}{11}$
B.$\infty$
C.$-\infty$
D.$0$
E.$\frac{76}{-11}$
F.$\frac{38}{27}$
G.$-\frac{76}{-11}$
H.$1$
I.$38$
\testStop
\kluczStart
A
\kluczStop



\zadStart{Przykład z Wikieł P 4.2b moja wersja nr 743}
Obliczyć granicę $\lim\limits_{x\to\ 38}\frac{x^{2}-38^{2}}{(x-38)(x-29)}$.
\zadStop
\rozwStart{Patryk Wirkus}{Martyna Czarnobaj}
$$\frac{x^{2}-38^{2}}{(x-38)(x-29)}=\frac{x+38}{x-29}$$

$$\lim\limits_{x\to\ 38}\frac{x^{2}-38^{2}}{(x-38)(x-29)}=[\frac{0}{0}]=\lim\limits_{x\to\ 38}\frac{x+38}{x-29}=2 \cdot \frac{38}{38-29} = \frac{76}{9}$$
\rozwStop
\odpStart
$\frac{76}{9}$
\odpStop
\testStart
A.$\frac{76}{9}$
B.$\infty$
C.$-\infty$
D.$0$
E.$\frac{76}{-9}$
F.$\frac{38}{29}$
G.$-\frac{76}{-9}$
H.$1$
I.$38$
\testStop
\kluczStart
A
\kluczStop



\zadStart{Przykład z Wikieł P 4.2b moja wersja nr 744}
Obliczyć granicę $\lim\limits_{x\to\ 38}\frac{x^{2}-38^{2}}{(x-38)(x-31)}$.
\zadStop
\rozwStart{Patryk Wirkus}{Martyna Czarnobaj}
$$\frac{x^{2}-38^{2}}{(x-38)(x-31)}=\frac{x+38}{x-31}$$

$$\lim\limits_{x\to\ 38}\frac{x^{2}-38^{2}}{(x-38)(x-31)}=[\frac{0}{0}]=\lim\limits_{x\to\ 38}\frac{x+38}{x-31}=2 \cdot \frac{38}{38-31} = \frac{76}{7}$$
\rozwStop
\odpStart
$\frac{76}{7}$
\odpStop
\testStart
A.$\frac{76}{7}$
B.$\infty$
C.$-\infty$
D.$0$
E.$\frac{76}{-7}$
F.$\frac{38}{31}$
G.$-\frac{76}{-7}$
H.$1$
I.$38$
\testStop
\kluczStart
A
\kluczStop



\zadStart{Przykład z Wikieł P 4.2b moja wersja nr 745}
Obliczyć granicę $\lim\limits_{x\to\ 38}\frac{x^{2}-38^{2}}{(x-38)(x-33)}$.
\zadStop
\rozwStart{Patryk Wirkus}{Martyna Czarnobaj}
$$\frac{x^{2}-38^{2}}{(x-38)(x-33)}=\frac{x+38}{x-33}$$

$$\lim\limits_{x\to\ 38}\frac{x^{2}-38^{2}}{(x-38)(x-33)}=[\frac{0}{0}]=\lim\limits_{x\to\ 38}\frac{x+38}{x-33}=2 \cdot \frac{38}{38-33} = \frac{76}{5}$$
\rozwStop
\odpStart
$\frac{76}{5}$
\odpStop
\testStart
A.$\frac{76}{5}$
B.$\infty$
C.$-\infty$
D.$0$
E.$\frac{76}{-5}$
F.$\frac{38}{33}$
G.$-\frac{76}{-5}$
H.$1$
I.$38$
\testStop
\kluczStart
A
\kluczStop



\zadStart{Przykład z Wikieł P 4.2b moja wersja nr 746}
Obliczyć granicę $\lim\limits_{x\to\ 38}\frac{x^{2}-38^{2}}{(x-38)(x-35)}$.
\zadStop
\rozwStart{Patryk Wirkus}{Martyna Czarnobaj}
$$\frac{x^{2}-38^{2}}{(x-38)(x-35)}=\frac{x+38}{x-35}$$

$$\lim\limits_{x\to\ 38}\frac{x^{2}-38^{2}}{(x-38)(x-35)}=[\frac{0}{0}]=\lim\limits_{x\to\ 38}\frac{x+38}{x-35}=2 \cdot \frac{38}{38-35} = \frac{76}{3}$$
\rozwStop
\odpStart
$\frac{76}{3}$
\odpStop
\testStart
A.$\frac{76}{3}$
B.$\infty$
C.$-\infty$
D.$0$
E.$\frac{76}{-3}$
F.$\frac{38}{35}$
G.$-\frac{76}{-3}$
H.$1$
I.$38$
\testStop
\kluczStart
A
\kluczStop



\zadStart{Przykład z Wikieł P 4.2b moja wersja nr 747}
Obliczyć granicę $\lim\limits_{x\to\ 39}\frac{x^{2}-39^{2}}{(x-39)(x-2)}$.
\zadStop
\rozwStart{Patryk Wirkus}{Martyna Czarnobaj}
$$\frac{x^{2}-39^{2}}{(x-39)(x-2)}=\frac{x+39}{x-2}$$

$$\lim\limits_{x\to\ 39}\frac{x^{2}-39^{2}}{(x-39)(x-2)}=[\frac{0}{0}]=\lim\limits_{x\to\ 39}\frac{x+39}{x-2}=2 \cdot \frac{39}{39-2} = \frac{78}{37}$$
\rozwStop
\odpStart
$\frac{78}{37}$
\odpStop
\testStart
A.$\frac{78}{37}$
B.$\infty$
C.$-\infty$
D.$0$
E.$\frac{78}{-37}$
F.$\frac{39}{2}$
G.$-\frac{78}{-37}$
H.$1$
I.$39$
\testStop
\kluczStart
A
\kluczStop



\zadStart{Przykład z Wikieł P 4.2b moja wersja nr 748}
Obliczyć granicę $\lim\limits_{x\to\ 39}\frac{x^{2}-39^{2}}{(x-39)(x-4)}$.
\zadStop
\rozwStart{Patryk Wirkus}{Martyna Czarnobaj}
$$\frac{x^{2}-39^{2}}{(x-39)(x-4)}=\frac{x+39}{x-4}$$

$$\lim\limits_{x\to\ 39}\frac{x^{2}-39^{2}}{(x-39)(x-4)}=[\frac{0}{0}]=\lim\limits_{x\to\ 39}\frac{x+39}{x-4}=2 \cdot \frac{39}{39-4} = \frac{78}{35}$$
\rozwStop
\odpStart
$\frac{78}{35}$
\odpStop
\testStart
A.$\frac{78}{35}$
B.$\infty$
C.$-\infty$
D.$0$
E.$\frac{78}{-35}$
F.$\frac{39}{4}$
G.$-\frac{78}{-35}$
H.$1$
I.$39$
\testStop
\kluczStart
A
\kluczStop



\zadStart{Przykład z Wikieł P 4.2b moja wersja nr 749}
Obliczyć granicę $\lim\limits_{x\to\ 39}\frac{x^{2}-39^{2}}{(x-39)(x-5)}$.
\zadStop
\rozwStart{Patryk Wirkus}{Martyna Czarnobaj}
$$\frac{x^{2}-39^{2}}{(x-39)(x-5)}=\frac{x+39}{x-5}$$

$$\lim\limits_{x\to\ 39}\frac{x^{2}-39^{2}}{(x-39)(x-5)}=[\frac{0}{0}]=\lim\limits_{x\to\ 39}\frac{x+39}{x-5}=2 \cdot \frac{39}{39-5} = \frac{78}{34}$$
\rozwStop
\odpStart
$\frac{78}{34}$
\odpStop
\testStart
A.$\frac{78}{34}$
B.$\infty$
C.$-\infty$
D.$0$
E.$\frac{78}{-34}$
F.$\frac{39}{5}$
G.$-\frac{78}{-34}$
H.$1$
I.$39$
\testStop
\kluczStart
A
\kluczStop



\zadStart{Przykład z Wikieł P 4.2b moja wersja nr 750}
Obliczyć granicę $\lim\limits_{x\to\ 39}\frac{x^{2}-39^{2}}{(x-39)(x-7)}$.
\zadStop
\rozwStart{Patryk Wirkus}{Martyna Czarnobaj}
$$\frac{x^{2}-39^{2}}{(x-39)(x-7)}=\frac{x+39}{x-7}$$

$$\lim\limits_{x\to\ 39}\frac{x^{2}-39^{2}}{(x-39)(x-7)}=[\frac{0}{0}]=\lim\limits_{x\to\ 39}\frac{x+39}{x-7}=2 \cdot \frac{39}{39-7} = \frac{78}{32}$$
\rozwStop
\odpStart
$\frac{78}{32}$
\odpStop
\testStart
A.$\frac{78}{32}$
B.$\infty$
C.$-\infty$
D.$0$
E.$\frac{78}{-32}$
F.$\frac{39}{7}$
G.$-\frac{78}{-32}$
H.$1$
I.$39$
\testStop
\kluczStart
A
\kluczStop



\zadStart{Przykład z Wikieł P 4.2b moja wersja nr 751}
Obliczyć granicę $\lim\limits_{x\to\ 39}\frac{x^{2}-39^{2}}{(x-39)(x-8)}$.
\zadStop
\rozwStart{Patryk Wirkus}{Martyna Czarnobaj}
$$\frac{x^{2}-39^{2}}{(x-39)(x-8)}=\frac{x+39}{x-8}$$

$$\lim\limits_{x\to\ 39}\frac{x^{2}-39^{2}}{(x-39)(x-8)}=[\frac{0}{0}]=\lim\limits_{x\to\ 39}\frac{x+39}{x-8}=2 \cdot \frac{39}{39-8} = \frac{78}{31}$$
\rozwStop
\odpStart
$\frac{78}{31}$
\odpStop
\testStart
A.$\frac{78}{31}$
B.$\infty$
C.$-\infty$
D.$0$
E.$\frac{78}{-31}$
F.$\frac{39}{8}$
G.$-\frac{78}{-31}$
H.$1$
I.$39$
\testStop
\kluczStart
A
\kluczStop



\zadStart{Przykład z Wikieł P 4.2b moja wersja nr 752}
Obliczyć granicę $\lim\limits_{x\to\ 39}\frac{x^{2}-39^{2}}{(x-39)(x-10)}$.
\zadStop
\rozwStart{Patryk Wirkus}{Martyna Czarnobaj}
$$\frac{x^{2}-39^{2}}{(x-39)(x-10)}=\frac{x+39}{x-10}$$

$$\lim\limits_{x\to\ 39}\frac{x^{2}-39^{2}}{(x-39)(x-10)}=[\frac{0}{0}]=\lim\limits_{x\to\ 39}\frac{x+39}{x-10}=2 \cdot \frac{39}{39-10} = \frac{78}{29}$$
\rozwStop
\odpStart
$\frac{78}{29}$
\odpStop
\testStart
A.$\frac{78}{29}$
B.$\infty$
C.$-\infty$
D.$0$
E.$\frac{78}{-29}$
F.$\frac{39}{10}$
G.$-\frac{78}{-29}$
H.$1$
I.$39$
\testStop
\kluczStart
A
\kluczStop



\zadStart{Przykład z Wikieł P 4.2b moja wersja nr 753}
Obliczyć granicę $\lim\limits_{x\to\ 39}\frac{x^{2}-39^{2}}{(x-39)(x-11)}$.
\zadStop
\rozwStart{Patryk Wirkus}{Martyna Czarnobaj}
$$\frac{x^{2}-39^{2}}{(x-39)(x-11)}=\frac{x+39}{x-11}$$

$$\lim\limits_{x\to\ 39}\frac{x^{2}-39^{2}}{(x-39)(x-11)}=[\frac{0}{0}]=\lim\limits_{x\to\ 39}\frac{x+39}{x-11}=2 \cdot \frac{39}{39-11} = \frac{78}{28}$$
\rozwStop
\odpStart
$\frac{78}{28}$
\odpStop
\testStart
A.$\frac{78}{28}$
B.$\infty$
C.$-\infty$
D.$0$
E.$\frac{78}{-28}$
F.$\frac{39}{11}$
G.$-\frac{78}{-28}$
H.$1$
I.$39$
\testStop
\kluczStart
A
\kluczStop



\zadStart{Przykład z Wikieł P 4.2b moja wersja nr 754}
Obliczyć granicę $\lim\limits_{x\to\ 39}\frac{x^{2}-39^{2}}{(x-39)(x-14)}$.
\zadStop
\rozwStart{Patryk Wirkus}{Martyna Czarnobaj}
$$\frac{x^{2}-39^{2}}{(x-39)(x-14)}=\frac{x+39}{x-14}$$

$$\lim\limits_{x\to\ 39}\frac{x^{2}-39^{2}}{(x-39)(x-14)}=[\frac{0}{0}]=\lim\limits_{x\to\ 39}\frac{x+39}{x-14}=2 \cdot \frac{39}{39-14} = \frac{78}{25}$$
\rozwStop
\odpStart
$\frac{78}{25}$
\odpStop
\testStart
A.$\frac{78}{25}$
B.$\infty$
C.$-\infty$
D.$0$
E.$\frac{78}{-25}$
F.$\frac{39}{14}$
G.$-\frac{78}{-25}$
H.$1$
I.$39$
\testStop
\kluczStart
A
\kluczStop



\zadStart{Przykład z Wikieł P 4.2b moja wersja nr 755}
Obliczyć granicę $\lim\limits_{x\to\ 39}\frac{x^{2}-39^{2}}{(x-39)(x-16)}$.
\zadStop
\rozwStart{Patryk Wirkus}{Martyna Czarnobaj}
$$\frac{x^{2}-39^{2}}{(x-39)(x-16)}=\frac{x+39}{x-16}$$

$$\lim\limits_{x\to\ 39}\frac{x^{2}-39^{2}}{(x-39)(x-16)}=[\frac{0}{0}]=\lim\limits_{x\to\ 39}\frac{x+39}{x-16}=2 \cdot \frac{39}{39-16} = \frac{78}{23}$$
\rozwStop
\odpStart
$\frac{78}{23}$
\odpStop
\testStart
A.$\frac{78}{23}$
B.$\infty$
C.$-\infty$
D.$0$
E.$\frac{78}{-23}$
F.$\frac{39}{16}$
G.$-\frac{78}{-23}$
H.$1$
I.$39$
\testStop
\kluczStart
A
\kluczStop



\zadStart{Przykład z Wikieł P 4.2b moja wersja nr 756}
Obliczyć granicę $\lim\limits_{x\to\ 39}\frac{x^{2}-39^{2}}{(x-39)(x-17)}$.
\zadStop
\rozwStart{Patryk Wirkus}{Martyna Czarnobaj}
$$\frac{x^{2}-39^{2}}{(x-39)(x-17)}=\frac{x+39}{x-17}$$

$$\lim\limits_{x\to\ 39}\frac{x^{2}-39^{2}}{(x-39)(x-17)}=[\frac{0}{0}]=\lim\limits_{x\to\ 39}\frac{x+39}{x-17}=2 \cdot \frac{39}{39-17} = \frac{78}{22}$$
\rozwStop
\odpStart
$\frac{78}{22}$
\odpStop
\testStart
A.$\frac{78}{22}$
B.$\infty$
C.$-\infty$
D.$0$
E.$\frac{78}{-22}$
F.$\frac{39}{17}$
G.$-\frac{78}{-22}$
H.$1$
I.$39$
\testStop
\kluczStart
A
\kluczStop



\zadStart{Przykład z Wikieł P 4.2b moja wersja nr 757}
Obliczyć granicę $\lim\limits_{x\to\ 39}\frac{x^{2}-39^{2}}{(x-39)(x-19)}$.
\zadStop
\rozwStart{Patryk Wirkus}{Martyna Czarnobaj}
$$\frac{x^{2}-39^{2}}{(x-39)(x-19)}=\frac{x+39}{x-19}$$

$$\lim\limits_{x\to\ 39}\frac{x^{2}-39^{2}}{(x-39)(x-19)}=[\frac{0}{0}]=\lim\limits_{x\to\ 39}\frac{x+39}{x-19}=2 \cdot \frac{39}{39-19} = \frac{78}{20}$$
\rozwStop
\odpStart
$\frac{78}{20}$
\odpStop
\testStart
A.$\frac{78}{20}$
B.$\infty$
C.$-\infty$
D.$0$
E.$\frac{78}{-20}$
F.$\frac{39}{19}$
G.$-\frac{78}{-20}$
H.$1$
I.$39$
\testStop
\kluczStart
A
\kluczStop



\zadStart{Przykład z Wikieł P 4.2b moja wersja nr 758}
Obliczyć granicę $\lim\limits_{x\to\ 39}\frac{x^{2}-39^{2}}{(x-39)(x-20)}$.
\zadStop
\rozwStart{Patryk Wirkus}{Martyna Czarnobaj}
$$\frac{x^{2}-39^{2}}{(x-39)(x-20)}=\frac{x+39}{x-20}$$

$$\lim\limits_{x\to\ 39}\frac{x^{2}-39^{2}}{(x-39)(x-20)}=[\frac{0}{0}]=\lim\limits_{x\to\ 39}\frac{x+39}{x-20}=2 \cdot \frac{39}{39-20} = \frac{78}{19}$$
\rozwStop
\odpStart
$\frac{78}{19}$
\odpStop
\testStart
A.$\frac{78}{19}$
B.$\infty$
C.$-\infty$
D.$0$
E.$\frac{78}{-19}$
F.$\frac{39}{20}$
G.$-\frac{78}{-19}$
H.$1$
I.$39$
\testStop
\kluczStart
A
\kluczStop



\zadStart{Przykład z Wikieł P 4.2b moja wersja nr 759}
Obliczyć granicę $\lim\limits_{x\to\ 39}\frac{x^{2}-39^{2}}{(x-39)(x-22)}$.
\zadStop
\rozwStart{Patryk Wirkus}{Martyna Czarnobaj}
$$\frac{x^{2}-39^{2}}{(x-39)(x-22)}=\frac{x+39}{x-22}$$

$$\lim\limits_{x\to\ 39}\frac{x^{2}-39^{2}}{(x-39)(x-22)}=[\frac{0}{0}]=\lim\limits_{x\to\ 39}\frac{x+39}{x-22}=2 \cdot \frac{39}{39-22} = \frac{78}{17}$$
\rozwStop
\odpStart
$\frac{78}{17}$
\odpStop
\testStart
A.$\frac{78}{17}$
B.$\infty$
C.$-\infty$
D.$0$
E.$\frac{78}{-17}$
F.$\frac{39}{22}$
G.$-\frac{78}{-17}$
H.$1$
I.$39$
\testStop
\kluczStart
A
\kluczStop



\zadStart{Przykład z Wikieł P 4.2b moja wersja nr 760}
Obliczyć granicę $\lim\limits_{x\to\ 39}\frac{x^{2}-39^{2}}{(x-39)(x-23)}$.
\zadStop
\rozwStart{Patryk Wirkus}{Martyna Czarnobaj}
$$\frac{x^{2}-39^{2}}{(x-39)(x-23)}=\frac{x+39}{x-23}$$

$$\lim\limits_{x\to\ 39}\frac{x^{2}-39^{2}}{(x-39)(x-23)}=[\frac{0}{0}]=\lim\limits_{x\to\ 39}\frac{x+39}{x-23}=2 \cdot \frac{39}{39-23} = \frac{78}{16}$$
\rozwStop
\odpStart
$\frac{78}{16}$
\odpStop
\testStart
A.$\frac{78}{16}$
B.$\infty$
C.$-\infty$
D.$0$
E.$\frac{78}{-16}$
F.$\frac{39}{23}$
G.$-\frac{78}{-16}$
H.$1$
I.$39$
\testStop
\kluczStart
A
\kluczStop



\zadStart{Przykład z Wikieł P 4.2b moja wersja nr 761}
Obliczyć granicę $\lim\limits_{x\to\ 39}\frac{x^{2}-39^{2}}{(x-39)(x-25)}$.
\zadStop
\rozwStart{Patryk Wirkus}{Martyna Czarnobaj}
$$\frac{x^{2}-39^{2}}{(x-39)(x-25)}=\frac{x+39}{x-25}$$

$$\lim\limits_{x\to\ 39}\frac{x^{2}-39^{2}}{(x-39)(x-25)}=[\frac{0}{0}]=\lim\limits_{x\to\ 39}\frac{x+39}{x-25}=2 \cdot \frac{39}{39-25} = \frac{78}{14}$$
\rozwStop
\odpStart
$\frac{78}{14}$
\odpStop
\testStart
A.$\frac{78}{14}$
B.$\infty$
C.$-\infty$
D.$0$
E.$\frac{78}{-14}$
F.$\frac{39}{25}$
G.$-\frac{78}{-14}$
H.$1$
I.$39$
\testStop
\kluczStart
A
\kluczStop



\zadStart{Przykład z Wikieł P 4.2b moja wersja nr 762}
Obliczyć granicę $\lim\limits_{x\to\ 39}\frac{x^{2}-39^{2}}{(x-39)(x-28)}$.
\zadStop
\rozwStart{Patryk Wirkus}{Martyna Czarnobaj}
$$\frac{x^{2}-39^{2}}{(x-39)(x-28)}=\frac{x+39}{x-28}$$

$$\lim\limits_{x\to\ 39}\frac{x^{2}-39^{2}}{(x-39)(x-28)}=[\frac{0}{0}]=\lim\limits_{x\to\ 39}\frac{x+39}{x-28}=2 \cdot \frac{39}{39-28} = \frac{78}{11}$$
\rozwStop
\odpStart
$\frac{78}{11}$
\odpStop
\testStart
A.$\frac{78}{11}$
B.$\infty$
C.$-\infty$
D.$0$
E.$\frac{78}{-11}$
F.$\frac{39}{28}$
G.$-\frac{78}{-11}$
H.$1$
I.$39$
\testStop
\kluczStart
A
\kluczStop



\zadStart{Przykład z Wikieł P 4.2b moja wersja nr 763}
Obliczyć granicę $\lim\limits_{x\to\ 39}\frac{x^{2}-39^{2}}{(x-39)(x-29)}$.
\zadStop
\rozwStart{Patryk Wirkus}{Martyna Czarnobaj}
$$\frac{x^{2}-39^{2}}{(x-39)(x-29)}=\frac{x+39}{x-29}$$

$$\lim\limits_{x\to\ 39}\frac{x^{2}-39^{2}}{(x-39)(x-29)}=[\frac{0}{0}]=\lim\limits_{x\to\ 39}\frac{x+39}{x-29}=2 \cdot \frac{39}{39-29} = \frac{78}{10}$$
\rozwStop
\odpStart
$\frac{78}{10}$
\odpStop
\testStart
A.$\frac{78}{10}$
B.$\infty$
C.$-\infty$
D.$0$
E.$\frac{78}{-10}$
F.$\frac{39}{29}$
G.$-\frac{78}{-10}$
H.$1$
I.$39$
\testStop
\kluczStart
A
\kluczStop



\zadStart{Przykład z Wikieł P 4.2b moja wersja nr 764}
Obliczyć granicę $\lim\limits_{x\to\ 39}\frac{x^{2}-39^{2}}{(x-39)(x-31)}$.
\zadStop
\rozwStart{Patryk Wirkus}{Martyna Czarnobaj}
$$\frac{x^{2}-39^{2}}{(x-39)(x-31)}=\frac{x+39}{x-31}$$

$$\lim\limits_{x\to\ 39}\frac{x^{2}-39^{2}}{(x-39)(x-31)}=[\frac{0}{0}]=\lim\limits_{x\to\ 39}\frac{x+39}{x-31}=2 \cdot \frac{39}{39-31} = \frac{78}{8}$$
\rozwStop
\odpStart
$\frac{78}{8}$
\odpStop
\testStart
A.$\frac{78}{8}$
B.$\infty$
C.$-\infty$
D.$0$
E.$\frac{78}{-8}$
F.$\frac{39}{31}$
G.$-\frac{78}{-8}$
H.$1$
I.$39$
\testStop
\kluczStart
A
\kluczStop



\zadStart{Przykład z Wikieł P 4.2b moja wersja nr 765}
Obliczyć granicę $\lim\limits_{x\to\ 39}\frac{x^{2}-39^{2}}{(x-39)(x-32)}$.
\zadStop
\rozwStart{Patryk Wirkus}{Martyna Czarnobaj}
$$\frac{x^{2}-39^{2}}{(x-39)(x-32)}=\frac{x+39}{x-32}$$

$$\lim\limits_{x\to\ 39}\frac{x^{2}-39^{2}}{(x-39)(x-32)}=[\frac{0}{0}]=\lim\limits_{x\to\ 39}\frac{x+39}{x-32}=2 \cdot \frac{39}{39-32} = \frac{78}{7}$$
\rozwStop
\odpStart
$\frac{78}{7}$
\odpStop
\testStart
A.$\frac{78}{7}$
B.$\infty$
C.$-\infty$
D.$0$
E.$\frac{78}{-7}$
F.$\frac{39}{32}$
G.$-\frac{78}{-7}$
H.$1$
I.$39$
\testStop
\kluczStart
A
\kluczStop



\zadStart{Przykład z Wikieł P 4.2b moja wersja nr 766}
Obliczyć granicę $\lim\limits_{x\to\ 39}\frac{x^{2}-39^{2}}{(x-39)(x-34)}$.
\zadStop
\rozwStart{Patryk Wirkus}{Martyna Czarnobaj}
$$\frac{x^{2}-39^{2}}{(x-39)(x-34)}=\frac{x+39}{x-34}$$

$$\lim\limits_{x\to\ 39}\frac{x^{2}-39^{2}}{(x-39)(x-34)}=[\frac{0}{0}]=\lim\limits_{x\to\ 39}\frac{x+39}{x-34}=2 \cdot \frac{39}{39-34} = \frac{78}{5}$$
\rozwStop
\odpStart
$\frac{78}{5}$
\odpStop
\testStart
A.$\frac{78}{5}$
B.$\infty$
C.$-\infty$
D.$0$
E.$\frac{78}{-5}$
F.$\frac{39}{34}$
G.$-\frac{78}{-5}$
H.$1$
I.$39$
\testStop
\kluczStart
A
\kluczStop



\zadStart{Przykład z Wikieł P 4.2b moja wersja nr 767}
Obliczyć granicę $\lim\limits_{x\to\ 39}\frac{x^{2}-39^{2}}{(x-39)(x-35)}$.
\zadStop
\rozwStart{Patryk Wirkus}{Martyna Czarnobaj}
$$\frac{x^{2}-39^{2}}{(x-39)(x-35)}=\frac{x+39}{x-35}$$

$$\lim\limits_{x\to\ 39}\frac{x^{2}-39^{2}}{(x-39)(x-35)}=[\frac{0}{0}]=\lim\limits_{x\to\ 39}\frac{x+39}{x-35}=2 \cdot \frac{39}{39-35} = \frac{78}{4}$$
\rozwStop
\odpStart
$\frac{78}{4}$
\odpStop
\testStart
A.$\frac{78}{4}$
B.$\infty$
C.$-\infty$
D.$0$
E.$\frac{78}{-4}$
F.$\frac{39}{35}$
G.$-\frac{78}{-4}$
H.$1$
I.$39$
\testStop
\kluczStart
A
\kluczStop



\zadStart{Przykład z Wikieł P 4.2b moja wersja nr 768}
Obliczyć granicę $\lim\limits_{x\to\ 39}\frac{x^{2}-39^{2}}{(x-39)(x-37)}$.
\zadStop
\rozwStart{Patryk Wirkus}{Martyna Czarnobaj}
$$\frac{x^{2}-39^{2}}{(x-39)(x-37)}=\frac{x+39}{x-37}$$

$$\lim\limits_{x\to\ 39}\frac{x^{2}-39^{2}}{(x-39)(x-37)}=[\frac{0}{0}]=\lim\limits_{x\to\ 39}\frac{x+39}{x-37}=2 \cdot \frac{39}{39-37} = \frac{78}{2}$$
\rozwStop
\odpStart
$\frac{78}{2}$
\odpStop
\testStart
A.$\frac{78}{2}$
B.$\infty$
C.$-\infty$
D.$0$
E.$\frac{78}{-2}$
F.$\frac{39}{37}$
G.$-\frac{78}{-2}$
H.$1$
I.$39$
\testStop
\kluczStart
A
\kluczStop



\zadStart{Przykład z Wikieł P 4.2b moja wersja nr 769}
Obliczyć granicę $\lim\limits_{x\to\ 40}\frac{x^{2}-40^{2}}{(x-40)(x-3)}$.
\zadStop
\rozwStart{Patryk Wirkus}{Martyna Czarnobaj}
$$\frac{x^{2}-40^{2}}{(x-40)(x-3)}=\frac{x+40}{x-3}$$

$$\lim\limits_{x\to\ 40}\frac{x^{2}-40^{2}}{(x-40)(x-3)}=[\frac{0}{0}]=\lim\limits_{x\to\ 40}\frac{x+40}{x-3}=2 \cdot \frac{40}{40-3} = \frac{80}{37}$$
\rozwStop
\odpStart
$\frac{80}{37}$
\odpStop
\testStart
A.$\frac{80}{37}$
B.$\infty$
C.$-\infty$
D.$0$
E.$\frac{80}{-37}$
F.$\frac{40}{3}$
G.$-\frac{80}{-37}$
H.$1$
I.$40$
\testStop
\kluczStart
A
\kluczStop



\zadStart{Przykład z Wikieł P 4.2b moja wersja nr 770}
Obliczyć granicę $\lim\limits_{x\to\ 40}\frac{x^{2}-40^{2}}{(x-40)(x-7)}$.
\zadStop
\rozwStart{Patryk Wirkus}{Martyna Czarnobaj}
$$\frac{x^{2}-40^{2}}{(x-40)(x-7)}=\frac{x+40}{x-7}$$

$$\lim\limits_{x\to\ 40}\frac{x^{2}-40^{2}}{(x-40)(x-7)}=[\frac{0}{0}]=\lim\limits_{x\to\ 40}\frac{x+40}{x-7}=2 \cdot \frac{40}{40-7} = \frac{80}{33}$$
\rozwStop
\odpStart
$\frac{80}{33}$
\odpStop
\testStart
A.$\frac{80}{33}$
B.$\infty$
C.$-\infty$
D.$0$
E.$\frac{80}{-33}$
F.$\frac{40}{7}$
G.$-\frac{80}{-33}$
H.$1$
I.$40$
\testStop
\kluczStart
A
\kluczStop



\zadStart{Przykład z Wikieł P 4.2b moja wersja nr 771}
Obliczyć granicę $\lim\limits_{x\to\ 40}\frac{x^{2}-40^{2}}{(x-40)(x-9)}$.
\zadStop
\rozwStart{Patryk Wirkus}{Martyna Czarnobaj}
$$\frac{x^{2}-40^{2}}{(x-40)(x-9)}=\frac{x+40}{x-9}$$

$$\lim\limits_{x\to\ 40}\frac{x^{2}-40^{2}}{(x-40)(x-9)}=[\frac{0}{0}]=\lim\limits_{x\to\ 40}\frac{x+40}{x-9}=2 \cdot \frac{40}{40-9} = \frac{80}{31}$$
\rozwStop
\odpStart
$\frac{80}{31}$
\odpStop
\testStart
A.$\frac{80}{31}$
B.$\infty$
C.$-\infty$
D.$0$
E.$\frac{80}{-31}$
F.$\frac{40}{9}$
G.$-\frac{80}{-31}$
H.$1$
I.$40$
\testStop
\kluczStart
A
\kluczStop



\zadStart{Przykład z Wikieł P 4.2b moja wersja nr 772}
Obliczyć granicę $\lim\limits_{x\to\ 40}\frac{x^{2}-40^{2}}{(x-40)(x-11)}$.
\zadStop
\rozwStart{Patryk Wirkus}{Martyna Czarnobaj}
$$\frac{x^{2}-40^{2}}{(x-40)(x-11)}=\frac{x+40}{x-11}$$

$$\lim\limits_{x\to\ 40}\frac{x^{2}-40^{2}}{(x-40)(x-11)}=[\frac{0}{0}]=\lim\limits_{x\to\ 40}\frac{x+40}{x-11}=2 \cdot \frac{40}{40-11} = \frac{80}{29}$$
\rozwStop
\odpStart
$\frac{80}{29}$
\odpStop
\testStart
A.$\frac{80}{29}$
B.$\infty$
C.$-\infty$
D.$0$
E.$\frac{80}{-29}$
F.$\frac{40}{11}$
G.$-\frac{80}{-29}$
H.$1$
I.$40$
\testStop
\kluczStart
A
\kluczStop



\zadStart{Przykład z Wikieł P 4.2b moja wersja nr 773}
Obliczyć granicę $\lim\limits_{x\to\ 40}\frac{x^{2}-40^{2}}{(x-40)(x-13)}$.
\zadStop
\rozwStart{Patryk Wirkus}{Martyna Czarnobaj}
$$\frac{x^{2}-40^{2}}{(x-40)(x-13)}=\frac{x+40}{x-13}$$

$$\lim\limits_{x\to\ 40}\frac{x^{2}-40^{2}}{(x-40)(x-13)}=[\frac{0}{0}]=\lim\limits_{x\to\ 40}\frac{x+40}{x-13}=2 \cdot \frac{40}{40-13} = \frac{80}{27}$$
\rozwStop
\odpStart
$\frac{80}{27}$
\odpStop
\testStart
A.$\frac{80}{27}$
B.$\infty$
C.$-\infty$
D.$0$
E.$\frac{80}{-27}$
F.$\frac{40}{13}$
G.$-\frac{80}{-27}$
H.$1$
I.$40$
\testStop
\kluczStart
A
\kluczStop



\zadStart{Przykład z Wikieł P 4.2b moja wersja nr 774}
Obliczyć granicę $\lim\limits_{x\to\ 40}\frac{x^{2}-40^{2}}{(x-40)(x-17)}$.
\zadStop
\rozwStart{Patryk Wirkus}{Martyna Czarnobaj}
$$\frac{x^{2}-40^{2}}{(x-40)(x-17)}=\frac{x+40}{x-17}$$

$$\lim\limits_{x\to\ 40}\frac{x^{2}-40^{2}}{(x-40)(x-17)}=[\frac{0}{0}]=\lim\limits_{x\to\ 40}\frac{x+40}{x-17}=2 \cdot \frac{40}{40-17} = \frac{80}{23}$$
\rozwStop
\odpStart
$\frac{80}{23}$
\odpStop
\testStart
A.$\frac{80}{23}$
B.$\infty$
C.$-\infty$
D.$0$
E.$\frac{80}{-23}$
F.$\frac{40}{17}$
G.$-\frac{80}{-23}$
H.$1$
I.$40$
\testStop
\kluczStart
A
\kluczStop



\zadStart{Przykład z Wikieł P 4.2b moja wersja nr 775}
Obliczyć granicę $\lim\limits_{x\to\ 40}\frac{x^{2}-40^{2}}{(x-40)(x-19)}$.
\zadStop
\rozwStart{Patryk Wirkus}{Martyna Czarnobaj}
$$\frac{x^{2}-40^{2}}{(x-40)(x-19)}=\frac{x+40}{x-19}$$

$$\lim\limits_{x\to\ 40}\frac{x^{2}-40^{2}}{(x-40)(x-19)}=[\frac{0}{0}]=\lim\limits_{x\to\ 40}\frac{x+40}{x-19}=2 \cdot \frac{40}{40-19} = \frac{80}{21}$$
\rozwStop
\odpStart
$\frac{80}{21}$
\odpStop
\testStart
A.$\frac{80}{21}$
B.$\infty$
C.$-\infty$
D.$0$
E.$\frac{80}{-21}$
F.$\frac{40}{19}$
G.$-\frac{80}{-21}$
H.$1$
I.$40$
\testStop
\kluczStart
A
\kluczStop



\zadStart{Przykład z Wikieł P 4.2b moja wersja nr 776}
Obliczyć granicę $\lim\limits_{x\to\ 40}\frac{x^{2}-40^{2}}{(x-40)(x-21)}$.
\zadStop
\rozwStart{Patryk Wirkus}{Martyna Czarnobaj}
$$\frac{x^{2}-40^{2}}{(x-40)(x-21)}=\frac{x+40}{x-21}$$

$$\lim\limits_{x\to\ 40}\frac{x^{2}-40^{2}}{(x-40)(x-21)}=[\frac{0}{0}]=\lim\limits_{x\to\ 40}\frac{x+40}{x-21}=2 \cdot \frac{40}{40-21} = \frac{80}{19}$$
\rozwStop
\odpStart
$\frac{80}{19}$
\odpStop
\testStart
A.$\frac{80}{19}$
B.$\infty$
C.$-\infty$
D.$0$
E.$\frac{80}{-19}$
F.$\frac{40}{21}$
G.$-\frac{80}{-19}$
H.$1$
I.$40$
\testStop
\kluczStart
A
\kluczStop



\zadStart{Przykład z Wikieł P 4.2b moja wersja nr 777}
Obliczyć granicę $\lim\limits_{x\to\ 40}\frac{x^{2}-40^{2}}{(x-40)(x-23)}$.
\zadStop
\rozwStart{Patryk Wirkus}{Martyna Czarnobaj}
$$\frac{x^{2}-40^{2}}{(x-40)(x-23)}=\frac{x+40}{x-23}$$

$$\lim\limits_{x\to\ 40}\frac{x^{2}-40^{2}}{(x-40)(x-23)}=[\frac{0}{0}]=\lim\limits_{x\to\ 40}\frac{x+40}{x-23}=2 \cdot \frac{40}{40-23} = \frac{80}{17}$$
\rozwStop
\odpStart
$\frac{80}{17}$
\odpStop
\testStart
A.$\frac{80}{17}$
B.$\infty$
C.$-\infty$
D.$0$
E.$\frac{80}{-17}$
F.$\frac{40}{23}$
G.$-\frac{80}{-17}$
H.$1$
I.$40$
\testStop
\kluczStart
A
\kluczStop



\zadStart{Przykład z Wikieł P 4.2b moja wersja nr 778}
Obliczyć granicę $\lim\limits_{x\to\ 40}\frac{x^{2}-40^{2}}{(x-40)(x-27)}$.
\zadStop
\rozwStart{Patryk Wirkus}{Martyna Czarnobaj}
$$\frac{x^{2}-40^{2}}{(x-40)(x-27)}=\frac{x+40}{x-27}$$

$$\lim\limits_{x\to\ 40}\frac{x^{2}-40^{2}}{(x-40)(x-27)}=[\frac{0}{0}]=\lim\limits_{x\to\ 40}\frac{x+40}{x-27}=2 \cdot \frac{40}{40-27} = \frac{80}{13}$$
\rozwStop
\odpStart
$\frac{80}{13}$
\odpStop
\testStart
A.$\frac{80}{13}$
B.$\infty$
C.$-\infty$
D.$0$
E.$\frac{80}{-13}$
F.$\frac{40}{27}$
G.$-\frac{80}{-13}$
H.$1$
I.$40$
\testStop
\kluczStart
A
\kluczStop



\zadStart{Przykład z Wikieł P 4.2b moja wersja nr 779}
Obliczyć granicę $\lim\limits_{x\to\ 40}\frac{x^{2}-40^{2}}{(x-40)(x-29)}$.
\zadStop
\rozwStart{Patryk Wirkus}{Martyna Czarnobaj}
$$\frac{x^{2}-40^{2}}{(x-40)(x-29)}=\frac{x+40}{x-29}$$

$$\lim\limits_{x\to\ 40}\frac{x^{2}-40^{2}}{(x-40)(x-29)}=[\frac{0}{0}]=\lim\limits_{x\to\ 40}\frac{x+40}{x-29}=2 \cdot \frac{40}{40-29} = \frac{80}{11}$$
\rozwStop
\odpStart
$\frac{80}{11}$
\odpStop
\testStart
A.$\frac{80}{11}$
B.$\infty$
C.$-\infty$
D.$0$
E.$\frac{80}{-11}$
F.$\frac{40}{29}$
G.$-\frac{80}{-11}$
H.$1$
I.$40$
\testStop
\kluczStart
A
\kluczStop



\zadStart{Przykład z Wikieł P 4.2b moja wersja nr 780}
Obliczyć granicę $\lim\limits_{x\to\ 40}\frac{x^{2}-40^{2}}{(x-40)(x-31)}$.
\zadStop
\rozwStart{Patryk Wirkus}{Martyna Czarnobaj}
$$\frac{x^{2}-40^{2}}{(x-40)(x-31)}=\frac{x+40}{x-31}$$

$$\lim\limits_{x\to\ 40}\frac{x^{2}-40^{2}}{(x-40)(x-31)}=[\frac{0}{0}]=\lim\limits_{x\to\ 40}\frac{x+40}{x-31}=2 \cdot \frac{40}{40-31} = \frac{80}{9}$$
\rozwStop
\odpStart
$\frac{80}{9}$
\odpStop
\testStart
A.$\frac{80}{9}$
B.$\infty$
C.$-\infty$
D.$0$
E.$\frac{80}{-9}$
F.$\frac{40}{31}$
G.$-\frac{80}{-9}$
H.$1$
I.$40$
\testStop
\kluczStart
A
\kluczStop



\zadStart{Przykład z Wikieł P 4.2b moja wersja nr 781}
Obliczyć granicę $\lim\limits_{x\to\ 40}\frac{x^{2}-40^{2}}{(x-40)(x-33)}$.
\zadStop
\rozwStart{Patryk Wirkus}{Martyna Czarnobaj}
$$\frac{x^{2}-40^{2}}{(x-40)(x-33)}=\frac{x+40}{x-33}$$

$$\lim\limits_{x\to\ 40}\frac{x^{2}-40^{2}}{(x-40)(x-33)}=[\frac{0}{0}]=\lim\limits_{x\to\ 40}\frac{x+40}{x-33}=2 \cdot \frac{40}{40-33} = \frac{80}{7}$$
\rozwStop
\odpStart
$\frac{80}{7}$
\odpStop
\testStart
A.$\frac{80}{7}$
B.$\infty$
C.$-\infty$
D.$0$
E.$\frac{80}{-7}$
F.$\frac{40}{33}$
G.$-\frac{80}{-7}$
H.$1$
I.$40$
\testStop
\kluczStart
A
\kluczStop



\zadStart{Przykład z Wikieł P 4.2b moja wersja nr 782}
Obliczyć granicę $\lim\limits_{x\to\ 40}\frac{x^{2}-40^{2}}{(x-40)(x-37)}$.
\zadStop
\rozwStart{Patryk Wirkus}{Martyna Czarnobaj}
$$\frac{x^{2}-40^{2}}{(x-40)(x-37)}=\frac{x+40}{x-37}$$

$$\lim\limits_{x\to\ 40}\frac{x^{2}-40^{2}}{(x-40)(x-37)}=[\frac{0}{0}]=\lim\limits_{x\to\ 40}\frac{x+40}{x-37}=2 \cdot \frac{40}{40-37} = \frac{80}{3}$$
\rozwStop
\odpStart
$\frac{80}{3}$
\odpStop
\testStart
A.$\frac{80}{3}$
B.$\infty$
C.$-\infty$
D.$0$
E.$\frac{80}{-3}$
F.$\frac{40}{37}$
G.$-\frac{80}{-3}$
H.$1$
I.$40$
\testStop
\kluczStart
A
\kluczStop





\end{document}
