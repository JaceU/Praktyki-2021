\documentclass[12pt, a4paper]{article}
\usepackage[utf8]{inputenc}
\usepackage{polski}

\usepackage{amsthm}  %pakiet do tworzenia twierdzeń itp.
\usepackage{amsmath} %pakiet do niektórych symboli matematycznych
\usepackage{amssymb} %pakiet do symboli mat., np. \nsubseteq
\usepackage{amsfonts}
\usepackage{graphicx} %obsługa plików graficznych z rozszerzeniem png, jpg
\theoremstyle{definition} %styl dla definicji
\newtheorem{zad}{} 
\title{Multizestaw zadań}
\author{Jacek Jabłoński}
%\date{\today}
\date{}
\newcounter{liczniksekcji}
\newcommand{\kategoria}[1]{\section{#1}} %olreślamy nazwę kateforii zadań
\newcommand{\zadStart}[1]{\begin{zad}#1\newline} %oznaczenie początku zadania
\newcommand{\zadStop}{\end{zad}}   %oznaczenie końca zadania
%Makra opcjonarne (nie muszą występować):
\newcommand{\rozwStart}[2]{\noindent \textbf{Rozwiązanie (autor #1 , recenzent #2): }\newline} %oznaczenie początku rozwiązania, opcjonarnie można wprowadzić informację o autorze rozwiązania zadania i recenzencie poprawności wykonania rozwiązania zadania
\newcommand{\rozwStop}{\newline}                                            %oznaczenie końca rozwiązania
\newcommand{\odpStart}{\noindent \textbf{Odpowiedź:}\newline}    %oznaczenie początku odpowiedzi końcowej (wypisanie wyniku)
\newcommand{\odpStop}{\newline}                                             %oznaczenie końca odpowiedzi końcowej (wypisanie wyniku)
\newcommand{\testStart}{\noindent \textbf{Test:}\newline} %ewentualne możliwe opcje odpowiedzi testowej: A. ? B. ? C. ? D. ? itd.
\newcommand{\testStop}{\newline} %koniec wprowadzania odpowiedzi testowych
\newcommand{\kluczStart}{\noindent \textbf{Test poprawna odpowiedź:}\newline} %klucz, poprawna odpowiedź pytania testowego (jedna literka): A lub B lub C lub D itd.
\newcommand{\kluczStop}{\newline} %koniec poprawnej odpowiedzi pytania testowego 
\newcommand{\wstawGrafike}[2]{\begin{figure}[h] \includegraphics[scale=#2] {#1} \end{figure}} %gdyby była potrzeba wstawienia obrazka, parametry: nazwa pliku, skala (jak nie wiesz co wpisać, to wpisz 1)

\begin{document}
\maketitle


\kategoria{Wikieł/z1.84b}
\zadStart{Zadanie z Wikieł z1.84b) moja wersja nr [nrWersji]}
%[p1]:[2,3,4,5,6,7,8]
%[p2]:[2,3,4,5,6,7,8,9,10]
%[r1]=([p1]-5)*5
%[r2]=[r1]-[p2]
%[r22]=[r2]*(-1)
%[r5]=0
%[r5b]=[r5] + 2
%[r5c]=[r5] + 3
%[r5d]=[r5] + 4
%[r5e]=[r5] + 5
%[r5f]=[r5] + 6
%[r5g]=[r5] - 1
%[r5h]=[r5] - 2 
%[r5i]=[r5] - 3
%[r22]>0
Rozwiązać równanie:
b) $[p1] \cdot 5^x - [p2] \cdot 5^{x-1} = 5^{x+1} - \frac{[r22]}{5}$
\zadStop
\rozwStart{Jacek Jabłoński}{}
$$[p1] \cdot 5^x - [p2] \cdot 5^{x-1} = 5^{x+1} - \frac{[r22]}{5}$$
$$[p1] \cdot 5^x - [p2] \cdot 5^x \cdot 5^{-1} - 5^x \cdot 5 = - \frac{[r22]}{5}$$
$$5^x ( [p1] - \frac{[p2]}{5} - 5) = - \frac{[r22]}{5}$$
$$5^x ( \frac{[r1]}{5} - \frac{[p2]}{5} ) = -\frac{[r22]}{5} $$
$$5^x (\frac{[r2]}{5}) = -\frac{[r22]}{5} $$
$$5^x = -\frac{[r22]}{5} \cdot \frac{5}{[r2]} $$
$$5^x = 1 $$
$$5^x = 5^0$$
$$x=0$$
\rozwStop
\odpStart
$$x=0$$
\odpStop
\testStart
A. $$x = 0$$
B. $$x = [r5b]$$
C. $$x = [r5c]$$
D. $$x = [r5d]$$
E. $$x = [r5e]$$
F. $$x = [r5f]$$
G. $$x = [r5g]$$
H. $$x = [r5h]$$
I. $$x = [r5i]$$
\testStop
\kluczStart
A
\kluczStop



\end{document}