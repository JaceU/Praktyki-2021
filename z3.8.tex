\documentclass[12pt, a4paper]{article}
\usepackage[utf8]{inputenc}
\usepackage{polski}

\usepackage{amsthm}  %pakiet do tworzenia twierdzeń itp.
\usepackage{amsmath} %pakiet do niektórych symboli matematycznych
\usepackage{amssymb} %pakiet do symboli mat., np. \nsubseteq
\usepackage{amsfonts}
\usepackage{graphicx} %obsługa plików graficznych z rozszerzeniem png, jpg
\theoremstyle{definition} %styl dla definicji
\newtheorem{zad}{} 
\title{Multizestaw zadań}
\author{Robert Fidytek}
%\date{\today}
\date{}
\newcounter{liczniksekcji}
\newcommand{\kategoria}[1]{\section{#1}} %olreślamy nazwę kateforii zadań
\newcommand{\zadStart}[1]{\begin{zad}#1\newline} %oznaczenie początku zadania
\newcommand{\zadStop}{\end{zad}}   %oznaczenie końca zadania
%Makra opcjonarne (nie muszą występować):
\newcommand{\rozwStart}[2]{\noindent \textbf{Rozwiązanie (autor #1 , recenzent #2): }\newline} %oznaczenie początku rozwiązania, opcjonarnie można wprowadzić informację o autorze rozwiązania zadania i recenzencie poprawności wykonania rozwiązania zadania
\newcommand{\rozwStop}{\newline}                                            %oznaczenie końca rozwiązania
\newcommand{\odpStart}{\noindent \textbf{Odpowiedź:}\newline}    %oznaczenie początku odpowiedzi końcowej (wypisanie wyniku)
\newcommand{\odpStop}{\newline}                                             %oznaczenie końca odpowiedzi końcowej (wypisanie wyniku)
\newcommand{\testStart}{\noindent \textbf{Test:}\newline} %ewentualne możliwe opcje odpowiedzi testowej: A. ? B. ? C. ? D. ? itd.
\newcommand{\testStop}{\newline} %koniec wprowadzania odpowiedzi testowych
\newcommand{\kluczStart}{\noindent \textbf{Test poprawna odpowiedź:}\newline} %klucz, poprawna odpowiedź pytania testowego (jedna literka): A lub B lub C lub D itd.
\newcommand{\kluczStop}{\newline} %koniec poprawnej odpowiedzi pytania testowego 
\newcommand{\wstawGrafike}[2]{\begin{figure}[h] \includegraphics[scale=#2] {#1} \end{figure}} %gdyby była potrzeba wstawienia obrazka, parametry: nazwa pliku, skala (jak nie wiesz co wpisać, to wpisz 1)

\begin{document}
\maketitle


\kategoria{Wikieł/Z3.8}
\zadStart{Zadanie z Wikieł Z 3.8 ) moja wersja nr [nrWersji]}
%[p1]:[1,2,3,4,6,7,8,10,11,12,14]
%[a]=random.randint(1,10)
%[b]=random.randint(1,10)
%[c]=random.randint(1,10)
%[d]=random.randint(1,10)
%[a2]=[a]*[p1]+[b]+[c]+[d]
Zbadać monotoniczność ciągu określonego rekurencyjnie.
$$
 \left\{ \begin{array}{ll}
a_{1}= [p1] & \\
a_{n+1}=[a]a_{n}+[b]n^{2}+[c]n+[d]  & \mbox{dla }n\geq1
\end{array} \right.
$$
\zadStop
\rozwStart{Wojciech Przybylski}{}
$$a_{2}=a_{1+1}=[a]\cdot[p1]+[b]\cdot(1)^{2}+[c]\cdot1+[d]=[a2] $$
$$\mbox{Dowód indukcyjny:}$$
$$\mbox{I Sprawdzenie } [a2]=a_{2}>a_{1}=[a]$$
$$\mbox{II Założenie niech }n\geq1 \mbox{ takie, że } a_{n+1}>a_{n}$$
$$\mbox{III Krok indukcyjny }\mbox{Udowodnimy, że } a_{n+2}>a_{n+1}$$
$$a_{n+2}=[a]a_{n+1}+[b](n+2)^{2}+[c](n+2)+[d]=$$
$$=[a]a_{n+1}+[b](n^{2}+2n+1+2n+3)+[c](n+1+1)+[d]=$$
$$=[a]a_{n+1}+[b](n+1)^{2}+[c](n+1)+[d]+[b](2n+4)>$$
$$---\mbox{z założenia }a_{n+1}>a_{n} \mbox{ oraz } [b](2n+4)>0---$$
$$>[a]a_{n}+[b](n+1)^{2}+[c](n+1)+[d]=a_{n+1}$$
$$a_{n+2}>a_{n+1}$$
$$\mbox{Dowód indukcyjny jest spełniony ciąg rekurencyjny jest rosnący}$$
\rozwStop
\odpStart
Ciąg rekurencyjny jest ciągiem rosnącym.
\odpStop
\testStart
A. Ciąg rekurencyjny jest ciągiem rosnącym.\\
B. Ciąg rekurencyjny jest ciągiem malejącym.\\
C. Ciąg rekurencyjny jest ciągiem nierosnącym.\\
D. Ciąg rekurencyjny jest ciągiem stałym\\
E. Ciąg rekurencyjny nie jest monotoniczny.
\testStop
\kluczStart
A
\kluczStop



\end{document}