\documentclass[12pt, a4paper]{article}
\usepackage[utf8]{inputenc}
\usepackage{polski}
\usepackage{amsthm}  %pakiet do tworzenia twierdzeń itp.
\usepackage{amsmath} %pakiet do niektórych symboli matematycznych
\usepackage{amssymb} %pakiet do symboli mat., np. \nsubseteq
\usepackage{amsfonts}
\usepackage{graphicx} %obsługa plików graficznych z rozszerzeniem png, jpg
\theoremstyle{definition} %styl dla definicji
\newtheorem{zad}{} 
\title{Multizestaw zadań}
\author{Patryk Wirkus}
%\date{\today}
\date{}
\newcommand{\kategoria}[1]{\section{#1}}
\newcommand{\zadStart}[1]{\begin{zad}#1\newline}
\newcommand{\zadStop}{\end{zad}}
\newcommand{\rozwStart}[2]{\noindent \textbf{Rozwiązanie (autor #1 , recenzent #2): }\newline}
\newcommand{\rozwStop}{\newline}                                           
\newcommand{\odpStart}{\noindent \textbf{Odpowiedź:}\newline}
\newcommand{\odpStop}{\newline}
\newcommand{\testStart}{\noindent \textbf{Test:}\newline}
\newcommand{\testStop}{\newline}
\newcommand{\kluczStart}{\noindent \textbf{Test poprawna odpowiedź:}\newline}
\newcommand{\kluczStop}{\newline}
\newcommand{\wstawGrafike}[2]{\begin{figure}[h] \includegraphics[scale=#2] {#1} \end{figure}}

\begin{document}
\maketitle

\kategoria{Wikieł/5.22a}


\zadStart{Zadanie z Wikieł Z 5.22 a) moja wersja nr 1}

Wykazać, że dana funkcja jest rosnąca $f(x) = x^{31}+x$.
\zadStop
\rozwStart{Patryk Wirkus}{Szymon Tokarski}
Wzór naszej funkcji wygląda następująco:
$$f(x) = x^{31}+x.$$
Aby sprawdzić monotoniczność funkcji, policzmy pochodną funkcji f(x),\\ $f^{'}(x) = 31x^{31-1}+1 > 0$.\\
Zgodnie z twierdzeniem o monotoniczności funkcja ta jest rosnąca w\\ przedziale $(-\infty,\infty)$.
\rozwStop
\odpStart
Funkcja f(x) jest rosnąca.
\odpStop
\testStart
A. Funkcja f(x) jest rosnąca.
\testStop
\kluczStart
A
\kluczStop



\zadStart{Zadanie z Wikieł Z 5.22 a) moja wersja nr 2}

Wykazać, że dana funkcja jest rosnąca $f(x) = x^{11}+x$.
\zadStop
\rozwStart{Patryk Wirkus}{Szymon Tokarsk}
Wzór naszej funkcji wygląda następująco:
$$f(x) = x^{11}+x.$$
Aby sprawdzić monotoniczność funkcji, policzmy pochodną funkcji f(x),\\ $f^{'}(x) = 11x^{11-1}+1 > 0$.\\
Zgodnie z twierdzeniem o monotoniczności funkcja ta jest rosnąca w\\ przedziale $(-\infty,\infty)$.
\rozwStop
\odpStart
Funkcja f(x) jest rosnąca.
\odpStop
\testStart
A. Funkcja f(x) jest rosnąca.
\testStop
\kluczStart
A
\kluczStop



\zadStart{Zadanie z Wikieł Z 5.22 a) moja wersja nr 3}

Wykazać, że dana funkcja jest rosnąca $f(x) = x^{3}+x$.
\zadStop
\rozwStart{Patryk Wirkus}{Szymon Tokarsk}
Wzór naszej funkcji wygląda następująco:
$$f(x) = x^{3}+x.$$
Aby sprawdzić monotoniczność funkcji, policzmy pochodną funkcji f(x),\\ $f^{'}(x) = 3x^{3-1}+1 > 0$.\\
Zgodnie z twierdzeniem o monotoniczności funkcja ta jest rosnąca w\\ przedziale $(-\infty,\infty)$.
\rozwStop
\odpStart
Funkcja f(x) jest rosnąca.
\odpStop
\testStart
A. Funkcja f(x) jest rosnąca.
\testStop
\kluczStart
A
\kluczStop



\zadStart{Zadanie z Wikieł Z 5.22 a) moja wersja nr 4}

Wykazać, że dana funkcja jest rosnąca $f(x) = x^{15}+x$.
\zadStop
\rozwStart{Patryk Wirkus}{Szymon Tokarsk}
Wzór naszej funkcji wygląda następująco:
$$f(x) = x^{15}+x.$$
Aby sprawdzić monotoniczność funkcji, policzmy pochodną funkcji f(x),\\ $f^{'}(x) = 15x^{15-1}+1 > 0$.\\
Zgodnie z twierdzeniem o monotoniczności funkcja ta jest rosnąca w\\ przedziale $(-\infty,\infty)$.
\rozwStop
\odpStart
Funkcja f(x) jest rosnąca.
\odpStop
\testStart
A. Funkcja f(x) jest rosnąca.
\testStop
\kluczStart
A
\kluczStop



\zadStart{Zadanie z Wikieł Z 5.22 a) moja wersja nr 5}

Wykazać, że dana funkcja jest rosnąca $f(x) = x^{5}+x$.
\zadStop
\rozwStart{Patryk Wirkus}{Szymon Tokarsk}
Wzór naszej funkcji wygląda następująco:
$$f(x) = x^{5}+x.$$
Aby sprawdzić monotoniczność funkcji, policzmy pochodną funkcji f(x),\\ $f^{'}(x) = 5x^{5-1}+1 > 0$.\\
Zgodnie z twierdzeniem o monotoniczności funkcja ta jest rosnąca w\\ przedziale $(-\infty,\infty)$.
\rozwStop
\odpStart
Funkcja f(x) jest rosnąca.
\odpStop
\testStart
A. Funkcja f(x) jest rosnąca.
\testStop
\kluczStart
A
\kluczStop



\zadStart{Zadanie z Wikieł Z 5.22 a) moja wersja nr 6}

Wykazać, że dana funkcja jest rosnąca $f(x) = x^{17}+x$.
\zadStop
\rozwStart{Patryk Wirkus}{Szymon Tokarsk}
Wzór naszej funkcji wygląda następująco:
$$f(x) = x^{17}+x.$$
Aby sprawdzić monotoniczność funkcji, policzmy pochodną funkcji f(x),\\ $f^{'}(x) = 17x^{17-1}+1 > 0$.\\
Zgodnie z twierdzeniem o monotoniczności funkcja ta jest rosnąca w\\ przedziale $(-\infty,\infty)$.
\rozwStop
\odpStart
Funkcja f(x) jest rosnąca.
\odpStop
\testStart
A. Funkcja f(x) jest rosnąca.
\testStop
\kluczStart
A
\kluczStop



\zadStart{Zadanie z Wikieł Z 5.22 a) moja wersja nr 7}

Wykazać, że dana funkcja jest rosnąca $f(x) = x^{7}+x$.
\zadStop
\rozwStart{Patryk Wirkus}{Szymon Tokarsk}
Wzór naszej funkcji wygląda następująco:
$$f(x) = x^{7}+x.$$
Aby sprawdzić monotoniczność funkcji, policzmy pochodną funkcji f(x),\\ $f^{'}(x) = 7x^{7-1}+1 > 0$.\\
Zgodnie z twierdzeniem o monotoniczności funkcja ta jest rosnąca w\\ przedziale $(-\infty,\infty)$.
\rozwStop
\odpStart
Funkcja f(x) jest rosnąca.
\odpStop
\testStart
A. Funkcja f(x) jest rosnąca.
\testStop
\kluczStart
A
\kluczStop



\zadStart{Zadanie z Wikieł Z 5.22 a) moja wersja nr 8}

Wykazać, że dana funkcja jest rosnąca $f(x) = x^{19}+x$.
\zadStop
\rozwStart{Patryk Wirkus}{Szymon Tokarsk}
Wzór naszej funkcji wygląda następująco:
$$f(x) = x^{19}+x.$$
Aby sprawdzić monotoniczność funkcji, policzmy pochodną funkcji f(x),\\ $f^{'}(x) = 19x^{19-1}+1 > 0$.\\
Zgodnie z twierdzeniem o monotoniczności funkcja ta jest rosnąca w\\ przedziale $(-\infty,\infty)$.
\rozwStop
\odpStart
Funkcja f(x) jest rosnąca.
\odpStop
\testStart
A. Funkcja f(x) jest rosnąca.
\testStop
\kluczStart
A
\kluczStop



\zadStart{Zadanie z Wikieł Z 5.22 a) moja wersja nr 9}

Wykazać, że dana funkcja jest rosnąca $f(x) = x^{9}+x$.
\zadStop
\rozwStart{Patryk Wirkus}{Szymon Tokarsk}
Wzór naszej funkcji wygląda następująco:
$$f(x) = x^{9}+x.$$
Aby sprawdzić monotoniczność funkcji, policzmy pochodną funkcji f(x),\\ $f^{'}(x) = 9x^{9-1}+1 > 0$.\\
Zgodnie z twierdzeniem o monotoniczności funkcja ta jest rosnąca w\\ przedziale $(-\infty,\infty)$.
\rozwStop
\odpStart
Funkcja f(x) jest rosnąca.
\odpStop
\testStart
A. Funkcja f(x) jest rosnąca.
\testStop
\kluczStart
A
\kluczStop



\zadStart{Zadanie z Wikieł Z 5.22 a) moja wersja nr 10}

Wykazać, że dana funkcja jest rosnąca $f(x) = x^{21}+x$.
\zadStop
\rozwStart{Patryk Wirkus}{Szymon Tokarsk}
Wzór naszej funkcji wygląda następująco:
$$f(x) = x^{21}+x.$$
Aby sprawdzić monotoniczność funkcji, policzmy pochodną funkcji f(x),\\ $f^{'}(x) = 21x^{21-1}+1 > 0$.\\
Zgodnie z twierdzeniem o monotoniczności funkcja ta jest rosnąca w\\ przedziale $(-\infty,\infty)$.
\rozwStop
\odpStart
Funkcja f(x) jest rosnąca.
\odpStop
\testStart
A. Funkcja f(x) jest rosnąca.
\testStop
\kluczStart
A
\kluczStop





\end{document}
