\documentclass[12pt, a4paper]{article}
\usepackage[utf8]{inputenc}
\usepackage{polski}
\usepackage{amsthm}  %pakiet do tworzenia twierdzeń itp.
\usepackage{amsmath} %pakiet do niektórych symboli matematycznych
\usepackage{amssymb} %pakiet do symboli mat., np. \nsubseteq
\usepackage{amsfonts}
\usepackage{graphicx} %obsługa plików graficznych z rozszerzeniem png, jpg
\theoremstyle{definition} %styl dla definicji
\newtheorem{zad}{} 
\title{Multizestaw zadań}
\author{Radosław Grzyb}
%\date{\today}
\date{}
\newcounter{liczniksekcji}
\newcommand{\kategoria}[1]{\section{#1}} %olreślamy nazwę kateforii zadań
\newcommand{\zadStart}[1]{\begin{zad}#1\newline} %oznaczenie początku zadania
\newcommand{\zadStop}{\end{zad}}   %oznaczenie końca zadania
%Makra opcjonarne (nie muszą występować):
\newcommand{\rozwStart}[2]{\noindent \textbf{Rozwiązanie (autor #1 , recenzent #2): }\newline} %oznaczenie początku rozwiązania, opcjonarnie można wprowadzić informację o autorze rozwiązania zadania i recenzencie poprawności wykonania rozwiązania zadania
\newcommand{\rozwStop}{\newline}                                            %oznaczenie końca rozwiązania
\newcommand{\odpStart}{\noindent \textbf{Odpowiedź:}\newline}    %oznaczenie początku odpowiedzi końcowej (wypisanie wyniku)
\newcommand{\odpStop}{\newline}                                             %oznaczenie końca odpowiedzi końcowej (wypisanie wyniku)
\newcommand{\testStart}{\noindent \textbf{Test:}\newline} %ewentualne możliwe opcje odpowiedzi testowej: A. ? B. ? C. ? D. ? itd.
\newcommand{\testStop}{\newline} %koniec wprowadzania odpowiedzi testowych
\newcommand{\kluczStart}{\noindent \textbf{Test poprawna odpowiedź:}\newline} %klucz, poprawna odpowiedź pytania testowego (jedna literka): A lub B lub C lub D itd.
\newcommand{\kluczStop}{\newline} %koniec poprawnej odpowiedzi pytania testowego 
\newcommand{\wstawGrafike}[2]{\begin{figure}[h] \includegraphics[scale=#2] {#1} \end{figure}} %gdyby była potrzeba wstawienia obrazka, parametry: nazwa pliku, skala (jak nie wiesz co wpisać, to wpisz 1)
\begin{document}
\maketitle
\kategoria{Beger/c1.14e}
\zadStart{Zadanie z Beger C 1.14e moja wersja nr [nrWersji]}
%[p1]:[3,5,7,9,11]
%[p2]:[1,2,3,4,5,6,7,8,9,10]
%[Delta]=[p1]**2-4*[p2]
%[DeltaPlus]=-[Delta]
%[Delta]<0
Obliczyć całkę niewymierną.
$$\int \frac{1}{\sqrt{x^{2}+[p1]x+[p2]}} \,dx$$
\zadStop
\rozwStart{Radosław Grzyb}{}
Do obliczenia całki wykorzystamy pomocniczy wzór:
$$ax^{2}+bx+c = a[(x+\frac{b}{2a})^{2}-\frac{\Delta}{4a^{2}}]$$
Obliczmy najpierw deltę:
$$\Delta=[p1]^2-4\cdot[p2]=[Delta]$$
Podstawiając do wzoru otrzymujemy:
$$(x+\frac{[p1]}{2})^{2}-\frac{[Delta]}{4}$$
A więc:
$$\int \frac{1}{\sqrt{(x+\frac{[p1]}{2})^{2}-\frac{[Delta]}{4}}} \,dx=\int \frac{1}{\sqrt{(x+\frac{[p1]}{2})^{2}+\frac{[DeltaPlus]}{4}}} \,dx$$
Podstawiamy $t=x+\frac{[p1]}{2}$, a więc $dt=dx$:
$$\int\frac{1}{\sqrt{t^{2}+\frac{[DeltaPlus]}{4}}} \,dt$$
Korzystając z gotowego wzoru na całkę  $\int \frac{1}{\sqrt{x^{2}+q}} \,dx = \ln(|x+\sqrt{x^2+q}|)+C$ otrzymujemy:
$$\int\frac{1}{\sqrt{t^{2}+\frac{[DeltaPlus]}{4}}} \,dt=\ln(|t+\sqrt{t^2+\frac{[DeltaPlus]}{4}}|)+C=\ln(|x+\frac{[p1]}{2}+\sqrt{(x+\frac{[p1]}{2})^{2}+\frac{[DeltaPlus]}{4}}|)+C$$
\rozwStop
\odpStart
$$\ln(|x+\frac{[p1]}{2}+\sqrt{(x+\frac{[p1]}{2})^{2}+\frac{[DeltaPlus]}{4}}|)+C$$
\odpStop
\testStart
A.$$\arcsin(\frac{2(x-\frac{[p1]}{2})}{[DeltaPlus]})+C$$
B.$$\ln(|x+\frac{[p1]}{2}+\sqrt{(x+\frac{[p1]}{2})^{2}+\frac{[DeltaPlus]}{4}}|)+C$$
C.$$\ln(|\frac{2(x-\frac{[p1]}{2})}{[DeltaPlus]}|)+C$$
D.$$\tg(\frac{2(x-\frac{[p1]}{2})}{[DeltaPlus]})+C$$
\testStop
\kluczStart
B
\kluczStop
\end{document}