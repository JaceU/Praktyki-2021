\documentclass[12pt, a4paper]{article}
\usepackage[utf8]{inputenc}
\usepackage{polski}

\usepackage{amsthm}  %pakiet do tworzenia twierdzeń itp.
\usepackage{amsmath} %pakiet do niektórych symboli matematycznych
\usepackage{amssymb} %pakiet do symboli mat., np. \nsubseteq
\usepackage{amsfonts}
\usepackage{graphicx} %obsługa plików graficznych z rozszerzeniem png, jpg
\theoremstyle{definition} %styl dla definicji
\newtheorem{zad}{} 
\title{Multizestaw zadań}
\author{Robert Fidytek}
%\date{\today}
\date{}
\newcounter{liczniksekcji}
\newcommand{\kategoria}[1]{\section{#1}} %olreślamy nazwę kateforii zadań
\newcommand{\zadStart}[1]{\begin{zad}#1\newline} %oznaczenie początku zadania
\newcommand{\zadStop}{\end{zad}}   %oznaczenie końca zadania
%Makra opcjonarne (nie muszą występować):
\newcommand{\rozwStart}[2]{\noindent \textbf{Rozwiązanie (autor #1 , recenzent #2): }\newline} %oznaczenie początku rozwiązania, opcjonarnie można wprowadzić informację o autorze rozwiązania zadania i recenzencie poprawności wykonania rozwiązania zadania
\newcommand{\rozwStop}{\newline}                                            %oznaczenie końca rozwiązania
\newcommand{\odpStart}{\noindent \textbf{Odpowiedź:}\newline}    %oznaczenie początku odpowiedzi końcowej (wypisanie wyniku)
\newcommand{\odpStop}{\newline}                                             %oznaczenie końca odpowiedzi końcowej (wypisanie wyniku)
\newcommand{\testStart}{\noindent \textbf{Test:}\newline} %ewentualne możliwe opcje odpowiedzi testowej: A. ? B. ? C. ? D. ? itd.
\newcommand{\testStop}{\newline} %koniec wprowadzania odpowiedzi testowych
\newcommand{\kluczStart}{\noindent \textbf{Test poprawna odpowiedź:}\newline} %klucz, poprawna odpowiedź pytania testowego (jedna literka): A lub B lub C lub D itd.
\newcommand{\kluczStop}{\newline} %koniec poprawnej odpowiedzi pytania testowego 
\newcommand{\wstawGrafike}[2]{\begin{figure}[h] \includegraphics[scale=#2] {#1} \end{figure}} %gdyby była potrzeba wstawienia obrazka, parametry: nazwa pliku, skala (jak nie wiesz co wpisać, to wpisz 1)

\begin{document}
\maketitle


\kategoria{Wikieł/Z5.21 a}
\zadStart{Zadanie z Wikieł Z 5.21 a) moja wersja nr [nrWersji]}
%[a]:[4,6,8,10,12,14,18,20,22]
%[b]=int([a]/2)
%[a]!=0
Wyznaczyć przedziały monotoniczności funkcji $f(x)=\frac{[a]}{x}-x^2$.
\zadStop
\rozwStart{Joanna Świerzbin}{}
$$f(x)=\frac{[a]}{x}-x^2$$
\\
$$D_f:x \neq 0 $$
\\
$$f'(x)=\left(\frac{[a]}{x}-x^2\right)' = \frac{-[a]}{x^2} -2x = \frac{-[a]-2x^3}{x^2}$$
\begin{enumerate}
\item 
$$\frac{-[a]-2x^3}{x^2}<0$$
$$-[a]-2x^3<0$$
$$x^3>-\frac{[a]}{2}$$
$$x>-\sqrt[3]{[b]} \land x\neq 0$$ 
$$ x \in ( -\sqrt[3]{[b]},0) \cup (0, \infty) $$
\item 
$$\frac{-[a]-2x^3}{x^2}>0$$
$$-[a]-2x^3>0$$
$$x^3<-\frac{[a]}{2}$$
$$x<-\sqrt[3]{[b]} \land x\neq 0$$ 
$$ x \in (-\infty , -\sqrt[3]{[b]})  $$
\end{enumerate}
Funkcja $f$ jest: \\ rosnąca dla $x \in \left(-\infty , -\sqrt[3]{[b]}\right)$ \\ malejąca dla $x \in \left( -\sqrt[3]{[b]},0\right) \cup \left(0, \infty \right)$.
\rozwStop
\odpStart
$f$ jest: \\ rosnąca dla $x \in \left(-\infty , -\sqrt[3]{[b]}\right)$ \\ malejąca dla $x \in \left( -\sqrt[3]{[b]},0\right) \cup \left(0, \infty \right)$.
\odpStop
\testStart
A. $f$ jest: \\ rosnąca dla $x \in \left(-\infty , -\sqrt[3]{[b]}\right)$ \\ malejąca dla $x \in \left( -\sqrt[3]{[b]},0\right) \cup \left(0, \infty \right)$.\\
B. $f$ jest: \\ rosnąca dla $x \in \left(-\infty , -\sqrt[3]{[b]}\right)$ \\ malejąca dla $x \in \left( -\sqrt[3]{[b]},\infty \right)$.\\
C. $f$ jest: \\ rosnąca dla $x \in \left(-\infty , \sqrt[3]{[b]}\right)$ \\ malejąca dla $x \in \left( \sqrt[3]{[b]},\infty \right)$.\\
D. $f$ jest: \\ rosnąca dla $x \in \left(-\infty , -[b] \right)$ \\ malejąca dla $x \in \left( -[b],0\right) \cup \left(0, \infty \right)$.\\
E. $f$ jest: \\ rosnąca dla $x \in \left(-[b], -\sqrt[3]{[b]}\right)$ \\ malejąca dla $x \in \left( -\sqrt[3]{[b]},0\right) \cup \left(0, \infty \right)$.\\
F. $f$ jest: \\ rosnąca dla $x \in \left(-\infty , -\sqrt[3]{[b]}\right)$ \\ malejąca dla $x \in \left( -\sqrt[3]{[b]},0\right) \cup \left(0, [b] \right)$.
\testStop
\kluczStart
A
\kluczStop



\end{document}