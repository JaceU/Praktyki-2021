\documentclass[12pt, a4paper]{article}
\usepackage[utf8]{inputenc}
\usepackage{polski}

\usepackage{amsthm}  %pakiet do tworzenia twierdzeń itp.
\usepackage{amsmath} %pakiet do niektórych symboli matematycznych
\usepackage{amssymb} %pakiet do symboli mat., np. \nsubseteq
\usepackage{amsfonts}
\usepackage{graphicx} %obsługa plików graficznych z rozszerzeniem png, jpg
\theoremstyle{definition} %styl dla definicji
\newtheorem{zad}{} 
\title{Multizestaw zadań}
\author{Laura Mieczkowska}
%\date{\today}
\date{}
\newcounter{liczniksekcji}
\newcommand{\kategoria}[1]{\section{#1}} %olreślamy nazwę kateforii zadań
\newcommand{\zadStart}[1]{\begin{zad}#1\newline} %oznaczenie początku zadania
\newcommand{\zadStop}{\end{zad}}   %oznaczenie końca zadania
%Makra opcjonarne (nie muszą występować):
\newcommand{\rozwStart}[2]{\noindent \textbf{Rozwiązanie (autor #1 , recenzent #2): }\newline} %oznaczenie początku rozwiązania, opcjonarnie można wprowadzić informację o autorze rozwiązania zadania i recenzencie poprawności wykonania rozwiązania zadania
\newcommand{\rozwStop}{\newline}                                            %oznaczenie końca rozwiązania
\newcommand{\odpStart}{\noindent \textbf{Odpowiedź:}\newline}    %oznaczenie początku odpowiedzi końcowej (wypisanie wyniku)
\newcommand{\odpStop}{\newline}                                             %oznaczenie końca odpowiedzi końcowej (wypisanie wyniku)
\newcommand{\testStart}{\noindent \textbf{Test:}\newline} %ewentualne możliwe opcje odpowiedzi testowej: A. ? B. ? C. ? D. ? itd.
\newcommand{\testStop}{\newline} %koniec wprowadzania odpowiedzi testowych
\newcommand{\kluczStart}{\noindent \textbf{Test poprawna odpowiedź:}\newline} %klucz, poprawna odpowiedź pytania testowego (jedna literka): A lub B lub C lub D itd.
\newcommand{\kluczStop}{\newline} %koniec poprawnej odpowiedzi pytania testowego 
\newcommand{\wstawGrafike}[2]{\begin{figure}[h] \includegraphics[scale=#2] {#1} \end{figure}} %gdyby była potrzeba wstawienia obrazka, parametry: nazwa pliku, skala (jak nie wiesz co wpisać, to wpisz 1)

\begin{document}
\maketitle


\kategoria{Wikieł/Z1.36s}
\zadStart{Zadanie z Wikieł Z 1.36 s) moja wersja nr [nrWersji]}
%[a]:[2,3,4,5,6,7,8,9,10,11,12,13,14,15,16,17,18,19,20]
%[b]:[2,3,4,5,6,7,8,9,10,11,12,13,14,15,16,17,18,19,20]
%[c]:[3,4,5,6,7,8,9,10,11,12,13,14,15,16,17,18,19,20]
%[delta]=[b]**2-4*[a]*[c]
%[p]=(pow([delta],1/2))
%[p1]=[p].real
%[p2]=int([p1])
%[m]=2*[a]
%[e]=[b]+[p2]
%[f]=[b]-[p2]
%[uu1]=[f]/[m]
%[uu2]=[e]/[m]
%[u1]=int([uu1])
%[u2]=int([uu2])
%[up1]=math.sqrt([u1])
%[up]=int([up1])
%[up2]=math.sqrt([u2])
%[upp]=int([up2])
%[uu1].is_integer()==True and [uu2].is_integer()==True and [up1].is_integer()==True and [up2].is_integer()==True and [b]**2>4*[a]*[c] and [p].is_integer()==True
Rozwiązać nierówność $[a]x^4-[b]x^2+[c]<0$.
\zadStop
\rozwStart{Laura Mieczkowska}{}
$$[a]x^4-[b]x^2+[c]<0$$
$$t=x^2 \Rightarrow [a]t^2-[b]t+[c]<0$$ 
$$\triangle=[b]^2-4\cdot[a]\cdot [c]=[delta] \Rightarrow \sqrt{\triangle}=[p2]$$
$$t=\frac{[f]}{[m]} \vee t=\frac{[e]}{[m]}$$
$$x^2=[u1] \vee x^2=[u2]$$
$$x=[up] \vee x=-[up] \vee x=[upp] \vee x=-[upp]$$
$$x\in(-[upp];-[up])\cup([up];[upp])$$
\odpStart
$x\in(-[upp];-[up])\cup([up];[upp])$
\odpStop
\testStart
A. $x\in(-[b];-[up])\cup([up];[upp])$ \\
B. $x\in(-[upp];[up])$ \\
C. $x\in\emptyset$ \\
D. $x\in(-[upp];-[up])\cup([up];[upp])$ 
\testStop
\kluczStart
D
\kluczStop



\end{document}