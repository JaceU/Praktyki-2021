\documentclass[12pt, a4paper]{article}
\usepackage[utf8]{inputenc}
\usepackage{polski}

\usepackage{amsthm}  %pakiet do tworzenia twierdzeń itp.
\usepackage{amsmath} %pakiet do niektórych symboli matematycznych
\usepackage{amssymb} %pakiet do symboli mat., np. \nsubseteq
\usepackage{amsfonts}
\usepackage{graphicx} %obsługa plików graficznych z rozszerzeniem png, jpg
\theoremstyle{definition} %styl dla definicji
\newtheorem{zad}{} 
\title{Multizestaw zadań}
\author{Robert Fidytek}
%\date{\today}
\date{}
\newcounter{liczniksekcji}
\newcommand{\kategoria}[1]{\section{#1}} %olreślamy nazwę kateforii zadań
\newcommand{\zadStart}[1]{\begin{zad}#1\newline} %oznaczenie początku zadania
\newcommand{\zadStop}{\end{zad}}   %oznaczenie końca zadania
%Makra opcjonarne (nie muszą występować):
\newcommand{\rozwStart}[2]{\noindent \textbf{Rozwiązanie (autor #1 , recenzent #2): }\newline} %oznaczenie początku rozwiązania, opcjonarnie można wprowadzić informację o autorze rozwiązania zadania i recenzencie poprawności wykonania rozwiązania zadania
\newcommand{\rozwStop}{\newline}                                            %oznaczenie końca rozwiązania
\newcommand{\odpStart}{\noindent \textbf{Odpowiedź:}\newline}    %oznaczenie początku odpowiedzi końcowej (wypisanie wyniku)
\newcommand{\odpStop}{\newline}                                             %oznaczenie końca odpowiedzi końcowej (wypisanie wyniku)
\newcommand{\testStart}{\noindent \textbf{Test:}\newline} %ewentualne możliwe opcje odpowiedzi testowej: A. ? B. ? C. ? D. ? itd.
\newcommand{\testStop}{\newline} %koniec wprowadzania odpowiedzi testowych
\newcommand{\kluczStart}{\noindent \textbf{Test poprawna odpowiedź:}\newline} %klucz, poprawna odpowiedź pytania testowego (jedna literka): A lub B lub C lub D itd.
\newcommand{\kluczStop}{\newline} %koniec poprawnej odpowiedzi pytania testowego 
\newcommand{\wstawGrafike}[2]{\begin{figure}[h] \includegraphics[scale=#2] {#1} \end{figure}} %gdyby była potrzeba wstawienia obrazka, parametry: nazwa pliku, skala (jak nie wiesz co wpisać, to wpisz 1)

\begin{document}
\maketitle


\kategoria{Wikieł/Z1.9a}
\zadStart{Zadanie z Wikieł Z 1.9 a) moja wersja nr [nrWersji]}
%[a]:[2,3,4]
%[b]:[2,3,4]
%[c]:[2,3,4]
%[bk]=[b]*[b]
%[bk2]=[bk]-2
%[ak]=[a]*[a]
%[ab2]=2*[a]*[b]
%[4a]=4*[a]
%[6ak]=6*[ak]
%[4a3]=4*[a]*[a]*[a]
%[xkw]=[6ak]*2+[ab2]
%[ck]=[c]*[c]
%[4c]=4*[c]
%[6ck]=6*[ck]
%[4c3]=4*[c]*[c]*[c]
%[aczw]=[a]*[a]*[a]*[a]
%[cczw]=[c]*[c]*[c]*[c]
%[doczwartej]=2-[bk]
%[dotrzeciej]=[4a]-[4c]
%[dodrugiej]=[6ak]+[ab2]+[6ck]
%[dopierwszej]=[4a3]-[4c3]
%[wolny]=[aczw]-[ak]+[cczw]
%[a]>[c]
Uprościć wyrażenie: $(x+[a])^4-([b]x^2-[a])^2+(x-[c])^4$
\zadStop
\rozwStart{Pascal Nawrocki}{Jakub Ulrych}
Stosujemy wzory skróconego mnożenia (polecam naukę korzystania z trójkąta Pascala, aby nie musieć uczyć się wzorów na pamięć) $(a+b)^4=a^4+4a^3b+6a^2b^2+4ab^3+b^4$, $(a-b)^4=a^4-4a^3b+6a^2b^2-4ab^3+b^4$ oraz $(a-b)^2=a^2-2ab+b^2$. Zatem: \newline
$$(x+[a])^4-([b]x^2-[a])^2+(x-[c])^4 =$$
$$=x^4+4x^3\cdot[a]+6x^2\cdot[a]^2+4x[a]^3+[a]^4-([b]^2\cdot x^4-2\cdot[b]\cdot x^2\cdot[a]+[a]^2)+x^4-4x^3\cdot[c]+6x^2\cdot[c]^2-4x[c]^3+[c]^4=$$
$$=x^4+[4a]x^3+[6ak]x^2+[4a3]x+[aczw]-[bk]x^4+[ab2]x^2-[ak]+x^4-[4c]x^3+[6ck]x^2-[4c3]x+[cczw]=$$
Teraz musimy tylko uporządkować. To znaczy, że $x^4$ z $x^4$, $x^3$ z $x^3$ etc. dobrym pomysłem kiedy jest tego dużo, jest podkreślanie/zakreślanie kolorem liczb które będziemy łączyć.
$$=[doczwartej]x^4+[dotrzeciej]x^3+[dodrugiej]x^2+[dopierwszej]x+[wolny]$$
\odpStop
$[doczwartej]x^4+[dotrzeciej]x^3+[dodrugiej]x^2+[dopierwszej]x+[wolny]$
\testStart
A.$[doczwartej]x^4+[dotrzeciej]x^3+[dodrugiej]x^2+[dopierwszej]x+[wolny]$
\\
B.$-[bk2]x^4+[wolny]$
\\
C.$[wolny]$
\\
D.$-[xkw]x^2-[wolny]$
\testStop
\kluczStart
A
\kluczStop
\end{document}