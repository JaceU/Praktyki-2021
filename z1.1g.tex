\documentclass[12pt, a4paper]{article}
\usepackage[utf8]{inputenc}
\usepackage{polski}

\usepackage{amsthm}  %pakiet do tworzenia twierdzeń itp.
\usepackage{amsmath} %pakiet do niektórych symboli matematycznych
\usepackage{amssymb} %pakiet do symboli mat., np. \nsubseteq
\usepackage{amsfonts}
\usepackage{graphicx} %obsługa plików graficznych z rozszerzeniem png, jpg
\theoremstyle{definition} %styl dla definicji
\newtheorem{zad}{} 
\title{Multizestaw zadań}
\author{Laura Mieczkowska}
%\date{\today}
\date{}
\newcounter{liczniksekcji}
\newcommand{\kategoria}[1]{\section{#1}} %olreślamy nazwę kateforii zadań
\newcommand{\zadStart}[1]{\begin{zad}#1\newline} %oznaczenie początku zadania
\newcommand{\zadStop}{\end{zad}}   %oznaczenie końca zadania
%Makra opcjonarne (nie muszą występować):
\newcommand{\rozwStart}[2]{\noindent \textbf{Rozwiązanie (autor #1 , recenzent #2): }\newline} %oznaczenie początku rozwiązania, opcjonarnie można wprowadzić informację o autorze rozwiązania zadania i recenzencie poprawności wykonania rozwiązania zadania
\newcommand{\rozwStop}{\newline}                                            %oznaczenie końca rozwiązania
\newcommand{\odpStart}{\noindent \textbf{Odpowiedź:}\newline}    %oznaczenie początku odpowiedzi końcowej (wypisanie wyniku)
\newcommand{\odpStop}{\newline}                                             %oznaczenie końca odpowiedzi końcowej (wypisanie wyniku)
\newcommand{\testStart}{\noindent \textbf{Test:}\newline} %ewentualne możliwe opcje odpowiedzi testowej: A. ? B. ? C. ? D. ? itd.
\newcommand{\testStop}{\newline} %koniec wprowadzania odpowiedzi testowych
\newcommand{\kluczStart}{\noindent \textbf{Test poprawna odpowiedź:}\newline} %klucz, poprawna odpowiedź pytania testowego (jedna literka): A lub B lub C lub D itd.
\newcommand{\kluczStop}{\newline} %koniec poprawnej odpowiedzi pytania testowego 
\newcommand{\wstawGrafike}[2]{\begin{figure}[h] \includegraphics[scale=#2] {#1} \end{figure}} %gdyby była potrzeba wstawienia obrazka, parametry: nazwa pliku, skala (jak nie wiesz co wpisać, to wpisz 1)

\begin{document}
\maketitle


\kategoria{Wikieł/Z1.1g}
\zadStart{Zadanie z Wikieł Z 1.1 g) moja wersja nr [nrWersji]}
%[a]:[2,3,4,5,6,7,8,9,10,11,12,13,14,15,16,17,18,19,20]
%[b]:[2,3,4,5,6,7,8,9,10,11,12,13,14,15,16,17,18,19,20]
%[c]:[2,3,4,5,6,7,8,9,10,11,12,13,14,15,16,17,18,19,20]
%[d]:[2,3,4,5,6,7,8,9,10,11,12,13,14,15,16,17,18,19,20]
%[e]=[a]*[d]
%[f]=[b]*[d]
%[g]=[c]*[d]
%[ep1]=math.sqrt([e])
%[ep]=int([ep1])
%[fp1]=math.sqrt([f])
%[fp]=int([fp1])
%[gp1]=math.sqrt([g])
%[gp]=int([gp1])
%[sum]=[ep]-[fp]+[gp]
%[ep1].is_integer()==True and [fp1].is_integer()==True and [gp1].is_integer()==True and [d]!=[a] and [d]!=[b]
Obliczyć wartość wyrażenia $(\sqrt{[a]}-\sqrt{[b]}+\sqrt{[c]})\cdot \sqrt{[d]}$.
\zadStop
\rozwStart{Laura Mieczkowska}{}
$$(\sqrt{[a]}-\sqrt{[b]}+\sqrt{[c]})\cdot \sqrt{[d]}=\sqrt{[a]\cdot[d]}-\sqrt{[b]\cdot[d]}+\sqrt{[c]\cdot[d]}=\sqrt{[e]}-\sqrt{[f]}+\sqrt{[g]}=[ep]-[fp]+[gp]=[sum]$$

\odpStart
$[sum]$
\odpStop
\testStart
A. $[sum]$ \\
B. $[a]$ \\
C. $1$ \\
D. $-[b]$ 
\testStop
\kluczStart
A
\kluczStop



\end{document}