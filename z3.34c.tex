\documentclass[12pt, a4paper]{article}
\usepackage[utf8]{inputenc}
\usepackage{polski}

\usepackage{amsthm}  %pakiet do tworzenia twierdzeń itp.
\usepackage{amsmath} %pakiet do niektórych symboli matematycznych
\usepackage{amssymb} %pakiet do symboli mat., np. \nsubseteq
\usepackage{amsfonts}
\usepackage{graphicx} %obsługa plików graficznych z rozszerzeniem png, jpg
\theoremstyle{definition} %styl dla definicji
\newtheorem{zad}{} 
\title{Multizestaw zadań}
\author{Robert Fidytek}
%\date{\today}
\date{}
\newcounter{liczniksekcji}
\newcommand{\kategoria}[1]{\section{#1}} %olreślamy nazwę kateforii zadań
\newcommand{\zadStart}[1]{\begin{zad}#1\newline} %oznaczenie początku zadania
\newcommand{\zadStop}{\end{zad}}   %oznaczenie końca zadania
%Makra opcjonarne (nie muszą występować):
\newcommand{\rozwStart}[2]{\noindent \textbf{Rozwiązanie (autor #1 , recenzent #2): }\newline} %oznaczenie początku rozwiązania, opcjonarnie można wprowadzić informację o autorze rozwiązania zadania i recenzencie poprawności wykonania rozwiązania zadania
\newcommand{\rozwStop}{\newline}                                            %oznaczenie końca rozwiązania
\newcommand{\odpStart}{\noindent \textbf{Odpowiedź:}\newline}    %oznaczenie początku odpowiedzi końcowej (wypisanie wyniku)
\newcommand{\odpStop}{\newline}                                             %oznaczenie końca odpowiedzi końcowej (wypisanie wyniku)
\newcommand{\testStart}{\noindent \textbf{Test:}\newline} %ewentualne możliwe opcje odpowiedzi testowej: A. ? B. ? C. ? D. ? itd.
\newcommand{\testStop}{\newline} %koniec wprowadzania odpowiedzi testowych
\newcommand{\kluczStart}{\noindent \textbf{Test poprawna odpowiedź:}\newline} %klucz, poprawna odpowiedź pytania testowego (jedna literka): A lub B lub C lub D itd.
\newcommand{\kluczStop}{\newline} %koniec poprawnej odpowiedzi pytania testowego 
\newcommand{\wstawGrafike}[2]{\begin{figure}[h] \includegraphics[scale=#2] {#1} \end{figure}} %gdyby była potrzeba wstawienia obrazka, parametry: nazwa pliku, skala (jak nie wiesz co wpisać, to wpisz 1)

\begin{document}
\maketitle


\kategoria{Wikieł/Z3.34c}
\zadStart{Zadanie z Wikieł Z 3.34 c) moja wersja nr [nrWersji]}
%[a]:[2,3,4,5,6,7,8,9]
%[b]:[2,3,4,5,6,7,8]
%[c]:[2,3,4,5,6,7,8,9]
%[d]=6
%[b3]=[b]*3
%[a6]=[a]*6
%[wy]=[c]+[b3]+[a6]
%math.gcd([wy],[d])==1
Obliczyć granicę ciągu $$a_n=\frac{[a]}{n}\binom{n}{1}+\frac{[b]}{n^2}\binom{n}{2}+\frac{[c]}{n^3}\binom{n}{3}.$$
\zadStop
\rozwStart{Aleksandra Pasińska}{}
$$\lim_{n\rightarrow \infty}\biggl(\frac{[a]}{n}\binom{n}{1}+\frac{[b]}{n^2}\binom{n}{2}+\frac{[c]}{n^3}\binom{n}{3}\biggr)=$$
$$=\lim_{n\rightarrow \infty}\biggl(\frac{[a]}{n}\binom{n}{1}\biggr)+\lim_{n\rightarrow \infty}\biggl(\frac{[b]}{n^2}\binom{n}{2}\biggr)+\lim_{n\rightarrow \infty}\biggl(\frac{[c]}{n^3}\binom{n}{3}\biggr)=$$
$$=\lim_{n\rightarrow \infty}\biggl(\frac{[a]n(n-1)!}{n(n-1)!}\biggr)+\lim_{n\rightarrow \infty}\biggl(\frac{[b]n(n-1)(n-2)!}{2n^2(n-2)!}\biggr)+\lim_{n\rightarrow \infty}\biggl(\frac{[c]n(n-1)(n-2)(n-3)!}{6n^3(n-3)!}\biggr)=$$
$$=\lim_{n\rightarrow \infty}([a])+\lim_{n\rightarrow \infty}\biggl(\frac{[b](n-1)}{2n}\biggr)+\lim_{n\rightarrow \infty}\biggl(\frac{[c](n-1)(n-2)}{6n^2}\biggr)=$$
$$[a]+\frac{[b]}{2}+\lim_{n\rightarrow \infty}\biggl(\frac{[c](n^2-3n+2)}{6n^2}\biggr)=[a]+\frac{[b]}{2}+\frac{[c]}{6}=\frac{[wy]}{[d]}$$
\rozwStop
\odpStart
$\frac{[wy]}{[d]}$\\
\odpStop
\testStart
A.$\frac{[wy]}{[d]}$
B.$\infty$
C.$-\infty$
D.$1$
E.$9$
F.$e$
G.$4$
H.$7$
I.$-7$
\testStop
\kluczStart
A
\kluczStop



\end{document}