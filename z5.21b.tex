\documentclass[12pt, a4paper]{article}
\usepackage[utf8]{inputenc}
\usepackage{polski}

\usepackage{amsthm}  %pakiet do tworzenia twierdzeń itp.
\usepackage{amsmath} %pakiet do niektórych symboli matematycznych
\usepackage{amssymb} %pakiet do symboli mat., np. \nsubseteq
\usepackage{amsfonts}
\usepackage{graphicx} %obsługa plików graficznych z rozszerzeniem png, jpg
\theoremstyle{definition} %styl dla definicji
\newtheorem{zad}{} 
\title{Multizestaw zadań}
\author{Radosław Grzyb}
%\date{\today}
\date{}
\newcounter{liczniksekcji}
\newcommand{\kategoria}[1]{\section{#1}} %olreślamy nazwę kateforii zadań
\newcommand{\zadStart}[1]{\begin{zad}#1\newline} %oznaczenie początku zadania
\newcommand{\zadStop}{\end{zad}}   %oznaczenie końca zadania
%Makra opcjonarne (nie muszą występować):
\newcommand{\rozwStart}[2]{\noindent \textbf{Rozwiązanie (autor #1 , recenzent #2): }\newline} %oznaczenie początku rozwiązania, opcjonarnie można wprowadzić informację o autorze rozwiązania zadania i recenzencie poprawności wykonania rozwiązania zadania
\newcommand{\rozwStop}{\newline}                                            %oznaczenie końca rozwiązania
\newcommand{\odpStart}{\noindent \textbf{Odpowiedź:}\newline}    %oznaczenie początku odpowiedzi końcowej (wypisanie wyniku)
\newcommand{\odpStop}{\newline}                                             %oznaczenie końca odpowiedzi końcowej (wypisanie wyniku)
\newcommand{\testStart}{\noindent \textbf{Test:}\newline} %ewentualne możliwe opcje odpowiedzi testowej: A. ? B. ? C. ? D. ? itd.
\newcommand{\testStop}{\newline} %koniec wprowadzania odpowiedzi testowych
\newcommand{\kluczStart}{\noindent \textbf{Test poprawna odpowiedź:}\newline} %klucz, poprawna odpowiedź pytania testowego (jedna literka): A lub B lub C lub D itd.
\newcommand{\kluczStop}{\newline} %koniec poprawnej odpowiedzi pytania testowego 
\newcommand{\wstawGrafike}[2]{\begin{figure}[h] \includegraphics[scale=#2] {#1} \end{figure}} %gdyby była potrzeba wstawienia obrazka, parametry: nazwa pliku, skala (jak nie wiesz co wpisać, to wpisz 1)

\begin{document}
\maketitle


\kategoria{Wikieł/Z5.21b}
\zadStart{Zadanie z Wikieł Z 5.21 b) moja wersja nr [nrWersji]}
%[b]:[2,3,4,5,6,7,8,9,10,11,12]
%[c]:[2,3,4,5,6,7,8,9,10,11,12]
%[d]=3*[b]
%[f]=3*[c]
%[h]=[d]+[b]
%math.gcd([f],[h])==1
Wyznaczyć przedziały monotoniczności funkcji f:
$$f(x)=x^3([b]x-[c])$$
\zadStop
\rozwStart{Klaudia Klejdysz}{}
Dziedziną funkcji f jest zbiór liczb rzeczywistych, tj. $D_f=\mathbb{R}$.\\
Obliczamy pochodną funkcji
$$f'(x)=3x^2([b]x-[c])+x^3*[b]=[d]x^3-[f]x^2+[b]x^3=[h]x^3-[f]x^2\text{.}$$
Ponieważ:
$$f'(x)>0 \Leftrightarrow [h]x^3-[f]x^2>0\Leftrightarrow x^2([h]x-[f])>0\Leftrightarrow[h]x-[f]>0\Leftrightarrow[h]x>[f]\Leftrightarrow x>\frac{[f]}{[h]}$$
i podobnie:
$$f'(x)<0 \Leftrightarrow [h]x^3-[f]x^2<0\Leftrightarrow x^2([h]x-[f])<0\Leftrightarrow[h]x-[f]<0\Leftrightarrow[h]x<[f]\Leftrightarrow x<\frac{[f]}{[h]}$$
więc funkcja f jest malejąca w przedziale $(-\infty,\frac{[f]}{[h]})$, a rosnąca w przedziale $(\frac{[f]}{[h]},\infty)$.
\\
\rozwStop
\odpStart
rosnąca: $(\frac{[f]}{[h]},\infty)$, malejąca: $(-\infty,\frac{[f]}{[h]})$
\odpStop
\testStart
A. rosnąca: $(\frac{[f]}{[h]},\infty)$, malejąca: $(-\infty,\frac{[f]}{[h]})$\\
B. rosnąca: $([b],\infty)$, malejąca: $(-\infty,[b])$\\
C. rosnąca: $([c],\infty)$, malejąca: $(-\infty,[c])$\\
D. rosnąca: $(-\infty,\frac{[f]}{[h]})$, malejąca: $(\frac{[f]}{[h]},\infty)$\\
E. rosnąca: $(-\infty,[f])$, malejąca: $([h],\infty)$\\
F. rosnąca: $(-\infty,[h])$, malejąca: $([f],\infty)$
\testStop
\kluczStart
A
\kluczStop



\end{document}
