\documentclass[12pt, a4paper]{article}
\usepackage[utf8]{inputenc}
\usepackage{polski}

\usepackage{amsthm}  %pakiet do tworzenia twierdzeń itp.
\usepackage{amsmath} %pakiet do niektórych symboli matematycznych
\usepackage{amssymb} %pakiet do symboli mat., np. \nsubseteq
\usepackage{amsfonts}
\usepackage{graphicx} %obsługa plików graficznych z rozszerzeniem png, jpg
\theoremstyle{definition} %styl dla definicji
\newtheorem{zad}{} 
\title{Multizestaw zadań}
\author{Robert Fidytek}
%\date{\today}
\date{}
\newcounter{liczniksekcji}
\newcommand{\kategoria}[1]{\section{#1}} %olreślamy nazwę kateforii zadań
\newcommand{\zadStart}[1]{\begin{zad}#1\newline} %oznaczenie początku zadania
\newcommand{\zadStop}{\end{zad}}   %oznaczenie końca zadania
%Makra opcjonarne (nie muszą występować):
\newcommand{\rozwStart}[2]{\noindent \textbf{Rozwiązanie (autor #1 , recenzent #2): }\newline} %oznaczenie początku rozwiązania, opcjonarnie można wprowadzić informację o autorze rozwiązania zadania i recenzencie poprawności wykonania rozwiązania zadania
\newcommand{\rozwStop}{\newline}                                            %oznaczenie końca rozwiązania
\newcommand{\odpStart}{\noindent \textbf{Odpowiedź:}\newline}    %oznaczenie początku odpowiedzi końcowej (wypisanie wyniku)
\newcommand{\odpStop}{\newline}                                             %oznaczenie końca odpowiedzi końcowej (wypisanie wyniku)
\newcommand{\testStart}{\noindent \textbf{Test:}\newline} %ewentualne możliwe opcje odpowiedzi testowej: A. ? B. ? C. ? D. ? itd.
\newcommand{\testStop}{\newline} %koniec wprowadzania odpowiedzi testowych
\newcommand{\kluczStart}{\noindent \textbf{Test poprawna odpowiedź:}\newline} %klucz, poprawna odpowiedź pytania testowego (jedna literka): A lub B lub C lub D itd.
\newcommand{\kluczStop}{\newline} %koniec poprawnej odpowiedzi pytania testowego 
\newcommand{\wstawGrafike}[2]{\begin{figure}[h] \includegraphics[scale=#2] {#1} \end{figure}} %gdyby była potrzeba wstawienia obrazka, parametry: nazwa pliku, skala (jak nie wiesz co wpisać, to wpisz 1)

\begin{document}
\maketitle


\kategoria{Wikieł/Z3.23}
\zadStart{Zadanie z Wikieł Z 3.23 moja wersja nr [nrWersji]}
%[a]:[2,3,4,5,6,7,8,9,10,11,12,13,14,15,16,17,18,19,20]
%[2a]=2*[a]
%
Wyznaczyć ekstrema funkcji $f(x)=[a]x^2+[a]x^3+[a]x^4+\dots$, będącej sumą nieskończonego ciągu geometrycznego.
\zadStop
\rozwStart{Adrianna Stobiecka}{}
Funkcja jest sumą nieskończonego ciągu geometrycznego o pierwszym wyrazie $a_1=[a]x^2$ oraz $q=x$. Zakładamy $|q|<1$. 
$$|q|<1\qquad\Leftrightarrow\qquad|x|<1\qquad\Leftrightarrow\qquad x<1~~\land~~ x>-1\qquad\Leftrightarrow\qquad x\in(-1,1)$$
Mamy zatem założenie, że $x\in(-1,1)$. Następnie obliczamy sumę ciągu. Mamy zatem
$$f(x)=\frac{[a]x^2}{1-x}.$$
Przechodzimy do obliczena pochodnej funkcji $f(x)$ oraz znalezienia pierwiastków funkcji $f'(x)$.
$$f'(x)=\frac{2\cdot[a]x(1-x)-[a]x^2\cdot(-1)}{(1-x)^2}=\frac{[2a]x-[2a]x^2+[a]x^2}{(1-x)^2}=\frac{[2a]x-[a]x^2}{(1-x)^2}$$
$$f'(x)=0\Leftrightarrow\frac{[2a]x-[a]x^2}{(1-x)^2}=0\Leftrightarrow[2a]x-[a]x^2=0\Leftrightarrow-[a]x(x-2)=0$$
Funkcja $-[a]x(x-2)$ jest parabolą z pierwiastkami w punktach $x_1=0$ i $x_2=2$. Zauważamy, że punkt $x_2=2$ nie należy do przedziału $(-1,1)$. Natomiast $x_1=0\in(-1,1)$. Obliczmy jeszcze $f(0)$.
$$f(0)=\frac{[a]\cdot0}{1-0}=0$$
Otrzymujemy zatem, że funkcja $f(x)=\frac{[a]x^2}{1-x}$ dla $x\in(-1,1)$, osiąga minumum w punkcie $x=0$ równe $f(0)=0$.
\rozwStop
\odpStart
Minimum w punkcie $x=0$ równe $f(0)=0$
\odpStop
\testStart
A.Minimum w punkcie $x=[a]$ równe $f([a])=\frac{1}{[a]}$\\
B.Minimum w punkcie $x=[a]$ równe $f([a])=0$\\
C.Maksimum w punkcie $x=[a]$ równe $f([a])=\frac{1}{[a]}$\\
D.Minimum w punkcie $x=0$ równe $f(0)=0$\\
E.Maksimum w punkcie $x=[a]$ równe $f([a])=0$\\
F.Minimum w punkcie $x=0$ równe $f(0)=[a]$\\
G.Maksimum w punkcie $x=0$ równe $f(0)=[a]$\\
H.Maksimum w punkcie $x=0$ równe $f(0)=0$\\
I.Maksimum w punkcie $x=0$ równe $f(0)=[2a]$
\testStop
\kluczStart
D
\kluczStop



\end{document}
