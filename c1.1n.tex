\documentclass[12pt, a4paper]{article}
\usepackage[utf8]{inputenc}
\usepackage{polski}

\usepackage{amsthm}  %pakiet do tworzenia twierdzeń itp.
\usepackage{amsmath} %pakiet do niektórych symboli matematycznych
\usepackage{amssymb} %pakiet do symboli mat., np. \nsubseteq
\usepackage{amsfonts}
\usepackage{graphicx} %obsługa plików graficznych z rozszerzeniem png, jpg
\theoremstyle{definition} %styl dla definicji
\newtheorem{zad}{} 
\title{Multizestaw zadań}
\author{Robert Fidytek}
%\date{\today}
\date{}
\newcounter{liczniksekcji}
\newcommand{\kategoria}[1]{\section{#1}} %olreślamy nazwę kateforii zadań
\newcommand{\zadStart}[1]{\begin{zad}#1\newline} %oznaczenie początku zadania
\newcommand{\zadStop}{\end{zad}}   %oznaczenie końca zadania
%Makra opcjonarne (nie muszą występować):
\newcommand{\rozwStart}[2]{\noindent \textbf{Rozwiązanie (autor #1 , recenzent #2): }\newline} %oznaczenie początku rozwiązania, opcjonarnie można wprowadzić informację o autorze rozwiązania zadania i recenzencie poprawności wykonania rozwiązania zadania
\newcommand{\rozwStop}{\newline}                                            %oznaczenie końca rozwiązania
\newcommand{\odpStart}{\noindent \textbf{Odpowiedź:}\newline}    %oznaczenie początku odpowiedzi końcowej (wypisanie wyniku)
\newcommand{\odpStop}{\newline}                                             %oznaczenie końca odpowiedzi końcowej (wypisanie wyniku)
\newcommand{\testStart}{\noindent \textbf{Test:}\newline} %ewentualne możliwe opcje odpowiedzi testowej: A. ? B. ? C. ? D. ? itd.
\newcommand{\testStop}{\newline} %koniec wprowadzania odpowiedzi testowych
\newcommand{\kluczStart}{\noindent \textbf{Test poprawna odpowiedź:}\newline} %klucz, poprawna odpowiedź pytania testowego (jedna literka): A lub B lub C lub D itd.
\newcommand{\kluczStop}{\newline} %koniec poprawnej odpowiedzi pytania testowego 
\newcommand{\wstawGrafike}[2]{\begin{figure}[h] \includegraphics[scale=#2] {#1} \end{figure}} %gdyby była potrzeba wstawienia obrazka, parametry: nazwa pliku, skala (jak nie wiesz co wpisać, to wpisz 1)

\begin{document}
\maketitle
\kategoria{Wikieł/C1.1n}
\zadStart{Zadanie z Wikieł C 1.1n moja wersja nr [nrWersji]}
%[a]:[2,3,4,5,6,7]
%[b]:[2,3,4,5,6,7]
%[c]:[2,3,4,5,6,7]
%[d]:[2,3,4,5,6,7]
%[e]:[2,3,4,5,6,7,8,9]
%math.gcd([a],[e])==1 and math.gcd([b],[e])==1 and math.gcd([c],[e])==1 and math.gcd([d],[e])==1 and [a]!=[b] and [a]!=[c] and [a]!=[d] and [b]!=[c] and [b]!=[d] and [c]!=[d]
Oblicz całkę $$\int \frac{[a]\cdot [b]^{x}-[c]\cdot [d]^{x}}{[e]^{x}} dx.$$
\zadStop
\rozwStart{Justyna Chojecka}{}
$$\int \frac{[a]\cdot [b]^{x}-[c]\cdot [d]^{x}}{[e]^{x}} dx=\int\left( \frac{[a]\cdot [b]^{x}}{[e]^{x}}-\frac{[c]\cdot [d]^{x}}{[e]^{x}}\right) dx$$$$=\int\frac{[a]\cdot [b]^{x}}{[e]^{x}}dx-\int \frac{[c]\cdot [d]^{x}}{[e]^{x}} dx=[a] \int\frac{[b]^{x}}{[e]^{x}}dx-[c] \int \frac{[d]^{x}}{[e]^{x}}dx$$$$=[a] \int\left(\frac{[b]}{[e]}\right)^{x}dx-[c] \int\left( \frac{[d]}{[e]}\right)^{x}dx=\frac{[a]}{ln\left(\frac{[b]}{[e]}\right)}\left(\frac{[b]}{[e]}\right)^{x}-\frac{[c]}{ln\left(\frac{[d]}{[e]}\right)}\left(\frac{[d]}{[e]}\right)^{x}+C$$
\rozwStop
\odpStart
$\frac{[a]}{ln\left(\frac{[b]}{[e]}\right)}\left(\frac{[b]}{[e]}\right)^{x}-\frac{[c]}{ln\left(\frac{[d]}{[e]}\right)}\left(\frac{[d]}{[e]}\right)^{x}+C$
\odpStop
\testStart
A.$\frac{[a]}{ln\left(\frac{[b]}{[e]}\right)}\left(\frac{[b]}{[e]}\right)^{x}-\frac{[c]}{ln\left(\frac{[d]}{[e]}\right)}\left(\frac{[d]}{[e]}\right)^{x}+C$\\
B.$\frac{[a]}{ln\left(\frac{[b]}{[e]}\right)}\left(\frac{[b]}{[e]}\right)^{x}-\frac{[d]}{ln\left(\frac{[c]}{[e]}\right)}\left(\frac{[d]}{[e]}\right)^{x}+C$\\
C.$-\frac{[a]}{ln\left(\frac{[b]}{[e]}\right)}\left(\frac{[b]}{[e]}\right)^{x}-\frac{[c]}{ln\left(\frac{[d]}{[e]}\right)}\left(\frac{[d]}{[e]}\right)^{x}+C$\\
D.$\frac{[a]}{ln\left(\frac{[b]}{[e]}\right)}\left(\frac{[b]}{[e]}\right)^{x}-\frac{[c]}{ln\left(\frac{[e]}{[d]}\right)}\left(\frac{[d]}{[e]}\right)^{x}+C$\\
E.$\frac{[a]}{ln\left(\frac{[e]}{[b]}\right)}\left(\frac{[b]}{[e]}\right)^{x}+\frac{[c]}{ln\left(\frac{[d]}{[e]}\right)}\left(\frac{[d]}{[e]}\right)^{x}+C$\\
F.$\frac{[a]}{ln\left(\frac{[e]}{[b]}\right)}\left(\frac{[b]}{[e]}\right)^{x}-\frac{[c]}{ln\left(\frac{[d]}{[e]}\right)}\left(\frac{[d]}{[e]}\right)^{x}+C$\\
G.$\frac{[a]}{ln\left(\frac{[b]}{[e]}\right)}\left(\frac{[b]}{[e]}\right)^{x}+\frac{[c]}{ln\left(\frac{[d]}{[e]}\right)}\left(\frac{[d]}{[e]}\right)^{x}+C$\\
H.$\frac{[a]}{ln\left(\frac{[b]}{[e]}\right)}\left(\frac{[b]}{[e]}\right)^{x}+\frac{[c]}{ln\left(\frac{[e]}{[d]}\right)}\left(\frac{[d]}{[e]}\right)^{x}+C$\\
I.$-\frac{[a]}{ln\left(\frac{[b]}{[e]}\right)}\left(\frac{[b]}{[e]}\right)^{x}+\frac{[c]}{ln\left(\frac{[d]}{[e]}\right)}\left(\frac{[d]}{[e]}\right)^{x}+C$\\
\testStop
\kluczStart
A
\kluczStop



\end{document}