\documentclass[12pt, a4paper]{article}
\usepackage[utf8]{inputenc}
\usepackage{polski}

\usepackage{amsthm}  %pakiet do tworzenia twierdzeń itp.
\usepackage{amsmath} %pakiet do niektórych symboli matematycznych
\usepackage{amssymb} %pakiet do symboli mat., np. \nsubseteq
\usepackage{amsfonts}
\usepackage{graphicx} %obsługa plików graficznych z rozszerzeniem png, jpg
\theoremstyle{definition} %styl dla definicji
\newtheorem{zad}{} 
\title{Multizestaw zadań}
\author{Laura Mieczkowska}
%\date{\today}
\date{}
\newcounter{liczniksekcji}
\newcommand{\kategoria}[1]{\section{#1}} %olreślamy nazwę kateforii zadań
\newcommand{\zadStart}[1]{\begin{zad}#1\newline} %oznaczenie początku zadania
\newcommand{\zadStop}{\end{zad}}   %oznaczenie końca zadania
%Makra opcjonarne (nie muszą występować):
\newcommand{\rozwStart}[2]{\noindent \textbf{Rozwiązanie (autor #1 , recenzent #2): }\newline} %oznaczenie początku rozwiązania, opcjonarnie można wprowadzić informację o autorze rozwiązania zadania i recenzencie poprawności wykonania rozwiązania zadania
\newcommand{\rozwStop}{\newline}                                            %oznaczenie końca rozwiązania
\newcommand{\odpStart}{\noindent \textbf{Odpowiedź:}\newline}    %oznaczenie początku odpowiedzi końcowej (wypisanie wyniku)
\newcommand{\odpStop}{\newline}                                             %oznaczenie końca odpowiedzi końcowej (wypisanie wyniku)
\newcommand{\testStart}{\noindent \textbf{Test:}\newline} %ewentualne możliwe opcje odpowiedzi testowej: A. ? B. ? C. ? D. ? itd.
\newcommand{\testStop}{\newline} %koniec wprowadzania odpowiedzi testowych
\newcommand{\kluczStart}{\noindent \textbf{Test poprawna odpowiedź:}\newline} %klucz, poprawna odpowiedź pytania testowego (jedna literka): A lub B lub C lub D itd.
\newcommand{\kluczStop}{\newline} %koniec poprawnej odpowiedzi pytania testowego 
\newcommand{\wstawGrafike}[2]{\begin{figure}[h] \includegraphics[scale=#2] {#1} \end{figure}} %gdyby była potrzeba wstawienia obrazka, parametry: nazwa pliku, skala (jak nie wiesz co wpisać, to wpisz 1)

\begin{document}
\maketitle


\kategoria{Wikieł/Z1.4a}
\zadStart{Zadanie z Wikieł Z 1.4 a) moja wersja nr [nrWersji]}
%[a]:[2,3,4,5,6,7,8,9,10]
%[b]:[2,3,4,5]
%[c]:[2,3,4,5]
%[d1]=[b]**3
%[d2]=[c]**3
%[d]=[d1]*[d2]
%[bkw]=[b]*[b]
%[ckw]=[c]*[c]
%[ulamek1]=1/[b]
%[ulamek2]=1/[c]
%[cb]=[b]*[c]
%[r1]=[c]-[b]
%[w]=[a]+[r1]
%[c]>[b] and [ulamek1]>[ulamek2]
Obliczyć wartość wyrażenia $[a]+\sqrt[3]{[d]}\cdot \big(\frac{[b]^{-2}}{[b]^{-1}}-\frac{[c]^{-2}}{[c]^{-1}}\big)$.
\zadStop
\rozwStart{Laura Mieczkowska}{}
$$[a]+\sqrt[3]{[d]}\cdot \big(\frac{[b]^{-2}}{[b]^{-1}}-\frac{[c]^{-2}}{[c]^{-1}}\big)
=[a]+\sqrt[3]{[b]^3\cdot[c]^3}\cdot \big(\frac{(\frac{1}{[b]})^{2}}{\frac{1}{[b]}}-\frac{(\frac{1}{[c]})^{2}}{\frac{1}{[c]}}\big)=$$
$$=[a]+[b]\cdot[c]\cdot\big(\frac{1}{[bkw]}\cdot[b]-\frac{1}{[ckw]}\cdot[c]\big)=[a]+[cb]\cdot\frac{[r1]}{[cb]}=[w]$$


\odpStart
$[w]$
\odpStop
\testStart
A. $[w]$ \\
B. $[a]$ \\
C. $[cb]$ \\
D. $1$ 
\testStop
\kluczStart
A
\kluczStop



\end{document}