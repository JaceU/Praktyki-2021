\documentclass[12pt, a4paper]{article}
\usepackage[utf8]{inputenc}
\usepackage{polski}

\usepackage{amsthm}  %pakiet do tworzenia twierdzeń itp.
\usepackage{amsmath} %pakiet do niektórych symboli matematycznych
\usepackage{amssymb} %pakiet do symboli mat., np. \nsubseteq
\usepackage{amsfonts}
\usepackage{graphicx} %obsługa plików graficznych z rozszerzeniem png, jpg
\theoremstyle{definition} %styl dla definicji
\newtheorem{zad}{} 
\title{Multizestaw zadań}
\author{Jacek Jabłoński}
%\date{\today}
\date{}
\newcounter{liczniksekcji}
\newcommand{\kategoria}[1]{\section{#1}} %olreślamy nazwę kateforii zadań
\newcommand{\zadStart}[1]{\begin{zad}#1\newline} %oznaczenie początku zadania
\newcommand{\zadStop}{\end{zad}}   %oznaczenie końca zadania
%Makra opcjonarne (nie muszą występować):
\newcommand{\rozwStart}[2]{\noindent \textbf{Rozwiązanie (autor #1 , recenzent #2): }\newline} %oznaczenie początku rozwiązania, opcjonarnie można wprowadzić informację o autorze rozwiązania zadania i recenzencie poprawności wykonania rozwiązania zadania
\newcommand{\rozwStop}{\newline}                                            %oznaczenie końca rozwiązania
\newcommand{\odpStart}{\noindent \textbf{Odpowiedź:}\newline}    %oznaczenie początku odpowiedzi końcowej (wypisanie wyniku)
\newcommand{\odpStop}{\newline}                                             %oznaczenie końca odpowiedzi końcowej (wypisanie wyniku)
\newcommand{\testStart}{\noindent \textbf{Test:}\newline} %ewentualne możliwe opcje odpowiedzi testowej: A. ? B. ? C. ? D. ? itd.
\newcommand{\testStop}{\newline} %koniec wprowadzania odpowiedzi testowych
\newcommand{\kluczStart}{\noindent \textbf{Test poprawna odpowiedź:}\newline} %klucz, poprawna odpowiedź pytania testowego (jedna literka): A lub B lub C lub D itd.
\newcommand{\kluczStop}{\newline} %koniec poprawnej odpowiedzi pytania testowego 
\newcommand{\wstawGrafike}[2]{\begin{figure}[h] \includegraphics[scale=#2] {#1} \end{figure}} %gdyby była potrzeba wstawienia obrazka, parametry: nazwa pliku, skala (jak nie wiesz co wpisać, to wpisz 1)

\begin{document}
\maketitle


\kategoria{Wikieł/z1.84d}
\zadStart{Zadanie z Wikieł z1.84d) moja wersja nr [nrWersji]}
%[p1]:[2,3,4]
%[p2]:[2,3,4,5]
%[p3]:[2,3,4]
%[a]=random.randint(2,4)
%[b]=random.randint(2,4)
%[c]=random.randint(2,4)
%[r1]=int(math.pow(2,[p1]))
%[r2]=[p3]
%[r3]=int(math.pow(2,[r2]))
%[r4]=[p1]*[a]
%[r5]=[r2]*[b]
%[r6]=[r5]*[c]
%[r7]=[p1]-([r5]*[p2])
%[r8]=[r6]+[r4]
%[r9]=[r8]*[p2]
%[r10]=int(math.gcd([r9],[r7]))
%[r11]=int([r9]/[r10])
%[r12]=int(([r7]/[r10])*(-1))
%[ra]=[r11]+[a]
%[rb]=[r11]+[b]
%[rc]=[r11]+[c]
%[rra]=[r12]+[a]
%[rrb]=[r12]+[b]
%[rrc]=[r12]+[c]
Rozwiązać równanie:
d) $[r1]^{\frac{1}{[p2]} x - [a]} = [r3]^{[b] (x + [c])}$
\zadStop
\rozwStart{Jacek Jabłoński}{}
$$[r1]^{\frac{1}{[p2]} x - [a]} = [r3]^{[b] (x + [c])}$$
$$2^{[p1](\frac{1}{[p2]} x - [a])} = 2^{[r2]([b] (x + [c]))}$$
$$[p1](\frac{1}{[p2]} x - [a]) = [r2]([b] (x + [c]))$$
$$\frac{[p1]}{[p2]}x - [r4] = [r5]x + [r6] $$
$$\frac{[p1]}{[p2]}x - [r5]x = [r6]+[r4]$$
$$x(\frac{[r7]}{[p2]}) = [r8]$$
$$x=[r8] \cdot \frac{[p2]}{[r7]}$$
$$x=\frac{[r9]}{[r7]}$$
$$x=- \frac{[r11]}{[r12]}$$
\rozwStop
\odpStart
$$x=- \frac{[r11]}{[r12]}$$
\odpStop
\testStart
A. $$x=- \frac{[r11]}{[r12]}$$
B. $$x = \frac{[ra]}{[r12]}$$
C. $$x = \frac{[r11]}{[rra]}$$
D. $$x = \frac{[rb]}{[r12]}$$
E. $$x = \frac{[r11]}{[rrb]}$$
F. $$x = \frac{[rc]}{[r12]}$$
G. $$x = \frac{[r11]}{[rrc]}$$
H. $$x = \frac{[ra]}{[rrb]}$$
I. $$x = \frac{[rc]}{[rra]}$$
\testStop
\kluczStart
A
\kluczStop



\end{document}