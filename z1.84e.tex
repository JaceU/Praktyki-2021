\documentclass[12pt, a4paper]{article}
\usepackage[utf8]{inputenc}
\usepackage{polski}

\usepackage{amsthm}  %pakiet do tworzenia twierdzeń itp.
\usepackage{amsmath} %pakiet do niektórych symboli matematycznych
\usepackage{amssymb} %pakiet do symboli mat., np. \nsubseteq
\usepackage{amsfonts}
\usepackage{graphicx} %obsługa plików graficznych z rozszerzeniem png, jpg
\theoremstyle{definition} %styl dla definicji
\newtheorem{zad}{} 
\title{Multizestaw zadań}
\author{Jacek Jabłoński}
%\date{\today}
\date{}
\newcounter{liczniksekcji}
\newcommand{\kategoria}[1]{\section{#1}} %olreślamy nazwę kateforii zadań
\newcommand{\zadStart}[1]{\begin{zad}#1\newline} %oznaczenie początku zadania
\newcommand{\zadStop}{\end{zad}}   %oznaczenie końca zadania
%Makra opcjonarne (nie muszą występować):
\newcommand{\rozwStart}[2]{\noindent \textbf{Rozwiązanie (autor #1 , recenzent #2): }\newline} %oznaczenie początku rozwiązania, opcjonarnie można wprowadzić informację o autorze rozwiązania zadania i recenzencie poprawności wykonania rozwiązania zadania
\newcommand{\rozwStop}{\newline}                                            %oznaczenie końca rozwiązania
\newcommand{\odpStart}{\noindent \textbf{Odpowiedź:}\newline}    %oznaczenie początku odpowiedzi końcowej (wypisanie wyniku)
\newcommand{\odpStop}{\newline}                                             %oznaczenie końca odpowiedzi końcowej (wypisanie wyniku)
\newcommand{\testStart}{\noindent \textbf{Test:}\newline} %ewentualne możliwe opcje odpowiedzi testowej: A. ? B. ? C. ? D. ? itd.
\newcommand{\testStop}{\newline} %koniec wprowadzania odpowiedzi testowych
\newcommand{\kluczStart}{\noindent \textbf{Test poprawna odpowiedź:}\newline} %klucz, poprawna odpowiedź pytania testowego (jedna literka): A lub B lub C lub D itd.
\newcommand{\kluczStop}{\newline} %koniec poprawnej odpowiedzi pytania testowego 
\newcommand{\wstawGrafike}[2]{\begin{figure}[h] \includegraphics[scale=#2] {#1} \end{figure}} %gdyby była potrzeba wstawienia obrazka, parametry: nazwa pliku, skala (jak nie wiesz co wpisać, to wpisz 1)

\begin{document}
\maketitle


\kategoria{Wikieł/z1.84e}
\zadStart{Zadanie z Wikieł z1.84e) moja wersja nr [nrWersji]}
%[p1]:[2,3,4]
%[p2]:[1,2,3]
%[p3]:[1,2,3]
%[a]=random.randint(2,6)
%[b]=random.randint(2,6)
%[c]=random.randint(2,6)
%[r1]=int(math.pow(3,[p3]))
%[r2]=int(math.pow(3,[p2]))
%[r3]=int(math.pow(3,[p1]))
%[r4]=[p1]*[a]*(-1)
%[r5]=[p2]*[b]
%[r6]=[p2]*[c]
%[r7]=-[r6] + [p3] - [r4]
%[r8]=[p1]-[r5]
%[rrr]=int(math.gcd([r7],[r8]))
%[rr7]=int([r7]/[rrr])
%[rr8]=int([r8]/[rrr])
%[f1a]=[r7]+1
%[f1b]=[r7]+2
%[f1c]=[r7]+4
%[f2a]=[r8]+1
%[f2b]=[r8]+2
%[f2c]=[r8]+4
%[r8]!=0 and [f2a]!=0 and [f2b]!=0 and [f2c]!=0
Rozwiązać równanie:
e) $(\frac{1}{[r3]})^{[a]-x} = [r1] \cdot [r2]^{[b] x - [c]}$
\zadStop
\rozwStart{Jacek Jabłoński}{}
$$(\frac{1}{[r3]})^{[a]-x} = [r1] \cdot [r2]^{[b] x - [c]}$$
$$[r3]^{-([a]-x)} = 3^{[p3]} \cdot 3^{[p2]([b] x - [c])}$$
$$3^{-[p1]([a]-x)} = 3^{[p2]([b]x-[c]) + [p3]}$$
$$ -[p1]([a]-x) = [p2]([b]x-[c]) + [p3]$$
$$[r4] + [p1]x = [r5]x - [r6] +[p3]$$
$$x([p1]-[r5]) = [r7]$$
$$[r8]x = [r7]$$
$$ x = \frac{[rr7]}{[rr8]}$$
\rozwStop
\odpStart
$$ x = \frac{[rr7]}{[rr8]}$$
\odpStop
\testStart
A. $$ x = \frac{[rr7]}{[rr8]}$$
B. $$ x = \frac{[f1a]}{[r8]}$$
C. $$ x = \frac{[f1b]}{[r8]}$$
D. $$ x = \frac{[f1c]}{[r8]}$$
E. $$ x = \frac{[f1a]}{[r8]} + \frac{[f1a]}{[f2c]}$$
F. $$ x = [f1a]$$
G. $$ x = [f2a]$$
H. $$ x = [f1c]$$
I. $$ x = [f2c]$$
\testStop
\kluczStart
A
\kluczStop



\end{document}