\documentclass[12pt, a4paper]{article}
\usepackage[utf8]{inputenc}
\usepackage{polski}

\usepackage{amsthm}  %pakiet do tworzenia twierdzeń itp.
\usepackage{amsmath} %pakiet do niektórych symboli matematycznych
\usepackage{amssymb} %pakiet do symboli mat., np. \nsubseteq
\usepackage{amsfonts}
\usepackage{graphicx} %obsługa plików graficznych z rozszerzeniem png, jpg
\theoremstyle{definition} %styl dla definicji
\newtheorem{zad}{} 
\title{Multizestaw zadań}
\author{Robert Fidytek}
%\date{\today}
\date{}
\newcounter{liczniksekcji}
\newcommand{\kategoria}[1]{\section{#1}} %olreślamy nazwę kateforii zadań
\newcommand{\zadStart}[1]{\begin{zad}#1\newline} %oznaczenie początku zadania
\newcommand{\zadStop}{\end{zad}}   %oznaczenie końca zadania
%Makra opcjonarne (nie muszą występować):
\newcommand{\rozwStart}[2]{\noindent \textbf{Rozwiązanie (autor #1 , recenzent #2): }\newline} %oznaczenie początku rozwiązania, opcjonarnie można wprowadzić informację o autorze rozwiązania zadania i recenzencie poprawności wykonania rozwiązania zadania
\newcommand{\rozwStop}{\newline}                                            %oznaczenie końca rozwiązania
\newcommand{\odpStart}{\noindent \textbf{Odpowiedź:}\newline}    %oznaczenie początku odpowiedzi końcowej (wypisanie wyniku)
\newcommand{\odpStop}{\newline}                                             %oznaczenie końca odpowiedzi końcowej (wypisanie wyniku)
\newcommand{\testStart}{\noindent \textbf{Test:}\newline} %ewentualne możliwe opcje odpowiedzi testowej: A. ? B. ? C. ? D. ? itd.
\newcommand{\testStop}{\newline} %koniec wprowadzania odpowiedzi testowych
\newcommand{\kluczStart}{\noindent \textbf{Test poprawna odpowiedź:}\newline} %klucz, poprawna odpowiedź pytania testowego (jedna literka): A lub B lub C lub D itd.
\newcommand{\kluczStop}{\newline} %koniec poprawnej odpowiedzi pytania testowego 
\newcommand{\wstawGrafike}[2]{\begin{figure}[h] \includegraphics[scale=#2] {#1} \end{figure}} %gdyby była potrzeba wstawienia obrazka, parametry: nazwa pliku, skala (jak nie wiesz co wpisać, to wpisz 1)

\begin{document}
\maketitle


\kategoria{Wikieł/Z1.98d}
\zadStart{Zadanie z Wikieł Z 1.98 d) moja wersja nr [nrWersji]}
%[a]:[2,3,4,5,6,7,8,9]
%[b]:[2,3,4,5,6,7,8]
%[c]:[2,3,4,5,6,7,8]
%[d]:[1,2,3,4,5,6,7,8,9,10,11,12,13,14,15]
%[e]:[2,3]
%[aa]=pow([a],[e])
%[ff]=([aa]-[d])/[c]
%[f]=int([ff])
%[gg]=[d]/[c]
%[g]=int([gg])
%[aa]<100 and [f]!=0 and [ff].is_integer()==True and [gg].is_integer()==True and [f]>[b]
Wyznaczyć dziedzinę funkcji $f(x) = \sqrt{\log_{\frac{1}{[a]}}{([c]x+[d])}+[e]} + \sqrt{x^2-[b]x}$
\zadStop
\rozwStart{Małgorzata Ugowska}{}
Warunki do sprawdzenia:\\
1.) $\log_{\frac{1}{[a]}}{([c]x+[d])}+[e] \ge 0$\\
2.) $[c]x+[d] >0$\\
3.) $x^2-[b]x \ge 0$\\
Sprawdzamy, kiedy spełniony jest warunek 1.)
$$\log_{\frac{1}{[a]}}{([c]x+[d])}+[e] \ge 0$$
$$\log_{[a]}{([c]x+[d])} \le [e]$$
$$\log_{[a]}{([c]x+[d])} \le \log_{[a]}{[aa]}$$
$$[c]x+[d] \le [aa]$$
$$x \le \frac{[aa]-[d]}{[c]}$$
$$x \le [f]$$
Sprawdzamy, kiedy spełniony jest warunek 2.)
$$[c]x+[d] >0$$
$$ x >-[g]$$
Sprawdzamy, kiedy spełniony jest warunek 3.)
$$x^2-[b]x \ge 0$$
$$x(x-[b]) \ge 0$$
$$x \in (-\infty, 0] \cup [[b], \infty)$$
Bierzemy czę\'sć wspólną warunków i otrzymujemy:
$$D = (-[g],0] \cup [[b], [f]]$$
\rozwStop
\odpStart
$D =(-[g],0] \cup [[b], [f]]$
\odpStop
\testStart
A. $(-[b],0] \cup [1, [f]]$\\
B. $\emptyset $\\
C. $(-[g],0] \cup [[b], [f]]$\\
D. $(-[g],0] \cup [[b], [f])$\\
E. $(-[g],0) \cup ([b], [f])$\\
F. $(-[g], [f])$\\
G. $(-[b], [f]]$
\testStop
\kluczStart
C
\kluczStop



\end{document}