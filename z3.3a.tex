\documentclass[12pt, a4paper]{article}
\usepackage[utf8]{inputenc}
\usepackage{polski}

\usepackage{amsthm}  %pakiet do tworzenia twierdzeń itp.
\usepackage{amsmath} %pakiet do niektórych symboli matematycznych
\usepackage{amssymb} %pakiet do symboli mat., np. \nsubseteq
\usepackage{amsfonts}
\usepackage{graphicx} %obsługa plików graficznych z rozszerzeniem png, jpg
\theoremstyle{definition} %styl dla definicji
\newtheorem{zad}{} 
\title{Multizestaw zadań}
\author{Robert Fidytek}
%\date{\today}
\date{}
\newcounter{liczniksekcji}
\newcommand{\kategoria}[1]{\section{#1}} %olreślamy nazwę kateforii zadań
\newcommand{\zadStart}[1]{\begin{zad}#1\newline} %oznaczenie początku zadania
\newcommand{\zadStop}{\end{zad}}   %oznaczenie końca zadania
%Makra opcjonarne (nie muszą występować):
\newcommand{\rozwStart}[2]{\noindent \textbf{Rozwiązanie (autor #1 , recenzent #2): }\newline} %oznaczenie początku rozwiązania, opcjonarnie można wprowadzić informację o autorze rozwiązania zadania i recenzencie poprawności wykonania rozwiązania zadania
\newcommand{\rozwStop}{\newline}                                            %oznaczenie końca rozwiązania
\newcommand{\odpStart}{\noindent \textbf{Odpowiedź:}\newline}    %oznaczenie początku odpowiedzi końcowej (wypisanie wyniku)
\newcommand{\odpStop}{\newline}                                             %oznaczenie końca odpowiedzi końcowej (wypisanie wyniku)
\newcommand{\testStart}{\noindent \textbf{Test:}\newline} %ewentualne możliwe opcje odpowiedzi testowej: A. ? B. ? C. ? D. ? itd.
\newcommand{\testStop}{\newline} %koniec wprowadzania odpowiedzi testowych
\newcommand{\kluczStart}{\noindent \textbf{Test poprawna odpowiedź:}\newline} %klucz, poprawna odpowiedź pytania testowego (jedna literka): A lub B lub C lub D itd.
\newcommand{\kluczStop}{\newline} %koniec poprawnej odpowiedzi pytania testowego 
\newcommand{\wstawGrafike}[2]{\begin{figure}[h] \includegraphics[scale=#2] {#1} \end{figure}} %gdyby była potrzeba wstawienia obrazka, parametry: nazwa pliku, skala (jak nie wiesz co wpisać, to wpisz 1)

\begin{document}
\maketitle


\kategoria{Wikieł/Z3.3a}
\zadStart{Zadanie z Wikieł Z 3.3 a)  moja wersja nr [nrWersji]}
%[p1]:[3,4,5,6,7,8,9,10,11,12,13,14,15]
%[p2]=random.randint(2,10)
%[p3]=random.randint(2,100)
%[p4]=random.randint(2,10)
%[p5]=random.randint(2,10)
%[p6]=random.randint(1,100)
%[p0]:[1]
%[a]=[p1]-[p0]
%[b]=[p2]-[p0]
%[c]=[p4]-[p0]
%[d]=[p5]-[p0]
%[ab]=[a]-[b]
%[r]=round([p3]/([ab]+0.0000001),2)
%[cr]=round([c]*[r],2)
%[dr]=round([d]*[r],2)
%[cd]=round([cr]*[dr],2)
%[cdd]=round([cr]+[dr],2)
%[cdp3]=round([cd]-[p6],2)
%[del]=round([cdd]*[cdd]-4*[cdp3],2)
%[pdel]=round(math.sqrt([del]),2)
%[a1]=round((-[cdd]-[pdel])/2,2)
%[a2]=round((-[cdd]+[pdel])/2,2)
%[ab]!=0 and [del]>0 and [p1]!=[p2] and [p4]!=[p5]


Znaleźć wyraz pierwszy $a_{1}$ oraz różnicę $r$ ciągu arytmetycznego $(a_{n})$, w którym $a_{[p1]}-a_{[p2]}=[p3]$ oraz $a_{[p4]}\cdot a_{[p5]}=[p6].$
\zadStop
\rozwStart{Maja Szabłowska}{}
Z pierwszego równania otrzymujemy:
$$a_{[p1]}-a_{[p2]}=[p3] \Rightarrow a_{1}+[a]r-a_{1}-[b]r=[p3]$$
$$[ab]r=[p3] \Rightarrow r=[r]$$
Korzystamy z wyliczonej zmiennej:
$$a_{[p4]}\cdot a_{[p5]}=[p6] \Rightarrow (a_{1}+[c]r)(a_{1}+[d]r)=[p6] $$
$$(a_{1}+[cr])(a_{1}+[dr])=[p6] \Rightarrow a_{1}^{2}+[cr]a_{1}+[dr]a_{1}+[cd]=[p6] $$

$$a_{1}^{2}+[cdd]a_{1}+[cdp3]=0 $$
$$\Delta=[cdd]^{2}-4\cdot1\cdot([cdp3]) \Rightarrow \Delta=[del] \Rightarrow \sqrt{\Delta}=[pdel]$$
$$a_{1}=\frac{-([cdd])-[pdel]}{2}=[a1]$$ $$\lor$$ $$ a_{1}=\frac{-([cdd])+[pdel]}{2}=[a2] $$
\rozwStop
\odpStart
$r=[r], a_{1}=[a1] \lor a_{1}=[a2]$
\odpStop
\testStart
A.$r=[r], a_{1}=[a1] \lor a_{1}=[a2]$
B.$r=[r], a_{1}=[dr] \lor a_{1}=[a2]$
C.$r=[r], a_{1}=[a1] \lor a_{1}=[cr]$
D.$r=[r], a_{1}=[p6] \lor a_{1}=[a2]$
E.$r=[p3], a_{1}=[a1] \lor a_{1}=[a2]$
F.$r=[cdd], a_{1}=[a1] \lor a_{1}=[a2]$
G.$r=[r], a_{1}=[a1] \lor a_{1}=[del]$
H.$r=[pdel], a_{1}=[a1] \lor a_{1}=[r]$
I.$r=[r], a_{1}=[p5] \lor a_{1}=[a2]$
\testStop
\kluczStart
A
\kluczStop



\end{document}