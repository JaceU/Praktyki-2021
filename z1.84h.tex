\documentclass[12pt, a4paper]{article}
\usepackage[utf8]{inputenc}
\usepackage{polski}
\usepackage{amsthm}  %pakiet do tworzenia twierdzeń itp.
\usepackage{amsmath} %pakiet do niektórych symboli matematycznych
\usepackage{amssymb} %pakiet do symboli mat., np. \nsubseteq
\usepackage{amsfonts}
\usepackage{graphicx} %obsługa plików graficznych z rozszerzeniem png, jpg
\theoremstyle{definition} %styl dla definicji
\newtheorem{zad}{} 
\title{Multizestaw zadań}
\author{Radosław Grzyb}
%\date{\today}
\date{}
\newcounter{liczniksekcji}
\newcommand{\kategoria}[1]{\section{#1}} %olreślamy nazwę kateforii zadań
\newcommand{\zadStart}[1]{\begin{zad}#1\newline} %oznaczenie początku zadania
\newcommand{\zadStop}{\end{zad}}   %oznaczenie końca zadania
%Makra opcjonarne (nie muszą występować):
\newcommand{\rozwStart}[2]{\noindent \textbf{Rozwiązanie (autor #1 , recenzent #2): }\newline} %oznaczenie początku rozwiązania, opcjonarnie można wprowadzić informację o autorze rozwiązania zadania i recenzencie poprawności wykonania rozwiązania zadania
\newcommand{\rozwStop}{\newline}                                            %oznaczenie końca rozwiązania
\newcommand{\odpStart}{\noindent \textbf{Odpowiedź:}\newline}    %oznaczenie początku odpowiedzi końcowej (wypisanie wyniku)
\newcommand{\odpStop}{\newline}                                             %oznaczenie końca odpowiedzi końcowej (wypisanie wyniku)
\newcommand{\testStart}{\noindent \textbf{Test:}\newline} %ewentualne możliwe opcje odpowiedzi testowej: A. ? B. ? C. ? D. ? itd.
\newcommand{\testStop}{\newline} %koniec wprowadzania odpowiedzi testowych
\newcommand{\kluczStart}{\noindent \textbf{Test poprawna odpowiedź:}\newline} %klucz, poprawna odpowiedź pytania testowego (jedna literka): A lub B lub C lub D itd.
\newcommand{\kluczStop}{\newline} %koniec poprawnej odpowiedzi pytania testowego 
\newcommand{\wstawGrafike}[2]{\begin{figure}[h] \includegraphics[scale=#2] {#1} \end{figure}} %gdyby była potrzeba wstawienia obrazka, parametry: nazwa pliku, skala (jak nie wiesz co wpisać, to wpisz 1)
\begin{document}
\maketitle
\kategoria{Wikieł/Z1.84h}
\zadStart{Zadanie z Wikieł Z 1.84h moja wersja nr [nrWersji]}
%[p1]:[2,3,4,5,6,7,9,10]
%[p2]:[2,4,5,8,10,16]
%[c1]=[p1]**2
%[c2]=[p1]**3
%[c3]=[p2]/2
%[wynik]=1/[p2]
%[zlywynik1]=1-[wynik]
%[zlywynik2]=-[wynik]+4
%[zlywynik3]=2*[wynik]-5
Rozwiązać równanie:
$$\sqrt[3]{[p1]}\cdot\left(\frac{1}{[c2]}\right)^{\frac{1-[p2]x}{3}}+[c1]^{[c3]x}=[p1]+\sqrt[3]{[p1]}$$
\zadStop
\rozwStart{Radosław Grzyb}{}
$$\sqrt[3]{[p1]}\cdot[p1]^{[p2]x-1}+[p1]^{[p2]x}=[p1]+\sqrt[3]{[p1]}$$
$$\frac{\sqrt[3]{[p1]}}{[p1]}\cdot[p1]^{[p2]x}+[p1]^{[p2]x}=[p1]+\sqrt[3]{[p1]}$$
$$(\frac{\sqrt[3]{[p1]}}{[p1]}+1)\cdot[p1]^{[p2]x}=[p1]+\sqrt[3]{[p1]}$$
Dzielimy obie strony równania przez $(\frac{\sqrt[3]{[p1]}}{[p1]}+1)$ i otrzymujemy:
$$[p1]^{[p2]x}=[p1]^{1}$$
Logarytmując obustronnie otrzymujemy finalny wynik:
$$[p2]x=1\implies x=[wynik]$$
\rozwStop
\odpStart
$[wynik]$
\odpStop
\testStart
A.$[zlywynik1]$
B.$[wynik]$
C.$[zlywynik2]$
D.$[zlywynik3]$
\testStop
\kluczStart
B
\kluczStop
\end{document}