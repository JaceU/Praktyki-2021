\documentclass[12pt, a4paper]{article}
\usepackage[utf8]{inputenc}
\usepackage{polski}

\usepackage{amsthm}  %pakiet do tworzenia twierdzeń itp.
\usepackage{amsmath} %pakiet do niektórych symboli matematycznych
\usepackage{amssymb} %pakiet do symboli mat., np. \nsubseteq
\usepackage{amsfonts}
\usepackage{graphicx} %obsługa plików graficznych z rozszerzeniem png, jpg
\theoremstyle{definition} %styl dla definicji
\newtheorem{zad}{} 
\title{Multizestaw zadań}
\author{Robert Fidytek}
%\date{\today}
\date{}\documentclass[12pt, a4paper]{article}
\usepackage[utf8]{inputenc}
\usepackage{polski}

\usepackage{amsthm}  %pakiet do tworzenia twierdzeń itp.
\usepackage{amsmath} %pakiet do niektórych symboli matematycznych
\usepackage{amssymb} %pakiet do symboli mat., np. \nsubseteq
\usepackage{amsfonts}
\usepackage{graphicx} %obsługa plików graficznych z rozszerzeniem png, jpg
\theoremstyle{definition} %styl dla definicji
\newtheorem{zad}{} 
\title{Multizestaw zadań}
\author{Robert Fidytek}
%\date{\today}
\date{}
\newcounter{liczniksekcji}
\newcommand{\kategoria}[1]{\section{#1}} %olreślamy nazwę kateforii zadań
\newcommand{\zadStart}[1]{\begin{zad}#1\newline} %oznaczenie początku zadania
\newcommand{\zadStop}{\end{zad}}   %oznaczenie końca zadania
%Makra opcjonarne (nie muszą występować):
\newcommand{\rozwStart}[2]{\noindent \textbf{Rozwiązanie (autor #1 , recenzent #2): }\newline} %oznaczenie początku rozwiązania, opcjonarnie można wprowadzić informację o autorze rozwiązania zadania i recenzencie poprawności wykonania rozwiązania zadania
\newcommand{\rozwStop}{\newline}                                            %oznaczenie końca rozwiązania
\newcommand{\odpStart}{\noindent \textbf{Odpowiedź:}\newline}    %oznaczenie początku odpowiedzi końcowej (wypisanie wyniku)
\newcommand{\odpStop}{\newline}                                             %oznaczenie końca odpowiedzi końcowej (wypisanie wyniku)
\newcommand{\testStart}{\noindent \textbf{Test:}\newline} %ewentualne możliwe opcje odpowiedzi testowej: A. ? B. ? C. ? D. ? itd.
\newcommand{\testStop}{\newline} %koniec wprowadzania odpowiedzi testowych
\newcommand{\kluczStart}{\noindent \textbf{Test poprawna odpowiedź:}\newline} %klucz, poprawna odpowiedź pytania testowego (jedna literka): A lub B lub C lub D itd.
\newcommand{\kluczStop}{\newline} %koniec poprawnej odpowiedzi pytania testowego 
\newcommand{\wstawGrafike}[2]{\begin{figure}[h] \includegraphics[scale=#2] {#1} \end{figure}} %gdyby była potrzeba wstawienia obrazka, parametry: nazwa pliku, skala (jak nie wiesz co wpisać, to wpisz 1)

\begin{document}
\maketitle


\kategoria{Wikieł/Z1.129s}
\zadStart{Zadanie z Wikieł Z 1.129 s) moja wersja nr [nrWersji]}
%[p1]:[2,3,4,5,6,7,8,9,10]
%[p2]=random.randint(2,10)
%[p3]:[2,3,4,5,6,7,8,9,10]
%[p4]:[2,3,4,5,6,7,8,9,10]
%[p5]=random.randint(2,10)
%[p5p4]=[p5]+[p4]
%[p5p42]=-[p5]+[p4]
%[a]=-[p2]+[p1]
%[b]=[p2]+[p1]
%[pp3]=round(math.sqrt([p3]),2)
%[p5p42]<[p5p4] and [pp3]<[a]

Wyznaczyć dziedzinę naturalną funkcji.
$$f(x)=\frac{\arcsin\frac{x-[p1]}{[p2]}-\log([p3]-x^{2})}{\sqrt{|x-[p4]|-[p5]}}$$
\zadStop

\rozwStart{Maja Szabłowska}{}
$$\sqrt{|x-[p4]|-[p5]}\neq 0 \quad \land \quad |x-[p4]|-[p5]\geq0$$
$$|x-[p4]|-[p5]>0$$
$$|x-[p4]|>[p5]$$
$$x-[p4]>[p5]\quad \lor  \quad x-[p4]<-[p5]$$
$$x>[p5p4] \quad \lor  \quad x<[p5p42]$$
$$x\in (-\infty, [p5p42])\cup([p5p4],\infty)$$

$$-1\leq\frac{x-[p1]}{[p2]}\leq 1$$
$$-[p2]\leq x-[p1]\leq [p2]$$
$$[a]\leq x \leq [b]$$
$$x\in [[a],[b]]$$

$$[p3]-x^{2} > 0$$
$$([pp3]-x)([pp3]+x)>0$$
$$x\in(-[pp3],[pp3])$$

Ostatecznie: $x\in\emptyset$
\rozwStop
\odpStart
$x\in\emptyset$
\odpStop
\testStart
A.$x\in\emptyset$
B.$x\in[e^{[p2]},\infty)$
C.$x\in(-\infty, 0)$
D.$x\in(-\infty, -[p2]] \cup [\ln[p1],\infty)$
E.$x\in[[p1],\infty)$
F.$x\in([p2],\infty)$
G.$x\in\mathbb{R}$

\testStop
\kluczStart
A
\kluczStop



\end{document}
