\documentclass[12pt, a4paper]{article}
\usepackage[utf8]{inputenc}
\usepackage{polski}

\usepackage{amsthm}  %pakiet do tworzenia twierdzeń itp.
\usepackage{amsmath} %pakiet do niektórych symboli matematycznych
\usepackage{amssymb} %pakiet do symboli mat., np. \nsubseteq
\usepackage{amsfonts}
\usepackage{graphicx} %obsługa plików graficznych z rozszerzeniem png, jpg
\theoremstyle{definition} %styl dla definicji
\newtheorem{zad}{} 
\title{Multizestaw zadań}
\author{Robert Fidytek}
%\date{\today}
\date{}
\newcounter{liczniksekcji}
\newcommand{\kategoria}[1]{\section{#1}} %olreślamy nazwę kateforii zadań
\newcommand{\zadStart}[1]{\begin{zad}#1\newline} %oznaczenie początku zadania
\newcommand{\zadStop}{\end{zad}}   %oznaczenie końca zadania
%Makra opcjonarne (nie muszą występować):
\newcommand{\rozwStart}[2]{\noindent \textbf{Rozwiązanie (autor #1 , recenzent #2): }\newline} %oznaczenie początku rozwiązania, opcjonarnie można wprowadzić informację o autorze rozwiązania zadania i recenzencie poprawności wykonania rozwiązania zadania
\newcommand{\rozwStop}{\newline}                                            %oznaczenie końca rozwiązania
\newcommand{\odpStart}{\noindent \textbf{Odpowiedź:}\newline}    %oznaczenie początku odpowiedzi końcowej (wypisanie wyniku)
\newcommand{\odpStop}{\newline}                                             %oznaczenie końca odpowiedzi końcowej (wypisanie wyniku)
\newcommand{\testStart}{\noindent \textbf{Test:}\newline} %ewentualne możliwe opcje odpowiedzi testowej: A. ? B. ? C. ? D. ? itd.
\newcommand{\testStop}{\newline} %koniec wprowadzania odpowiedzi testowych
\newcommand{\kluczStart}{\noindent \textbf{Test poprawna odpowiedź:}\newline} %klucz, poprawna odpowiedź pytania testowego (jedna literka): A lub B lub C lub D itd.
\newcommand{\kluczStop}{\newline} %koniec poprawnej odpowiedzi pytania testowego 
\newcommand{\wstawGrafike}[2]{\begin{figure}[h] \includegraphics[scale=#2] {#1} \end{figure}} %gdyby była potrzeba wstawienia obrazka, parametry: nazwa pliku, skala (jak nie wiesz co wpisać, to wpisz 1)

\begin{document}
\maketitle


\kategoria{Wikieł/Z3.34b}
\zadStart{Zadanie z Wikieł Z 3.34 b) moja wersja nr [nrWersji]}
%[a]:[1,2,3,4,5,6,7,8,9,10,11,12,13,14,15,16,17,18,19,20,21,22,23,24,25,26,27,28]
Obliczyć granicę ciągu $$a_n=sin\sqrt{n+[a]}-sin\sqrt{n}.$$
\zadStop
\rozwStart{Aleksandra Pasińska}{}
$$\lim_{n\rightarrow \infty}(sin\sqrt{n+[a]}-sin\sqrt{n})$$\\
Korzystamy z wzoru na różnice funkcji trygonometrycznych:
$$sin\alpha-sin\beta=2cos\bigg(\frac{\alpha+\beta}{2}\bigg)sin\bigg(\frac{\alpha-\beta}{2}\bigg)$$
$$\lim_{n\rightarrow \infty}\sqrt{n+[a]}=\sqrt{\infty+[a]}=\sqrt{\infty}$$
$$\lim_{n\rightarrow \infty}\sqrt{n}=\sqrt{\infty}$$
$$\lim_{n\rightarrow \infty}2cos\bigg(\frac{\sqrt{n+[a]}+\sqrt{n}}{2}\bigg)sin\bigg(\frac{\sqrt{n+[a]}-\sqrt{n}}{2}\bigg)=$$
$$=2cos\bigg(\frac{\sqrt{\infty}+\sqrt{\infty}}{2}\bigg)sin\bigg(\frac{\sqrt{\infty}-\sqrt{\infty}}{2}\bigg)=2cos\bigg(\frac{\sqrt{\infty}+\sqrt{\infty}}{2}\bigg)sin\bigg(\frac{0}{2}\bigg)=$$
$$=2cos\bigg(\frac{\sqrt{\infty}+\sqrt{\infty}}{2}\bigg)sin(0)=2cos\bigg(\frac{\sqrt{\infty}+\sqrt{\infty}}{2}\bigg)\cdot 0=0$$
\rozwStop
\odpStart
$0$\\
\odpStop
\testStart
A.$0$
B.$\infty$
C.$-\infty$
D.$1$
E.$9$
F.$e$
G.$4$
H.$7$
I.$-7$
\testStop
\kluczStart
A
\kluczStop



\end{document}