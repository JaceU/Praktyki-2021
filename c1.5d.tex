\documentclass[12pt, a4paper]{article}
\usepackage[utf8]{inputenc}
\usepackage{polski}
\usepackage{amsthm}  %pakiet do tworzenia twierdzeń itp.
\usepackage{amsmath} %pakiet do niektórych symboli matematycznych
\usepackage{amssymb} %pakiet do symboli mat., np. \nsubseteq
\usepackage{amsfonts}
\usepackage{graphicx} %obsługa plików graficznych z rozszerzeniem png, jpg
\theoremstyle{definition} %styl dla definicji
\newtheorem{zad}{} 
\title{Multizestaw zadań}
\author{Radosław Grzyb}
%\date{\today}
\date{}
\newcounter{liczniksekcji}
\newcommand{\kategoria}[1]{\section{#1}} %olreślamy nazwę kateforii zadań
\newcommand{\zadStart}[1]{\begin{zad}#1\newline} %oznaczenie początku zadania
\newcommand{\zadStop}{\end{zad}}   %oznaczenie końca zadania
%Makra opcjonarne (nie muszą występować):
\newcommand{\rozwStart}[2]{\noindent \textbf{Rozwiązanie (autor #1 , recenzent #2): }\newline} %oznaczenie początku rozwiązania, opcjonarnie można wprowadzić informację o autorze rozwiązania zadania i recenzencie poprawności wykonania rozwiązania zadania
\newcommand{\rozwStop}{\newline}                                            %oznaczenie końca rozwiązania
\newcommand{\odpStart}{\noindent \textbf{Odpowiedź:}\newline}    %oznaczenie początku odpowiedzi końcowej (wypisanie wyniku)
\newcommand{\odpStop}{\newline}                                             %oznaczenie końca odpowiedzi końcowej (wypisanie wyniku)
\newcommand{\testStart}{\noindent \textbf{Test:}\newline} %ewentualne możliwe opcje odpowiedzi testowej: A. ? B. ? C. ? D. ? itd.
\newcommand{\testStop}{\newline} %koniec wprowadzania odpowiedzi testowych
\newcommand{\kluczStart}{\noindent \textbf{Test poprawna odpowiedź:}\newline} %klucz, poprawna odpowiedź pytania testowego (jedna literka): A lub B lub C lub D itd.
\newcommand{\kluczStop}{\newline} %koniec poprawnej odpowiedzi pytania testowego 
\newcommand{\wstawGrafike}[2]{\begin{figure}[h] \includegraphics[scale=#2] {#1} \end{figure}} %gdyby była potrzeba wstawienia obrazka, parametry: nazwa pliku, skala (jak nie wiesz co wpisać, to wpisz 1)
\begin{document}
\maketitle
\kategoria{Beger/c1.5d}
\zadStart{Zadanie z Beger C 1.5d moja wersja nr [nrWersji]}
%[p2]:[2,3,4,5,6,7,8,9,10,11,12]
Obliczyć, całkując przez podstawienie.
$$\int \frac{1}{\sqrt{1-x^2}\arccos([p2]x)} \,dx$$
\zadStop
\rozwStart{Radosław Grzyb}{}
Podstawiamy:
$$u=\arccos([p2]x) \implies du=-\frac{[p2]}{\sqrt{1-x^2}}dx \implies dx=-\frac{\sqrt{1-x^2}}{[p2]}du$$
A więc:
$$\int \frac{1}{\sqrt{1-x^2}\arccos([p2]x)}\cdot\left(-\frac{\sqrt{1-x^2}}{[p2]}\right) \,du=-\frac{1}{[p2]}\int \frac{1}{u} \,du$$
Otrzymaliśmy prostą do policzenia całkę, a zarazem finalny wynik:
$$-\frac{1}{[p2]}\int \frac{1}{u} \,du=-\frac{1}{[p2]}\ln|u|+C=-\frac{1}{[p2]}\ln|\arccos([p2]x)|+C$$
\rozwStop
\odpStart
$$-\frac{1}{[p2]}\ln|\arccos([p2]x)|+C$$
\odpStop
\testStart
A.$$-\frac{1}{[p2]\arctan^{[p2]}(x)}+C$$
B.$$\arctan(\frac{x}{\sqrt{[p2]}})+C$$
C.$$-\frac{1}{[p2]}\ln|\arccos([p2]x)|+C$$
D.$$\frac{1}{3}\ln|\frac{x}{[p2]}|+C$$
\testStop
\kluczStart
C
\kluczStop
\end{document}