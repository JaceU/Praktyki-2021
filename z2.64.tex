\documentclass[12pt, a4paper]{article}
\usepackage[utf8]{inputenc}
\usepackage{polski}

\usepackage{amsthm}  %pakiet do tworzenia twierdzeń itp.
\usepackage{amsmath} %pakiet do niektórych symboli matematycznych
\usepackage{amssymb} %pakiet do symboli mat., np. \nsubseteq
\usepackage{amsfonts}
\usepackage{graphicx} %obsługa plików graficznych z rozszerzeniem png, jpg
\theoremstyle{definition} %styl dla definicji
\newtheorem{zad}{} 
\title{Multizestaw zadań}
\author{Robert Fidytek}
%\date{\today}
\date{}
\newcounter{liczniksekcji}
\newcommand{\kategoria}[1]{\section{#1}} %olreślamy nazwę kateforii zadań
\newcommand{\zadStart}[1]{\begin{zad}#1\newline} %oznaczenie początku zadania
\newcommand{\zadStop}{\end{zad}}   %oznaczenie końca zadania
%Makra opcjonarne (nie muszą występować):
\newcommand{\rozwStart}[2]{\noindent \textbf{Rozwiązanie (autor #1 , recenzent #2): }\newline} %oznaczenie początku rozwiązania, opcjonarnie można wprowadzić informację o autorze rozwiązania zadania i recenzencie poprawności wykonania rozwiązania zadania
\newcommand{\rozwStop}{\newline}                                            %oznaczenie końca rozwiązania
\newcommand{\odpStart}{\noindent \textbf{Odpowiedź:}\newline}    %oznaczenie początku odpowiedzi końcowej (wypisanie wyniku)
\newcommand{\odpStop}{\newline}                                             %oznaczenie końca odpowiedzi końcowej (wypisanie wyniku)
\newcommand{\testStart}{\noindent \textbf{Test:}\newline} %ewentualne możliwe opcje odpowiedzi testowej: A. ? B. ? C. ? D. ? itd.
\newcommand{\testStop}{\newline} %koniec wprowadzania odpowiedzi testowych
\newcommand{\kluczStart}{\noindent \textbf{Test poprawna odpowiedź:}\newline} %klucz, poprawna odpowiedź pytania testowego (jedna literka): A lub B lub C lub D itd.
\newcommand{\kluczStop}{\newline} %koniec poprawnej odpowiedzi pytania testowego 
\newcommand{\wstawGrafike}[2]{\begin{figure}[h] \includegraphics[scale=#2] {#1} \end{figure}} %gdyby była potrzeba wstawienia obrazka, parametry: nazwa pliku, skala (jak nie wiesz co wpisać, to wpisz 1)

\begin{document}
\maketitle


\kategoria{Wikieł/Z2.64}
\zadStart{Zadanie z Wikieł Z 2.64 moja wersja nr [nrWersji]}
%[a1]:[1,2,3,4,5,6,7,8,9,10,11,12,13,14,15,16,17,18,19,20,21,22,23,24,25,26,27,28,29,30]
%[a2]=1
%[b1]:[1,2,3,4,5,6,7,8,9,10,11,12,13,14,15,16,17,18,19,20,21,22,23,24,25,26,27,28,29,30]
%[b2]=[a2]*2
%[aa1]=[a1]*[a1]
%[aa2]=[a2]*[a2]
%[bb1]=[b1]*[b1]
%[bb2]=[b2]*[b2]
%[4aa1]=[aa1]*4
%[a]=[bb1]-[4aa1]
%[a3]=3*[aa1]
%[d]=[a]+[a3]
%[aae2]=[aa2]*[a]
%[a]>0
Napisać równanie hiperboli, której osiami symetrii są osie układu, mając dane współrzędne dwóch punktów A([a1],[a2]), B(-[b1],-[b2]) należących do tej hiperboli.
\zadStop
\rozwStart{Aleksandra Pasińska}{}
$$-\frac{x^2}{a^2}+\frac{y^2}{b^2}=1$$
$$\left\{ \begin{array}{ll}
-\frac{[aa1]}{a^2}+\frac{[aa2]}{b^2}=1/\cdot4\\
-\frac{[bb1]}{a^2}+\frac{[bb2]}{b^2}=1
\end{array} \right.$$
$$\left\{ \begin{array}{ll}
-\frac{[4aa1]}{a^2}+\frac{[bb2]}{b^2}=4\\
-\frac{[bb1]}{a^2}+\frac{[bb2]}{b^2}=1
\end{array} \right.$$
$$-\frac{[4aa1]}{a^2}+\frac{[bb2]}{b^2}+\frac{[bb1]}{a^2}-\frac{[bb2]}{b^2}=3$$
$$\frac{[a]}{a^2}=3,a^2=\frac{[a]}{3}$$
$$-\frac{[a3]}{[a]}+\frac{[aa2]}{b^2}=1$$
$$\frac{[aa2]}{b^2}=\frac{[d]}{[a]}$$
$$b^2=\frac{[aae2]}{[d]}$$
$$-\frac{3}{[a]}x^2+\frac{[d]}{[aae2]}y^2=1$$
$$-3x^2+[d]y^2=[aae2]$$
\rozwStop
\odpStart
$-3x^2+[d]y^2=[aae2]$\\
\odpStop
\testStart
A.$-3x^2+[d]y^2=[aae2]$
B.$ 3x^2+[d]y^2=[aae2]$
C.$-3x^2+y^2=[aae2]$
D.$x^2+[d]y^2=[aae2]$
E.$-x^2+[d]y^2=[aae2]$
F.$-3x^2+[d]y=[aae2]$
G.$-3x+[d]y^2=[aae2]$
H.$-3x^2+[d]y^3=[aae2]$
I.$-3x^3+[d]y^2=[aae2]$
\testStop
\kluczStart
A
\kluczStop



\end{document}