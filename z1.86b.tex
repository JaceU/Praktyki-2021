\documentclass[12pt, a4paper]{article}
\usepackage[utf8]{inputenc}
\usepackage{polski}

\usepackage{amsthm}  %pakiet do tworzenia twierdzeń itp.
\usepackage{amsmath} %pakiet do niektórych symboli matematycznych
\usepackage{amssymb} %pakiet do symboli mat., np. \nsubseteq
\usepackage{amsfonts}
\usepackage{graphicx} %obsługa plików graficznych z rozszerzeniem png, jpg
\theoremstyle{definition} %styl dla definicji
\newtheorem{zad}{} 
\title{Multizestaw zadań}
\author{Jacek Jabłoński}
%\date{\today}
\date{}
\newcounter{liczniksekcji}
\newcommand{\kategoria}[1]{\section{#1}} %olreślamy nazwę kateforii zadań
\newcommand{\zadStart}[1]{\begin{zad}#1\newline} %oznaczenie początku zadania
\newcommand{\zadStop}{\end{zad}}   %oznaczenie końca zadania
%Makra opcjonarne (nie muszą występować):
\newcommand{\rozwStart}[2]{\noindent \textbf{Rozwiązanie (autor #1 , recenzent #2): }\newline} %oznaczenie początku rozwiązania, opcjonarnie można wprowadzić informację o autorze rozwiązania zadania i recenzencie poprawności wykonania rozwiązania zadania
\newcommand{\rozwStop}{\newline}                                            %oznaczenie końca rozwiązania
\newcommand{\odpStart}{\noindent \textbf{Odpowiedź:}\newline}    %oznaczenie początku odpowiedzi końcowej (wypisanie wyniku)
\newcommand{\odpStop}{\newline}                                             %oznaczenie końca odpowiedzi końcowej (wypisanie wyniku)
\newcommand{\testStart}{\noindent \textbf{Test:}\newline} %ewentualne możliwe opcje odpowiedzi testowej: A. ? B. ? C. ? D. ? itd.
\newcommand{\testStop}{\newline} %koniec wprowadzania odpowiedzi testowych
\newcommand{\kluczStart}{\noindent \textbf{Test poprawna odpowiedź:}\newline} %klucz, poprawna odpowiedź pytania testowego (jedna literka): A lub B lub C lub D itd.
\newcommand{\kluczStop}{\newline} %koniec poprawnej odpowiedzi pytania testowego 
\newcommand{\wstawGrafike}[2]{\begin{figure}[h] \includegraphics[scale=#2] {#1} \end{figure}} %gdyby była potrzeba wstawienia obrazka, parametry: nazwa pliku, skala (jak nie wiesz co wpisać, to wpisz 1)

\begin{document}
\maketitle


\kategoria{Wikieł/z1.86b}
\zadStart{Zadanie z Wikieł z1.86b) moja wersja nr [nrWersji]}
%[p1]:[2,3,4,5,6]
%[p2]:[2,3,4,5,6]
%[p3]:[2,3,4,5,6]
%[p4]:[2,3,4,5,6]
%[r1]=[p3]+[p4]
%[delta]=[r1]*[r1]
%[pdelta]=int(math.pow([delta],(1/2)))
%[r2]=int(([r1]-[pdelta])/(2*[r1]))
%[r3]=int(([r1]+[pdelta])/(2*[r1]))
%[f1a]=[r2]-1
%[f2a]=[r2]-2
%[f3a]=[r2]-3
%[f1b]=[r3]+1
%[f2b]=[r3]+2
%[f3b]=[r3]+3
%[p1]!=[p2] and [p3]!=[p4] and [delta]>0 and [p1]<[p2]
Rozwiązać nierówność:
b) $(\frac{[p1]}{[p2]})^{[p3]x^2-[p4]x} \geq (\frac{[p2]}{[p1]})^{[p4]x^2-[p3]x} $
\zadStop
\rozwStart{Jacek Jabłoński}{}
$$(\frac{[p1]}{[p2]})^{[p3]x^2-[p4]x} \geq (\frac{[p2]}{[p1]})^{[p4]x^2-[p3]x}$$
$$(\frac{[p1]}{[p2]})^{[p3]x^2-[p4]x} \geq (\frac{[p1]}{[p2]})^{[p3]x-[p4]x^2}$$
$$[p3]x^2-[p4]x \leq [p3]x-[p4]x^2 $$
$$[r1]x^2 - [r1]x \leq 0 $$
$$\Delta=[delta]$$
$$\sqrt{\Delta} = [pdelta]$$
$$t_1=\frac{[r1]-[pdelta]}{2 \cdot [r1]} = [r2]$$
$$t_2=\frac{[r1]+[pdelta]}{2 \cdot [r1]} = [r3]$$ 
$$x \in <[r2] ; [r3]>$$
\rozwStop
\odpStart
$$x \in <[r2] ; [r3]>$$
\odpStop
\testStart
A. $$x \in <[r2] ; [r3]>$$
B. $$x \in <[f1a] ; [r3]>$$
C. $$x \in <[r2] ; [f1b]>$$
D. $$x \in <[f2a] ; [r3]>$$
E. $$x \in <[r2] ; [f2b]>$$
F. $$x \in <[f3a] ; [r3]>$$
G. $$x \in <[r2] ; [f3b]>$$
H. $$x \geq [r2]$$
I.$$x \leq [r3]$$
\testStop
\kluczStart
A
\kluczStop



\end{document}