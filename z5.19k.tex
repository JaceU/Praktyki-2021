\documentclass[12pt, a4paper]{article}
\usepackage[utf8]{inputenc}
\usepackage{polski}

\usepackage{amsthm}  %pakiet do tworzenia twierdzeń itp.
\usepackage{amsmath} %pakiet do niektórych symboli matematycznych
\usepackage{amssymb} %pakiet do symboli mat., np. \nsubseteq
\usepackage{amsfonts}
\usepackage{graphicx} %obsługa plików graficznych z rozszerzeniem png, jpg
\theoremstyle{definition} %styl dla definicji
\newtheorem{zad}{} 
\title{Multizestaw zadań}
\author{Robert Fidytek}
%\date{\today}
\date{}
\newcounter{liczniksekcji}
\newcommand{\kategoria}[1]{\section{#1}} %olreślamy nazwę kateforii zadań
\newcommand{\zadStart}[1]{\begin{zad}#1\newline} %oznaczenie początku zadania
\newcommand{\zadStop}{\end{zad}}   %oznaczenie końca zadania
%Makra opcjonarne (nie muszą występować):
\newcommand{\rozwStart}[2]{\noindent \textbf{Rozwiązanie (autor #1 , recenzent #2): }\newline} %oznaczenie początku rozwiązania, opcjonarnie można wprowadzić informację o autorze rozwiązania zadania i recenzencie poprawności wykonania rozwiązania zadania
\newcommand{\rozwStop}{\newline}                                            %oznaczenie końca rozwiązania
\newcommand{\odpStart}{\noindent \textbf{Odpowiedź:}\newline}    %oznaczenie początku odpowiedzi końcowej (wypisanie wyniku)
\newcommand{\odpStop}{\newline}                                             %oznaczenie końca odpowiedzi końcowej (wypisanie wyniku)
\newcommand{\testStart}{\noindent \textbf{Test:}\newline} %ewentualne możliwe opcje odpowiedzi testowej: A. ? B. ? C. ? D. ? itd.
\newcommand{\testStop}{\newline} %koniec wprowadzania odpowiedzi testowych
\newcommand{\kluczStart}{\noindent \textbf{Test poprawna odpowiedź:}\newline} %klucz, poprawna odpowiedź pytania testowego (jedna literka): A lub B lub C lub D itd.
\newcommand{\kluczStop}{\newline} %koniec poprawnej odpowiedzi pytania testowego 
\newcommand{\wstawGrafike}[2]{\begin{figure}[h] \includegraphics[scale=#2] {#1} \end{figure}} %gdyby była potrzeba wstawienia obrazka, parametry: nazwa pliku, skala (jak nie wiesz co wpisać, to wpisz 1)

\begin{document}
\maketitle


\kategoria{Wikieł/Z5.19 k}
\zadStart{Zadanie z Wikieł Z 5.19 k) moja wersja nr [nrWersji]}
%[a]:[2,3,4,5,6,7,8,9]
%[b]:[2,3,4,5,6,7,8,9]
%[a]!=0
Oblicz granicę $\lim_{x \rightarrow 0} \left( \frac{[a]}{x} \right)^{\tg([b]x)}$.
\zadStop
\rozwStart{Joanna Świerzbin}{}
$$\lim_{x \rightarrow 0} \left( \frac{[a]}{x} \right)^{\tg([b]x)} = \lim_{x \rightarrow 0} e^{\ln\left(\left( \frac{[a]}{x} \right)^{\tg([b]x)}\right)}
= \lim_{x \rightarrow 0} e^{\tg([b]x)\ln\left( \frac{[a]}{x}\right)}= $$
$$= e^{ \lim_{x \rightarrow 0}\tg([b]x)\ln\left( \frac{[a]}{x}\right)} $$
Obliczmy  $\lim_{x \rightarrow 0}\tg([b]x)\ln\left( \frac{[a]}{x}\right)$
$$\lim_{x \rightarrow 0}\tg([b]x)\ln\left( \frac{[a]}{x}\right) = \lim_{x \rightarrow 0}\frac{\ln\left( \frac{[a]}{x}\right)}{\frac{1}{\tg([b]x)}}$$
Otrzymujemy $ \left[ \frac{\infty}{-\infty} \right] $ więc możemy skorzystać z twierdzenia de l'Hospitala.
$$\lim_{x \rightarrow 0}\frac{\left(\ln\left( \frac{[a]}{x}\right)\right)'}{\left(\frac{1}{\tg([b]x)}\right)'} = 
\lim_{x \rightarrow 0}\frac{\frac{x}{[a]}\left( \frac{[a]}{x}\right)'}{\frac{-1}{\tg^2(x)}\left(\tg([b]x)\right)'} = 
\lim_{x \rightarrow 0}\frac{\frac{-[a]x}{[a]x^2}}{\frac{-[b]}{\tg^2([b]x) \cos^2([b]x)}} =$$
$$ = \lim_{x \rightarrow 0}\frac{\frac{1}{x}}{\frac{[b]}{\sin^2([b]x)}} =
\lim_{x \rightarrow 0}\frac{\sin^2([b]x)}{[b]x}= \frac{1}{[b]}\lim_{x \rightarrow 0}\frac{\sin^2([b]x)}{x} $$
Otrzymujemy $ \left[ \frac{0}{0} \right] $ więc możemy skorzystać z twierdzenia de l'Hospitala.
$$\frac{1}{[b]}\lim_{x \rightarrow 0}\frac{\left(\sin^2([b]x)\right)'}{\left(x\right)'}  = \frac{1}{[b]}\lim_{x \rightarrow 0}\frac{2\sin([b]x)(\sin([b]x))'}{1} = \frac{2}{[b]}\lim_{x \rightarrow 0}[b]\sin([b]x)\cos([b]x)=$$
$$ = 2 \lim_{x \rightarrow 0}\sin([b]x) \lim_{x \rightarrow 0} \cos([b]x)= 0$$
Podstawmy do początkowego przykładu.
$$e^{ \lim_{x \rightarrow 0}\tg([b]x)\ln\left( \frac{[a]}{x}\right)}=e^0=1$$
\rozwStop
\odpStart
$ \lim_{x \rightarrow 0} \left( \frac{[a]}{x} \right)^{\tg([b]x)} =1 $
\odpStop
\testStart
A. $ \lim_{x \rightarrow 0} \left( \frac{[a]}{x} \right)^{\tg([b]x)} =1 $\\
B. $ \lim_{x \rightarrow 0} \left( \frac{[a]}{x} \right)^{\tg([b]x)} =\infty $\\
C. $ \lim_{x \rightarrow 0} \left( \frac{[a]}{x} \right)^{\tg([b]x)} =0 $\\
D. $ \lim_{x \rightarrow 0} \left( \frac{[a]}{x} \right)^{\tg([b]x)} =e $\\
E. $ \lim_{x \rightarrow 0} \left( \frac{[a]}{x} \right)^{\tg([b]x)} =[a] $\\
F. $ \lim_{x \rightarrow 0} \left( \frac{[a]}{x} \right)^{\tg([b]x)} =-1 $
\testStop
\kluczStart
A
\kluczStop



\end{document}