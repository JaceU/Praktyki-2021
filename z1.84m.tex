\documentclass[12pt, a4paper]{article}
\usepackage[utf8]{inputenc}
\usepackage{polski}

\usepackage{amsthm}  %pakiet do tworzenia twierdzeń itp.
\usepackage{amsmath} %pakiet do niektórych symboli matematycznych
\usepackage{amssymb} %pakiet do symboli mat., np. \nsubseteq
\usepackage{amsfonts}
\usepackage{graphicx} %obsługa plików graficznych z rozszerzeniem png, jpg
\theoremstyle{definition} %styl dla definicji
\newtheorem{zad}{} 
\title{Multizestaw zadań}
\author{Jacek Jabłoński}
%\date{\today}
\date{}
\newcounter{liczniksekcji}
\newcommand{\kategoria}[1]{\section{#1}} %olreślamy nazwę kateforii zadań
\newcommand{\zadStart}[1]{\begin{zad}#1\newline} %oznaczenie początku zadania
\newcommand{\zadStop}{\end{zad}}   %oznaczenie końca zadania
%Makra opcjonarne (nie muszą występować):
\newcommand{\rozwStart}[2]{\noindent \textbf{Rozwiązanie (autor #1 , recenzent #2): }\newline} %oznaczenie początku rozwiązania, opcjonarnie można wprowadzić informację o autorze rozwiązania zadania i recenzencie poprawności wykonania rozwiązania zadania
\newcommand{\rozwStop}{\newline}                                            %oznaczenie końca rozwiązania
\newcommand{\odpStart}{\noindent \textbf{Odpowiedź:}\newline}    %oznaczenie początku odpowiedzi końcowej (wypisanie wyniku)
\newcommand{\odpStop}{\newline}                                             %oznaczenie końca odpowiedzi końcowej (wypisanie wyniku)
\newcommand{\testStart}{\noindent \textbf{Test:}\newline} %ewentualne możliwe opcje odpowiedzi testowej: A. ? B. ? C. ? D. ? itd.
\newcommand{\testStop}{\newline} %koniec wprowadzania odpowiedzi testowych
\newcommand{\kluczStart}{\noindent \textbf{Test poprawna odpowiedź:}\newline} %klucz, poprawna odpowiedź pytania testowego (jedna literka): A lub B lub C lub D itd.
\newcommand{\kluczStop}{\newline} %koniec poprawnej odpowiedzi pytania testowego 
\newcommand{\wstawGrafike}[2]{\begin{figure}[h] \includegraphics[scale=#2] {#1} \end{figure}} %gdyby była potrzeba wstawienia obrazka, parametry: nazwa pliku, skala (jak nie wiesz co wpisać, to wpisz 1)

\begin{document}
\maketitle


\kategoria{Wikieł/z1.84m}
\zadStart{Zadanie z Wikieł z1.84m) moja wersja nr [nrWersji]}
%[p1]:[3,4,5]
%[p2]:[2,3,4,5,6,7,8,9,10,11,12,13,14,15,16,17,18,19,20,21,22,23,24,25,26,27,28,29,30]
%[p3]:[2,3,4,5,6]
%[a]=int(math.pow(2,[p1]))
%[delta]=abs(1-4*(-[p2]))
%[pdelta]=int(math.pow([delta],(1/2)))
%[t1]=int((-1-[pdelta])/2)
%[t2]=int((-1+[pdelta])/2)
%[c1]=math.sqrt([delta])
%[c2]=math.isqrt([delta])
%[c3]=math.gcd([t1],2)
%[c4]=math.gcd([t2],[p3])
%[r1]=int(math.pow(2,[p3]))
%[f1]=[p3]+1
%[f2]=[p3]+2
%[f3]=[p3]+3
%[f11]=[p1]+1
%[f22]=[p1]+2
%[f33]=[p1]+3
%[delta]>0 and not([c1]!=[c2]) and not([c4]!=[p3]) and not([r1]!=[t2]) 
Rozwiązać równanie:
m) $[a]^{2x} + [a]^x = [p2]$
\zadStop
\rozwStart{Jacek Jabłoński}{}
$$[a]^{2x} + [a]^x = [p2]$$
$$[a]^{2x} + [a]^x - [p2] = 0$$
$$t = [a]^x$$
$$t^2 + t - [p2] = 0 $$
$$\Delta = [delta]$$
$$\sqrt{\Delta} = [pdelta]$$
$$t_1 = \frac{-1-[pdelta]}{2 \cdot 1} = [t1]$$
$$t_2 = \frac{-1+[pdelta]}{2 \cdot 1} =[t2]$$
$$t_2 = [a]^x $$
$$[t2] = 2^{[p1]x} $$
$$2^[p3]=2^{[p1]x}$$
$$[p3]=[p1]x$$
$$x=\frac{[p3]}{[p1]}$$
\rozwStop
\odpStart
$$ x=\frac{[p3]}{[p1]} $$
\odpStop
\testStart
A. $$ x=\frac{[p3]}{[p1]} $$
B. $$ x=\frac{[f1]}{[p1]} $$
C. $$ x=\frac{[f2]}{[p1]} $$
D. $$ x=\frac{[f3]}{[p1]} $$
E. $$ x=\frac{[p1]}{[f11]} $$
F. $$ x=\frac{[p1]}{[f22]} $$
G. $$ x=\frac{[p1]}{[f33]} $$
H. $$ x=[f1] $$
I. $$ x=[f2] $$
\testStop
\kluczStart
A
\kluczStop



\end{document}