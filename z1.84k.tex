\documentclass[12pt, a4paper]{article}
\usepackage[utf8]{inputenc}
\usepackage{polski}
\usepackage{amsthm}  %pakiet do tworzenia twierdzeń itp.
\usepackage{amsmath} %pakiet do niektórych symboli matematycznych
\usepackage{amssymb} %pakiet do symboli mat., np. \nsubseteq
\usepackage{amsfonts}
\usepackage{graphicx} %obsługa plików graficznych z rozszerzeniem png, jpg
\theoremstyle{definition} %styl dla definicji
\newtheorem{zad}{} 
\title{Multizestaw zadań}
\author{Radosław Grzyb}
%\date{\today}
\date{}
\newcounter{liczniksekcji}
\newcommand{\kategoria}[1]{\section{#1}} %olreślamy nazwę kateforii zadań
\newcommand{\zadStart}[1]{\begin{zad}#1\newline} %oznaczenie początku zadania
\newcommand{\zadStop}{\end{zad}}   %oznaczenie końca zadania
%Makra opcjonarne (nie muszą występować):
\newcommand{\rozwStart}[2]{\noindent \textbf{Rozwiązanie (autor #1 , recenzent #2): }\newline} %oznaczenie początku rozwiązania, opcjonarnie można wprowadzić informację o autorze rozwiązania zadania i recenzencie poprawności wykonania rozwiązania zadania
\newcommand{\rozwStop}{\newline}                                            %oznaczenie końca rozwiązania
\newcommand{\odpStart}{\noindent \textbf{Odpowiedź:}\newline}    %oznaczenie początku odpowiedzi końcowej (wypisanie wyniku)
\newcommand{\odpStop}{\newline}                                             %oznaczenie końca odpowiedzi końcowej (wypisanie wyniku)
\newcommand{\testStart}{\noindent \textbf{Test:}\newline} %ewentualne możliwe opcje odpowiedzi testowej: A. ? B. ? C. ? D. ? itd.
\newcommand{\testStop}{\newline} %koniec wprowadzania odpowiedzi testowych
\newcommand{\kluczStart}{\noindent \textbf{Test poprawna odpowiedź:}\newline} %klucz, poprawna odpowiedź pytania testowego (jedna literka): A lub B lub C lub D itd.
\newcommand{\kluczStop}{\newline} %koniec poprawnej odpowiedzi pytania testowego 
\newcommand{\wstawGrafike}[2]{\begin{figure}[h] \includegraphics[scale=#2] {#1} \end{figure}} %gdyby była potrzeba wstawienia obrazka, parametry: nazwa pliku, skala (jak nie wiesz co wpisać, to wpisz 1)
\begin{document}
\maketitle
\kategoria{Wikieł/Z1.84k}
\zadStart{Zadanie z Wikieł Z 1.84k moja wersja nr [nrWersji]}
%[p2]:[5,13,29,53,68,85]
%[p3]:[2,3,5,6,7]
%[c1]=[p3]**2
%[c2]=[p2]-4
%[wynik]=int(math.sqrt([c2]))
%[wynik2]=int(-[wynik])
%[zlywynik1]=int(1-[wynik])
%[zlywynik11]=-[zlywynik1]+1.5
%[zlywynik2]=int(10-[wynik])
%[zlywynik22]=-[zlywynik2]+10
%[zlywynik3]=int(2*[wynik]-5)
%[zlywynik33]=-[zlywynik3]+21
Rozwiązać równanie:
$$(\sqrt{[p3]})^{x^{2}-[p2]}=\frac{1}{[c1]}$$
\zadStop
\rozwStart{Radosław Grzyb}{}
$$(\sqrt{[p3]})^{x^{2}-[p2]}=[p3]^{-2}$$
$$(\sqrt{[p3]})^{x^{2}-[p2]}=\sqrt{[p3]}^{-4}$$
Pierwiastkując obie strony równania otrzymujemy:
$$x^{2}-[p2]=-4$$
$$x^{2}=[c2]$$
$$x=[wynik] \vee [wynik2]$$
\rozwStop
\odpStart
$[wynik] \vee [wynik2]$
\odpStop
\testStart
A.$[zlywynik1] \vee [zlywynik11]$
B.$[wynik] \vee [wynik2]$
C.$[zlywynik2] \vee [zlywynik22]$
D.$[zlywynik3] \vee [zlywynik33]$
\testStop
\kluczStart
B
\kluczStop
\end{document}