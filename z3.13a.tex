\documentclass[12pt, a4paper]{article}
\usepackage[utf8]{inputenc}
\usepackage{polski}

\usepackage{amsthm}  %pakiet do tworzenia twierdzeń itp.
\usepackage{amsmath} %pakiet do niektórych symboli matematycznych
\usepackage{amssymb} %pakiet do symboli mat., np. \nsubseteq
\usepackage{amsfonts}
\usepackage{graphicx} %obsługa plików graficznych z rozszerzeniem png, jpg
\theoremstyle{definition} %styl dla definicji
\newtheorem{zad}{} 
\title{Multizestaw zadań}
\author{Robert Fidytek}
%\date{\today}
\date{}
\newcounter{liczniksekcji}
\newcommand{\kategoria}[1]{\section{#1}} %olreślamy nazwę kateforii zadań
\newcommand{\zadStart}[1]{\begin{zad}#1\newline} %oznaczenie początku zadania
\newcommand{\zadStop}{\end{zad}}   %oznaczenie końca zadania
%Makra opcjonarne (nie muszą występować):
\newcommand{\rozwStart}[2]{\noindent \textbf{Rozwiązanie (autor #1 , recenzent #2): }\newline} %oznaczenie początku rozwiązania, opcjonarnie można wprowadzić informację o autorze rozwiązania zadania i recenzencie poprawności wykonania rozwiązania zadania
\newcommand{\rozwStop}{\newline}                                            %oznaczenie końca rozwiązania
\newcommand{\odpStart}{\noindent \textbf{Odpowiedź:}\newline}    %oznaczenie początku odpowiedzi końcowej (wypisanie wyniku)
\newcommand{\odpStop}{\newline}                                             %oznaczenie końca odpowiedzi końcowej (wypisanie wyniku)
\newcommand{\testStart}{\noindent \textbf{Test:}\newline} %ewentualne możliwe opcje odpowiedzi testowej: A. ? B. ? C. ? D. ? itd.
\newcommand{\testStop}{\newline} %koniec wprowadzania odpowiedzi testowych
\newcommand{\kluczStart}{\noindent \textbf{Test poprawna odpowiedź:}\newline} %klucz, poprawna odpowiedź pytania testowego (jedna literka): A lub B lub C lub D itd.
\newcommand{\kluczStop}{\newline} %koniec poprawnej odpowiedzi pytania testowego 
\newcommand{\wstawGrafike}[2]{\begin{figure}[h] \includegraphics[scale=#2] {#1} \end{figure}} %gdyby była potrzeba wstawienia obrazka, parametry: nazwa pliku, skala (jak nie wiesz co wpisać, to wpisz 1)

\begin{document}
\maketitle


\kategoria{Wikieł/Z3.13a}
\zadStart{Zadanie z Wikieł Z 3.13 a) moja wersja nr [nrWersji]}
%[a]:[2,3,4,5,6,7,8,9]
%[b]:[2,3,4,5,6,7,8,9]
%[2b]=2*[b]
%[a]!=[b]
Obliczyć granicę ciągu 
$$a_n=\big(\sqrt{[a]^n+[b]}-\sqrt{[a]^n-[b]}\big).$$
\zadStop
\rozwStart{Adrianna Stobiecka}{}
$$\lim_{n\to\infty}\big(\sqrt{[a]^n+[b]}-\sqrt{[a]^n-[b]}\big)$$
$$=\lim_{n\to\infty}\frac{\big(\sqrt{[a]^n+[b]}-\sqrt{[a]^n-[b]}\big)\big(\sqrt{[a]^n+[b]}+\sqrt{[a]^n-[b]}\big)}{\sqrt{[a]^n+[b]}+\sqrt{[a]^n-[b]}}=(*)$$
W liczniku skorzystamy ze wzoru $(a-b)(a+b)=a^2-b^2$.
$$(*)=\lim_{n\to\infty}\frac{[a]^n+[b]-[a]^n+[b]}{\sqrt{[a]^n+[b]}+\sqrt{[a]^n-[b]}}=\lim_{n\to\infty}\frac{[2b]}{\sqrt{[a]^n+[b]}+\sqrt{[a]^n-[b]}}$$
$$\lim_{n\to\infty}\frac{[2b]}{\sqrt{[a]^n}\big(\sqrt{1+\frac{[b]}{[a]^n}}+\sqrt{1-\frac{[b]}{[a]^n}}\big)}=(**)$$
Wiemy, że $$\lim_{n\to\infty}\frac{[b]}{[a]^n}=0.$$
Mamy zatem:
$$(**)=\lim_{n\to\infty}\frac{[2b]}{2\sqrt{[a]^n}}=\lim_{n\to\infty}\frac{[b]}{\sqrt{[a]^n}}=0$$
\rozwStop
\odpStart
$0$
\odpStop
\testStart
A.$[a]$
B.$[b]$
C.$-[b]$
D.$0$
E.$-\infty$
F.$-[a]$
G.$1$
H.$\infty$
I.$-1$
\testStop
\kluczStart
D
\kluczStop



\end{document}
