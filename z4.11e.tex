\documentclass[12pt, a4paper]{article}
\usepackage[utf8]{inputenc}
\usepackage{polski}

\usepackage{amsthm}  %pakiet do tworzenia twierdzeń itp.
\usepackage{amsmath} %pakiet do niektórych symboli matematycznych
\usepackage{amssymb} %pakiet do symboli mat., np. \nsubseteq
\usepackage{amsfonts}
\usepackage{graphicx} %obsługa plików graficznych z rozszerzeniem png, jpg
\theoremstyle{definition} %styl dla definicji
\newtheorem{zad}{} 
\title{Multizestaw zadań}
\author{Robert Fidytek}
%\date{\today}
\date{}
\newcounter{liczniksekcji}
\newcommand{\kategoria}[1]{\section{#1}} %olreślamy nazwę kateforii zadań
\newcommand{\zadStart}[1]{\begin{zad}#1\newline} %oznaczenie początku zadania
\newcommand{\zadStop}{\end{zad}}   %oznaczenie końca zadania
%Makra opcjonarne (nie muszą występować):
\newcommand{\rozwStart}[2]{\noindent \textbf{Rozwiązanie (autor #1 , recenzent #2): }\newline} %oznaczenie początku rozwiązania, opcjonarnie można wprowadzić informację o autorze rozwiązania zadania i recenzencie poprawności wykonania rozwiązania zadania
\newcommand{\rozwStop}{\newline}                                            %oznaczenie końca rozwiązania
\newcommand{\odpStart}{\noindent \textbf{Odpowiedź:}\newline}    %oznaczenie początku odpowiedzi końcowej (wypisanie wyniku)
\newcommand{\odpStop}{\newline}                                             %oznaczenie końca odpowiedzi końcowej (wypisanie wyniku)
\newcommand{\testStart}{\noindent \textbf{Test:}\newline} %ewentualne możliwe opcje odpowiedzi testowej: A. ? B. ? C. ? D. ? itd.
\newcommand{\testStop}{\newline} %koniec wprowadzania odpowiedzi testowych
\newcommand{\kluczStart}{\noindent \textbf{Test poprawna odpowiedź:}\newline} %klucz, poprawna odpowiedź pytania testowego (jedna literka): A lub B lub C lub D itd.
\newcommand{\kluczStop}{\newline} %koniec poprawnej odpowiedzi pytania testowego 
\newcommand{\wstawGrafike}[2]{\begin{figure}[h] \includegraphics[scale=#2] {#1} \end{figure}} %gdyby była potrzeba wstawienia obrazka, parametry: nazwa pliku, skala (jak nie wiesz co wpisać, to wpisz 1)

\begin{document}
\maketitle


\kategoria{Wikieł/Z4.11e}
\zadStart{Zadanie z Wikieł Z 4.11 e) moja wersja nr [nrWersji]}
%[a]:[2,3,4,5,6,7]
%[b]:[2,3,4,5,6,7]
%[c]:[2,3,4,5,6,7]
%[a]=random.randint(2,10)
%[b]=random.randint(2,10) 
%[c]=random.randint(2,10)
%[a2]=[a]*[a]
%[ca]=[c]*[a]
%[ba]=[b]*[a]
%[cab]=[ca]+[b]
%[b1]=[ba]+1
Obliczyć granicę funkcji $\lim_{x \to \infty}arcctg\big(\frac{[c]x^{2}}{x+[a]}-\frac{[b]x+1}{x-[a]}\big)$.
\zadStop
\rozwStart{Jakub Ulrych}{Pascal Nawrocki}
$$\lim_{x \to \infty}arcctg\bigg(\frac{[c]x^{2}}{x+[a]}-\frac{[b]x+1}{x-[a]}\bigg)$$
$$\lim_{x \to \infty}arcctg\bigg(\frac{[c]x^{2}(x-[a])-([b]x+1)(x+[a])}{x^{2}-[a]^{2}}\bigg)$$
$$\lim_{x \to \infty}arcctg\bigg(\frac{[c]x^{3}-[ca]x^{2}-[b]x^{2}-[ba]x-x-[a]}{x^{2}-[a2]}\bigg)$$
$$\lim_{x \to \infty}arcctg\bigg(\frac{[c]x^{3}-[cab]x^{2}-[b1]x-[a]}{x^{2}-[a2]}\bigg)$$
$$\lim_{x \to \infty}arcctg\bigg(\frac{x^{2}([c]x-[cab]-\frac{[b1]}{x}-\frac{[a]}{x^{2}})}{x^{2}(1-\frac{[a2]}{x^{2}})}\bigg)$$
$$\lim_{x \to \infty}arcctg\bigg(\frac{[c]x-[cab]}{1}\bigg)=\lim_{x \to \infty}arcctg\bigg([c]x-[cab]\bigg)=0\text{, ponieważ }\lim_{x \to \infty}\bigg([c]x-[cab]\bigg)=\infty$$
\rozwStop
\odpStart
$$0$$
\odpStop
\testStart
A.$0$
B.$\infty$
C.$-\infty$
D.$1$
\testStop
\kluczStart
A
\kluczStop



\end{document}