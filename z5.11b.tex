\documentclass[12pt, a4paper]{article}
\usepackage[utf8]{inputenc}
\usepackage{polski}

\usepackage{amsthm}  %pakiet do tworzenia twierdzeń itp.
\usepackage{amsmath} %pakiet do niektórych symboli matematycznych
\usepackage{amssymb} %pakiet do symboli mat., np. \nsubseteq
\usepackage{amsfonts}
\usepackage{graphicx} %obsługa plików graficznych z rozszerzeniem png, jpg
\theoremstyle{definition} %styl dla definicji
\newtheorem{zad}{} 
\title{Multizestaw zadań}
\author{Robert Fidytek}
%\date{\today}
\date{}
\newcounter{liczniksekcji}
\newcommand{\kategoria}[1]{\section{#1}} %olreślamy nazwę kateforii zadań
\newcommand{\zadStart}[1]{\begin{zad}#1\newline} %oznaczenie początku zadania
\newcommand{\zadStop}{\end{zad}}   %oznaczenie końca zadania
%Makra opcjonarne (nie muszą występować):
\newcommand{\rozwStart}[2]{\noindent \textbf{Rozwiązanie (autor #1 , recenzent #2): }\newline} %oznaczenie początku rozwiązania, opcjonarnie można wprowadzić informację o autorze rozwiązania zadania i recenzencie poprawności wykonania rozwiązania zadania
\newcommand{\rozwStop}{\newline}                                            %oznaczenie końca rozwiązania
\newcommand{\odpStart}{\noindent \textbf{Odpowiedź:}\newline}    %oznaczenie początku odpowiedzi końcowej (wypisanie wyniku)
\newcommand{\odpStop}{\newline}                                             %oznaczenie końca odpowiedzi końcowej (wypisanie wyniku)
\newcommand{\testStart}{\noindent \textbf{Test:}\newline} %ewentualne możliwe opcje odpowiedzi testowej: A. ? B. ? C. ? D. ? itd.
\newcommand{\testStop}{\newline} %koniec wprowadzania odpowiedzi testowych
\newcommand{\kluczStart}{\noindent \textbf{Test poprawna odpowiedź:}\newline} %klucz, poprawna odpowiedź pytania testowego (jedna literka): A lub B lub C lub D itd.
\newcommand{\kluczStop}{\newline} %koniec poprawnej odpowiedzi pytania testowego 
\newcommand{\wstawGrafike}[2]{\begin{figure}[h] \includegraphics[scale=#2] {#1} \end{figure}} %gdyby była potrzeba wstawienia obrazka, parametry: nazwa pliku, skala (jak nie wiesz co wpisać, to wpisz 1)

\begin{document}
\maketitle


\kategoria{Wikieł/Z5.11b}
\zadStart{Zadanie z Wikieł Z 5.11 b) moja wersja nr [nrWersji]}
%[a]:[2,3,4,5,6,7,8,9]
%[b]:[2,3,4,5,6,7,8,9]
%[c]=[b]*[b]
%[d]=2*[b]
%[e]=[a]*[b]
%[f]=2*[a]*[d]
%[g]=[a]*[a]*[a]
%[h]=[d]*[d]*[d]
%[g1]=[a]*[a]
%[h1]=[d]*[d]
%[i]=[e]-[f]
%[j]=[i]*[h]
%[k]=[d]*[g]
%[a1]=[a]*[d]
%[a2]=[a1]-[e]
%[a3]=[a2]*[a2]
%[wspdziel]=math.gcd([j],[k])
%[j1]=-1*int([j]/[wspdziel])
%[k1]=int([k]/[wspdziel])
%[l1]=[a2]*[h1]
%[m1]=[d]*[g1]
%[wspdziel2]=math.gcd([l1],[m1])
%[l2]=int([l1]/[wspdziel2])
%[m2]=int([m1]/[wspdziel2])
%[j2]=[j1]*[a]
%[k2]=[k1]*[d]
%[j3]=[j2]*[m2]-[k2]*[l2]
%[k3]=[k2]*[m2]
%[wspdziel3]=math.gcd([j3],[k3])
%[l3]=int([j3]/[wspdziel3])
%[m3]=int([k3]/[wspdziel3])
%[n]=[c]*[b]
%[n2]=[n]*[h1]
%[wspdziel4]=math.gcd([n2],[a3])
%[n4]=int([n2]/[wspdziel4])
%[a4]=int([a3]/[wspdziel4])
%[o]=[c]*[d]
%[wspdziel5]=math.gcd([o],[a2])
%[o5]=int([o]/[wspdziel5])
%[a5]=int([a2]/[wspdziel5])
%[n6]=[n4]*[a]
%[a6]=[a4]*[d]
%[n8]=[o5]*[a6]-[n6]*[a5]
%[a8]=[a6]*[a5]
%[wspdziel6]=math.gcd([n8],[a8])
%[n7]=int([n8]/[wspdziel6])
%[a7]=int([a8]/[wspdziel6])
%[p1]=([j1]*[a4])+([n4]*[k1])
%[r1]=([k1]*[a4])-([j1]*[n4])
%[wspdziel7]=math.gcd([p1],[r1])
%[p2]=int([p1]/[wspdziel7])
%[r2]=-1*int([r1]/[wspdziel7])
%[k]!=0 and math.gcd([a],[d])==1 and [j]<0 and [m2]!=1 and [p2]>0 and [r2]>0
Wyznaczyć równania stycznych do krzywych $y=f(x)$ i $y=g(x)$ w punktach ich przecięcia, a następnie obliczyć tangens kąta ostrego między tymi krzywymi, jeżeli:\\
$f(x)=\frac{[b]x+[a]}{x^2}, \ \ \ \ \ g(x)=\frac{[c]}{[b]x+[a]}$.
\zadStop
\rozwStart{Joanna Świerzbin}{}
$$f(x)=\frac{[b]x+[a]}{x^2}, \ \ \ \ \ g(x)=\frac{[c]}{[b]x+[a]}$$
$$\frac{[b]x+[a]}{x^2}=\frac{[c]}{[b]x+[a]}$$
$$\frac{[b]x+[a]}{x^2}=\frac{[c]}{[b]x+[a]} \ \ \Big\backslash \cdot x^2([b]x+[a])$$
$$({[b]x+[a]})^2=[c]x^2$$
$$[b]^2x^2+2\cdot[a]\cdot[b]x+[a]^2=[c]x^2$$
$$2\cdot[b]x+[a]=0$$
$$x=-\frac{[a]}{[d]}$$
\\
Równanie stycznej do wykresu funkcji $y=f(x)$ w punkcie $P(x_0,f(x_0))$:
$$ y-f(x_0)=f'(x_0)(x-x_0)$$
\\
$$f'(x)=\left( \frac{[b]x+[a]}{x^2} \right)'=\frac{([b]x+[a])'x^2-([b]x+[a])(x^2)'}{x^4} = $$
$$ =\frac{[b]x^2-([b]x+[a])2x}{x^4} = \frac{[b]x^2-2\cdot[b]x^2-2\cdot[a]x}{x^4} = $$
$$ = \frac{-[b]x^2-2\cdot[a]x}{x^4} = \frac{-[b]x-2\cdot[a]}{x^3}$$
$$f'\left(-\frac{[a]}{[d]}\right)= \frac{-[b]\left(-\frac{[a]}{[d]}\right)-2\cdot[a]}{\left(-\frac{[a]}{[d]}\right)^3}=
\frac{\left(\frac{[a]\cdot [b]}{[d]}\right)-2\cdot[a]}{\left(-\frac{[a]}{[d]}\right)^3}=\frac{\frac{[e]-[f]}{[d]}}{-\frac{[g]}{[h]}}=$$
$$=\frac{\frac{[i]}{[d]}}{-\frac{[g]}{[h]}}=-\frac{([i])\cdot[h]}{[d]\cdot[g]}=-\frac{([j])}{[k]} = \frac{[j1]}{[k1]}$$
$$f\left(-\frac{[a]}{[d]}\right)= \frac{[b]\left(-\frac{[a]}{[d]}\right)+[a]}{\left(-\frac{[a]}{[d]}\right)^2}
= \frac{\left(-\frac{[a]\cdot[b]}{[d]}\right)+[a]}{\left(-\frac{[a]}{[d]}\right)^2}= \frac{\left(\frac{-[e]+[a1]}{[d]}\right)}{-\frac{[g1]}{[h1]}}=$$
$$= \frac{\frac{[a2]}{[d]}}{-\frac{[g1]}{[h1]}}=- \frac{[a2]\cdot[h1]}{[d]\cdot[g1]}=-\frac{[l1]}{[m1]}=-\frac{[l2]}{[m2]}$$
\\
$$y_1=\frac{[j1]}{[k1]}\left(x+\frac{[a]}{[d]}\right)-\frac{[l2]}{[m2]}=\frac{[j1]}{[k1]}x +\frac{[j1]\cdot[a]}{[k1]\cdot[d]}-\frac{[l2]}{[m2]}=
\frac{[j1]}{[k1]}x +\frac{[j2]}{[k2]}-\frac{[l2]}{[m2]}$$
$$y_1=\frac{[j1]}{[k1]}x +\frac{[l3]}{[m3]}$$
\\
$$g'(x)=\left( \frac{[c]}{[b]x+[a]} \right)'= \frac{-[c]\cdot[b]}{([b]x+[a])^2} =  - \frac{[n]}{([b]x+[a])^2}$$
$$g'\left( -\frac{[a]}{[d]} \right)= - \frac{[n]}{\left([b]\left( -\frac{[a]}{[d]} \right)+[a]\right)^2}=- \frac{[n]}{\left(-\frac{[a]\cdot [b]}{[d]}+[a]\right)^2}=
- \frac{[n]}{\left(\frac{-[e]+[a1]}{[d]}\right)^2} =$$
 $$= - \frac{[n]}{\frac{[a3]}{[h1]}} = - \frac{[n]\cdot[h1]}{[a3]} =- \frac{[n2]}{[a3]} =- \frac{[n4]}{[a4]}$$
$$g\left( -\frac{[a]}{[d]} \right)= \frac{[c]}{[b]\left( -\frac{[a]}{[d]} \right)+[a]}=\frac{[c]}{ \frac{-[e]+[a1]}{[d]}} =\frac{[c]}{ \frac{[a2]}{[d]}} =
\frac{[c]\cdot[d]}{ [a2]}=\frac{[o5]}{ [a5]}$$
\\
$$y_2=- \frac{[n4]}{[a4]}x - \frac{[n4]\cdot[a]}{[a4]\cdot[d]} +\frac{[o5]}{ [a5]} = - \frac{[n4]}{[a4]}x - \frac{[n6]}{[a6]} +\frac{[o5]}{ [a5]} $$
$$y_2= - \frac{[n4]}{[a4]}x $$
\\
Wzór na tangens kąta między krzywymi $f$ i $g$:
$$\tg(\alpha)=\Big|\frac{f'(x_0)-g'(x_0)}{1+f'(x_0)g'(x_0)}\Big| $$
$$y_1'=\frac{[j1]}{[k1]}$$
$$y_2'=-\frac{[n4]}{[a4]}$$
$$\tg(\alpha)=\Bigg|\frac{ \frac{[j1]}{[k1]} +\frac{[n4]}{[a4]} }{1+\left( \frac{[j1]}{[k1]} \right) \left(-\frac{[n4]}{[a4]}\right)}\Bigg| =
\Bigg|\frac{ \frac{[j1]\cdot[a4]+[n4]\cdot[k1]}{[k1]\cdot[a4]}}{1- \frac{[j1]\cdot[n4]}{[k1]\cdot[a4]} }\Bigg|
=\Bigg|\frac{ \frac{[j1]\cdot[a4]+[n4]\cdot[k1]}{[k1]\cdot[a4]}}{\frac{[k1]\cdot[a4]-[j1]\cdot[n4]}{[k1]\cdot[a4]}}\Bigg|= $$
$$=\Bigg|\frac{[j1]\cdot[a4]+[n4]\cdot[k1]}{[k1]\cdot[a4]-[j1]\cdot[n4]}\Bigg|= \frac{[p2]}{[r2]}$$
$$\tg(\alpha)=\frac{[p2]}{[r2]}$$
\rozwStop
\odpStart
$y_1=\frac{[j1]}{[k1]}x +\frac{[l3]}{[m3]}, \ \ \ \ y_2= - \frac{[n4]}{[a4]}x, \ \ \ \ \tg(\alpha)=\frac{[p2]}{[r2]}$
\odpStop
\testStart
A. $y_1=\frac{[j1]}{[k1]}x +\frac{[l3]}{[m3]}, \ \ \ \ y_2= - \frac{[n4]}{[a4]}x, \ \ \ \ \tg(\alpha)=\frac{[p2]}{[r2]}$\\
B. $y_1=x +\frac{[l3]}{[m3]}, \ \ \ \ y_2= - \frac{[n4]}{[a4]}x, \ \ \ \ \tg(\alpha)=\frac{[p2]}{[r2]}$\\
C. $y_1=\frac{[j1]}{[k1]}x, \ \ \ \ y_2= - \frac{[n4]}{[a4]}x, \ \ \ \ \tg(\alpha)=\frac{[p2]}{[r2]}$\\
D. $y_1=\frac{[j1]}{[k1]}x +\frac{[l3]}{[m3]}, \ \ \ \ y_2= - \frac{[n4]}{[a4]}x+\frac{[l3]}{[m3]}, \ \ \ \ \tg(\alpha)=\frac{[p2]}{[r2]}$\\
E. $y_1=\frac{[j1]}{[k1]}x +\frac{[l3]}{[m3]}, \ \ \ \ y_2= \frac{[l3]}{[m3]}, \ \ \ \ \tg(\alpha)=\frac{[p2]}{[r2]}$\\
F. $y_1=\frac{[j1]}{[k1]}x, \ \ \ \ y_2= - \frac{[n4]}{[a4]}x+ \frac{[l3]}{[m3]}, \ \ \ \ \tg(\alpha)=\frac{[p2]}{[r2]}$
\testStop
\kluczStart
A
\kluczStop



\end{document}