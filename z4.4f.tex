\documentclass[12pt, a4paper]{article}
\usepackage[utf8]{inputenc}
\usepackage{polski}
\usepackage{amsthm}  %pakiet do tworzenia twierdzeń itp.
\usepackage{amsmath} %pakiet do niektórych symboli matematycznych
\usepackage{amssymb} %pakiet do symboli mat., np. \nsubseteq
\usepackage{amsfonts}
\usepackage{graphicx} %obsługa plików graficznych z rozszerzeniem png, jpg
\theoremstyle{definition} %styl dla definicji
\newtheorem{zad}{} 
\title{Multizestaw zadań}
\author{Patryk Wirkus}
%\date{\today}
\date{}
\newcommand{\kategoria}[1]{\section{#1}}
\newcommand{\zadStart}[1]{\begin{zad}#1\newline}
\newcommand{\zadStop}{\end{zad}}
\newcommand{\rozwStart}[2]{\noindent \textbf{Rozwiązanie (autor #1 , recenzent #2): }\newline}
\newcommand{\rozwStop}{\newline}                                           
\newcommand{\odpStart}{\noindent \textbf{Odpowiedź:}\newline}
\newcommand{\odpStop}{\newline}
\newcommand{\testStart}{\noindent \textbf{Test:}\newline}
\newcommand{\testStop}{\newline}
\newcommand{\kluczStart}{\noindent \textbf{Test poprawna odpowiedź:}\newline}
\newcommand{\kluczStop}{\newline}
\newcommand{\wstawGrafike}[2]{\begin{figure}[h] \includegraphics[scale=#2] {#1} \end{figure}}

\begin{document}
\maketitle

\kategoria{Wikieł/Z4.4f}


\zadStart{Przykład z Wikieł Z 4.4f moja wersja nr 1}


Obliczyć granicę funkcji $\lim\limits_{x\to\ 0}\frac{x}{tan(7 \cdot x)}$.
\zadStop
\rozwStart{Patryk Wirkus}{Szymon Tokarski}
$$\lim\limits_{x\to\ 0}\frac{x}{tan(7 \cdot x)}=\lim\limits_{x\to\ 0}\frac{x \cdot cos(7 \cdot x)}{sin(7 \cdot x)}=\lim\limits_{x\to\ 0}\frac{cos(7 \cdot x)}{\frac{sin(7 \cdot x)}{x}}=\lim\limits_{x\to\ 0}\frac{cos(7 \cdot x)}{7 \cdot \frac{sin(7 \cdot x)}{7 \cdot x}} = \frac{1}{7}$$
\rozwStop
\odpStart
$\frac{1}{7}$
\odpStop
\testStart
A.$\frac{1}{7}$
B.$\infty$
C.$-\infty$
D.$0$
E.$-\frac{1}{7}$
F.$7$
G.$-7$
H.$tan(7 \cdot x)$
I.$sin(7 \cdot x)$
\testStop
\kluczStart
A
\kluczStop



\zadStart{Przykład z Wikieł Z 4.4f moja wersja nr 2}


Obliczyć granicę funkcji $\lim\limits_{x\to\ 0}\frac{x}{tan(9 \cdot x)}$.
\zadStop
\rozwStart{Patryk Wirkus}{Szymon Tokarski}
$$\lim\limits_{x\to\ 0}\frac{x}{tan(9 \cdot x)}=\lim\limits_{x\to\ 0}\frac{x \cdot cos(9 \cdot x)}{sin(9 \cdot x)}=\lim\limits_{x\to\ 0}\frac{cos(9 \cdot x)}{\frac{sin(9 \cdot x)}{x}}=\lim\limits_{x\to\ 0}\frac{cos(9 \cdot x)}{9 \cdot \frac{sin(9 \cdot x)}{9 \cdot x}} = \frac{1}{9}$$
\rozwStop
\odpStart
$\frac{1}{9}$
\odpStop
\testStart
A.$\frac{1}{9}$
B.$\infty$
C.$-\infty$
D.$0$
E.$-\frac{1}{9}$
F.$9$
G.$-9$
H.$tan(9 \cdot x)$
I.$sin(9 \cdot x)$
\testStop
\kluczStart
A
\kluczStop



\zadStart{Przykład z Wikieł Z 4.4f moja wersja nr 3}


Obliczyć granicę funkcji $\lim\limits_{x\to\ 0}\frac{x}{tan(11 \cdot x)}$.
\zadStop
\rozwStart{Patryk Wirkus}{Szymon Tokarski}
$$\lim\limits_{x\to\ 0}\frac{x}{tan(11 \cdot x)}=\lim\limits_{x\to\ 0}\frac{x \cdot cos(11 \cdot x)}{sin(11 \cdot x)}=\lim\limits_{x\to\ 0}\frac{cos(11 \cdot x)}{\frac{sin(11 \cdot x)}{x}}=\lim\limits_{x\to\ 0}\frac{cos(11 \cdot x)}{11 \cdot \frac{sin(11 \cdot x)}{11 \cdot x}} = \frac{1}{11}$$
\rozwStop
\odpStart
$\frac{1}{11}$
\odpStop
\testStart
A.$\frac{1}{11}$
B.$\infty$
C.$-\infty$
D.$0$
E.$-\frac{1}{11}$
F.$11$
G.$-11$
H.$tan(11 \cdot x)$
I.$sin(11 \cdot x)$
\testStop
\kluczStart
A
\kluczStop



\zadStart{Przykład z Wikieł Z 4.4f moja wersja nr 4}


Obliczyć granicę funkcji $\lim\limits_{x\to\ 0}\frac{x}{tan(13 \cdot x)}$.
\zadStop
\rozwStart{Patryk Wirkus}{Szymon Tokarski}
$$\lim\limits_{x\to\ 0}\frac{x}{tan(13 \cdot x)}=\lim\limits_{x\to\ 0}\frac{x \cdot cos(13 \cdot x)}{sin(13 \cdot x)}=\lim\limits_{x\to\ 0}\frac{cos(13 \cdot x)}{\frac{sin(13 \cdot x)}{x}}=\lim\limits_{x\to\ 0}\frac{cos(13 \cdot x)}{13 \cdot \frac{sin(13 \cdot x)}{13 \cdot x}} = \frac{1}{13}$$
\rozwStop
\odpStart
$\frac{1}{13}$
\odpStop
\testStart
A.$\frac{1}{13}$
B.$\infty$
C.$-\infty$
D.$0$
E.$-\frac{1}{13}$
F.$13$
G.$-13$
H.$tan(13 \cdot x)$
I.$sin(13 \cdot x)$
\testStop
\kluczStart
A
\kluczStop



\zadStart{Przykład z Wikieł Z 4.4f moja wersja nr 5}


Obliczyć granicę funkcji $\lim\limits_{x\to\ 0}\frac{x}{tan(17 \cdot x)}$.
\zadStop
\rozwStart{Patryk Wirkus}{Szymon Tokarski}
$$\lim\limits_{x\to\ 0}\frac{x}{tan(17 \cdot x)}=\lim\limits_{x\to\ 0}\frac{x \cdot cos(17 \cdot x)}{sin(17 \cdot x)}=\lim\limits_{x\to\ 0}\frac{cos(17 \cdot x)}{\frac{sin(17 \cdot x)}{x}}=\lim\limits_{x\to\ 0}\frac{cos(17 \cdot x)}{17 \cdot \frac{sin(17 \cdot x)}{17 \cdot x}} = \frac{1}{17}$$
\rozwStop
\odpStart
$\frac{1}{17}$
\odpStop
\testStart
A.$\frac{1}{17}$
B.$\infty$
C.$-\infty$
D.$0$
E.$-\frac{1}{17}$
F.$17$
G.$-17$
H.$tan(17 \cdot x)$
I.$sin(17 \cdot x)$
\testStop
\kluczStart
A
\kluczStop



\zadStart{Przykład z Wikieł Z 4.4f moja wersja nr 6}


Obliczyć granicę funkcji $\lim\limits_{x\to\ 0}\frac{x}{tan(19 \cdot x)}$.
\zadStop
\rozwStart{Patryk Wirkus}{Szymon Tokarski}
$$\lim\limits_{x\to\ 0}\frac{x}{tan(19 \cdot x)}=\lim\limits_{x\to\ 0}\frac{x \cdot cos(19 \cdot x)}{sin(19 \cdot x)}=\lim\limits_{x\to\ 0}\frac{cos(19 \cdot x)}{\frac{sin(19 \cdot x)}{x}}=\lim\limits_{x\to\ 0}\frac{cos(19 \cdot x)}{19 \cdot \frac{sin(19 \cdot x)}{19 \cdot x}} = \frac{1}{19}$$
\rozwStop
\odpStart
$\frac{1}{19}$
\odpStop
\testStart
A.$\frac{1}{19}$
B.$\infty$
C.$-\infty$
D.$0$
E.$-\frac{1}{19}$
F.$19$
G.$-19$
H.$tan(19 \cdot x)$
I.$sin(19 \cdot x)$
\testStop
\kluczStart
A
\kluczStop



\zadStart{Przykład z Wikieł Z 4.4f moja wersja nr 7}


Obliczyć granicę funkcji $\lim\limits_{x\to\ 0}\frac{x}{tan(21 \cdot x)}$.
\zadStop
\rozwStart{Patryk Wirkus}{Szymon Tokarski}
$$\lim\limits_{x\to\ 0}\frac{x}{tan(21 \cdot x)}=\lim\limits_{x\to\ 0}\frac{x \cdot cos(21 \cdot x)}{sin(21 \cdot x)}=\lim\limits_{x\to\ 0}\frac{cos(21 \cdot x)}{\frac{sin(21 \cdot x)}{x}}=\lim\limits_{x\to\ 0}\frac{cos(21 \cdot x)}{21 \cdot \frac{sin(21 \cdot x)}{21 \cdot x}} = \frac{1}{21}$$
\rozwStop
\odpStart
$\frac{1}{21}$
\odpStop
\testStart
A.$\frac{1}{21}$
B.$\infty$
C.$-\infty$
D.$0$
E.$-\frac{1}{21}$
F.$21$
G.$-21$
H.$tan(21 \cdot x)$
I.$sin(21 \cdot x)$
\testStop
\kluczStart
A
\kluczStop



\zadStart{Przykład z Wikieł Z 4.4f moja wersja nr 8}


Obliczyć granicę funkcji $\lim\limits_{x\to\ 0}\frac{x}{tan(23 \cdot x)}$.
\zadStop
\rozwStart{Patryk Wirkus}{Szymon Tokarski}
$$\lim\limits_{x\to\ 0}\frac{x}{tan(23 \cdot x)}=\lim\limits_{x\to\ 0}\frac{x \cdot cos(23 \cdot x)}{sin(23 \cdot x)}=\lim\limits_{x\to\ 0}\frac{cos(23 \cdot x)}{\frac{sin(23 \cdot x)}{x}}=\lim\limits_{x\to\ 0}\frac{cos(23 \cdot x)}{23 \cdot \frac{sin(23 \cdot x)}{23 \cdot x}} = \frac{1}{23}$$
\rozwStop
\odpStart
$\frac{1}{23}$
\odpStop
\testStart
A.$\frac{1}{23}$
B.$\infty$
C.$-\infty$
D.$0$
E.$-\frac{1}{23}$
F.$23$
G.$-23$
H.$tan(23 \cdot x)$
I.$sin(23 \cdot x)$
\testStop
\kluczStart
A
\kluczStop



\zadStart{Przykład z Wikieł Z 4.4f moja wersja nr 9}


Obliczyć granicę funkcji $\lim\limits_{x\to\ 0}\frac{x}{tan(27 \cdot x)}$.
\zadStop
\rozwStart{Patryk Wirkus}{Szymon Tokarski}
$$\lim\limits_{x\to\ 0}\frac{x}{tan(27 \cdot x)}=\lim\limits_{x\to\ 0}\frac{x \cdot cos(27 \cdot x)}{sin(27 \cdot x)}=\lim\limits_{x\to\ 0}\frac{cos(27 \cdot x)}{\frac{sin(27 \cdot x)}{x}}=\lim\limits_{x\to\ 0}\frac{cos(27 \cdot x)}{27 \cdot \frac{sin(27 \cdot x)}{27 \cdot x}} = \frac{1}{27}$$
\rozwStop
\odpStart
$\frac{1}{27}$
\odpStop
\testStart
A.$\frac{1}{27}$
B.$\infty$
C.$-\infty$
D.$0$
E.$-\frac{1}{27}$
F.$27$
G.$-27$
H.$tan(27 \cdot x)$
I.$sin(27 \cdot x)$
\testStop
\kluczStart
A
\kluczStop



\zadStart{Przykład z Wikieł Z 4.4f moja wersja nr 10}


Obliczyć granicę funkcji $\lim\limits_{x\to\ 0}\frac{x}{tan(29 \cdot x)}$.
\zadStop
\rozwStart{Patryk Wirkus}{Szymon Tokarski}
$$\lim\limits_{x\to\ 0}\frac{x}{tan(29 \cdot x)}=\lim\limits_{x\to\ 0}\frac{x \cdot cos(29 \cdot x)}{sin(29 \cdot x)}=\lim\limits_{x\to\ 0}\frac{cos(29 \cdot x)}{\frac{sin(29 \cdot x)}{x}}=\lim\limits_{x\to\ 0}\frac{cos(29 \cdot x)}{29 \cdot \frac{sin(29 \cdot x)}{29 \cdot x}} = \frac{1}{29}$$
\rozwStop
\odpStart
$\frac{1}{29}$
\odpStop
\testStart
A.$\frac{1}{29}$
B.$\infty$
C.$-\infty$
D.$0$
E.$-\frac{1}{29}$
F.$29$
G.$-29$
H.$tan(29 \cdot x)$
I.$sin(29 \cdot x)$
\testStop
\kluczStart
A
\kluczStop



\zadStart{Przykład z Wikieł Z 4.4f moja wersja nr 11}


Obliczyć granicę funkcji $\lim\limits_{x\to\ 0}\frac{x}{tan(31 \cdot x)}$.
\zadStop
\rozwStart{Patryk Wirkus}{Szymon Tokarski}
$$\lim\limits_{x\to\ 0}\frac{x}{tan(31 \cdot x)}=\lim\limits_{x\to\ 0}\frac{x \cdot cos(31 \cdot x)}{sin(31 \cdot x)}=\lim\limits_{x\to\ 0}\frac{cos(31 \cdot x)}{\frac{sin(31 \cdot x)}{x}}=\lim\limits_{x\to\ 0}\frac{cos(31 \cdot x)}{31 \cdot \frac{sin(31 \cdot x)}{31 \cdot x}} = \frac{1}{31}$$
\rozwStop
\odpStart
$\frac{1}{31}$
\odpStop
\testStart
A.$\frac{1}{31}$
B.$\infty$
C.$-\infty$
D.$0$
E.$-\frac{1}{31}$
F.$31$
G.$-31$
H.$tan(31 \cdot x)$
I.$sin(31 \cdot x)$
\testStop
\kluczStart
A
\kluczStop



\zadStart{Przykład z Wikieł Z 4.4f moja wersja nr 12}


Obliczyć granicę funkcji $\lim\limits_{x\to\ 0}\frac{x}{tan(33 \cdot x)}$.
\zadStop
\rozwStart{Patryk Wirkus}{Szymon Tokarski}
$$\lim\limits_{x\to\ 0}\frac{x}{tan(33 \cdot x)}=\lim\limits_{x\to\ 0}\frac{x \cdot cos(33 \cdot x)}{sin(33 \cdot x)}=\lim\limits_{x\to\ 0}\frac{cos(33 \cdot x)}{\frac{sin(33 \cdot x)}{x}}=\lim\limits_{x\to\ 0}\frac{cos(33 \cdot x)}{33 \cdot \frac{sin(33 \cdot x)}{33 \cdot x}} = \frac{1}{33}$$
\rozwStop
\odpStart
$\frac{1}{33}$
\odpStop
\testStart
A.$\frac{1}{33}$
B.$\infty$
C.$-\infty$
D.$0$
E.$-\frac{1}{33}$
F.$33$
G.$-33$
H.$tan(33 \cdot x)$
I.$sin(33 \cdot x)$
\testStop
\kluczStart
A
\kluczStop



\zadStart{Przykład z Wikieł Z 4.4f moja wersja nr 13}


Obliczyć granicę funkcji $\lim\limits_{x\to\ 0}\frac{x}{tan(37 \cdot x)}$.
\zadStop
\rozwStart{Patryk Wirkus}{Szymon Tokarski}
$$\lim\limits_{x\to\ 0}\frac{x}{tan(37 \cdot x)}=\lim\limits_{x\to\ 0}\frac{x \cdot cos(37 \cdot x)}{sin(37 \cdot x)}=\lim\limits_{x\to\ 0}\frac{cos(37 \cdot x)}{\frac{sin(37 \cdot x)}{x}}=\lim\limits_{x\to\ 0}\frac{cos(37 \cdot x)}{37 \cdot \frac{sin(37 \cdot x)}{37 \cdot x}} = \frac{1}{37}$$
\rozwStop
\odpStart
$\frac{1}{37}$
\odpStop
\testStart
A.$\frac{1}{37}$
B.$\infty$
C.$-\infty$
D.$0$
E.$-\frac{1}{37}$
F.$37$
G.$-37$
H.$tan(37 \cdot x)$
I.$sin(37 \cdot x)$
\testStop
\kluczStart
A
\kluczStop



\zadStart{Przykład z Wikieł Z 4.4f moja wersja nr 14}


Obliczyć granicę funkcji $\lim\limits_{x\to\ 0}\frac{x}{tan(39 \cdot x)}$.
\zadStop
\rozwStart{Patryk Wirkus}{Szymon Tokarski}
$$\lim\limits_{x\to\ 0}\frac{x}{tan(39 \cdot x)}=\lim\limits_{x\to\ 0}\frac{x \cdot cos(39 \cdot x)}{sin(39 \cdot x)}=\lim\limits_{x\to\ 0}\frac{cos(39 \cdot x)}{\frac{sin(39 \cdot x)}{x}}=\lim\limits_{x\to\ 0}\frac{cos(39 \cdot x)}{39 \cdot \frac{sin(39 \cdot x)}{39 \cdot x}} = \frac{1}{39}$$
\rozwStop
\odpStart
$\frac{1}{39}$
\odpStop
\testStart
A.$\frac{1}{39}$
B.$\infty$
C.$-\infty$
D.$0$
E.$-\frac{1}{39}$
F.$39$
G.$-39$
H.$tan(39 \cdot x)$
I.$sin(39 \cdot x)$
\testStop
\kluczStart
A
\kluczStop





\end{document}
