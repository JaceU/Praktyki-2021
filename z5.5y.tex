\documentclass[12pt, a4paper]{article}
\usepackage[utf8]{inputenc}
\usepackage{polski}

\usepackage{amsthm}  %pakiet do tworzenia twierdzeń itp.
\usepackage{amsmath} %pakiet do niektórych symboli matematycznych
\usepackage{amssymb} %pakiet do symboli mat., np. \nsubseteq
\usepackage{amsfonts}
\usepackage{graphicx} %obsługa plików graficznych z rozszerzeniem png, jpg
\theoremstyle{definition} %styl dla definicji
\newtheorem{zad}{} 
\title{Multizestaw zadań}
\author{Robert Fidytek}
%\date{\today}
\date{}
\newcounter{liczniksekcji}
\newcommand{\kategoria}[1]{\section{#1}} %olreślamy nazwę kateforii zadań
\newcommand{\zadStart}[1]{\begin{zad}#1\newline} %oznaczenie początku zadania
\newcommand{\zadStop}{\end{zad}}   %oznaczenie końca zadania
%Makra opcjonarne (nie muszą występować):
\newcommand{\rozwStart}[2]{\noindent \textbf{Rozwiązanie (autor #1 , recenzent #2): }\newline} %oznaczenie początku rozwiązania, opcjonarnie można wprowadzić informację o autorze rozwiązania zadania i recenzencie poprawności wykonania rozwiązania zadania
\newcommand{\rozwStop}{\newline}                                            %oznaczenie końca rozwiązania
\newcommand{\odpStart}{\noindent \textbf{Odpowiedź:}\newline}    %oznaczenie początku odpowiedzi końcowej (wypisanie wyniku)
\newcommand{\odpStop}{\newline}                                             %oznaczenie końca odpowiedzi końcowej (wypisanie wyniku)
\newcommand{\testStart}{\noindent \textbf{Test:}\newline} %ewentualne możliwe opcje odpowiedzi testowej: A. ? B. ? C. ? D. ? itd.
\newcommand{\testStop}{\newline} %koniec wprowadzania odpowiedzi testowych
\newcommand{\kluczStart}{\noindent \textbf{Test poprawna odpowiedź:}\newline} %klucz, poprawna odpowiedź pytania testowego (jedna literka): A lub B lub C lub D itd.
\newcommand{\kluczStop}{\newline} %koniec poprawnej odpowiedzi pytania testowego 
\newcommand{\wstawGrafike}[2]{\begin{figure}[h] \includegraphics[scale=#2] {#1} \end{figure}} %gdyby była potrzeba wstawienia obrazka, parametry: nazwa pliku, skala (jak nie wiesz co wpisać, to wpisz 1)

\begin{document}
\maketitle


\kategoria{Wikieł/Z5.5y}
\zadStart{Zadanie z Wikieł Z 5.5 y) moja wersja nr [nrWersji]}
%[a]:[2,3,4,5,6,7,8,9]
%[b]:[2,3,4,5,6,7,8,9]
%[c]:[2,3,4,5,6,7,8,9]
%[d]:[2,3,4,5,6,7,8,9]
%[e]=random.randint(2,10)
%[f]=random.randint(2,10)
%[g]=random.randint(2,10)
%[h]=random.randint(2,10)
%[i]=random.randint(2,10)
%[j]=random.randint(2,10)
%[gf]=[g]*[f]
%[u1]=1-[b]
%[u2]=1-[d]
%[u3]=-1-[b]
%[u4]=-[g]-[h]
%[b]<[d] and (1/[b])!=([g]/[h]) and [g]<[h] and math.gcd([a],[b])==1 and math.gcd([c],[d])==1 and math.gcd([e],[b])==1 and math.gcd([g],[h])==1 and math.gcd([f],[h])==1
Wyznacz pochodną funkcji \\ $f(x)=[a]x^{\frac{1}{[b]}}+[c]x^{\frac{1}{[d]}}+[e]x^{-\frac{1}{[b]}}+[f]x^{-\frac{[g]}{[h]}}+\frac{[i]}{\sqrt{[j]}} $.
\zadStop
\rozwStart{Joanna Świerzbin}{}
$$ f(x)=[a]x^{\frac{1}{[b]}}+[c]x^{\frac{1}{[d]}}+[e]x^{-\frac{1}{[b]}}+[f]x^{-\frac{[g]}{[h]}}+\frac{[i]}{\sqrt{[j]}}  $$
$$ f'(x)=\left([a]x^{\frac{1}{[b]}}+[c]x^{\frac{1}{[d]}}+[e]x^{-\frac{1}{[b]}}+[f]x^{-\frac{[g]}{[h]}}+\frac{[i]}{\sqrt{[j]}}\right)' =  $$
$$ =\frac{[a]}{[b]}x^{\frac{1}{[b]}-1}+\frac{[c]}{[d]}x^{\frac{1}{[d]}-1}-\frac{[e]}{[b]}x^{-\frac{1}{[b]}-1}-\frac{[g]\cdot[f]}{[h]}x^{-\frac{[g]}{[h]}-1} =  $$
$$ =\frac{[a]}{[b]}x^{\frac{[u1]}{[b]}}+\frac{[c]}{[d]}x^{\frac{[u2]}{[d]}}-\frac{[e]}{[b]}x^{\frac{[u3]}{[b]}}-\frac{[gf]}{[h]}x^{\frac{[u4]}{[h]}} $$
\rozwStop
\odpStart
$ f'(x)  =\frac{[a]}{[b]}x^{\frac{[u1]}{[b]}}+\frac{[c]}{[d]}x^{\frac{[u2]}{[d]}}-\frac{[e]}{[b]}x^{\frac{[u3]}{[b]}}-\frac{[gf]}{[h]}x^{\frac{[u4]}{[h]}} $
\odpStop
\testStart
A. $ f'(x)  =\frac{[a]}{[b]}x^{\frac{[u1]}{[b]}}+\frac{[c]}{[d]}x^{\frac{[u2]}{[d]}}-\frac{[e]}{[b]}x^{\frac{[u3]}{[b]}}-\frac{[gf]}{[h]}x^{\frac{[u4]}{[h]}} $\\
B. $ f'(x)  =\frac{[a]}{[b]}x^{\frac{1}{[b]}}+\frac{[c]}{[d]}x^{\frac{[u2]}{[d]}}-\frac{[e]}{[b]}x^{\frac{[u3]}{[b]}}-\frac{[gf]}{[h]}x^{\frac{[u4]}{[h]}} $\\
C. $ f'(x)  =\frac{[a]}{[b]}x^{\frac{[u1]}{[b]}}+\frac{[c]}{[d]}x^{\frac{1}{[d]}}-\frac{[e]}{[b]}x^{\frac{[u3]}{[b]}}-\frac{[gf]}{[h]}x^{\frac{[u4]}{[h]}} $\\
D. $ f'(x)  =x^{\frac{[u1]}{[b]}}+x^{\frac{[u2]}{[d]}}-x^{\frac{[u3]}{[b]}}-x^{\frac{[u4]}{[h]}} $\\
E. $ f'(x) =\frac{[a]}{[b]}x^{\frac{[u1]}{[b]}}+\frac{[c]}{[d]}x^{\frac{[u2]}{[d]}}+\frac{[e]}{[b]}x^{\frac{[u3]}{[b]}}+\frac{[gf]}{[h]}x^{\frac{[u4]}{[h]}} $\\
F. $ f'(x)  =\frac{[a]}{[b]}x^{\frac{[u1]}{[b]}}+\frac{[c]}{[d]}x^{\frac{[u2]}{[d]}} $
\testStop
\kluczStart
A
\kluczStop
\end{document}