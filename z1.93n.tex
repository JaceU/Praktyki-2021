\documentclass[12pt, a4paper]{article}
\usepackage[utf8]{inputenc}
\usepackage{polski}

\usepackage{amsthm}  %pakiet do tworzenia twierdzeń itp.
\usepackage{amsmath} %pakiet do niektórych symboli matematycznych
\usepackage{amssymb} %pakiet do symboli mat., np. \nsubseteq
\usepackage{amsfonts}
\usepackage{graphicx} %obsługa plików graficznych z rozszerzeniem png, jpg
\theoremstyle{definition} %styl dla definicji
\newtheorem{zad}{} 
\title{Multizestaw zadań}
\author{Robert Fidytek}
%\date{\today}
\date{}
\newcounter{liczniksekcji}
\newcommand{\kategoria}[1]{\section{#1}} %olreślamy nazwę kateforii zadań
\newcommand{\zadStart}[1]{\begin{zad}#1\newline} %oznaczenie początku zadania
\newcommand{\zadStop}{\end{zad}}   %oznaczenie końca zadania
%Makra opcjonarne (nie muszą występować):
\newcommand{\rozwStart}[2]{\noindent \textbf{Rozwiązanie (autor #1 , recenzent #2): }\newline} %oznaczenie początku rozwiązania, opcjonarnie można wprowadzić informację o autorze rozwiązania zadania i recenzencie poprawności wykonania rozwiązania zadania
\newcommand{\rozwStop}{\newline}                                            %oznaczenie końca rozwiązania
\newcommand{\odpStart}{\noindent \textbf{Odpowiedź:}\newline}    %oznaczenie początku odpowiedzi końcowej (wypisanie wyniku)
\newcommand{\odpStop}{\newline}                                             %oznaczenie końca odpowiedzi końcowej (wypisanie wyniku)
\newcommand{\testStart}{\noindent \textbf{Test:}\newline} %ewentualne możliwe opcje odpowiedzi testowej: A. ? B. ? C. ? D. ? itd.
\newcommand{\testStop}{\newline} %koniec wprowadzania odpowiedzi testowych
\newcommand{\kluczStart}{\noindent \textbf{Test poprawna odpowiedź:}\newline} %klucz, poprawna odpowiedź pytania testowego (jedna literka): A lub B lub C lub D itd.
\newcommand{\kluczStop}{\newline} %koniec poprawnej odpowiedzi pytania testowego 
\newcommand{\wstawGrafike}[2]{\begin{figure}[h] \includegraphics[scale=#2] {#1} \end{figure}} %gdyby była potrzeba wstawienia obrazka, parametry: nazwa pliku, skala (jak nie wiesz co wpisać, to wpisz 1)

\begin{document}
\maketitle


\kategoria{Wikieł/Z1.93n}
\zadStart{Zadanie z Wikieł Z 1.93 n) moja wersja nr [nrWersji]}
%[n]:[2,3,4,5,6,7,8,9,10]
%[c]=[n]*2
Rozwiązać równanie $[n]^{\log^2_{[n]}{x}}+ x^{\log_{[n]}{x}} = [c]$
\zadStop
\rozwStart{Małgorzata Ugowska}{}
Dziedzina: $x \in (0, \infty)$
Rozwiązujemy równanie:
$$ [n]^{\log^2_{[n]}{x}}+ x^{\log_{[n]}{x}} = [c] \quad \Longleftrightarrow \quad \Big([n]^{\log_{[n]}{x}}\Big)^{\log_{[n]}{x}}+ x^{\log_{[n]}{x}} = [c] $$
$$ \Longleftrightarrow \quad x^{\log_{[n]}{x}}+ x^{\log_{[n]}{x}} = [c] \quad \Longleftrightarrow \quad  2x^{\log_{[n]}{x}} = [c] $$
$$ \Longleftrightarrow \quad x^{\log_{[n]}{x}} = [n] \quad \Longleftrightarrow \quad \log_{x}{[n]}=\log_{[n]}{x}$$
$$ \Longleftrightarrow \quad \frac{1}{\log_{[n]}{x}} = \log_{[n]}{x} \quad \Longleftrightarrow \quad 1 = \log^2_{[n]}{x} $$
$$ \Longleftrightarrow \quad \log_{[n]}{x} = 1 \quad \vee \quad \log_{[n]}{x} = -1  \quad \Longleftrightarrow \quad x= [n]^{-1} = \frac{1}{[n]} \quad \vee \quad x = [n]$$
\rozwStop
\odpStart
$x \in \{\frac{1}{[n]} , [n] \}$
\odpStop
\testStart
A. $x \in \{-[n], [n] \}$\\
B. $x \in \{\frac{1}{[n]} , [n] \}$\\
C. $x \in \{-1, 1 \}$\\
D. $x \in \{-\frac{1}{[n]} , \frac{1}{[n]}  \}$\\
E. $x \in \{0, 1 \}$
\testStop
\kluczStart
B
\kluczStop



\end{document}