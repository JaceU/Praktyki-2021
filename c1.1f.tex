\documentclass[12pt, a4paper]{article}
\usepackage[utf8]{inputenc}
\usepackage{polski}

\usepackage{amsthm}  %pakiet do tworzenia twierdzeń itp.
\usepackage{amsmath} %pakiet do niektórych symboli matematycznych
\usepackage{amssymb} %pakiet do symboli mat., np. \nsubseteq
\usepackage{amsfonts}
\usepackage{graphicx} %obsługa plików graficznych z rozszerzeniem png, jpg
\theoremstyle{definition} %styl dla definicji
\newtheorem{zad}{} 
\title{Multizestaw zadań}
\author{Robert Fidytek}
%\date{\today}
\date{}
\newcounter{liczniksekcji}
\newcommand{\kategoria}[1]{\section{#1}} %olreślamy nazwę kateforii zadań
\newcommand{\zadStart}[1]{\begin{zad}#1\newline} %oznaczenie początku zadania
\newcommand{\zadStop}{\end{zad}}   %oznaczenie końca zadania
%Makra opcjonarne (nie muszą występować):
\newcommand{\rozwStart}[2]{\noindent \textbf{Rozwiązanie (autor #1 , recenzent #2): }\newline} %oznaczenie początku rozwiązania, opcjonarnie można wprowadzić informację o autorze rozwiązania zadania i recenzencie poprawności wykonania rozwiązania zadania
\newcommand{\rozwStop}{\newline}                                            %oznaczenie końca rozwiązania
\newcommand{\odpStart}{\noindent \textbf{Odpowiedź:}\newline}    %oznaczenie początku odpowiedzi końcowej (wypisanie wyniku)
\newcommand{\odpStop}{\newline}                                             %oznaczenie końca odpowiedzi końcowej (wypisanie wyniku)
\newcommand{\testStart}{\noindent \textbf{Test:}\newline} %ewentualne możliwe opcje odpowiedzi testowej: A. ? B. ? C. ? D. ? itd.
\newcommand{\testStop}{\newline} %koniec wprowadzania odpowiedzi testowych
\newcommand{\kluczStart}{\noindent \textbf{Test poprawna odpowiedź:}\newline} %klucz, poprawna odpowiedź pytania testowego (jedna literka): A lub B lub C lub D itd.
\newcommand{\kluczStop}{\newline} %koniec poprawnej odpowiedzi pytania testowego 
\newcommand{\wstawGrafike}[2]{\begin{figure}[h] \includegraphics[scale=#2] {#1} \end{figure}} %gdyby była potrzeba wstawienia obrazka, parametry: nazwa pliku, skala (jak nie wiesz co wpisać, to wpisz 1)

\begin{document}
\maketitle


\kategoria{Dymkowska, Beger/C1.1f}
\zadStart{Zadanie z Dymkowska, Beger C 1.1f moja wersja nr [nrWersji]}
%[c]:[2,3,4,5,6,7]
%[b]=[c]+1
%[a]=[b]+1

Obliczyć całkę $$\int \frac{x^{[a]}-[b]x^{[b]}+[c]x^{[c]}-1}{x^{[b]}}dx$$
\zadStop
\rozwStart{Laura Mieczkowska}{}
$$\int \frac{x^{[a]}-[b]x^{[b]}+[c]x^{[c]}-1}{x^{[b]}}dx=\int \bigg(\frac{x^{[a]}}{x^{[b]}}-\frac{[b]x^{[b]}}{x^{[b]}}+\frac{[c]x^{[c]}}{x^{[b]}}-\frac{1}{x^{[b]}}\bigg)dx=$$
$$=\int xdx-\int [b]dx+\int \frac{[c]}{x}dx-\int \frac{1}{x^{[b]}}dx=$$
$$=\frac{x^2}{2}-[b]x+[c]lnx+\frac{1}{[c]x^{[c]}}+C$$
\rozwStop
\odpStart
$\frac{x^2}{2}-[b]x+[c]lnx+\frac{1}{[c]x^{[c]}}+C$
\odpStop
\testStart
A. $\frac{x^2}{2}-[c]x+[a]lnx+\frac{1}{[c]x^{[c]}}+C$ \\
B. $\frac{x^2}{2}-[b]x+[c]lnx+\frac{1}{[c]x^{[c]}}+C$ \\
C. $\frac{x^2}{2}+[b]x+[c]lnx+\frac{1}{x^{[c]}}+C$ \\
D. $\frac{x^2}{2}-[b]x-[c]lnx-\frac{1}{[c]x^{[c]}}+C$
\testStop
\kluczStart
B
\kluczStop



\end{document}