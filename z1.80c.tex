\documentclass[12pt, a4paper]{article}
\usepackage[utf8]{inputenc}
\usepackage{polski}

\usepackage{amsthm}  %pakiet do tworzenia twierdzeń itp.
\usepackage{amsmath} %pakiet do niektórych symboli matematycznych
\usepackage{amssymb} %pakiet do symboli mat., np. \nsubseteq
\usepackage{amsfonts}
\usepackage{graphicx} %obsługa plików graficznych z rozszerzeniem png, jpg
\theoremstyle{definition} %styl dla definicji
\newtheorem{zad}{} 
\title{Multizestaw zadań}
\author{Robert Fidytek}
%\date{\today}
\date{}
\newcounter{liczniksekcji}
\newcommand{\kategoria}[1]{\section{#1}} %olreślamy nazwę kateforii zadań
\newcommand{\zadStart}[1]{\begin{zad}#1\newline} %oznaczenie początku zadania
\newcommand{\zadStop}{\end{zad}}   %oznaczenie końca zadania
%Makra opcjonarne (nie muszą występować):
\newcommand{\rozwStart}[2]{\noindent \textbf{Rozwiązanie (autor #1 , recenzent #2): }\newline} %oznaczenie początku rozwiązania, opcjonarnie można wprowadzić informację o autorze rozwiązania zadania i recenzencie poprawności wykonania rozwiązania zadania
\newcommand{\rozwStop}{\newline}                                            %oznaczenie końca rozwiązania
\newcommand{\odpStart}{\noindent \textbf{Odpowiedź:}\newline}    %oznaczenie początku odpowiedzi końcowej (wypisanie wyniku)
\newcommand{\odpStop}{\newline}                                             %oznaczenie końca odpowiedzi końcowej (wypisanie wyniku)
\newcommand{\testStart}{\noindent \textbf{Test:}\newline} %ewentualne możliwe opcje odpowiedzi testowej: A. ? B. ? C. ? D. ? itd.
\newcommand{\testStop}{\newline} %koniec wprowadzania odpowiedzi testowych
\newcommand{\kluczStart}{\noindent \textbf{Test poprawna odpowiedź:}\newline} %klucz, poprawna odpowiedź pytania testowego (jedna literka): A lub B lub C lub D itd.
\newcommand{\kluczStop}{\newline} %koniec poprawnej odpowiedzi pytania testowego 
\newcommand{\wstawGrafike}[2]{\begin{figure}[h] \includegraphics[scale=#2] {#1} \end{figure}} %gdyby była potrzeba wstawienia obrazka, parametry: nazwa pliku, skala (jak nie wiesz co wpisać, to wpisz 1)

\begin{document}
\maketitle


\kategoria{Wikieł/Z1.80c}
\zadStart{Zadanie z Wikieł Z 1.80 c)  moja wersja nr [nrWersji]}

%[p1]:[2,3,4,5,6,7,8,9]
%[p2]=random.randint(1,20)
%[a]:[2,3,4,5,6,7,8,9,10,11,12,13,14,15]
%[ak]=pow([a],2)
%[a4]=pow([a],4)
%[u]=int([a4]/[ak])
%[d]=int(-[p2]-[u])
%[c]:[2,3,4,5,6,7,8,9,10,11,12,13,14,15]
%[del]=[p1]*[p1]-4*[d]
%[pdel]=round(math.sqrt(abs([del])),2)
%[x1]=round((-[p1]-[pdel])/2,2)
%[x2]=round((-[p1]+[pdel])/2,2)
%([a4]/[ak])==[c] and [del]>0


Rozwiązać równanie 
$$(x^{2}+[p1]x-[p2])^{-\frac{1}{4}}=\frac{\sqrt{[a]}}{[a]}$$
\zadStop

\rozwStart{Maja Szabłowska}{}
$$(x^{2}+[p1]x-[p2])^{-\frac{1}{4}}=\frac{\sqrt{[a]}}{[a]}$$
$$x^{2}+[p1]x-[p2]=\left(\frac{\sqrt{[a]}}{[a]}\right)^{-4}$$
$$x^{2}+[p1]x-[p2]=\left(\frac{[a]}{\sqrt{[a]}}\right)^{4}$$
$$x^{2}+[p1]x-[p2]=\frac{[a4]}{[ak]}$$
$$x^{2}+[p1]x-[p2]=[u]$$
$$x^{2}+[p1]x+([d])=0$$
$$\Delta=[p1]^2 - 4\cdot1\cdot([d])=[del]\Rightarrow \sqrt{\Delta}=[pdel]$$
$$x_{1}=\frac{-[p1]-[pdel]}{2}=[x1], \quad x_{2}=\frac{-[p1]+[pdel]}{2}=[x2]$$
\rozwStop
\odpStart
$x_{1}=[x1], \quad x_{2}=[x2]$
\odpStop
\testStart
A.$x_{1}=[x1], \quad x_{2}=[x2]$\\
B.$x_{1}=[d], \quad x_{2}=[x2]$\\
D.$x_{1}=[del], \quad x_{2}=[pdel]$\\
E.$x_{1}=[a4], \quad x_{2}=0$\\
F.$x_{1}=[ak], \quad x_{2}=[del]$\\
G.$x_{1}=[x1], \quad x_{2}=[d]$\\
H.$x_{1}=[a], \quad x_{2}=[c]$\\
\testStop
\kluczStart
A
\kluczStop



\end{document}
