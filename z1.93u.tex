\documentclass[12pt, a4paper]{article}
\usepackage[utf8]{inputenc}
\usepackage{polski}

\usepackage{amsthm}  %pakiet do tworzenia twierdzeń itp.
\usepackage{amsmath} %pakiet do niektórych symboli matematycznych
\usepackage{amssymb} %pakiet do symboli mat., np. \nsubseteq
\usepackage{amsfonts}
\usepackage{graphicx} %obsługa plików graficznych z rozszerzeniem png, jpg
\theoremstyle{definition} %styl dla definicji
\newtheorem{zad}{} 
\title{Multizestaw zadań}
\author{Robert Fidytek}
%\date{\today}
\date{}
\newcounter{liczniksekcji}
\newcommand{\kategoria}[1]{\section{#1}} %olreślamy nazwę kateforii zadań
\newcommand{\zadStart}[1]{\begin{zad}#1\newline} %oznaczenie początku zadania
\newcommand{\zadStop}{\end{zad}}   %oznaczenie końca zadania
%Makra opcjonarne (nie muszą występować):
\newcommand{\rozwStart}[2]{\noindent \textbf{Rozwiązanie (autor #1 , recenzent #2): }\newline} %oznaczenie początku rozwiązania, opcjonarnie można wprowadzić informację o autorze rozwiązania zadania i recenzencie poprawności wykonania rozwiązania zadania
\newcommand{\rozwStop}{\newline}                                            %oznaczenie końca rozwiązania
\newcommand{\odpStart}{\noindent \textbf{Odpowiedź:}\newline}    %oznaczenie początku odpowiedzi końcowej (wypisanie wyniku)
\newcommand{\odpStop}{\newline}                                             %oznaczenie końca odpowiedzi końcowej (wypisanie wyniku)
\newcommand{\testStart}{\noindent \textbf{Test:}\newline} %ewentualne możliwe opcje odpowiedzi testowej: A. ? B. ? C. ? D. ? itd.
\newcommand{\testStop}{\newline} %koniec wprowadzania odpowiedzi testowych
\newcommand{\kluczStart}{\noindent \textbf{Test poprawna odpowiedź:}\newline} %klucz, poprawna odpowiedź pytania testowego (jedna literka): A lub B lub C lub D itd.
\newcommand{\kluczStop}{\newline} %koniec poprawnej odpowiedzi pytania testowego 
\newcommand{\wstawGrafike}[2]{\begin{figure}[h] \includegraphics[scale=#2] {#1} \end{figure}} %gdyby była potrzeba wstawienia obrazka, parametry: nazwa pliku, skala (jak nie wiesz co wpisać, to wpisz 1)

\begin{document}
\maketitle


\kategoria{Wikieł/Z1.93u}
\zadStart{Zadanie z Wikieł Z 1.93 u) moja wersja nr [nrWersji]}
%[a]:[1,2,3,4,5,6,7,8,9,10,11,12,13,14]
%[b]:[1,2,3,4,5,6,7,8,9,10,11,12,13,14]
%[c]:[1,2,3,4,5,6,7,8,9,10,11,12,13,14]
%[d]:[1,2,3,4,5,6,7,8,9,10,11,12,13,14]
%[e]=[a]*[d]
%[f]=[c]*[b]
%[g]=[d]-[b]
%[h]=[b]*[d]
%[i]=[e]+[f]-[h]
%[j]=[c]-[a]-[g]
%[delta]=([j]**2)-(4*[i])
%[pr2]=(pow([delta],(1/2)))
%[pr1]=[pr2].real
%[pr]=int([pr1])
%[zz1]=([j]-[pr])/2
%[zz2]=([j]+[pr])/2
%[z1]=int([zz1])
%[z2]=int([zz2])
%[x1]=(pow(10,[z1]))
%[x2]=(pow(10,[z2]))
%[p1]=(pow(10,[z1]+1))
%[p2]=(pow(10,[z2]+2))
%[d1]=pow(10,-[d])
%[delta]>0 and [pr2].is_integer()==True and [d]>[b] and [j]>0 and [i]>0 and [zz1].is_integer()==True and [zz2].is_integer()==True and [z2]!=[b] and [z1]!=[b] and [x1]!=[d1] and [x2]!=[d1] 
Rozwiązać równanie $\frac{[a]}{[b]-\log{x}} + \frac{[c]}{\log{x}+[d]} = 1$
\zadStop
\rozwStart{Małgorzata Ugowska}{}
Szukamy dziedziny:
$$x>0$$
$$[b]-\log{x} \ne 0 \quad \Longrightarrow \quad x \ne 10^{[b]}$$ 
$$\log{x} +[d] \ne 0 \quad \Longrightarrow \quad x \ne 10^{-[d]}$$
Następnie rozwiązujemy równanie: 
$$\frac{[a]}{[b]-\log{x}} + \frac{[c]}{\log{x}+[d]} = 1 $$
$$ \Longleftrightarrow \quad \frac{[a](\log{x}+[d]) + [c]([b]-\log{x})}{([b]-\log{x})(\log{x}+[d])} = 1 $$
$$\Longleftrightarrow \quad  \frac{[a]\log{x}+[e] + [f]-[c]\log{x}}{([b]-\log{x})(\log{x}+[d])} = 1 $$
$$ \Longleftrightarrow \quad [a]\log{x}+[e] + [f]-[c]\log{x}= ([b]-\log{x})(\log{x}+[d]) $$
$$ \Longleftrightarrow \quad [a]\log{x}+[e] + [f]-[c]\log{x}= - \log^2{x} -[g]\log{x} + [h]$$
$$ \quad \Longleftrightarrow \quad \log^2{x} - [j]\log{x}+[i]=0 $$
Podstawiamy $y=\log{x}$ i mamy:
$$y^2 -[j]y+[i]=0$$
$$ \bigtriangleup = [j]^2 - 4 \cdot [i] = [delta] \quad  \Longrightarrow \quad \sqrt{\bigtriangleup} = [pr]$$
$$y_1=\frac{[j]-\sqrt{\bigtriangleup}}{2} = [z1] \quad \land \quad y_2=\frac{[j]+\sqrt{\bigtriangleup}}{2} =[z2]$$
dla $y=[z1]$:
$$\log{x} = [z1] \quad  \Longrightarrow \quad x = 10^{[z1]} = [x1]$$
dla $y=[z2]$:
$$\log{x} = [z2] \quad  \Longrightarrow \quad x = 10^{[z2]} = [x2]$$
\rozwStop
\odpStart
$x \in \{[x1], [x2]\}$
\odpStop
\testStart
A. $x \in \{[delta], [c]\}$\\
B. $x \in \{-1, 1\}$\\
C. $x \in \{[p1], [p2]\}$\\
D. $x \in \{[z1], [z2]\}$\\
E. $x \in \{[x1], [x2]\}$
\testStop
\kluczStart
E
\kluczStop



\end{document}