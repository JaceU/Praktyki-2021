\documentclass[12pt, a4paper]{article}
\usepackage[utf8]{inputenc}
\usepackage{polski}

\usepackage{amsthm}  %pakiet do tworzenia twierdzeń itp.
\usepackage{amsmath} %pakiet do niektórych symboli matematycznych
\usepackage{amssymb} %pakiet do symboli mat., np. \nsubseteq
\usepackage{amsfonts}
\usepackage{graphicx} %obsługa plików graficznych z rozszerzeniem png, jpg
\theoremstyle{definition} %styl dla definicji
\newtheorem{zad}{} 
\title{Multizestaw zadań}
\author{Robert Fidytek}
%\date{\today}
\date{}
\newcounter{liczniksekcji}
\newcommand{\kategoria}[1]{\section{#1}} %olreślamy nazwę kateforii zadań
\newcommand{\zadStart}[1]{\begin{zad}#1\newline} %oznaczenie początku zadania
\newcommand{\zadStop}{\end{zad}}   %oznaczenie końca zadania
%Makra opcjonarne (nie muszą występować):
\newcommand{\rozwStart}[2]{\noindent \textbf{Rozwiązanie (autor #1 , recenzent #2): }\newline} %oznaczenie początku rozwiązania, opcjonarnie można wprowadzić informację o autorze rozwiązania zadania i recenzencie poprawności wykonania rozwiązania zadania
\newcommand{\rozwStop}{\newline}                                            %oznaczenie końca rozwiązania
\newcommand{\odpStart}{\noindent \textbf{Odpowiedź:}\newline}    %oznaczenie początku odpowiedzi końcowej (wypisanie wyniku)
\newcommand{\odpStop}{\newline}                                             %oznaczenie końca odpowiedzi końcowej (wypisanie wyniku)
\newcommand{\testStart}{\noindent \textbf{Test:}\newline} %ewentualne możliwe opcje odpowiedzi testowej: A. ? B. ? C. ? D. ? itd.
\newcommand{\testStop}{\newline} %koniec wprowadzania odpowiedzi testowych
\newcommand{\kluczStart}{\noindent \textbf{Test poprawna odpowiedź:}\newline} %klucz, poprawna odpowiedź pytania testowego (jedna literka): A lub B lub C lub D itd.
\newcommand{\kluczStop}{\newline} %koniec poprawnej odpowiedzi pytania testowego 
\newcommand{\wstawGrafike}[2]{\begin{figure}[h] \includegraphics[scale=#2] {#1} \end{figure}} %gdyby była potrzeba wstawienia obrazka, parametry: nazwa pliku, skala (jak nie wiesz co wpisać, to wpisz 1)

\begin{document}
\maketitle


\kategoria{Wikieł/Z1.36q}
\zadStart{Zadanie z Wikieł Z 1.36 q) moja wersja nr [nrWersji]}
%[f]:[1,2,3,4,5,9,8,11]
%[e]:[1,2,4,5,6,11,12,13,16,18,19,20,21]
%[g]:[1,2,3,4,5,6,7,8,9,10,11,12,13,14]
%[b]=random.randint(2,200)
%[c]=random.randint(1,200)
%[d]=[b]*[b] + 4*([c]) 
%[z]=int(math.sqrt([d]))
%[a]=int((-[b]-[z])/2)
%[s]=int((-[b]+[z])/2)
%[z]-math.sqrt([d])==0  and [a]-((-[b]-[z])/2)==0 and  [s]-((-[b]+[z])/2)==0
Rozwiąż poniższą nierówność w zbiorze liczb rzeczywistych: 
$$x(x+[b]) \le [c]$$.
\zadStop
\rozwStart{Barbara Bączek}{}
$$x(x+[b]) \le [c] $$ 
W celu rozwiązania powyższej nierówności, początkowo rozwiązujemy równanie:
$$x(x+[b]) = [c]$$ 
$$x^2+[b]x -[c] = 0$$
$$\Delta= [b]^2 + 4\cdot[c]=[d], \sqrt{\Delta}=[z]$$
$$x_1=[a], \hspace{0.5cm} x_2=[s]$$
Zatem rozwiązaniem nierówności $x(x+[b]) \le [c] $ jest:
$$ x \in [[a],[s]]$$
\rozwStop
\odpStart
$x \in [[a],[s]]$
\odpStop
\testStart
A.$ x \in [[a],[s]]$
B.$ x \in [[a], [s])$
C.$ x \in ([a],[s]]$
D.$[[a], \infty)$
E.$ x \in \mathbb{R}$
G.$(-\infty, [a]] \cup [[s], \infty)$
H.$ x \in ([a],[s])$
\testStop
\kluczStart
A
\kluczStop



\end{document}