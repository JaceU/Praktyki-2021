\documentclass[12pt, a4paper]{article}
\usepackage[utf8]{inputenc}
\usepackage{polski}

\usepackage{amsthm}  %pakiet do tworzenia twierdzeń itp.
\usepackage{amsmath} %pakiet do niektórych symboli matematycznych
\usepackage{amssymb} %pakiet do symboli mat., np. \nsubseteq
\usepackage{amsfonts}
\usepackage{graphicx} %obsługa plików graficznych z rozszerzeniem png, jpg
\theoremstyle{definition} %styl dla definicji
\newtheorem{zad}{} 
\title{Multizestaw zadań}
\author{Robert Fidytek}
%\date{\today}
\date{}
\newcounter{liczniksekcji}
\newcommand{\kategoria}[1]{\section{#1}} %olreślamy nazwę kateforii zadań
\newcommand{\zadStart}[1]{\begin{zad}#1\newline} %oznaczenie początku zadania
\newcommand{\zadStop}{\end{zad}}   %oznaczenie końca zadania
%Makra opcjonarne (nie muszą występować):
\newcommand{\rozwStart}[2]{\noindent \textbf{Rozwiązanie (autor #1 , recenzent #2): }\newline} %oznaczenie początku rozwiązania, opcjonarnie można wprowadzić informację o autorze rozwiązania zadania i recenzencie poprawności wykonania rozwiązania zadania
\newcommand{\rozwStop}{\newline}                                            %oznaczenie końca rozwiązania
\newcommand{\odpStart}{\noindent \textbf{Odpowiedź:}\newline}    %oznaczenie początku odpowiedzi końcowej (wypisanie wyniku)
\newcommand{\odpStop}{\newline}                                             %oznaczenie końca odpowiedzi końcowej (wypisanie wyniku)
\newcommand{\testStart}{\noindent \textbf{Test:}\newline} %ewentualne możliwe opcje odpowiedzi testowej: A. ? B. ? C. ? D. ? itd.
\newcommand{\testStop}{\newline} %koniec wprowadzania odpowiedzi testowych
\newcommand{\kluczStart}{\noindent \textbf{Test poprawna odpowiedź:}\newline} %klucz, poprawna odpowiedź pytania testowego (jedna literka): A lub B lub C lub D itd.
\newcommand{\kluczStop}{\newline} %koniec poprawnej odpowiedzi pytania testowego 
\newcommand{\wstawGrafike}[2]{\begin{figure}[h] \includegraphics[scale=#2] {#1} \end{figure}} %gdyby była potrzeba wstawienia obrazka, parametry: nazwa pliku, skala (jak nie wiesz co wpisać, to wpisz 1)

\begin{document}
\maketitle


\kategoria{Wikieł/Z1.68f}
\zadStart{Zadanie z Wikieł Z 1.68 f) moja wersja nr [nrWersji]}
%[a]:[2,3,4,5,6,7,8,9,10]
%[b]:[2,3,4,5,6,7,8,9,10]
%[c]=[a]+[b]
%[d]=[a]*[b]
%[a]<[b]
Rozwiązać równanie $\Big| \frac{x^2-[c]x+[d]}{x^2+[c]x+[d]} \Big| = \frac{(x-[a])([b]-x)}{(x+[a])(x+[b])}$.
\zadStop
\rozwStart{Małgorzata Ugowska}{}
Dziedzina: $ x \notin\{ -[b], -[a]\}$
$$\Big| \frac{x^2-[c]x+[d]}{x^2+[c]x+[d]} \Big| = \frac{(x-[a])([b]-x)}{(x+[a])(x+[b])}$$
$$\Big| \frac{(x-[a])(x-[b])}{(x+[a])(x+[b])}\Big| = - \frac{(x-[a])(x-[b])}{(x+[a])(x+[b])}$$
$$\frac{(x-[a])(x-[b])}{(x+[a])(x+[b])} \le 0$$
$$(x-[a])(x-[b])(x+[a])(x+[b]) \le 0$$
$$x \in [-[b],-[a]] \cup [[a], [b]] \quad \land \quad x \notin\{ -[b], -[a]\} \quad \Longrightarrow \quad x \in (-[b],-[a]) \cup [[a], [b]] $$
\rozwStop
\odpStart
$x \in (-[b],-[a]) \cup [[a], [b]]$
\odpStop
\testStart
A. $x \in (-[b],-[a]) \cup [[a], [b]]$\\
B. $x \in \emptyset $\\
C. $x \in [-[b],-[a]] \cup [[a], [b]]$\\
D. $x \in (-[b], [b]]$\\
E. $x \in (-[a], [b]]$\\
F. $x \in [-[a], [b]]$\\
G. $x \in (-[b],-[a]) \cup ([a], [b])$
\testStop
\kluczStart
A
\kluczStop



\end{document}