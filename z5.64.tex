\documentclass[12pt, a4paper]{article}
\usepackage[utf8]{inputenc}
\usepackage{polski}

\usepackage{amsthm}  %pakiet do tworzenia twierdzeń itp.
\usepackage{amsmath} %pakiet do niektórych symboli matematycznych
\usepackage{amssymb} %pakiet do symboli mat., np. \nsubseteq
\usepackage{amsfonts}
\usepackage{graphicx} %obsługa plików graficznych z rozszerzeniem png, jpg
\theoremstyle{definition} %styl dla definicji
\newtheorem{zad}{} 
\title{Multizestaw zadań}
\author{Robert Fidytek}
%\date{\today}
\date{}
\newcounter{liczniksekcji}
\newcommand{\kategoria}[1]{\section{#1}} %olreślamy nazwę kateforii zadań
\newcommand{\zadStart}[1]{\begin{zad}#1\newline} %oznaczenie początku zadania
\newcommand{\zadStop}{\end{zad}}   %oznaczenie końca zadania
%Makra opcjonarne (nie muszą występować):
\newcommand{\rozwStart}[2]{\noindent \textbf{Rozwiązanie (autor #1 , recenzent #2): }\newline} %oznaczenie początku rozwiązania, opcjonarnie można wprowadzić informację o autorze rozwiązania zadania i recenzencie poprawności wykonania rozwiązania zadania
\newcommand{\rozwStop}{\newline}                                            %oznaczenie końca rozwiązania
\newcommand{\odpStart}{\noindent \textbf{Odpowiedź:}\newline}    %oznaczenie początku odpowiedzi końcowej (wypisanie wyniku)
\newcommand{\odpStop}{\newline}                                             %oznaczenie końca odpowiedzi końcowej (wypisanie wyniku)
\newcommand{\testStart}{\noindent \textbf{Test:}\newline} %ewentualne możliwe opcje odpowiedzi testowej: A. ? B. ? C. ? D. ? itd.
\newcommand{\testStop}{\newline} %koniec wprowadzania odpowiedzi testowych
\newcommand{\kluczStart}{\noindent \textbf{Test poprawna odpowiedź:}\newline} %klucz, poprawna odpowiedź pytania testowego (jedna literka): A lub B lub C lub D itd.
\newcommand{\kluczStop}{\newline} %koniec poprawnej odpowiedzi pytania testowego 
\newcommand{\wstawGrafike}[2]{\begin{figure}[h] \includegraphics[scale=#2] {#1} \end{figure}} %gdyby była potrzeba wstawienia obrazka, parametry: nazwa pliku, skala (jak nie wiesz co wpisać, to wpisz 1)

\begin{document}
\maketitle


\kategoria{Wikieł/Z5.64}
\zadStart{Zadanie z Wikieł Z 5.64 ) moja wersja nr [nrWersji]}
%[a]:[2,3,4,5,6,7,8,9]
%[b]:[2,3,4,5,6,7,8,9]
%[e]:[2,3,4,5,6,7,8,9]
%[c]=random.randint(2,20)
%[d]=random.randint(2,20)
%[a3]=[a]*3
%[b2]=[b]*2
%[ea3]=[e]*[a3]
%[ce]=int([c]/[e])
%[m]=int(-([ce]+[ea3])/[b2])
%[b2m]=abs([b2]*[m])
%[x1]=[e]-1
%[x2]=[e]+1
%[f1]=[a3]*[x1]*[x1]-[b2m]*[x1]+[c]
%[f2]=[a3]*[x2]*[x2]-[b2m]*[x2]+[c]
%math.gcd([c],[e])==[e] and math.gcd(([ce]+[ea3]),[b2])==[b2] and [f1]<0 and [f2]>0
Wyznaczyć takie $m$, aby funkcja $f(x)=[a]x^{3}+[b]mx^{2}+[c]x+[d]$ osiągała ekstremum w $x=[e]$. Zbadać, czy jest to minimum czy maksimum.
\zadStop
\rozwStart{Wojciech Przybylski}{Pascal Nawrocki}
$$ f(x)=[a]x^{3}+[b]mx^{2}+[c]x+[d]\hspace{5mm} \mathcal{D}_{f}=\mathbb{R}$$
$$ f'(x)=[a3]x^{2}+[b2]mx+[c]$$
$$\mbox{ wiemy że } x=[e] \mbox{, więc otrzymujemy równanie z niewiadomą "z" }$$
$$f'(x)=(x-[e])([a3]x-z)=[a3]x^{2}-zx-[ea3]x+[e]z$$
$$[e]z=[c]\Rightarrow z=[ce]$$
$$-[ce]-[ea3]=[b2]m \Rightarrow m=[m]$$
$$f'(x)=[a3]x^{2}-[b2m]x+[c],\hspace{3mm}x_{1}=[e]-1=[x1],\hspace{2mm} x_{2}=[e]+1=[x2]$$
$$f'([x1])=[a3]\cdot[x1]^{2}-[b2m]\cdot[x1]+[c]=[f1]<0$$
$$f'([x2])=[a3]\cdot[x2]^{2}-[b2m]\cdot[x2]+[c]=[f2]>0$$
Parametr $m=[m]$ natomiast ekstremum w $x=[e]$ jest minimum.
\rozwStop
\odpStart
Parametr $m=[m]$, natomiast ekstremum w $x=[e]$ jest minimum.
\odpStop
\testStart
A. Parametr $m=[m]$, natomiast ekstremum w $x=[e]$ jest minimum.\\
B. Parametr $m=[m]$, natomiast ekstremum w $x=[e]$ jest maksimum.\\
C. Parametr $m=[ce]$, natomiast ekstremum w $x=[e]$ jest minimum.\\
D. Parametr $m=[ce]$, natomiast ekstremum w $x=[e]$ jest maksimum.\\
E. Parametr $m=[e]$, natomiast ekstremum w $x=[e]$ jest minimum.\\
F. Nie istnieje taki parametr $m$, który by pasował do tej funkcji.
\testStop
\kluczStart
A
\kluczStop



\end{document}