\documentclass[12pt, a4paper]{article}
\usepackage[utf8]{inputenc}
\usepackage{polski}

\usepackage{amsthm}  %pakiet do tworzenia twierdzeń itp.
\usepackage{amsmath} %pakiet do niektórych symboli matematycznych
\usepackage{amssymb} %pakiet do symboli mat., np. \nsubseteq
\usepackage{amsfonts}
\usepackage{graphicx} %obsługa plików graficznych z rozszerzeniem png, jpg
\theoremstyle{definition} %styl dla definicji
\newtheorem{zad}{} 
\title{Multizestaw zadań}
\author{Robert Fidytek}
%\date{\today}
\date{}
\newcounter{liczniksekcji}
\newcommand{\kategoria}[1]{\section{#1}} %olreślamy nazwę kateforii zadań
\newcommand{\zadStart}[1]{\begin{zad}#1\newline} %oznaczenie początku zadania
\newcommand{\zadStop}{\end{zad}}   %oznaczenie końca zadania
%Makra opcjonarne (nie muszą występować):
\newcommand{\rozwStart}[2]{\noindent \textbf{Rozwiązanie (autor #1 , recenzent #2): }\newline} %oznaczenie początku rozwiązania, opcjonarnie można wprowadzić informację o autorze rozwiązania zadania i recenzencie poprawności wykonania rozwiązania zadania
\newcommand{\rozwStop}{\newline}                                            %oznaczenie końca rozwiązania
\newcommand{\odpStart}{\noindent \textbf{Odpowiedź:}\newline}    %oznaczenie początku odpowiedzi końcowej (wypisanie wyniku)
\newcommand{\odpStop}{\newline}                                             %oznaczenie końca odpowiedzi końcowej (wypisanie wyniku)
\newcommand{\testStart}{\noindent \textbf{Test:}\newline} %ewentualne możliwe opcje odpowiedzi testowej: A. ? B. ? C. ? D. ? itd.
\newcommand{\testStop}{\newline} %koniec wprowadzania odpowiedzi testowych
\newcommand{\kluczStart}{\noindent \textbf{Test poprawna odpowiedź:}\newline} %klucz, poprawna odpowiedź pytania testowego (jedna literka): A lub B lub C lub D itd.
\newcommand{\kluczStop}{\newline} %koniec poprawnej odpowiedzi pytania testowego 
\newcommand{\wstawGrafike}[2]{\begin{figure}[h] \includegraphics[scale=#2] {#1} \end{figure}} %gdyby była potrzeba wstawienia obrazka, parametry: nazwa pliku, skala (jak nie wiesz co wpisać, to wpisz 1)

\begin{document}
\maketitle


\kategoria{Wikieł/Z1.100}
\zadStart{Zadanie z Wikieł Z 1.100) moja wersja nr [nrWersji]}
%[a]:[2,3,4,5,6,7,8,9]
%[b]:[2,3,4,5,6,7,8,9]
%[c]=random.randint(2,10)
%[d]=random.randint(2,10)
%[h1]=[b]-(2*[a]*[c])
%[h2]=-1*[h1]
%[dziel]=math.gcd([h2],[a])
%[h3]=int([h2]/[dziel])
%[a3]=int([a]/[dziel])
%[a]!=0 and [h1]<0 and [a3]!=1
Dana jest funkcja $f(x)=\log([a]x^2-[b]x+[d])$. Znaleźć takie wartości $h$, aby $f([c]-h)<f([c])$.
\zadStop
\rozwStart{Joanna Świerzbin}{}
$$f(x)=\log([a]x^2-[b]x+[d])$$
$$f([c]-h)<f([c])$$
$$\log([a]([c]-h)^2-[b]([c]-h)+[d])<\log([a]\cdot[c]^2-[b]\cdot [c]+[d])$$
$$\log([a]([c]^2-2\cdot[c]\cdot h +h^2)-[b]\cdot[c]+[b]h+[d])<\log([a]\cdot[c]^2-[b]\cdot [c]+[d])$$
$$\log([a]\cdot[c]^2-2\cdot[a]\cdot[c]\cdot h +[a]h^2-[b]\cdot[c]+[b]h+[d])<\log([a]\cdot[c]^2-[b]\cdot [c]+[d])$$
$$[a]\cdot[c]^2-2\cdot[a]\cdot[c]\cdot h +[a]h^2-[b]\cdot[c]+[b]h+[d]<[a]\cdot[c]^2-[b]\cdot [c]+[d]$$
$$-2\cdot[a]\cdot[c]\cdot h +[a]h^2+[b]h<0$$
$$[h1] h +[a]h^2<0$$
$$h([h1] +[a]h)<0$$
Znajdźmy miejscą zerowe:
$$h([h1] +[a]h)=0$$
$$h=0 \lor  h=\frac{[h3]}{[a3]}$$
więc
$$h \in \left(0,\frac{[h3]}{[a3]}\right)$$
\rozwStop
\odpStart
$h \in \left(0,\frac{[h3]}{[a3]}\right)$
\odpStop
\testStart
A. $h \in \left(0,\frac{[h3]}{[a3]}\right)$\\
B. $h \in \left(0, [h3]\right)$\\
C. $h \in \left(0,[a3] \right)$\\
D. $h \in \left(0,1 \right)$\\
E. $h \in \left(0,\infty\right)$\\
F. $h \in \mathbb{R}$\\
\testStop
\kluczStart
A
\kluczStop



\end{document}